\chapter{伟大的南汉国}

\section{保卫南粤的海洋家族:封州刘氏}

\indent 1997年,德国海床勘探公司(Seabed Explorations)和印尼老海成公司(P.T. Sulung Segarajaya)在雅加达以北的印坦油田附近海域打捞到了一艘巨大的古代沉船残骸。该船长30米、宽10米,运载了大量瓷器、铅币、银锭和带有狰狞兽纹的爪哇铜器,当为往返于东亚大陆与东南亚之间的商船。由个别物件上所刻汉字来看,“印坦沉船”上的货物出于公元920年至960年之间。其中,银锭的数量十分惊人,共有97枚,总重量几达5000两,相当于宋初帝国全年收入的1\%强。银锭上的文字更进一步显示,它们都铸造于南粤境内。在这样一艘普通商船上便有如此多的南粤银锭,彼时南粤之富庶、重商可见一斑。而那时的南粤,正处于南汉国时代\footnote{杜希德、思鉴:《沉船遗宝:一艘十世纪沉船上的中国银锭》,《唐研究》第十卷,页383—431}。南汉国究竟是一个什么样的政权,能够使我们的伟大祖先创造如此繁荣的文明奇迹?欲解答这一问题,便需从南汉国的起源讲起。

南汉国的建立者系封州土豪刘氏。据刘氏自述,其祖籍为河南上蔡,又迁居闽越之仙游,后至南海经商,定居番禺\footnote{《广东通史》古代上册,页628}。然而,种种历史记载表明,这一切不过是刘氏为在帝国内部争取合法身份的说辞,正如士氏自称其祖先来自山东。刘氏家族的真正祖先,应当是来粤定居的波斯人或阿拉伯人。到9世纪,这个家族已经完全融入了南粤社会,成为地地道道的粤人\footnote{刘氏族属系一史家聚讼不已的问题,主要分为“大食波斯说”、“俚僚说”、“中原说”三派。“中原说”虽然文献证据最充分,但攀附北方祖先实为南粤史上粤人极其常用的策略,16世纪以后出现的宗族都有大量的族谱材料能够“证明”自己的祖先来自北方。因此,用健全常识判断,文献材料越多的说法反而越不可靠。至于“俚僚说”也很不可靠,仅是部分学者毫无根据的推测。由刘氏在南海经商、定居番禺(广州)来看,他们是阿拉伯、波斯后裔的可能性最大。关于史家对此问题的讨论,参见周加胜:《南汉国研究》,页7—9}。在859—873年之间,刘氏家主刘安至粤东为官,任潮州长史\footnote{周加胜:《南汉国研究》,页7},其子刘谦于驻粤唐军中任下级军官“牙校”。879年,黄巢流寇军入侵南粤,已继任一家之主的刘谦迎来了生死存亡的考验\footnote{《广东通史》古代上册,页628。}在大肆屠杀广州居民并侵占南粤大部分土地后,黄巢曾一度有割据岭南的打算。他还曾为此上表唐廷,希望获授“广州节度使同平章事兼安南都护”的官职。然而,由于流寇多为北人,不习南粤的湿热气候。至当年年底,流寇军已因“大疫”减员了三至四成。与此同时,容州、高州、韶州等地的南粤土豪及唐军残部亦纷纷起兵袭杀流寇。在南粤军民和瘟疫的双重打击下,黄巢被迫放弃长久侵占南粤的打算,全军北上,转战湖湘、吴越、江淮等地,于880年十二月攻陷唐帝国首都长安\footnote{《广东通史》古代上册,页588—589}。在黄巢据粤时英勇抗击流寇的南粤军民中,活跃着刘谦的身影。对于刘谦在此次战争的具体表现,史书记载语焉不详。然而,从其中“击群盗,屡有功”这几个字来看,刘谦一定立下了赫赫战功,对促成黄巢流寇撤离南粤起了很大作用。公元883年,为封赏刘谦的功劳,唐帝国授其封州刺史、贺水(今贺江)镇遏使,令其镇守西江上游的交通要道封州。在此前后,刘谦还受到唐岭南东道节度使韦宙的赏识,得以娶其侄女为妻,生子刘隐\footnote{公元627年,唐于今两广、福建、云南一部、越南北部置岭南道。862年,岭南道分为岭南东道、西道,东道治广州、西道治邕州,此为两广分治之始。}。据说,韦宙之妻曾因刘谦“非我族类”为由反对这桩婚事,韦宙则称:

\begin{quote}

此人(刘谦)非常流也,他日吾子孙或可依之\footnote{孙光宪:《北梦琐言》卷6}。

\end{quote}

此后发生的事证明,韦宙并未看错人。驻守封州期间,刘谦“抚纳流亡,节费养士”\footnote{梁廷柟:《南汉书》卷1《烈祖纪》},不但恢复了当地的社会秩序,亦训练出了一支有一万兵力、百余艘战舰的精锐部队,成为当地土豪化的军事领袖\footnote{欧阳修:《新五代史》卷65《南汉世家》}。公元894年,刘谦病逝。临终前,他对长子刘隐说了如下一番话:

\begin{quote}

今五岭盗贼方兴,吾有精甲犀械。尔勉建功,时哉不可失也\footnote{梁廷柟:《南汉书》卷1《烈祖纪》}。
\end{quote}

此段遗嘱中,刘谦指出五岭一带“盗贼”遍布,刘隐应靠刘氏的精锐部队趁机建功立业。揆诸当时形势,可知刘谦的话语实则别有暗示。当时,华北食人流寇首领秦宗权的余部刘建锋、马殷已率部横扫江右、湖湘。四年后,盘踞潭州(今长沙)的马殷更被唐廷任命为武安军节度使,开启了南楚政权的历史。几乎与此同时,著名的王潮、王审邦、王审知三兄弟亦已控制闽越之地。不久后,王审知将成为闽王,开创闽国。上述变局使南粤与湘、赣、闽的边界五岭陷入动荡不安。在此国际形势下,南粤如何处理与岭北各政权间的关系、如何生存下去,实为一迫在眉睫的问题。刘谦令刘隐解决五岭盗贼问题,事实上是希望他依靠父辈留下的家底站出来保卫南粤。这一遗嘱与任嚣对赵佗的临终托付非常相似,充分显示了刘谦的拳拳爱粤之心与责任感。

刘谦死后,军中共推刘岩为主。次年,已摇摇欲坠的唐廷仍未放弃控制南粤的野心,于南粤置清海军,以唐宗室薛王李知柔任之\footnote{周加胜:《南汉国研究》,页13}。896年十一月,李知柔行抵湖湘,未受马殷阻拦。这时,广州牙将卢琚、谭弘玘突然起兵反唐,意图阻击李知柔入境。谭玘率军西进占领端州(肇庆),又与刘隐联络,提出刘隐若举兵响应便能娶其女为妻。唐帝国末期,牙将的反复叛乱一直困扰着各藩镇。刘隐明白,卢、谭二将举兵反唐并非多么有爱于南粤,实则为满足个人野心。他们之所以欲拉刘隐入伙,实际上是为了利用封州刘氏的精兵。若此次反唐战争胜利,则卢、谭二将便能在保存自身实力的情况下消耗刘氏兵力。若战争失利,毫无政治德性的二人更有可能将“反叛”之名扣在刘氏头上,迅速倒向唐军,与侵略者一同将刘氏斩尽杀绝\footnote{周加胜:《南汉国研究》,页14}。权衡利弊之后,刘隐决定对二人开战。他首先假装答应谭弘玘的要求,向端州派出一支迎亲船队。船队于夜色中驶至端州后,伏于其中的士兵立即跃出发动奇袭,斩杀了谭弘玘。随后,刘隐军又顺流而下,以奇袭攻占广州城,击毙卢玘。随后,雄踞广州的刘隐以威武的军容将李知柔迎入南粤\footnote{周加胜:《南汉国研究》,页13}。震慑于刘氏军容的李知柔不得不任命刘隐为行军司马,令其掌管军事、赋税。至此,刘隐已在事实上控制了南粤。

对于刘隐的上位,一批身在南粤的唐帝国、将领颇为不满,乃密谋发动叛乱。公元898年十二月,粤北韶州刺史曾衮举兵攻广州,广州将王璙以战舰应之,两者皆被刘隐迅速讨平。其后,又有韶州将刘潼叛乱,被刘隐击败、斩杀\footnote{周加胜:《南汉国研究》,页14}。在短时间内,刘隐连破两股叛军,确立了其在南粤无可撼动的地位。至900年,唐廷以宰相徐彦若代李知柔镇清海军。徐彦若虽为唐帝国宰相,却无力号令粤人,只能将一切政事交给刘隐。昔日对南粤肆行暴虐的唐帝国,到此时可以说是衰微已极了。901年,徐彦若病死。次年,接替他的新节度使崔远仅行至江陵便因惧怕刘隐而不再前进。唐昭宗只得于903年召回崔远。904年,包揽唐帝国朝政的梁王朱温收受了刘隐的贿赂,迫使唐廷以刘隐为清海军节度使\footnote{《广东通史》古代上册,页629}。这样,无论在名义上还是实际上,刘隐都已经是南粤的统治者了。907年,朱温废唐哀帝而篡位,建立后梁。刘隐继续不断以金银、犀角、象牙、宝货、名香等自海外贸易得来的珍宝贿赂朱温,朱温亦投桃报李,对其不断加封\footnote{刘隐对朱温行贿之数量十分惊人。史载:“广州献金银、犀角、象牙,杂宝货及名香等,合估数十万。是月(908年四月)客省引进使韦坚使广州,及还,以银、茶上献,其估凡五百万。”见王钦若:《册府元龟》卷197}:908年,朱温命刘隐兼任静海军(越南北部,即交趾)节度使、领安南都护;909年,封刘隐为南平王。910年,进封南海王\footnote{《广东通史》古代上册,页629}。刘隐早已看穿了流寇出身的朱温贪财好货的本质,故能利用其这一弱点不断为自己、为南粤谋利。随着刘隐被封王,后梁帝国已基本上承认了南粤的独立,这实为南粤外交史上的一次重大胜利。

晚年的刘隐一边贿赂并利用朱温,一边受困于岭北侵略者持续不断的威胁。公元900年,桂、宜、岩、柳、象五州在马殷的煽动下“皆降于湖南”\footnote{周加胜:《南汉国研究》,页18}。902年,割据赣南的虔州刺史卢光稠率军入侵南粤,侵占潮州、韶州。刘隐命麾下猛将指挥使苏章反攻韶州,结果得到其三弟刘岩的如下劝谏:

\begin{quote}

韶州有虔潮之援,击之则两州必应,首尾受敌。宜计取,不宜直攻\footnote{梁廷柟:《南汉书》卷2《高祖纪》}。

\end{quote}

刘岩之语可谓至论,惜乎刘隐不听,导致苏章军于韶州城南中伏惨败。刘隐系一大度之人。经此大败,他认识到了自己军事才能的不足,遂将兵权完全交给刘岩。此时,南粤已到了万分危险的时刻,粤北、粤东皆被卢光稠侵占,西江上游西部(今广西境内)被马殷占领,刘氏的大本营封州已直接暴露在湖湘侵略者的兵锋前。在此后六年内,由于刘岩一直忙于抵抗卢光稠的进攻,因此忽视了对湖湘的防御。908年,马殷再度发动强大攻势,昭、贺、梧、蒙、龚、富六州皆被湖湘占领,但封州依然被牢牢掌握在粤人手中。两年后,又有容、高两州降于马殷\footnote{周加胜:《南汉国研究》,页17—18}。至此,马殷不但夺取了相当于今日广西的土地,亦将其魔爪深入粤西,与卢光稠一起对刘氏形成了三面包围之势。面对南粤土地一日日的沦丧于湖湘、江右侵略者之手的惨景,刘隐病倒了。此时的他虽已被朱温“封”为南海王,却完全高兴不起来,更无力思考自己当初将军权交给刘岩是否合适。公元911年三月,刘隐在广州病逝,刘岩顺理成章地继承了他的位置\footnote{司马光:《资治通鉴》卷268《后梁纪三》}。

令人惊喜的是,上天并未抛弃南粤,刘岩刚一上台就遇到了收复失地的天赐良机。当年年底,随着卢光稠与刘隐同年病死,叛将黎求杀死了卢光稠之子卢延昌,陷入严重内讧的虔州势力无暇南顾\footnote{欧阳修:《新五代史》卷41《杂传第二十九》}。趁此有利机会,刘岩于年底一举出兵收复了韶州、潮州,将江右侵略军完全赶出南粤。接着,他又迅速挥军西进,发起收复粤西的战役\footnote{陈欣:《南汉国史》,页85}。面对闪电般攻来的南粤军,驻守湖湘的容州守将姚彦章大惊失色,决定弃城,并丧心病狂地命军队将容州的百姓全部强制带往湖湘。刘岩随即光复容州、高州两地。这样一来,虽然刘岩在容州只得到了一座空城,但相当于今天广东之地域内的侵略军已全部被他赶走。不久后,刘岩的军队又沿西江东进,连复邕州、新州。至此,除西江上游的部分土地仍被马殷侵占外,刘岩已光复了绝大部分南粤领土\footnote{陈欣:《南汉国史》,页86},南粤暂时解除了被岭北侵略者灭亡的危险。

封州刘氏是一个因南粤发达的海洋贸易而产生的家族。他们虽然来自阿拉伯或波斯,却已完全视南粤为自己的家园。在黄巢大洪水肆虐南粤时,是刘谦站出来与流寇进行了殊死战斗。在唐帝国崩溃时,是刘隐站出来与后梁周旋、为南粤谋取自立。在来自湖湘与江右的侵略军肆意蹂躏南粤的河山时,是刘岩站出来发动绝地反击,收复了南粤的大部分土地,保护了南粤刚刚得到的自由。经过 kóydèy 父子三人的奋斗,南粤经受住了大洪水的考验,浴火重生了。现在,是时候该由刘岩考虑如何使南粤彻底脱离帝国、获得完全自由的问题了。


\section{多国体系中的南粤保卫者:南汉高祖}

\indent 公元915年,已安然自立的刘岩作出了与后梁帝国彻底决裂的决断。是年,他向朱温“求封”为南越王,遭到拒绝,遂对幕僚、部将们讲出了那个时代南粤的最强音:

\begin{quote}
	
今中国纷纷,孰为天子!安能梯航万里,远事伪廷乎\footnote{司马光:《资治通鉴》卷269《后梁纪四》}!

\end{quote}

从此之后,刘岩断绝了对朱温的“进贡”,自称“南越王”。公元917年八月,他在广州举行了隆重的登基大典,称帝并建国号“大越”,改元乾亨,是为南汉高祖。此外,他追尊祖父刘安仁为太祖、父刘谦为代祖、兄刘隐为烈祖,以广州为兴王府\footnote{司马光:《资治通鉴》卷270《后梁纪五》}。次年,又改国号为“南汉”。乾亨四年(920),高祖仿唐制开科举,首科进士十余人,以南海人简文会为状元\footnote{仇巨川:《羊城古钞》,页338。对于南汉国时期史事纪念,本书全部采用南汉之年号。}。九年,高祖听说有一条白龙出现于南宫三清殿,便改年号为“白龙”、改己名为“刘䶮”,以应此“祥瑞”,显示了立国之初意气风发的面貌\footnote{欧阳修:《新五代史》卷65《南汉世家》}。至此,继南越国之后,又一个伟大的南粤本土帝国便在南海与南岭之间诞生了。时隔千余年,我们的祖先再一次有了自己的皇室。

称帝之后,南汉高祖积极推行对外开放政策,大力发展海外贸易。此外,他还与江淮的吴国结成紧密的同盟,共同抗击后梁。在南粤与江淮之间的海路上,商船纵横往来,形成了一片“结连淮海,阻塞梯航”的繁荣景象\footnote{《广东通史》古代上册,页660}。乾亨三年(919),高祖又与南粤的死对头马楚政权和解,迎娶马殷之女为皇后。五年(921),后梁帝国命吴越王钱镠发兵进攻南粤,钱镠以“山川隔越,地方攘扰”为托词按兵不动。对于吴越的义举,高祖非常感激,乃与吴越互相通好,以父兄之礼事钱镠,两国之间经由海路的贸易往来亦连绵不断。大有六年(933),钱镠病逝,高祖遣人赴杭州致祭,使吴越文穆王钱元瓘十分感动,谱写了一段关于粤吴友谊的动人佳话。对蜀地的王氏前蜀政权,高祖也遣使通好,维护了粤蜀之间的友谊。白龙元年(925),高祖又将其女增城公主嫁给滇地大长与国皇帝郑仁旻,结成了牢固的粤滇同盟。至于据有荆南的南平国,则更是卑词厚币,向高祖俯首称臣\footnote{《广东通史》古代上册,页648}。这样,南汉高祖便用核平而有礼的外交手段与复国之后的东亚诸邦结成了广泛的同盟,共抗邪恶的中原帝国。这一壮举,无疑是一千年后东南互保的预演,亦是一块标志着诸夏友谊的伟大丰碑。

南汉高祖虽然奉行核平外交路线,但在外敌入侵南粤时,他绝不会妥协。乾亨七年(923),闽国发兵入侵南汉东界之梅口镇。次年,高祖亲征闽国进行报复,进至汀、漳之地而退。大有六年(933),马楚政权发兵围困刘氏龙兴之地封州,高祖命猛将苏章领“神弩军”三千、战舰百艘增援。苏章到达战场后,沉铁索于水中,于贺江两岸设巨轮以挽铁索,并筑长堤以隐之,置伏兵于堤中。马楚舰队来攻时,苏章军佯装败退,将侵略者引入两堤之间,突然转轮升起铁索截断其归路。随后,南汉伏兵大起,以强弩自两岸猛射,全歼了这股侵略者,又一次取得了南粤反侵略战争的伟大胜利\footnote{《广东通史》古代上册,页646}。

南汉高祖虽建立了反抗中原帝国的邪恶同盟、领导粤人连连取得反侵略战争的胜利,然亦做出过侵略他国的残忍之事,那便是侵略越南之战。唐帝国崩溃前,越南人与我们的祖先不分彼此,一同合力抗击岭北帝国的暴虐统治,那时的越南史与南粤史是连为一体的。但在10世纪初,越南本地土豪曲氏站出来建立了自己的国家,不但与中华帝国彻底切割,亦与南粤进行了切割。公元906年,洪州(今属越南海阳省)富豪曲承裕(曲先主)被交州人推举为静海军节度使,割据自保。次年,曲承裕去世,其子曲颢(曲中主)继承节度使之位。又过了一年,贪图贿赂的朱温突然加封刘岩为静海军节度使,由此引发了南汉高祖与曲氏对交州统治权的争夺。做为交州本地人,曲氏的统治无疑更具有合法性,高祖对交州的野心则是罪恶的。乾亨元年(917),曲颢去世,其子曲承美继位,是为曲后主。曲后主对南汉采取强硬态度,不向高祖称臣,使高祖大为恼怒\footnote{陈重金:《越南通史》,页48}。大有三年(930),高祖遂派兵攻打交州,俘曲后主,侵占其地\footnote{欧阳修:《新五代史》卷65《南汉世家》。此处采取《新五代史》之说法,将曲后主被俘之战系于大有三年(930)。然而,越南方面史料显示,此战发生于乾亨七年(923)。参见吴士连:《大越史记全书》}。

在处置曲后主的问题上,高祖展现了高尚的美德,将其释放,曲氏对交州的统治至此终结。然而,不管高祖如何大度,不义的侵略战争毕竟是不得人心的。第二年,不甘受南粤统治的越南人便展开了反抗。曲氏旧将杨廷艺发动了起义,将南汉军全部驱逐,光复了越南的国土。六年后,杨廷艺被其部将矫公羡杀害。杨廷艺部将闻之大怒,起兵讨伐矫公羡。矫公羡走投无路,乃遣人向南汉高祖求援。野心不死的南汉高祖视之为重夺交州的天赐良机,立即命太子刘弘操率率先锋南下进攻吴权,并自领大军为后援。南粤与越南之战,至此进入高潮\footnote{陈重金:《越南通史》,页50}。

大有十年(938),闻知刘弘操引军来犯的吴权迅速讨杀矫公羡,率军于白藤江迎敌。战前,吴权一面传檄将士鼓舞士气,一面命人砍伐树木、制造顶端包有铁皮的木桩,植于江心。待涨潮时,吴权命水军向南汉军挑战。年轻气盛的刘弘操不知是计,竟命水军向前进攻,越南军佯败而退。不久后,江水退潮,铁头木桩全部露出水面,戳破了许多南汉战舰。越南军随即反身猛攻,一举击败了这支南汉先锋军,将自刘弘操以下的大半官兵杀死。消息传至南汉后军,丧失了太子与许多忠勇将士的高祖大声恸哭,无心再战,遂收兵回到兴王府,自此不再侵犯交州\footnote{欧阳修:《新五代史》卷65《南汉世家》}。第二年,吴权在古螺称王,建立了越南吴朝,彻底终结了越南屈辱的北属时代\footnote{陈重金:《越南通史》,页57}。

白藤江之战是南粤与越南这两个伟大民族间的决战。在此之后,崭新的越南民族国家便出现在东亚大地上,不但获得了完全的自由,亦脱离了南粤史的发展轨迹。高祖做为侵略方,为实现个人野心进行了可耻的侵略战争,其行为实为我南粤史上的污点,应当予以彻底否定。然而在这场战争中,高祖并未如历来的岭北侵略者一样誓不罢休地进攻交州。在太子阵亡后,他便意识到了不义之战定然无法胜利,体面地主动结束了这场战争,显示了远高于岭北帝国统治者的政治德性。

据宋人司马光在《资治通鉴》中记载,南汉高祖在晚年变得十分残暴,设“灌鼻、割舌、支解、刳剔、砲炙、烹蒸”等酷刑虐待臣民\footnote{司马光:《资治通鉴》卷283《后晋纪四》}。对于此种出自帝国士人之手的记载,我们必须要小心对待。事实上,高祖晚年虽有“纵耳目之好”、“兴土木之工”等失德之举,但并无直接记载证明他曾使用过 这些酷刑。对于劝谏他应节省民力的左仆射黄损,他亦未施加任何肉体迫害,仅不许黄损任宰相,并对左右近臣发牢骚称“我殊不喜此老狂”,显得十分理智\footnote{梁廷柟:《南汉书》卷10《黄损传》}。因此,司马光的记载极有可能属于夸大、污蔑之辞。高祖晚年虽有过失贪纵之举,但应不会如此残暴。

大有十五年(942)四月,南汉高祖驾崩,结束了他风云激荡的一生,享年54岁。做为在错综复杂的多国体系中的南粤保卫者,他一面长袖善舞地推行联合吴越、江淮、湖湘、巴蜀、大滇共抗中原帝国的政策,保障了东亚诸邦的自由,一面又坚定地反击任何企图进犯南粤的侵略者,守护了南粤的自由。他虽有侵略越南、晚年失德这两大恶行,但这些劣迹丝毫不影响他的伟大。在今天的广州城隍庙中,他做为主神接受着人们的供奉与香火,守护着他为之奋斗了一生的南粤河山。一代代粤人对他虔诚的礼拜与祈祷,便是粤人会永远铭记他、怀念他的明证。

\section{血腥宫廷:殇帝、中宗朝的政治斗争与军事外交}

\indent 南汉高祖驾崩后,因其长子、次子皆已不在人世,皇位遂由三子刘弘度继承。刘弘度更名为刘玢,改元光天,是为南汉殇帝。殇帝生长于深宫,系一纨绔子弟。据说,他曾在高祖在世时聚集一批无赖子弟劫夺往来商人的财物,并受到高祖的溺爱与纵容\footnote{司马光:《资治通鉴》卷279《后唐纪八》}。登基之后,他更召大批伶人入宫,经常通宵达旦地饮酒作乐,将政事交给四弟晋王刘弘熙处理\footnote{《广东通史》古代上册,页666}。由于殇帝生活奢靡,南粤百姓的税负较高祖朝更为沉重,一场大规模民变因此爆发。

在殇帝上台后不久,博罗县吏张遇贤便聚众起兵,称“中天八国王”,建元“永乐”,发动了反对南汉的战争\footnote{马令:《南唐书》卷26《张遇贤传》}。张遇贤军皆着红衣,自称“赤军子”,人数很快膨胀至数万\footnote{《广东通史》古代上册,页672},攻破循州(今惠州、河源、梅州大部,州治在龙川),杀刺史刘传\footnote{梁廷柟:《南汉书》卷13《刘传传》}。殇帝派出精兵征讨,却被张遇贤军全歼于钱帛馆。随后,张遇贤军乘胜攻占潮州、正州(今河源),占据了整个粤东\footnote{陈欣:《南汉国史》,页119}。

在此情形下,殇帝仍然毫无振作之举,一味耽于享乐。此外,他又猜疑心极重,常常担心臣下谋反。每次召集宗室、群臣入宫宴饮时,他总令宦官在宫门对他们进行搜身\footnote{司马光:《资治通鉴》卷283《后晋纪四》}。然而,殇帝并未想到有反意者恰恰是他十分信任的四弟刘弘熙。刘弘熙是个心思深沉、手段毒辣的人,颇有意于皇位。为了麻痹殇帝,他向宫中“日进声妓”,诱导殇帝进一步堕落\footnote{《广东通史》古代上册,页666}。殇帝的荒淫终于引发其诸弟的厌恶,五弟越王弘昌、十弟循王弘杲皆倒向了弘熙一边。三王训练了一批熟习“角抵”(摔跤)的死士,将他们献给殇帝。而这些人的秘密使命,实为伺机杀死殇帝。光天二年(943)三月,这批死士趁殇帝滥醉之机发动政变,弑殇帝于长春宫\footnote{欧阳修:《新五代史》卷65《刘玢传》}。刘弘熙随即入宫称帝,改元应乾,改名刘晟,是为南汉中宗\footnote{司马光:《资治通鉴》卷283《后晋纪四》}。这时,中宗年仅24岁。

中宗在登基后立即露出了极度残酷的面目,开始卸磨杀驴。政变成功后不久,他便将屠刀对准了一同参与政变的循王、越王。当时,朝野上下对于中宗篡位之举议论纷纷。光天二年(943)五月,循王建议中宗将弑杀殇帝的死士除掉,以塞众人之口。中宗闻之佯装大怒,于深夜召循王入宫杀之。接着,中宗又令刺客杀死了越王。杀掉二王后,他进一步决定除掉所有皇弟,以防止有人重演自己的行为。乾和二年(944),中宗遣人毒杀五弟弘泽于邕州。次年,杀七弟弘雅,弑杀殇帝的死士亦被全部处死\footnote{陈欣:《南汉国史》,页125}。至乾和五年(947),他更是丧心病狂地将六弟弘弼、十一弟弘暐、十三弟弘简、十三弟弘建、十四弟弘建、十五弟弘济、十六弟弘道、十七弟弘昭、十九弟弘益及他们的儿子全部屠杀,尽收其妻女入宫,肆行淫乐。乾和十二、十三年间(954—955),仅剩的十二弟弘邈、十八弟弘政也被他杀害。这样,中宗便屠杀了除自己外的所有高祖诸子\footnote{陈欣:《南汉国史》,页125—127}。这种不顾法统和兄弟之情的疯狂杀戮、甚至奸淫诸弟妻女的行为,无论以何种标准来看都是禽兽不如的。中宗对亲人的大屠杀,实为我南粤史上的最大污点。

在对待粤东民变的问题上,中宗亦采取了不问请由、一律剿杀的残酷政策。光天二年(943),南汉军再度发动反攻,夺回循州。张遇贤连战连败,竟轻率地放弃粤东,全军越岭进攻赣南之虔州。张遇贤的行为使他完全丧失了南粤父老乡亲的支持,变为远离乡土的流寇。当时,虔州已被南唐政权占领。在南唐军的突袭下,张遇贤军惨败,其本人被俘送金陵杀死\footnote{《广东通史》古代上册,页673;公元937年,李昇逼迫吴主禅让,建立南唐政权,据有江淮、江右之地。}。持续一年、轰动一时的粤东民变,就这样惨淡地收场了。

南唐虽在进攻张遇贤军的军事行动中与南汉联合,但这并不意味着其视南汉为盟友。当时在位的南唐中主李璟系一野心极大之人,极力推行对外扩张政策。乾和三年(945),南唐灭闽。九年(951),攻占马楚都城潭州。至此,南汉的大部分邻国皆已被南唐占领,南唐帝国如泰山压顶般威胁着南汉。然而,南唐军攻陷潭州亦令马楚政权陷入穷途末路,使南汉获得了一统岭南的绝好机会。同年,中宗命亲信宦官吴怀恩率军溯西江而上,接连攻克蒙、桂、宜、连、梧、严、富、柳、龚、象等州,“尽有岭南之地”,收复了南粤陷于湖湘之手的全部失土。随着马楚的彻底灭亡,南汉与南唐立即进入战争状态,中宗决定先下手为强。当年十二月,中宗派宦官潘崇彻、将军谢贯越岭北伐,大破南唐军于义章(今湖南宜章),攻取郴州、桂阳监两地\footnote{《广东通史》古代上册,页647}。次年,马楚旧将刘言、王逵起兵于湖湘,于十一月赶走南唐军。十二月,潘崇彻率军于义章之蠔石击败王逵,斩首万余级,使南汉军不但未步南唐军之后尘撤离湖湘,更得以在岭北确保了一块稳定的占领区\footnote{刘应麟:《南汉春秋》卷1《世家》}。

中宗收复失地统一岭南之举,无疑是他伟大的历史功绩。至于北伐击破南唐军一事,更是他惩戒国际体系破坏者的义举。但是,当南唐军撤离湖湘、不再威胁南粤之后,他仍然不从岭北撤军,便将南粤由自卫反击者变成了侵略湖湘的一方,使义战变为不义。此后,南汉不得不将大量资源投入对湘南占领区维持中,使粤人的血汗大量地消耗于岭北。这一孤悬岭北的占领区亦迫使南粤在后来深深卷入岭北争霸战争,为后来南汉国的灭亡埋下了伏笔。这一历史教训告诉我们,南粤若要维持自己的自由,不但应与岭北帝国抗争到底,亦应尊重邻国的自由,不可贪图南岭以北的土地。背靠南岭、面向南海、与世界连为一体,方能使我南粤真正雄踞于地球之上,恢复本来面目。

\section{走向灭亡的南汉国:后主朝}


\indent 乾和十六年(958)年八月,南汉中宗病死,其十七岁的长子刘继兴继位,改元大宝,改名刘鋹,是为南汉后主。中宗本为残忍僭主,不但屠杀诸弟,亦信用吴怀恩、潘崇彻等宦官。耳濡目染之下,后主也对宦官十分亲近,宠信宦者陈延寿。大宝三年(960),陈延寿对后主进言称:

\begin{quote}
	陛下所以得立,由先帝尽杀群弟故也\footnote{李焘:《续资治通鉴长编》卷1}。
\end{quote}


后主对陈延寿之言深以为然,遂效仿将屠刀对准了亲弟。后主毕竟远不如中宗残暴,“仅”杀死了二弟璇兴、三弟庆兴,未加害四保兴、五弟崇兴。然而在享乐方面,后主却远过其父。继位伊始,后主便花费巨资兴建万政殿,“以银为殿衣,间以云母”。此外,他又以大量珍珠、玳瑁装饰所居宫殿,于殿中放置用“鱼英”、椰子壳制作的种种精美器皿。所谓“鱼英”,乃由鱼头骨烧化炼成之物。就算在海外贸易发达、物产丰富的南粤,鱼英亦是稀有的珍宝。关于后主奢靡的日常生活,史书如是记载:

\begin{quote}
	中官陈延寿作诸淫巧,动糜斗金。离宫数十,帝不时游幸,常至月余或旬日,率以豪民为课户,供千人馔\footnote{吴任臣:《十国春秋》卷60《后主本纪》}。
\end{quote}


由“离宫数十”这四个字来看,后主兴建的宫室绝不仅有万政殿一座。为在离宫中举行千人规模的盛宴,后主竟向富有的“豪民”课税。每次游幸离宫时,后主都会带上一个被他爱称为“媚猪”的波斯女子,对她极尽宠爱之能事\footnote{梁廷柟:《南汉书》卷7《后妃列传》}。除修建宫殿外,后主还大建佛寺。作为阿拉伯或波斯人的后代,南汉皇室的祖先很可能信奉伊斯兰教或拜火教,但完全粤化的他们早已抛弃了祖先的宗教信仰。高祖同时信仰佛、道,中宗、后主则笃信佛教。在兴王府四面,后主参考星空中二十八宿的布局修建了二十八座佛寺\footnote{梁廷柟:《南汉书》卷6《后主纪》},保存至今、宏伟壮丽的广州大佛寺便是这些佛寺之一。此外,后主又在宫中供奉一个名叫樊胡子的外国女巫。樊胡子自称玉皇大帝降身,常端坐殿内帐中对后主宣示祸福,呼后主为“太子皇帝”\footnote{欧阳修:《新五代史》卷65《南汉世家》}。可见除佛教外,来路不明的奇异巫术亦得到后主的青睐。

为维持奢靡的生活与宗教活动,后主增加了种种苛捐杂税。自高祖时代起,南汉即积极鼓励对外贸易。那时,南汉国的威名响彻伊斯兰世界,阿拉伯人将南汉称作“月季花王朝”。据阿拉伯史书记载,10世纪时的广州已有与翁蛮(今阿曼)、波斯湾畔之西拉甫、八哈剌因、俄波拉、巴士拉等地往返直航的船只\footnote{陈欣:《南汉国史》,页280}。对来粤贸易的各国商人,高祖皆十分优待,不但恢复了广州的繁荣,亦创立了南汉以商税为军国用度大宗的财政格局\footnote{陈欣:《南汉国史》,页137}。对于这一点,宋神宗曾如是评论:

\begin{quote}
	(南汉)内足自富,外足抗中国者,亦由笼海商得术也\footnote{杨仲良:《宋通鉴长编纪事本末》卷66}。
\end{quote}

对于满载货物的船只、兴旺发达的贸易,后主自然视为利源,下令“计口以税,虽船居皆不免”,并于各城镇、圩市设镇、场、务等机构,征收名目繁多的商税。过度征敛使南汉朝廷渐渐失去了商人与土豪的信任,部分土豪甚至有避税逃亡岭北之举\footnote{《广东通史》古代上册,页671}。他们绝非不爱南粤之人,是后主的贪婪将他们逼上了这一步。安土重迁、热爱乡土是我南粤人自古以来的良好美德。身为粤人的皇帝,后主不但不能使南粤百姓安居于家乡,还令一些人不得不背井离乡、逃往岭北,可谓十分昏庸无道。更加过分的是,后主又出奇地信任宦官。中宗虽曾命宦官吴怀恩、潘崇彻等人带兵打仗,但对宦官的使用尚有节制。后主为创造完全忠于自己的官僚系统,竟肆无忌惮地以宦官代替士人担任外朝官职。当时,宦官“有为三师、三公者,但其上加‘内’字”。这批“内三师”、“内三公”掌握了实权,由士人组成的台省官员则被称为“门外人”,无权参预大政\footnote{《广东通史》古代上册,页669}。据司马光在《资治通鉴》中记载,后主朝的宦官人数多达二万\footnote{司马光:《资治通鉴》卷294}。这一数字虽被后世史家广泛引用,但由于出自宋帝国士人之手,颇有夸张之嫌。纵然如此,由当时宦官已经遍布官僚机构、控制了朝政来看,后主朝的宦官当为数不少。急剧膨胀的宦官队伍必然增大了宫廷开销、进一步提高了南粤百姓的税负。

就在后主不顾国家根本、肆意妄为时,一个比南唐更为强大、更为邪恶的帝国已出现在岭北。大宝三年(960),赵匡胤篡夺后周帝位,建立北宋帝国,是为宋太祖。五年(962),宋军攻灭荆南、侵占湖湘。八年(965),后蜀灭亡,巴蜀沦入宋帝国之手,遭到宋军疯狂杀掠,南唐、吴越随即对宋表示臣服。面对宋帝国侵略军的步步进逼,后主既不减轻赋税以团结人心,又不依托南岭天险布置防线,反而孟浪地屡屡在湖湘发动攻势。大宝七年(964)、八年(965),南汉军连续两次从湘南占领区出发,进攻被宋帝国占据的桂阳、江华、潭州等地。由于对敌情毫无了解,轻敌的南汉军接连被宋潭州防御使潘美击败,湘南重要据点郴州亦在宋军的反攻下失守。大宝十一年(968),南汉军进攻道州,第三次被宋军打败\footnote{陈欣:《南汉国史》,页153—154}。在后主的虐政下,南汉军已不再具有高祖、中宗时代的强大战斗力,因而接连战败。经过在湖湘的三次作战,后主依然未能摸清宋帝国的具体实力,宋太祖却已看透了南汉的虚弱。大宝十三年(970),宋太祖以潘美为贺州道行营兵马都部署,命其率军攻灭南汉。

在此危急时刻,后主陷入了无将可用的窘境。郴州失守后,后主起用中宗朝宿将吴怀恩为桂州团练使,令其领兵镇守西江上游要地。大宝九年(966),吴怀恩在桂州监督修治战舰,因督下过严而被心怀不满的工匠杀死,后主以另一宿将潘崇彻代之。然而,潘崇彻领兵不到一年便受到后主猜忌,被剥夺了军权\footnote{陈欣:《南汉国史》,页155—156}。当宋帝国侵略军来攻时,南汉的情形 hày :

\begin{quote}
	旧将多以谗构诛死,宗室翦灭殆尽,掌兵者唯宦人数辈\footnote{伯颜:《宋史》卷481《南汉世家》}。
\end{quote}


中宗、后主不顾法统、屠杀诸弟,至此终于自食其果。大宝十三年(970)八月,宋军兵临贺州(今湘、粤、桂交界处)城下。后主连忙召集群臣商议对策,决定请出潘崇彻抗击强敌。然而,潘崇彻对后主无故罢其军权一事尚耿耿于怀,以目疾推辞。无奈之下,后主只得命大将伍彦柔率军救援贺州。伍彦柔乃一不知军略的莽夫,率战舰一味冒进。潘美遂设下圈套,退兵二十里,将伍彦柔全军引诱上岸,再以伏兵攻击。混乱之中,这支南汉援军彻底溃败,“死者十七八”,伍彦柔亦被俘杀。不久后,贺州城亦告失守,时为大宝十三年(970)九月\footnote{陈欣:《南汉国史》,页161}。宋帝国侵略军在贺州大造战船,声言欲沿贺江、西江顺流东下直取广州。直到此时,潘崇彻才终于答应出山,获授内太师、马步军都统,率三万大军进屯贺江。然而,这一切不过是潘美声东击西的奸计。十月,宋军突然向西进攻,接连侵占富州(今广西昭平)、昭州(今广西平乐)、桂州(今广西桂林)。潘崇彻心灰意冷,自此避战自保。十一月,宋帝国侵略军重回贺州,向东进攻,先破开建寨,凶残地屠杀了数千人,而后又招降天险骑田岭之守军,攻破连州。十二月,侵略军逼近韶州,守将李承渥率军出战,两军战于城外莲华峰下。激战中,李承渥将战象置于阵前,每象载战士十余人,军容十分威武。然而,在侵略军弩手的密集射击下,战象兵被击溃,受惊的大象返身践踏己军,导致南汉军彻底崩溃,李承渥仅以身免,韶州城随即失守\footnote{李焘:《续资治通鉴长编》卷11}。次年正月,英州、雄州被侵略军攻占,潘崇彻向侵略军投降。至此,粤北已完全落入了宋帝国的魔掌\footnote{陈欣:《南汉国史》,页162}。

这时,后主早已大惊失色。进退失据的他听信老宫女梁鸾真之言,以其养子郭崇岳为招讨使,与大将植廷晓统兵六万驻守兴王府城外百余里的马逕,布下了南汉的最后一道防线。潘美率军于距马逕十里处下寨,以游骑频频挑战。郭崇岳系一毫无将才之人,部下兵士亦多为由韶州、英州逃归的败军,士气低落,乃立栅坚守不出。情急之下,后主竟决定抛弃南粤的河山与军民,取船十余艘,欲载金宝、嫔妃逃亡海上。对此深感愤怒的宦官乐范与卫兵千余人将船只扣留,阻止了后主的出逃计划。无计可施的后主遂作出了最无耻的决定:遣使向侵略者屈膝投降\footnote{陈欣:《南汉国史》,页163}。

可是,此时的潘美仍不打算接受南汉的投降,逮捕了南汉的使者。他要将所有为南粤的自由而战的义士都屠杀干净,然后再考虑受降问题。见使者一去不回,后主十分害怕,命郭崇岳加强戒备,又令四弟祯王保兴率兵北上增援马逕守军。大宝十四年(971)二月四日,最后的决战在马逕爆发了。是日,植晓廷对胆怯的郭崇岳发表了一段悲壮的说话:

北军乘席卷之势,其锋不可当也。吾士旅虽众,然皆伤罢之余。今不驱策而前,亦坐受其毙矣\footnote{李焘:《续资治通鉴长编》卷12}!

直到今天,这段话依然掷地有声,回荡在每一位热爱南粤、保卫南粤的义士心中。伟大的南粤爱国者廷晓植说出了南粤的战斗宣言:无论岭北侵略者有多么强大,南粤都会反抗到底。就算一时难以战胜侵略军,粤人亦一定会为了自由挺身而战,绝不引颈受戮、绝不坐以待毙。决战当天,面对来攻的宋帝国侵略军,廷晓植率前军奋斗到了最后一刻,壮烈地力战而死,成为了永远的英雄。指挥后军的郭崇岳吓得肝胆俱裂,急忙奔回栅中固守。入夜时分,潘美命宋军乘风将火炬投向木栅。一时之间“万炬齐发”,火借风势,彻底吞噬了南汉军的工事。郭崇岳在一片混乱中被乱兵杀死,刘保兴逃归兴王府\footnote{李焘:《续资治通鉴长编》卷12}。这样,南汉军的最后一支的有生力量,便在侵略军残酷的火攻中消失了。

败报传至兴王府,一批宦官于黄昏时分开始焚烧府库宫殿。他们天真地认为,只要把财宝烧掉,宋帝国就不会再对南粤有兴趣了。熊熊大火燃烧了整夜,后主用十余年时间修建的宫殿、聚敛的财宝纷纷化为乌有。次日,即大宝十四年(971)二月五日,宋帝国侵略军进至白田,后主穿素服、骑白马出城投降。侵略者随即进入兴王府,俘南汉祯王刘保兴以下宗室、官属九十七人。南汉的六十州、二百一十四县,全部进入宋帝国版图。经历四帝、历时54年的南汉国至此灭亡,南粤第五次落入了岭北帝国的魔掌\footnote{陈欣:《南汉国史》,页163}。

此后,被俘的南汉后主君臣作为降虏一直生活在宋都汴京,受到宋帝国的“优待”。为了活命,后主恬不知耻地讨好宋帝,曾以珍珠结成种种奇异形状“进献”,颇得宋太祖欢心。公元976年,宋太祖在消灭南唐后死去。978年,吴越“纳土归宋”。979年,宋太宗将屠刀指向十国仅存的硕果、三晋之地的北汉。在进攻北汉前,宋太宗大宴群臣,后主亦在坐。宴会上,后主无耻地向宋太宗说:

\begin{quote}
	朝廷武灵及远,四方僭窃之主,今日尽在坐中。旦夕平太原,刘继元(北汉末帝)又至。臣率先来朝,愿得执梃,为诸国降王长\footnote{李焘:《续资治通鉴长编》卷20}。
\end{quote}

南汉高祖的子孙,就这样谄媚地拜倒在宋帝国皇帝的脚下。岭北帝国对于南粤的侮辱、南汉后主本人的无耻,皆令人愤怒、悲叹。后主最终还是没有保住自己的性命。公元980年,他离奇地暴死于汴京,很有可能系中毒身亡,时年39岁\footnote{陈欣:《南汉国史》,页168—169}。这一故事告诉我们,岭北帝国无论戴着怎样的面具,其阴毒、残暴都是不会改变的。宋帝国号称宽仁,实为东亚诸邦的死敌。宋灭南汉,制造了开建寨大屠杀;灭南唐,制造了江州屠城;灭后蜀,制造了成都大屠杀;灭北汉,又火烧太原城……南粤与东亚各邦如要争取自由,便必须丢弃一切幻想,同大一统帝国抗争到底。

南汉这一以“月季花王朝”之名声震世界的王朝灭亡了。做为海洋贸易的产物,南汉中宗、后主的失德并不影响这一王朝在对外贸易方面的伟大成就,更不会抹杀高祖、中宗为南粤自由而战的历史功绩。来自阿拉伯或波斯的南汉皇室能够完全粤化、建立南粤本土王朝,无疑说明了我南粤社会的坚韧。南汉国在历史上的存在,即已表明了南粤的伟大。它浓厚的海洋精神则更是鞭策着一代代粤人继续向海洋前进,把南粤的种子撒向五洲七海。


