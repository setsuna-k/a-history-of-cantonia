\chapter{伟大的南越国}

\section{南越国的建立}

\indent 公元前206年,暴秦灭亡,席卷华夏世界的第一次大洪水却仍未终止。在南粤,赵佗处在历史的十字路口上:究竟是将南粤变为“天下”的一部分、加入到在洪水中瓜分暴秦尸体的争霸战争中,还是让南粤自立、使她免遭洪水的侵害?
当时,赵佗的官职是龙川令,管辖着秦南海郡东部重镇龙川县。而在南粤掌握军政大权之人,则是南海尉任嚣。暴秦令“战功赫赫”的赵佗屈居于任嚣之下,不知是否有借任嚣防范赵佗之意。若有的话,暴秦无疑是失算了。其时任嚣已是重病将死之人,自知时日无多的他将赵佗召至南海郡城番禺,说了如下一番推心置腹的话:

\begin{quote}
	闻陈胜等作乱,秦为无道,天下苦之。项羽、刘季、陈胜、吴广等州郡各共兴军聚众,虎争天下,中国扰乱,未知所安,豪杰畔秦相立。南海僻远,吾恐盗兵侵地至此,吾欲兴兵绝新道,自备,待诸侯变,会病甚。且番禺负山险,阻南海,东西数千里,颇有中国人相辅,此亦一州之主也,可以立国。郡中长吏无足与言者,故召公告之\footnote{ 司马迁:《史记》卷113《南越列传》}。
\end{quote}

此段说话中,任嚣点出了南粤背靠南岭、面朝南海、东西数千里、足以自守的地利,希望赵佗依靠暴秦留下的武力自立,避免大洪水波及南粤。不久后,任嚣将南海尉之职交予赵佗,驾鹤西去。赵佗果然没有辜负任嚣的嘱托,他立即向粤北的横浦、阳山、湟溪关发出了一道简短的命令:

\begin{quote}
	盗兵且至,急绝道聚兵自守\footnote{司马迁:《史记》卷113《南越列传》}!
\end{quote}

在这里,赵佗将岭北流窜至粤的散兵游勇称作“盗兵”,要求南岭各关口断绝道路,以兵自守。南粤终于结束了第一次北属,与岭北决裂了。

其后,赵佗挥军从南海出发,对桂林、象郡展开了疾风烈火般的攻击,两地的暴秦长吏纷纷被击毙,赵佗的部下接管了整个南粤。三年后,公元前203年,出生于赵国真定的赵佗自称南越武王,采取“与辑百越”的政策,自己亦完全采纳了百越人的礼仪、发式与饮食习惯,成为了归化越人。至此,一个强有力的南粤本土国家、伟大的南越国,在南海与南岭之间出现了!

\section{永远的英雄:南越武帝}

\indent 公元前202年,项羽在奋战后消失于巫江畔的乱军之中,华夏世界重返封建体系的努力失败,继承暴秦遗产的汉帝国出现。在大一统史观的叙事体系下,从公元前202年开始,东亚大陆进入了汉代。然而,只要我们将目光移向与西汉初年同时的百越之地,就会明白实际情况其实远更复杂。

在吴越南端,东瓯王摇统治着东瓯族国家东瓯国。在八闽之地,闽越王无诸控制着强大的闽越族国家闽越国。今天广东、江西、福建的交界处,也就是日后客家人的聚居区,闽越国南武侯织于公元前195年被汉朝册封为“南海王”,建立了南海国。这些国家与南越国一起,在百越之地构成了一片连续的越人国家。至于南粤以西,在今天被称为“西南”的区域,则有夜郎、靡莫、滇、卭都等被汉人统称为“西南夷”的独立王国或者部族联盟。所有这些政权一同构成了复杂的多国体系,从东南、南、西南三个方向以弧形包围了汉帝国南疆。与乏味单调的大一统帝国相比,此种多国体系无疑能够催生更复杂的宪制与国际关系,拥有更宽广的历史路径。因此,南越国一方面要考虑与汉帝国的关系,另一方面又要处理其它周边政权的关系。在这样复杂的国际舞台上,南越武王赵佗开始了保卫南粤的奋斗。

公元前196年,已将异姓王消灭殆尽、时日无多的汉高祖刘邦将贪婪的目光瞄向了尚未臣服于他的南越国。这一年,他派遣大夫陆贾出使南越,“册封”南越武王为“南越王”。为了南粤的和平及对汉通商之利,南越武王选择了接受“册封”,将南越国变为汉帝国名义上的藩属国。这绝不是卑躬屈膝的行为,仅仅是南越武王为对付汉帝国的威胁而采取的权宜之计。一旦汉帝国威胁到南越国的生存,南越武王一定会毫不犹豫地与之战斗到底。不久后发生的事情,即说明了这一点。

公元前183年春,汉帝国的实际掌权者吕后实行了“别异蛮夷”政策,禁止对南越国出口金铁、田器、母马、母牛、母羊、母畜。汉帝国背信弃义、擅自挑起争端的行为令武王感到由衷愤怒,他先后派遣内史藩、中尉高、御史平三名高官出使长安,对汉廷据理力争\footnote{ 梁廷柟:《南越五主传》,页7—8}。然而,吕后却卑鄙地扣留了三名使者,继续推行“别异蛮夷”政策。武王明白,辩论已经无效。能够捍卫南粤的方法,只有战斗一途。他直接指出了汉帝国吞灭南越国的狼子野心:

\begin{quote}
	别异蛮夷,隔绝器物,此必长沙王计也,欲倚中国,击灭南越而并亡之,自为功也\footnote{司马迁:《史记》卷113《南越列传》}。
\end{quote}

长沙国系汉朝分封于湖湘的异姓王国,其王室吴氏是刘邦异姓王大清洗中的唯一幸存者。长沙国对汉帝国的忠诚,由此可见。为保卫南粤而决意一战的武王立即自称“南越武帝”,发兵北伐长沙国,破其数邑。南越国与汉帝国之间,至此进入战争状态。

面对南越国的反抗,汉帝国露出了极度凶残的面目。残忍的吕后令人屠杀了武帝留在北方的几乎所有亲属,并捣毁了武帝父母的坟墓\footnote{ 张荣芳:《南越国史》,页193。}。公元前181年9月,一支庞大的汉帝国侵略军在隆虑侯周灶的指挥下抵达南岭,准备彻底消灭南越国。在武帝的指挥下,南越子弟兵同仇敌忾地坚守南岭各关口,“据险筑城”,令侵略军“兵不能逾岭”,只得退守长沙。在长时间的僵持中,越化的南越军完全能够适应湿热的气候,北人组成的汉军则饱受疫病折磨。经一年多的对峙,公元前179年,吕后病死,汉帝国侵略军狼狈北撤\footnote{ 张荣芳:《南越国史》,页195。}。在南越武帝的指挥下,不屈的南粤赢得了第二次抗击外敌的伟大胜利。越汉战争的胜利令南越国声威大震,成为汉帝国以南的一流强国。这时,不但西瓯、雒越人臣服于南越国的统治下,就连闽越、夜郎等国亦慑于南越国的兵威与财力,成为南越的臣属\footnote{ 张荣芳、黄淼章:《南越国史》,页156—158。}。这时,南越国的领土与属国的疆域已有“东西万余里”,南越武帝更是“乘黄屋左纛,称制,与中国侔”,南越国成为了足可与汉帝国相提并论的超强国家\footnote{ 司马迁:《史记》卷113《南越列传》}!

色厉内荏的汉帝国见武力无法征服南越国,遂决定采取怀柔政策。在汉军北撤的同一年,汉文帝在武帝的家乡河北真定寻访到了武帝残存的堂兄弟,给予官爵。同时,汉文帝还派人为武帝在真定的祖坟“置守邑,岁时奉祀”,并派遣曾出使过南越国的太中大夫的陆贾再次出使南越\footnote{ 司马迁:《史记》卷113《南越列传》}。陆贾到达南越后,武帝携他遍游国中名胜,并将一封书信交给陆贾,让其带回长安。这封书信充分展示了武帝为保卫南粤而展现出的高超外交智慧,值得全文引用:

\begin{quote}
	蛮夷大长老臣佗昧死再拜上书皇帝陛下:老夫故粤吏也,高皇帝幸赐臣佗玺,以为南粤王,使为外臣,时内贡职。孝惠皇帝即位,义不忍绝,所以赐老夫者厚甚。高后自临用事,近细士,信谗臣,别异蛮夷,出令曰:“毋予蛮夷外粤金铁田器;马、牛、羊即予,予牡,毋与牝。”老夫处辟,马、羊、羊齿已长,自以祭祀不修,有死罪,使内史藩、中尉高、御史平凡三辈上书谢过,皆不反。又风闻老夫父母坟墓已坏削,兄弟宗族已诛论。吏相与议曰:“今内不得振于汉,外亡以自高异。”故更号为帝,自帝其国,非敢有害于天下也。高皇后闻之大怒,削去南粤之籍,使使不通。老夫窃疑长沙王谗臣,故敢发兵以伐其边。且南方卑湿,蛮夷中西有西瓯,其众半羸,南面称王;东有闽粤,其众数千人,亦称王;西北有长沙,其半蛮夷,亦称王。老夫故敢妄窃帝号,聊以自娱。老夫身定百邑之地,东西南北数千万里,带甲百万有余,然北面而臣事汉,何也?不敢背先人之故。老夫处粤四十九年,于今抱孙焉。然夙兴夜寐,寝不安席,食不甘味,目不视靡曼之色,耳不听钟鼓之音者,以不得事汉也。今陛下幸哀怜,复故号,通使汉如故,老夫死骨不腐,改号不敢为帝矣!谨北面因使者献白璧一双,翠鸟千,犀角十,紫贝五百,桂蠹一器,生翠四十双,孔雀二双。昧死再拜,以闻皇帝陛下\footnote{ 班固:《汉书》卷95《西南夷两粤朝鲜传》。赵佗之自称,原文作“蛮夷大长老夫臣佗”,殊难解。据胡守为考证,“夫”似为衍文,当作“蛮夷大长老臣佗”,今据胡氏之考证改。参见胡守为:《岭南古史》页38—39。}。
\end{quote}

书信中,武帝自称“蛮夷大长老”,充分展现了他以南粤为家,自认为本土“蛮夷”的自我认同。武帝的用词看似卑下,实则处处声讨着汉帝国的不义:正是汉帝国采取“别异蛮夷”政策、扣押南越使臣破坏了越汉之间曾经友好的关系,武帝称帝于番禺则更是反击汉帝国杀害其家人、掘毁其先人坟墓的正义之举。接下来,武帝又将挑起两国战争的责任推到汉帝国属国长沙国的头上,暗示汉文帝越汉之间仍有谈判余地。至于武帝乞求汉帝国给予封号的文字,正如日后的许多越南皇帝在一次次击退中华帝国的侵略大军后仍向帝国乞封一样,无疑是权宜之计,目的是为了避免战争长期化并导致祖国不得安宁。毕竟,外强中干的中华帝国十分容易满足于“万国来朝”的虚名。武帝乞封乃是在吃准了汉帝国的弱点后针锋相对地采用的巧妙护国策略,充满了在帝国威胁下保卫南粤的政治智慧。在书信最后一段,武帝称“老夫身定百邑之地,东西南北数千万里,带甲百万有余”,无疑是在向汉帝国表明:如果你们想把战争进行下去,南越国有实力奉陪到底。值得注意的是,武帝指出在南越国治下的西瓯人仍有自己的王。这表明南越国是尊重百越各族群习惯法与政治共同体的,绝不是碾平一切的大一统吏治国家。

我们不知道汉文帝在看到这封书信时是何种表情。可以推测的是,他的内心应该是无可奈何的。于是,南越国与汉帝国恢复了和平,武帝对外宣布“去帝制、黄屋左纛”,并对汉称臣。公元前156年,汉文帝死,汉景帝继位。在汉景帝朝,南越国曾有遣使赴汉“朝请”之举。当然,这一切不过是欺骗汉帝国、并使汉帝国自欺欺人地沉浸在“南越臣服”的幻象中的伪装。在南越国中,武帝一如既往地自称皇帝\footnote{司马迁:《史记》卷113《南越列传》}。南越国的这一政策,与日后越南人对付中华帝国的策略可谓如出一辙。

公元前137年,南越武帝驾崩,结束了他长达120年的波澜壮阔的人生,亦离开了他守护了67年的南粤。南越武帝人生的前后半段可谓截然两分。在前半生,他作为被暴秦灭亡了祖国的赵人降虏,为暴秦卖命,率领暴秦侵略军屠杀南粤百姓,并代表暴秦第二次灭亡了古蜀人的国家。然而,在洪水滔天的历史节点,他终于做出了正义的决断,毅然与帝国与自己的过去切割,使自己与数十万南下殖民者归化为越人,以南粤为家,建立了伟大的南越国。在后半生,武帝不但在礼仪、习俗上全面越化,亦为了保卫南粤殚精竭虑、不停奋斗。当汉帝国威胁到南越国的生存时,他更是毅然举兵北上,纵然自己留在北地的家人遭到汉帝国屠杀、先人坟墓被汉帝国掘毁,也要保卫南粤的自由。他后半生高尚的政治德性使他成为南粤史上的伟人之一,亦使他成为激励南粤人为自由而战的精神象征。他120岁的高寿则似乎在暗示,德性充沛的人一定会得到上天的奖赏。

\section{自立与归汉:文帝、明王、哀王朝的宪制斗争}

\indent 南越武帝赵佗以120岁高龄去世,他的继承人南越文帝赵眜是他的孙子,据说还是那位在古螺城殉情的赵仲始之子。在历史上,文帝的存在感远不如武帝强烈。然而,今天的我们仍能通过一处庞大的历史遗迹直观地体会到他曾经的存在,那便是著名的广州南越王墓。

公元1983年,一支工程队挖开了广州越秀区一座海拔不足50米的小山象岗山,一座大型陵墓出现在人们眼前。在用起重机打开两扇巨大的石门后,埋葬南越文帝的地宫在经历了2100余年的岁月后重见天日。地宫中出土的种种陪葬品足以令人震惊:十余件结构复杂、百越特色鲜明的铜熏炉骄傲地展示着南越国现金的铸造技术;九件铜提筒上装饰着各色花纹,其中一件上雕刻着四艘首尾相衔的羽人船,每船有羽人五名,有的在划桨、有的手持兵器、有的在杀人,展现着南越国军人杀俘祭海神的场景,反映了越人的虔诚与武德。七百余件铁器及一副铁制铠甲,表明当时的南越国已在大规模使用铁器。至于包裹文帝遗体的精美的丝缕玉衣,共由2300枚玉篇连缀而成,比西汉中山靖王刘胜墓中那件有名的金缕玉衣还要早十二年。最令人惊叹的,则是一件闪闪发光、由波斯舶来、盛放着十粒来自两河流域的药丸的银盒,以及五支长达1.2米的非洲象象\footnote{ 关于南越王墓出土文物的情形,参见张荣芳、黄淼章:《南越国史》,页376—388。}牙……这些陪葬品显示,南越国不但十分越化,更是一个重视海洋、海外贸易发达的国家。

对于南粤而言,南岭用以是抵御岭北帝国的坚固城墙,南海则是广阔的大后方。如前所述,早在史前时代,南粤的百越先民们便已驾驶着海船驰骋于南海波涛中。在南越国时代,南粤的海外贸易十分繁盛,来自波斯、两河流域、东非、东南亚的商品萃聚于番禺城,南越国人频繁地往来于东南亚及南海诸岛之间\footnote{ 张荣芳、黄淼章:《南越国史》,页314—315。}。究竟应当依托南岭、面向大海、建立一个自由的南粤,还是依附于大一统的汉帝国?围绕这一根本宪制问题而进行的政治斗争,是武帝去世后南越国政治史的主线。

文帝继位的第三年,即公元前135年,南越属国闽越乘武帝新丧之机发兵攻打南越的边境城邑。文帝缺乏武帝的果断,担心若南越国擅自发兵抵抗将触怒汉帝国,竟上疏汉廷求援。其时,汉朝的皇帝是刚刚上台四年的汉武帝。年轻的汉武帝就此获得了干涉越人内部事务的借口,因而对南越的“恭顺”大喜过望。在汉武帝的命令下,汉军从会稽、豫章两路出发,进攻闽越。同时,汉使唐蒙前往南越,将出兵的消息告知文帝。在南越国中,接待唐蒙的南越人为其准备了丰盛的食物,其中包括由巴蜀进口的枸酱。可是,南粤的热情好客却引来汉帝国居心叵测的暗算:唐蒙不怀好意地询问接待者枸酱产自何地,接待者直率地答称此酱系由西北方顺牂牁江(西江)而来。回到长安后,唐蒙经一番调查,发现牂牁江上游的南越属国夜郎实为蜀越枸酱贸易的中转站。于是,一个卑劣的计划在唐蒙心中浮现,他向汉武帝上疏称:

\begin{quote}
	南越王黄屋左纛,地东西万余里,名为外臣,实一州主也。今以长沙、豫章往,水道多绝,难行。窃闻夜郎所有精兵,可得十余万,浮船牂柯江,出其不意,此制越一奇也。诚以汉之强,巴蜀之饶,通夜郎道,为置吏,易甚\footnote{ 司马迁:《史记》卷116《西南夷列传》}。
\end{quote}

此一计划的要点在于:汉帝国不由正面进攻难以翻越的五岭,而应先制服越国西侧的夜郎,再由此顺江而下压制南越。在南越文帝对汉帝国毫无敌意,视之为盟友时,汉帝国居然仍在处心积虑地思考灭亡南越的计策。汉帝国之卑鄙无耻,于此可见一斑。汉武帝立即同意了唐蒙的提议,任其为中郎将,命他“将千人、赉食量及衣重者万余人”由巴蜀入夜郎,以贿赂手段收买了夜郎侯多同及当地各部落,并以夜郎之地为犍为郡\footnote{ 梁廷柟:《南越五主传》,页28。}。南越从此失去了西侧的藩篱。

出人意料的是,汉军尚未逾岭,闽越王郢即被其弟馀善发动政变杀害。其后,馀善无耻地将郢的头颅献给汉军,宣告投降,汉帝国不废一兵一卒即控制了闽越。为分化闽越,汉武帝立无渚子孙丑为“越繇王,奉闽越先祭祀”,又以卖国者馀善为东越王,“与繇王并处”\footnote{ 司马迁:《史记》卷114《东越列传》}。自此,南越又失去了东侧的藩篱。

在这之后,汉武帝的野心仍未满足。他派遣中大夫严助“以处分闽越事谕意南越”。面对汉使严助,南越文帝张惶无措,说出了如下一番丧权辱国的话:

\begin{quote}
天子乃为臣兴兵诛闽越,臣死无法报德……国新被寇,使者行矣,眜方日夜束装,入见天子\footnote{梁廷柟:《南越五主传》,页18。传统文献将文帝之名误作“赵胡”,此段记载原文为“胡方日夜束装”。然据南越王墓考古材料,文帝之名实为“赵眜”,乃于引文中改“胡”为“眜”。}。
\end{quote}

严助北去后,文帝的大臣们对于他轻易许下亲自“朝见”汉武帝的承诺感到震惊与愤怒。他们争相劝谏,指出如果文帝亲自入长安,将很可能被汉廷扣留、无法回国\footnote{梁廷柟:《南越五主传》,页18。}。毕竟,武帝时代吕后扣押三名南越使者的教训如在昨日,汉帝国糟糕的政治德性使南粤不可能相信其任何承诺。在群臣的劝谏下,文帝终于改变了主意,对汉帝国“称病,竟不入见。”然而,此前文帝对汉使严助屈辱的承诺已覆水难收,他的太子赵婴齐仍不得不赴长安“入宿卫”\footnote{司马迁:《史记》卷113《南越列传》}。这一举动,实为二十四年后南越国灭亡的导火索。

除了软弱地在汉使面前进退失据外,文帝并未在政治史上留下太多痕迹。公元前122年,文帝驾崩,结束了他并不光彩的一生。然而,南越国的苦难至此才刚刚开始。从长安回到南粤继承帝位的赵婴齐削去帝号,库藏了南越皇帝的玉玺,是为南越明王。明王卑微的态度使汉帝国自此“益易事南越”,屡次遣使敦促明王“入朝”,欲以此将他扣留在长安。明王明白自己若随汉使北上,定然有去无回,因此效仿文帝称病,“坚不肯入见”。不过,懦弱的他还是派出了王子赵次公入长安“宿卫”,一如文帝当年软弱地将他派往长安。更令人悲愤的是,明王不但在外交上继承文帝的卖国路线,在内政上更是任意妄为、疯狂地破坏南越国的法统。“乐擅杀生”的明王是一个残暴任性的君主。早年在长安时,他曾娶河北邯郸女子樛氏,生子赵兴。回到南粤后,他悍然不顾武帝制定的与越人通婚的政策,竟立樛氏为后,立赵兴为太子,全然不理他与越人王妃所生的长子赵建德\footnote{梁廷柟:《南越五主传》,页20。}。大概是因为明王过于残暴自恣,群臣无法阻止起有效的反对。公元前115年,明王病死,结束了他短暂而荒唐的统治。赵兴继位,是为南越哀王。因哀王年幼,樛氏居然获得太后身份、成为了南越国的实际掌权者。此时,南越国的法统已被明王破坏殆尽。

做为一名对南越国毫无认同感的北人,樛氏对南粤毫无感情,一心希望举南粤之地尽归汉朝。她力劝哀王出卖自己的国家、依附汉帝国。汉武帝闻知南越国中樛氏掌权、哀王年少的消息后,大喜过望。公元前113年,汉武帝令樛氏在北方的旧情人安国少季率使团出使南越国,要求樛氏、哀王“入朝,比内诸侯”\footnote{司马迁:《史记》卷113《南越列传》}。同时,汉卫尉路博德屯兵桂阳,以武力威胁做为安国少季的后盾。安国少季一到番禺,早已饥渴难耐的樛氏便与他厮混在一起。很快,两人便达成了两项彻彻底底的卖国协议:1)南越国如汉帝国“内诸侯”,三年一朝;2)解除边关防御。汉武帝迅速批准了这一协议,并增添了附加条款:3)“赐”南越丞相吕嘉银印,及内史、中尉、太傅印;4)废南越法,除黥劓刑,改用汉法;5)汉使留于南越国内“镇抚”。对于这些更为屈辱的条款,樛氏不但没有任何反对,反而高高兴兴地接受了它们。樛氏与哀王更开始整饬行装,为“入朝”汉武帝做准备\footnote{司马迁:《史记》卷113《南越列传》}。至此,南越国已经沦为与长沙国一样的地位,失去了一切的自主能力,几乎等同于被汉帝国吞并。

南越国最危险的时刻,到来了。

\section{南粤的大宪章运动:吕嘉革命}

\indent 在祖国马上就要灭亡的情况下,一个男子为了挽救南粤的自由站了出来,他就是南越国丞相吕嘉。

吕嘉是越人,自武帝时代起便已为官。在文帝、明王、哀王三朝,他都担任丞相。他的家族中官至长吏者达七十余人,男子皆娶王室女子,女子则尽嫁王族。在越人当中,吕嘉丞相有着极高的威望\footnote{司马迁:《史记》卷113《南越列传》}。面对樛氏与哀王肆意践踏南越武帝开创的伟大国度、面对祖国的沦亡,吕嘉感到出离愤怒。在哀王数次拒绝了他的劝谏后,他决心起兵反抗,继承南越武帝的遗志,为南粤的自由战斗到底。为了麻痹汉帝国,吕嘉首先称病不出,拒绝与汉使相见。汉使、樛氏、哀王对吕嘉的异动皆有所察觉,狠毒的樛氏遂计划借汉使之手杀害他。

为杀吕嘉,太后与哀王于宫中置酒,邀请汉使及群臣入宫赴宴,计划在宴会中下毒手。时为南越国之将的吕嘉之弟察觉到了危险,在吕嘉入宫后率兵驻于宫外。宴会开始后,樛氏不怀好意地对吕嘉说:“南越内属,国之利也,而相君苦不便行,何也?”意图以此语激怒汉使。正在汉使迟疑之间,警觉的吕嘉已察觉到了危险,起身急走而出。樛氏气急败坏,亲自持矛欲刺吕嘉,却被尚存良知的哀王阻止,吕嘉得以顺利逃出宫门与其弟的军队会合,率军回到家中。

此时,南越国群臣与百姓对于樛氏、哀王无耻卖国行径的不满终于爆发了。他们纷纷支持吕嘉丞相,几乎无人愿意听从那个与汉使安国少季私通的北人太后。番禺城中,困守王宫的汉使、樛氏与哀王已被宫外愤怒的南越爱国者们彻底包围,如同坐困于孤岛上一般。走投无路的樛氏欲下达杀害吕嘉的命令,却根本无人执行\footnote{司马迁:《史记》卷113《南越列传》}。

闻知樛氏与哀王已被支持吕嘉的官民困于宫中,汉武帝决定派出一支小部队越过边境以进行威慑。在汉武帝的命令下,汉济北相韩千秋及樛氏之弟樛乐统率两千人马跨过了边境。汉武帝失算了,他彻底低估了南粤人为自由而战的意志,更不知他面对的是一个怎样坚韧的国度。在汉军入境的刺激下,番禺城中南越国官民们的怒火彻底喷发了,一场由吕嘉丞相指挥的伟大革命开始了。吕嘉丞相决定武力进攻王宫,并发布了檄文:

\begin{quote}
	王年少,太后中国人也,又与使者乱,专欲内属,尽持先王宝器入献天子以自媚。多从人,行至长安,虏卖以为僮仆。取自脱一时之利,无顾赵氏社稷,为万世虑计之意\footnote{司马迁:《史记》卷113《南越列传》}。
\end{quote}

檄文中,吕嘉丞相列举了樛氏对南越国犯下的不可饶恕的罪行:首先,身为“中国人”的她把持朝政,与汉使私通,一意卖国投降,甚至为了讨好汉朝皇帝而将历代先王的宝器献出。其次,她还将南粤人贩卖到长安去做汉人的僮仆,做着人口贩子的生意。如此只顾一己之利的无耻行为,是一定会毁掉南越国的。为了南越武帝开创的伟大事业、为了南粤的自由,樛氏必须死。这一慷慨激昂的檄文,可称得上是有史以来第一篇南粤自立宣言。

在南越官民的支持下,吕嘉丞相的革命军攻入宫中,杀死了樛氏与哀王、杀光了汉帝国使团。吕嘉革命胜利了、南粤胜利了,文帝朝以来媚北归汉的外交路线被一扫而空,南粤回到了越人手中。 这场革命之于南粤的意义堪比大宪章运动之于英国的意义,足以证明面对北来僭主的威胁,南粤本土的志士们绝不会屈服,而是会反抗到底,保卫家邦,正如吕嘉丞相与南越国官民们所做的那样。


\section{南越国的灭亡}

\indent 吕嘉革命胜利后,由越人王妃所生、时为术阳侯的赵建德被立为王。此时,韩千秋、樛乐率领的两千汉军已攻破数城,正向南越都城番禺逼近。吕嘉丞相采取诱敌深入之策,将这股侵略者诱至番禺城附近全部消灭,侵略军头目韩千秋、樛乐皆被击毙。然而,这场小胜并不能左右大局,真正的考验尚未到来。闻知败报的汉武帝下令展开动员,欲以大军消灭南越国。在诏令中,汉武帝蛮横地将赵建德、吕嘉指责为造反者,宣布“令罪人及江淮以南楼船十万师往讨之”\footnote{司马迁:《史记》卷113《南越列传》}。公元前112年秋,侵略军集结完毕,整装待发。做为文明收割者汉武帝治下的军队,他们的实力远强于当年吕后派出的侵略军\footnote{参见刘仲敬:《经与史》,页150。}。他们的战斗序列如下:

\begin{quote}
	伏波将军路博德出桂阳,下湟水;
	
	楼船将军杨仆出豫章,下浈水;
	
	叛越归汉的南海人“归义侯”郑严、田甲分别任戈船将军、下濑将军,并出零陵,一下漓水、一下苍梧。
	
	除以上四人指挥十万侵略军外,又以粤奸“驰义侯”何遗发犍为郡夜郎之兵下牂牁江,命八校尉发巴蜀罪人,自西侧进攻,与路博德合击番禺。此外,东越王馀善亦出兵八千配合汉帝国作战,自东侧入侵粤东之揭阳\footnote{梁廷柟:《南越五主传》,页28-29。}。
\end{quote}

夜郎人与闽越人并不甘心做汉帝国侵略南粤的炮灰。夜郎部落“且兰国”的酋长因汉帝国强制征兵而发动反抗,杀死汉使,汉武帝不得不令巴蜀罪人转攻且兰国。兵驻揭阳的东越王馀善亦采取“阴持两端”的机会主义态度,“暗与南越通使”\footnote{刘仲敬:《经与史》,页150。}。可惜的是, 这些举动皆不能阻止汉军主力的南下。四路汉军中,郑严、田甲两路多有由降汉越人编成的伪军,并不积极推进。路博德一路全由罪人组成,战斗力不强,且要翻越衡山、骑田岭,因此亦进展缓慢。唯杨仆一路多有精卒,战斗力甚强。杨仆一路首先击溃梅岭(大庾岭)的南越守军,成为第一支深入南粤境内的侵略军\footnote{刘仲敬:《经与史》,页150。}。

面对汉帝国侵略军入境,吕嘉丞相毫无退缩之意,积极迎战。他命禆将庾胜赴大庾岭一带筑城固守,并于番禺城北各险要处布下防线。经过一年激烈的交战,至公元前111年秋,杨仆军攻陷寻峡(今清远飞来峡),顺北江而下,逼近番禺城西北二十里处的天险石门。石门一带江流狭窄,两山并峙,系番禺北面的门户。吕嘉命南越军积石于此处江中,阻截杨仆军战船南下的水道。然而,杨仆军经猛烈进攻后最终突破石门天险,缴获了一批南越军的船只与粮食。攻破石门后,杨仆军继续推进,终于攻至番禺城下,全军数万人列营于城外东南侧以待路博德军来会。不久后,姗姗来迟的路博德终于率千余先头部队到达战场,列营于城外西北侧\footnote{梁廷柟:《南越五主传》,页29。}。南越国的最后时刻,就要来临了。

番禺城依山傍水而建,曾经南越历代帝王扩建,乃一坚城。此刻,赵建德与吕嘉率南越军民正坚守城中,准备进行最后的抵抗。路博德到达战场后,杨仆军从东南面对番禺城发动了猛攻。至黄昏时分,番禺守军开始不支败退,杨仆军突入城中,大肆纵火。生性残酷的杨仆更命军士将投降者“皆缚以为虏”,并挖出墓中死人冒充南越军尸体,“自夸多获”\footnote{梁廷柟:《南越五主传》}。与此同时,路博德则在城外按兵不动,阴险地“遣使者招降者,赐印,复纵令相招”\footnote{司马迁:《史记》卷113《南越列传》}。这时,天已黑了下来。在一片火海中,番禺守军的士气崩溃了。为了逃离杨仆的魔掌,他们大批地逃向西北方,向路博德投降。次日晨,一切都结束了。番禺城中坚持抵抗的南越军民都已被侵略者杀死、烧死,幸存者都已向路博德投降。昔日繁华的南越国首都番禺城,此时已变成一片正在燃烧的废墟。

在战斗的最后关头,赵建德与吕嘉率数百部属乘船入海西去,准备继续与侵略者战斗。然而,有“降者贵人”向路博德透露了他们的逃亡路线。一批刚刚投降的无耻粤奸立即率船队向西追击,疯狂搜捕他们曾经的王与丞相。最后,赵建德、吕嘉分别被原南越校尉司马苏弘、越郎都稽俘虏。二人被俘后,吕嘉丞相被粤奸都稽斩首、首级被送往汉武帝处。不久后,南越王赵建德亦被侵略者杀害。汉武帝更将赵建德的头颅悬于长安宫殿之北阙,以夸耀汉帝国的“武功”。经历五王、93年的南越国,灭亡了。

当汉帝国侵略南越国时,汉武帝正在帝国境内巡视。汉武帝将其听闻南越灭亡之消息及收到吕嘉首级的地点分别改名为“闻喜县”、“获嘉县”,以表达心中的狂喜。侵略者洋洋得意,对一群无耻的大小粤奸进行了“封赏”,以奖励他们在灭亡南越国时做出的“卓越贡献”:驻守西江上游的南越武帝之孙苍梧王赵光未作任何抵抗即对汉投降,被封为“随桃侯”;揭阳县令史定降汉,被封为“安道侯”;桂林监居翁“谕瓯雒以四十余万口属汉”,被封为“湘城侯”;瓯雒左将黄同斩西瓯王,被封为“下鄜侯”;苏弘、都稽因俘获赵建德、吕嘉,分别被封为“海常侯”、“临蔡侯”…\footnote{参见欧大任:《百越先贤志》,页23;梁廷柟:《南越五主传》,页30。}…

与众多向侵略者献媚的粤奸相比,吕嘉丞相的形象异常伟岸。做为一名越人,他无愧于自己的百越血统,无愧于自己的百越祖先,亦没有背叛南越武帝开创的伟大事业、没有背叛南粤的自由。在南粤即将沦陷的时刻,是他挺身而出发动革命,告诉帝国南粤仍有有血性之人。在侵略者的大军入境的时刻,是他组织南越国军民进行不屈的抵抗。在番禺城陷落的时刻,是他仍不放弃希望,继续战斗。他奋斗到了最后一刻,如一颗流星般划过南粤历史的天空,留下了稍纵即逝的耀眼光辉。他 没有背叛南粤军民,正如项羽没有背叛江东父老、罗伯特·李将军没有背叛弗吉尼亚。他化身为一个符号,象征着南粤的不屈精神与战斗意志。被他扶立的南越末代君主赵建德亦没有辱没南越武帝的威名,壮烈殉国。

在文明收割者汉武帝的打击下,南越国灭亡了。然而,南粤反抗帝国、争取自由的斗争绝不会停止。在这之后的两千年里,南粤人将一次又一次地为自由而战、为尊严而战。


