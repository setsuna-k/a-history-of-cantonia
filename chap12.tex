\chapter{在群山与大海中战斗:南粤大航海时代的开启}

\section{蛮族的反抗:海南、广西与粤西北}

\indent 16世纪,当西方人来到南粤、珠三角的精英迅速自我华夏化时,海南、广西和粤西北的百姓尚未太受到此种历史剧变的影响。他们依然维持着千万年来的传统生活方式,依靠百越习惯法自由地生活,不断进行可歌可泣的起义。随着西方人的到来,驻粤明帝国侵略军部分掌握了先进的西式武器,得以更大规模地屠杀南粤起义军,犯下了无数令人发指的罪行。

在海南岛上,俚人的直系后裔黎人世世代代地生活着、繁衍着。英雄的海南人曾在第二次北属时代初期发动持续近七十年的反抗,迫使汉帝国放弃海南。公元6世纪,是他们最早向冼夫人效忠,率先投入冼夫人保卫南粤自由的伟大斗争中。明帝国侵占海南后,因岛上的土著共同体至为坚固,洪武社会主义未能打散当地原有社会结构。明帝国虽在岛上设立了由流官统治的府、州、县,并驻有近两万名侵略军,但岛上的大部分民众依旧由黎人各部落酋长管理。这些酋长被明帝国“封”为“土官”,管理着被称为“峒”的土著共同体,仅在名义上对明臣服\footnote{贺喜:《亦神亦祖:粤西南信仰构建的社会史》,页66—67}。

海南的明帝国流官多为贪酷残忍之辈。他们大肆向黎人勒索财物,索取耕牛、麖皮、蜂蜡等物。岛上的侵略军则时时诬陷黎人“反叛”,进而杀良冒功。1469年,儋州七方峒黎民在符那南、符英率领下发动起义,明帝国以不久前残酷屠杀了大藤峡起义瑶民的刽子手韩雍前往镇压。韩雍一面以剥皮抽肠、凌迟处死等酷刑恐吓黎民,一面进兵。至次年,符英壮烈战死,起义失败。六年后,落窑峒黎民在符那推率领下发动起义,又遭镇压\footnote{贺喜:《亦神亦祖:粤西南信仰构建的社会史》,页77—78}。

血淋淋的屠杀吓不倒英勇的海南武士。1501年,震动全岛的符南蛇起义爆发了。符南蛇于1470年出生于七方峒黎人酋长家庭中,七岁开始从师读书、学习儒家经典。他从小便富有正义感,立志要为家乡、为南粤的自由而战。十四岁时,他曾写下ni 样的语句:

\begin{quote}

意欲重洗山河。不甘随地老,立志破天荒\footnote{苏英博主编:《海南名人辞典》,页33}!

\end{quote}

1496年,海南发生水旱之灾,饥民流离失所。在此情况下,岛上的明帝国官僚不但不加救助,反而依旧横征暴敛,致使饿殍遍野。为拯救一方父老的生命,符南蛇决定起兵反抗。1501年夏,他召集海南各黎峒首领于七方峒誓师,斩杀勾结明官压榨同胞的酋长符那月,刻令箭传示岛上三府十州。全岛之民领其令箭、纷纷响应。至闰七月,起义军人数膨胀至万余,开始围攻儋州。八月,围攻昌化。九月,分兵进攻临高。岛上的二万余侵略军分五路“围剿”,但因骄横冒进,很快便有一路全军覆没,被击毙两千余人,其余四路遂望风奔溃。至十二月,义军人数已高达十万,在保吉大破岛上明军主力,毙敌数千人。至此,海南岛近半地域已被解放\footnote{《广东通史》古代下册,页202—203}。

声势浩大的起义令明廷极为震恐。明帝国拼凑了十万大军,在两广总督毛锐的指挥下气势汹汹地渡过琼州海峡展开大规模镇压。毛锐集中重兵围攻义军各战略据点,使义军首尾不能相顾。经一年战斗,义军控制区日益缩小。1502年十二月,符南蛇在危局之下率兵出击,中箭负伤,退回七方峒赴水殉难。侵略军随后攻陷七方峒,展开惨绝人寰的屠戮,杀害2560余人、俘1400人、毁房屋1200余所\footnote{《广东通史》古代下册,页203}。

符南蛇牺牲了,但他英勇反抗的精神绝不会消失。在这种精神的激励下,海南百姓又于1539—1541年、1549—1550年、1597—1599年、1612年接连发动起义。在起义中,无论是山中的黎民还是沿海平地被称为“汉人”的编户齐民都拿起武器一同战斗。这时,明帝国侵略军已部分掌握了西式火器,屠杀效率更为提高,每次镇压都有成千上万人倒在侵略者的屠刀下。然而,明帝国始终不能终止海南人的反抗。至明末,明帝国在海南投入的兵力“无虑十万”,伤亡惨重、“财力耗蔽”,彻底陷入战争泥潭\footnote{《广东通史》古代下册,页204}。在后来的清帝国治下,海南一直未被完全纳入帝国的编户齐民体系。直到1887年,张之洞、冯子材等人仍致力于在海南岛中央的五指山中修筑道路\footnote{贺喜:《亦神亦祖:粤西南信仰构建的社会史》,页92}。

在整个16世纪,广西境内的起义亦贯彻始终。从明初开始,广西柳州城外僮人的起义便接连不断。在马平县(柳州附廓县),当地的起义民众转战于柳江两岸,屡屡袭击明帝国侵略军,使明廷大为头痛。1507年秋,明两广总督陈金指挥十万大军兵分五路大举进攻马平。陈金系一胆小怯懦的鼠辈。他装出“大兵进剿”的样子,实则根本不敢与起义军交锋。至次年春,陈金用兵半年、未发一矢。为向朝廷交差,他丧心病狂地指使部下屠杀了附近的7000余名和平居民,将他们的首级砍下来当做“战利品”报功。明廷则以此为“大捷”,对陈金加官进爵\footnote{《广西通史》卷1,页353}。此后,马平僮民又在1522、1544年两次起义,一度切断柳州至庆远的水道。直至1545年,起义才被明两广总督张岳残酷镇压下去,死难起义军民达4000人以上\footnote{《广西通史》卷1,页354—355}。

上林、忻州两县间的八寨也发生了波澜壮阔的起义。八寨地区方圆250里,当地瑶、僮之民在深山中结寨而居,筑有八到十个大寨,有村128座、万余人口。在14—15世纪,当地百姓发动过多次反明起义,曾于1457年逼近南宁。明帝国视八寨为眼中钉、肉中刺,急欲处之而后快。1528年四月,明廷以大儒王守仁(阳明)总制两广、江西、湖广军务,率六千军队分三路进攻八寨。八寨百姓以木石、镖枪、药弩为武器,依山凭险节节抵抗,双方均伤亡惨重。直至今日,广西境内仍有一首叙述当时战况的民谣:

\begin{quote}

天蒙蒙,喊杀又喊冲。
帽子飞上天,鞋子滚下坡。
天蒙蒙,相杀在田中。
蒺藜刺进肉,惨叫震九重。
天灰灰,人头成山堆。
人头还比猪头贱,谁见眼泪不纷飞\footnote{《广西通史》卷1,页347}?

\end{quote}

经过如此惨烈的战斗,明帝国侵略军在付出巨大代价后镇压了八寨百姓,屠杀2000余人。这场战争成为王守仁死去前的谢幕之战,受到后世无数无耻文人的吹捧。然而,帝国文人光鲜照人的“赫赫武功”决不能掩盖其丧心病狂的残酷暴行。此后,八寨百姓依然没有屈服。在1570年代,他们发动了全民大起义。1579年十一月,明两广总督刘尧诲、广西巡抚张任纠集十万大军对八寨展开了灭绝性屠戮。经103天的血战,侵略军杀害、俘虏16900余人。在如此惨烈的屠杀下,八寨残民所剩无几,持续两百余年的八寨大起义被彻底镇压了\footnote{《广西通史》卷1,页348}。

在漓江中游阳朔、昭平之间的延绵三百里的府江,英勇的反抗一直没有停止过。当地的明帝国屯军与瑶人联合起来,一同反抗明帝国的暴虐压迫。在明初,当地屯军首领陈华四率部与瑶民一同作战,进行了长达十余年的抗明战争。直至15世纪初,明帝国才将ni 次反抗镇压,并令人发指地将当地编户“汉人”屠戮殆尽。1494年,蒙山县农民在覃照扶、蓝公永的率领下号称“飞天过海”,发动大起义。义军攻克永安、修仁、荔浦,击毙明指挥张敞,声势十分浩大。当年冬,明监察御史林廷选、两广总督闵珪率六万大军进行镇压。义军依托山势节节抗击,被避入崖高山险的最后据点通天岩。在通天岩,义军用简陋的弓箭、石块进行了长达一个月的壮烈抵抗,终因寡不敌众战败。此次起义中,明帝国侵略军毁村寨180座、屠杀6000余人、掳走400余人。1511年,又有贺县农民覃公浪联合怀集及广东连山百姓发动大规模起义。两广义军在广西境内会师,控制府江沿线达三个月之久,终被明广西总兵柳文镇压而失败。侵略军再次大开杀戒,毁村寨200余座、屠杀4470人、掳走1200余人。经此两次残酷屠杀,伟大的府江百姓仍未放弃斗争。1545年,贺县、连山百姓再次联合,在倪仲亮、邓良朝、梁荣李三弟的率领下大举出击,活跃于两广边界,并一度攻入湘南衡、永、郴、桂等州县,沉重打击了明帝国的凶焰。两年后,明两广总督张岳、总兵陈圭调集八万大军,以东西对进的方式进攻义军。经两个月激战,双方均伤亡惨重,寡不敌众的义军终告失败。明帝国侵略军则对府江地区展开第三次大屠杀,屠戮3000余人、掳走280余人、270余头马牛、1500余件器械\footnote{《广西通史》卷1,页347—348}。在府江的历次起义中,操南粤本土语言的瑶人与操西南官话的明初“屯军”后裔并肩作战、视若兄弟,一同谱写了南粤历史上的一段追求自由的悲壮史诗,亦用事实有力回击了今日一些偏激分子视讲西南官话的桂柳民系为应予驱逐的“非我族类”的谬论。明帝国侵略军在短短半个世纪内制造的“府江三屠”,则是岭北帝国对我南粤欠下的又一笔血债,应被今日的南粤人永远牢记。

在距明广西省城桂林仅80里的古田县,壮烈的起义亦持续了近百年。1500年前后,当地民众在韦朝威、覃万贤的率领下起兵反明。在一次战斗中,韦朝威战死、覃万贤下落不明,起义军的领导权遂归入韦朝威之子韦银豹之手。此后,义军在韦银豹的率领下采取声东击西的游击战术,屡挫侵略军锋焰。至1520年代,义军已攻克古田、洛容县城,其控制区到达桂林近郊,切断桂林、柳州间的江道。此后四十年中,义军不断对外出击,控制区扩展到桂东北的绝大部分地区。1564、1565年,义军两次攻克桂林城,杀明广西参政黎民衷,势力达到顶点\footnote{《广西通史》卷1,页531—532}。韦银豹起义军的长期战斗使明帝国疲于奔命,决定调集庞大部队予以一次解决。1570年,明广西巡抚殷正茂调集十四万大军进攻古田。经三个多月的激战,起义军战败,韦银豹遭叛徒出卖、被送往北京杀害,倒在侵略军屠刀下的古田军民达7000人以上。明帝国更将古田改为“永宁州”,得意洋洋地宣称他们已经平定了当地人的抵抗,当地将“永远安宁”\footnote{《广西大百科全书 历史上》,页421}。此种自欺欺人的行为无疑反映了明帝国意图彻底镇压粤人的痴心妄想。事实上,只要岭北帝国依然侵占着南粤、侵害着粤人的自由,粤人便永远不会屈服、永远不会抛弃对自由的向往。

在粤西北的罗旁山区,当地以瑶民为主的百姓自家乡被明帝国侵占起就承受着惨绝人寰的压榨。明帝国对他们横征暴敛,驱赶他们入山砍伐竹木,视他们如奴隶一般。1457年,驻守当地的明参将范信竟诬陷安泰、永平二乡之民为“贼”,灭绝人性地将二乡百姓全部屠戮。进入16世纪后,明帝国又对当地的特产楠漆、砂仁、黄蜡、蜂糖、皮张、竹木强行抽税,使当地民不聊生。早在15世纪,英勇的罗旁人就曾举行过多次武装反抗。这些反抗虽都被血腥镇压下去,但更大的反抗随之而来。1515年,一支一千余人的罗旁起义军曾一度攻克粤西高州,沉重打击了明帝国在南粤的暴虐统治。此后数十年间,明帝国屡屡对罗旁用兵,皆不能获胜。当地百姓中流传着这样的歌谣:

\begin{quote}

官有万兵,我有万山。兵来我去,兵去我还\footnote{王时阶:《明代罗旁瑶族农民起义》}。

\end{quote}

到16世纪中期,明帝国已对罗旁连续用兵数十年,除损失累累外仍一无所获。1565年,明帝国制定了在罗旁山区“随山刊木,设立营堡,就将近田地给与戍兵耕种”的战略。此种阴谋使罗旁百姓的武装斗争暂停了十年。在ni 十年中,罗旁人并没有屈服,而是在酝酿更大规模的反抗\footnote{《广东通史》古代下册,页205}。1575年,一场规模宏大的起义爆发了。当时,入山屯垦的明兵已经松懈。起义军趁夜袭击侵略军营堡,“杀兵焚寨”。时任明两广总督的刽子手殷正茂见起义军势大,急忙上疏明廷请求“进剿”罗旁。不久后,殷正茂离任,其继任者凌云翼继续上疏,提出以二十万大军进攻的“大征罗旁”之策,得到明廷批准。次年二月,凌云翼在高州、肇庆两地纠集自广西、浙江等地开来的二十万侵略军,开始向罗旁山区进发,这是明帝国规模最大的一次对粤用兵。十一月,侵略军完成对罗旁山区的包围。十二月二十日,他们兵分十路开始进攻。起义军“竖栅为寨”,用简陋的木棍、毒弩、石块为武器,与装备了西式火器的侵略军展开殊死搏斗,使侵略军每前进一步都要付出巨大代价。经数个月血战,侵略军一边缓慢推进一边大肆屠杀,以“搜山索峒,几无遗类”的手段攻破义军村寨564所、杀害16104人、俘虏23100余人,逃进深山冻饿而死的起义军民更是不知凡几。他们中的许多人都在山中坚持战斗到了最后一刻,绝不向侵略者屈服。1577年三月十五日,战役结束,明帝国在当地设罗定直隶州,留兵两万驻扎。至1579年,罗旁起义军最后的余部也惨遭镇压。明帝国随即将罗旁瑶民的土地全部没收,于1581年将田58410亩分给当地侵略军\footnote{《广东通史》古代下册,页205—206}。轰轰烈烈的罗旁大起义,就这样在明帝国血红的屠刀下被残酷镇压了。罗旁军民宁愿冻饿而死也要坚持抵抗的事迹,反映了我南粤人至为刚健的战斗精神。这段壮烈的历史,足以令今天的我们感到无比自豪。

14—16世纪之间,在明帝国的持续屠杀下,桂北的语言结构发生了重大变化。当地土著语言日渐式微,以“平话”的名称存留至今。由明帝国屯军带来的西南官话逐步成为当地主流语言,使桂林、柳州一带形成了桂柳民系。需要注意的是,桂柳民系的血统绝非全为来自岭北的军人,他们的大部分血统仍来自被帝国编户的南粤土著。这些土著中的不少人讲起了西南官话,采用了发明北方祖先、建立宗祠、发明宗族的形式给自己的共同体套上一层“合法”外衣,逃过了明帝国的屠杀。因此,称桂柳民系全是北方殖民者无疑是偏颇的。但无论如何,桂北在明帝国治下沦为官话区都是南粤文明的重大损失。在桂柳民系真正融入岭南本土之前,桂北一直是南粤身上的巨大伤口。

\section{十六世纪的粤东海上群雄}

\indent 在粤东潮州的韩江流域生活着南粤的潮汕民系。自10—11世纪以来,大批闽人自人口日益密集的闽越向西移民,与粤东的土著僚人后裔畲人混居,融合为一个讲闽南语系语言潮语、又与闽南人相当不同的新族群。勤劳的他们迅速地开发了韩江三角洲,使当地成为一块适合耕种的沃土。自第六次北属时期起,他们被南粤其余居民称为“福佬”,以示他们的祖先来自福建\footnote{参见陈训先:《潮汕先民探源》}。在15—16世纪,随着韩江三角洲开发的日益成熟,潮州出现了地少人多的问题。因此,当地百姓纷纷出海谋生,从事国际贸易。然而,施行愚昧海禁政策的明帝国却极力阻止他们出海。这样一来,潮州人便自15世纪起不断发动武装反抗,全力冲破明帝国的海禁政策,为天赋的自由贸易权利而战。

早在1426年,便有潮州通事(翻译)刘秀招引日本船至饶平东里港湾,当地各村百姓纷纷“赴船领货”,公开进行外贸。明帝国虽于五年后重申海禁,但根本不能阻止潮州人的反抗。1460年,渔民黄于一、林乌铁在揭阳县夏岭首建义旗。当时,明帝国官僚在夏岭纵容无赖侵占百姓田产。极富正义感的黄、林二人遂发动父老乡亲武装反抗,乘船四处袭击侵略军。他们在当年袭击海阳县,击败了一大股侵略者。明帝国对此又惊又急,将他们诬称为“海寇”,纠集大兵进行镇压。同年,明潮州知府周瑄(三晋阳曲人)首先设计诱捕林乌铁,将其残酷杀害,接着进兵夏岭,与义军鏖战四十余日,杀害了黄于一。失去领袖的起义军民群龙无首,陷入大乱。这时,周瑄突然摆出一副仁慈面目,称“盗魁既得,余可抚而下也。”受其蛊惑的夏岭军民乃纷纷解甲出降,上缴大海船150艘,受“抚”者多达1237户。卑鄙至极的周瑄见夏岭百姓已被解除武装,就派兵对他们展开惨无人道的大屠杀,在当地造成了数千人死亡、幸存者逃散的惨剧\footnote{郑广南:《中国海盗史》,页166}。

经此惨烈屠杀,夏岭百姓并未屈服。战事结束后,幸存者返回家乡,埋葬亲人朋友的尸体,继续拿起武器战斗。1464年,第二次夏岭起义爆发了。义军四处攻城掠地,纵横于海阳、揭阳二县,击毙揭阳守将刘琛、通判刘恭。明廷乃再以大军“征剿”夏岭,又一次疯狂屠戮起义军民,将三千余名幸存者强行迁入内陆\footnote{陈春声:《从“倭乱”到“迁海”——明末清初潮州地方动乱与乡村社会变迁》}。义军余部逃出重围,转战于赣南、闽西山区,一度攻占江西安远县和福建上杭县,声势颇大。至1466年,这支义军亦在明帝国的重兵围剿下归于失败,死者1400余人\footnote{郑广南:《中国海盗史》,页167}。

夏岭起义失败了,但潮州人的反抗绝不会停止。在16世纪上半页,他们一次又一次起兵,纵横海上,沉重打击了明帝国的海禁政策。这些起义虽都被镇压下去,但更大规模的反抗随之而来。1523年,吴越发生日本“贡使”自相残杀的“宁波争贡”事件。事后,明帝国裁撤浙江、福建市舶司,在吴越、闽越厉行海禁,并于1548年制造双屿大屠杀。在此情况下,大批吴闽沿海居民纷纷与日本海商合流,展开了激烈的抗明战争。当时,日本正处于波澜壮阔的战国时代,许多在战争中失去领地的武士纷纷浮海冒险,成为起义军的中坚作战力量,明帝国因而诬称义军为“倭寇”。然而,就连明帝国官僚自己也不得不承认,“倭寇”中的大部分并非日本人:

\begin{quote}

(倭寇中)夷人十一,流人十二,宁、绍十五,漳、泉、福人十九\footnote{转引自郑广南:《中国海盗史》,页181}。

\end{quote}

可见,“倭寇”中的大多数人都是吴越宁波、绍兴及闽越漳州、泉州、福州的沿海百姓,他们是一支不折不扣的抗明起义军。在吴越,徽州大英雄王直的船队纵横海上数年,一直受到其吴越同胞的热烈欢迎。1557年,大吴奸、明浙直总督胡宗宪以极度卑鄙的手段诱捕王直,于两年后将其杀害。此后,“倭寇”转战于闽、吴沿海,于1565年被所谓的“抗倭名将”戚继光、俞大猷杀戮殆尽。在此期间,亦有相当数量的粤东百姓高举义旗,与闽、吴两地的“倭寇”一同抗明,从而演出了十六世纪南粤历史中最壮阔的一幕。

1544年,潮州饶平县人李大用起兵反明,集船百艘袭柘林下岱山,后登陆作战失利,被迫撤退。在海上,义军船队遭遇风暴,李大用本人及大部分官兵皆不幸落水身亡,只有其部将林国显率船两艘逃脱。林国显乃李大用同乡,人称“小尾佬”。他召集部众、重整旗鼓,向往自由的百姓纷纷来投,势力很快便壮大起来。怯懦的明帝国侵略军无法战胜他,便阴险地囚禁了他的儿子,强迫他投降。林国显断然拒绝招降,亲赴日本招兵买马。不久后,他就威风凛凛地带着一批日本“海贼”杀回南粤,攻克上里林家围、福建诏安县梅岭,与日本海商大举贸易,完全无视明帝国的海禁政策。林国显开创的通商路线给粤闽百姓带来巨大益处,明帝国对其无可奈何,只得恶狠狠地称“梅岭人悉从盗”。不幸的hay ,林国显于1565年在潮阳海面与侵略军舰队的一次战斗中意外阵亡。他牺牲后,粤东人的反抗并未中止。不久后,他的族孙林凤、侄女婿吴平便会创造出惊人的事业\footnote{郑广南:《中国海盗史》,页213}。当时,在南澳岛上活跃着的另一支海上义军便是来自闽越的吴平部。吴平系闽越诏安四都人,率部常年活跃于南澳岛与闽越浯屿之间。1565年初,吴平率大批闽、粤义军登陆,进攻粤东北重镇梅州,后又以百余艘船、万余部众进攻闽越诏安。时分任福建总兵和南赣总兵的两大刽子手戚继光、俞大猷决定联手进攻南澳岛,彻底屠尽岛上的粤、闽义军。当年九月,戚继光下令强征南澳岛附近的所有民船用以运兵,亲率步兵登陆岛上,与吴平部展开激战。十月初,俞大猷率船三百艘赶到,从海路封死了吴平的退路。戚继光的步兵步步紧逼,将岛上义军逼入最后的据点云澳,发动总攻。吴平率众据木栅坚守。栅破之后,义军拼死搏杀,战斗到最后一刻。大批勇士不愿当侵略军的俘虏,纷纷投海、跳崖而死,其余的人都被侵略军杀害,死难的粤、闽义军多达15000人,吴平本人则率残部在粤东、粤西沿海转战一阵后逃亡越南方向,不知所踪。对于戚继光、俞大猷不留俘虏、斩草除根的残酷行径,一首民谣如是描述:

\begin{quote}

俞龙戚虎,杀贼如土\footnote{转引自郑广南:《中国海盗史》,页223—224}。

\end{quote}

南澳岛大屠杀绝不会打消粤东人的反抗意志,只会使他们对残暴的明帝国更加愤怒,展开更大规模的抵抗。仅仅一年后,潮州惠来县人林道乾就率船五十艘占领南澳岛,使该岛再次变为抗明基地。林道乾出生于一个小康家庭,幼时读过几年私塾,是个沉毅勇武、是非分明的人。他曾在惠来担任县吏,后因不满于衙门的贪酷而顶撞知县,率众出海投靠吴平。他与吴平麾下爱将曾一本(闽越诏安人)交好,一同联合“倭寇”、互相扶持。吴平死后,两人的关系变得微妙起来。明帝国官僚描述当时的情形称:

\begin{quote}

海寇林道乾、曾一本、吴平辈,乘倭啸聚,初不过数十人……及倭灭,而吴平统有其众,流毒沿海。道乾、一本亦各树党援,以平为犄角,以抗王师。后平窜海,莫知所往,党溃散,于是道乾、一本复纠合之。林、曾二贼,其势大焰,势不相下,互相雄长,为岭东连年大患\footnote{郑广南:《中国海盗史》,页226}。

\end{quote}

林道乾进入南澳岛后,俞大猷率兵大举“进剿”。林道乾闻警,当即率众转移至台湾海峡中的澎湖列岛。因俞大猷率兵紧追,林道乾舰队又继续东航至台湾北港,在此地泊舟,于打鼓山下登陆,一面修养生息扩大舰队、一面计划东山再起。林道乾的部队对台湾原住民非常友好,一则美丽的传说由此诞生。据说,林道乾的小妹曾随舰队同至北港。部队登陆后,小妹在打鼓山之巅埋下金银财宝十八箱,打鼓山之巅遂被命名为“埋金山”。从此,埋金山上长满了甘美异常的奇花异果。曾有一些樵夫见过、品尝过这些花果,但若他们以后还想再找到这些花果,便只能一无所获。这一如《桃花源记》般美丽的传说之所以能出现,与林道乾和原住民间的友谊是分不开的。在台湾人心目中,林道乾兄妹带来了财宝和花果,是他们的恩人。这则传说反映的,无疑是粤台间的一段历史佳话\footnote{郑广南:《中国海盗史》,页227—228}。

不久后,一批日本“海贼”来到北港,与林道乾舰队联合。然双方很快便发生火拼,林道乾失利,只得泛海前往浡泥(今文莱)。在浡泥休养生息后,林道乾舰队再次跨过南海返回潮州,时时出兵袭扰粤西雷州、海南岛等处。1567年十二月,林道乾率三千兵力包围澄海县溪东寨。当地明帝国侵略军无力抵御,只得请求乡绅陈求默组织民兵防御。至次年三月,溪东寨中已经粮尽,可附近的明军皆袖手旁观、不发一矢。寨破之后,心灰意冷的陈求默遂率残部二百余人逃入闽南山区,开始反明作战。林道乾因在溪东寨遇到强烈抵抗,破寨后采取了报复行动,屠杀男女千余人\footnote{陈春声:《从“倭乱”到“迁海”——明末清初潮州地方动乱与乡村社会变迁》}。此次暴行,实为林道乾人生中的一大污点。

溪东寨既破,粤东明帝国侵略军大为震恐,纷纷避战不出。林道乾军得以纵横海阳、潮阳、揭阳、澄海诸县村落,如入无人之境。束手无策的明帝国官僚遂想出一条阴险的毒计。他们决定“招抚”林道乾,并向林道乾赠送土地,以利用其力进攻与其有矛盾的曾一本。林道乾果然“受抚”,被“安插”于潮阳县下尾村。在潮州境内一同“受抚”的“海寇”,还有朱良宝、魏朝义、莫应敷,三者分别被“安插”于澄海县南洋寨、大家井、南澳岛东湖寨。当时,曾一本纵横南粤的整个海岸线,势头正盛。1568年二月,他率舰队突袭雷州,击毙明守备李茂才,接着又开入珠江口,一度进至广州城下。林道乾、朱良宝等人在“受抚”后非但未如明帝国期望的那样出兵攻击曾一本,反而对其进行接济\footnote{郑广南:《中国海盗史》,页230}。明帝国驱使粤人、闽人自相残杀的毒计,至此全盘落空。次年四月,明两广总督熊桴、福建巡抚涂泽民、总兵俞大猷等集闽粤之兵大举“会剿”曾一本。五月二十六日,曾一本在潮州广洋澳作战失利、不幸遭俘杀,其部众死者数千人\footnote{凌云翼:《苍梧总督军门志》卷21}。

曾一本死后数年内,林道乾仍据守下尾村,坐拥良田千余亩,大兴海上贸易。他不顾明帝国的海禁政策,开辟澄海县河门渡港口,积极从事南洋贸易,实力愈加壮大。粤东各地饱受明帝国残害的百姓纷纷来到下尾村投靠林道乾,使时为内阁首辅的明帝国权臣张居正亦不得不承认:

\begin{quote}
	广中数年多盗,非民之好乱。本于吏治不清,贪官为害耳\footnote{转引自郑广南:《中国海盗史》,页230}!
\end{quote}

张居正所言虽有一定道理,但并未抓住问题的实质。事实上,无论统治南粤的岭北帝国官僚是贪官还是“清官”,粤人都绝不会放弃对自由和尊严的向往、都会斗争到底。1573年,双手沾满粤人鲜血的刽子手、明两广总督殷正茂决定以大军“剿灭”林道乾。林道乾率部进入南洋寨,与其盟友朱良宝合兵一处。朱良宝系南洋寨本地人,深得父老信任。他于是年二月率船七十艘袭击粤西阳江,旋即返回家乡积极备战。朱良宝率领南洋寨军民深挖壕沟、建筑高垒,与部下同甘共苦。其部下官兵皆削发明志,发誓为朱良宝效死。不久后,明军先锋到达南洋寨下。南洋寨军民居高临下,以“佛郎机”火炮、西式火绳枪射击,继而开寨出击,尽歼敌兵。张居正闻之极为震怒,下令惩办失职官员,以总兵张元勋率兵再攻。1574年二月,张元勋率大军出征,传檄珠三角、粤东各处发兵相助。侵略军兵临南洋寨下之日,以大字书写榜文“有投降者免死”。然半个月内,寨中无一人回应,侵略军遂于三月十日发起进攻。经过激战,南洋寨陷落,自朱良宝以下官兵1250人皆宁死不屈,“至死犹斗”,全部壮烈战死。南洋寨军民的英勇战斗感动了时人。他们被比作汉初壮烈的“田横五百士”,受到粤人及一切热爱自由的人们的永远怀念\footnote{凌云翼:《苍梧总督军门志》卷21}。

南洋寨陷落后,魏朝义、莫应敷等人皆相继毁寨投降。明帝国声称他们在南洋寨中找到了林道乾的遗体,并对其遗体施以凌迟之刑。然而,这不过又是明帝国无耻的自吹自擂。事实上,林道乾在乱军中奇迹般地逃了出去。他浮海离粤到达柬埔寨,由此开始了他异常传奇的后半生。他在柬埔寨改名为林梧梁,向柬埔寨国王献上宝物,获得“把水使”之职,得以继续于南海中活动。他曾率两千柬军袭击暹罗(泰国),失利退去。1578年,林道乾潜回潮州,尽发昔日所藏财物,招募百人而还。两年后,明廷听暹罗使臣说林道乾正在柬埔寨,乃遣使赴柬,要求柬人捉拿林道乾。林道乾察觉到危险,于1581年率两千部署突出重围,到达泰国南部的穆斯林自治城邦国家北大年。北大年苏丹见林道乾能力出众,便将其招为驸马,令其皈依伊斯兰教、掌管北大年港口。后来,林道乾在测试新制火炮时不幸因爆炸事故身亡\footnote{李才进主编:《三湾史略》}。北大年百姓怀念他的善政,便给当地起了“道乾港”的别称,以示对这位传奇南粤海上英雄的缅怀\footnote{张廷玉:《明史》卷323《外国四》}。

除林道乾外,另一位纵横南洋的海上英雄是饶平人林凤。他曾率军进攻西属菲律宾,将南粤的海上兵威推至极致。在介绍林凤的伟绩前,我们不应绕过著名的许栋、许朝光父子。许栋乃饶平县黄岗人。他身体健壮、武艺高强,率领一批乡民常年纵横于粤东、闽南海上,与日本“海贼”并肩作战,给明帝国造成了不小的麻烦。1553年,许栋被养子许朝光(潮阳人)谋杀。许朝光尽领养父旧部,继续战斗,攻占潮阳牛田洋,在当地建立税收体系,实与独立政权无异。1563年,对许朝光束手无策的明帝国官府只得对其进行招降。许朝光虽答应赴潮州城接受“招抚”,但从其“受抚”过程来看,无论如何都更像明帝国在向他投降:

\begin{quote}

(许朝光)乃驾船数十艘溯流上,旌旗蔽空,甲光耀日。舣舟老鸦洲,跨高马,佩长剑,其党数百人翼之入城,受宴出\footnote{(康熙)《澄海县志》卷19}。

\end{quote}

许朝光在明帝国官僚面前耀武扬威的姿态,真是大伸我南粤志气。因侵略军再也不敢进攻他,许朝光便在南澳岛上的东湖边公开建立城寨,势力更加壮大。可惜的是,许朝光在次年被其谋反的部下莫应敷杀害,其众在莫应敷统领下继续据有东湖寨。不久后,明帝国官方将一篇祭文赠予许朝光,表彰他生前的“归顺”之举,充分暴露了明帝国色厉内荏的实质\footnote{陈春声:《从“倭乱”到“迁海”——明末清初潮州地方动乱与乡村社会变迁》}。

这一时期另一位著名的南粤海上英雄是饶平乌石人张琏。他是饶平的一名库吏,为人精明能干,富有反抗精神。1558年,明帝国的潮州地方官借口“防倭”、以酷刑逼迫百姓赋役,激发民变,张琏遂趁机登高一呼、万众响应。很快,程乡、大埔、小靖等地的起义军便纷纷投入他麾下,汇聚成一支十万人的抗暴大军。极度惊恐的明饶平知县林丛槐亲自求见张琏,妄图“招抚”他。对此,张琏大笑着说出了一句豪气万丈的宣言:

\begin{quote}

尘埃亦知天子哉\footnote{毛奇龄:《后鉴录》卷4}!

\end{quote}

1560年五月,在粤东父老的拥戴下,张琏加冕称帝,自称“飞龙人主”,开科取士。至此,在南粤东端的南海之滨,一个与明帝国分庭抗礼的政权出现了。同年,张琏联合“倭寇千余人”,率军大举出击,攻入闽西汀州、漳州、延平、建宁及赣南宁都、连城、瑞金等地,所向披靡,造成“三省骚动”之局。次年五月,张琏由赣南回师南粤,进攻兴宁、长乐、龙川等县。明两广总督张臬亲率重兵赴龙川与张琏对阵,实施坚壁清野之策,将县城周围的房屋拆毁无遗,并纵兵大肆蹂躏附近百姓,使“乡落七八十里皆罹其害”。侵略军的暴行使更多百姓投入张琏麾下,义军规模日益浩大\footnote{郑广南:《中国海盗史》,页217—218}。惊慌不已的明世宗不得不亲自“下诏”,命令张臬指挥由各省调来的二十万兵力“大举攻剿”。在侵略军的疯狂进攻下,张琏不得不返回饶平,将部众分为四队极力抵抗,凄苦至极的防御战持续了将近一年。至1562年六月,侵略军逼近饶平城栅下,纵火焚烧周围各寨,并以“万金、官指挥”的赏格悬赏捉拿张琏。就在战争进行到最危急的关头之际,义军中一个名叫郭玉镜的无耻叛徒竟贪图赏金,“卖琏以献”,起义遂告失败。在这场战争中,侵略军对饶平军民进行了无比残酷的大屠杀,遇难者多达三万人\footnote{郑广南:《中国海盗史》,页218—219}。

对于张琏的结局,史料中的记载扑朔迷离。明帝国官方自然大肆宣传他们已将张琏处决。然而,《明史》却称张琏此后突出重围、浮海前往南洋,在苏门答腊南曾出现过粤人政权的旧港担任“蕃舶长”,管理着当地的对外贸易。在1577年时,有人曾在旧港见过他,那时他已是一个拥有一排店铺和外国船的豪商了\footnote{松浦章:《中国的海贼》,页65—66}。由明帝国的一贯无耻和善于自我吹嘘来看,张琏被处死的说法是站不住脚的。张琏应是到达旧港成了一个南粤大海商,继续他的传奇生涯。 

接下来,笔者便要介绍掀起16世纪南粤航海高潮的大英雄林凤了。16世纪中期,林凤出生于潮州饶平县,乳名阿凤。他的家族是当地的一个小土豪,族人多有从事海外贸易者。从小他便受族人熏陶,向往海上自由自在的生活,是一个乐观开朗、充满幻想的大男孩。19岁那年,他联合一班同乡好友扬帆出海,开始了“海贼”生涯。他的船队在潮州、闽南沿海活动,专门抢掠官吏的钱财,并将财物分给百姓。大批百姓遂争相加入这一侠盗的队伍,不少人甚至带着一家老小。数年之内,他已在澎湖占据了一块基地,拥有一只战船300余艘、兵民数千人的队伍,俨然成了一个海上小国的国王。他的队伍里不仅粤人、闽人,还有许多日本海贼。一名自称“萧公”(Sioco)的日本武士,甚至成了他的心腹爱将。

1567年十月,林凤率领部众进行了第一次攻坚战,他的目标是惠来县的神泉镇。此地离县城十五里,地处海滨。明廷在此筑城守卫,专门防备商贾出洋贸易。当时,林凤巨大的舰队骤然出现在神泉镇外的洋面,其部队迅速登陆攻破神泉镇,并以此为据点。此后,林凤展开了一轮又一轮的袭击:1573年,占据南澳岛之钱澳。七月,至福建福宁州,大破明帝国侵略军船队,发炮击毙明把总刘国宾。十二月,攻澄海,再次大破侵略军。1574年二月,林凤率船队攻入珠江口,直至广州城下。四月,向南航行,大掠海南岛,攻占文昌县清澜港,取得大捷,毙敌上万。六月,至澎湖休整,复东航占据台湾北端之鸡笼。入冬时分,复袭击潮州、惠州沿海。林凤的船队忽东忽西,所到之处,明帝国在南粤的统治机器一片糜烂。明两广总督殷正茂、福建巡抚刘尧诲气急败坏,发大兵进剿,围林凤于南澳岛。然而,就在殷、刘二人弹冠相庆,认为林凤马上就能被擒获时,林凤却又一次飘走了:趁明军不备,林凤率62艘船、5500人悄悄潜出,不知所踪。得知林凤远遁,已对林凤闻风丧胆的明军诸将纷纷“椎羊酒相贺”。刘尧诲亦连忙向明廷报捷,为“奋战”的诸位文武大员请功。然而,他们没有高兴多久。很快,他们通过谍报得了惊人的消息——林凤不但没有失踪,而且毫发无损。十一月二十九日,他的船队已出现在吕宋(菲律宾)的马尼拉湾。当时,守卫马尼拉城的,正是刚占领菲律宾没多久的西班牙人\footnote{郑广南:《中国海盗史》,页236}。也就是说,林凤要对纵横四海的西班牙帝国动手了! 

林凤舰队刚一靠近菲律宾海岸便俘获西班牙小艇一艘。岸上的西班牙官员连忙派小艇向马尼拉城中的菲律宾总督拉维柴立斯(Lavezaris)告急,然该艇亦被林凤截获。在马尼拉城尚未察觉之际,林凤命日本武士萧公率700名先锋乘艇登陆突袭马尼拉。这批勇士勇往直前,在西人毫无防备的情况下杀进马尼拉市区,冲入西军军长高梯(Goiti)住宅将其击毙。城中西人这才反应过来,纷纷抵抗。萧公见一时难以取胜,便率兵撤回船上,向林凤汇报战况。林凤乃动员全部兵力登陆,兵分三路向马尼拉发起总攻:第一路进攻马尼拉大街、第二路沿海岸线推进、第三路沿巴石河推进。此时,西人已完成布防。经过激烈的战斗,三路林凤军皆告失利。林凤只得率部向北转移,占领彭加钖南。林凤友好地对待当地百姓,使许多菲律宾人都成了他的帮手\footnote{郑广南:《中国海盗史》,页367}。

菲律宾的西班牙人惊恐万状,连忙实施紧急动员,拼凑起一支五千人的部队,对彭加钖南展开进攻。西军的第一次进攻被林凤军的顽强抵抗粉碎,只得对林凤军围而不攻,双方陷入僵持。这时,厚颜无耻的明帝国竟与西班牙帝国勾结,妄图在菲律宾一举消灭林凤。1575年春,明军把总王望高率两艘战船自福建到达菲律宾,与西军会师共围林凤。不久后,西班牙人称林凤将不日成擒,王望高遂扬帆回航。拉维柴立斯特意派遣十二名传教士跟随王望高而去,以求对闽通商。然而,两大帝国的阴谋落空了。同年八月四日,林凤突然率船三十艘趁夜突围,返航潮州,粉碎了西人活捉他的妄想\footnote{郑广南:《中国海盗史》,页368—369;门多萨:《大中华帝国史》,页116}。回到潮州后,他继续招兵购舰,势力有所恢复。

不幸的是,与西班牙人协同作战的明帝国舰队已在南粤海面严阵以待。林凤舰队回航后不久便在海丰广洋海面与明两广总督凌云翼、福建巡抚刘尧诲调集的粤、闽水师发生激战。这一次,师老兵疲的林凤军再未能取胜。他们损失了1800余人、40余艘船只,全军近乎失去战斗力。万般无奈下,林凤的部将马志善、李成为保住官兵的生命率1712名部下、53艘海船投降。热爱自由的林凤见“众心已散”,又绝不愿对明屈服,便在暗夜中驾船逃出包围,从此不知所踪。有人说,他和林道乾、张琏一样跑到了南洋\footnote{郑广南:《中国海盗史》,页237}。

对驻粤、闽明帝国官僚来说,没有捉到林凤是一件大失脸面的事。在对明廷的报告中,他们只得重点宣传自己的“大捷”,支支吾吾地将林凤的行踪搪塞过去。林凤虽然失败了,但他在南海波涛中力抗暴明、西班牙两大帝国的壮举足以使人神往不已,他不愧是十六世纪南粤海上群雄中最为优秀的人物。十六世纪的粤东海上群雄虽然先后被愚昧残暴的明帝国镇压,但他们中的许多人都到达南洋各地,将南粤文明洒遍东南亚,书写了南粤大航海时代波澜壮阔的序章。此后,我们伟大祖先的航海事业绝不会消沉下去。到17世纪上半页,一名令人震撼的英雄出现了,他便是曾与著名的闽越郑氏集团分庭抗礼的南粤“大海盗”刘香。

\section{十七世纪前期粤闽海上武装的和战:刘香与郑芝龙}

\indent 今日的香江之所以得名,与一个有趣的传说有关。据说,古时珠江口的一座岛上上有个名唤“香姑”的女海盗。她武艺高强、貌美如花,乃是个英姿飒爽的女武者。当地人十分倾慕她,遂将她居住的岛屿称作“香姑岛”,简称“香岛”。香岛上有座天然良港,被人称为“香港”。有学者认为,香姑是并不存在的人物,但有关她的故事并非空穴来风,实与17世纪初当地一个名叫刘香的海上英雄有关。美丽的香姑传说,便是由刘香的事迹演化而来的\footnote{刘志文:《广东民俗大观 下》,页797}。

对于刘香的早年经历,史籍中记载极少,我们只知他是南丫岛(今属香港)人\footnote{亦有学者认为刘香系闽人,参见郑广南:《中国海盗史》,页255}。刘香大约出生于16、17世纪之交,早年可能曾赴南洋闯荡,与澳门的葡萄牙人、菲律宾的西班牙人有密切交往,当过西班牙人的买办\footnote{张嵚:《就这样收复台湾》,页131}。他身材矮小,勇猛异常,手下有一支规模达上百艘船、亦商亦盗的舰队,时常袭击、抢掠明军战船,被明帝国视为“海寇”。他的部下与他关系融洽,亲昵地称他为“香佬”\footnote{郑广南:《中国海盗史》,页255}。1620年代,闽越沿海因荷兰东印度公司的介入发生了巨大变化。当时,荷兰人因进攻澳门失利,遂将注意力转向东面,于1622年攻占澎湖、1624年占领台湾西南部,修建著名的热兰遮城堡,控制了台湾海峡的制海权,意图进一步打开闽荷贸易之门。然而,由于闽越的商品皆操于闽商之手,荷人急需与亦商亦盗的闽人海上武装合作\footnote{徐晓望:《闽商发展史 总论卷 古代部分》,页164}。1625年,荷人开始向闽越海上武装发放橘白蓝三色的荷兰国旗,并邀请他们到台湾定居,鼓励他们为荷人做生意、攻击西、葡两国的船只。荷人手下一位精通葡萄牙语的闽人翻译也想分一杯羹,便向荷人辞职,当起了为荷兰东印度公司服务的海盗。这位翻译,便是大名鼎鼎的郑芝龙\footnote{欧阳泰:《1661,决战热兰遮:中国对西方的第一次胜利》,页25—26}。

郑芝龙,字飞黄,小名一官,闽越泉州南安人,出生于1604年。他的父亲是泉州的一名小吏,家庭中有许多人都前往澳门、南洋谋生。十八岁那年,郑芝龙和他的亲弟芝豹、芝虎随舅父黄程前往澳门“闯世界”。当时的澳门城邦是个各国商贾云集的国际城市。在这样一个新奇的地方,聪明的郑芝龙很快学会了葡萄牙语,并受洗皈依天主教,教名尼古拉(Nicolas),因而被西方人称为尼古拉·一官。当时,泉州商人李旦正有一批货物欲从澳门运往日本。黄程见郑芝龙聪明能干,就命他随李旦同往日本。不到二十岁的郑芝龙就这样漂洋过海,踏上了日本的土地\footnote{郑广南:《中国海盗史》,页242}。

李旦是闽日交流史上的传奇人物。李旦居于日本九州长崎平户。在九州的粤、闽富商中,他势力最大,于台湾魍港(今嘉义布袋镇)设有对日贸易港,并与荷兰人、英国人有不少贸易往来。1600年关原合战后,德川家康成为日本的实际统治者。三年后,家康于江户建立幕府,开启了日本历史的新时代。1605年,家康将幕府将军之位交给其子秀忠,自称“大御所”,移居江户西面的骏府城。当时,丰臣秀吉之子秀赖正盘踞于大坂城,乃全日本唯一不向幕府臣服的大名。家康视丰臣家为眼中钉,急欲除之而后快。听闻李旦富可敌国,家康非常想与其合作,乃于1612年八月将李旦召至骏府城面谈。此后,双方建立了合作关系。在1614—1615年江户幕府消灭丰臣家的大坂冬之阵、大坂夏之阵中,家康曾受李旦资助\footnote{徐健竹:《郑芝龙、颜思齐、李旦的关系及其开发台湾考》,《明史研究论丛》(第三辑)}。郑芝龙随李旦东渡日本时,李旦已经年迈。因年轻的郑芝龙容貌俊秀,有龙阳之好的李旦便以其为义子,将其当做男宠\footnote{欧阳泰:《1661,决战热兰遮:中国对西方的第一次胜利》,页20}。

郑芝龙不止是李旦的男宠,更是他十分信任的部下。每当李旦与荷兰人会面时,郑芝龙皆随行。当时,荷人正占据着澎湖,与明军冲突不断。正是在李旦和郑芝龙的调解下,荷兰人才离开澎湖,前往台湾西南部。在李旦的安排下,郑芝龙娶平户日本女子田川松为妻,于1624年生下著名的郑成功。但在郑成功还未出生时,郑芝龙便已抛下妻子离开日本——当时,荷兰人刚刚到达台湾西南部,需要翻译。在李旦的派遣下,精通西方语言的郑芝龙担任了这项工作\footnote{欧阳泰:《1661,决战热兰遮:中国对西方的第一次胜利》,页23}。

如前所述,郑芝龙于1625年辞去翻译工作,成为一名为荷兰人服务的海盗。同年,李旦病逝,他庞大的舰队转入郑芝龙手中。一时之间,郑芝龙的舰队成为闽、粤沿海最强大的海上武装,各地海商争相来投。同年十二月十八日,以郑芝龙为首的十八位闽、粤大海商在台湾北港溪出海口会面,结成强大的反明军事同盟“十八芝”。除郑芝龙外,其余十七人的名字为:郑芝龙、郑芝虎、郑芝豹、郑芝莞、郑芝凤(以上四人为芝龙之弟)、李国助(李旦之子)、杨天生、陈衷纪、施大瑄、洪旭、甘辉、杨六、杨七、钟斌、刘香、李魁奇、何斌、郭怀一\footnote{张培忠:《海权战略》}。值得注意的是,这些人中除刘香外皆为闽人。这表明,因明帝国在16世纪下半页的持续镇压、屠杀,南粤的海上势力受到了严重挫折,实力已不如闽越。

我们不知道刘香究竟是在何时、何地投奔郑芝龙的。但能够确定的是,此后三年乃两人合作的蜜月期。1626年三月初,郑芝龙率十八芝水军出海进攻闽越漳浦镇。十日,舰队泊金门。十八日,大军攻打中左所(今厦门),于金门、厦门两岛竖旗招兵,十日之间便有数千百姓来投。当时驻守厦门的明将为都督俞咨皋,乃刽子手俞大猷之子。他见郑芝龙军势大,不敢出兵,便以重利引诱杨六、杨七两人反水。四月,利欲熏心的两人竟果真无耻地投降了明帝国。在此情形下,郑芝龙仍不放弃战斗,而是派兵于海澄县登陆,袭扰附近乡村地区。当时,驻闽的明帝国文武官员皆是贪狠之辈。他们争相派出家丁、军队劫掠民财,使百姓民不聊生。相反,郑军军纪严明,绝不焚杀,凡经过一地皆救济贫民,百姓遂“归之如流水”。随着郑军的日益壮大,明福建巡抚朱一冯惊惧至极,急催俞咨皋率舰队出战。俞咨皋乃一纨绔子弟,全不知兵。当年九月,他纠集兴化、漳州、泉州之兵至中左所“会剿”,结果惨败于将军澳,仅以身免,被明廷投入监狱。郑军长驱直入,一举攻克中左所。经此一战,“十八芝”威名大震,在闽南沿海攻城掠地,如入无人之境。至1628年,郑芝龙已坐拥海船千艘、兵力十万,控制了台湾海峡制海权,成为令荷兰人都感到害怕的存在\footnote{郑广南:《中国海盗史》,页247—250}。

在此历史节点上,如果郑芝龙宣布独立,那么“十八芝”很可能会解放闽、粤,将丧失已久的自由带回两者。然而,他却选择向明帝国投降,希望以此保住自己的权位。1628年,明思宗“谕令”福建巡抚熊文灿“招抚”沿海“海寇”。郑芝龙于当年九月悍然“受抚”,成为明帝国的“海防游击”\footnote{郑广南:《中国海盗史》,页250}。南粤和闽越的历史机会就这样错过了。此后的郑芝龙沦为明帝国的打手,将屠刀对准了“十八芝”中的昔日盟友。

杨六、杨七成了首批被郑芝龙清洗的老盟友。此二人曾于1626年投降明帝国,不久后即幡然悔悟,重举抗明义旗。1629年六月,郑成功出兵攻杀杨六、杨七、收编其众\footnote{郑广南:《中国海盗史》,页253}。接下来被遭难的是李魁奇。李魁奇曾于1628年与郑芝龙一同“受抚”,后因心有不甘与钟斌(“十八芝”之一)浮海而去,继续与明帝国为敌,受到百姓的热烈支持,其舰队规模很快就膨胀至400艘船。李魁奇舰队连破两城,并于1629年六月一度兵临海澄城下。郑芝龙被他打得接连战败,折损了杨天生、陈衷纪(均为“十八芝”成员)二员大将。无奈之下,郑芝龙只得向其老主顾荷兰人求助。时任台湾总督的汉斯·普特曼斯(Hans Putmans)欣然应之,于1630年二月率一支小舰队开往金门料罗湾与李魁奇军海战。当时,李魁奇过于骄傲轻敌,视钟斌为部下,对其颐气指使。忍无可忍的钟斌遂暗中与荷人约降,在开战后率船绕至李魁奇舰队背后开火,与荷舰前后夹攻。一片混乱中,李魁奇中火枪身亡,全军溃败。战后,荷人将李魁奇残部交予郑芝龙,与其一同举行了盛大的庆祝活动。普特曼斯如是说明他支持郑芝龙的理由:

\begin{quote}

如此一来,我们即可获得一名真诚可靠的盟友。没有人会比他对我们更有利\footnote{欧阳泰:《1661,决战热兰遮:中国对西方的第一次胜利》,页28—29}。

\end{quote}

荷兰人高估了郑芝龙的道德,因为他接下来的清洗目标便是钟斌。1630年十一月,郑芝龙、郑芝豹以明军的身份率舰出海进攻钟斌,于平林湾、崇武、平海洋面三战三胜。钟斌惨败,放弃巨舰,乘小船逃亡南澳岛。郑芝龙入粤追击,迫使钟斌回航闽越,泊于泉州附近。明思宗得知郑芝龙的“捷报”,令其务必消灭钟斌,郑芝龙亦干得更加卖力。1631年五月,郑芝龙舰队从外海发起突袭,于柑桔洋围攻钟斌舰队。钟斌兵败势穷,投海而死\footnote{郑广南:《中国海盗史》,页254}。

数年来,郑芝龙的倒行逆施一直被刘香看在眼里。勇武善战、重视责任与荣耀乃是我南粤人的固有美德,刘香则是此种美德的化身。对于郑芝龙甘做明帝国鹰犬、肆意杀戮老盟友的行为,刘香怎能忍受?自郑芝龙降明起,他便与之决裂,独自率舰纵横于北起宁波、南至雷州的广阔海岸线上,到处袭击明帝国侵略军。逼死钟斌后,郑芝龙不再派船前往台湾通商。普特曼斯为之大怒,认为郑芝龙背信弃义的行为着实“虚狡、奸诈”。他拟定了一份击败郑芝龙的计划,并派兵支援刘香\footnote{欧阳泰:《1661,决战热兰遮:中国对西方的第一次胜利》,页30;郑广南:《中国海盗史》,页30}。在荷兰人的帮助下,刘香连战连捷,一度突入郑芝龙的重要据点中左所,焚毁其大批舰船。1632年正月,刘香舰队袭击闽越南安、同安、海澄,对当地人进行了报复性劫掠。八月,他率船200余艘、兵力近万北上吴越宁波、台州、温州,烧掠不已,使沿海百姓逃散一空。九月,他的舰队掉头南下,登陆攻陷闽安镇,逼近福州。刘香联合荷人北征吴、闽之举固然沉重打击了明帝国与郑芝龙的嚣张气焰,但他抢掠百姓的行为无疑非常过分,使他丧失了当地民心。当时正率兵驻守福州的郑芝龙迅速率舰队出海,在连江县海面击退师老兵疲的刘香舰队,给刘香军造成千余人的损失\footnote{郑广南:《中国海盗史》,页256}。远征失利后,刘香于冬天顺风退入南粤沿海,郑军紧追不舍。1633年二月,两军战于珠江洋面赤头岗,刘香军再告失利,阵亡600余人,继续西退。六月,两军又战于雷州半岛洋面,刘香军第三次失利,损失139人\footnote{张培忠:《海权战略》,页44}。刘香舰队再无可退之处,便冒险向东航行,躲过郑军的侦察,最终抵达澎湖。

此时,台湾的荷兰人已对依靠刘香打败郑芝龙失去信心,决定亲自出手。1633年7月,普特曼斯率九艘荷舰出征,于7日(农历六月二日)占领南澳岛,12日突袭厦门(中左所)。当时,郑芝龙的主力舰队正在福宁州,厦门港内停泊着数十艘毫无防备的郑军、明廷船只。荷军猛烈开炮,并派兵登上敌船纵火,击沉大型战船30艘、小型战船20艘,而荷方仅阵亡1人\footnote{欧阳泰:《1661,决战热兰遮:中国对西方的第一次胜利》,页38}。傍晚,荷军又绕至厦门北部,以零伤亡轻易击沉郑军战舰10艘、明军战舰5艘。14日,130艘明舰兵分两路于金门岛向荷军发起反攻,被迅速击退、伤亡惨重。经此一连串大胜,普特曼斯产生了骄傲轻敌的情绪。他强迫厦门、金门、烈屿、鼓浪屿一带的村庄每周向荷军交出25只猪、100只鸡、25头牛及大批财物,否则就派兵上岸烧掠。附近百姓别无选择,只得照办。据荷方统计,普特曼斯舰队在厦门之战后六周内一共掠取了价值64017枚银币的财物,相当于今日的1300万美元\footnote{欧阳泰:《1661,决战热兰遮:中国对西方的第一次胜利》,页39}。至9月,刘香亦派出约数十艘小船与普特曼斯舰队会师,形成了一支粤荷联合舰队\footnote{欧阳泰:《1661,决战热兰遮:中国对西方的第一次胜利》,页39}。

此时,郑芝龙正在明福建巡抚邹维琏的督促下积极备战。邹维琏是个愚昧顽固的帝国官僚,十分痛恨西方人。在他的要求下,郑芝龙不敢怠慢,组织起一支规模极其庞大的舰队,并向普特曼斯送去一封战书,用杀气腾腾的口吻写道:

\begin{quote}

皇帝岂能容许一条贱狗将头搁在他的枕头上!……你想打仗,就到厦门来吧,这样中国的官员即可以看着我们打败你!

\end{quote}

普特曼斯对郑芝龙的挑衅不屑一顾。他称:

\begin{quote}
他们若是对我们发动攻击,愿神协助我们奉他的圣名战胜敌人,打垮这个满是鸡奸者的奸邪民族\footnote{以上两段皆转引自欧阳泰:《1661,决战热兰遮:中国对西方的第一次胜利》,页40}!
\end{quote}

在普特曼斯的指挥下,粤荷联合舰队在厦门外海不远的金门岛南海岸下锚,此处正是三年前普特曼斯打败李魁奇的料罗湾。联合舰队布下的阵型,是9艘荷舰居中、50艘粤舰居两翼。粤舰和荷舰一样挂着荷兰东印度公司的蓝色旗帜,威风凛凛,静候郑军到来\footnote{欧阳泰:《1661,决战热兰遮:中国对西方的第一次胜利》,页40}。

1633年11月22日(农历九月十六日)拂晓,郑芝龙共计150艘船的庞大舰队绕过金门岛,出现在联合舰队的视野中。这支舰队的主力是50艘战舰,其中有30艘为仿欧战舰,每艘配有30至36门火炮,与荷舰的装炮量一样多。其余战舰则只有6到8门小炮,与刘香的战舰相同\footnote{欧阳泰:《1661,决战热兰遮:中国对西方的第一次胜利》,页35}。此外,郑芝龙还拥有100艘满载易燃物的火攻帆船。这些火攻船的前头装有被称为“狼牙”的铁钩,可使火攻船与敌舰钩在一起燃烧。这些火攻船上同样载有大炮与士兵,外表伪装得与小型战船无异\footnote{欧阳泰:《1661,决战热兰遮:中国对西方的第一次胜利》,页42}。

郑芝龙的欺敌战术成功了。当时,战场上刮着强劲的东风。骄傲的普特曼斯未发现任何异常,下令处于逆风位置的舰队原地不动,以火力阻击敌舰。然而,郑军的100艘火攻船却出乎他意料地一炮未发,径直冲了过来。这些小船的移动速度极快。在联合舰队还没回过神时,一艘火攻船已钩上一艘荷舰。在腾空而起的爆炸声中,两舰被火海吞没,数十名荷军和16名郑军全部沉入水底\footnote{欧阳泰:《1661,决战热兰遮:中国对西方的第一次胜利》,页45}。

郑军之所以表现得如此疯狂,与郑芝龙在战前对他们许下的承诺有关。郑芝龙提出,若有火攻船能焚毁敌舰便奖励白银200两。全船16人如能全员幸存,每人即可分得12两以上。重赏之下,这些火攻船不要命地前仆后继,像疯似的寻找目标。直到这时,普特曼斯才反应过来联合舰队所面临的危局。在战后写下的回忆文字中,他这样说:

\begin{quote}
这时候我们才意识到:整支舰队其实是一批火攻船。他们根本无意作战,而是要冲向我们,纵火焚烧我们的船只\footnote{欧阳泰:《1661,决战热兰遮:中国对西方的第一次胜利》,页43}。

\end{quote}

他的醒悟实在太晚了。此时,大批荷舰、粤舰已被火攻船撞上,幸存战舰陷入一片混乱。在漫天火海中,普特曼斯丧失了战斗意志,怯懦地抛下陷入围攻的荷舰和所有粤舰,率5艘荷舰突出重围,落荒而逃。后来,一艘荷舰竟奇迹般地冲了出来。郑芝龙见荷舰逃跑,率其战舰穷追不舍。荷舰逆风而行,全速逃命。西方世界强大的科技力量终于有所展现。郑军虽然顺风,但还是追不上荷舰。最终,郑芝龙见荷舰越逃越远,只得作罢\footnote{欧阳泰:《1661,决战热兰遮:中国对西方的第一次胜利》,页44}。这时,残存粤舰已沉没殆尽。惨烈的料罗湾海战,就这样以郑芝龙的胜利告终。

在这场大决战中,普特曼斯的轻敌和怯懦导致了惨痛的代价。荷军损失战舰3艘,阵亡、被俘93人\footnote{欧阳泰:《1661,决战热兰遮:中国对西方的第一次胜利》,页44}。至于50艘南粤战舰则大多悲惨地沉入海底,舰上战士的伤亡数字甚至都无人统计。数以千百计的优秀南粤儿郎,就这样悲惨地葬送在异乡的海底,成为难以找到回家之路的孤魂。这实为我南粤航海史上至为沉痛的一幕。

经此一战,郑芝龙已成为闽、粤沿海最强大的海上霸主。他坐拥3000艘各类船只,麾下军队超过20万人,其中有大量日本人、朝鲜人、黑人,是西太平洋上一股任何人都不能忽视的力量\footnote{张培忠:《海权战略》,页49}。普特曼斯在卑怯地逃回台湾后,很快就和郑芝龙达成和解。不久后,受郑芝龙控制的商人便遍布台湾,积极与荷兰人进行贸易\footnote{欧阳泰:《1661,决战热兰遮:中国对西方的第一次胜利》,页46}。1634年,荷兰人为讨好郑芝龙,决定解除与刘香的盟友关系。巴达维亚殖民当局做出了不与刘香签订任何协定、不允许刘香舰队驻扎于澎湖、出没于漳州、台湾热兰遮的决议,并威胁刘香,称其若不遵守上述规定,荷军便会联合明帝国军队一同将其“剿灭”\footnote{醉罢君山:《海上家国 十七世纪中荷战争全纪录》,页91}。至此,刘香被荷兰人彻底抛弃了。

对荷兰人背信弃义的行为,刘香极感愤怒。他清楚地记得郑芝龙是如何卑劣地杀害昔日的盟友李魁奇、杨六、杨七、钟斌的,也清楚地记得郑芝龙是如何卑鄙地利用荷兰人消灭异己的。他更明白,郑芝龙是个向明帝国投降、毫无政治德性与原则的恶棍,不但背叛了当初一同与他盟誓反明的兄弟,也背叛了曾对他寄予厚望的沿海百姓。而荷兰人竟与这样一个人结盟,着实背离了国际外交中的正义。正直、有责任感乃是我南粤人共有的优良秉性。面对不义,粤人一定会勇敢战斗。身为一名南粤的海上武者,刘香下定了决心。尽管他已只剩下几十艘战舰,但无论郑芝龙与荷兰人有多么强大,他都要与之战斗到底。这不单是为了正义,亦是为了死难的“十八芝”兄弟报仇。

1634年初,刘香舰队在台湾海峡频频活动,截获了十几艘闽越船只。其后,他们向台湾发起进攻。4月7日,刘香率战舰出现于热兰遮以北的浅海区。曼普斯特下令全城备战,荷军严阵以待。8日凌晨2时,刘香军的600名勇士在暗夜中登陆,用云梯悄然爬过热兰遮城墙,突入城内,一举摧毁两座碉堡、炸毁一批火药。这时,荷军被从睡梦中惊醒,开始抵抗。勇士们亦不恋战,顺利撤出。此后,刘香舰队在热兰遮城外徘徊数日,因荷军防备森严,并不贸然进攻。13日,刘香暂时放弃攻台计划,率舰队撤回澎湖,在当地袭击了一只荷兰商船,俘荷人30名\footnote{醉罢君山:《海上家国 十七世纪中荷战争全纪录》,页92}。

刘香与荷兰人的战争暂时告一段落了,但他与郑芝龙和明帝国的对决尚未结束。此后一年内,刘香舰队在粤、闽沿海四处袭扰,使明帝国官僚苦不堪言。明两广总督熊文灿希望“招抚”刘香,被他严词拒绝。身为一名南粤武者,他绝不会像郑芝龙那样出卖自己的父老,向残暴与不义妥协。气急败坏的熊文灿只得催促郑芝龙火速出兵“剿灭”刘香。1635年4月8日,郑芝龙庞大的舰队与仅剩五十艘船的刘香舰队在陆丰海面的田尾洋遭遇,双方展开最后的决战。虽然郑军兵力远强于刘香,但郑芝龙仍不敢轻视这位宿敌。开战后,他亲自指挥前哨各舰突击,希望亲自了结与刘香的恩怨。战局的进展令人瞠目结舌。刘香舰队虽然为数不多,但爆发出了强大的战斗力。激战中,郑芝龙之弟芝虎、芝鹄所乘战舰沉没,全船300余人无一幸免。闻知两名亲弟战死,郑芝龙怒极,命全军蜂拥而上。刘香知道,最后的时刻到来了。他决定自尽,宁死亦不能让自己的尸首被郑芝龙和明帝国羞辱。在刘香的命令下,他的座舰被将士们举火焚毁。在冲天燃起的红莲烈火中,刘香和全舰将士化为灰烬,将自己的骨灰洒进了南粤的大海。曾在粤、闽、吴、台海上纵横征战的刘香舰队,就这样全军覆没,消逝在南海波涛中\footnote{张培忠:《海权战略》,页44—45}。战后,郑军斩获刘香军首级622颗。至于与舰同沉,未被郑军打捞的将士尸首,更是不计其数\footnote{郑广南:《中国海盗史》,页257}。

刘香的逝去,标志着16—17世纪最后一支可与明帝国抗衡的南粤舰队的烟消云散。在明帝国及其鹰犬郑芝龙的残酷剿杀下,那个如德雷克般为南粤的自由冲杀于巨浪中的大英雄刘香消失了,南海上强大的粤人武装也暂时消失了。在即将到来的明末清初大洪水中,南粤缺乏如刘香般的海上英雄,终使南粤在野蛮的“迁海”(详见下章)中蒙受了极度惨痛的损失。1644年,李自成的流寇大军攻入北京,明思宗在煤山自缢,恶贯满盈、屠杀了无数粤人的明帝国终于走向终结。两年后,清军进入闽越,惯于投机的郑芝龙又向清帝国屈膝投降,最后被如囚犯般送往北京,失去了一切,最后于1661年被清帝国灭族。与此同时,他的儿子郑成功继承了他在闽越的事业,创造了另一个传奇故事。在清军进入闽越的同一年,他们也打进了南粤。南粤,迎来了近代史上首次神圣的卫国战争。









