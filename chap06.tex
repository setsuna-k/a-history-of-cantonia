\chapter{第四次北属}

\section{唐帝国治下的南粤}

公元698年,伟大的冯冼家族在武周的屠刀下归于沉寂,南粤最大的本土自立势力被帝国消灭了。705年,唐中宗复辟,武则天下台,而帝国屠杀粤人的政策却并未停止。在唐中宗上台第二年,广州都督周仁轨率一支两万人的侵略军对钦州僚人豪族宁氏发动了进攻,斩杀其首领宁承基,“杀掠其部众殆尽”\footnote{司马光:《资治通鉴》卷280}。至此,经武则天以来的反复屠杀,粤人已经完全丧失了自由,第四次沦为帝国的奴隶。

面对岭北帝国如此疯狂的羞辱与屠戮,我们的伟大而英勇的祖先当然不会坐以待毙、更不会被侵略者的淫威吓倒。722年,欢州(今越南乂安)人梅叔鸾见唐朝官僚多行残暴不法之事,愤而聚众起兵,筑垒称帝,惨遭唐军镇压\footnote{越南人俗称梅叔鸾为“梅黑帝”,参见陈重金:《越南通史》,页43}。公元728年,一场更大规模的起义爆发了。是年,泷州(今罗定)僚人酋长陈行范、广州僚人酋长冯璘、何游鲁发动了震撼全粤的反唐大起义。在短短的时间内,起义军攻陷四十余城,几乎解放了整个南粤。陈行范自称天子、冯璘自称南越王、何游鲁自称定国大将军。其时,唐帝国正处在所谓的“开元盛世”。岭北帝国的盛世便等同于南粤的苦难。在唐玄宗的命令下,大粤奸宦官杨思勗(化州人)率领十万侵略军开入南粤。在保卫泷州城的战斗中,冯璘、何游鲁皆慷慨战死。唐帝国侵略军长驱直入,打到广州城以南的云际、盘辽二洞(在今深圳境内),杀害了撤退至此的陈行范。这场轰轰烈烈的大起义便这样被唐帝国镇压下去,倒在侵略者屠刀下的南粤军民多达六万人。对于被俘的起义军将士,粤奸杨思勗指示麾下唐军施以酷刑虐杀,“或生剥面皮,或以刀犁发际、扯去头发”,对我们的祖先犯下了不可饶恕的滔天罪行\footnote{关于此次起义,参见欧阳修:《新唐书》卷207《列传第一百三十二宦者上》;司马光:《资治通鉴》卷213《唐纪二十九》;《广西大百科全书·历史上》,页223}。至791年,交趾唐林郡(今越南山西省境内)又发生了冯兴起义。起义军攻破龙编,横征暴敛的唐安南都护高正平在人民的怒火中惊惧而死。不久后,龙兴病逝,军民立其子冯安为继承人。同年七月,在唐军的镇压下,起义失败。然而,交趾人仍立祠祭祀冯兴,视他为父母,亲昵地称之为“布盖大王”\footnote{越南人称父为“布”,母为“盖”,“布盖大王”即“父母大王”。见陈重金:《越南通史》,页43}。

经过唐帝国的轮番屠杀迫害,南粤俚人与僚人的自治权日渐丧失。唐中叶以降,俚人渐渐在史籍中消失了\footnote{《广东通史》古代上册},僚人则仅在粤西北仍有少量分布\footnote{《广东省志·少数民族志》}。他们当中,居住地离帝国神经节点较近者成为帝国的编户齐民,被冠以“汉人”这一降虏标记极强的称呼;居住在山区中的人们虽保留了原有的习俗与部落,但被迫承担起沉重的徭役,被称为“瑶人”\footnote{刘志伟:《如何理解帝国边缘的南岭》}。至此,南粤已经丧失了在国际交往中的发言权。在9世纪的南诏*唐朝战争中,南粤沦为两大帝国的战场,受到了残酷的蹂躏。

南诏于公元8世纪形成于滇地,是一个足可与唐帝国分庭抗礼的伟大国家。公元859年,唐帝国以南诏王世隆之名犯唐太宗、玄宗之“讳”为借口与之决裂。次年十二月,饱受唐帝国暴政之苦的交趾百姓引南诏军三万余人攻陷龙编,从而引发了诏唐之间对南粤的惨烈争夺。861年,唐军重占交趾,南诏军又出兵攻克邕州(今南宁)。863年,南诏国军重夺交趾,置安南节度使。三年后,唐安南都护高骈又夺取邕州,复消灭三万驻交趾之南诏军\footnote{李半仙:《南诏编年简史》}。双方长期的拉锯战使给南粤百姓制造了巨大的苦难,仅交趾一地的死者即超过十五万\footnote{陈重金:《越南通史》,页45}。

诏唐战争使唐帝国遭受重创,亦使我们的祖先认清了唐帝国的外强中干。公元868年,桂林戍卒发动了震撼东亚大陆的“庞勋兵变”,起义军经湖湘、吴越一路北上,在一年多的时间里控制了江淮之地,切断了唐帝国榨取东南膏脂的通道。伟大的庞勋兵变虽然最终被镇压了,但亦开启了唐帝国的末日\footnote{关于庞勋兵变的细节,可参看卢建荣:《咆哮彭城:唐代淮上军民抗争史》}。史书中“唐亡于黄巢,实祸基于桂林”一语,可谓佳论\footnote{欧阳修:《新唐书》卷222《列传第一百四十七中南蛮》}。在南粤与南诏的共同努力下,摇摇欲坠的唐帝国即将土崩瓦解。东亚大陆诸邦的大解放时代,很快就要到来了。

\section{光怪陆离的国际贸易港口:广州}

在叙述唐帝国的最终崩溃之前,让我们先放缓一下脚步,看一看南粤的中心广州城在当时有着怎样的面貌。在第四次北属时期,南粤的国际贸易航线较旧时更为扩大,而这一航线的东端就在广州城。对此,《新唐书》有详细记述:

\begin{quote}
	
广州东南海行,二百里至屯门山(今深圳南头或香港屯门),乃帆风西行,二日至九州石(今海南东北部七洲列岛)。又南二日至象石(今海南东北独珠山)。又西南三日行,至占不劳山(今越南岘港东南占婆岛),山在环王国东二百里海中。又南二日行至陵山(今越南燕子岬)。又一日行,至门毒国(今越南归仁)。又一日行,至古笪国(今越南芽庄)。又半日行,至奔陀浪洲(今越南藩朗)。又两日行,到军突弄山(今越南昆仑岛)。又五日行至海硖(马六甲海峡),蕃人谓之"质",南北百里,北岸则罗越国(今马来半岛南端),南岸则佛逝国(今印尼苏门答腊旧港)。佛逝国东水行四五日,至诃陵国(今印尼爪哇岛),南中洲之最大者。又西出硖,三日至葛葛僧祗国(似即黑人峡),在佛逝西北隅之别岛,国人多钞暴,乘舶者畏惮之。其北岸则个罗国(今马来西亚吉打)。个罗西则哥谷罗国(今泰国克拉地峡西南)。又从葛葛僧只四五日行,至胜邓洲(今印尼)。又西五日行,至婆露国(今印尼苏门答腊婆罗师岛)。又六日行,至婆国伽蓝洲(今印度尼科巴群岛)。又北四日行,至师子国(今斯里兰卡),其北海岸距南天竺大岸百里。又西四日行,经没来国(今印度西南阔伦),南天竺之最南境。又西北经十余小国,至婆罗门西境。又西北二日行,至拔狖国(今印度孟买附近)。又十日行,经天竺西境小国五,至提狖国(今巴基斯坦卡拉奇),其国有弥兰太河,一曰新头河,自北渤昆国来,西流至提狖国北,入于海。又自提狖国西二十日行,经小国二十余,至提罗卢和国(今伊朗阿巴丹附近),一曰罗和异国,国人于海中立华表,夜则置炬其上,使舶人夜行不迷。又西一日行,至乌剌国(今伊拉克奥波拉),乃大食国之弗利剌河(今幼发拉底河),南入于海。小舟溯流二日至末罗国(今伊拉克巴士拉),大食重镇也。又西北陆行千里,至茂门王所都缚达城(今伊拉克巴格达)。自婆罗门南境,从没来国至乌剌国,皆缘海东岸行;其西岸之西,皆大食国,其西最南谓之三兰国(今坦桑尼亚桑给巴尔)。自三兰国正北二十日行,经小国十余,至设国(今也门席赫尔)。又十日行,经小国六七,至萨伊瞿和竭国(今阿曼马斯喀特西南),当海西岸。又西六七日行,经小国六七,至没巽国(今阿曼苏哈尔)。又西北十日行,经小国十余,至拔离謌磨难国(今巴林)。又一日行,至乌剌国,与东岸路合\footnote{欧阳修:《新唐书》卷43《地理志》}。
\end{quote}

可见,在7至9世纪之间,我们的伟大祖先的航海水平较第二次北属时已远更进步,拥有横跨印度洋、到达伊斯兰世界与东非的能力。海外贸易的勃兴使整个南粤都变得异常繁荣富裕,就连粤北的连州都有两成左右的人口从事工商业\footnote{《广东通史》古代上册,页530}。至于广州,更是一派邸店林立的繁华景象。繁荣的商品阶级、富足的城市生活带来了酒类消费的高潮,广州城内“生酒行”鳞次栉比,“皆系女人招呼。\footnote{《太平御览》卷845《饮食部三》}”夜市亦是通宵喧哗,一片热闹\footnote{《广东通史》古代上册,页533}。作为东亚与阿拉伯货物的集散地,当时江淮、吴越、巴蜀与唐帝国两京的商人将本地商品运往广州,分销于南粤和海外,又运回货物,形成了巨大的循环物流。时人描述这一情形称:

\begin{quote}

南海,有国之重镇,北方之东西,中土之士庶,䑸连毂击,合会于其间者,日千百焉\footnote{转引自《广东通史》古代上册,页533}。

\end{quote}

发达的海外贸易使南粤与伊斯兰世界间产生了密切的人员交往。公元900年左右,一位巴格达的阿拉伯医生记载称,他交了一个在当地居住了一年,用五个月时间学会了阿拉伯语的“中国”朋友\footnote{斯塔夫里阿诺斯:《全球通史》第十二章《欧亚大陆文化高度发达的核心区》}。他的在位朋友,很可能是从广州出发的南粤人。至于在南粤定居的阿拉伯人、波斯人和印度人就更多了。早在公元401年,罽宾国(今克什米尔)僧人昙摩耶舍就在广州建立了王苑朝延寺的大雄宝殿。该寺庙在1151年改名为光孝寺,沿用至今\footnote{执经生:《南粤编年简史》}。时至今日,该寺宏伟的木构大雄宝殿仍耸立在广州城中,接受着无数善男信女的香火。527年,更有一位非常著名的印度僧人来到广州,他便是禅宗始祖达摩。达摩在其居停之处建立了西来庵,该庵于一千余年后改名华林寺,成为佛教禅宗最重要的寺庙。至676年,禅宗再次与南粤发生紧密联系。是年,于楚地黄梅得到禅宗五祖弘忍传授衣钵的南粤新州(今新兴)僚人慧能来到光孝寺听住持印宗法师弘法,遇两僧人辩论风幡,一人说是风动、一人说是幡动,争论不休。慧能对他们说:“不是风动,不是幡动,仁者心动”,从而制造了禅宗史上著名的“风幡公案”。印宗法师大为称奇,遂拜慧能为师,为之祝发\footnote{《坛经》}。今天,光孝寺中仍保留着六祖慧能瘗发塔、风幡堂等建筑,无声地述说着在段精彩的历史。六祖慧能圆寂后,其肉身更是神奇地千年不腐,被安放于韶州南华寺供奉,直至被来自北方的红卫兵破坏。除此之外,广州城内又有建于537年的宝庄严寺,该寺后改名六榕寺。自4、5世纪以降,光孝寺、华林寺、六榕寺一直是来粤印度僧人的居所\footnote{科大卫:《皇帝与祖宗:华南的国家与宗族》,页23—24}。

广州城内阿拉伯人、波斯人的数量比印度人还要多。622年,穆罕默德创立伊斯兰教,随即派门徒四人前往东方传教。五年后,四人中的艾比*宛葛素和侨居广州的阿拉伯人一同修建了东亚首座清真寺怀圣寺。今天,这座清真寺的仍不断传出唤拜声,吸引着由欧、亚、非等地来到南粤的穆斯林。到8世纪,广州城西侧已经形成了非常成熟的穆斯林社区“蕃坊”。9世纪前期,更有许多穆斯林和粤人通婚、在南粤置办产业,广州呈现一派“蕃獠与华人错居相婚嫁,多占田营第舍”的景象。如果帝国官吏胆敢阻挠他们,这些穆斯林便与粤人团结在一起,“相挺为乱”\footnote{科大卫:《皇帝与祖宗:华南的国家与宗族》,页23;《广东通史》古代上册,页553}。他们融入了南粤社会,与粤人一起反抗帝国官僚的残暴统治,成为了南粤文明中一块极具异域色彩的拼图。公元758年,广州的阿拉伯、波斯海商发起反唐暴动,一度占领广州城、大掠府库,逼使唐广州刺史韦利见弃城而逃\footnote{司马光:《资治通鉴》卷220《唐纪三十六》}。763年,唐帝国于广州设市舶使司,对广州的海外贸易展开了更为制度化的盘剥\footnote{科大卫:《皇帝与祖宗:华南的国家与宗族》,页23;《广东通史》古代上册,页553}。根据唐帝国规定,仅广州一地就要向宫廷“进贡”银、生沉香、甲香、石斛、䵶鼊皮、蚺蛇胆、詹香糖、藤簟、竹簟、竺席、水马、荔枝等物产。整个南粤二十五州需要“进贡”的土产,更达36种之多\footnote{《广东通史》古代上册,页463}。残酷的盘剥使广州人发起了接连的反抗。769年,番禺人冯崇道在广州发动起义,兵败遇害\footnote{欧阳修:《新唐书》卷131《列传第五十六宰相宗室》}。773年,被盘剥得忍无可忍的广州商人拥唐循州刺史哥叔晃据广州城起兵,击毙岭南节度使吕崇贲。唐帝国派路嗣恭接任岭南节度使,以大军镇压。哥舒晃与广州军民同仇敌忾,进行了艰苦卓绝、长达三年的战斗,最终不幸兵败。丧心病狂的路嗣恭随即大开杀戒,不但斩杀了哥舒晃,还对广州商人滥施屠戮,“没其家财宝数百万贯。\footnote{关于哥舒晃起义,参见杜佑:《通典》卷188《边防四》;刘昫:《旧唐书》卷122《列传第七十二路嗣恭》}”至9世纪初,唐帝国在岭南抽取的商业税已与“两税”(农业税)相当,足见当时南粤海外贸易之繁盛与被盘剥程度之重\footnote{《广东通史》古代上册,页532}。

无能的唐帝国一面疯狂盘剥着南粤的财富、屠杀着我们的祖先,一面又极度无能、毫无保护南粤的能力。在公元868年的桂林庞勋兵变之后,唐帝国江河日下。875年,山东濮州人王仙芝聚众起事,曹县人黄巢响应之,唐末大洪水爆发。三年后,王仙芝战死,黄巢尽收其众,南下江淮、吴越、闽越、南粤。公元879年五月,黄巢的流寇大军仅用了不到一天的时间便攻陷广州,随即展开了惨绝人寰的大屠杀。对于在场惊天惨剧,阿拉伯史学名著《黄金草原》有如下描述:

他(黄巢)迅速向广州城进军,该城的居民由伊斯兰教徒、基督徒、犹太人、祆教徒与其他“中国人”组成。他将该城紧紧围住,在遭到国王(按:指唐帝国皇帝)军队的袭击时,他把在支军队击溃了,掳掠了些女子。后来,他率领的士兵比任何时候都更为众多,用武力夺城并屠杀了该城数量众多的居民。据估计,在面对刀剑的逃亡中死于兵器或水难的穆斯林、基督徒、犹太人与祆教徒共达20万人\footnote{马苏第著、耿昇译:《黄金草原》,页166}。

面对席卷而来的大洪水,唐帝国衰弱的驻粤军队根本无力抵御。很快,除容州、韶州、高州外,整个南粤便都被黄巢流寇军侵占。就在z 时,一个有阿拉伯或波斯血统的南粤土豪家族站了出来。他们不但将彻底赶走流寇,亦将完全恢复南粤的自由,使南粤以“月季花王朝”的名字声震欧亚大陆。他们便是在南粤史上至关重要的封州(今封开)刘氏家族。他们开创的南粤本土王朝,就是伟大的南汉国。




