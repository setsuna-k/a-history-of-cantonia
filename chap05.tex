\chapter{蛮族自立:冯冼时代}

\section{陈霸先与南粤土豪}

陈霸先,吴越吴兴长城(今浙江长兴)人。他自称系汉代名士陈实之后,应是受到南朝以来攀附门第风气的影响。《南史》称他“其本甚微”,表明他应是一个地道的吴越土著\footnote{李延焘:《南史》卷9《陈本纪上第九》}。陈霸先本为一小吏,后受梁广州刺史萧映赏识,随其入粤为官。公元542年,土豪卢子略起兵反梁,围萧映于广州。陈霸先时任监西江督护、高要太守,控扼西江下游要地,乃率兵顺流东下,镇压了卢子略起义,收降其大批旧部。同年,萧映病死,梁廷又授陈霸先交州司马,令其与交州刺史杨瞟南下镇压交州的李贲起义\footnote{胡守为:《岭南古史》,页193}。

李贲出自交州俚族土豪家庭。正是他发动起义、宣布独立建国,建立了越南史上著名的“前李朝”。公元541年,因原交州刺史萧谘“裒刻失众心”,李贲连结数州豪杰同时起义\footnote{姚思廉:《陈书》卷1《本纪第一高祖上》}。542年,与卢子略围攻广州同时,李贲起义军攻克交州州治龙编。至544年,李贲乃自称“李南帝”、建国号曰“万春”、改元“天德”,完全脱离了帝国的统治\footnote{陈重金:《越南通史》,页39}。

545年,杨㬓、陈霸先率梁帝国侵略军进攻万春国。梁军虽很快便攻占龙编,但双方的战争异常艰苦,持续时间达三年之久。548年,李南帝因患病崩于屈僚洞\footnote{陈重金:《越南通史》}。次年,陈霸先又于九真镇压了李贲余部的抵抗\footnote{陈重金:《越南通史》}。此后,梁帝国认为已经“讨平”交州,乃授陈霸先“振远将军、西江督护、高要太守、督七郡诸军事”\footnote{陈重金:《越南通史》。据越南人的记载,此后李贲余部并未完全被消灭,万春国仍继续存在。李南帝死后,其部下赵光复于549年自称“越王”,继续保持独立。571年,李南帝之侄李佛子起兵讨灭赵越王而登基,是为后李南帝。直到602年,后李南帝方向隋帝国投降。相关记载,参见陈重金:《越南通史》,页39—40;《钦定越史通鉴纲目》卷4,页6—14。然而,相关史事在帝国史书中未见记载。无论事实如何,皆可知梁帝国其实远未“平定”交州,交州本土土豪一直到602年皆维持着某种自立状态。}。陈霸先一时之间风头无两,成为南粤事实上的军事霸主。

陈霸先身为南人,却代表梁帝国镇压了李南帝的起义,不能不说是他人生的一大污点。然而,日后成为陈武帝、开创了南人土豪洞主政权陈朝的陈霸先绝非一个帝国奴\footnote{关于陈朝立国基础的论述,参见万绳楠整理:《陈寅恪魏晋南北朝讲演录》,页241—242}。在南粤时,他一边镇压土豪起义,一边积极罗致流寓岭南的吴越人与本地土豪。例如,在镇压卢子略起义后,他便将卢军骁将杜僧明(高邮人)、周文育(淳安人)收入麾下,使二人成为其左膀右臂。在镇压李南帝起义之后,又有始兴曲江(今广东韶关)土豪侯安都召集甲兵三千人投入陈霸先麾下\footnote{陈霸先罗致的寓粤吴人、南粤土豪颇多。详情可参看胡守为:《岭南古史》,页195—197}。在样一来,陈霸先的军队已带上了浓重的南粤与吴越色彩。在历史节点到来时,他的抉择将决定南粤的命运。

\section{冼夫人的崛起}

与陈霸先在南粤的崛起同时,粤西高凉地区出现了一位强有力的女性蛮族领袖,她的名字叫做冼英,又被人称为冼夫人。在今日粤西及海南人心目中,冼夫人已成为一位保护一方平安的地方神,受到人们虔诚的敬拜。仅高州地区,现存的冼夫人庙便有二百余座\footnote{贺喜:《亦神亦祖:粤西南信仰构建的社会史》,页194}。其中,最古老的旧城村冼夫人庙据说始建于隋治南粤时期,已有约1400年的历史。该庙有三进,分前、中、后殿,最后一进正殿中供奉着冼夫人及其丈夫冯宝。在正殿之中,安放着一面复制铜鼓,象征着冼夫人深厚的百越传统。冼夫人庙之旁则有一间同样为三进的冯氏宗祠。该祠堂在旧城村两千多名冯姓人口的社会凝结核,安放了自一世祖冯宝公至第五十二代祖先在内的近700个祖宗牌位。当地的冯姓村人皆认为自己是冯宝与冼夫人的后代,视冼夫人庙与冯氏宗祠为最重要的信仰和祭祀中心\footnote{贺喜:《亦神亦祖:粤西南信仰构建的社会史》,页195—196}。

在粤西人眼中,冼夫人不仅仅是威严的神明与祖先,亦亲切得犹如家人一般。在今日的高州,人们亲昵地称呼她为“冼太”、“冼太阿婆”。究竟是何种原因使得粤西人对冼夫人如此尊敬又如此亲切?欲回答这一问题,便必须回到那风云激荡的历史场景中去寻找答案。

公元548年,原为东魏将领、后向萧梁投降的羯人侯景于寿阳(今安徽淮南)竖起反旗,著名的侯景之乱爆发。在此次大乱里,统治帝国达四十五年之久的梁武帝萧衍饿死于重围中的建康城,萧梁帝国在大洪水中走向了它的末日。在当时的南粤,陈霸先刚刚崛起为军事霸主。他虽然团结了一批流寓吴人与南粤豪族,却仍不能控制遍布着百越部落的粤西。在当时的粤西,生活着两个百越族群。一个被称为“僚”(被帝国史书污蔑性地写为“獠”),系上古时代雒越人的直系后裔\footnote{《广东通史》古代上册}。他们居无定所,“巢居海曲,每岁一移”,用铜鼓召集部众,并时常野蛮地掠人为食\footnote{胡守为:《岭南古史》,页251}。相形之下,另一个被称为“俚”的族群的文明程度似乎更高。俚人乃西瓯、雒越、南粤人的后代,是骁勇善战的部落战士。他们延续着南粤本土部落自百越时代以来的传统,以铜鼓为部落中的至关重要之物。据孙吴治粤时期的史料记载:

\begin{quote}
	(俚人)风俗好杀,多构仇怨。欲相攻击,鸣此鼓(铜鼓)集众,到者如云,有是鼓者极为豪强\footnote{胡守为:《岭南古史》,页248}。
\end{quote}


可见,作战、仇杀是俚人部落间关系的常态。每当战斗即将到来时,铜鼓便发挥召集部落战斗人员的作用。在战斗中,俚人亦会敲响铜鼓。若一部落被另一部落打败,那么胜利者便会夺取失败者的铜鼓,视之为重要战利品。俚人虽然经常作战,但绝不残忍卑鄙,“质直尚信”是他们的美德。《隋书》曾在样描述俚人的风俗:

\begin{quote}
	巢居穴处,尽力农事。刻木以为符契,言誓则至死不改。
\end{quote}

由是可知,勤劳耕作、遵守承诺是俚人的美德。他们会以性命承担自己立下的誓言,互相之间充满信任。他们是道德高尚的部落民,未曾被帝国的武断之治污染,保留着刚健勇武的特点,享受着百越人自古以来理所当然的自由。

公元513年,粤西高凉郡(今阳江至吴川一带)俚人酋长冼氏的女儿冼英出生,她便是著名的冼夫人\footnote{关于冼夫人生卒年的考证,参见冯仁鸿:《试论冼夫人享年——兼谈冯太守生卒年庚》,}。据史书记载,冼氏“世为南越首领,跨据山洞,部落十余万家。\footnote{魏征:《隋书》卷84《列女第四十五冼夫人传》}”若以一家五口计算,则冼氏控制的人口至少亦有五十万。当时,粤西尚为一未被帝国深入“开发”的区域,帝国官员仅能龟缩于郡城、县城内,对于城外的蛮族世界几乎毫无控制力。高凉郡的实际控制者,实为冼氏统率的俚人部落。在百越的传统中,许多女子与男子一样骁勇善战。南粤历史上,如二征姐妹、赵妪一般的越人巾帼英雄不绝如缕。冼夫人便是又一位在样的巾帼豪杰。在少女时代,她便随父母参加部落战争,积累了“行军用师”的经验。她亦曾用百越的传统习惯法“劝亲族为善”,以信义凝结部落共同体内部成员\footnote{魏征:《隋书》卷84《列女第四十五冼夫人传》}。

冼夫人16岁时,她的家族与同胞遇到了空前的危机。公元528年,梁西江督护孙冏、新州刺史卢子雄纠集数万大军,越过阳春北界,大举入侵高凉俚人的土地。残暴的梁帝国侵略军采取恐怖屠杀政策,将大批俚人村寨变为尸山血海。在保卫家园的战斗中,冼夫人的父亲、叔父壮烈战死。率领部族抵抗侵略者的任务,突然落到了年少的冼夫人与她的兄长冼挺身上。兄妹二人勇敢地承担起了这一重任,率部退守今电白、阳春、高州间的山区,利用对地形的熟悉与侵略者进行了长达七年的周旋。在俚人神出鬼没的游击战下,梁帝国侵略军伤亡惨重,毫无进展。至535年,在粤西山中丢下了成千上万具尸体的梁军停止进攻,俚人抗击梁军入侵的战争宣告结束\footnote{关于此次七年战争,参见刘国光:《爱国爱民,恩威并施——浅论冼夫人治政之道》,《南方论刊》2004年第1期,页57—58}。这一场持续了七年的义战,是南粤自击毙秦尉屠睢、海南岛民驱逐汉军之后的又一次伟大胜利。它又一次向岭北帝国昭示,粤人绝不会屈服。为了保卫南粤的自由,我们的伟大祖先会与侵略者进行永不停息的斗争,令南粤大地成为侵略者的坟墓。

在惨烈的战争中,冼夫人凭借着她的勇武与美德成为冼氏当之无愧的领袖\footnote{史载,冼夫人曾劝阻其兄冼挺“侵略傍郡”的行为,其在部落中的地位当在其兄之上。参见魏征:《隋书》卷84《列女第四十五冼夫人传》}。在击败梁帝国侵略军的同一年,她迎来了幸福的婚姻,与26岁的梁高凉太守冯宝成亲。冯宝出身于鲜卑化的北燕皇室,其曾祖父冯业于436年“浮海归宋”,被安置于南粤。自冯业至冯宝之父冯融,冯氏在粤西“三世为守牧”,成为为南朝宋、齐、梁镇守南土的地方官。身为北人的冯氏得不到俚人的信任,完全无法号令城外的俚人部落。至此,冯融为冯宝娶冼夫人为妻,力图使冯氏融入南粤本土的部落共同体中。冯氏的主动粤化终于产生了效果:在冯宝与冼夫人成亲后,粤西在两人治下出现了“政令有序”的安定局面。自此开始,冯氏便与冼氏统治下的俚人融为一个紧密的共同体,开始了保卫南粤的奋斗。

在婚后第五年,冯宝与冼夫人派兵渡过琼州海峡,招降了海南岛上的一千余个村落。在冼夫人的要求下,梁帝国于海南岛上置“崖州”,承认了冯冼对海南的拓殖。这一举动对于岭北帝国而言颇有象征意义:自公元前46年汉军放弃海南岛后,经586年的岁月,海南岛又一次进入了帝国的行政区划中。然而,此时海南的实际控制者无疑是冯冼,而非萧梁朝廷\footnote{魏征:《隋书》卷84《列女第四十五冼夫人传》}。

公元548年冬,侯景叛军围困梁都建康的消息传至南粤。时为西江督护的陈霸先决定北上“勤王”、讨伐侯景。然而,心怀不轨的广州刺史元景仲却在此关键时刻起兵响应侯景。549年七月,经半年多的僵持,陈霸先调集义兵发动攻势,元景仲于绝望中自缢\footnote{姚思廉:《陈书》卷1《本纪第一高祖上》},陈霸先遂迎梁宗室萧勃为新刺史\footnote{贺喜:《亦神亦祖:粤西南信仰构建的社会史》,页23。}。然而,萧勃亦是个野心勃勃、在政治上不甘寂寞的人。他暗中联络南康(今江西赣州)土豪蔡路养,指使其发兵阻击陈霸先。公元550年,陈霸先率一支不足三万人的南粤“勤王”军从粤北始兴出发,翻越大庾岭,开始了他的北伐。在南康城外,蔡路养已纠集二万人马严阵以待。然而,这些乌合之众根本无法抵挡轻锐的南粤勇士。经过一场激战,蔡路养军惨败,其本人落荒而逃,南康遂落入陈霸先与南粤兵之手\footnote{胡守为:《岭南古史》,页198}。

与此同时,不甘寂寞的萧勃仍在想方设法掣肘陈霸先。他命令早已有意投降侯景的高州刺史李迁仕率兵北上,参与北伐战事。李迁仕急欲利用冯冼麾下精悍的俚人战士,遂召见冯宝。关键时刻,冯宝正欲出发与李迁仕会面,幸被识破李氏阴谋的冼夫人及时阻止。夫妻两人的对话如下:

\begin{quote}
	宝欲往,夫人止之曰:“刺史无故不合诏太守,必欲诈君共为反耳。”
	宝曰:“何以知之?”
	夫人曰:“刺史被召援台(即台城,建康皇宫所在地),乃称有疾,铸兵聚众,而后唤君。今者若往,必留质,追君兵众。此意可见,愿且无行,以观其势。\footnote{魏征:《隋书》卷84《列女第四十五冼夫人传》}”
\end{quote}

此段对话中,冼夫人指出了李迁仕召见冯宝一事的蹊跷之处,认为若冯宝赴召,定会被叛心已显的李氏扣押,彻底丧失兵权。其后的事实证明,冼夫人的推断完全正确。不久后,李迁仕军占领大皋口(今江西吉安),果然竖起了反旗。他又遣骁将杜平虏占领灨石(今江西万安),在陈霸先的北伐道路上制造了巨大障碍。一时之间,陈霸先陷入了难以继续北进的困境。在此紧要关头,冼夫人做出了出兵帮助陈霸先的重要决断。率兵出征前,军事经验丰富的她对冯宝严密地分析了战局:

\begin{quote}
	平虏,骁将也,领兵入灨石,即与官兵相拒,势未得还。迁仕在州,无能为也。若君自往,必有战斗。宜遣使诈之,卑辞厚礼,云身未敢出,欲遣妇往参。彼闻之喜,必无防虑。于是我将千余人,步担杂物,唱言输赕,得至栅下,贼必可图\footnote{魏征:《隋书》卷84《列女第四十五冼夫人传》}。
\end{quote}


冼夫人提出的作战计划,是由她本人率千余精锐佯装向叛军输送赎罪财物(输赕),从而接近李迁仕盘踞着的大皋口,一举克之。由于杜平虏正在灨石阻击陈霸先,根本无法回援。为降低叛军的戒备心理,此次出征冯宝不应露面,而应由身为女子的冼夫人指挥。这一计划大胆巧妙的可谓有勇有谋,充分表现了冼夫人高超的战争艺术。不久后,战局走向一如冼夫人所料。大皋口守军见“夫人众皆担物,不设备”,遂被冼夫人所率的千余俚人精兵迅速击溃,李迁仕狼狈地退守宁都\footnote{魏征:《隋书》卷84《列女第四十五冼夫人传》。不同史书对大皋口之战的记载不同。《隋书》称大皋口系由冼夫人攻下,本书即采信此种说法。然而,《陈书》中的记载则称,攻下大皋口之人乃陈霸先麾下猛将周文育、杜僧明。参见姚思廉:《陈书》卷8《周文育传》}。其后,陈霸先与冼夫人这两位越人的英雄在灨石会面。551年,在陈冼联军的合击下,李迁仕被擒杀。在冼夫人的帮助下,陈霸先北伐的道路被打通了。

打败李迁仕后,冼夫人率领俚人子弟兵凯旋还乡。她对于与陈霸先的灨石之会印象深刻,认定陈霸先必能做出一番更大的事业,遂对冯宝说:

\begin{quote}
	陈都督大可畏,极得众心。我观此人必能平贼,君宜厚资之\footnote{魏征:《隋书》卷84《列女第四十五冼夫人传》}。
\end{quote}


冼夫人对陈霸先的期许果然没有落空。公元552年,陈霸先与萧梁名将王僧辩攻占建康,侯景败死,惨烈的侯景之乱至此落下帷幕。对于南人而言,此次乱事虽是一次惨烈的大洪水,亦有很强的正面效果。两晋之际由北方南渡的南朝士族,经此一乱而被消灭殆尽。两年后,又发生了西魏军攻陷江陵、将残存士族尽数北掳的剧变。持续二百余年的南朝士族政治遂告终结,南人豪强的政权呼之欲出\footnote{万绳楠整理:《陈寅恪魏晋南北朝讲演录》,页200}。同年,陈霸先将亲近北齐的王僧辩擒杀。公元557年十月,陈霸先逼迫梁敬帝萧方智向他“禅让”,建立陈朝,是为陈武帝。

面对陈武帝步步“篡夺”皇位的行为,身为萧梁宗室的广州刺史萧勃无法置之不理。557年二月,即陈霸先登基前八个月,萧勃于广州起兵反陈。三月,萧勃率军越大庾岭,进至南康,其前锋更已到达巴山(今江西抚州西)、跖口(今江西南昌南)。在时,陈武帝麾下名将猛将周文育发动猛烈反击,击溃萧勃之前军。消息传到南康,萧勃军中大惧,其本人被乱兵杀死。不久,陈军进至粤北始兴,南粤诸郡遂纷纷向陈武帝投降。

在这场易代战乱中,冼夫人依靠她的威望约束粤西各部落,使当地未受兵戈之祸,“数州晏然”\footnote{魏征:《隋书》卷84《列女第四十五冼夫人传》}。陈武帝上台后封冯宝为“护国侯”,允许其居于粤西阳春郡城,完全承认了冯冼在粤西的自立。

对南人来说,侯景之乱与江陵之变起到了清道夫的作用,不但宣告了华夏君统的断绝,亦消灭了南朝的北人士族\footnote{关于江陵陷落标志着华夏君统断绝的论述,参见刘仲敬:《经与史》,页192}。冼夫人的自立、陈霸先的政权,也就是南粤与吴越的自由,都建基于此。经过梁末剧变,吴越与南粤解放了,南人解放了。吴越人陈霸先依靠冼夫人与南粤兵夺回南人自由的故事成为了一段佳话,承载着百越联合抗击北人帝国的伟大历史。然而,这一自由是脆弱的。在不久的将来,冒名顶替华夏的鲜卑帝国将继承北人士族的未竟之业,对南人的自由发起更为疯狂的进攻。

\section{守护南粤的“圣母”:冼夫人在历史节点的决断}

公元558年,冯宝病逝于阳春。其时,他与冼夫人的儿子冯仆只有九岁,冯冼家族迎来了危机时刻。若陈武帝欲以冯氏无合格继承人为借口实行削藩,那么南粤刚刚争取到的自由又将失去。然而,陈武帝终究不是北人皇帝,而是南人豪强的盟主。他不会如专制大君一般卸磨杀驴,而是会继续与支持他的豪强们亲善。同年,在冼夫人的安排下,九岁的冯仆在南粤各部落酋长的护卫下前往丹阳朝见陈武帝\footnote{关于冼夫人安排各部酋长护送冯仆往见陈武帝的情节,参见钟万全:《巾帼夫人冼夫人》,页42},向其献上由扶南(柬埔寨)进口的犀杖。陈武帝亦投桃报李,授冯仆为阳春太守\footnote{魏征:《隋书》卷84《列女第四十五冼夫人传》}。据南朝制度规定,如有人欲为官,需“年三十而仕”\footnote{杜佑:《通典》卷14《选举二》}。如此看来,陈武帝将太守之职授予九岁的冯仆,实为违犯常规的“法外加恩”。陈武帝对冯冼家族的信任,由此可见一斑。一年后,陈武帝驾崩,结束了他为南人自由而奋斗的壮阔人生。此后十年间,冯仆在冼夫人的监督下渐渐成长为一个合格的俚人领袖。据说,在母亲的安排下,他迎娶了被阳春人称作“金花夫人”的表亲冼氏,冯冼之间的关系因此变得更为紧密\footnote{《冼冯大事记》}。与此同时,冼夫人成为了高州、罗州(今湛江一带)、新州(今台山)、泷州(今罗定)的实际统治者,将阳春郡城治理得日渐富庶,成为商业、金属冶炼业发达的粤西核心都会\footnote{钟万全:《巾帼英雄冼夫人》,页43}。

陈朝初年,代表陈廷统治南粤者系广州刺史欧阳。欧阳系长沙临湘人,早年曾任清远太守,后于侯景之乱时投奔陈武帝。陈武帝对他颇为信任,于翻越大庾岭北伐前任命他留守粤北要地始兴\footnote{胡守为:《岭南古史》,页197}。在萧勃之乱中,欧阳一度倒向叛军,后又反正并得到陈武帝的宽恕,被任命为广州刺史。公元563年,欧阳去世,结束了他反复无常的一生,其子欧阳纥继承父职。六年后,对欧阳氏在粤势力深感忧虑的陈宣帝遣老臣沈恪任广州刺史,召欧阳纥入京任左卫将军,意图调虎离山。充满野心的欧阳纥不欲离粤,乃于广州发动叛乱\footnote{胡守为:《岭南古史》,页203}。

作乱前夕,欧阳纥以商议军情为名召见冯仆,将其诱捕并关押于广州。欧阳起兵后,更派人将冯仆的求救信送予冼夫人,意图迫使她投降。此时,冼夫人已年近六十,丧夫多年的她难以再承受丧子之痛。然而,为了南粤的自由、为了报答陈朝对南粤的恩义,冼夫人慨然说出了她的答复:

\begin{quote}
	我为忠贞,经今两代,不能惜汝,辄负国家\footnote{魏征:《隋书》卷84《列女第四十五冼夫人传》}。
\end{quote}

对冼夫人而言,陈朝与之前统治南粤的帝国截然不同,乃是一代表南人利益的政权。维护陈朝在个“国家”,便等同于维护南粤。冼夫人不顾身家地捍卫南粤自由的精神足以感天动地,令无数爱粤之人击节赞叹。在冼夫人的身上,无疑展现着我们的伟大祖先最优秀的品质。

回绝了冯仆的求救后,冼夫人立刻动员粤西各部酋长出兵守卫疆界,等待陈军支持。不久后,陈车骑将军章昭达率军自湟水道(今连州)入粤,欧阳纥亦亲自率大队人马北上迎击,两军相持于湟水。次年(570年)三月,在与章昭达通信后,冼夫人率大军自阳春出征,轻取防守空虚的广州城,将冯仆自狱中救出。至八月,在冼、章两军的夹攻下,叛军于浛洭(今清远)一带的北江江面上展开最后的困兽之斗。叛军设置了横江铁链、木桩,并以大船守江。然而,他们完全不是粤西蛮族的对手。在冼夫人与冯仆的指挥下,来自粤西山区、敏捷矫健的俚人官兵乘竹木排撞开了叛军在江中设置的障碍物,并乘风放火焚烧敌船,使叛军陷入混乱。章昭达大军随即顺流而下,一举击溃叛军,俘虏了欧阳纥,将其送往建康斩首\footnote{钟万全:《巾帼英雄冼夫人》,页46—47}。缺乏政治德性的投机者欧阳氏父子,自此彻底退出了历史舞台。在冼夫人的战斗下,南粤终究没有落到他们手中、没有成为资助野心家争霸天下的奶牛。对于冼夫人母子的平叛之功,陈宣帝大加封赏,封冯仆为信都侯、平越中郎将、石龙太守,并特别遣使持节封冼夫人为中郎将、石龙太夫人,赐予其与刺史相同的出入仪杖\footnote{钟万全:《巾帼英雄冼夫人》,页46—47}。至此,冼夫人已经成为无可置疑的南粤领袖,南粤已接近完全自立。

公元584年,冯仆以35岁之龄英年早逝,留下了冯魂、冯暄、冯盎这三位年少的子嗣。作为家族中的最高长辈,时年已72岁的冼夫人一面祖母的身份养育着三个爱孙,一面以南粤守护者的姿态警惕地注视着岭北的动向。589年,惊天剧变发生了。当年二月,隋帝国侵略军在晋王杨广的指挥下渡过长江,占领建康,消灭了陈朝。然而,隋军止步于岭北,迟迟不敢翻越南岭进入他们并不熟悉的南粤。在此微妙关头,我们的祖先决定不再忠于任何势力,而是宣告独立。在岭南各郡的拥戴下,冼夫人被加上“圣母”称号,成为了保境安民的南粤君主\footnote{魏征:《隋书》卷84《列女第四十五冼夫人传》}。这样,在南越国灭亡后整整七百年,南粤又一次彻彻底底地脱离了帝国统治。漫长的第三次北属时代,至此正式结束。

冼夫人君临南粤的壮举使美丽富饶的粤土躲过了陈隋之际的易代战乱。在此期间,隋文帝派出总管韦洸接管南粤。韦洸见冼夫人势大,不敢贸然过岭。双方的僵持持续了九个月。至当年十一月,杨广遣人将陈后主的劝降书信送予冼夫人,并送回了当年冯仆献给陈武帝的扶南犀杖。见到这一象征粤吴联盟的信物,冼夫人终于明白,陈朝永远地灭亡了,那个南人自己的政权永远消逝了。她召集数千名酋长、土豪,在他们面前“尽日恸哭”\footnote{魏征:《隋书》卷84《列女第四十五冼夫人传》}。冼夫人知道,失去了盟友的南粤若与巨兽般的隋帝国放手一战,将很可能毁掉来之不易的和平与繁荣,使南粤大地再次生灵涂炭。为了保卫南粤的山河与百姓,她必须隐忍、必须妥协。直到今天,我们仍然能够感受到这位78岁的老人内心的极度痛苦、体会到她对南粤无比深沉的爱。她已为南粤付出了太多太多,亦承担了无比沉重的责任。行文至此,笔者不禁暂时搁笔,与她一同放声痛哭!

公元589年十一月,冼夫人遣长孙冯魂率众迎韦洸入广州城,南粤各地皆向隋帝国臣服。隋文帝“追赠”冯宝为广州总管、谯国公,“册封”冼夫人为谯国夫人\footnote{魏征:《隋书》卷84《列女第四十五冼夫人传》}。对于冼夫人在无奈下的降隋之举,并不是所有南粤爱国者都会赞同。公元590年,番禺人王仲宣发动反隋起义,围攻广州。在守城战中,韦洸中流矢战\footnote{司马光:《资治通鉴》卷177《隋纪一》}死、冯魂亦阵亡\footnote{《冯冼大事记》}。冼夫人令冯暄率兵救援广州,却遭到了冯暄的拒绝。原来,王仲宣的部将陈佛智是冯暄的好友。冯暄实在不愿向自己的朋友举起屠刀、不愿参加这场粤人自相残杀的悲剧。万分痛苦的冼夫人把冯暄关入狱中,将救援任务移交给三孙冯盎\footnote{《冯冼大事记》}。善战的冯盎进兵神速,很快便取得胜利,斩杀陈佛智,又于南海与隋军合兵击败王仲宣。如同郭马一样,王仲宣的最终下落未被史籍记录,他很可能被隋帝国残酷杀害了。冼夫人与王仲宣的战争,实为两个南粤保卫者间发生的惨痛悲剧。

击败王仲宣后,年近耄耋之龄的冼夫人“亲被甲,乘介马,张锦伞,领彀骑”,威风凛凛地巡阅南粤各地。苍梧、冈州(今新会)、梁化(今惠州一带)、藤州(今广西藤县东北)、罗州等地的豪酋皆向她表示效忠,南粤再次安定下来。冼夫人的这一举动表面上是在胁迫南粤土豪对隋臣服,实则亦在向隋帝国宣示:她虽然已经年老,但仍能号令整个南粤。如隋帝国逼之太甚,南粤将在她的率领下战斗到底。

震慑于冼夫人的赫赫威势,隋帝国不得不妥协,允许她开设谯国夫人幕府,自置长史以下官属,有权调遣南粤六州兵马,且能“便宜行事”。这样,阳春便成为南粤事实上的政治中心,广州则仅是缺乏实权的隋帝国流官驻地。此外,冯暄亦被释放并授罗州刺史,冯盎获授高州刺史\footnote{魏征:《隋书》卷84《列女第四十五冼夫人传》}。这样,经过一番豪气万丈的武力展示,南粤在事实上回到了粤人手中。

在人生的最后十余年中,冼夫人继续勤勤勉勉地守护着南粤,不敢有丝毫懈怠。对于虐待粤人的隋帝国流官,她更是予以坚决的打击。因隋番州总管赵讷贪虐导致百姓逃亡,她遣长史张融上疏极论赵讷恶行。隋文帝不得不派人诛杀赵讷,让冼夫人“招慰亡叛”。公元600年前后,年近九十的冼夫人最后一次巡阅南粤大地。凡她所过之处,各地俚人、僚人无不热情欢迎、表示效忠。这一次,她既是在用尽最后一丝气力守护南粤,又是在向南粤告别。长途巡阅彻底拖垮了她的身体。公元602年,90岁的冼夫人永远离开了人世、离开了她守护了近一个世纪的粤土与粤人。她可以休息了,可以在彼岸世界成为神明、平静地庇佑南粤了。今天粤西、海南无数座冼夫人庙中连绵不绝的香火,便是粤人永远不会忘记她的最好证明。只要木棉花仍会盛开、只要南岭仍然挺立、只要南海仍未干枯,粤人便会永远视她为如家人般亲切的“冼太”、“冼太阿婆”,永远尊经她、怀念她。事实上,她从未离开。

\section{与岭北强权周旋的现实主义者:冯盎}

公元602年,冼夫人去世,次孙冯暄、三孙冯盎一同继承了她的事业。冯暄性格善良温和,在讨伐王仲宣之战时曾因念及旧情抗命,冯盎则自年少时起便有勇有谋\footnote{刘昫:《旧唐书》卷113《列传第五十九冯盎》}。当时,粤西北成州(今封开)及粤东潮州的僚人乘冼夫人新丧之机发动叛乱。冯盎意识到这是一个利用隋帝国提高其地位的良机。在冯暄作出反应之前,大胆的冯盎已火速北上驰至隋都大兴(长安),请隋文帝允许他出兵讨伐“叛僚”。为试探冯盎,隋文帝命重臣左仆射杨素与之议论战局。史书中虽未记载两人之具体谈了什么,但可以确定的是,杨素在谈话之后对冯盎十分欣赏,说出了如下评语:

\begin{quote}
	不意蛮夷中有此人,大可奇也\footnote{刘昫:《旧唐书》卷113《列传第五十九冯盎》}!
\end{quote}

冯盎以其优异的才能折服了隋文帝君臣,隋文帝同意派出长江、南岭间的部分军队援助他。在隋军支援下,冯盎很快便平定了成州、潮州的僚人叛乱,获授金紫光禄大夫、汉阳(今甘肃西和县西)太守\footnote{两唐书皆称冯盎在讨伐成州、潮州叛乱时调动了“江岭兵”,“江岭兵”之“江”很可能指长江。参见刘昫:《旧唐书》卷113《列传第五十九冯盎》;欧阳修:《新唐书》卷35《诸夷蕃将》。此处之汉阳并非现代之汉阳,见《广东通史》古代上册,页426}。在公元612年的隋炀帝杨广征高句丽之役中,活跃着冯盎及其长子冯智戴的身影。从战后他被升为武卫大将军来看,他在辽东战场上应是立下过战功的\footnote{《广东通史》古代上册,页426}。

隋帝国将冯盎派往西北边地任官,无疑是不想让如此优秀的人才为南粤效力。在隋帝国的官僚体系中,冯盎步步高升,看似对帝国十分忠诚。然而,这一切不过是他捞取政治资本、赚取战争与行政经验的手段。在隋帝国真正面临危机时,他并不会为隋室卖命。公元613年,杨素之子杨玄感于黎阳(今河南浚县东北)起兵反隋,隋末大洪水爆发。616年,江右饶州人林士弘于豫章称帝,定国号“楚”,拥众十余万,雄据全赣。618年,又有梁朝皇族后裔萧铣于湖湘岳阳称帝,恢复萧梁,号称拥兵四十万,据有全湘。同年,隋炀帝遇弑于江都,李渊于长安称帝建国,是为唐高祖。至此,中原已完全陷入易代战乱中,湖湘、江右则各出现了一个对南粤虎视眈眈的强大政权。在此危机关头,仁柔的冯暄未能如他的祖母冼夫人一样站出来保卫南粤,更不能团结全粤土豪。在番禺和新兴,已有高法澄、冼宝彻两人公开宣布投靠林士弘,四处破城杀官,意图将岭北大洪水带进南粤。在此危机关头,冯盎做出了惊人的决定。为了拯救粤人的家园,冯盎父子由关中一路奔回南粤。对于他们一路上的经历,帝国的史书十分吝惜笔墨,完全没有记载,我们甚至不知道他们具体走了哪条路线。然而,从当时岭北的混乱情况来看,这一超长途一定充满了难以种种难以想象的危险。当冯盎越过五岭,如天神一般回到了他阔别近二十年的南粤时,大批土豪、酋长立刻热情地欢迎他、投入他的麾下。很快,他的军队便膨胀到了五万人。冯盎随即率军闪击高法澄、冼宝彻。在和二人的军队交战时,冯盎每次都于阵前脱下甲胄,高呼“若等识我耶!?\footnote{《广东通史》古代上册,页426}”在两个粤奸与他们的部下完全没有料到冯盎居然会回来,皆大惊失色,纷纷丢下武器,或投降、或奔逃。很快,两人便都被俘获。

直至今天,冯盎在阵前释胄高呼的英姿仍令人神往不已。而他长途奔回南粤保护此一方河山的壮举,更使人无比感动。他不愧是冼夫人之孙,不愧是俚人战士与鲜卑骑士的后代,不愧是出类拔萃的南粤武士。由于他的壮举,南粤得以避免了大洪水的侵袭。对于他的这一伟大功绩,我们绝不应忘记。

在击败依附于林士弘的高、冼两人后,冯盎又做出了看似违背常理的决定,向林士弘表示效忠\footnote{对于冯盎击败高、冼二人及依附林士弘之时间的问题,史籍记载混乱。《旧唐书》系冯盎击败高、冼之战于620年,《新唐书》系之于隋亡时(当为618年),而《资治通鉴》又系冯盎依附林士弘于616年。然616年时隋未亡,冯盎尚未回粤,无依附林士弘的可能。若冯盎先依附林士弘,再讨伐忠于林的高、冼二人,于理不合。因此,依附林士弘当在击败高、冼二人之后。参见司马光:《资治通鉴》;刘昫:《旧唐书》卷113《列传第五十九冯盎》;欧阳修:《新唐书》卷35《诸夷蕃将》}。当时,林士弘在与萧铣的战争中正处于下风,其都城豫章已被攻取。冯盎做出如此举动,一则料定虚弱的林士弘已难以对南粤造成实际威胁,二则为避免与林士弘产生进一步的不必要纷争。此后,林士弘果然未再于南粤生事,冯盎的外交策略取得了成功。至公元620年,包括广州、苍梧、珠崖等地在内的岭南“二十余州,地数千里”皆已成为冯盎的势力范围,各地土豪、酋长纷纷向他效忠。除据有粤西南北部湾沿岸、雷州半岛的钦州僚人豪酋宁氏仍向萧铣效忠外,南粤境内已无冯盎的敌手\footnote{《广东通史》古代上册,页427}。冯盎遂自称“总管”,成为南粤实际上的君主,兄长冯暄则成了他的部下。至此,南粤再次完全脱离了岭北的控制。

在冯盎称“总管”时,岭北形势发生剧变。621年,萧铣被唐帝国俘杀。次年,林士弘亦被唐军彻底打败,含恨病死。统一了岭北的唐军虎视眈眈,随时可能入侵南粤。在此期间,有部下劝冯盎称“南越王”,被冯盎拒绝。冯盎给出的理由是:

\begin{quote}

吾居越五世矣,牧伯唯我一姓,子女玉帛吾有也。人生富贵,如我希矣。常恐忝先业,尚自王哉\footnote{刘昫:《旧唐书》卷113《列传第五十九冯盎》}?

\end{quote}

如冯盎所言,南粤各地地方官已皆为冯氏族人,冯氏拥有巨大的财富,不宜再冒得罪唐帝国的危险称王,以免毁掉冯宝与冼夫人创造的一切。作为一个善于利用岭北势力为南粤的利益服务的人,冯盎决定将唐帝国变为新的利用对象。公元622年,冯盎向唐岭南道宣慰大使李靖投降。唐高祖划出高州、罗州、白州(今广西博白)、崖州、儋州、林州(今广西桂平)、振州(今三亚东北)为冯盎的领地,授其上柱国、高州总管、封越国公,并拜其子冯智戴为春州(今阳春)刺史、冯智彧为东合州(今雷州)刺史\footnote{刘昫:《旧唐书》卷113《列传第五十九冯盎》}。此外,钦州的宁氏亦向唐帝国投降,得以保留原有领地\footnote{《广东通史》古代上册,页434}。

投降唐帝国后,冯盎不再掌控全粤,但冯氏的根本重地粤西仍被他牢牢控制在手中,海南岛也仍是他的领地。可以说,冯氏的核心利益并未受到损害,南粤大地亦得以免遭兵火,冯盎的对唐战略不可谓之失败。事实上,因为冯盎被唐帝国授予高官,他甚至还能利用唐人给他的地位扮演“官军”,“合法”地扩张冯氏的直辖领地。公元623年,冯盎遣冯暄、谈殿率军攻陷宁氏领地南越州(今廉州)\footnote{欧阳修:《新唐书》卷222《列传第一百四十七南蛮》;数年后,冯暄于唐太宗统治初期去世}。公元627年,即唐太宗上台的第二年,冯盎又出兵攻陷新州\footnote{《冯冼大事记》}。在扩张领地的同时,冯盎每年皆派其子赴长安“朝见”唐帝,以示并无“叛”心。对于冯盎的此种行为,刚登基不久的唐太宗颇为震怒,欲发江淮之卒大举讨伐,却被魏征以“天下初定,疮痍未复”为由劝阻\footnote{欧阳修:《新唐书》卷35《诸夷蕃将》}。于是,唐太宗在628年初春向冯盎发出了一道杀气腾腾的敕书,其中有如下字样:

\begin{quote}
	卿已破新州,复劫数县,恐百姓涂炭,无容不即防御。闻卿自悔前愆,令子入侍,更令旋旆,不入卿境。此是朕惜卿本诚,意存含育。卿即有心识,应具朕怀……(卿)若其掠夺不止,衅恶日彰,欲人不言,其□□也。□至五月末以来宜遣一子,尽心奏闻。若无使至,朕即发兵屠戮卿之党羽,一举必无遗类\footnote{许敬宗:《文林词馆》。此敕流传至今,已有缺字,然不甚影响理解。转引自贺喜:《亦神亦祖:粤西南信仰构建的社会史》,页32}。
\end{quote}


此敕被学者称为《前敕》,是一封露骨的威胁信。唐太宗首先自称已对冯盎多有让步,接着语气一转,提出若冯盎不遣子“入朝”,便要出兵尽诛其众。这样一封带着“屠戮卿之党羽”、“必无遗类”等血腥字样的书信出自帝国“仁君”唐太宗李世民之手,不得不说是一个绝妙的讽刺。无耻的帝国奴文人为了制造唐太宗的“仁君”形象而炮制了大批伪史。然而,大概是出于对南粤的轻视,百密一疏的他们漏过了这唐太宗寄予“蛮夷”的威胁信,使今天的我们仍能通过它看清唐太宗的凶残面目。当这份材料摆到我们面前时,帝国伪史发明家终于因为轻视我们的祖先而付出了代价。

事实证明,唐太宗的在次威胁只是虚张声势。在冯盎遣次子冯智戴“入朝”后,唐太宗见好就收,不敢采取任何敌对行动。相反,唐太宗还在630年罢免了与冯盎不和的唐广州都督刘感\footnote{《广东通史》古代上册,页558}。可见,冯盎不但牢牢控制着粤西和海南,亦对广州城内的事务有很大影响力。631年,唐太宗向冯盎寄出了第二封威胁信,要求冯盎本人“入朝”:


\begin{quote}
	(卿)宜驰传暂至京师,旬日□□□尽心曲,便命旋轸,委以南方,子子孙孙,长飨福禄。倘其必存首鼠,不识事机,积恶期于灭身,强梁不得其死,自取夷戮,断在不疑。大兵一临,悔无所及\footnote{许敬宗:《文林词馆》。转引自贺喜:《亦神亦祖:粤西南信仰构建的社会史》,页34}。
\end{quote}


在这封被称为《后敕》的威胁信中,唐太宗继续采取虚张声势的凶残腔调,意图胁迫冯盎“入朝”。面对威胁,冯盎自有打算,善于利用对手的他希望再一次借助唐帝国增强自己的威势。于是,冯盎便在同一年亲赴长安“朝见”唐太宗,得到了丰厚的赏赐。通过对唐妥协,冯盎巩固了自己作为“官军”的名分。回到南粤后,他立即召集南粤各地豪酋,一同进攻罗州、窦州(今信宜)与他敌对的僚人部落。在战斗中,冯盎勇武异常,于阵前持弩连发七箭而中七人,使敌军大惊而逃。此战胜利后,冯盎不但又一次确认了南粤豪酋对他的忠诚,亦向唐太宗漂亮地展示了他的肌肉。唐太宗不得不默认他的地位,对他采取拉拢姿态,“赏赐不可胜数”\footnote{刘昫:《旧唐书》卷113《列传第五十九冯盎》633年十二月,唐太宗与太上皇(唐高祖)于宫中设宴款待“入朝”的冯智戴与突厥颉利可汗。宴会中,太上皇命颉利可汗起舞,又命“南蛮酋长”冯智戴咏诗,随即笑称:“胡越一家,自古未有也。”唐帝国便这样满足于“胡越一家”的幻像,全然不顾冯氏只是在利用他们而已。见司马光:《资治通鉴》卷194《唐纪十》}。冯盎以高州为中心,坐拥“地方二千里,珍货充积”的领地,成为了南粤无可争议的军事霸主\footnote{司马光:《资治通鉴》卷193《唐纪九》}。

公元646年,冯盎去世,结束了他传奇的一生。由于他的出生时间史籍缺载,我们并不知道他享年几何。作为一个善于外交斡旋的人,他清楚地知道该在何时妥协、何时展示实力。他的每一次选择都能恰到好处地利用岭北强权为南粤获利,隋唐帝国及林士弘皆成为了他用以保卫南粤的工具。南粤因而能够一次次化险为夷,在隋唐帝国的阴影下顽强地生存下来。然而,冯盎绝不是个无底线的马基雅维利主义者,更不是只会卖弄聪明的智术师。作为一名勇武过人的南粤武士,一旦南粤需要他挺身而出,他便会身先士卒地忘情战斗。他由关中历尽险阻返回故土拯救南粤的故事,更足以令每一个爱粤志士为之动容。我们伟大祖先的智慧与勇武,皆由他完美地呈现了出来。他是南粤史上的外交天才,更是当之无愧的南粤战神。

\section{冯氏的灭亡}

随着冯盎离开人世,南粤再无如他一般有威望的军事领袖,唐帝国终于敢对冯氏动手了。唐帝国的第一步行动是将冯氏的领地一分为三,分别交给冯盎的三个儿子。公元649年,即冯盎去世后仅三年,唐廷从高州中分出恩州(州治在今恩平东北),命冯智戴为刺史,高州刺史则由冯智戣担任\footnote{《广东通史》古代上册,页560}。次年,又置潘州(州治在今茂名),以冯智玳为刺史\footnote{《冯冼大事记》}。冯盎在世时,身为长子的冯智戴曾多次代表其父“入朝”,与唐廷中的一些高官颇有交游。650年,他迎娶了唐礼部尚书许敬宗之女,声势一时极盛\footnote{周忠泰:《南北朝至隋唐间岭南冯氏家族记事》}。其时唐太宗已死,唐高宗在位。为弄清楚冯氏究竟有多大实力,唐高宗派御史许瓘南下调查,却被冯智戴长子冯子游率数十人击铜鼓而擒拿。冯子游更“上奏”唐廷,称许瓘有罪。唐高宗不得不再派御史杨璟处理此事,而杨璟竟也险些被冯子游扣留。《新唐书》记载此事称:

\begin{quote}
	璟至,卑词以结之(“之”指冯子游),委罪以瓘。子游喜,遗金二百两、银五百两,璟不受,子游曰:“君不取此,且留不得归!\footnote{《新唐书》卷110《冯盎附子猷传》。冯子猷实名子游,有冯氏族人墓志铭可证,见《集赠潘州刺史冯君衡墓铭》}”
\end{quote}

可见,此时的冯氏领地虽已被分割,仍拥有相当大的自主权。此后,冯氏的历史变得模糊不清。公元676年,唐高宗开始在南粤推行“南选”制度,规定每四年将一批五品以上的官员调往岭南,取代粤人土官\footnote{《广东通史》古代上册,页450}。随着帝国流官一批批进入南粤,冯氏、宁氏等豪酋对南粤政治的影响力日渐衰落,他们的活动被帝国文人记录下来的机会也越来越少。令人痛心的是,安于现状的他们并未对唐帝国“温水煮青蛙”地控制南粤的行为做出有效抵抗。690年,武则天篡位,南粤豪酋的灭顶之灾随即到来。这时,已经没有像冼夫人、冯盎那样的英雄能够站出来了。

武周革命是一场打击门阀士族的重大政治变革。经此革命,东亚大陆向原子化社会迈出了关键的一步。在武则天看来,“割据”南粤达百年以上的“蛮夷”冯冼家族一如由鲜卑骑士后裔组成的关陇集团,是她建立扁平化帝国的巨大障碍,必须彻底铲除\footnote{关于武周革命的性质,参见刘仲敬:《经与史》}。693年,武则天首先屠杀了3400余名被流放到南粤的“流人”,其中多有李唐皇族。次年,她又诬称冯氏煽动僚人叛乱,杀害冯氏族人36家。697年,武则天命李千里为“讨击使”、率江淮之兵“征冯”。当年十月,武周侵略军自北面发动进攻,首先进入春州。经两个月交战,冯子游于十二月二十四日阵亡。两天后,其子冯梧在保卫春州西城县(今阳春三甲镇)的战斗中战死\footnote{冯峥:《岭南冯氏大迁徙》}。次年秋冬时分,武周军攻入高州州治良德城(今电白霞洞镇),杀害冯君衡(冯智戣之子),俘其子女\footnote{钟万全:《女皇武则天对阳春、高州的极大灾祸》}。李千里更将冯君衡之子、年幼的冯元一阉割,送入唐帝国宫廷充当宦官。在“征冯”之役中,武周军对春州(阳春)境内的南粤军民展开了灭绝人性的大屠杀,曾在冼夫人时代人烟辐辏、以金属冶炼与商业著称的春州至此沦为鬼蜮。直到四百年后,当北宋名儒周敦颐前往春州做官时,他看到的仍是一片“林木蔽天,迷雾笼罩”的情景\footnote{钟万全:《武则天“讨击”高凉的严重劫难》}。武周军对南粤犯下的罪行实在是深重至极,足以令人目呲尽裂!经此一役,冯冼族人或死于武周军屠刀下,或迁徙远方避祸\footnote{冯峥:《岭南冯氏大迁徙》}。至此,自公元535年起延续了163年的南粤冯冼家族,就在武则天的暴行下沉寂了。

冯氏的大部分男丁为保卫南粤的自由壮烈战死了。承载着国恨家仇的小宦官冯元一后被一高姓宦官收养,改名高力士。公元756年,当唐玄宗被乱兵困于马嵬驿时,他身边能够信任的人竟唯有高力士。唐玄宗信任高力士,是一个鲜卑血统的东方帝国专制大君,信任一个南粤土著出身、有鲜卑血统、做为降虏被帝国奴化的太监。唐玄宗与高力士这两个蛮族后裔,上演着岭北帝国羞辱南粤以及帝国宫廷内主奴互动的活剧,着实令人浩叹。在个故事表明,在武周革命之后,南粤俚人与鲜卑骑士都已被帝国彻底吞噬。在对奇怪的君臣在马嵬驿落寞的身影,便无声地诉说了这一场吊诡的惨烈悲剧。




