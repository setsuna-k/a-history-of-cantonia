\chapter{决定宪制的四场战争}

\section{对外开放路径的锁定:鸦片战争}

\indent 鸦片,本为18—19世纪间粤英贸易物之一种。因一系列阴差阳错的原因,纵横四海的大英帝国与侵占南粤的清帝国竟为之爆发了一场战争,进而极度深重地影响了此后百余年间南粤的对外关系,使南粤对外开放的路线得以锁定。在20世纪发明“中华民族”、“大东亚共荣者”的眼中,鸦片战争乃“全体中国人”乃至“全体黄种人”的莫大耻辱,象征着西方帝国主义势力开始全面入侵“中国”或“大东亚”。然而,若我们站在南粤的立场上考察此次战争,便会发现事实绝非如此简单。

鸦片是一种提炼自罂粟的产品,它富含吗啡,不但能够止痛和治疗腹泻,亦可使人精神愉快,从而对其上瘾。重度上瘾者若久不服用鸦片,便会萎靡不振。18—19世纪,世界各地都有大量鸦片成瘾者。在英国伦敦,许多人饮啤酒时要在其中添加鸦片,男性工人们则往往要在上工前服用鸦片丸以提神。在清帝国境内,情况也差不多,军队中多有瘾君子,许多苦力甚至“靠鸦片活着,鸦片就是他们的酒肉”\footnote{蓝诗玲:《鸦片战争》,页22}。

自17世纪烟草进入东亚起,吸烟之风便已蔓延整个东亚大陆。18世纪前期,一种新奇的吸烟方法传至清帝国境内,那便是吸食在鸦片浆汁中浸泡过的烟草。1729年,在全世界皆未视鸦片为违禁物的情况下,清帝国颁布法令,提出要用刑罚惩治贩卖鸦片者。但在此后一个世纪内,该法令形同具文。由于鸦片中80\%—90\%的吗啡能通过烟具冒出的或人呼出的烟挥发掉,吸食鸦片对人体的危害比直接服用的危害要小得多。在18、19世纪之交,鸦片不但在清帝国的穷人中大受欢迎,亦令许多富人为之痴狂。吸鸦片成为一种上流社会的社交活动,包括清帝国皇室在内的许多人都有此种习惯,由玉、象牙、龟壳制成的高级烟具因而成为一种时髦的尚品\footnote{蓝诗玲:《鸦片战争》,页27}。

1757年,英国东印度公司攻占孟加拉,随后垄断了在印度的鸦片制造权和贸易权。1784年,英国国会为打击茶叶走私贸易而将茶税降至12.5\%。自此,茶叶在英国成为人人皆可消费的廉价商品,导致其英人对闽、粤茶叶的需求激增,进而令大批白银从英人手中流入清帝国。为解决这一问题,英国东印度公司开始向广州输送鸦片以挽回白银外流\footnote{徐承恩:《躁郁的城邦:香港民族源流史》,页86}。至1816年,平均每年有3210箱鸦片流入广州。1831年,这一数字增至16500箱。1834年,英国东印度公司失去对清贸易专利权,大批私商争相加入鸦片贸易。这些商人与旗帮海盗留在南粤沿海的走私贸易网络联合,使鸦片不受任何控制地通过南粤流入清帝国各地。1839年一年内,流入清帝国的鸦片数量达到惊人的40000箱\footnote{魏斐德:《大门口的陌生人:1839—1861年间华南的社会动乱》,页33}。

鸦片贸易逆转了英清之间的白银流向,令清帝国君臣大为烦恼。1838年6月2日,清鸿胪寺卿黄爵滋“上奏”道光帝,杀气腾腾地提出应以一年为限强制戒烟,到期仍吸食鸦片者格杀勿论。道光帝将此奏折下发各地将军、督抚讨论,得到29人的回复,其中仅有8人表示支持。就在此事即将不了了之时,道光帝忽于同年10—11月间得知已有皇室成员染上鸦片瘾,且连与北京近在咫尺的天津亦出现了大量鸦片。道光帝对此表示震怒,遂于11月9日命时任湖广总督的林则徐入京\footnote{茅海建:《天朝的崩溃:鸦片战争再研究》,页90—93}。

林则徐系闽越侯官人,曾于1809年以幕僚身份协助清福建当局凶残镇压闽越英雄蔡牵领导的海上反清起义,是个极度无耻的闽奸和帝国奴。他于1811年中进士后平步青云,在1837年出任湖广总督。他不但极力支持黄爵滋的残酷提案,且身体力行,不待清廷下令就在湘、楚境内查抄了价值1.2万两的鸦片。道光帝对这条忠实的走狗十分满意,便决定以之办理广州禁烟事务。林则徐到达北京后,与道光帝连续密谈八次,每次皆谈至深夜,两人相得深欢。事后,道光帝对林则徐下达了“鸦片务须杜绝”的训令。林则徐随即出京南下,于1839年3月10日以“钦差大臣”的身份抵达广州\footnote{茅海建:《天朝的崩溃:鸦片战争再研究》,页96—102}。在他到达广州前两个月,与他臭味相投的两广总督邓廷桢已开始狂热地禁烟,逮捕了345名吸食者\footnote{魏斐德:《大门口的陌生人:1839—1861年间华南的社会动乱》,页37}。到达广州后,林则徐立即投入工作,于3月21日从外商手中强行收缴鸦片1037箱。三天后,他又下令停止广州外贸、封锁十三行商馆、撤走外商服务的所有粤人,将商馆内的350名外商全部软禁\footnote{茅海建:《天朝的崩溃:鸦片战争再研究》,页105}。广州市民则在林则徐的暴政下蒙受了更大的苦难。在林则徐的命令下,大批百姓被逮捕入狱。他们中的大部分人既不是鸦片贩也不是吸食者,只是被抓来充数或遭人陷害的无辜百姓。告密者通过分得被诬陷者的家产发了横财,无辜的人挤满了监狱,许多人惨遭斩首,更多人则在恶劣的囚禁条件和狱卒的虐待下如苍蝇般死去\footnote{魏斐德:《大门口的陌生人:1839—1861年间华南的社会动乱》,页39}。

当此惊变发生时,英国对清贸易商务总监查理·义律(Charles Elliot)正在澳门。义律出身于贵族家庭,他的祖父是名伯爵,父亲则是个军人,曾在对土耳其人的战争中表现出超凡的勇敢。自1815年加入海军后,义律在海上磨砺了十五个春秋。1830年,他被调往英属圭亚那担任“奴隶保护官”,富有同情心的他因而成了一个坚定的废奴主义者。1834年,他被任命为英国首任对华贸易总监律劳卑(Lord Napier)的侍从长,与贪婪成性的粤海关展开过多次周旋,有一次甚至被凶暴的清兵打了脑袋。自1836年继任律劳卑之职后,他一面维护英商的权利,一面又对鸦片贸易大加抨击。他曾说:

\begin{quote}

一项大宗贸易要依赖于一项稳定持续进行的走私,来买卖一种价格昂贵、又经常性地大起大落的邪恶的奢侈品,是不可能有好结果的\footnote{蓝诗玲:《鸦片战争》,页81}。

\end{quote}

由此可见,义律乃是一个极富正义感与骑士精神的英国绅士。当澳门听说林则徐已封锁商馆时,他十分愤怒,立即穿上全套海军服乘船直驶广州,并给英国外相巴麦尊(Henry Palmerston)写下一封毅然决然的快信:


\begin{quote}

我决心赶到商馆,否则就牺牲在路上\footnote{蓝诗玲:《鸦片战争》,页84}。

\end{quote}

强行进入商馆后,义律升起英国国旗,极大鼓舞了被困外商的士气。三天后,义律对林则徐做出上交20283箱鸦片的承诺,并对外商们保证英国政府一定会赔偿他们的损失。林则徐得知此事,遂于3月29日恢复对商馆的食物供应。在整个4月份内,上交鸦片的行动一直顺利进行。5月22日,林则徐要求以英商颠地(John Dent)为首的十六名大鸦片商做出“具结”,保证以后不再涉足清帝国。在义律的劝说下,十六人皆做出保证。两天后,义律得以与他们一同离开广州。6月3日,林则徐根据道光帝的命令在虎门当众销毁19176箱又2117袋鸦片,是为被大肆吹嘘的“虎门销烟”\footnote{茅海建:《天朝的崩溃:鸦片战争再研究》,页105—106}。至此,问题似乎已在义律的勇敢斡旋下得到解决。然而,不甘罢休的林则徐却立即再次挑起事端。

虎门销烟后,林则徐又向已回到澳门的义律发出“命令”,要求义律敦促英商以“货尽没官,人即正法”的格式“具结”,保证不再贩卖鸦片,否则就不准英人前往广州通商。义律对林则徐的突然要求大为反感,下令英船不得驶入广州港。7月7日,一群醉酒的英国水手在九龙尖沙咀和当地村民发生械斗,一位名叫林维喜的村民不幸重伤身亡,是为“林维喜事件”。林则徐乃揪住此事大做文章,要求义律将肇事水手交给清帝国审判。清帝国的刑罚以灭绝人性著称,其凌迟之刑更是骇人听闻。义律遂坚决拒绝,在英船上自立法庭,判处肇事水手3—6个月的监禁和15—20英镑的罚款。8月16日,林则徐以义律拒绝交凶为借口亲自率兵进驻香山,要求英人立即离开澳门。24日,澳葡当局屈从于林则徐的压力,将英人全部驱逐。义律和英商因此居无定所,只得乘船泊于香港、九龙洋面。31日,林则徐又下令沿海地区严防英人上岸取水\footnote{茅海建:《天朝的崩溃:鸦片战争再研究》,页126}。至此,义律等人已陷入绝境。

在林维喜事件中,林则徐如此强硬的态度绝非有爱于粤人,因为在他的暴政下早已有数以百计的无辜广州市民惨死。义律对肇事水手的惩罚虽失之过轻,但做为一个高贵的英国贵族,他绝无理由将自己的水手交给凶残的东方帝国。凡此种种事件,无不表明林则徐是个极度忠实的帝国奴。在清帝国官僚中,林则徐以清廉有操守著称。但就是这种帝国士大夫的模范人物都表现得如此丧心病狂,可见侵占南粤的清帝国已经愚昧疯狂到了何种程度。

1839年9月3日下午2时,走投无路的义律率三艘小船至九龙,要求当地清帝国官员提供食物,遭到拒绝,乃于4时45分向清舰开炮。清舰依托岸炮还击,双方交火至6时半,以英舰主动撤离告终。在时间不长的交战中,英人有数人负伤,清军亦仅有二死四伤\footnote{茅海建:《天朝的崩溃:鸦片战争再研究》,页128}。此次冲突虽小,却拉开了英清鸦片战争的序幕。第一场决定南粤近现代宪制的战争,终于爆发了。11月3日,双方舰队又在穿鼻短暂交火,清军战死15人、英军无人伤亡。此后,英舰驶往澳门,静待本土作出反应\footnote{茅海建:《天朝的崩溃:鸦片战争再研究》,页130}。

面对林则徐的倒行逆施,英国维多利亚女王迅速做出回应。1840年1月16日,她在国会发表演说,表示自己将密切关注清帝国的动向,并会捍卫“我国臣民利益与王室尊严”。4月7日,英国下议院经激烈辩论,最终以271票对262票的微弱优势通过对清开战决议。6月28日,英国远征军总司令懿律(George Elliot)、海军司令伯麦(James Bremer)率战舰16艘、武装轮船4艘、6000余兵力抵达南粤洋面,在留下战舰4艘、武装轮船1艘封锁珠江口后全军北上,于7月6日克定海(今浙江舟山),8月9日抵天津大沽口洋面,向清廷递交控诉林则徐在南粤残害外商的信件。因英军骤然逼近北京,道光帝大惊失色,忙命直隶总督琦善与英人谈判。登上英舰的琦善畏惧英人之坚船利炮,又不敢违抗道光帝,遂息事宁人地称只要英舰“返棹南返”,清帝便会派出钦差大臣前往广东与英人和谈。单纯的英人以为如此便真能解决事端,乃于9月15日同意返粤谈判。另一方面,对世界大势极度无知的道光帝以为“英夷”之所以北上大沽口,不过是为其在广东受到的“委屈”伸冤,乃以琦善替换林则徐。11月29日,新钦差大臣琦善抵达广州,林则徐被革职\footnote{茅海建:《天朝的崩溃:鸦片战争再研究》,页164—205}。对道光帝忠心耿耿、对南粤犯下滔天罪行的清帝国鹰犬林则徐,就这样被他的主子无情地抛弃了。

1840年12月3日,身处广州的琦善向在澳门的义律发出照会,双方议和正式开始。此后一个月内,两人往来照会达十五通。义律虽一直要求面谈,可胆小如鼠的琦善却一直避而不见。谈判中,琦善仅同意赔偿烟价600万元,但对义律割让岛屿、开厦门、定海为商埠的要求则置之不理。30日,道光帝认为谈判已无以再进,遂改“抚”为“剿”,要求琦善开战。1841年1月5日,义律也失去耐心,谈判宣告破裂。7日,英军以38人负伤的代价攻克虎门大角、沙角炮台,击毙清军自参将陈连升以下282人、击伤462人。在英军凌厉攻势的震慑下,琦善不得不在瞒住道光帝的情况下前往虎门与义律见面。2月12日,在经过长达半天的当面交涉后,二人拟定了一批条文,是为《穿鼻草约》,其主要内容为:

\begin{enumerate}
	\item 清帝国割让香港本岛及港口予英国。
	\item 清帝国赔偿600万银元。
	\item 清帝国开放广州为通商口岸。
	\item 英军撤出沙角、大角炮台,归还定海。
\end{enumerate}

由于事涉割地,琦善不敢在条约上签字,便欺骗义律称签字需延期十天,随后溜回广州。此后,他便把签字的事完全搁置下来,称病不出。见琦善迟迟不履约,英军乃再次发动攻势。2月26日,清方重兵把守的威远、靖远、镇远、横档、永安、巩固炮台亦告失守,清军阵亡自广东水师提督关天培以下250人、负伤100人、被俘上千人,而英军的损失不过hay 微不足道的轻伤5人\footnote{茅海建:《天朝的崩溃:鸦片战争再研究》,页227—231}。至此,清帝国精心经营的虎门炮台群已全被英军攻克。27日,英军克乌涌炮台,毙清军446人。3月2日,克琶洲炮台。3日,克琵洲炮台。6日,克猎德、二沙尾炮台。至此,英舰已进抵广州城下的珠江江面。

在此期间,已转为主张“剿夷”的道光帝以弈山为靖逆将军,以隆文、杨芳为参赞,由各地调集大军进入广州。2月20日,道光帝收到广东巡抚怡良关于琦善“私许香港”的密折,颇为震怒,下令将琦善“锁拿进京”。怡良系林则徐密友,对琦善私下议和的行为早已侦知。对他而言,此举无疑是给林则徐复仇的大好机会。做为旗人,琦善实为道光帝的奴才。私智膨胀的他试图在义律和道光之间长袖善舞、两头欺骗,却最终在道光帝的一怒之下落得如此下场,着实可悲可笑\footnote{茅海建:《天朝的崩溃:鸦片战争再研究》,页233}。

3月5日,参赞大臣杨芳抵达广州,为守城清军带来了一线希望。杨芳系湘西松桃厅人,曾于1828年率军深入西域活捉白山派反清首领张格尔,乃清帝国悍将。此时,杨芳已有71岁,早已丧失了过去的锐气。在他到达的第二天,英军攻克猎德、二沙炮台,但他完全不以为意,反而命人大肆收购西洋钟表、娈童、美女供他取乐。此外,愚昧至极的他又认为英舰火力之所以如此强大,是因为“必有善邪教者伏其内”,便“别出心裁”地搜集妇女用过的马桶,欲以此“至阴”之物破解“英夷”的“邪术”。在杨芳的命令下,清兵在城外江上大扎木排,置马桶于其上,摆出了所谓的“阴门阵”。这一令人喷饭的“奇术”自然不可能阻挡英舰。13日,英军再次进攻,克大黄滘炮台;18日上午,克凤凰岗、永靖炮台、西炮台、海珠炮台,并在西关登陆,于下午4时攻占十三行商馆,升起了英国国旗,随即向广州城垣开炮数十发。至此,英军已突入广州市区,兵临城墙之下。如此窘境下,杨芳只得欺瞒道光帝,称自己已取得大胜,仅在乌涌炮台就打死了446名英军\footnote{茅海建:《天朝的崩溃:鸦片战争再研究》,页263}。

直到这时,义律仍然试图与清帝国谈判。攻占商馆当天,他便向杨芳发出照会,要求面谈。杨芳拒绝见面,但同意进行书面交涉。3月20日,杨芳同意恢复广州外贸,并于4月3日与怡良一同将此事“上奏”道光帝。半个月后,收到此奏的道光帝勃然大怒,下令将二人革职留任。14日,弈山携带大批随从耀武扬威地抵达广州。随后,由清帝国各地南下的清军源源不断地开入广州城。至5月初,广州守军兵力已膨胀至2.5万,其中1.7万为由岭北开来的援军。这些援军中以杨芳麾下的湖湘兵军纪最为败坏。这些湘兵屯于东校场,离当时的麻风院不过二三里。入城购物的麻风病妇女必经过东校场,许多人都遭到抢劫、奸污,部分湘兵因此染上了麻风病。这些愚昧残暴的湘兵认为儿童的肉可以治疗麻风病,竟丧心病狂地掠杀、煮食广州儿童,不少儿童的父母、兄长也在争夺中被他们杀害。此外,他们还时常劫杀商贾,将广州城搅得鸡犬不宁。当时,广州城内有许多协助守城的番禺、南海乡勇。面对湖湘侵略者骇人听闻的暴行,义愤填膺的他们群起袭杀落单湘兵。一日,一名乡勇在城中被湘兵杀害,数百名南海乡勇便涌入由湘兵驻守的贡院进行报复,打得湘兵抱头鼠窜\footnote{江波:《道光时期广州社会治安研究》}。

当广州城内一片混乱时,弈山还在筹划对占领商馆的英军发动奇袭,却发现能听命于他的部队只有1700名水勇。5月21日白天,义律判明清军即将进攻,遂疏散商馆的所有英商,其本人于下午5时则登上英舰。深夜11时,弈山的奇袭终于开始,约百艘燃烧着的火船由上游漂向英舰,装载着清兵的战船则紧随其后。早有准备的英舰巧妙地避开了火船,一些火船甚至被冲到江岸并引发岸上大火。跟在后面的清军战船见火攻失败,遂一哄而散。22日,英舰溯江而上,击毁清军为再次火攻而准备的43艘战船、32艘火筏。至此,英军已扫清江面之敌。24日下午3时,360名英军组成的右翼纵队登陆西关,于两小时后再次攻占商馆。然而,这不过是英军为隐藏主攻方向而进行的佯动。夜9时,在郭富(Hugh Gough)少将的指挥下,2400名英军组成的左翼纵队开始在广州西侧的缯步登陆\footnote{茅海建:《天朝的崩溃:鸦片战争再研究》,页284—285}。25日晨,英军左翼纵队登陆完毕,开始向广州城北高地推进,于上午10时攻取越秀山的制高点和镇海楼,击毙清军500人。当夜,英军在越秀山顶架好了火炮。在5月21日—25日间的战斗中,英军的损失不过是9死68伤\footnote{茅海建:《天朝的崩溃:鸦片战争再研究》,页285}。

当5月26日黎明到来时,庞大的广州城已暴露在英军黑洞洞的炮口下\footnote{蓝诗玲:《鸦片战争》,页207}。hay 日,弈山被一群苦力堵在大佛寺前,他们想知道这位曾经不可一世的“靖逆将军”究竟有没有办法守住城池。弈山虽早已被英国人吓破了胆,却仍在苦力面前架子十足,命随从将为首的几个苦力当场斩首。大佛寺临近双门底大街,正是城中人烟最为密集的闹市区。目睹这一幕的人群一片惊慌,纷纷逃散,整个城市随之陷入混乱。湘兵、驻防旗兵乃至乡勇一面互相攻击、一面抢掠市民,市民中也出现了不少趁火打劫者。假如英军趁乱进攻,那么他们将迅速占领广州城。然而,当时正居于商馆的义律却派人致信郭富,要求他们不得进入广州市区,其理由为:

\begin{quote}

保护广州的人民,增进他们对我们的好感,可能是我们在这个国家主要的政治任务\footnote{蓝诗玲:《鸦片战争》,页211}。

\end{quote}

义律的义举使广州得以避免进一步破坏。至此,弈山已走投无路,只能与英人议和。当日傍晚,受弈山指派的广州知府余保纯前往商馆与义律展开谈判。义律提出,广州需缴纳600万元赎城费,弈山、杨芳所率岭北援兵必需在六日内撤至城外二百里处,如此方许在广东境内停战。次日晨,弈山应允义律的全部条件\footnote{茅海建:《天朝的崩溃:鸦片战争再研究》,页287}。至此,战事似乎已经停止了。然而,一次意外却引发了一场在南粤史上至关重要的武装冲突。

5月29日,一群英军利用休战中的闲暇前往越秀山附近的村庄闲逛。他们来到三元里东华村,对村民墓葬产生了浓厚兴趣,竟打开几口棺材查看里面的尸体。一些印度兵甚至还冲进一户村民家中,下流地骚扰妇女。这些人在胡作非为了一番后回到营地。他们这种不顾南粤风俗肆意妄为的举动严重激怒了村民。在宗族士绅的组织下,村民们拿起火绳枪、长矛、刀棍、鱼叉等武器,鸣锣集合附近村庄的居民,向越秀山前进。到30日上午10时,已有超过5000名乡勇聚集在英军营地前。郭富大吃一惊,连忙备战,于下午1时派出一支分遣队进攻,将村民打退了5公里。这时,越来越多的村民从四面八方涌来,乡勇的队伍膨胀到7500人。乡勇们重整阵势,举起旗帜冒着英军的排枪向前进攻。突然间,一场倾盆大雨降临在战场上,使能见度降到只有几米。大雨淹没了田间小道,电闪雷鸣使人心惊,英军的火枪失灵了。趁此机会,乡勇英勇地直前搏战,以长矛刺杀英军,英军只得后撤。下午4时许,大部分英军都成功撤回了营地,只损失了几个掉队者。但他们惊讶地发现,有一个连队没有回来。于是,一支由两个手持防水枪的水兵连组成的救援队从营地出发,在暗夜的暴雨中寻找这个连队。过了很久,他们才终于在一块稻田中找到这个被包围的连队。该连正聚苦苦支撑,用匆匆擦干的火枪偶尔开火以驱散来攻的乡勇,已有十余人伤亡。救援队放了一阵密集的排枪,将猝不及防的乡勇打得四散而退,随后带着这个失魂落魄连队返回,于9时到达营地\footnote{魏斐德:《大门口的陌生人:1839—1861年间华南的社会动乱》,页39}。在这场被称为“三元里之战”的战斗中,英军付出了7死42伤的代价\footnote{魏斐德称此战英军1死15伤,显然严重低估英军伤亡,见魏斐德:《大门口的陌生人:1839—1861年间华南的社会动乱》,页17。关于三元里之战中英军伤亡的详尽分析,参见茅海建:《天朝的崩溃:鸦片战争再研究》,页126}。

次日,来自附近八九十个村庄的武装村民纷纷赶到,乡勇的队伍膨胀到2万人。他们逼近四方炮台,在几座山丘下布下庞大的阵势。这时,英军已布下严密的防线。若乡勇贸然进攻,必然蒙受可怕的屠杀。郭富不欲再开杀戒,遂致信广州知府余保纯,威胁说如果敌对行动再进行下去,他将攻打广州市区。不多时,余保纯及南海、番禺的知县来到英军营地,与一名英军上尉一起下山进入乡勇的队列。余保纯对领导乡勇的士绅说,与英人的和约已经签订,英人不会再继续进攻了。听闻此言,乡勇遂纷纷散去,一场大屠杀得以避免。同日,广州的600万元赎城费全部交齐。6月1日,弈山、杨芳率残兵败将垂头丧气地退出广州城,移驻城北之金山寺。英军亦登上军舰,全体撤至香港\footnote{茅海建:《天朝的崩溃:鸦片战争再研究》,页287}。鸦片战争在南粤境内的战事至此结束。

三元里之战是南粤乡民一场守卫乡土的义举。纵然大英帝国十分伟大,但英兵不顾南粤风俗、侮辱妇女的行为着实有错,乡勇奋起抵抗完全正义。然而,日后的大一统主义者却将三元里之战发明为“中华民族抵抗外来侵略者的壮举”,无疑是极度无耻的摘桃子行为。事实上,和英军比起来,防守广州城的清军远更残暴。在这场战斗中,数千南粤乡民在一天时间内给英军造成的伤亡,居然达到2.5万清军在五天内给英军造成伤亡的一半以上,并成功击退英军,我南粤人为乡土而战时所迸发出的强大战斗力于此可见一斑。三元里之战是南粤与大英这两个伟大文明在互不了解的情况下进行的一场较量,一如日本历史上的萨英战争。三元里的光荣属于南粤,而绝不属于大一统帝国。

此后,英清鸦片战争又持续了一年多。1842年8月29日,清英双方代表在英舰“皋华丽”号上签订《南京条约》,战争以清帝国的失败告终。据该条约规定,清帝国不但要对英赔偿2100万银元,还要开放广州、厦门、福州、宁波、上海五个通商口岸并割让香港岛。此外,英国进出口货物之税款需由英清共同商定。1843年,英清双方又签订做为《南京条约》附件的《虎门条约》,英国获得了领事裁判权和片面最惠国待遇。其后,美、法两国接踵而至。1844年,美清在澳门望厦村签订《望厦条约》,美国亦获得协定关税、领事裁判权、片面最惠国待遇等权利。同年,法清签订《黄埔条约》,法国不但获得与美国相同的权利,还获准在五个通商口岸修建教堂、坟地\footnote{茅海建:《天朝的崩溃:鸦片战争再研究》,页482—545}。1849年8月,葡人强行关闭清帝国设在澳门的海关,并发兵攻击拱北关闸,从此拒交地租银,澳门在实质上完全成为葡萄牙的土地\footnote{黄鸿钊:《澳门史》}。至此,持续近百年的广州“一口通商”制度被打破,西方势力开始大举进入东亚大陆。南粤得以摆脱清帝国的巨大束缚,以更快的速度融入世界大潮。经过这场战争,南粤的对外开放路径锁定了。

鸦片战争不但锁定了南粤的外交路径,也深刻改变了南粤社会。在这场战争中,大批南粤乡勇被动员起来,在宗族、乡村共同体的凝聚下成为一支异常强大的武装力量,珠三角一带的基层社会在实质上落入土豪之手。战争结束后,清帝国无力完全解散他们,只能承认此种社会格局。在此后十余年间,南粤土豪、清帝国、西方人将展开复杂的三角冲突与博弈,在惊涛骇浪中进一步塑造南粤的宪制。

\section{土豪自治的巩固:洪兵战争}

\indent 1842年,鸦片战争结束。然而,这不过预示着南粤境内更大规模冲突和战争的开始。在《南京条约》英文本的第二条中,写有允许英国人携带家眷居住于五个通商口岸的“Cities and Towns(城市与乡镇)”的字样。然而在汉文本中,Cities and Towns却被译为“港口”。如此含糊不清的表述为英清之间的新冲突埋下了祸根,而新崛起的南粤土豪则做为第三方势力加入此种斗争,使形势变得十分复杂。此后数年内,三方围绕由此种文本差异引发的矛盾展开了一系列复杂的斗争。

在英人看来,Cites and Towns无疑包括广州城墙内的区域。然而,南粤士绅显然不这么看,他们认为英国人必须像以前那样呆在商馆。三元里之战后,珠三角士绅们的仇英情绪高涨。1842年11月,他们刊刻了一通名为《全粤义士义民公檄》的告示,声称绝不允许“犬羊成性”、“狼心兽面”的英人入城\footnote{王金铻,邢康主编:《爱国主义教育辞典》,页220}。1843年7月,清钦差大臣耆英宣布广州城门将向英人开放。广州士绅立即致书耆英,称假如英人入城,他们将发动“团练义民十余万众”群起抵抗。耆英只得告知英国公使璞鼎查(Henry Pottinger),称由于众怒难犯,英人不能立即入城,此事遂暂时搁置下来\footnote{魏斐德:《大门口的陌生人:1839—1861年间华南的社会动乱》,页85—86}。1844年,璞鼎查退休,态度强硬的德庇时(John Davis)代之。在德庇时看来,英军没有在三元里之战中痛下杀手乃一巨大错误。1846年初,德庇时将时已就任两广总督的耆英叫到香港,当面威胁称如果清帝国再不允许英人进入广州城,英军便会动武。在此压力下,耆英只得于1月13日与广东巡抚黄恩彤联名发布告示,宣布即将允许英人入城。士绅们迅速迎战,于次日在广州的大街小巷贴满了告示,宣城只要英人胆敢入城就格杀勿论。15日,大骚乱爆发。是日,因听说广州知府刘浔正受耆英委派与英人秘密谈判,几千名仇英情绪高涨的市民展开暴动,揪住刘浔的发辫将他暴打了一顿,并涌入衙门放火烧掉他的官服。16日晨,面对群情激愤的暴民,广州城内被吓呆的清帝国官员们纷纷表示支持“忠义百姓”,要求将“英夷”挡在城外。耆英和黄恩彤别无选择,只得服从全城官民,照会德庇时称英人不可入城\footnote{魏斐德:《大门口的陌生人:1839—1861年间华南的社会动乱》,页89—91}。

7月8日,一群英国商人在商馆附近和一批广州小贩发生争吵,双方展开搏斗,有三名粤人死在了英人的枪下。闻知此事的士绅们又一次群情激愤,四处散发反英传单。千钧一发之际,德庇时要求肇事英商交出200银元的罚金交给耆英,此事遂告平息。不久后,有六名外出游览的英人在佛山附近遭到村民袭击,所幸都保住了性命。德庇时对此大为光火,要求耆英惩办肇事者,耆英的态度则是置之不理。被激怒的英人决定发动一次军事进攻。1847年4月1日,英军突袭并攻占了虎门炮台,一支部队更在商馆登陆。耆英惊慌失措,只得对三名肇事者施以鞭笞之刑,并向德庇时承诺将在1849年4月6日开放广州城门\footnote{魏斐德:《大门口的陌生人:1839—1861年间华南的社会动乱》,页100}。在士绅们看来,南粤又一次被清帝国出卖了,珠三角的反英情绪随之更加高涨。同年12月6日,六名外出打猎的英国人在南海县的小村庄黄竹岐被愤怒的乡民杀死。德庇时再次暴怒,称若耆英不惩办凶手,他将把商馆里的所有英国人撤走,并随时准备开战。无奈的耆英唯有再次屈服,派出驻防八旗军占领黄竹岐,将四名肇事者当场斩首,并对另外十一人处以不同程度的刑罚\footnote{魏斐德:《大门口的陌生人:1839—1861年间华南的社会动乱》,页101—102}。

1848年初,身在北京的道光帝听说耆英已将黄竹岐事件的肇事者处决,表示十分不满。他认为,既然南粤人有严重的反洋情绪,那清帝国正可将之用来反英。耆英的官运自此到头了。2月3日,他和黄恩彤被撤职,代之以对西方人态度强硬的徐广缙和叶名琛。3月16日,态度温和的文翰(George Bonham)取代德庇时出任英国公使,这一人事变动预示着英人将要面临挫折。次年2月,由于已临近耆英与德庇时约定的英人入城时间,文翰乘战舰到达虎门,要求与徐广缙面议此事。老奸巨猾的徐广缙先“密召诸乡团练,先后至者逾十万人”,在广州城中制造浩大的反英声势,随后于17日乘舟赴虎门登上英舰,携翻译进入文翰的私人舱房展开谈判。寒暄过后,当文翰提出入城问题时,徐广缙却开始滔滔不绝地称允许英人入城只是耆英做出的承诺,与他本人无关,但他可以为此事向朝廷“请旨”。不欢而散后,徐广缙回到广州,编造了文翰试图扣留他,但他大义凛然地扬长而去的故事。与此同时,他又与叶名琛“上奏”道光帝,称广州城中已有十万乡勇随时准备对英开战。假如朝廷执意要让英人入城,那么乡勇很可能哗变,造成“内外交讧”之局。4月1日,在朝廷的批复尚未送达广州时,徐广缙做出了一个冒险的举动:他对文翰伪造了一份“圣旨”,称朝廷绝不会违背民意允许英人入城。温和的文翰不待深究便软化下来。在虚张声势地对徐广缙做出一番警告后,他便再不提及入城之事。4月29日,道光帝的批复抵达广州,指出“进城一事实属万不可行”,这一态度正与徐广缙伪造的“圣旨”一致\footnote{魏斐德:《大门口的陌生人:1839—1861年间华南的社会动乱》,页121—122}。徐广缙赌赢了,他用卑鄙的欺骗和小聪明暂时阻止了英人入城。直到第二次鸦片战争爆发前,英人都未再提及入城之事。

就在英人入城之事搁置下来时,一场巨型风暴正在南粤大地酝酿,那便是著名的太平天国之乱。日后的太平天国天王洪秀全本为广州花县的客家农家子(关于客家人入粤,详见下一节),于1812年出生。他幼丧父母,在村塾中以教书为生,读过一些经史著作。然而,他却三次科场失利,连秀才都考不中,因而产生了反社会人格。1836年,他翻阅到在广州应试时得自基督徒梁发(关于梁发,参见第十七章)的传教小册子《劝世良言》一书,将书中所描述的场景与自己生病时的幻觉相对比,此后便认为自己是上帝之子、耶稣之弟,已受到上帝派遣到人间斩杀化身为满洲皇帝的“阎罗妖”。他遂回到家乡创立“拜上帝教”,说动好友冯云山一起砸烂村中孔子牌位,两人于1844年一同前往广西桂平、武宣交界山区中传教。不久后,洪秀全回到广东,留下冯云山在当地继续传教。1847年,洪秀全在广州东石角教堂学习了数个月,但传教士因他对教义认识不够拒绝为他受洗,他遂再赴桂平。这时,冯云山已在当地紫荆山客家大姓的帮助下发展了数千名信徒。由于当地的广府、客家两族矛盾颇深,大批客家人争相入教,四处砸毁主要由广府人的庙宇,与广府人为主的团练为敌。1847年底,冯云山一度被土豪团练逮捕,洪秀全心灰意冷地回到广东。1848年,在群龙无首的情况下,信徒杨秀清、萧朝贵分别以假扮被上身的方式伪托“天父(上帝)”、“天兄下凡传言”,使教中人心得定。不久后,教徒通过行贿将冯云山救出。在冯云山的邀请下,洪秀全于1849年7月回到紫荆山。这时,桂东北一带爆发灾荒,流离失所的灾民纷纷入教,使拜上帝教教众之数过万,其中近半为客家人。进入1850年后,拜上帝会与团练屡次冲突,势力愈加壮大。洪秀全乃于1851年1月11日在桂平金田村率两万人誓师,竖起反清大旗。3月23日,洪秀全于武宣“登基”称“太平王”,后改称“天王”。秋,太平军陷永安(今广西蒙山),于12月在永安分封杨秀清、萧朝贵、冯云山、韦昌辉、石达开为“东王”、“西王”、“南王”、“北王”、“翼王”。次年4月,太平军围攻桂林失利,转陷全州,屠戮百姓数千,随后北上,于蓑衣渡遭遇清知府江忠源所率“楚勇”(湘人团练)阻击,冯云山阵亡。经此挫折,太平军转变方向,于5月19日离开广西进入湖南境内,陷道州、郴州,裹挟五万余湘南百姓。至此,太平军已完全脱离南粤本土,成为一支活跃于岭北的流寇大军。此后,他们于1853年攻下南京,改其名为“天京”,与清帝国分庭抗礼,并与清帝国一同对吴越、江淮、荆楚百姓犯下了滔天罪行,造成高达数千万的大洪水式人口损失。直到1864年清军攻下天京,ni 场浩劫方告一段落\footnote{关于太平天国之研究可谓汗牛充栋,此处所述通史知识可参看罗尔纲:《太平天国全史》}。如果说太平军初起时在南粤境内时的反清活动尚有值得肯定之处,那么与南粤土豪为敌的他们在逃出南粤之后,便不再对南粤有什么牵挂,岭北人在他们的队伍中占了多数。此后,他们所思所想的唯有在岭北与清帝国交战,丝毫没有为南粤而战的打算。我们可以认为,太平军在离开南粤那一刻起便已彻底叛离了南粤,成为一支劫掠东亚大陆各邦的流寇,化为东亚各邦的凶恶敌人。做为被南粤社会排挤出去的渣滓,他们绝不能代表南粤。

不过,在拜上帝教信徒中却有一批与太平军迥然相异者,他们便是凌十八率领的反清起义军。在洪秀全金田誓师,开始北上作战后,这批英雄选择留下为南粤的自由而战。凌十八本名凌才锦,其家族为客家人,在粤西高州信宜农村以种植染料为生,兼营茶铺,生活富裕、人丁兴旺。因他在家族的堂兄弟中排行十八,故人称“凌十八”。凌十八生于1819年,自幼博览群书。成年后,他担任塾师,写得一手好字,是个光明磊落、富有正义感的读书人。1849年,桂东北的灾荒侵袭至粤西,当地百姓不但承受着饥饿的折磨,还要被清帝国官吏疯狂盘剥。为拯救一方父老,凌十八投笔从戎,前往广西金田加入拜上帝教,随后受命回乡发展信众。1850年7月,他聚集了上千百姓拜旗起义,日夜打造军器,使清帝国官僚为之震恐。信宜知县官步宵纠集起两千余名乡勇围攻起义军据点大寮,被打得大败亏输,官步宵借口生病狼狈逃回县城。9月,高州知府胡美彦见起义军势大,遂改“剿”为“抚”。凌十八乃佯装“归降”,在保留部众的情况下继续招兵买马。1851年2月14日,在听闻洪秀全已于金田誓师起兵后,凌十八率两三千部众西进入桂,欲与洪秀全会合。但因双方联络不畅,起义军未能找到洪秀全所在位置,便转攻广西陆川、博白,进而包围西江上游连接两广的重镇玉林。清帝国忙调集正规军、乡勇8000余人解围。经月余激战,起义军损失2000余人,凌十八只得率残部退往广东化州一带。经过休整,到春夏之交,起义军人数增至四五千人,乃于8月7日攻克罗定县之城镇罗境圩。起义军在当地不屠不掠,开征轻税,与清帝国形成鲜明对比,得到百姓的热情支持。在百姓的帮助下,义军构筑了包括大量炮台、掩体、壕沟、陷坑、长达二十公里的防御工事,布置上千挺火器,将罗境圩建成一座巨大的要塞。清帝国侵略军多次进攻,都被打得抱头鼠窜。气急败坏之下,徐广缙只得亲自出马,下令侵略军构筑长三千七百余丈、宽丈余的长壕,妄图以围困战术困死起义军\footnote{《广东军事人物志》,页71—73}。

在长期围困下,凌十八一直保持着高昂的斗志,不断激励起义军民们奋斗下去。然而,粮食在一天天地减少,坐困孤城终非长久之策。11月25日,凌十八亲率义军进行了一次突围,但被严阵以待的清军击退,伤亡一千多人。1852年5月22日,徐广缙奉命前往广西讨伐太平军,叶名琛则往罗定“围剿”凌十八起义军。这时,义军将士已无粮可吃,只能靠树皮充饥,许多人都饿得骨瘦如柴,但依然保持着旺盛的斗志。7月28日,叶名琛指挥侵略军对罗境圩展开总攻,包括凌十八在内有能战之力的1100多名义军将士全部搏斗到最后一刻,在歼灭300多名侵略者后壮烈战死在阵地上。凶残的叶名琛随之下令将因饥饿倒在镇内的800余名义军活活烧死,只留下268名俘虏。此役,义军的视死如归令叶名琛大为震惊。在事后给清廷的“奏报”中,他这样说:

\begin{quote}

“逆党”之坚忍信从,实为历来各“匪徒”所罕见。

\end{quote}

凌十八悲壮地牺牲了,而南粤大地上的杀戮还远未停止。1853年春,一场惊天动地的大战在珠三角爆发,那便是由天地会发动的“洪兵之乱”。欲明白这场战争,便需首先了解天地会这一组织。天地会又称“三合会”或“洪门”,是一个遍布粤、闽、南洋、湘、赣的秘密反清组织。据天地会自己的说法,该会起自闽越福州莆田县九连山中的少林寺。据说,该寺僧人武艺高强,颇通军略,曾在康熙或乾隆时派出僧人128名助清帝征服“西鲁国”。然而,清帝却因忌惮于他们的强大战斗力而卸磨杀驴,遣人趁夜纵火焚寺,仅有18名寺僧逃离。这些幸存者在逃到黄泉村时遭遇清兵,战死13人,仅剩的五人逃至一处高山上的庙宇中,在附近溪水中发现了一个闪着红光、带有“反清复明”字样的香炉。这时,他们遇到五名陌生勇士的帮助,十人歃血为盟,举起反清义旗。后来,他们在与清军的战斗中失利,遂分为五大房遍布清帝国各地,继续与清帝国斗争\footnote{洪门起源之传说非常丰富、精彩,本书不拟详述。关于详细的洪门起源传说,参见平山周:《中国秘密社会史》,页13—21}。

这一故事的编造、拼凑痕迹十分明显。比如,清帝国境内确实有一座以习武著称的少林寺,但事实上该寺在河南;再比如,历史上从未有过一个叫“西鲁国”的国家曾与清帝国为敌。无论该传说的真实性如何,反满都是它的核心之一。这表明尽管经历了18世纪的百年承平,但17世纪时清帝国侵略南粤时犯下的滔天罪行从未被粤人遗忘。依托反满这面旗帜,天地会可以吸收众多会众。

天地会还有另一个颇为引人注目的主张,那便是其描绘出的社会蓝图。自16世纪南粤社会华夏化后,粤人即骄傲地视自己为华夏文明的真正传人,鄙视一切北方蛮族。在天地会的主张中,一切苦难都是满洲胡虏造成的。只要消灭清朝、恢复明朝,便能重建美好而纯正的华夏文明,创立一个尧舜复出的“公平正直之世”。这一主张对于高度认同华夏的粤人本就很有亲和力,其强烈的乌托邦色彩更能吸引大批对社会不满者入会。16世纪以来,南粤社会中虽然遍布着以儒家血缘理论为纽带的宗族共同体,但在各宗族之间、各宗族内部仍有严重的贫富分化,许多小族、贫民、佃户都对大族、富人怀有强烈的阶级仇恨。这样,加入天地会便为他们改变社会地位提供了一种可能。天地会具有一套异常隐秘而繁琐的入会仪式,互不相识的天地会会众则需互对暗号方能相认。此套暗号自成系统、异常复杂,见于下:

\begin{quote}

于道上试人是否会员,则叩以“汝为瞎子否?”其人如答言“我非瞎子,我目较汝为大”,即为会员之符征。
若欲于饮茶时试之,则以右手之拇指置茶碗缘,第二指置茶碗底,执茶碗以献,左手之拇指与第二指屈曲,余三指伸出,置于右手之侧。若其人为会员者,必以同法受之。
凡贡献饮食物三种时,必取其居中之一物,谓之忠臣。
伸右手,令拇指与前指屈曲,余三指伸直,左手亦然。唯以伸直之三指按胸前,此即所以表天。如伸右手,令拇指与第一、第二指伸直,他二指屈曲,而以左手之拇指与第一、第二指伸直,按胸上,即所以表地。若伸右手,令拇指与小指伸直,余三指屈曲,左手亦然,以置胸上,即所以表人。此表人者,谓之龙头凤尾。三法连演,即所以表明为三合会员。

\end{quote}

天地会起事之时,会众之家人会收到特别保护:

\begin{quote}

三合会起事以后,有保护家族之法。凡会员之家,门上必贴方形之红巾,外面做“洪”字,里面书“英”字,室内四隅,必竖立三尺六寸长之绿竹。若是者,即为会员家之特征\footnote{以上两段引文,引自平山周:《中国秘密社会史》,页66—67}。

\end{quote}

由以上描述可见,天地会具有极强的秘密性,其会众身份不向社会、甚至不向同会之人随意公开。若有人加入天地会,那么其家人也会受到组织关照。对于阶级地位较低下、生活较穷困者来说,天地会能给他们提供许多宗族共同体无法提供的好处。在19世纪,南粤社会事实上并行着两种秩序,一为由儒化士绅土豪主导、以血缘为纽带的宗族秩序,一为由小族、佃户、贫民主导、以秘密会社为纽带的地下秩序。在历史节点中,这两种难以兼容、由不同阶级主导的秩序必将进行你死我活的冲突,决定究竟由何者来主导南粤此后的历史进程。

1852年夏,严重的洪涝灾害蹂躏了花县和广州间的乡村地带。同年秋,徐广缙离任,叶名琛升任两广总督,旗人柏贵接任广东巡抚。叶名琛和柏贵对受洪灾折磨的百姓毫无赈济之举,反而加紧征收赋税,导致社会矛盾急剧激化。1853年初,大批佃农和无业游民开始涌入天地会。至6月,天地会众开始在珠三角乡村中公开进行抢劫、绑架活动,东莞、新会、顺德、香山郊外的大部分地区遍布他们的身影。农村对广州城的粮食供应已受到严重影响,恐慌在市民中蔓延。10月,成群的天地会众出现在广州郊区,他们的劫掠行为进一步加剧了广州人的恐慌。11月,惠州境内也出现了大批公开活动的天地会众,叶名琛发现他已联系不上惠州府城了\footnote{魏斐德:《大门口的陌生人:1839—1861年间华南的社会动乱》,页162—164}。1854年初,叶名琛开始着手镇压天地会。他派出一支军队前往东莞,对当地农村百姓施以不分老幼妇孺的残酷屠杀。清帝国侵略军的暴行使更多人充满仇恨地加入天地会。五月十五日,东莞天地会首领何六在石龙镇竖旗,宣布公开反清,于二十二日攻占县城。六月十一日,佛山天地会首领陈开聚会众7000人起兵于石湾镇附近的大雾冈,随即占领佛山,建国号“大宁”。在广州北郊的佛岭,有自称“统领水陆兵马兼事粮饷大元帅”的李文茂起兵响应。在东郊的燕塘,有陈显良起兵响应;在河南(海珠岛),则有林洸隆起兵。六月十九日,花县天地会首领甘先起兵于远龙圩,于次日攻占县城。此外,大批珠江疍民亦追随天地会起兵反清。天地会众皆头戴红巾或腰缠红带,自称“洪兵”,又称“红兵”\footnote{宋其蕤、冯粤松:《广州军事史》下,页33}。惨烈的洪兵战争,就这样突然爆发了。

疍民是南粤社会中的一个特殊族群。广东沿海遍布港湾,珠江、韩江流域河道纵横,多有居民以船为居。他们熟悉水性,靠捕鱼和为商人担任水手为生,亦会从事水上贸易活动。在南粤大规模华夏化以前,他们与陆上居民并无太大的冲突。16世纪,随着南粤精英开始建构南粤小华夏,许多水上居民弃舟登岸,将自己打扮成儒化宗族积极开发沙田,获得大量财富,从而变为陆上居民。那些未能成功建构宗族的人则继续留在水上,日渐受到陆上儒化宗族的歧视。由于陆上土地已被儒化宗族占领,他们难以在岸上立足。自17世纪中期起,陆上居民更不准他们上岸,并发明了一套历史叙事以证明疍民的“非华夏”出身:

\begin{quote}

自秦始皇发诸尝逋亡人、赘婿、贾人略取扬越,以谪徙民与越杂处。又适治狱吏不直者,筑南方越地。又以一军处番禺之都,一军戍台山之塞,而任嚣、尉佗所将率楼船士十余万,其后皆家于越,生长子孙,故嚣谓佗曰:“颇有中国人相辅。”今粤人大抵皆中国种,自秦汉以来,日滋月盛,不失中州清淑之气。其真鄼发文身越人,则今之徭、僮、平鬃、狼、黎、岐、蛋(疍)诸族是也\footnote{屈大均:《广东新语》卷7,页233}。

\end{quote}

此种叙事模式下,陆上居民被打造成南下秦兵与岭北殖民者的后代,瑶、僮、疍等未儒化成功的群体则被说成是百越后代。这一叙事无疑与真实历史图景相差甚远,因为在16、17世纪以前陆上人和水上人的分野远没有后来那么明显。分子人类学研究的结果显示,当代南粤四成男性的父系Y染色体来自百越祖先,母系mtDNA中的百越血统更占八成。可以说,当代南粤人的大部分祖先都是百越人,水上人和陆上人的大部分血统皆来自百越\footnote{徐杰舜、李辉:《岭南民族源流史》,页460}。因此,此种将陆上人定义为华夏后裔,将疍民定义为非华夏后裔的说法只是陆上儒化精英为排斥水上人而发明出来的。事实上,疍民和陆上人的对立并非族群对立,而是一种阶级对立。

对饱受陆上宗族歧视的疍民来说,加入天地会是他们改变阶级地位的大好机会。洪兵蜂起后,遂有大批疍民争相投靠,使洪兵得到了强大的水军。疍民迅速控制了广州内河(广州城与海珠岛间之珠江江面),迫使叶名琛惊恐地下令关闭广州城门。当时,广州城中的八旗、绿营兵仅有5300余人,兵力极其单薄\footnote{宋其蕤、冯粤松:《广州军事史》下,页33}。值此关头,南粤土豪团练的向背遂成为左右局势发展的最关键因素。

洪兵的队伍中包含大批啸聚而来的无业游民。他们平时偷鸡摸狗,战时就化为打家劫舍的凶恶强盗。在洪兵占领区,他们没有建起任何税收体系,只满足于劫掠勒索富户,称之“打单”。若有富户抗拒,便会惨遭他们杀害\footnote{蒋祖缘、方志钦:《简明广东史》,页466}。洪兵虽然反清,但他们要摧毁的是南粤土豪自治的社会秩序。因此,珠三角土豪们纷纷组织团练乡勇抵抗洪兵,和清帝国结成了机会主义的联盟,正如15世纪时面对黄萧养流寇的威胁与明帝国结盟的南粤土豪。短短时间内,叶名琛就从广州、东莞、新安、香山、新会、潮州等地招募到27000名乡勇,其中有16000人被调至广州参与守城\footnote{宋其蕤、冯粤松:《广州军事史》下,页34}。至此,南粤土豪成为抗击洪兵的主力。

六月下旬,洪兵陈开、甘先、陈显良、林洸隆部从西、北、东、南四个方向大举包围广州城,疍民水军则在珠江江面游弋。围城洪兵总兵力高达20万,然互不统属,多为只知劫掠的乌合之众。二十六日,怒海狂涛般的洪兵开始从西、北、东三个方向大举攻城,叶名琛居镇海楼指挥守军极力防御,勉强将之击退。叶名琛随后派兵五千出城向北反攻,在牛栏冈中伏,败回城中\footnote{宋其蕤、冯粤松:《广州军事史》下,页34}。洪兵见强攻难以得手,便采取围困战术,漫长的围城战开始了。此后半年内,洪兵不断对广州城墙发动冲击,但一次次被人数不多的守军击退。以团练为主的守军虽在人数上占绝对劣势,但以守卫乡土为己任他们却拥有远高于洪兵的斗志。长期对峙有利于乡勇,但绝不利于洪兵。很快,由于广州近郊的村庄都已被抢光、烧光,围城洪兵陷入无饷可筹的窘境,许多人退出了队伍,还有些人转向更富庶的地区继续“打单”,甚至有6000人向清军投降。留在城墙下的各部则因分赃不匀互相仇视,时常火并\footnote{魏斐德:《大门口的陌生人:1839—1861年间华南的社会动乱》,页170}。闰七月十五日,清绿营参将卫邦佐(东莞人)率兵出城突袭燕塘,一举击溃洪兵陈显良部。陈显良率残部逃至黄埔之新造,广州城东面的威胁解除。在珠三角各地,土豪们纷纷组织团练攻击洪兵。在广州、佛山之间的大沥堡,有举人欧阳泉、麦佩金集合附近士绅立团练局,召集乡勇抵抗洪兵。在广州北郊,有士绅组织的“北路平定会”与洪兵为敌。在黄埔,则有大族吴氏组织乡勇同新造的洪兵交手。在东莞,退到乡村的知县华廷杰于1854年底率团练重夺县城,何六向北败逃\footnote{魏斐德:《大门口的陌生人:1839—1861年间华南的社会动乱》,页171}。至此,洪兵大势已去。1855年一月上旬,洪兵李文茂、甘先部在广州守军和城外团练的夹攻下溃败,何六、甘先率败兵转攻韶州,失利后窜入湖湘,李文茂则退往南海县之九江,广州北面的威胁宣告解除\footnote{宋其蕤、冯粤松:《广州军事史》下,页35}。而在此之前,广州西面的陈开部已自行撤退。

在各路洪兵中,陈开部战斗力最强。据有南粤第二大城市佛山的他们不但拥有充足的饷源,更雇佣了一批美国、荷兰水手为他们制造火器和枪弹。1854年九月二十七日,叶名琛派出一支1500人的正规军乘船从水路西进,意图突袭佛山。这支清军在佛山附近成功登陆,但他们留在岸边的26艘战舰、帆船却被偷偷尾随他们的洪兵俘获。洪兵立即将舰炮对准岸上清军猛烈开火。销烟过后,清军全军覆没,有1200人战死、300人被俘\footnote{《广东通史》近代上册,页254}。

至1854年末,昔日繁华的佛山已被洪兵压榨得犹如被挤干的橘子,难以继续支付洪兵的开销。十一月三日,陈开部将和尚能要求佛山商人“助饷”,被忍无可忍的商人们囚禁。商人们随即发动市民在街头构筑木栅,如1449年时的先辈一样展开了勇敢的抵抗。洪兵难以取胜,便放火焚烧,城中燃起大火。这时,卑鄙狡诈的和尚能诡称只要商人们释放他,他便会率领部下灭火。天真的商人们信以为真,竟放虎归山。获释的和尚能恩将仇报,立即率兵继续纵火,并对佛山百姓展开疯狂屠戮。在凶残的屠杀中,佛山人的反抗惨遭镇压,数以万计的平民惨死于洪兵屠刀下,全城有49条街道遭焚,三分之一的城区被毁。经此浩劫,佛山再也无力支撑陈开部的开销了。十一月十二日,陈开率部撤离满目疮痍的佛山,向西退去\footnote{《广东通史》近代上册,页255}。十二月一日,大沥团练和清军开入佛山,所见者皆是仍在闷烧的废墟\footnote{魏斐德:《大门口的陌生人:1839—1861年间华南的社会动乱》,页177}。陈开西撤后与李文茂部会合,一同退至肇庆。1855年四月六日,因清军自西江上、下游两面攻来,陈开、李文茂放弃肇庆,转攻广西。四月二十一日,洪兵包围浔州城。经近四个月的围攻,浔州于八月十七日城破,清知府刘体舒被杀。占领浔州后,陈开将之改名“秀京”,建立“大成国”,改元“洪德”,并蓄发易服、开铸货币“洪德通宝”。此后,他们屡次出兵攻击桂林、柳州等地。1862年,清军攻破秀京,大成国残部逃亡广西大容山中,于1866年被彻底消灭\footnote{冯先知:《中国历代重大战争详解 近代战争史》下册,页275—277}。至此,广东境内的洪兵战争告一段落。随着天地会在与团练的斗争中惨败,能够威胁到南粤土豪的秘密会社受遭受重创。南粤土豪在这场阶级斗争中获胜,土豪自治的社会秩序在南粤得到了空前的巩固。

将洪兵逐出广东后,残忍至极的叶名琛发动了一场超大规模的残酷屠杀,刚刚经历洪兵残酷杀掠的广东百姓因此又遭到更为惨痛的劫难。根据叶名琛的命令,不但天地会众和无业游民要被处死,许多曾被迫向洪兵交过保护费的村庄也难逃一劫。据一名目击此次屠杀的英国人描述,参与屠杀者除清帝国侵略军外,还有许多被强迫的平民:

\begin{quote}

在很多情况下,围捕叛匪的任务是强加在居民头上的。因为如果不服从命令,就要毁掉他们的村子\footnote{转引自魏斐德:《大门口的陌生人:1839—1861年间华南的社会动乱》,页178}。

\end{quote}
	
在叶名琛的淫威下,各地士绅、团练不得不每日和清兵一同搜捕所谓的“从逆者”,否则他们自己的村庄就会遭到屠戮。每天由各地被解送至广州者多达七八百人,他们全部被杀。1855年夏,刚从美国耶鲁大学留学归粤的香山人容闳(关于容闳,详见第十六章)寓居广州,目击了这场令人发指的暴行:

\begin{quote}
统计是夏所杀,凡七万五千余人。以予所知,其中强半皆无辜冤死。予寓去刑场才半英里。一日,余忽发奇想,思赴刑场一觇其异。至则但见场中流血成渠,道旁无首之尸纵横遍地。盖以杀戮过重,不及掩埋……此累累之陈尸,最新暴露者亦已两三日。地上之土,吸血既饱,皆作赭色。余血盈科而登,汇为污地。空气中毒菌之弥漫,殆不可以言语形容……此种情形,非独当时观者酸鼻。至今言之,犹令人欲作三日呕。人或告之,是被杀者有与暴动毫无关系,徒以一般虎狼胥吏,敲诈不遂,遂任意诬陷置之死地云。似此不分良莠之屠戮,不独今世纪中无事可与比拟,即古昔尼罗(Nero)王之残暴,及法国革命时代之惨剧,杀人亦无如是之多。罪魁祸首,惟两广总督叶名琛一人实尸其咎\footnote{容闳:《容闳自述》,页33}。

\end{quote}

由容闳的叙述可知,在1855年夏,仅广州城内就处决了7.5万人,其中多有遭诬陷的无辜者。事实上,全广东的遇难者总数要远远超过7.5万。时在广东的英国人施嘉士(Scarth John)如是说:

\begin{quote}
在广州,六个月内被处死者七万。在肇庆,其数尤多。在其他处,被斩者整千整百计。被投入江中溺死者,整百整千计——皆以十余人绑在一起,一批一批地投下。我曾亲见其腐烂的尸骸集体漂流于江面——其中有不少妇女。许多人是生前被割碎而死的。我曾亲见此惨状——无数无肢体的,无首级的尸骸,满布于刑场。据估计:自乱事起后,在一年之内,全省被杀的人民在百万以上,其中在广州处死者逾十万人\footnote{转引自《广东通史》近代上册,页264—265}。

\end{quote}

叶名琛究竟在1855年屠杀了多少粤人?根据革命党在20世纪初的宣传,“广东生灵之伤于官军手者,百余万人”\footnote{平山周:《中国秘密社会史》,页31}。这一数字与施嘉士的估计吻合,应是较为夸大的。据叶名琛向清廷的“奏报”,他一共处决了4.7万名“叛匪”\footnote{魏斐德:《大门口的陌生人:1839—1861年间华南的社会动乱》,页179}。这当然远远小于实数,应是他对自己滥杀无辜的掩饰。真实的杀人数字很可能只有他自己知道。1858年,叶名琛在第二次鸦片战争中被英人俘至印度加尔各答。当英人询问他在1855年究竟杀了多少人时,他的反应是:

\begin{quote}
他狞笑一声说,他曾杀了十万个男女。他又自炫云,如果连他下令毁灭的村镇包括计算,为四倍前数哩……而他犹以未能铲绝全部根苗为憾\footnote{}。

\end{quote}
由此可知,在血腥的1855年,倒在叶名琛屠刀下的粤人达40万之众,说他是南粤史上自尚可喜之后最残忍的屠夫是不为过的。这是继清初大屠杀后清帝国对南粤犯下的又一起滔天罪行,必将被一代代南粤人永世铭记。在进行了如此残酷的大屠杀后,叶名琛的末日马上就要到了。1856年,第二次鸦片战争爆发,大屠夫叶名琛即将迎来与其德性相匹配的下场。

\section{对外合作的启动:第二次鸦片战争}

\indent 自鸦片战争后,清帝国即一直阻挠《南京条约》的正常履行,禁止英国人进入广州城。对英人来说,《南京条约》仅使清帝国开放了五个口岸,并未全境开放通商。因此,英人对《南京条约》造成的结果深为不满,迫切希望修约。据1844年清美《望厦条约》规定,“所有贸易及海面各款恐不无稍有变通之处,应候十二年后,两国派员公平酌办”。又据清英《虎门条约》、清法《黄埔条约》规定,英、法两国在清享有片面最惠国待遇,故可均沾美国在清享有的一切利益。因此,英、法、美三国于1854对清帝国提出修约要求,但未得到回应。当时,英、法两国正在对俄进行克里米亚战争,清帝国则忙于镇压太平天国之乱,均无暇认真交涉,修约之事便暂时不了了之。1856年,克里米亚战争以英法的胜利告终,两国遂联合美国,于当年5月照会叶名琛,再提修约之事。当时,刚刚在南粤进行了惨绝人寰的大屠杀的叶名琛正不可一世,傲慢地报之以沉默。至此,一场新战争一触即发\footnote{}。

10月8日,一场小冲突点燃了战火。是日,自厦门开往广州的南粤商船“亚罗号”泊于黄埔,突遭清广东水师强行登船,十二名水手均以“走私”罪名被扣留,船上悬挂的英国国旗则被扯落。“亚罗号”曾于1855年9月底在香港注册,领有执照,曾以英国船的身份活动了一年。当时,港英政府(关于香港城邦的建立,详见第十六章)给该船发出的执照已过期十一天,可该船仍悬挂着英国国旗。虽然该船在理论上已不再属于英国,但清帝国无理扣押船员、侮辱国旗的举动仍令英国驻广州代理领事巴夏礼(Harry Parkes)深感愤怒。他当即向叶名琛抗议,要求清方在四十八小时内放还水手、赔礼道歉并惩办肇事军官。两天后,叶名琛仅答应释放九人,被巴夏礼拒绝。10月16日,英国公使、港督包令(John Bowring)向叶名琛发出如下照会:

\begin{quote}

如不速为弥补,自饬本国水师,将和约缺陷补足\footnote{转引自茅海建:《苦命天子:咸丰皇帝弈詝》,页166}。

\end{quote}

至此,英方动武之意已十分明显,但叶名琛仍傲慢地置之不理。10月21日,巴夏礼向叶名琛发出最后通牒,要求其释放全部水手并道歉。叶名琛称他可以释放水手,但不会道歉。这样一来,双方再无谈判余地,包令下令驻港英军进攻广州。23日,西摩尔(Michael Seymour,又译“西马縻各厘”)少将率三艘英舰越过虎门进至广州郊外,攻克猎德炮台,将清方的150门火炮全部钉塞,第二次鸦片战争爆发。叶名琛一筹莫展,干脆做了缩头乌龟,下令水师后撤,禁止对英舰开火。24日,英军克凤凰岗炮台;25日,克海珠炮台、占领十三行商馆,兵临广州城下。对此,叶名琛所做的只有可笑地下令停止广州外贸,而这时西关的商馆已落入英军之手。27日,西摩尔少将照会叶名琛,令其允许西方人自由出入广州城,叶名琛仍旧置之不理,英军遂开始每隔五至七分钟炮击一次两广总督署。炮火之下,叶名琛的卫士逃散一空,但他本人却滑稽地端坐官椅,摆出一副镇定姿态,并传令广州军民“杀夷一人,赏银三十元”。这一命令没有得到任何回应。28日,英军开始猛轰广州南城墙,于当日夜轰开一道六米多宽的缺口。29日下午,百余名英军由此突入广州城,占领总督署,遍寻叶名琛而不得。原来,叶名琛已怯懦地躲进了广东巡抚衙门。英军由于兵力单薄,只得暂且退到城外。此战,英军的损失不过是3死11伤\footnote{关于此战,见茅海建:《苦命天子:咸丰皇帝弈詝》,页167;姚红军:《第二次鸦片战争清军与英、法联军伤亡人数、原因及其影响》}。

英军退走后,叶名琛连忙向咸丰帝大肆吹嘘,说他接连两次击退英军,毙伤“英夷”四百多人,甚至还打死了英军统帅“西马縻各厘”。愚昧的咸丰帝信以为真,于12月对叶名琛做出如下批示:

\begin{quote}

倘该酋(包令)因连败之后自知悔祸,来求息事,该督(叶名琛)自可设法驾驭,以泯争端;如其仍肆鸱张,亦不可迁就议和,致起要求之患……叶名琛熟悉夷情,必有驾驭之法,著即相机妥办\footnote{转引自茅海建:《苦命天子:咸丰皇帝弈詝》,页169}。
\end{quote}

1857年1月,兵力不足的英军退出商馆,不久后撤离珠江。叶名琛得意洋洋,以为他的鸵鸟政策已经奏效,便再次向咸丰帝报捷。咸丰帝闻之大喜,令叶名琛全权处理“夷务”,并表示自己不会“遥制”此事。这对沉浸在自我欺骗带来的喜悦中的君臣还不知道他们的末日就要到了。“亚罗号”事件后,态度强硬的英国首相巴麦尊决意正式对清用兵,并积极与美、法两国联络,谋求共同行动。3月20日,英国任命额尔金伯爵(Lord Elgin)高级专使负责对清开战\footnote{茅海建:《苦命天子:咸丰皇帝弈詝》,页171}。恰在此时,法国亦欲对清用兵。早在1854年,有一名为马赖(Auguste Chadelaine)的法国神父来到香港,随即转往广西最西端的西林县传教。西林地处深山,经济落后,居民多有僮人、彝人。马赖对当地百姓深感同情。在给同事的书信中,他这样说:

\begin{quote}
这里的教徒们都是处在极贫困及很低文化状态……亲爱的同工!请你常常关心我的苦寒的教徒吧!他们被抛弃在遥远的广西西林,请你想办法去支援他们吧!我十分担心这些贫苦教徒的未来\footnote{邵雍:《中国近代对外关系研究》,页53}。
\end{quote}

在闭塞落后的西林,马赖凭借一己之力发展了数以百计的教徒。他积极帮助百姓戒除鸦片,用西方医学为他们治病,并对男女教徒一视同仁,获得了很多女信众的信任。西林治安混乱,匪贼横行,马赖就积极维护当地秩序,并曾依靠西式审判程序判处一名据说犯过杀人罪的盗匪无罪。马赖的行为引起了当地清帝国政府的恐慌。虽然清帝国官吏大都是些残酷虐待百姓、双手沾满平民鲜血的恶棍,但他们十分擅长于贼喊捉贼,摆出一副正义的面孔指责马赖包庇杀人犯,并诬陷他奸淫妇女。1856年2月25日,马赖被西林知县张鸣凤下令逮捕,惨遭鞭笞之刑。四天后,受尽折磨的他惨死于站笼中,灭绝人性的刽子手更剖开他的胸膛,将他仍在跳动的心脏煎熟分食。清帝国之愚昧、野蛮,于此表现得淋漓尽致\footnote{邵雍:《中国近代对外关系研究》,页51}。

这一被称为“马神甫事件”的惨剧为法国对清开战提供了理由。1857年4月,法国皇帝拿破仑三世以葛罗男爵(J.B.L.Gros)为高级专使领兵攻清。至11月,英法联军在香港完成集结,其中包括英舰43艘、英军1万人、法舰10艘。12月12日,额尔金、葛罗照会叶名琛,要求其在十天内允许西方人进入广州城、就“亚罗”号事件和马神甫事件道歉、与英法进行修约谈判。对于这一通牒,叶名琛仍旧置之不理\footnote{茅海建:《苦命天子:咸丰皇帝弈詝》,页172}。15日,英法联军舰队驶入珠江口,仅用730名登陆水兵就控制了河南。28日,联军开始炮击总督府、广州南门、大新街、双门底大街等处,全城乱做一团。联军主力在广州城东侧登陆,攻克东固炮台。这时,叶名琛发现t 再也不能依靠团练了,因为他残酷的大屠杀已使粤人完全丧失了保护他的兴趣,土豪们纷纷拒绝入城参战,与抵抗洪兵时形成了鲜明对比。在为数一万的广州守军中,仅有3000人为乡勇。当联军的炮击开始后,大部分乡勇都逃出了北门,只有1000名东莞乡勇在坚持战斗。次日晨,5000名联军对广州城展开总攻,英军水兵、英军步兵和法军水兵、法军步兵很快便各自经北大门、小北门、东门突破城防,与守军展开巷战。到下午2时,广州城已被联军控制。次日,广州驻防八旗将军穆克德讷率残存守军在西北城墙上打出白旗,广州城正式落入英法联军之手。此次作战,联军伤亡仅为10死101伤,其中英军8死71伤、法军2死30伤\footnote{姚红军:《第二次鸦片战争清军与英、法联军伤亡人数、原因及其影响》}。

城破时,叶名琛怯懦地化妆出逃,换上苦力的衣服躲进了越华书院。不久后,因英军搜捕甚急,他又钻进广州八旗副都统署后花园的八角亭内。经过细致的搜索,1858年1月5日,英军终于俘获了叶名琛,将肥头大耳的他押上英舰“刚毅”号。2月22日,他踏上了被送往印度加尔各答的旅程,并获英人允许携带产自清帝国的粮食。到达加尔各答后,他被囚禁于威廉炮台。直到此时,这个丧心病狂的杀人恶魔仍不知悔改。他不但以炫耀的语调向英人描述自己屠杀粤人的细节,更惺惺作态地自比“海上苏武”,写下了“向戎何必求免死,苏卿无恙劝加餐”这种令人作呕的诗句。在吃完随身粮食后,身为死心塌地的清帝国走狗的他便开始绝食,表示自己要效法商代忠臣伯夷、叔齐“不食周粟”。英人当然不会可怜他,正义当然不会放过他。1859年4月9日,在极度痛苦与孤独中,叶名琛在加尔各答如一条野狗般死去,获得了他应得的下场\footnote{韩泰伦:《目击天安门》卷1,页166—167}。这真是件大快人心的喜事。

联军攻占广州时,穆克德讷及柏贵坚守着岗位,一边安抚民众,一边阻止乱兵暴民杀掠百姓,直到成为俘虏。当时,广州城内的5000名联军只配有两名翻译,而他们将要统治100—200万市民。因此,他们必须寻找一位代理人,而性情谨慎温和的柏贵实为不二人选。在广州行商和士绅的恳求下,柏贵于1858年1月8日接受了联军的条件,从而成为联军的傀儡。他被限制在由英军看守的巡抚衙门中,将大部分权力移交给了由巴夏礼主持的“联军委员会”,穆克德讷则将驻防八旗的武器全部上交,城内治安转由英法士兵组成的警察部队负责,一百多名粤人则被招进联军委员会中担任幕僚。1月26日,联军攻克广州的消息传到北京,本以为能够收到捷报的咸丰帝大吃一惊,忙将叶名琛解职,代之以黄宗汉\footnote{魏斐德:《大门口的陌生人:1839—1861年间华南的社会动乱》,页204}。2月15日,咸丰帝又令乡居的侍郎罗悖衍(顺德人)、太常寺卿龙元禧(顺德人)、给事中苏廷魁(高要人)举办团练。被称为“三绅”的三人于3月28日在顺德成立“广东团练总局”,又于十七天后将之迁往花县。南粤土豪对继续抗击西方人兴趣寥寥。经多方筹备,三绅仅从花县、东莞、新安、南海、石井等地招募到了万余团练\footnote{黄勋拔:《广东省志政治纪要》,页43}。6月1日,千余名新安籍乡勇在新安士绅陈桂籍的率领下集结于白云山下,似有攻城之意。两天后,800名联军士兵出城驱散了他们。然而,三绅却将此次规模不大的失败粉饰成一场歼敌百余人的胜仗。11日,黄宗汉到达惠州,在当地设立了新的总督衙门,并准备与三绅展开积极合作。不过,咸丰帝无法再被此种虚幻的“胜利”鼓舞了,因为英法联军已于5月20日攻克大沽口,并在6月26日逼迫清帝国签下了《天津条约》。从此,英法公使可以居于北京,清帝国增开营口、登州、台南、汕头、琼州、汉口、九江、南京、镇江为通商口岸,对英赔偿白银400万两、对法赔偿200万两,西方人则可在清帝国境内传教、游历、通商\footnote{魏斐德:《大门口的陌生人:1839—1861年间华南的社会动乱》,页201—202}。7月21日,三绅组织7000名乡勇对广州城发动了一次奇袭,他们一度攀上了西北城垣,但最终在联军的猛烈炮火下付出了惨重伤亡,狼狈撤退。不久后,咸丰帝要求三绅停止进攻的“圣旨”到达南粤,指出:

\begin{quote}
现在夷人仍踞省城,既不与官绅为难,亦只可暂与相安。其民夷仇杀之案,无关大局者,仍毋庸与闻\footnote{梁廷柟:《筹办夷务始末》卷30}。
\end{quote}

至此,团练的主动进攻停止了。三绅开始裁撤团练,将其规模缩小到三四千人。然而,咸丰帝仍毫无信义地向罗悖衍发出密令,要求其继续率团练袭击联军。因此,双方虽然已在名义上停战,但仍不断有落单的联军士兵在广州附近惨遭杀害、俘虏。至12月,联军开始巡逻广州近郊。1859年1月11日,经过三天激战,1300名联军攻克团练在广州北郊的重要据点石井村。20日,一小支联军溯西江而上进入佛山,受到当地官绅的热烈欢迎。2月8日,联军攻克花县,将三绅赶往顺德。19日,联军又进一步逆西江而上,攻占了由苏廷魁部把守的肇庆\footnote{魏斐德:《大门口的陌生人:1839—1861年间华南的社会动乱》,页206—207}。至此,广东团练总局的抵抗已被联军完全粉碎。

攻占石井时,联军缴获了咸丰帝发给罗悖衍的密令。巴夏礼立即将之交给额尔金,后者则对清方假称koy 绝不相信皇帝会如此背信弃义。无奈之下,咸丰帝只得于1月29日对黄宗汉“下诏”称:

\begin{quote}
著黄宗汉严拿伪造(密令)之人,尽法惩办\footnote{魏斐德:《大门口的陌生人:1839—1861年间华南的社会动乱》,页207}。
\end{quote}

这样,咸丰帝就自己否认了他曾发布的密令,明确下令南粤团练不得再做抵抗。对于愚昧残暴的清帝国,南粤人早就非常不耐烦了。南粤人惊喜地发现,英法联军处事公正、纪律也比清帝国侵略军好得多。自1859年2月下旬起,只要联军巡逻兵出现在一座村庄,村中士绅、耆老便会热情地列队欢迎\footnote{魏斐德:《大门口的陌生人:1839—1861年间华南的社会动乱》,页207}。5月,由于十三行商馆已在战争中被焚,联军委员会将十三行以南的沙面岛辟为租界,其中西部264亩之地为英租界、东部66亩之地为法租界\footnote{王永平:《广州发展纪事》,页77}。1860年10月,英法联军攻克北京,火烧圆明园,逼迫清帝国签订《北京条约》。在条约中,清帝国承认《天津条约》有效,增开天津为商埠,割让九龙司地方一区给英国,并将对英法的赔款额各自提升到800万两。战争结束了,但联军对广州的占领还未结束。柏贵已在1859年病逝,此后联军委员会便开始直接统治广州。1861年9月3日,新任两广总督劳崇光与英方签订《沙面租借协定》,规定英方以每年39.6万文的费用永久租借沙面。此后,沙面形同英法两国领土,十三行的外商纷纷迁往该岛\footnote{王永平:《广州发展纪事》,页77}。在广州城及其附近,联军委员会努力打击人口贩卖,将对百姓滥用酷刑的清帝国官吏投入监狱,并免除了清帝国对广州商铺的苛捐杂税,赢得了成千上万粤人由衷的爱戴。联军士兵一次次巡逻着乡村地区,打击盗匪,维护农民们的安全。珠三角土豪和百姓视联军为秩序的维护者,时常请求联军委员会出面革除地方上的弊政。经过二十余年的交手,光明磊落的南粤人终于认识到了西方人的伟大之处。南粤人曾因保卫家园的热情奋起与西方人作战。现在,南粤人又怀着同样的热情抛弃愚昧腐朽的清帝国,与西方人展开了全面合作。这一事实雄辩地证明,勇敢淳朴的粤人不但是真正的勇士,亦是积极融入世界大潮的文明开化者。通过与联军的合作,南粤人得以证明自己乃有资格雄踞于世界之上的优秀民族。然而,美好的日子总是那么短暂。随着战争结束,英法联军将珠三角交还清帝国的日子也逼近了。1861年10月21日,联军在越秀山上列队集合,这里正是二十年前英军与三元里乡勇作战时的营地。在令人唏嘘的沉默中,联军整队离去。广州市民怀着不舍与复杂的心情目送他们离去,并眼睁睁地看着自己的城市又一次落入清帝国之手\footnote{魏斐德:《大门口的陌生人:1839—1861年间华南的社会动乱》,页208—212}。

\section{族群边界的稳定:土客战争}

\indent 几乎与鸦片战争和洪兵战争同时,另一场极度血腥的战争在南粤大地上演。对南粤来说,这场夺走了100万条生命的战争至关重要,它是划定南粤内部族群边界、确立族群关系的关键一役。这场战争的一方是世居南粤、讲粤语的广府土著,另一方则是17世纪以来大举进入珠三角的客家人。今天,广府、客家已和而不同地居住在南粤的土地上,共同延续着南粤的伟大文明。但在仅仅一个半世纪前,双方却是不共戴天的死敌。双方何以形成这样的仇恨,又何以最终走向共存?欲了解此点,便需首先了解这场战争的起因。

在今天的东亚大陆上,客家人是个分布广泛的族群。他们的主要聚居区位于粤东北、闽西和赣南,在八桂、湖湘、巴蜀、吴越、海南、台湾乃至陕西亦有他们的身影。据著名客家学者罗香林在20世纪中期的说法,客家人系北方汉人的正统后裔,曾经历过五次迁徙:两晋之际,大批中原士族南下吴越,是为客家的首次迁徙;唐帝国崩溃后,闽主王审知大力延揽士大夫,一批士族由吴越南迁至闽、赣,是为客家的第二次迁徙;两宋之际,金兵攻入江右,部分士族由闽、赣迁至粤东、粤北,是为第三次迁徙;明清之际,大批北人为躲避战祸逃入作为客家大本营的粤闽赣交界区,是为第四次迁徙;19世纪后期,因土客战争、洪兵战争的影响,许多客家人迁往台湾、海南、香港、澳门乃至世界各地,是为第五次迁徙\footnote{刘平:《被遗忘的战争——咸丰同治年间广东土客大械斗研究》,页8—9}。“五次迁徙说”在现存的客家宗族族谱中得到了印证,为客家人提供了一套完整的中原起源理论。然而,只要我们仔细考究一下历史,便会发现这种说法是颇有问题的。

当代分子人类学的研究成果显示,客家人的父系Y染色体虽有七八成来自北方祖先,但他们的母系mtDNA有接近七成来自百越祖先\footnote{李辉:《岭南民族源流史》,页454;李晓昀、苏敏、黄海花、李辉、田东萍、高玉霞:《潮汕人与广府、客家人母系遗传背景的分析》}。由母系血统中百越成分占据如此优势来看,当代客家人的百越血统和北方血统至少是旗鼓相当的。客家人的百越血统来自粤闽赣交界区原住民畲人。“畲”为刀耕火种之意,畲人在字面意义上即指刀耕火种者。畲人是百越人的一支。早在隋唐之际,畲人就已在粤闽赣交界处形成。那里群山纵横,又处在帝国数个行政区之间,向来为帝国鞭长莫及之处,系逃避帝国赋役、逃避战争、向往自由者的天然乐土。自两宋之际起,便不断有北人进入赣南、闽西与畲人一同生活。在他们的影响下,畲人摆脱了刀耕火种的生产方式。他们的语言和畲语互相影响,形成了一种新的共同语,这便是被后世称为客家话的语言。他们在丧葬、婚姻、饮食等方面的习俗也全面畲化,变得与畲人难分彼此。到15—16世纪,这一血统混杂畲北、习俗畲化的族群进一步向人口较少、土地较多的粤东、粤北迁徙。他们被明帝国广东当局称为“客户”,用以与被称为“主户”的南粤本地居民相区别。当时,粤东北的程乡(今梅州)已成为“客户”的重要聚居区,粤北韶州、南雄两府亦成为“客户”的天下。自两宋之际湘赣流寇屠掠粤北后,当地人口数百年来一直难以恢复。“客户”的涌入令当地重获生机,逐渐恢复了繁荣的面貌。在南雄始兴县,“客户”占当地人口的七成。在韶州,当地出现了“主户少而客户多”的人口格局。东江、北江、韩江上游的长宁、永安、连平、和平、大埔、平远、镇平成为纯粹的“客住县”,程乡、兴宁、龙川、河源、始兴、英德、仁化、长乐等地亦分布着大批“客户”\footnote{李辉:《岭南民族源流史》,页458—459}。16、17世纪之交,“客户”继续向西南迁徙,进入惠州府境内的海丰、归善、博罗等地。至此,他们已经逼近了南粤的中心珠三角\footnote{李辉:《岭南民族源流史》,页459}。17世纪后期的“迁界”浩劫后,他们又继续迁至人口损失严重的广州府之番禺、东莞、香山、新安、花县、清远、龙门、从化、三水、新宁,肇庆府之高要、广宁、新兴、四会、鹤山、高明、开平、阳春等县,有的人甚至西进至粤西高州、雷州、廉州、罗定等地,并到达广西\footnote{刘平:《被遗忘的战争——咸丰同治年间广东土客大械斗研究》,页12}。这些地区大多为广府人聚居区。对这些语言风俗与自己差异极大的“客户”,广府人呼为“客民”、“客家”\footnote{李辉:《岭南民族源流史》,页459}。

自16世纪以来,广府人即展开了华夏化过程。他们以自己华夏嫡传的身份而自豪,蔑称客家话为“入耳嘈嘈”的“南蛮鴃舌”。在清帝国治粤前期,由于客家人的教育水平不如广府人,其内部尚未出现稳定的士大夫群体,因此并未对这一攻击做出有效回应。1733年,清帝国将客家人聚集的粤东北兴宁、长乐、平远、镇平、程乡五县划为由广东省直隶的嘉应州。1807年,又升嘉应州为嘉应府。此种行政变动表明,清帝国察觉到了南粤客家人聚居区的独特性。1815年,一位名叫徐旭(惠州和平人)的客家士人在惠州丰湖书院任教时对koy 的学生讲了ni 样一番话:

\begin{quote}
	

博罗、东莞某乡,近因小故,激成土客斗案,经两县会营弹压,由绅耆调解,始息。院内诸生,询余何谓土与客?答以:客者对土而言,寄居该地之谓也……(客家人)有由赣而闽,沿海至粤者,有由湘赣逾岭至粤者……粤之土人,称该地为客,该地之人亦自称为客……今日之客人,其先乃宋之中原衣冠旧族,忠义之后也\footnote{转引自程美宝:《地域文化与国家认同:晚清以来“广东文化”观的形成》,页71}。
\end{quote}

这段被徐旭的学生记录下来的说话,系现存最早的客家人“中原起源说”。值得注意的是,徐旭在此自称为“客”,认同了广府人送给自己族群的称谓。由此段记载可知,客家的“中原起源说”实为近二百余年内形成的理论,乃19世纪初的客家人在面对广府人攻击时做出的反击。客家人因被华夏化广府人指为“非华夏”的“南蛮”、“犵獠”,故徐旭便发明出一套“客家华夏论”进行反击。稍后,客家举人林达泉(大埔人)将此理论推进一步,提出了更为“激进”的观点:

\begin{quote}
客家多中原衣冠之遗,或避汉末之乱,或随东晋、南宋渡江而来。凡膏腴之地,先为土著占据,故客家所居之地多硗瘠,其语言多合中原之音韵\footnote{转引自刘平:《被遗忘的战争——咸丰同治年间广东土客大械斗研究》,页15}。
\end{quote}

在此,林达泉将客家人塑造为汉、晋、宋等帝国的南渡“中原衣冠之遗”,将广府人称为“土著”。此外,他还有更大胆的发明:

\begin{quote}
由是观之,大江以北,无所谓客,北即客之主。大江以南,客无异客,客乃土之耦。生今世而欲求唐虞三代之遗风流俗,客其一线之延也\footnote{转引自刘平:《被遗忘的战争——咸丰同治年间广东土客大械斗研究》,页15}。
\end{quote}

可见,林达泉已提出“北即客之主”,称客家人为唐虞三代文明的最后继承者。这样的叙述洗清了客家人的“南蛮”身份,反而暗示广府人才是“南蛮”土著。这样一来,在19世纪前期,广府、客家两族便各自自命华夏正统,对身为“南蛮”的对方展开口诛笔伐。

广客两族之所以如此仇恨对方,与两者对资源的争夺很有关系。做为移民,南粤客家人饱受流离之苦。由于粤闽赣交界处群山纵横、地少人多,客家人只能不断向外迁徙。为了在迁徙的过程中生存下去,他们多以宗族为单位成群结队地行进,到达目的地后亦聚族而居。为了防止盗匪、野兽的侵害,他们以宗族为单位结成一个个军事共同体。在此种共同体中,强壮的男性皆是骁勇的战士,瘦弱者被蔑称为“末朝人”\footnote{刘平:《被遗忘的战争——咸丰同治年间广东土客大械斗研究》,页48—49}。女性则皆不缠足,负责照顾老幼、纺织、烹饪。客家人每定居于一处皆修筑起高大的围楼。这些围楼一般有三至六层,每层高三米、有约三十座房间。这些房间的大小和结构大致相同,通过走廊相互连通。一般来讲,通风、采光良好的三楼以上是卧室,二楼用作粮仓,一楼的大厅则是宗族议事和举办庆典的场所。有的围楼内还设有学堂,专供教育族中子弟。当有敌人来攻时,围楼中的壮丁便携武器进入各房间,通过窗户和枪眼向外射击。可以说,每一座客家围楼都称得上是坚固的堡垒\footnote{袁君煊:《客家围屋军事防御艺术管窥》}。

如果说客家人是新阑入南粤的蛮族勇士,那么广府人就是骁勇善战的原住民。在漫长的历史上,广府人及其祖先为南粤的自由与尊严进行过一次次血肉搏杀,涌现出了无数英雄儿女。自16世纪宗族化以来,每个广府宗族都是一个坚固的军事共同体。在明末清初大洪水中,这些共同体纷纷武装自己,一面与各路侵略者展开激战,一面抵挡盗匪的侵袭。据《广东新语》记载,当时新会县的宗族筑有六七丈高的碉楼。一到战时,妇孺即避入其中\footnote{屈大均:《广东新语》卷8}。明末至土客战争前夕开平县滘堤洲大族司徒氏的组织形态,更为我们观察军事化的广府宗族提供了绝好案例:


\begin{quote}

滘堤旧称素封,俗非悍健,邻匪垂涎久之。一旦乘间窃发,于斯时也,内无山谷之固,外多烽烟之警,干戈满目,风声鹤唳,人心惶惶,朝不谋夕。使非有人焉维持其间,其祸可胜言乎……(福相公)勇敌千夫,谋出完全,乃集族中父老而预筹之,结土团,谋积聚,备器械,时训练,倡勇敢,守要害,人心皆定,战守皆宜。邻贼望风惊畏,不敢轻萌窥伺,间有妄想思吞噬者。公率乡人奋其义勇,摧彼凶锋。兵刃始交,莫不惊顾而遁。其时家室无虞,骨肉相保,鸡犬不惊,田庐如故……迄今二百年间,生齿日繁,人文蔚起,士农工商者各安其业,田原茂美,井里雍和,出作入息,相与优游于光天化日之下\footnote{转引自刘平:《被遗忘的战争——咸丰同治年间广东土客大械斗研究》,页56}。
\end{quote}

由此段记载可知,司徒氏在一位被尊称为“福相公”的长老带领下团结族中子弟积极自守,使一族之人在大洪水中生存下来,并在其后的长期和平中繁衍为人丁兴旺的富裕大族。如司徒氏一般的广府宗族数不胜数。在洪兵战争和第二次鸦片战争中,这些宗族组织的团练决定了战争走向。他们击败了洪兵对南粤土豪自治的颠覆,并抛弃了叶名琛,积极与英法联军展开合作。可以说,广府宗族是股能够决定南粤历史走向的伟大力量。

自17世纪起,广府人和客家人间围绕土地资源展开了激烈的竞争。做为移民的客家人往往充当广府土著的佃户,久之便依靠庞大的军事、宗族共同体力量将土地据为己有。此种历史记忆一直持续到现在,在粤语中留下了“客家变地主”这一俗语。相对地,客家人也会在一些他们立足未稳的地方遭到广府人欺凌。在平静的18世纪,随着两族人口不断增长,此种冲突越来越多,双方频频械斗\footnote{刘平:《被遗忘的战争——咸丰同治年间广东土客大械斗研究》,页63—67}。至19世纪中期,随着洪兵战争爆发,南粤社会陷入全面战乱,土客战争呼之欲出。

1854年七月十九日,一队洪兵攻陷开平县城,在城中大肆杀掠。这支洪兵以广府人为主,对客家人颇为凶残\footnote{刘平:《被遗忘的战争——咸丰同治年间广东土客大械斗研究》,页115}。当时,珠三角西部有高要县客籍举人马从龙正在兴办团练。马从龙欲趁此机会报复,遂向叶名琛请缨出击,得到批准。八月,客勇一举夺回开平县城,随后开始大举袭击开平境内的广府村庄。在恩平县,客勇亦闻风而起,借攻剿洪兵之名杀掠广府人,土客战争由是爆发。十一月,客勇攻破恩平横陂村,将全村六七百名男女老幼尽数屠戮。此后半年内,恩平客勇开始对广府人大施暴虐,平毁广府村庄四百多座,涌入县城的广府难民达数万之多。另一方面,遭到袭击的广府人亦以宗族、村庄为单位群起反抗,并报复性地毁灭了许多客家村落。从1855年开始,广府、客家人的族群战争蔓延到高要、高明、鹤山、新宁(今台山)等县。双方怀着深仇大恨疯狂地循环报复、互相“铲村”,任何一方失守的村庄都惨遭焚烧,来不及避难的村民大都成为刀下亡魂、部分妇女则被掳走。这一清形一直持续到1857年初。在此期间,大小冲突、死者成百上千的屠村惨案不计其数,笔者不可能将之一一列举。总而言之,数以万计的广府人与客家人在这些仇杀中失去了生命,数以百计的村庄被付之一炬\footnote{李天:《要明鹤恩开台六邑土客械斗始末》}。这着实是南粤史上至为惨痛的一幕。

1857年三月,随着开平、恩平、高要、高明、鹤山、新宁的广府士绅纷纷开设团练局,双方陷入僵持。因两族在两年半以来的疯狂仇杀中皆伤亡惨重,双方不约而同地停止了进攻。由此一直到1858年六月,双方只有小规模交战。就在局势趋于缓和时,一个名叫谭三才的开平籍港商掀起了新波澜\footnote{关于此一阶段复杂的战争进程,参见刘平:《被遗忘的战争——咸丰同治年间广东土客大械斗研究》,页65—217}。谭三才系开平长塘人,早年在英治香港经商,获得巨资。土客战争爆发后,koy 回乡组织团练,并购置了大批西式步枪。1858年七月,谭三才联络开平士绅成立“万全局”,集结广府土勇二万余人,会同从外县招募的数千名无业游民分路东进,大举袭击新宁西路的客家村落,意图消灭该县客民,双方战火重燃。客勇“蜂拥与御”,与谭三才之军鏖战三个月,使谭军伤亡过半,无力再进,只得退兵。十月,谭三才又与新宁东路的土勇联络,在当地成立“伟烈堂”,帮其购置洋枪,令其以3000余人进攻赤溪半岛的客家据点曹冲。曹冲云集着上万名兵强马壮的客民,系一处难攻不落的要塞。十二月,伟烈堂土勇屡攻曹冲,皆告失利\footnote{刘平:《被遗忘的战争——咸丰同治年间广东土客大械斗研究》,页164}。至此,谭三才只得求助于英国人。1859年初,谭三才以失窃案为借口请求港英政府派兵援助。英人信以为真,乃派兵40余名乘轮船赴曹冲,与谭三才约定于五月二日一同进攻“盗贼”。然而,英军却于五月一日擅自行动,于赤溪半岛之角咀登陆,直接进攻曹冲。这些英兵实在是过分轻敌了,他们没有想到客勇绝非不堪一击的清帝国侵略军。英军刚一进攻,就遭遇曹冲客勇激烈抵抗。客勇以大队人马正面迎击,又以一支奇兵迂回到海滩截断英军归路。经过短促而激烈的交手,这一小支英军全军覆没,阵亡30余人、被俘10余人,他们的40多支步枪也全被缴获。经审讯,客勇得知这些英兵系受诓骗而来,便让他们驾船返回。次日,不知英军已败的谭三才亲自指挥伟烈堂土勇分水陆两路攻打曹冲,结果中伏大败,遗弃洋枪百余条。七月、十一月,伟烈堂又对曹冲发动两次进攻,均被击退\footnote{刘平:《被遗忘的战争——咸丰同治年间广东土客大械斗研究》,页164—165}。连经挫折的谭三才已无计可施,其对新宁的进攻宣告失败。

就在广府人受挫于新宁之际,阳江的广府人又发动了新的攻势。1859年十二月,阳江土勇开始大举进攻客家村落。至1860年二月,该县客民已被基本肃清。三月,鹤山、开平、恩平、新宁、鹤山、阳春等县的土客士绅纷纷会议,“定盟相好”,双方再次停火。同年冬,新安县客绅李道昌率壮丁千余人入驻赤溪半岛,巩固了曹冲据点。新宁西路的客家人听闻东面的赤溪半岛系一片客民栖居的乐土,纷纷携家带口东迁。十一月,有赤水、深井等地的客家富户4000余人乘船东迁,路遇广府海盗陈列仔打劫,被夺走金银20余万两,男女被杀者计2000余人。这一惨剧之后不久,战争又一次爆发。1862年正月,新宁东路的广府人“因事与客失和,渝盟寻斗”,战火迅速蔓延至开平、恩平、高明、高要境内,双方互相“铲村”,各有死伤。在高明县,当地广府人势力较弱,全县中、西部三分之二的土地为客家人控制,连县城都于1857年被客勇占领。此次开战后,全县广府团练联合作战,于三月集结重兵包围县城。经近半年围攻,高明城内“粮民俱绝,食及草根,木叶、牛皮俱尽”,守城客勇渐渐不支。八月,土勇以八千斤重炮及火药轰塌城墙,蜂拥而入。城中幸存客家人除数十人出逃、数百人投降外,其余3000多人俱被杀死\footnote{刘平:《被遗忘的战争——咸丰同治年间广东土客大械斗研究》,页216}。八月至十月间,开平、恩平两县土勇控制了新宁西路的那扶一带,当地客民纷纷逃难至大门、深井等地,野栖露处,惨不堪言。流亡客民乃成立团练组织“福同团”,推举监生汤恩长为团长。汤恩长抽调3000名精壮编列队伍,组成一支战斗力颇强的精兵,于十二月三十日开始护送难民向曹冲逃亡,一路屡遭土勇截杀。1863年正月一日,汤恩长率大队人马进抵广海寨城前,要求守寨清军开门。但因城中的广府居民群情激愤,清兵不敢开城。二日,汤恩长率福同团攻城,一举打破城门。经激烈巷战,广海失守,福同团随即对守城军民展开残酷屠戮,“城内伏尸山积”,死者4000余人。在广海屠城发生之前,已被英法联军逼得走投无路的清广东当局一直对土客战争采取不闻不问的态度。但此次事件已属客勇攻陷清军驻守的寨城,严重威胁到了清帝国对南粤的统治,清军遂开始介入战事。三月十三日,清广东按察使吴昌寿、顺德协副将卫邦佐、汤骐照等率6000军队,在土勇的配合下自水陆两面包围广海城。围城前夕,城中客民大都已出逃,只有汤恩长仍率3000人留守。漫长的围城战持续到七月二十九日,汤恩长终因城中乏食率众开城出逃,遭清军、土勇分头堵截,死者上千人,余部随汤恩长逃至深井、大湖山等处\footnote{刘平:《被遗忘的战争——咸丰同治年间广东土客大械斗研究》,页170}。曾经强横一时的福同团,至此土崩瓦解。

当客勇精锐被围于广海城时,恩平、开平、新宁三县土勇趁机大举进攻聚集于新宁大隆洞山区的客民。当时,大隆洞一带有客民二十余万,多为老弱病残,缺衣少食。自1863年四月中旬起,大隆洞山区陷入重围。客民在山野间缺药少食,饱受瘟疫困扰,死者蔽野。清兵与土勇长驱直入,大肆屠杀,至六月中旬肃清大隆洞山区。除逃走者外,病死、饿死在大隆洞一带的客家人达数万人之多。与此同时,又有数万名以头人戴子贵为首的客家人被土勇围困于新宁、恩平交界处的那扶、深井山区,染病死者日以百计。1864年初,他们被迫向西北方向突围,进入阳春县境内,遭清军堵截,遂掉头转向东北进入新兴县境,围攻县城而不克,只得退入高明西部的五坑山区。这股客家人自称“大同军”,实则多为难民,然拥有数百名精锐骑兵。因此,在大同军进入五坑山区后,高明、恩平、新宁、鹤山等地被打得节节败退的客民遂纷纷涌入那里寻求庇护,当地客民人数在短时间内便膨胀至十余万人\footnote{李天:《要明鹤恩开台六邑土客械斗始末》}。十月,清广东当局急命雷琼副总兵卓兴统一万军队包围五坑山区。经一年围困,山中客家人终于饥饿难忍,决定出降。戴子贵“自缚负刃”,与百余随从至卓兴营中为部下求活。卓兴系潮州人,对客家人颇有亲近感(关于客家人与潮汕人的关系,详见下文),遂一面命人将戴子贵押往广州,一面向清广东当局请求宽恕山中客民,获得批准。1865年正月二月,卓兴在五坑点验客民完毕,随后率军将愿意迁出的客民护送至恩平、开平、新宁三县之那扶、金鸡、大门、深井等地“安插”居住\footnote{李天:《要明鹤恩开台六邑土客械斗始末》}。到这时,土客战争似乎已经落下帷幕。

然而,两族间的仇恨是不会这么轻易就消除的。完成“安插”任务后,卓兴即率兵撤离。恩、平、新三县广府团练待清兵一走,便立即群起攻杀客家人。五月,三县客家人被迫放弃刚刚入住的家园,逃亡新宁西路沿海山区。至年底,他们又在土勇的追击下逃至恩平西部那吉、岑洞、清湾的山泽中,靠采食野菜艰难维生。1866年,新任恩平知县罗德辅大力策动该县土勇“剿客”,迫使客民向北逃至新兴、阳春等地,复遭清兵、土勇堵截,只好退回原处竖栅筑垒自守。因山中乏粮,他们便四出劫掠,并连攻恩平县城两个月而不克。见变乱又起,湘军出身的清广东巡抚蒋益澧命总兵徐文秀率万名湘军前往恩平镇压。湘军系为清帝国消灭太平天国的精锐部队,战斗力十分强劲。当年六月,以湘军为主力的清军自新兴、阳春、阳江三面合围,徐文秀将湘军主力屯于恩平大槐,封锁那吉山口。数万客民困于夏雨连绵的山中,因瘟疫而死者甚众,只得缴械投降,将为首的百余人交给湘军处置。湘军开入那吉,将山中客民押往粤西高州、雷州、琼州及广西、湖广、福建等处“安插”。十二月,清军攻占五坑。1867年春,五坑客民亦被清军押出,送往清远、四会、韶州、嘉应、潮州、惠州、新宁、琼州等地。至此,仍在抵抗的客民据点就只剩赤溪半岛的曹冲了\footnote{李天:《要明鹤恩开台六邑土客械斗始末》}。

当时,聚集在曹冲的客民已增至二万多人。他们拥有广袤的良田和坚固的工事,足以自守。自1866年末起,蒋益澧即命三万湘军以土勇为向导自陆路、水师战船百余艘自水陆进攻曹冲。因曹冲客民皆为“杀戮余生,久经战斗,视死如归”者,清军虽有舰炮支援,仍屡攻不克。每当清军整队进攻时,客勇即以三十至四十人为一队冒死前进、分头截杀、起伏无常,使清军疲于奔命。1867年三月,蒋益澧统亲兵赶赴战场,在赤溪以西二十余里的狮山设下大营,指挥清军炮击客勇工事达月余之久。客勇支撑不住,终于同意开寨投降。四月二十二日,蒋益澧在大营设宴款待土客士绅,令双方“莅盟释憾”。五月,赤溪客民交出军械,蒋益澧班师广州,留下官吏与土客士绅商议善后事务,议定将整个赤溪半岛划归客家人居住,并将新宁县其余地区的客民田土交给广府人。至此,历时十三年的土客战争终于划上了句号。

历时十三年的土客战争终于结束了。在这场战争中,南粤的广府、客家两族均付出了极度沉重的代价。这场战争波及珠江口以西的十余个县,以鹤山、开平、恩平、高明、高要、新宁为主战场。战争波及了数以百万计的人口,使100万人丢失了性命。在死者当中,广府、客家两族各居半数,其中大部分都是死于屠杀、饥饿和瘟疫的平民。在战事最为惨烈、人口损失最为惨重的新宁县,客家人口由战前的30万锐减至3万,广府一方亦有17万人死亡。在屡经双方争夺的高明县,大战过后人烟稀少、荒田随处可见\footnote{刘平:《被遗忘的战争——咸丰同治年间广东土客大械斗研究》,页216—217}。经此惨烈战争,客家人在南粤的西进势头被广府人遏制住了。珠江口以西的大部分客家人或已死亡、或被驱逐,当代南粤广府人与客家人的族群地理边界由此定型。在战争中,土勇和客勇为保卫自己的亲人、家园皆进行了勇敢的战斗,双方进行了一场武士之间堂堂正正的对决。共同体的边界需要靠流血来划分。在流够血之后,崭新的秩序便应运而生。此后,两族不再进行如此惨烈的拼杀,而是认识到了对方的武勇、承认了对方的存在,一同以南粤为家,共同在战争的疮痍中重建南粤。很快,两者便将在惊涛骇浪中一同承受巨大的考验,一同亲如兄弟地为南粤的自由与尊严并肩作战。

在1839—1867年间,南粤大地上一共发生了两次鸦片战争、洪兵战争及土客战争这四场战争,超过150万南粤人为此丧失了生命,这是南粤近代史上自明末清初大洪水以来最惨痛的一页,我们应当为这些死难的同胞深深默哀。他们的血当然没有白流。在这二十九年中,南粤以150万条生命为代价确立了自己的宪制:鸦片战争锁定了南粤对外开放的历史路径、洪兵战争确立了南粤土豪自治的社会秩序、第二次鸦片战争证明粤人乃拥有足够资格与西方世界携手合作的开化民族、土客战争明确了南粤内部的族群边界。至此,一个开放、开化、土豪自治、各族和而不同地共存的南粤已经成型,南粤的宪制定型了,当代南粤民族距离被成功发明出来已只有一步之遥。










