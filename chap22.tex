\chapter{种子不死:1972年后的南粤}

\section{本土凝结核复兴与西方秩序回归:1972—1992年}

1972年3月10日,美国威尔斯利学院政治学系教授威廉·约瑟夫(William Joseph)率一批亚洲问题学者从香港越过边境,对深圳、广州进行短期访问,拍下了一批彩色照片。时隔23年,美国专家再次踏上南粤大地。此次访问虽规模不大,却意义非\footnote{《1972年,美国人镜头中的广州第六十一中学》}凡。它表明,南粤回归西方秩序的进程已经重启。此后,南粤将在这条道路上坚定前行。

与西方学者访粤相随的,是西方国家对南粤的技术输入。1973年2月,经毛泽东、周恩来批准,中共决定在此后三至五年内引进价值43亿美元的成套西方设备,包括13套大化肥、4套大化纤、3套石油化工、10个烷基苯工厂、43套综合采煤机组、3个大电站、一米七轧机、透平压缩机、燃气轮机及工业汽轮机制造厂等项目,是为“四三方案”。“四三方案”各项目多与化工有关,其最显著的作用在于粮食、布匹增产\footnote{史云、李丹慧:《中华人民共和国史第八卷:难以继续的“继续革命”》,页292}。在以美国为首的西方阵营看来,唯有由他们喂饱数以亿计的“中国人”,中共方能真心诚意地与西方合作,一同对抗苏联\footnote{布热津斯基曾在1978年有过相关讲话}。1974年10月,由法国进口、年产合成氨30万吨的广州化肥厂投产,这是西方国家在1949年后对南粤的首次技术输入。该厂使南粤粮食产量迅速增长,有效缓解了珠三角民众在中共治下普遍承受的饥饿\footnote{《我国20世纪70年代引进13套大化肥装置的由来及现状》}。

自1960年代末起,因列宁主义机器已在文革冲击下产生松动,南粤社会中的本土凝结核如雨后春笋般重新生长出来。因城市中的中共大员普遍受到冲击,忙于政治斗争,无法像从前那样管制经济,大批农民便依托原有的宗族、血缘纽带集资建设乡镇企业,做起生意。为规避中共的管控,他们将这些企业称作从属于人民公社、生产队的“社队企业”。今日大名鼎鼎的美的集团,便是由社队企业发展而来的。1968年,顺德北滘公社农民何享健依靠原有宗族纽带,与23名同乡集资5000元,创办了一个瓶盖厂。为伪装成“集体企业”,他们称之为“北滘公社塑料生产组”。随着生意越做越大,他们开始为广州的一家风扇厂生产配件,为日后成长为电器界巨头打下了基础\footnote{郭克莎主编:《2003年度中国企业最佳案例:人力资源》,页150}。在整个珠三角乃至整个南粤,如何享健般的例子不计其数。数以千万计的民众趁中共内讧辛勤建设家园,使生机重临南粤大地。到1975年,广西已有乡镇企业34472个,广东更多。在东亚各邦,类似的情况也在多处上演。对乡镇企业的大批兴起,中共当局相当惊恐。1975年6月,中共广西革委会下达命令,要求将农民“自留地”限制在耕地面积的5\%—7\%之间,农民不得私自售粮、从事私人运输、私人开荒,不得随意从事种植以外的“副业”\footnote{高德步:《中国民营经济史》,页54}。在此前后,中共广东当局发动狂热的“割资本主义尾巴”运动,将农民的大批财产毁坏、没收。例如,东莞县黄山大队原养鹅1600多只,竟被强迫杀至仅剩400多只\footnote{史云、李丹慧:《中华人民共和国史第八卷:难以继续的“继续革命”》,页285}。然而,中共高层明白,除非再展开一次如1950年代时一般的列宁主义格式化运动,否则便根本无法压制各邦自发秩序的强劲生长。当时,毛泽东、周恩来等人已近垂暮,无复当年“魄力”\footnote{史云、李丹慧:《中华人民共和国史第八卷:难以继续的“继续革命”》,页290}。1975年9月,中共中央只得“同意”农村开办“社队企业”。这不过是对既成事实的追认。

林彪倒台后,因一批林系将领遭到清洗,军队系统在中共党内地位日渐衰落。以周恩来为首的列宁主义官僚大举出击,迅速抢占军队系统空出来的生态位。1972年,在周恩来保护下,赵紫阳重返广东,出任革委会主任和省委书记。1973年底,为剪除林彪、黄永胜等人在广州军区的残余势力,经毛泽东同意,周恩来、叶剑英又将同林彪素有恩怨的许世友调任广州军区司令。赵紫阳本为文革前主政广东的列宁主义官僚,许世友则是曾在南京积极压制造反派的将领。赵、许二人在广州上任,标志着列宁主义官僚势力于南粤再起。此时的南粤社会正逐步恢复生机,列宁主义机器在南粤的威慑力已严重受损,受过粤人冲击的赵紫阳亦无力重现旧日凶焰。为清除林、黄势力,他只能重新起用曾被黄永胜打压的一批南粤本地干部。到1975年赵紫阳调离广东时,针对本地干部的“反地方主义运动”已难以为继。稍后,韦国清调任广东革委会主任兼省委书记,亦无力发动与1968年类似的屠杀\footnote{陈华昇:《广东“反地方主义”运动与派系冲突之分析(1949-1975)》,《中国大陆研究》,2008年第3期,第25页。}。

1974年1月,毛泽东、江青发动“批林批孔”运动,以批判儒家、林彪为名剑指周恩来。一批南粤原造反派趁此机会重新公开活动,要求中共当局为他们“平反”,少数人获得成功,有的人甚至得到官职。然而到6月,因慑于列宁主义官僚的庞大势力,毛泽东主动叫停运动,南粤造反派的“平反”运动也不了了之。但部分原造派人员因此重燃斗志,决定发起新的进攻。11月10日,一张作者署名“李一哲”、题为《关于社会主义的民主与法制》的大字报出现在广州北京路口,引发大批市民围观。该大字报虽仍打着拥共拥毛的旗号,却对中共的“社会主义体系”进行了尖锐抨击:

\begin{quote}

这种体系——即使我们也可承认我们具有某种体系的话,也绝不是异端于马克思主义体系之外的东西,我们只不过是企图以马克思主义的思想武器去对林彪体系影响、祸害所及的范围作一番清理罢了。事实上我们还远没有做到这一点。
常见的是某些领导者将党和人民给予的必要特殊照顾膨胀起来,变成政治和经济特权,并无限地触及到家族、亲友乃至实行特权的交换,通过走后门之类的渠道完成其子弟在政治、经济上实际的世袭地位,并且围绕着他们的私利,改变事业的社会主义方向,实行宗派主义的组织路线,扶植起一批特殊与人民利益并与人民利益相对立的新贵集团和势力来。
更重要的是,他们为了维护己得的特权和争取更多的特权,他们必须要打倒坚持原则的正值的革命同事,镇压起来反对他们特权的人民群众,非法地剥夺这些同志和群众的政治权利和经济权利\footnote{转引自史云、李丹慧:《中华人民共和国史第八卷:难以继续的“继续革命”》,页59}。

\end{quote}

该大字报虽仍运用中共的文革话语,却公开揭露中共干部的贪婪无耻,并直接攻击“社会主义体系”。在那时,它已足够引起中共当局的恐惧了。事发后,赵紫阳立即将大字报抄录上报“中央”,江青直斥之为“‘解放’后最反动的大字报”。不久,“李一哲”被警察逮捕。原来,“李一哲”是三名广州青年,分别是广美学生李天正、高中生陈一阳和原红旗派大佬、工人王希哲。此后,三人虽承受数年牢狱之灾,但并未丢掉性命\footnote{余宏檁:《“李一哲”事件始末》}。这表明,对南粤人的反抗,中共的列宁主义机器已越来越力不从心了。

1975年1月,因周恩来病重,毛泽东将中共中央日常工作交付重新起复的邓小平。身为列宁主义官僚大佬,邓小平一上台便着手“整顿”秩序,在陇海铁路沿线大肆清洗造派出身干部,与江青产生尖锐矛盾。因邓小平拒不肯定文革,毛泽东遂于11月发起“反击右倾翻案风”运动,将矛头对准邓。1976年1月8日,周恩来病死,失去保护伞的邓随后被软禁。28日,毛泽东起用在列宁主义官僚和文革派间摇摆不定的“中派”华国锋为国务院代总理,主持日常工作。自1月起,成批在文革初期失势的原老红卫兵聚集在北京天安门广场,以悼念周恩来为名猛烈抨击江青。他们的行动吸引了无数民众。广大民众虽憎恨老红卫兵,但已多年未有机会对中共政权发泄不满,遂纷纷赶往广场,示威者很快达到上百万。4月4日,示威者开始攻击军警。5日凌晨,毛泽东派出大批军警和工人民兵实施清场,驱散人群,逮捕数十人,是为“四五事件”。7日,毛泽东、江青、华国锋一致认为“四五事件”的后台是邓小平,将其撤职。许多南粤原造反派乃趁机再次活动,要求“平反”。韦国清则在广州坐立难安,一度准备联合许世友,以广州军区兵力发动反毛政变\footnote{《杜导正自爆文革后期曾参与政变》}。就在中共内讧又一次激化时,剧变发生。9月9日,毛泽东病死。10月6日,华国锋联合叶剑英等一批中共元老发动“怀仁堂政变”,逮捕江青、张春桥、王洪文、姚文元等文革派“四人帮”。

华国锋和中共元老虽将文革派击败,但可供他们选择的路径已经不多。由于中共已被毛泽东带上亲美路线,他们不可能重建苏式“天堂”,只能继续对西方开放门户。他们虽能在组织上清洗文革派分子,却不敢忤逆美国、不敢否认毛的外交革命,进而无法彻底否定毛泽东。他们只能把一切不满都发泄到林彪和“四人帮”身上,称毛只是晚年犯了“错误”。政变刚一成功,华国锋就发起“揭批查”运动,依靠各帮保党派清洗造反派人员。然而,南粤造反派早已在1968年毁灭殆尽,着实乏人可抓。在广东,中共当局只能逮捕少量在1974、1976年被“平反”的造派,顺便装腔作势地清洗几个最凶残的总派头目。至1983年5月运动结束时,广东仅有772人受到牵连\footnote{《当代广东简史》}。同年11月,中共中央由发起“清理三种人”运动,将4名旗派领袖和1名总派头目定性为“造反起家、帮派思想严重、打砸抢”的“三种人”,永不叙用。各市、县也随之划出一批“三种人”,他们中的大部分自然是旗派\footnote{《广东省、市革委会“群众代表”的升沉轨迹》}。在广西,由于各级官僚系统充斥着“保韦”的联指分子,当局一直没有进行清洗。直到1983年,中共中央才派员赴广西进行“文革处遗”。1985年,广西“文革处遗”结束,仅判处10人死刑、14人死缓、1841人有期徒刑。与被屠杀的数十万人相比,这些数字显得实在太小了。中共广西当局仅抛出了一些替罪羊,绝大多数双手沾满鲜血的联指凶手都得以逍遥法外,甚至有许多人身居高位,直到21世纪初仍在从政。大屠杀的元凶韦国清则于1977年进入中共中央军委,直到1989年病死于北京,没有受到哪怕一点点惩罚。南粤文革史的最后一页,就这样黯淡地翻过了。对中共来说,被屠杀的至少数十万南粤人根本不值一提。

由于华国锋继续亲美,南粤变得更加开放。1977年,在华国锋指示下,中共外贸部开始邀请南粤逃港者回粤办厂,惠州籍港商杨勋欣然应允。杨勋时年不满25岁,是个聪明的青年。1972年,他从家乡下海,用六小时游到香港,此后与先于他逃港的兄长一同在服装厂做工。聪慧的两兄弟辛勤奋斗,仅用两年便开办了自己的旭日服装厂。杨勋与中共谈判后,双方决定由旭日集团投资、派人管理、下订单、提供原材料,中共方面提供厂房、工人,产品返销出口。1978年3月,旭日集团的服装厂在顺德市容奇镇投产\footnote{《“改革开放给了我们报效祖国的机会”》}。大批香港企业随之进军南粤,将珠三角迅速变为工厂林立的来料加工中心。在1979—1996年间,香港主要制造业约80\%以上的工厂和加工工序转移至广东,其中94\%在珠三角,这些工厂很多都是由乡镇企业转型而成的。至1996年,珠三角的港资企业已达6.6万家,从业人员高达500万\footnote{赵弘:《总部经济新论:城市转型升级的新动力》,页84}。1978年10月12日,华国锋又批准中共下属的香港招商局在毗邻港澳之地建立“出口基地”,专门发展对外加工工业。次年7月20日,在招商局本土干部袁庚(宝安人)主持下,宝安县西端面积2.14平方公里的蛇口半岛成为“蛇口工业区”。在被黑暗封锁29年后,南粤终于出现了一座通往世界的小窗口\footnote{《华国锋批准了蛇口工业区的缔造》}。不过,华国锋的政治生命已在此时走向终结。早在1977年7月,邓小平即已在中共元老的呼吁下重入中央,他的支持者胡耀邦、赵紫阳也随之进入中共中央。由于华不愿彻底否定文革,遂与邓爆发激烈冲突。1978年11—12月,中共中央接连召开工作会议和十一届三中全会,以邓为首的中共元老激烈抨击华国锋,迫使其屈服。邓更表示反对毛泽东“以阶级斗争为纲”的政治路线,要将“工作重点”转移至“经济建设”。会后,华国锋淡出高层政治,邓小平、陈云等元老成为中共实际头目,“改革开放”正式拉开序幕。至此,列宁主义官僚完全夺回中共政权。

必须指出的是,无论华国锋还是邓小平皆是中共头目,在本质上没有区别。他们皆视南粤为中共的战利品,视粤人为为中共生产外贸商品、获取贸易利润的生物资源。因此,南粤的开放决不能归功于中共的“改革开放”,而应归功于粤港间两邦的兄弟情谊、南粤与西方国家的伟大友谊与粤人的勤勉奋斗。港人对粤人的血肉情谊,使珠三角与香港在经济上结为一体。粤人的乡镇企业和辛勤工作为港人提供了制造基地。如果没有中共政权的干扰,南粤早已成为与港澳、西方国家紧密相连的伟大国家、成为西方秩序在东亚大陆上的桥头堡。南粤大地上也不会仅有蛇口一个窗口,而是会出现千千万万个蛇口,变得远更繁荣昌盛。

邓小平刚一上台,便开始利用粤人为中共的罪恶战争服务。1975年,越共统一越南,随后倒向中共的敌人苏联,于1978年签订《苏越友好合作条约》。此后,中共与越共边境冲突不断。当时,柬埔寨已被中共走狗红色高棉控制,该政权在短短三到四年时间内便消灭了柬埔寨八分之一的人口。就算在共产政权中,红色高棉的疯狂残暴程度也是独一无人的。与它相比,越共的滔天罪行甚至都不值一提。1978年12月,越军进攻柬埔寨,一路受到热情欢迎,于次年1月攻克金边,推翻红色高棉。邓小平大惊失色,乃于1979年2月12日下令中共军队自桂、滇两方面入侵越南,美其名曰“对越自卫反击战”。由此,邓小平便可一面打击苏联在东南亚的势力,一面向苏联的大敌美国示好。侵桂共军中多有南粤本地兵,他们就这样成了中共的炮灰。17日,中共军大举侵越,越军顽强抵抗,双方均伤亡惨重。3月4日,中共军在付出沉重代价后攻陷越北重镇谅山,已成强弩之末,遂于6日进行总撤退,沿途将越北民生、工矿物资劫掠一空。17日,中共军完全撤出越南,“中越战争”结束,双方的小规模边境冲突则一直持续到1990年。此次战争,越共军伤亡2万余人,中共军伤亡21992人,其中6954人阵亡、14800人负伤、238人被俘\footnote{《解密:1979年中越战争双方伤亡人数触目惊心》}。在越北的群山与密林间,数千南粤子弟消失了,更多人带着伤残回乡,成为废人。失去自由的南粤,又一次交出了沉重的“血税”。

战争刚结束,规模庞大的第三次逃港潮就出现了。早在1974年11月,港英政府即宣布实行“抵垒政策”,规定凡“中国”居民成功偷渡至香港市区(界限街以南)者即可得到香港居民身份。如此宽松的政策使向往自由的南粤人大受鼓舞。一时之间,珠江上满是练习游泳的人群,人们虽喊着种种“革命”口号,实则在练习逃港泳技。1978年,随着粤港往来增多,中共的边境警备趋于松懈。至1979年晚春,因社会上风传港英政府将在女王登基日施行大赦,逃港人士可于三日内向港府申报成为永久居民,逃港潮又一次出现。当时,香港农民年收入可达1.3万元,对岸深圳农民的年收入只有134元,两者相差100倍。怀着对自由与美好生活的向往,来自惠阳、东莞、宝安的7万多人于1979年5月6日汇成数十条洪流涌向深圳,瞬间吞没了两个边防哨所。逃亡者既有城乡居民,也有受困于农村的知青,甚至还有少量粤籍共军士兵。所有人都怀着一个目的,那便是逃出黑暗。在这一天,有3万多人成功冲过边境,奔向自由。其余人有的死在海中,有的被军警抓捕。第二天,在距香港20公里的海面上,竟漂浮着数百具尸体。中共广东当局随即展开疯狂搜捕,将有逃港倾向者全部“收容”遣返。仅在广州,就有5万人被捕,全广东被捕者竟达30万,以珠三角尤多。尽管如此,人们仍未放弃对自由的向往。至当年底,已有75817人成功越境。在1980年1—8月间,又有36673人成功逃港\footnote{《1979年逃港30万人次 建特区后几天人群消失》}。面对规模空前的第三次逃港潮,邓小平、叶剑英等中共元老深感震惊。如果粤人持续逃港,中共在南粤的统治就要崩溃了。无奈之下,中共只得让步,于1980年8月26日公布《广东省经济特区条例》,宣布在深圳、珠海、汕头设置“经济特区”。到1988年4月13日,中共又将海南自广东分出,单独设省,成立“海南经济特区”。所谓的“经济特区”并非简单的出口加工区。“经济特区”干部拥有一定自主权,可以因地制宜地引进西方资金、技术、人才,实行“有利于经济发展”的法规和政策\footnote{《当代广东简史》}。它更类似欧洲史上的“特许状”城市,拥有一定自治权,但在本质上仍是为中共当局服务的。深圳、珠海分别毗邻香港、澳门,汕头、海南则与众多海外粤侨有紧密联系。中共希望以此榨取更多资金,并通过给予有限经济自由遏制粤人逃港潮。同年10月23日,因过多劳工涌入使香港人满为患,港英政府中止“抵垒政策”,对逃港者改行冷酷的“即捕即解政策”,规定以24—26日为最后的宽限期。通过收看香港电视频道得知这一消息的人们纷纷离家,开始最后一搏。三天之内,数以千计逃港者聚集在位于金钟道的港府办事处,日夜不停地领取香港身份证。26日,香港的大门关闭,整个南粤为之叹息\footnote{易之临:《世纪末风情——香港文化写真》,页88}。象征着南粤自由的逃港大潮,从此走入历史。但此后,南粤人仍然向往着自由,零星的偷渡逃港从来没有间断过。

1980年10月,经胡耀邦倡议,中共决定停止“上山下乡”,各邦知青纷纷返城。1982年12月,中共又宣布在农村建立乡级行政区,人民公社制度宣告解体。因大批青年返城、农民进城务工,南粤及东亚各邦人口流动性迅速增强。又因中共一以贯之地限制粤人建立维持秩序的社会组织,南粤各地治安状况随之恶化。在1980年代初,南粤各地刑事案件激增,社会秩序趋于紊乱。做为老牌列宁主义官僚,邓小平想到的解决办法便是大开杀戒。1983年8月25日,中共中央发起“严打”运动,决定“在三年内组织三个战役”,“杀掉一批有严重罪行、不杀不足以平民愤的犯罪分子”,恐怖的腥风血雨随之笼罩南粤和整个东亚大陆。至1984年10月,“严打”的“第一战役”结束,东亚各邦共有102.7万人被捕,其中86.1万人被判刑,包括2.4万例死刑。此后的“第二战役”、“第三战役”规模虽不如第一次大,但仍有数十万人被牵连。在南粤,“严打”一直持续到1987年,广东有7.4万人被捕,其中7.2万人被判刑。广西则有110903人被捕,其中99823人被中共当局“起诉”\footnote{侯敏:《中国检察年鉴》,页151}。被判处死刑者中虽有一些最有应得的刑事罪犯,但有相当多人都是无辜平民。史载:

\begin{quote}

当时刑法里的流氓罪最高是可判死刑的,(在广东,)有的乱搞男女关系的人就是按流氓罪从重被判了死刑。还有的只抢了一点点东西甚至只是一顶军帽就被枪毙了\footnote{引自《1983年广东“严打”:有人因抢一顶军帽被枪毙》}。

\end{quote}

偷窃一顶军帽便要处死,私生活稍不检点便被杀害。中共的冷血残暴,着实令人发指。在残酷的“严打”运动中,南粤有数以万计的无辜者被中共投入监狱、数以千计的无辜者惨遭杀害。这是中共对南粤欠下的又一笔血债。

1986年底,东亚各邦压抑已久的怒火爆发了。是年12月上旬,江淮合肥的4000多名学生走上街头,高呼“要求进行民主选举”。接着,北京、湖北、吴越、满洲相继爆发学潮,数万名高校学生罢课上街,展开声势浩大的游行。1987年1月1日,中共当局发动镇压,逮捕数十名北京学生,各邦学潮陆续平息,这便是“八九民运”的前奏“八六学潮”。事后,邓小平、陈云等中共元老认为学潮是奉行西化路线的总书记胡耀邦纵容出来的,称其“放任了资产阶级自由化”,对其大肆批判,命其辞职,代之以赵紫阳。同时,《人民日报》发表社论,猛烈抨击所谓的“资产阶级自由化”。此后,胡耀邦虽仍列席中共中央政治局,但已沦为闲人。1989年4月15日,胡耀邦病死。北京学生与民众随即以悼念胡耀邦为名涌上天安门广场,发起规模空前的示威,要求中共当局解决通货膨胀、处理失业、官员贪腐问题、开放新闻、结社自由,规模浩大的“八九民运”爆发。到5月中旬,聚集在广场上的民众已超过百万,学生发起绝食活动,促使东亚各邦400多个城市爆发大规模示威。当时,苏共总书记戈尔巴乔夫正访问北京,与中共修复关系。为报导此事的西方记者云集北京,却意外发现了更激动人心的一幕,纷纷用惊讶的语调报导着东亚的民主浪潮。在南粤各城市,大规模游行开始了,其中以广州、深圳人数尤多。各地学生、市民纷纷头缠白布涌上街头,高呼口号要求自由,控诉着中共的残暴统治。到5月下旬,广州街头的游行人数已达40万之巨。在香港,数以万计的民众也聚集街头,支持着各邦民众的抗暴义举。在此关头,中共高层又一次出现内讧。赵紫阳误判形势,认为示威者很可能在西方支持下获胜,遂接连发表同情学生的讲话。以邓小平为首的中共元老和总理李鹏则持强硬立场,决意动武。6月3日夜,30万共军以坦克为先导开入北京城,一路射杀抵抗他们的学生和市民。4日凌晨,共军控制天安门广场,完成武力“清场”。在血腥的屠杀中,近千名学生、市民失去了生命。然而,南粤人仍未放弃抵抗。5日,愤怒的学生和民众自发涌向广州海珠桥,以血肉之躯组成人墙,整整坚持了四天。到8日,因军队即将进城,人潮只得散去。与此同时,在中共的暴力镇压下,南粤及东亚各地的怒潮也纷纷被镇压下去。在巴蜀成都,甚至有数百人因此丧生\footnote{关于“八九民运”中的南粤情况,参见《六四情侣:于世文和陈卫的故事》;杨继绳:《罗征启访谈录之“八九风波”》}。

与镇压相伴的是大规模逮捕。在1989年春夏之交,东亚各邦被中共判刑者至少有1600人,许多人在经简单审讯后便遭处决,大批参与其事的教师、学生、官员则被政治污名化、永不重用。令人感动的是,香港和美英在第一时间伸出了援手。6月下旬,在美国中央情报局和港英政府支持下,香港民主派元老司徒华联合香港浸信会发起“黄雀行动”,对南粤和东亚各邦遭通缉者伸出援手。他们通过黑社会“新义安”组织偷渡,并买通了深圳、珠海的部分武警,使成批流亡者得以进入香港\footnote{关于“黄雀行动”,可参看《六四秘密大逃亡:“黄雀行动”》}。

“八九民运”后,赵紫阳被撤职。中共元老中以陈云为首的老左派对邓小平的“改革开放”发起攻击,起用上海市委书记江泽民为总书记,命其与李鹏执行所谓的“治理整顿”。邓小平随之“靠边站”,于11月8日宣布“退休”。此后三年内,因苏联、东欧共产极权国家相继倒台,老左派深为恐惧,命江奉行“稳定压倒一切”的政策。不过,中共高层的内讧并未对南粤经济产生太大影响。早在1985年,年事日高的叶剑英,为避免其死后广东的经济开放被老左派破坏,已安排其子叶选平出任省长,并以亲信林若等人襄赞、辅佐,使广东得以抵住老左派冲击。到1991年5月叶选平离任时,广东已成为“改革开放”的标志。1992年1—2月,走投无路的邓做出最后一搏,以“平民之身”进行“南巡”入粤,先后走访了广州、深圳、珠海,公开表示“市场经济不等于资本主义,社会主义也有市场”。江泽民见风使舵,立即投邓。在同年10月12—18日的中共十四大上,邓派击败老左派,大获全胜,提出建设所谓的“社会主义市场经济”。然而,一个族群若要实行真正的市场经济,其前提是有充分的自由。若一个族群缺乏政治上的自立,便必然丧失经济自主能力。若这一族群的文明程度、经济水平远高于身边的异族,而又不得不与他们同处于大一统暴政之下,便势必在市场经济带来的异族流动人口大潮中承受严重冲击。不幸的是,这一切都在南粤发生了。

\section{南粤民族意识的大觉醒:1992年—今}

1992年后,中共政权内部相对稳定,一直坚持“改革开放”政策。1993年10月,邓小平最后一次在公共场合出现,此后便不再露面。1997年2月邓小平病死后,江泽民平稳地维持着政局,并于2002年11月将总书记之位交给胡锦涛。江、胡是中共建政后培养的技术官僚,他们缺乏老列宁主义者的狠毒残忍,亦没有毛泽东的天马行空,因而对邓小平“萧随曹规”。直到老红卫兵出身的习近平于2012年11月15日上台,情况才开始有所变化。在江、胡时代,中共当局一直比较平稳地维持着“改革开放”政策。到目前为止,习近平亦未放弃“改革开放”。在此期间,南粤似乎得到了前所未有的发展。珠三角成为世界上工厂最密集的区域之一,到2001年,香港企业已在广东雇佣了超过1000万名制造业工人,其中位于珠三角的就超过800万。与此同时,数以百计的专业城镇如雨后春笋般兴起。比如,广州花都区的狮岭镇专门从事皮具制造业,其18万人口中竟有近12万从业,年产皮具4亿只,包揽了许多国际品牌的加工。更令人瞠目结舌的,则是数座巨型城市的兴起。东莞本是个有100多万人口的中等城市,但到2000年其人口已膨胀到645万,2010年更达到822万。深圳本是个仅有数万人口的边境市镇,但到2014年,其常住人口已高达1077.89万。至于南粤的心脏广州,在2015年已拥有令人咋舌的1350万常住人口。除港资外,侨资、台资、日资、韩资及西方各国的资本纷纷涌入南粤,南粤本土企业也大批出现,其规模使人惊叹。例如,台湾富士康集团在深圳龙华的厂区在2013年竟有27万人口,足可与一座城市相提并论\footnote{以上数据见于各种公开新闻,此不赘注}。因珠三角成为全球制造业中心之一,遂被称为“世界工厂”。

然而,此种发展模式是极度畸形的。首先,便是大量异族人口的涌入。如果1980年代和1990年代前期进入南粤的异族人口尚能被南粤同化,那么在这之后,随着异族人口在许多地区占据了相对多数,他们就不再愿意学习粤语了。对此,成长在深圳的笔者即有深刻认识。在笔者的幼儿园、小学时代(2002年之前),同班同学虽有相当数量的异族,但绝大部分人仍愿意说粤语、熟悉南粤文化。但到笔者上中学后,会说粤语的同学就越来越少了。据中共于2010年公布的数据,广东户籍人口为8500万,常住人口则达1.05亿。而就算在户籍人口之中,也有相当数量并非广东人。在数量惊人的外来人口中,除少数来自广西外,绝大部分都来自与南粤语言、文化、风俗差异极大的岭北各邦。除了少量归化者外,他们大都对南粤文化没有丝毫兴趣。以深圳为例,该市上千万人口中,竟有超过七成是异族,其中湘人至少超过100万,楚人、蜀人、赣人、吴越人、满洲人、河南人也各有数十万之多。再以海南三亚为例,该城目前的69万人口中竟有四成是异族,其中大部分来自满洲\footnote{《深圳千万常住人口籍贯排名:湖南湖北广西四川江西河南》}。南粤人彬彬有礼,尊重自发秩序,视自己的语言、文化为生命。但这些异族横行于南粤城市街头,毫不尊重南粤的固有风俗,将南粤的传统社会秩序破坏得体无完肤。事实上,南粤的大量本土、侨资、外资企业不但活跃了南粤的市场与社会,也为岭北落后各邦提供了数量众多的就业岗位,使之免于自然与人为制造的匮乏。虽然有些粤人企业家确实存在克扣工资、打骂劳工,乃至与中共当局勾结镇压劳工合理抗争等可耻行为,但南粤毕竟给岭北劳工提供了前所未有的机遇。而相当数量的岭北劳工受囿于自身的德性、知识结构以及相对封闭的生活、交际圈,使他们常以“中国”所谓的“悠久”历史与“伟大”文化为荣,流于虚无,与务实的南粤本地文化格格不入,甚至对南粤文明横加攻击,将对一部分粤人不良企业家的不满上升为对粤人的民族仇恨,使粤人深感愤怒。除劳工外,很多来自岭北的精英虽受过高等教育,却对南粤没有丝毫认同,仅将南粤当做“中国”的一部分,宣扬大一统思想,以其表面上得体的举止蒙骗了许多粤人。在深圳、东莞等南粤城市中,粤人甚至连一碗正宗云吞面都很难吃到了。在随处可见的湘、蜀、吴越、满洲餐馆之间,相信每一个粤人都会发问:这里还是我深爱的南粤吗?在各种重辣、重咸、重甜的异族料理中,甚至连口味典雅的南粤料理都已难寻!

更为致命的,则是中共政权的“推普”政策。自19世纪发明民族以来,粤语便是南粤各族群神圣的民族共同语。南粤人视之如生命,亦视之为自外于岭北的重要身份标志。自1967年香港TVB开播以来,香港电视节目便成了几乎所有南粤人深刻的童年记忆。不单广府人收看它,客家、潮汕人以及一部分愿意归化的异族也收看它。今天,南粤各地的80后、90后不管母语如何、族群如何,大都能说一口粤语,这在很大程度上便是拜香港大众传媒所赐。与粗俗、无趣的岭北电视节目相比,香港的电视节目是那样有趣、那样给人以启发。通过大河剧般壮阔的《大地恩情》,我们了解到了南粤人在民初大时代中的创业维艰;通过一部脍炙人口的武侠剧,我们了解到了仁、义、礼等最基本的东亚传统伦理;通过一部部粤译海外电影、动漫,我们了解到了西方和日本文明的高贵、典雅与树大根深。我们还有什么理由去收看CCTV、湖南卫视、浙江卫视之流呢?

既然粤语是南粤的灵魂,那么中共就要想尽办法对其进行限制、围剿,从而彻底消解南粤人的民族意识,这着实是万分阴毒的釜底抽薪之计。早在1956年,中共当局就制定了“以北京语音为标准音,以北方话为基础方言”的所谓“普通话”,以之为全“中国”的共同语。此后数十年间,虽然南粤及东亚各邦的知识分子大都能讲“普通话”,但他们仍以母语为自己的第一语言。南粤和东亚各邦百姓或许有不少人会说一些“普通话”,但他们在大多数时候仍讲着母语。然而,在2000年之后,中共的语言政策步步收紧。2005年10月,中共广电总局要求所有电视剧必须“讲普通话”。在此前后,一则“讲普通话,做文明人”的标语出现在南粤各中、小学。由于南粤的学校中涌入了大量不会粤语的岭北教师,他们自然不可能和孩子们用粤语交流。在他们影响下,南粤的小朋友竟然变得不爱说自己神圣的母语,而是用“文明”的“普通话”与家人交流。许多讲了一辈子母语的老人悲凉地发现,自己的孙辈竟然已经变成满开北语的“捞头”。

南粤学校中的“推普”过程绝不是“温情脉脉”的,它充满了侮辱性与强制性。只消看看“讲普通话,做文明人”的标语,我们便会明白它对南粤人进行着怎样的侮辱。2015年9月,一则出现于网络的留言揭露了广州第四十一中河南籍校长牛某的暴力“推普”行径。有关此事的新闻虽被中共迅速封杀,但我们仍能通过网上的只言片语了解到它是何等令人发指:

\begin{quote}

新学期已经开始,有唔少41中既同学反应宜家个牛X校长简直湿9,湿湿9,距利用校长私权,唔比老师上课用广东话就算,仲要要求老师一进校就用普通话交谈,各位同学既中午既粤语广播更系1cut再cut。粤语歌唔比唱,同学仔唔准系学校讲粤语,仲要求同学仔翻屋企都用普通话,根本无中生有。唔好以为距推普有热情,其实距系因为自己一句听唔明广东话,所以憎恨粤语。企图抹杀粤语文化,误人子弟,暴力推普。就系因为有d感既人渣校长,凭一己之力就可以搞到学校里面几千人立立乱,有不少高三同学完全不习惯,导致学习情绪低下。语言歧视\footnote{网帖《广州41中禁粤 校长要求系屋企学校都唔比用粤语》}。

\end{quote}

这则用不规范粤文书写的留言,足以表达该校学生在逆境中的愤怒之情。他们毕竟是高中生,已经拥有独立的自我意识。对于自我意识尚未发育成熟的小学生来说,这种“推普”是否还能激起如此激烈的反抗情绪呢?在历史上,南粤人曾为自由和尊严一次次挺身而战。然而,还没有哪个岭北政权是连南粤人的语言都要消灭的。现在,南粤人已经要为自己的语言而战了。

除“推普”外,中共还在其官方传媒上对南粤人进行百般丑化、羞辱。在1980、1990年代,香港流行文化曾强势输入东亚各邦,使粤语在岭北风靡一时。为与南粤文化对抗,中共当局便竭尽所能拼命贬低、矮化南粤文化。在CCTV粗陋低俗的“春节晚会”中,有几个粤人形象不是“唯利是图”、“贪婪自私”的呢?而这些粤人形象,又有几个是由粤人扮演的呢?一切尽在不言中。除此之外,一批鼓吹大一统的知识分子也动辄贬低南粤悠久的历史文化,称南粤“没有历史”,是“文化沙漠”。在最近二十余年来,这样的攻击实在太多,笔者已经无力一一列举。

尽管南粤文明已步入如此危险的境地,但在近二十余年来,南粤人仍创造了可喜的文明成就。首先,大批本土企业利用中共的“改革开放”政策强势崛起,广东成为“中国”最富庶的“省份”。据中共于2016年3月公布的数据,在2009—2016年间,广东上市公司数量连续八年位居“全国”之首。自1978年中共开始统计“各省”GDP以来,广东在1989—2015年间年nin 位居第一。需知,自1980年代起,中共当局便不断借手中权势安插岭北籍官员广东地方人事,从相对富裕的广东沿海等地,榨取资源,反哺经济改革滞后、官僚气息浓厚、意识保守的岭北内陆,以维持其利益结构与统治基础,并养活数量日增的官僚群体。而中共中央政府的这洋做法,也或多或少地遭到了中共广东当局中本土干部的有限抵制。自1990年代中后期起,在内陆“各省”虚高“GDP”增速以彰“经济大发展”之日,广东经济数据却刻意“缩水”,以免岭北的过分榨取。因此,说广东的经济实力居“各省”之首是毫无疑问的。广东的富庶,也为经济相对落后的亲邦广西提供了大量工作机会。据中共于2010年公布的数据,广西户籍人口为5159万、常住人口为4603万。也就是说,有相当数量的桂人流出广西,前往八桂之外工作。其中,有相当数量的人流向了广东。做为同属南粤、血肉相连的亲邦,直到今天,粤、桂之间维系着紧密联系。

其次,便是南粤传统宗族的复兴。自16世纪南粤华夏化以来,儒化宗族便是南粤最为坚实的传统小共同体。在岭北宗族已步入消亡之际,是百越后裔南粤人成为了华夏的嫡传。然而,自1950年代中共土改后,南粤宗族组织已趋向消亡,祠堂和传统庙宇也被拆毁殆尽。1960年代后期,随着南粤乡镇企业的快速兴起,以血缘为纽带的南粤宗族组织开始复兴。1982年后,随着人民公社解体,南粤民众迅速用传统方式自己组织起来。到1990年代,中共的乡、村两级政权已成为纯粹为自身利益而向民众索取的贪婪机构,不再负有组织民众的责任。在此情况下,南粤民众便重建祠堂、重修家谱,使宗族组织遍地复活。今天,我们依然能在几乎每个南粤村庄看到祠堂,便是南粤民众在近二十年来不断努力的结果\footnote{孙秀林:《华南的村治与宗族——一个功能主义的分析路径》}。在如此短的时间内,南粤的传统共同体便全面复活了,这不但是个伟大的奇迹,也是南粤强大自发秩序的最有利证明。此外,传统庙宇的全面复兴也十分引人注目。1995年11月1日,毁于文革初期的蛇口赤湾天后宫完成重建、对外开放,当时的场景足以使人感动落泪:

\begin{quote}

官方主持的剪彩仪式9:00开始,照例是奏乐(奏了可能不到三十秒的一段不伦不类的音乐)、鸣炮、介绍嘉宾、讲话(区长和国家文物局副局长两个讲话)剪彩、嘉宾进庙参观。就在嘉宾进庙不久,正殿的大门被一群妇女冲开,她们冲进来后,迅速地将贴在天后神像上的红纸撕下来,如获至宝地抱在怀里……
那些来拜神的人们与神的沟通是相当简单,也相当原始的,在神面前唱歌、跳舞,将衣服在与神有关的象征物上擦几下、将贴在神像上的红纸带回家中、在开光的仪式上将自己的神像摆上来,或摆上一些自己带来的神符在供桌上,等等,都是她们与神沟通的办法\footnote{刘志伟:《“官方”庙宇的意义转变——赤湾天后庙碑铭解析》,载郑振满主编:《碑铭研究》第二辑,页453—454}。

\end{quote}

就算经历列宁主义如此疯狂的摧残和毁灭,南粤的本土信仰仍深植于粤人心中。在这些妇女身上,南粤人的虔诚、淳朴、敬畏表现得一览无余。只要看看她们,我们便可知道南粤的宗族、信仰何以能在这么短的时间内复兴了。在无边黑暗中,不管强权如何摧残,南粤文明始终也没有灭亡,它存货在每一个南粤人的身上、心里。

1997年7月1日,香港沦入中共之手。1999年12月20日,澳门遭遇同样的厄运。随着港澳落入中共魔掌,西方与南粤文明结合而成的两座伟大城邦已丧失自由。虽然中共在表面上奉行邓小平于1982年提出的“一国两制”政策,仍维持港澳的原有政治、立法、司法、社会结构,但因中共当局对港澳的渗透日渐露骨,两个与南粤血肉相连的盟邦已然沦陷。2000年以后,南粤文明已步入危亡关头。一批有识之士的民族意识猛然觉醒,开始发出粤独的呐喊。

2006年,一个名叫“粤独嘅呐喊”的网站悄然诞生,开始大量发布阐释粤独理念、分析南粤历史、语言、文化的文章,对南粤各族群语言研究尤深。该网站的最新一篇文章发布于2016年7月16日,目前仍在积极活动。2007年,一个名为“大粤民国临时政府”的组织悄然诞生。在他们的网络阵地“大粤独立建国论坛”上,有两个名为“立法司法”和“国家行政机关”的版块,下设“大粤民国国会”、“大粤民国宪法法院”、“大粤民国国土安全部”、“大粤民国外交部”、“大粤民国教育部”、“大粤民国商务部”、“大粤民国能源及交通部”、“大粤民国民政及劳工部”、“大粤民国财政部”、“大粤民国农业及自然保育部”、“大粤民国司法部”、“大粤民国卫生部”、“大粤民国文化及康体部”、“大粤民国地政及住房部”等子版块。他们直斥岭北大一统强权为“支捞政府”,提出“大粤民国”乃“粤人嘅真正祖国”。2008年,他们在网上公布了一面目前流传甚广的南粤国旗,那便是著名的绿、咖啡、蓝三色木棉旗。他们这样解释旗帜的含义:

\begin{quote}
	
啡、绿、蓝,系我呢套方案嘅三基色,而木棉花同铜鼓,则系重要嘅象征物。
啡色,代表我哋系粤人,我大粤诸民族嘅直系先祖,系啡色肤色兼且带有矮黑人种血统嘅古南粤人,啡色意涵喻示我大粤诸民族嘅古南粤源起。
绿色,代表自由、和平、生机。绿色,经已系自由嘅代表色,我粤觉醒,大粤独立,系我大粤民国立国嘅核心理念;同睦邻和平而有尊严噉共处,系我粤国民之所愿;我粤地处东南亚北缘,北热带带嚟嘅阳光同雨水,令我粤绿意泱然,生机勃发。
蓝色,代表民主、海洋、睿智、永恒。粤内世居诸族嘅平等同民主,系我粤持久稳固嘅基石;我大粤族群同文明,有着显明嘅海洋色调;我粤国民,理性而智慧;我粤国运,恒长而久远。
大粤国旗,将依循现时世界上高上中下三色等分嘅国旗式样。啡色居中,代表我哋粤人将永远世居喺我粤国土,绿色喺上边,代表我哋粤人人人享有自由嘅空间,蓝色喺下低,代表我粤将建基喺民主嘅基石之上,正中间绽放嘅深红色木棉花,色彩浓重,系我粤人所认同嘅代表标志。

\end{quote}

这面旗帜完全展现了南粤文明的真精神,使人醍醐灌顶。该组织似乎不甚活跃,在2016年仅在论坛上发布三篇文章。但他们已经初步提出了未来南粤独立建国的行政、立法机构框架,设计了国旗,已经有了政府的雏形。他们的一些观点虽然失之偏激,真正执行的话必给南粤带来巨大灾难,但他们追求南粤自由的壮举和对南粤未来的一系列规划仍使人击节赞叹\footnote{他们不被笔者不认同的主张包括:视曾与南粤有铜鼓文明的滇、黔为“西粤”,试图将其攻占;不顾滇、黔、蜀已在近几百年来形成新民族共同体的事实,一味仇视西南官话人群,试图将其赶尽杀绝,并不承认巴蜀自为一国,甚至提出驱逐南粤桂柳民系的极端偏激思想,不顾桂柳人曾为南粤自由做出的伟大贡献;不顾西方秩序及各国之间平等交往的原则,反对国际仲裁,偏激地视南海诸岛为大粤民国固有领土。若他们的想法成真,南粤将虚掷资源于本土之外,重蹈南汉覆辙,无数南粤人的生命也将白白浪费在异域;南粤也将成为侵犯他族自由、挑战西方秩序、令国际社会厌恶的国家,终将被国际社会抛弃,从而背叛历史上无数先贤对南粤自由的追求。他们的相关论述,见其论坛。}。此后,2009年,一个名叫“大粤民进社”的网站出现,其最新一篇文章发布于2016年7月18日。在该网站首页上,振聋发聩地写有如下宣言:

\begin{quote}
文明冲突决战于五岭,蓝海必胜黄土。广东独立,广西独立,海南独立,香港独立,澳门独立,共建联邦。
\end{quote}

对于上述粤独组织的种种具体主张,笔者虽有不尽同意之处。不同组织间的主张,亦有互相矛盾之处。然而,他们对南粤自由与尊严的追求是完全一致的,他们热爱南粤、守护南粤的心是完全一样的。现在,他们虽然仍在网络上活动。但借助网络这一新媒体,他们必能将种子洒遍南粤大地,在未来的惊涛骇浪中创造奇迹。那面南粤国旗的广泛传播,便已证明了这一点。在短短时间内,粤独力量便已发展如此规模、发展得如此复杂,这雄辩地说明,我南粤直到今天仍有极其强劲的生命力!

事实上,南粤人已经开始行动了。2010年7月5日,中共广州政协提案委副主任纪可光向广州市政府提交建议,希望将主要使用粤语的广州电视台综合频道或新闻频道改为主要使用“普通话”播音。纪可光系满洲南下后代,他的母亲虽是南粤人,但他毫无爱粤之心,所思唯有为虎作伥。消息传出,广州市民愤怒了。11日,一群广州市民自发来到人民公园,手举“广州话起锚”标语,高唱粤语歌曲,伟大的“撑粤语行动”由此开始。面对粤人的反抗,中共立即摆出一副凶残面目。16日,广东省委书记汪洋公开宣称,他要通过推广“普通话”的方式对粤人进行“教化”。南粤人的怒火被彻底点燃了。25日下午5时,首次大游行爆发,上万人聚集在江南西地铁站A出口附近,展开声势浩大的示威,齐声高呼“掉哪妈,顶硬上!”。8月1日,又有数千民众聚集在人民公园支持粤语,其中包括200—300名赶来声援的香港人。遭中共警察粗暴“清场”后,他们并未被吓倒,而是转赴北京路、“烈士”陵园继续示威。中共警察遂再次动武,捉走其中7人,包括3名港人。次日,中共当局又以子虚乌有的“带头滋事、堵塞交通、扰乱公共场所秩序”罪名捉走3人。民众的游行虽然被镇压了,但中共广东当局亦十分惊恐。4日,汪洋无奈地表态,称当局并无“废粤”打算。经过英勇的抗争,南粤人赢得了“撑粤语行动”的伟大胜利。然而,这只是一次战斗的胜利,因为中共当局仍在暴力“推普”。在南粤获得真正的自由前,保卫粤语的战斗绝不会停止。

2011年,在粤东潮州,壮烈的战斗再度爆发。是年6月6日晚,在潮安县古巷镇,数千名向当地政府讨薪的巴蜀民工发起骚乱,四处追砍当地居民。此事起因虽为与政府勾结的欠薪老板虐打民工,但无论如何,这些异族都毫无理由对普通南粤人进行民族仇杀。事发次日,古巷居民为保卫家园勇敢地组织起来。他们组成自卫队,头戴黄色安全帽、身缠红布条,手持棍棒、刀具沿街巡逻,使异族暴民不敢进犯。8日,在中共潮州当局的控制下,骚乱平息。据中共公布的数据,骚乱中有18人受伤、1辆汽车被焚、17辆汽车受损。包庇欠薪老板的中共当局固然是制造此事的真正元凶,巴蜀民工诉求中的合理部分应当得到尊重。但当他们转向攻击南粤人时,事件就已转变为南粤人的民族自卫。在这场战斗中,勇敢的古巷人没有辱没潮汕人和南粤祖先的威名。他们守住了家园,保住了一方乡土的安宁。

同年9月21日,陆丰市乌坎村爆发新的抗争。是日,因村委会成员私自变卖存在土地,3000—4000名村民聚集于中共陆丰市政府大楼与派出所前示威,但未得到任何回应。其后,村民在村中组织起“乌坎村村民临时代表理事会”,多次打退前来进犯的中共警察。12月9日,陆丰市政府将乌坎民众定性为与“境外的某些机构、势力和媒体”相“勾结”者,捉走村民薛锦波,于三日后将其活活打死。被激怒的村民决定反抗,于2012年2月1日开始自行选举村委会。在村民的压力下,中共陆丰当局暂时让步,于16日将薛锦波的尸体归还村民,并交出90万元“抚恤金”和殓葬费。3月3—4日,村民将14名起义领袖选为村委会成员与村民小组代表,做出长期抗争的决议。此后,中共当局长期不归还土地,反而于2016年6月17日将村委会主任林祖恋逮捕。9月8日,林被判处有期徒刑3年1个月。13日清晨,中共警察突袭乌坎村,与村民爆发展开打斗,将多名村民代表带走。直到现在,乌坎村英勇的斗争仍在持续,牵动着每个南粤人的心。

自2012年11月习近平上台以来,江、胡较稳健的政治路线已日渐式微。做为老红卫兵出身的红二代,习自小受到列宁主义教育,不相信中共能与西方世界和平共处。2012年12月,习近平发动“反腐败运动”,对军队、官僚系统展开大清洗。到目前为止,已有至少10万名官员“落马”,其中包括大批被称为“客家帮”的南粤本土干部,如于2014年6月“落马”的广州市委书记万庆良(五华人)、于2016年6月自杀的广东省委副秘书长的刘小华(兴宁人)等。事实上,有实无名的广东“第四次反地方主义运动”。此外,习还大举营造对他的个人崇拜,其力度不但已超过江、胡,更有超越邓小平之势。在逐步确立其独裁统治的同时,习加紧在南海、东海不断挑衅以美国为首的西方秩序,蛮横地向东南亚各国及日本施压。2016年7月12日,联合国通过“南海仲裁案”,认定中共在南海的超常诉求违反《联合国海洋法》,要求其停止在南海活动,以习为首的中共则对此置之不理。一旦习近平完成对军队、官僚系统的清洗,中共将很可能彻底扭转江、胡的稳健路线,在南海实施更疯狂的冒险,从而引起美国及其盟友的严厉报复。即将召开于2017年秋的中共十九大将是习改变中共路线的绝佳节点。在此之后,历史究竟如何发展将无从得知。无论如何,只要中共与西方世界爆发冲突,面朝南海的南粤将沦为修罗场。值此特殊历史关头,南粤人更当自省。自1992年以后,南粤似乎得到了前所未有的发展。然而如前所述,这一发展却是相当畸形的,因为南粤并未获得自由。中共视南粤为榨取资源的富庶之地,对南粤予取予夺。为压制南粤人的反抗意识,中共又动用各种手段疯狂妖魔化南粤。更令人发指的是,在中共的打压和异族人口大潮的冲击下,南粤的语言、习俗都已危在旦夕。可以说,南粤文明已到了生死存亡的关头。现在,在中共疯狂挑战国际秩序的情况下,南粤已来到决断时刻。究竟是沉沦下去,被绑在中共的战车上步入浩劫?还是毅然决断,为南粤赢得完全的自由与尊严?这一问题,将由所有南粤人一同回答。在洪水滔天的前夜,让我们一同回忆南粤伟大先民的故事吧,让我们一同讲述这片热土上发生过的动人事迹吧。只要我们仍有血性,只要我们仍对自由与尊严有所向往,我们便绝不会愧对我们的伟大祖先、不会愧对我们的伟大历史、不会愧对我们的悠久文明,因为无数南粤史上的英雄男女已用行动告诉我们应该如何做了,那便是:

为了南粤的自由,为了南粤的尊严,挺身而战!

