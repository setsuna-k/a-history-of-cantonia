\chapter{无尽黑暗:1950—1972年的南粤}

\section{本土凝结核的毁灭:1950—1958年}

\indent 1949年8月下旬,在铁幕降临南粤前夕,为顺利完成对南粤全境的侵占与掌控,暂时避免不要冲突,中共决定以熟悉广东情况的粤人叶剑英出任新组建的中共华南分局第一书记,统一指挥两广军政与共军侵粤作战,又任命同为粤人的张云逸(文昌人)为第二书记,分管广西党政。而此前一直担任中共在粤地方党组织领导人的方方,则出任第三书记,协助叶剑英,分管广东党政。10月19日,在共军侵占广州的第六天,中共“广东省人民政府”在广州成立,叶剑英出任广东省主席,方方、古大存出任副主席。1950年2月1日,中共“广西省人民政府”在南宁成立,张云逸出任省主席。这样,在中共侵占两广之初,粤、桂的政权便主要落入粤籍干部手中\footnote{《简明广东史》,页822;《广西通史》,页611}。不过,侵粤共军头目则多为岭北人。1949年11月9日、12月9日,中共分别以四野第十五兵团、第十三兵团组建广东、广西军区。广东军区司令邓华(湖湘郴县人)、政委赖传珠(江右赣县人)举止粗俗、满身匪气,与南粤文化格格不入,常与方方、古大存等人发生冲突\footnote{杜襟南:《人世间:陈嘉(杜襟南)日记初叶(1933-1950》(下册),广州:中共广州市委党史研究室编印,2000年11月版,第965页。}。从一开始,中共南粤当局中就产生了内讧的萌芽。

不过,在1950年时,中共侵粤要员之间尚无暇开战。中共华南分局系驻武汉的中共中南局下级党组织。中南局是统管中共两广、江西、湖北、河南党务的机关。在以党统军的中共组织系统中,该局拥有极大权力。自1949年下半年起,为维持“新解放区”党政军系统的庞大开支,中南局便积极推行高额征粮政策,以高额累进制征收农村田赋,规定对人均农业年收入不满150斤稻谷者免征、对地主、富农征收四至五成重赋。然而,在南粤农村,人均年收入不足150斤者只有极少数赤贫者。在征粮过程中,中共基层干部任意羁押、吊打农民,滥加征额,使地主、富农纷纷破产,贫农、中农疲于奔命。1949年底,广西柳江、桂江、浔江、红河水流域发生严重水灾,导致1950年爆发春荒,全桂灾民达200万以上。许多农民无粮可交,却仍被征粮队疯狂逼迫,被饿死、逼死者到处出现。在此情况下,忍无可忍的民众终于揭竿而起,震撼全桂的伟大起义爆发了。当时,新桂系正规军虽已瓦解,但广西的数十万民团尚未解散,光是散布于民间的枪枝就有60万条。各村民团组织以兼任乡村小学校长的村长为小队长,仍保留着完好的组织。此外,全桂尚有3万余名原新桂系军人遍布各地,与民众紧密联系在一起。1950年1月25日,恭城县的4000多名民众在军官钟祖培率领下首揭义旗,一举摧毁中共乡政府13个,打响了大起义的第一枪。这次起义虽被共军迅速镇压,钟祖培也于2月27日被共军俘杀,但人民的怒火已被彻底点燃。在2月,北流、玉林、邕宁、东兰等县民众纷纷响应,先后消灭乡政府30个、区政府5座、解放县城3座。3月3日,十万大山起义爆发。千余起义民众成立“粤桂边反共救国军”,以军官之妻韦秀英为总指挥,迅速席卷了十万大山,兵力膨胀到2万余人。起义军在各地提出“打北佬”、“破仓分粮”等口号,杀死征粮队员、摧毁中共基层政权、夺回公粮,队伍迅速壮大。在1950年2—3月间,仅玉林、梧州、平乐三地区即有超过3000名中共干部和征粮队员毙命。起义军的杀人手段相当残忍,对被俘征粮队员往往以挖眼剖腹、砍去四肢、割掉耳鼻、火油浇身等手段虐杀,有的女俘死前还遭到轮奸\footnote{唐仁郭:《建国初期的广西剿匪斗争》}。这些过激手段固然有违人道,不应完全肯定,其中蹂躏女性的行径更令人极度不齿。但南粤人历来和平有礼、极少屠杀无辜。起义军之所以如此残酷,正是出于对中共政权的极度仇恨。他们并非天生残忍,是列宁主义暴政将他们逼成了这样。

面对民众的怒火,中共广西当局极度恐惧。他们一面将起义军诬称为“土匪”,一面积极组织镇压。1950年3月23—24日,中共广西省委召开高干会议,决定重点镇压桂东南地区起义。至7月底,共军相继镇压大容山、六万大山、十万大山、天堂山等地义军,起义军民被屠杀、俘虏者超过9万人。然而,民众没有被屠刀吓倒。共军的屠杀激起了更炽烈的怒火,更多人加入义军,继续进行英勇的斗争。10月11日,毛泽东致电广西当局,指责其在“剿匪工作”中有“宽大无边”的倾向,要求于次年5月1日前务必消灭“土匪”。随后,毛泽东电令叶剑英及中南局政治部主任陶铸(湖湘祁阳人)火速前往广西主持“剿匪”,张云逸则做出“自我批评”,不久后前往广州“休养”。陶铸是个极度狂热的列宁主义官僚。在他的指挥下,共军于10月中旬发起第二轮镇压。他特别要求:

\begin{quote}

镇压必须严厉,才能彰显力量……凡是“股匪头”与坚决的反革命分子一经捕获,一律处死刑。应该杀的人有多少杀多少,不要为数目所限,但也不许乱杀错杀\footnote{《广西十万大山女匪首韦秀英》}。

\end{quote}

所谓的“不许乱杀错杀”不过是句套话,“有多少杀多少”方为陶铸的真意。共军在南粤的战争机器彻底开动了。在共军的疯狂屠戮下,广西化为人间地狱。至12月25日,邕宁、横县、贵县、宾阳、来宾、迁江、兴业、玉林、博白、北流、容县等地义军都被镇压。30日,巾帼英雄韦秀英及少数亲兵在十万大山中的大禄镇被围。义军将士誓死不降,与共军战斗整夜。至天明时分,共军在轻重机枪、炮击炮的掩护下冲入义军指挥部,发现的只有大批尸体:韦秀英已和全体将士战斗到最后一刻,牺牲在阵地上\footnote{唐仁郭:《建国初期的广西剿匪斗争》}。

共军的屠杀远远没有中止。1950年底,陶铸、叶剑英召集广西省委、军区高干,提出以“剿匪”为“压倒一切的中心任务”。与此同时,疯狂的土地改革也伴随着共军的铁蹄在广西全面铺开。由于地主、富农中多有起义军干部,中共干部可以随心所欲地将他们指为“匪首”杀害,然后将土地分给“贫下中农”。在陶铸的严令下,共军向瑶山、桂南山区进发,又一次展开血腥屠杀。到1951年2月,两地义军都已失败,共军遂于3月攻入桂西北、桂东继续“剿匪”。5月,桂西北、桂东的反抗也渐渐沉寂下来。在陶铸主持“剿匪”的八个月中,遇害、被俘的起义军民多达33万人。随后,中欧共又展开遍及全桂的“清匪反霸”行动。至1952年底,残余的5.5万名起义军民也先后被俘、被杀\footnote{唐仁郭:《建国初期的广西剿匪斗争》}。在中共的疯狂屠杀下,历时三年的广西大起义就这样被残酷地镇压了。据中共在1952年12月公布的数据,共军在三年中动用两个军团、四个军、十七个师又一个团的兵力,共“歼灭”广西“土匪”512917人。陶铸事后声称,其中被杀者有4万人。这不过是他为掩饰暴行做出的谎言,因为据身与其事的中共干部回忆,被杀者至少有10万人,所有民团小队长皆被杀害\footnote{据《凤凰大视野》2012年8月节目《广西剿匪记》}。换言之,在1950—1952年间,中共消灭了全广西的村长及小学校长。到1954年5月,广西土改完成\footnote{何成学:《广西土地改革概述》}。至此,新桂系曾苦心经营十余年的广西社会彻底毁灭。在一片血红中,八桂百姓完全丧失了反抗能力,沦为任由列宁主义者蹂躏的俎上之肉。

在广东,血雨腥风于1950年冬全面降临。是年10月,中共开始在广东进行土改试点工作,由叶剑英、方方分任广东土改委员会正副委员长。叶、方虽是粤人,但他们在屠杀自己的同胞时毫不眨眼。至1951年3月,波及20余县的土改试点工作完成。一份对其中28个乡土改情况的综合统计报告显示,28乡共有地主4400人,其中枪决52人、自杀或keyta 原因致死者32人、判刑27人、逃亡42人,一般地主都进行过“斗争”,杀戮不可谓不重\footnote{陈遐瓒:《试论广东的土地改革》}。然而,中南局仍对广东的土改情况不满意。1951年4月,中南局激烈批评广东当局,称其“对敌不够狠,对群众不够热”。同月,毛泽东将在河南南阳推行激进土改政策的赵紫阳(河南滑县人)调至广东,任华南分局秘书长,并兼任广东土改委员会副主任,实际取代方方,成为广东土改运动负责人,开始推行远更疯狂的土改路线。1951年12月,毛又将大屠夫陶铸调任华南分局第四书记。1952年6月,毛将方方召至北京,对其做出当面批评:


\begin{quote}
你犯了两个错误。一是土改右倾,二是干部问题犯“地方主义”错误……(广东土改)迷失方向,我要打快板,方方打慢板\footnote{转引自朱执中:《村村见血:广东土改真相》}。
\end{quote}

以毛泽东为首的中共高层之所以如此急于在粤进行激进土改,与正在进行的韩战有关。1950年10月—1953年7月间,中共政权以苏联代理人的身份出兵朝鲜半岛,与美军正面交手。尽管得了苏联的各种援助与承诺,但参战后庞大的军费开支让中共财政上相形见绌,中共高层非常希望迅速在新占领地区推行土改,汲取资源,以维持在朝鲜的代理人战争。毛泽东发话后,揣摩上意的陶铸乃于7月对叶剑英进行点名批判。此后,叶剑英、方方彻底“靠边站”,广东土改大权落入陶铸、赵紫阳之手。在各地农村,疯狂的土改“复查”运动开始了。据古大存就粤东“复查”情况所写的报告,“复查”情形触目惊心:

\begin{quote}
复查一开始,农民就向地主追余粮。如东阁村先召开了党、团、妇、民兵大会,交代政府政策,说明斗争方法和目的,并组织了复查委员会。但一开始斗争时,群众控制不住,纷纷采用吊打办法,群众认为地主狡猾“抵吊”,东阁村七八个地主全部给吊打过,地主罗贵昌,三吊三出,拿出三两黄金;地主罗□培,掩藏白银,不肯承认,后给少儿队搜出,农民用篱笆竹打,愈打愈气,罗回家以后死掉了。他在学校念书的12岁儿子,也给少儿队私下拖去活埋,只剩下了头,以后又扒出来。男人打男人,妇女打妇女,一家地主四个人,有三个打得到现在还起不来。同甲乡四村,有六家地主,三家被打了。地主王和隆,欠一百多担谷,全家18口,儿子媳妇都给打了,只有二口没被吊打。地主陈得胜,全家16口,四口被吊打。四村吊得很凶,最初吊指头,然后吊单边,另一边则绑石头,吊得手指折断,但是一钱也没有吊出。东阁村农民过去受了地主的气,这次打得气喘不过来还要打,认为“打死地主,当他睡目”,“过去地主对我们狠,打吊一下算得什么?”有些群众见地主自杀,见死不救,如洪村地主婆跳水没人救,死了\footnote{转引自朱执中:《村村见血:广东土改真相》}。
\end{quote}

在陶铸、赵紫阳的疯狂政策下,死亡人数急剧攀升,连曾在1911年南粤独立战争中做出巨大贡献的民族英雄江孔殷也被斗死。为推进土改速度,两人还对对土改干部进行大清洗,以“整队”为名处决、查办了6515人。这些人大都是南粤本地出身的基层干部。他们虽已卖身投靠中共,但仍或多或少地留有一丁点对乡邦的感情,不愿太过疯狂地杀人,遂因“思想不纯”、有“地方主义”倾向、“出身”不良等问题大祸临头。这便是所谓的“第一次反地方主义运动”。运动过后,大批来自岭北的“南下干部”填补了空出的官缺。他们对南粤毫无感情,可以毫无心理负担地大砍大杀。到1953年4月,广东土改结束。据陶铸在1952年10月20日所作的总结报告,广东各地都发生了地主自杀事件,死者估计在1.6—1.7万人之间\footnote{陈遐瓒:《试论广东的土地改革》}。以陶铸在广西屠杀义军时缩报数字的表现来看,这一数字极有可能也是严重缩小的。至于这些死者中有多少是自杀的、有多少是被活活打死的,就更无法考究了。此外,在1950—1953年的“镇压反革命”运动中,广东又有3万名所谓的“土匪、恶霸、特务、反动党团骨干、反动会道门头子”被杀。这些人固然有不少是曾残民以逞的国民党分子,但有许多人仅仅是为了凑足杀人指标被随意滥杀的。因为根据中共中央规定,各地必须杀足占总人口数0.5—1‰的“反革命分子”。在1952年1月,因杀人数难以达到指标,中共华南分局甚至赤裸裸地地叫嚣“还有一万个人头。\footnote{叶曙明:《1951年广东镇反实录》}”如此看来,仅仅作最保守的估计,在1950—1953年间,广东就有近5万人被中共杀死,其中有极大比例的无辜者。

1953年,踏着无数粤人的尸骨,陶铸得意洋洋地出任中共华南分局书记。到1955年,因中南局、华南分局取消,他又就任广东省委书籍、广州省省长、广东军区第一政委,赵紫阳则作为他的副手担任省委副书记兼省委秘书长。至此,陶、赵这两个屠夫一手掌握了南粤大权。在陶铸主持下,中共华南分局自1954年起开始对粤、桂私营工商业进行所谓的“社会主义改造”。华南分局原本计划用八年时间完成“改造”。但到1955年底,因“改造”速度过慢,粤、桂中共当局遂紧跟中共中央步伐,决定在迅速完成“改造”。至1956年1月,“社会主义改造”在南粤完成,广东的所有行业都已实行“公私合营”体制,广西各主要城市的所有行业也确立了这一体制。在运动的最后阶段,中共当局完全采取“一刀切”的粗暴做法,将普通小商贩、小手工业者皆当做“资本主义经济”处理。在广州,数以万计的商贩被以“上山下乡”的名义遣送至农村。在惠阳,商户减少了四分之一。在梧州,原有的2345个商业网点减至1530个。大批人口随之失业,城市居民的日常生活也变得极其不便。在“公私合营”的招牌下,所有企业主都被剥夺了经营权,只能挂名闲职领取有限的定息,他们的产业已被中共干部夺走,还有许多人被直接打为“五毒”失去了一切。而就算这有限的定息也仅有十年期限。届时,他们便会失去一切,“公私合营”制度也会取消\footnote{关于南粤的社会主义改造,参见《当代广东简史》;张萌:《广西资本主义工商业的社会主义改造研究》}。自19世纪后期文明开化以来,还从未有任何政权对南粤企业过如此彻底的劫掠。相比之下,连丧心病狂的国民党都已显得太过温和。

1956年,南粤农村的土豪、士绅已几乎被彻底消灭了、南粤城市的商人已几乎被彻底抢光了,敢于反抗中共政权的南粤人已几乎被彻底杀光了。然而,以古大存、冯白驹为首的一大批南粤“本地干部”仍掌握着相当数量的基层政权。对这些与南粤乡土仍有微弱联系的干部,陶铸和赵紫阳自然不会放过。1956年1月28日,因长期受南下干部打压,未能在“革命成功”后获得相应的待遇,部分前琼崖纵队军人包围了中共临高县委、县政府,强迫县委书记、县长等接受他们提出的要求,是为“临高事件”。与此同时。与临高县相邻的那大县也发生了同样的情况。“临高事件”发生后,冯白驹受到了陶、赵等人批判,认为事件是海南本地干部“地方主义”情绪严重的表现。而在第一次反“地方主义”中未受牵连,但对陶、赵等执行激进政策不满的古大存则出面为冯白驹讲了话。他认为南下干部在土改、镇反等运动中作风不够“民主”,不了解广东地方情况,执行政策、开展运动过于教条、激进,不符合广东实际情况。他还提出南下干部利用政治运动,党同伐异,排斥地方干部,是“北方人整南方人”\footnote{傅高义著,高申鹏译:《共产主义下的广州:一个省会的规划与政治(1949-1968)》,广州:广东人民出版社,2008年版,第196-201页。}。

古大存的认识也代表多数广东本地干部,甚至部分南下干部的看法。他们不但怀疑中共广东省委的干部政策有宗派倾向,认为其干部政策“对北方人有利,歧视南方人”。且认为中共建国以来,中共高层给予广东的政策与重视不够,将广东视作与西方冷战的前线,非但拨给广东财政预算不足,还将广东大量资源运往北方,以致广东失业率高,工业建设缓慢,副食品供应紧张,发展滞后于满洲、华北等地区。而广东地方干部的这类不满,随着1957年4月“大鸣大放”运动的展开,与党外士人的意见与建议相互呼应,这引起了陶、赵等人尤其是陶铸的高度紧张。他担心古大存会以此为契机,联合在土改、“第一次反地方主义运动”中被整的本地干部展开反击,危及其在南粤的统治。

“大鸣大放”本是中共中央为听取党外人士意见发起的运动。然而,因许多党外人士在运动中攻击中共的一党专政制度,毛泽东、邓小平等人遂视之为“资产阶级右派分子的进攻”,于同年5月15日发动“反右运动”。在邓的亲自主持下,东亚各邦有55万人被划为“右派分子”遭到整肃。“反右”为陶铸清理冯白驹、古大存等本地干部提供了天赐良机。1957年夏天,当获悉山东、浙江、新疆等地亦有人借“反右”运动之机开展“反地方主义运动”、清洗本地干部后,陶铸顺势而动,指责广东本地干部破坏团结,并将冯白驹、古大存打为“地方主义反党集团头子”,于1957年12月、1958年4月先后免去两人一切职务。同时,陶铸又罗织各种罪名疯狂清洗,将一批占据地级领导岗位的本地干部打为四个“地方主义反党小集团”,上万名本地干部受到牵连。此后,广东80\%以上的高级领导岗位为南下干部所据,是为“第二次反地方主义运动”\footnote{David Shambaugh著,徐泽荣译:《赵紫阳的崛起与陷落》,香港:百姓文化事业公司,1990年版,第48-49页。}。1958年2月—6月间,广西当局亦展开同样的运动,将近三分之二的地、县级本地干部定为“地方主义集团”分子打倒,南下干部随之接管了空出来的职位\footnote{《关于对广西反地方主义和反地方民族主义问题的平反决议》}。

冯白驹和古大存本是出卖南粤、投靠中共的粤奸,他们领导的游击队曾为中共侵占南粤立下“汗马功劳”。然而,随着中共的“社会主义革命”愈加疯狂,他们这种人竟然也成了南粤本土势力的代表,成为中共的清洗对象。至此,南粤最后一点点能勉强被称为“凝结核”的力量也告消失,再也无人能阻挡对中共当局及南下干部的肆虐。在古大存被免职后一个月,一场空前疯狂的运动在南粤上演,前所未有的大劫难即将降临!

\section{列宁主义暴政下的浩劫:1957—1966年}

\indent 1958年1月,中共中央在南宁召开工作会议,毛泽东在会上批判周恩来等人在经济建设中思想保守。3月,中共中央又在成都召开各部门及各省、市、自治区党委第一书记参加的工作会议。毛泽东在会上发表讲话,提出“鼓足干劲,力争上游,多快好省地建设社会主义”的所谓“总路线”。5月5—23日,中共八大二次会议在北京召开,通过了毛提出的“总路线”,要求“全国”在工、农业生产方面实行跃进。当时,中共已在其占领区几乎完全消灭反对势力,对工商业的“社会主义改造”亦告完成。因受苏联“赶超美国”之口号的刺激,毛泽东、刘少奇等中共头目决定在钢铁及工业产品产量方面赶超英国,使所谓的“社会主义世界”将“帝国主义国家远远抛在后面”。自当年6月起,各邦中共官僚开始掀起“浮夸风”虚报粮食产量,制造出一批匪夷所思的“高产卫星”。惨绝人寰的“大跃进”运动,就此拉开序幕。

在广东,当“大跃进”开始后,老牌列宁主义分子陶铸、赵紫阳仿佛饿狗嗅到粪便一般,投入了狂热的“工作”。1958年7月25日,中共广东省委召开电话会议,要求各地立即展开“每亩一万斤”运动。31日,广东省委又进一步下达命令,要求各市、县“认清形势,去掉骄气,拼命奋战四月”,在年底之前达到产粮600亿斤的超高指标。由于该指标太过荒唐,引起不少质疑,陶铸便亲自上阵,于8月1日发表《驳“粮食增产有限论”》一文,提出广东水稻一年可种三造,亩产万斤也是很有可能的\footnote{《中国共产党广东历史1949—1978》第二卷,页336—338}。在广西,中共当局亦不甘落后。自1955年起,中共即以曾在越南工作的韦国清(东兰僮人)担任广西省省长。1958年3月,中共改“广西省”为所谓的“广西壮族自治区”,韦又出任区政府主席及区委书记,一手掌握广西大权。韦国清虽为僮人,但他深受毛泽东、邓小平信任,是个极度嗜血的杀人魔王。同年7月,韦国清召开广西各级党委书记会议,要求实现思想上的“跃进”\footnote{黄灵谋:《广西“大跃进”研究》}。在陶、韦等恶魔的鼓动下,荒诞至极的“浮夸风”迅速席卷南粤大地。9月3日,粤北连县田北社出现首颗“高产卫星”,当地水稻亩产量据说达到骇人的60437斤,中共广东省委机关报《南方日报》对此进行了专题报导。11日,广西首颗“卫星”诞生,环江县的红日人民公社据说实现了水稻亩产16227斤。18日,中共中央机关报《人民日报》又进一步报导,称环江的水稻亩产量实已达到130434斤\footnote{《中国共产党广东历史1949—1978》第二卷,页336—338}。26日,《人民日报》更出现一篇令人喷饭的报导,称番禺县出现了亩产番薯100万斤、甘蔗60万斤、水稻5万斤的“高产试验田”。11月14日,《南方日报》作出题为《万斤卫星飞满天,大片高产有奔头》的报导,称曲江、云浮、新兴、从化等地纷纷出现“万斤卫星”和“两万斤卫星”\footnote{《中国共产党广东历史1949—1978》第二卷,页339}。此后,南粤的“浮夸风”已不可遏制。环江红旗公社的“卫星”是“全国”各“卫星”中最大的一颗。此“卫星”出现后,大批岭北干部纷纷来到环江“取经”。当时的情形是:

\begin{quote}

(参观人员)首先见到的是巨大的标语口号,什么“共产主义是天堂,人民公社是桥梁”、“人有多大胆,地有多大产”。因为各地来到此处参观者众多,而安排是个问题,很多只是走马观花、浮光掠影地看一遍,无暇问及更多情况。到了田间,看到在一亩地中,稻子一棵挨着一棵,不见空隙,如同一个大稻子垛。当地领导特地安排一些讲解员,这些稻子是如何种植的?是如何管理的?经介绍是采用密植的方法,深翻土地,多施肥料,白天要用鼓风机向里边通风,晚上要有灯光照射。其实事实上是这样的:1958年8月2日,红旗公社的全社800人,经过苦战两天两夜,把附近18亩9分准备成熟的苗禾与粘粮谷,采用密植的办法,全部移到这块卫星试验田来……按照当时环江县早稻平均每亩685斤,这么多亩也不过是14000斤。为了达到放高产卫星的目的,只好在过秤时,把刚刚过秤的谷子,从前门走,又得从后门绕一圈拿回来过秤。这样反复多次,才有后面的亩产13万多斤\footnote{引自黄灵谋:《广西“大跃进”研究》}。

\end{quote}

一处如此,别处情况不问可知。与“浮夸风”相伴的是同样疯狂的“共产风”。1958年8月,中共中央在北戴河召开政治局扩大会议,叫嚣农业经济集体化、建设所谓的“人民公社”是“提前建成社会主义并过渡到共产主义”的必经之路。是月26日,中共广西区委发出在农村建立人民公社的“指示”。仅仅过了十八天,全广西就建成人民公社918个,入社农户404万户,占农村总户数97\%\footnote{周健:《广西民族团结的历史与现实研究》,页165}。广东省委随后紧跟,于9月11日做出在农村建立人民公社的决定。短短一个月内,广东亦实现了农村人民公社化,其速度之快难以想象。全广东119个县、市,共建立人民公社803个,入社农户790多万户,占农村总户数98.8\%,另有渔业公社81个,生产大队1093个,生产队5290个。如此快速的农业集体化进程便是所谓的“共产风”。南粤农民在土改中分得的山林、果树、耕牛、农具乃至住房、粪屋、猪牛栏至此几乎全被收归公有,农民们近乎被抢走了所有财产。根据规定,农村生产计划由公社统一安排、指挥,劳动力和物资由公社统一调拨,财务由公社统一收支,分配由公社统一计算,农民连生产、生活的自主权都被剥夺,丧失了一切自由,甚至连进食也必须在公社的公共食堂吃“大锅饭”\footnote{《当代广东简史》}。这样,大小中共干部便彻底掌握了所有南粤人的生杀大权,得以随意剥夺民众的一切,为所欲为。

按照中共的计划,农业“大跃进”和集体化是工业“大跃进”的前提,因为只有这样,中共才能动员全民“大炼钢铁”。据北戴河会议规定,1958年“全国”钢产量达到1070万吨,比1957年翻一番,1959年则要达到2700—3000万吨。1958年8月28日,韦国清发出叫嚣,要求广西“克服右倾保守思想”,在年底之前完成年产钢21万吨、铁70万吨、原煤400万吨的任务\footnote{黄灵谋:《广西“大跃进”研究》}。9月12日,陶铸发表政治报告,号召全广东“人民”为实现1958年产钢25万吨、产铁69万吨的目标而“奋斗”\footnote{《中国共产党广东历史1949—1978》第二卷,页345}。在中共当局的强迫下,全粤民众被一同动员起来以军事化组织方式“大炼钢铁”、生产煤炭,一颗颗工业“卫星”也相继出现。9月29日,柳州柳城出现日产煤85315吨的“卫星”,轰动一时\footnote{《中国共产党广东历史1949—1978》第二卷,页336—338}。10月2日,广东放出首颗钢铁“卫星”,宣称全广东在一天内炼出生铁1.2万吨,《南方日报》还专门为此发表了一篇题为《祝钢铁战线大捷》的社论。27日,广西忻城县更放出日产煤110.36万吨的特大“卫星”。由于农村劳动力大都被动员进行钢铁、煤炭生产,南粤各地的自然植被遭大面积破坏,土制“小高炉”通宵达旦地燃烧,染红了黑夜的天空,农田则乏人耕种。由于炼制过程极端不合理,这些钢铁、煤炭大都是不能用的废品。以广东为例,到1958年底,全广东虽然产钢20.6万吨、生铁43.79万吨,但合格品只有钢4.21万吨、生铁17.43万吨\footnote{《中国共产党广东历史1949—1978》第二卷,页346—348}。

疯狂的“浮夸风”、“共产风”迅速给南粤大地带来了沉重的灾难。自1959年初起,南粤各地相继出现粮食紧张。由于各地纷纷虚报粮食产量,中共当局遂在农村进行超高额的征粮活动。农民无粮可交,又被剥夺了进行有组织武装反抗的可能,唯有想尽一切办法将粮食藏起来。1959年1月,由于广东的征粮任务仍未完成,陶铸便亲率一支工作队,于该月中旬来到东莞县虎门公社亲自发起“反瞒产”运动。该月11日,在东莞县“反瞒产大会”上,陶铸丧心病狂地对与会基层干部叫喊道:

\begin{quote}

保证三餐干饭吃到底,全部粮食集中到公社,任何人不能保存粮食!

\end{quote}

在陶铸的压力下,东莞干部只得承认“隐瞒”了2000多万斤粮食。此后一星期内,在陶铸的亲自指挥下,穷凶极恶的中共干部从东莞农民手中抢夺了558万斤粮食和1.2万元资金。接着,赵紫阳率另一支工作队于23日到达雷州半岛最南端的雷南县,于此下令召集4000名当地干部开会,要求他们立即进行“反瞒产”斗争,并于25日指责全广东的基层干部都卷入了“瞒产”活动。赵紫阳杀气腾腾地表示,全广东有25—30亿公斤“瞒产”粮食。2月22日,毛泽东更代表中共中央亲自肯定赵紫阳的做法,要求在“全国”各地推广。25日,《人民日报》又发文力挺陶铸。就这样,遍及全广东的“反瞒产”运动开始了。在赵的严令下,各地干部挨家挨户地闯入农民家中,夺走他们的最后一点口粮,甚至连菜种都搜刮一空。若有干部不积极搜刮,便会被撤职、监禁乃至处死。至于拒绝交粮的农民,更是饱受虐打,成千上万的人被活活打死。与此同时,广西的“反瞒产”运动也在韦国清的指挥下开始,过程与广东一样残酷。自1959年2月起,大批农民开始如苍蝇一样死去。饿极了的民众试图冲击粮仓抢粮,被公社干部成批开枪打死。仅在环江县,就先后有数十人因“偷粮”被枪杀,其中包括一名12岁的男孩。更多人饱受折磨,遭受了捆绑吊打、铁丝穿手、坐水牢等酷刑\footnote{李昌玉:《广西环江:从反右到大跃进、大饥荒》}。在此期间,毛泽东虽曾认为“大跃进”出现异常,试图“纠左”。但在1959年7—8月的庐山会议上,彭德怀攻击人民公社制度,被毛泽东、刘少奇联手打为“反党集团”头目。此后,毛泽东停止“纠左”,转而发动“反对右倾机会主义”运动,将大批对“大跃进”态度消极的干部打成“右倾机会分子”。陶铸、韦国清等侵占东亚各邦的中共干部上行下效,一口气划出300多万名“右倾机会分子”。这样,“反瞒产”和“反右倾”运动便结合在一起,使用更大的灾难席卷东亚大陆。在南粤,民众被逼得走投无路,坠入了史无前例的悲惨境地。到1960年初,灾难还在继续。只需再看看环江当时的情形,便可知南粤人正面临着怎样的地狱:

\begin{quote}

(1960年,)柳州地委书记处书记朱渭川赶到环江,亲自出马,指导环江的反瞒产私分工作。按照上级的布置,洪华(环江县委书记)一直在积极部署环江县的“反右倾、反瞒产”运动。洪华告诉县委工作组:“粮食还有,都被下面瞒产私分了!不要心慈手软!不惜一切代价,千方百计也要把粮食弄出来!”就在洪华亲自抓典型的这次核实会上,被斗致伤又连饿带病,致使13人死亡。洪华恶狠狠地说:“这些人是社会主义的逃兵,斗死几个不要紧!”上行下效,全县反“反后手粮”的运动,毒打致死者达几百人之多。报了瞒产,就得交粮。社队所存的有限的口粮,甚至公共食堂的存粮,也被逼得当作“后手粮”交出来了。在这场“反瞒产”运动中,搞出“瞒产私分”的粮食2941万斤。结果,集体的存粮挖空了,农民的口粮搞光了,连种子量、饲料粮都收刮走\footnote{转引自李昌玉:《广西环江:从反右到大跃进、大饥荒》}。

\end{quote}

在中共当局极度残酷的压榨下,仅有16万人口的环江县竟有超过5万人饿死,每三人中就死了一个。在饥荒最严重的时候,人们互相杀食,县城的街道上空开叫卖着人肉。在罗定县,农民在1959—1960年间成批饿死,有的村庄甚至死得一人不剩。小蛇仔、蚱蜢、小虫早已被饥民吃光,吃人惨剧不断发生。当已有上万人饿死时,县委书记罗正时却高叫“不要大惊小怪”,说全县只饿死了14人。在高要县,粮荒自1959年初爆发。在春节前两天,该县县委出动4200多人下乡“打黑仓”,用短短一天就掠夺了1.2亿斤粮食\footnote{《我所知的大跃进大饥荒之广东》}。在始兴县,中共干部疯狂地叫嚣“敢打敢骂的才是好干部”,用近20种酷刑残酷折磨农民。虽然粮仓里堆满了粮食,但农民不但得不到足够的粮食,还被强迫每天劳动16小时,大多换上了水肿病,数以千计地饿死、累死。在合浦县,1959年12月因洪水发生了垮坝事件,造成大批人员伤亡。至1960年春夏之间,饥荒更蔓延到县城,造成8600多人死亡……凡此种种,仅是这次大饥荒的冰山一角。与194*3、1946两次大饥荒不同的是,饥民们被牢牢束缚在人民公社中,几乎没有逃荒的机会\footnote{《岭南纪事》,页224—226}。中共在此次饥荒中对粤人犯下的罪行,可谓罄竹难书!

到1960年9月底,因饿死人实在太多,“大跃进”难以为继,中共中央终于下令要在坚持“总路线”、“大跃进”、“人民公社”这“三面红旗”的前提下对经济进行“调整、巩固、充实、提高”。10月,领会上意的陶铸、韦国清相继当即下令遏制“共产风”,停建或缓建一些项目,使饥荒有所缓解。1961年1月14日,刘少奇在中共八届九中全会上正式提出“调整、巩固、充实、提高”方针,“大跃进”正式停止,然而饥荒的真正结束还要到一年多之后。这时,由于奄奄一息的农民已无法养活城镇,中共强令将各邦2000万城镇人口迁往农村。在南粤,随着城镇人口大量下乡,严酷的人民公社体制一时出现松动,许多人开始趁乱逃往香港。早在1956—1958年间,当中共在南粤进行“社会主义改造”时,就曾有20105人经宝安县逃往香港,是为“第一次逃港潮”。对南粤人来说,香港不但是与南粤血肉相连的亲邦,更是一座安全岛,粤人对香港并不陌生。只有在那里,南粤人才能过上自由的生活、得到做人的尊严。在漫长的南粤史上,粤人曾一次又一次为自由、为尊严而战。尽管中共的统治史无前例,将暴政与强制管控深入到南粤社会的每一处,但粤人绝不会因此放弃对自由与尊严的向往。自1961年起,规模庞大的“第二次逃港潮”爆发了。当时,粤港边境早已被中共的武装部队封锁,但数以十万计的南粤民众仍然不顾危险,想尽一切办法越过那道铁幕,每一个逃港者的故事都可拍成一部血泪、辛酸与精彩并存的电影。以笔者的一位亲人为例,他曾先后尝试过三次逃亡,在前两次都惨遭中共民兵殴打,第三次终于偷渡成功,抵达自由世界。逃港路线分西、中、东三条,深圳以西的逃亡者多选择西线,自蛇口、红树林一带游过深圳湾,用一个多小时游到香港元朗。中线逃港者则大多乘火车、汽车到达深圳,在夜间伺机从深圳河边翻越铁丝网、躲过军队和警犬进入香港。东线逃港者则多从惠阳、梅县、汕头出发而来,自惠州出发,徒步攀越梧桐山,或从盐田、大鹏、南澳一带游过大鹏湾进入香港。逃港之路是充满危险的,许多人惨遭中共军警逮捕、杀害,或在海中溺水而亡,甚至被鲨鱼咬死,但他们仍前仆后继地向香港前行。许多人手持棍棒,与把守关口的中共军警拼死搏杀,得以冲向自由。1961年7月20日,因逃港人数过多,而广东经济已经崩溃,中共广东当局不得不稍稍放宽限制,开设了几个非正式关口。到1962年5月中旬,逃港潮已不可遏制,每天都有上千人进入香港,深圳、宝安一带每天聚集着四五千名伺机逃港的民众。与此同时,逃港潮蔓延到广州。因风传港英当局会在女王生日时开放边境三天、大陆居民赴港无需通行证,购买港深线火车票的人数迅速激增22倍。在无边的黑暗中,人们用传言互相鼓舞,发誓只要还有一口气,就一定要逃到自由的世界\footnote{《我所知的大跃进大饥荒之广东》}。6月1日,临近香港的中共宝安县委向省委做出如下报告:

\begin{quote}

现在出现大逃亡风潮,不仅农村党员、团员,而且城镇机关的党团员也大量外逃。这次外逃发展非常迅速,来势甚凶。因而,从农村到城市群众思想都很混乱,农民无心生产,城镇有的工厂停工\footnote{转引自《我所知的大跃进大饥荒之广东》}。

\end{quote}

逃港潮是南粤人对中共暴政做出的英勇反抗,它惊动了中共中央。如果南粤人全部奔向自由,那中共在南粤的疯狂统治就维持不下去了。5月下旬,周恩来下令广东省委要将阻止逃港“作为当前第一位的工作来抓”。6月5日,广州东站出现大乱。数以万计的民众一批批冲向开往香港的火车,造成交通大堵塞。在广州的街道上,门窗紧闭,似乎全城市民都已奔向东站,连车站旁的树上都爬满了人。中共当局派出警察前往镇压,被激怒的人们便放火烧毁警车,将数名警察捉走。面对民众的怒潮,中共广东当局的大小头目惊慌失措,开会商讨对策。副省长曾生(惠阳人)称,应当立即实行军事戒严,遭到部分人反对。有人称,如果此事属“人民内部矛盾”,便无需出动军队。这时,陶铸发出了凶残的狂吼:

\begin{quote}
现在还叫什么人民内部矛盾?一般理解是敌我矛盾了,把公安局的车都砸了、烧了,还讲什么人民内部矛盾\footnote{转引自《我所知的大跃进大饥荒之广东》}?
\end{quote}

陶铸一锤定音下,一营军人乘车开赴广州东站,于6月6日开始凶残的镇压。他们武装封锁车站两侧,用暴力驱赶聚集其中的市民,凡反抗者尽被逮捕。到当日夜,广州东站沉寂下来,站中上只有全副武装的中共的戒严部队,1600多名平民已被捉走。14日,又有近千名市民聚集至东站,再次被中共军警驱散。在此前后,南粤各地的中共军警也加强戒备,全力阻止民众逃港\footnote{《我所知的大跃进大饥荒之广东》}。轰轰烈烈的“第二次逃港潮”,就这样被中共当局用暴力切断了。

在1961—1962年的逃港潮中,共有11万人加入逃亡,其中6万人成功逃至香港。面对汹涌而来的南粤难民,港英当局采取“即捕即遣”政策,派出数千军警围追堵截,将其中约4万人强制遣返。1962年5月,最感人的一幕发生了。在香港市区附近一座名为华山的小山上,聚集着3万名逃港者,他们与大约30万名香港人有着亲人、同乡、同学、朋友的关系,而当时的香港人口不过是300万。十余万名香港民众自发组织起来,带着食物和水赶到华山,用生命保护自己的亲邦兄弟。当逃港者被捉入汽车时,上万名香港市民卧倒在高温的马路上,用血肉之躯阻挡汽车,向逃港者呼喊着逃跑路径。在场的许多香港警察也不忍抓捕,同逃港者相拥而泣。在香港市民的掩护下,有1.5万人成功逃入市区\footnote{《大逃港之华山救亲》}。这段催人泪下的历史镜头,无疑是粤港友谊的最好见证。

在香港,逃港者的生活是艰难的,但也是充满希望的。他们已经失去了几乎一切财产,但获得了自由与尊严。他们有的在香港街角、空地处用木板钉出板屋,有些也到大楼天台上搭建,香港报刊后来常见“天台木屋”一词,最早即形容逃港者的居所。而逃港者抵达香港后,常常在家庭式作坊内从事粘纸盒、缝袜子、勾纱等工作,勤恳地维持生计。逃港者较低的薪酬要求,一方面为经济开始腾飞的香港提供了大量廉价劳动力,另一方面他们的勤奋努力,在自由世界也收到了回报。据不完全统计,20世纪末香港排名前一百位的富豪中,便有四十多人是上世纪60、70年代的逃港者,其中便包括金利来集团董事局主席曾宪梓、壹传媒集团主席黎智英、“期货教父”刘梦熊等人。不仅如此,香港“乐坛教父”罗文、“金牌编剧”梁立人等本地文化精英,也都曾是逃港者中的一员。今天,逃港者的后代早已成为地道的香港人。他们正与无数港人一道,为香港的自由而战。

此后不久,在南海的彼岸,又有一批南粤人获得了自由。1965年8月9日,以南粤潮汕、广府、客家移民后裔为主体的新加坡宣布脱离马来西亚联邦,建立独立共和国。在李光耀总理(祖籍大埔)等的带领下,凭借相对稳定的政治环境、经济政策,新加坡人以自由、务实的态度,抓住机会,承接欧美发达国家转型产业,并大力发展对外贸易与海上交通事业等,使新加坡经济得以高速发展,迅速崛起。新加坡的成功也为南粤未来的发展作出了榜样,该国的繁荣与当时南粤本土的凋敝形成了鲜明对比。这表明,纵然南粤本土已沦入黑暗,但南粤文明撒向海外的种子却在四处开花结果。哪怕是在最黑暗的时代,南粤也从未死去\footnote{参见C·玛丽·藤布尔著,欧阳敏译:《新加坡史》,北京:东方出版中心,2013年版。}。 

与获得自由的同胞相比,那些在1958—1962年大饥荒中惨死的人们是无比悲惨的。东亚各邦在“大跃进”中究竟死了多少人?这在目前的资料环境下恐怕是个谜,各种估算从1800万到4500万都有,就连毛派中的最狂热者也不得不承认死了几百万人,但他们把责任全部推到了刘少奇、邓小平身上。学者曹树基在对1462个县的统计资料进行比对后,得出有3245.8万人饿死的结论,这还不包括内蒙古、新疆和西藏。其中,广东和广西分别饿死了65.7万和93.1万人,各占粤、桂人口总数的1.71\%和4.63\%。也就是说,有158.8万名南粤人在这场大饥荒中死去了\footnote{曹树基:《大饥荒:1959—1961年的中国人口》}。与1943、1946年两次大饥荒不同,这次饥荒与天灾关系不大,几乎完全是中共制造的人祸。它的死亡人数不但超越了前两次饥荒,甚至超过了17世纪的“迁海”浩劫,可以说是南粤近现代史上空前绝后的大灾难。这是列宁主义分子在南粤欠下的深重血债。它必将被一代代南粤人永远铭记,成为南粤人永不能饶恕的丑恶罪行。它亦将激励一代代南粤人奋起战斗,与一切暴政进行最坚决的斗争,为自由、尊严、家园与祖国挺身而战。

1962年1—2月,中共中央在北京召集各省、市、地、县开会,史称“七千人大会”。在会上,毛泽东以退为进,就“大跃进”做出“自我批评”,邓小平、周恩来亦做出检讨,刘少奇却无耻地将自己的责任全部推卸干净,将罪责全部推到毛泽东身上,而坚定站在毛泽东一边的中共中央要员只有林彪。此后,毛泽东暂时退居“二线”,由刘少奇、邓小平出面组织中共中央工作。但毛泽东仍拥有最终决定权,一直准备反扑。

1962—1965年间,刘、邓采取稍显务实的经济政策,允许农民进行有限的自主经营、在一定程度上开放市场,使荒残的经济得以好转。但这不过是苏联“新经济政策”式的以退为进,目的不过是暂时与民休息,以代下一次收割。为反击刘、邓,毛泽东于1962年9月的中共八届十中全会上称农村出现了威胁“社会主义”制度的农民“单干”现象,称之为“单干风”,并提出中共党内存在“修正主义”倾向,有“资本主义复辟的危险”,全党“千万不要忘记阶级斗争”。当时,中共已与赫鲁晓夫治下的苏联决裂,将苏联打为“苏修”。毛泽东的此种论断,矛头无疑直指与苏联关系密切的刘、邓乃至周恩来。从是年10月开始,中共发动“社会主义教育”运动,要求在“全国”各人民公社开展“社会主义教育”,运用回忆、对比等方式诱导农民相信集体经济的“优越性”。在南粤,该运动在中共广东、广西当局的主导下迅速开始试点工作。在许多场合,农民的“忆苦”总是和刚刚过去的大饥荒相关,令中共干部大为尴尬。至1963年5月,中共中央又发动所谓的“四清”运动。毛泽东声称,为“反修防修”,应在农村进行一场“清工分,清账目,清财物,清仓库”的运动。他危言耸听地说,基层“有三分之一的领导权不在我们手里”,只有清除“四不清”干部,方能防止“资本主义复辟”。为与毛争夺运动领导权,刘少奇随后发出狂言,说基层政权不在中共手中的不止三分之一。1964年9月,在刘的命令下,大批由中共干部组成的“工作队”进入各邦农村,对基层干部展开残酷“斗争”,并时常滥及平民,死亡人数很快达到数万。在南粤,陶铸、韦国清皆积极执行刘的指示,疯狂清洗基层干部,数千民众惨遭殃及,死者相继。在揭阳县,甚至有超过四成公社都被视为“领导权不在我们手上”者,遭到残酷清洗\footnote{《当代广东简史》}。然而,中共分子间的此次内讧和即将到来的狂风暴雨相比仅是绵绵细雨。1964年12月,毛泽东在中共中央工作会议上抨击刘少奇,称运动的重点应是发动“群众”打击党内的“当权派”,而非清洗基层。到该年下半年,毛泽东基本放弃“四清运动”,转而策划一次前所未有的攻势。1966年5月16日,惊变发生。是日,在毛的要求下,中共中央政治局发布《五一六通知》,要求展开“无产阶级文化革命”,清洗“混进党里、政府里、军队里和各种文化界的资产阶级代表人物”,并称中共内部存在“睡在我们身胖的赫鲁晓夫”。这个所谓的“赫鲁晓夫”,自然指刘少奇乃至邓小平。中共高层间的内讧至此达于顶点。很快,“无产阶级文化大革命”的风暴便席卷东亚大陆,将南粤卷入狂潮\footnote{关于中共此段党史,相关研究甚多,此不赘注}。

\section{列宁主义机器的意外松动:1966—1972年}

\indent 1966年5月16日,中共中央政治局发布《五一六通知》,“文革”爆发。当时,因毛泽东尚在杭州观望局势,文革的主导权最初落入刘少奇、邓小平之手。刘、邓虽明白自己已成为毛要打倒的对象,但不敢公开违抗毛的命令,便试图转移斗争方向。在5月底,北京的大中学校已有激进学生响应毛的号召,成立了激进的“红卫兵”组织。6月初,刘、邓决定向北京大中学校派出工作组,试图压制学生运动,重演“反右”和“四清”。工作组进入学校后,将大批反对他们及学校党委的师生打成“右派”、“小爬虫”残酷迫害。从一开始,中共高层的内讧便蔓延到了社会。

在文革初期,如此多人响应毛的号召是有深刻社会背景的。在1949—1966年间,中共的列宁主义机器无日不肆虐于东亚大陆各邦,令民众敢怒不敢言。现在,名义上的中共最高领袖毛泽东突然号召“群众”清洗党政军系统中的“资产阶级分子”,给了民众一次向中共各级官僚复仇的机会。在南粤,毛的号召也迅速得到响应。1966年6月初,广州四十五中学的一批学生组织起一个“战斗小组”,贴出大字报攻击校领导。陶铸、赵紫阳等人如临大敌,连忙听从刘、邓命令,派出大批工作组进驻广东各机关单位及学校。6月11日,中共广州市委召开“全市中学教师大会”,要求全体师生一同横扫教育界的“牛鬼蛇神”。接着,工作组开始在全市范围内煽动学生殴打教师,造成大批教师自杀身亡。与此同时,进驻各机关单位、工厂的工作队也与各级领导勾结,肆无忌惮地将与领导关系不好及出身“黑五类”(中共认定的地主、富农、“反动派”、“坏分子”、“右派”)干部、民众打成“右派”,遭波及者竟达三分之一以上\footnote{刘国凯:《广州红旗派的兴亡》,页13—14}。与此同时,韦国清派出的工作组也进驻各大中学校及机关单位,诱导激进学生将学校党委当做替罪羊打倒,并号召学生们重现“反右”运动\footnote{广西文革大事年表编写小组:《广西文革大事年表》,页2}。

对刘少奇、邓小平等老牌列宁主义分子的伎俩,毛泽东早有准备。7月18日,他突然回到北京,随后开始激烈指责工作组“镇压学生运动”。24日,他又在中共中央常委会上抨击刘、邓,要求撤销工作组。8月1日,中共八届十一中全会在北京召开,毛的进攻更为猛烈,他称工作组是“站在资产阶级方面反对无产阶级”,并表态支持红卫兵。5日,他更在会上写下《炮打司令部——我的一张大字报》一文,称中共中央存在一个“资产阶级司令部”。18日,毛更出现在天安门城楼上,“检阅”了上百万红卫兵和“群众”组成的游行队伍。掌管军队的林彪站在他身边,号召红卫兵“大破一切剥削阶级的旧思想、旧文化、旧风俗、旧jab 惯”,文革随之转入更加暴烈的阶段。

列宁主义官僚系统如同一台严丝合缝的杀人机器,简单的冲击绝不可能使之停转。在工作组撤销、毛泽东首次接见红卫兵这段时间内,以周恩来、刘少奇、邓小平为首列宁主义分子又纷纷纵容自己的子女组织“红卫兵”,试图以此重夺运动主导权。这些年纪轻轻的红二代从小便对父辈杀人如麻的经历“如数家珍”,异常残暴嗜血。他们自称“老红卫兵”,叫嚣“老子英雄儿好汉,老子反动儿混蛋”,认为自己有“高贵”的“革命血统”,绝不能容许民众夺走他们的“天堂”。自7月底起,老红卫兵蜂起于北京城,用极其残忍的手段屠杀“黑五类”,连自己的老师和同学都不放过。此外,他们还砸毁大批文物古迹,自称在响应林彪的“破四旧”号召。在8月19日—9月2日间,他们在北京制造了“恐怖的红八月”,杀死至少10285名无辜平民、“没收”私宅数十万间。而中共事后公布的统计数据则称,当时死者只有1722人\footnote{该数据源自1985年中共进行的调查}。

在南粤,文革的进程比北京稍后。自8月中旬起,随着一批来自北京的老红卫兵窜入广州、南宁,广东、广西的老红卫兵组织纷纷出现。8月下旬,由侵粤中共干部子女组成的老红卫兵与来自北京的同类一道发起了“破四旧”和血腥的屠戮。在南宁,老红卫兵四处横行,有的甚至窜入乡村,将5000多人打成“牛鬼蛇神”横加虐待,抢夺了15万件物品\footnote{广西文革大事年表编写小组:《广西文革大事年表》,页7}。在广州,中共当局甚至从未公布过老红卫兵的“战果”。当时,中共高干、军干的子女多集中于广州市区北缘的中学。他们的父辈大都是南下干部,对南粤毫无认同感。他们虽在广州长大,但很多人连粤语都说不好,更不用说会对南粤有多大感情。当时,他们自称“毛泽东主义红卫兵”(简称“主义兵”),和北京老红卫兵一同身穿绿军装、腰束武装带,用普通话高声叫嚷着横行街市,大肆破坏文物古迹。在8月2*5、26两日,他们一连扫荡了广州的49座寺庙和清真寺,将广州两千年来的文明象征毁灭殆尽。在海幢寺,他们毁掉了四大金刚塑像。在华林寺,他们捣毁了五百罗汉。在怀圣寺附近,他们推倒了南粤本土穆斯林的“先贤古墓”\footnote{《广州破四旧狂潮》(上)}。最令人发指的,则是他们对石室圣心大教堂的破坏。

广州石室圣心大教堂位于一德路,原为大屠夫叶名琛的清两广总督署。第二次鸦片战争后,经法国皇帝拿破仑三世专门拨款,教堂于1863年在总督署废墟上奠基。当时,教廷任命的两广教区宗座牧监明稽章专从罗马、耶路撒冷各运来泥土一斤,在教堂东西墙角分别刻下“JERUSALEM 1863”、“ROME 1863”字样。1888年,教堂竣工。它坐北朝南,高三层58.5米、占地6000多平米,平面呈十字型,装饰着美轮美奂的彩色玻璃。教堂底层开有三个尖拱券门,二楼当中有个精美的圆形玫瑰窗,三楼则耸立着庄严的尖顶八角形钟楼,内有四口从法国运来的大铜钟。特别值得一提的是,教堂建造工程是由南粤工匠蔡孝(揭西客家人)指挥完成的。蔡孝自二十几岁起就投入工程中,一直到五十岁才完工,为之耗尽了半生心血。经他提议,教堂楼顶的出水口采用粤式狮子造型,大门也刻上了粤式木雕。这座教堂不但是南粤与西方友谊的见证,更是世界建筑史上的奇迹。它不但是东亚大陆最大的哥特式教堂,也是全球四座全石结构哥特式教堂之一。要知道,另外三座可是大名鼎鼎的巴黎圣母院、西敏寺和科隆大教堂。1966年8月25间,一切都毁灭了。史载,当时的情形是:

\begin{quote}

红卫兵把从教堂里搜出来的书籍、衣物、宗教器皿和画册,堆放在教堂前的空地上,放火焚烧。在教堂哥特式建筑的巨大窗口上,原装的拿破仑时代的彩色玻璃,被砸得稀烂,碎片散落一地。唱诗台上的脚风琴也被大卸八块。红卫兵们把神职人员押到空地上斗争,逼他们跪在烧书和物资的火堆前。焚烧后的纸灰漫天飞舞,直冲云霄\footnote{引自《广州破四旧狂潮》(上)}。

\end{quote}

大破坏过去后,精美的教堂被变为垃圾站,直到1979年才重新开放。中共从来热衷于毁灭一切美好事物,这些丧心病狂的老红卫兵自然也不例外。面对这巍峨的教堂,他们心中的自卑感骤然升起,激发起他们狂暴的破坏欲。与“破四旧”相伴的,自然是对“黑五类”的残酷虐杀。许多地主、富农出身的粤人侥幸活过了土改,却未能逃离这群只有十几岁的禽兽。“主义兵”窜入乡村,将原地主、富农捉回学校进行残酷的虐打,直到打死为止。当死者子女前来领尸时,他们还在用普通话高声叫嚣,说:“他妈的!把这些黑五类都打死才好!\footnote{刘国凯:《广州红旗派的兴亡》,页16}”

在1966年8—9月间,老红卫兵的“破四旧”和“红色恐怖”自北京开始,迅速蔓延至东亚大陆各地。对老红卫兵的行为,毛泽东一开始并未阻拦。在他看来,老红卫兵虽是列宁主义官僚用来对抗文革的外围组织,但他们可以帮自己清扫一大批“阶级敌人”。等到他们将“阶级敌人”清理得差不多了,他再出手也不迟。10月初,老红卫兵的“工作”已基本完成,毛遂向刘少奇、邓小平又一次开火,称他们执行了“资产阶级反动路线”(简称“资反线”)。林彪紧跟毛泽东,称刘、邓“同毛主席的路线相反”。11月底,不甘失败的北京老红卫兵成立“首都中学红卫兵联合行动委员会”(简称“联动”),随后高呼“坚决保卫革命的老干部”,先后六次冲击公安部。这些不知天高地厚的红二代们失算了。1967年1月,“联动”被毛泽东打成“反动组织”,其头目相继被捕,在监狱中过了三个月。被释放后,他们不复之前的锐气,因为他们的父辈已纷纷倒台,文革已朝向毛的本意发展。

自毛泽东批判“资反线”起,饱受工作组、老红卫兵压制的南粤学生、工人、市民、职员终于得到起义的机会。对毛来说,批判“资反线”只是他扳倒刘、邓的一步棋。对粤人来说,这却是个利用中共内讧冲击列宁主义机器的好机会。1949年以来,粤人已经忍得太久了。早在1966年8月中、下旬,广州大学生就已在建立自己的红卫兵组织。8月31日,中山大学“红旗公社”成立,成为广东造反派的领导组织。在此前后,广医、华工、华师也先后成立造反组织。自10月起,随着批判“资反线”运动传到南粤,横行一时的“主义兵”纷纷暂停活动。广州学生与市民下英勇地冲击中共各机关部门,销毁了一大批“黑材料”,这些材料多记载着中共干部在十几年来给大批民众罗织的罪名。此外,他们还解救被“主义兵”关押的民众和教师,使成百上千的无辜者免于惨死\footnote{刘国凯:《广州红旗派的兴亡》,页20—22}。10月,广西的造反派红卫兵组织也在南宁联合起来,将中共广西区委要员揪出办公室不断批斗。12月,粤、桂各城市中的工人、职员也开始行动,成立了一大批“造反派”,不断进行数万人规模的集会,要求中共广东省委、广西区委就执行“资反线”的行为做出解释\footnote{广西文革大事年表编写小组:《广西文革大事年表》,页10}。

这时,陶铸已被中共中央调往北京,广东民众的怒火一致对准了赵紫阳。因陶铸与林彪关系密切,在1966年8月,毛泽东本想以陶为倒刘干将,便将其定为中共中央政治局常委,在政治局中仅次于毛、林、周,位居第四。此后,毛又任命陶为“中央文化革命小组”(简称“中央文革”)顾问,希望他与自己合作。然而,陶铸却错判形势,误认为毛不会彻底将刘、邓打倒,对二人多有回护之意,遂被毛迅速抛弃。1967年1月4日,在陈伯达、康生、江青等文革派中坚的授意下,北京红卫兵宣布陶铸是“最大的保皇派”。此后,陶铸遭囚禁。同年8月,在天安门广场召开的百万人“批判刘、邓、陶大会”中,陶铸被造反派拳打脚踢、遍体鳞伤。1969年10月,已被折磨得奄奄一息的他被驱逐至合肥,度过了一个多月的残生,于11月30日在极端痛苦中死于癌症。陶铸自入粤以来一向精于揣摩上意,狂热地执行中共中央的各种疯狂命令,至少屠杀了78万南粤人\footnote{《文革大事记之七:陶铸被迫害致死真相》}。这次,他因为误判形势失去一切,获得了与自身德性相匹配的下场,这真是大快人心。

1967年1月,随着陶铸倒台,双手同样沾满粤人鲜血的赵紫阳岌岌可危。是月初,在赵紫阳授意下,主张维护原有体制的“地总”、“红总”相继成立\footnote{“地总”、“红总”正式名称为“毛泽东思想工人赤卫队广州地区总部”、“毛泽东思想广州红色工人总部”。}。这两个组织虽自称“群众组织”,其实大都由党团积极分子、南下干部构成,是中共统治的受益者,因而被以“中大红旗”为首的造反派视为“保守派”。事实上,这些人根本不是什么保守主义者,不过是列宁主义官僚的走狗,称之为“保党派”恐怕更贴切。1月6日,上海造反派“工总司”在王洪文、张春桥的指挥下发动“一月风暴”,夺走上海市委大权。8日,毛泽东亲自肯定了上海造反派的夺权行动,要求“全国”造反派向上海学jab 。至此,各邦中共大员已不敢公开镇压造反派,造反派纷纷冲击各大城市的核心党政机关。19—21日,桂林的数万名造反派在市体育场举行连续三天的集会,将韦国清等十余名中共大员揪至会场进行批斗。早已恨透韦国清的民众给他挂上沉重的铁牌、戴上高帽、逼他下跪,还把他拉上卡车四处游斗。一路上,市民争相向韦吐口水、掷石子,极尽羞辱之能事。22日,南宁的中共广西区委被韦国清的惨状惊得目瞪口呆,只得宣布支持造反派,23名区委要员随后被拉上卡车游斗,遭受了与韦相同的侮辱\footnote{广西文革大事年表编写小组:《广西文革大事年表》,页10}。同日,在“中大红旗”策划下,数以万计的广州学生、工人、职员、市民兵分三路,浩浩荡荡地冲向省委、省公安厅、公安局,将三者全部接管。赵紫阳成了俘虏,被押往中山大学主楼,乖乖交出省委大印,在造反派的看管下继续处理政务。2月初,广州造反派各组织又得出共识,认为中共广州军区早已被陶、赵渗透,决定冲击军区,扩大战果。广州军区成立于1955年,下辖广东、广西、湖南、湖北军区。军区司令黄永胜(湖北咸宁人)系林彪心腹,对造反派素无好感。2月7日凌晨1时,以“中大红旗”为首的十余个造反派组织高呼“打倒黄永胜”、“炮轰广州军区党委”等口号,以一波接一波的人浪冲向军区后勤部、政治部和司令部大院。由于造反派有毛泽东背书,黄永胜不敢动武,只得命部队组成人墙防御,双方展开激烈的徒手推打。至8日下午,冲击还在继续,学生们开始以广播呼吁士兵枪毙自己的军官。黄永胜只得向中共中央军委求援,得到毛泽东、林彪支持。夜8时,广州军区播放军委命令,称若造反派再不撤走便要自负后果。经此威胁,造反派只得撤退\footnote{刘国凯:《广州红旗派的兴亡》,页29}。

2月中旬,中共军队开始反击造派。在北京,叶剑英、徐向前等共军元老大闹怀仁堂,当面怒斥江青。接着,侵占各邦的共军相继出动,大肆逮捕造反派,在青海西宁发生了对造派的大屠杀。3月1日,侵粤共军对造反派的镇压开始。在广州,黄永胜于凌晨派大批部队逐户搜查,在保党派组织的配合下逮捕2000余人。在南宁,广西军区大举出动,捣毁了造派“工总”的总部。5日,广州近千名军警和保派“地总”人员涌入珠江电影制片厂,将造派“珠影东方红”的骨干捉走。与大逮捕相伴随的,是抄家、刑讯和反复殴打。“地总”、“红总”、“主义兵”的宣传车招摇过市,用普通话狂吼“坚决拥护军区取缔反革命组织”。当时,越共正与美国进行越南战争,南粤是中共运送援越物资的必经之地,在中共的国际布局上十分关键。中共中央担心若南粤大乱会使其援越行动受阻,遂决定让军队出面恢复秩序。13日,周恩来下达命令,要求黄永胜、韦国清分别对广东、广西实行“军事管制”\footnote{广西文革大事年表编写小组:《广西文革大事年表》,页22}。黄、韦二人终于拿到“尚方宝剑”,如释重负,相继组织起广东、广西“军事管制委员会”。然而,毛泽东并不能容忍军队继续镇压造反派。4月6日,毛泽东以军委名义向全军下令,要求共军各部不得擅自压制造派。4月上、中旬,不敢抗命的军队只好将监狱中的造派释放,政局又一次反覆,变得更为复杂\footnote{刘国凯:《广州红旗派的兴亡》,页38}。

在广西,造反派对韦国清的复出深感愤怒,决定对抗到底。4月22日,全桂数十个造反组织共万余人在南宁集会,宣布成立“南宁四二二火线指挥部”(简称“四二二”)。5月25日,韦国清的御用保党派“群众组织”也在南宁集会,宣布成立“广西无产阶级革命派联合指挥部”(简称“联指”)。此后,广西各地保造两派分别宣布拥护“联指”、“四二二”,保造之争迅速遍及全桂\footnote{广西文革大事年表编写小组:《广西文革大事年表》,页27—35}。在广东,黄永胜的态度更为微妙。4月19日,有意维持列宁主义秩序的周恩来亲自向黄传达“指示”,称“地总”、“红总”等保党派有大量“工人群众”,因此也是“广州工人的革命组织”,只是“有些偏于保守”。周恩来暧昧的态度使黄永胜既不敢公开镇压造派,又不敢公然支持保派,唯有摆出中立姿态。“地总”、“红总”、“主义兵”等组织乃迅速联合成“东风派”(简称“总派”),公然叫喊“三月东风浩荡,军管成绩辉煌”,称赞军队在1967年3月镇压造派的行为,呼造反派为“旗派”、“旗佬”。“中大红旗”、“华工红旗”、“红旗工人”等组织则自称“三面红旗”,联合广州各造反组织成立“红旗派”,蔑称东风派为“总派”。广东各地的保造组织也很快分别宣布拥护总、旗两派,使保造之争传遍全粤\footnote{刘国凯:《广州红旗派的兴亡》,页40}。南粤的保造两派虽皆高呼“毛主席万岁”,却有本质区别。红旗派与四二二中固然有不少激进的毛派分子,但他们中的许多人都是受尽中共暴政压迫的平民,仅是利用中共内讧发动起义,其中大多是粤人,有不少是“黑五类”出身。东风派与联指则大多是中共政权最忠实的走狗,其中出身岭北的南下干部及其子女尤多。从一开始,红旗派与四二二便有强烈的南粤民族起义军色彩。

红旗派刚成立不久便被黄永胜利用了。自陶、赵失势后,以东江纵队尹林平为首的部分长期受南下干部、南下大军压制的广东地方党干部,借批判陶铸、赵紫阳之机,于1967年4月联合部分造反组织,试图为在此前两次“反地方主义运动”中受打压的本地干部翻案,这自然引起视广州军区为“禁脔”的林彪、黄永胜等人的不满。当时,因黄永胜未公开镇压造反派,红旗派对其丧失警惕,视之为可以依靠的对象。在黄永胜的指使下,“中大红旗”等组织斥责尹林平等人为“地方主义黑司令部”,批判他们的翻案活动系“破坏文革”。5月,在江青支持下,林彪授意黄永胜又制造了“广东地下党”案,指责中共“革命”时期打入国民政府内部的部分广东地下党人为“叛徒”、“特务”。为此,与叶剑英关系密切的广东本地干部受到牵连,一大批在前两次反“地方主义”运动侥幸“漏网”的本地干部遂与尹林平等人一道被打为“地方主义”分子,受到关押与审查,不少人被批斗致死。据不完全统计,数千名本地干部牵涉上述两案,是为广东第三次“反地方主义运动”\footnote{陈华昇:《广东“反地方主义”运动与派系冲突之分析(1949-1975)》,《中国大陆研究》,2008年第3期,页22—23}。

自6月起,因造反派已失去利用价值,黄永胜遂指使总派攻击旗派。军队更暗中向总派发放枪支弹药,使其实力大增。7月20—21日,总派收买郊区农民大举围攻旗派据点华侨工厂,打死包括旗派8人。23日,广州城内发生大惨案,占据中山纪念堂的总派与在越秀山体育场集会的旗派爆发冲突。总派首先动手,捅死一名旗派工人,随后掳走11名旗派男女,在纪念堂内虐杀,手法“花样百出”,男子以刀剖腹、割耳后吊死,女子轮奸后杀害。为救回战友,上万旗派手持冷兵器猛攻纪念堂,总派以步枪还击,终因支撑不住,乘卡车逃走。是日,总派战死8人,旗派先后死者50余人。8月11日,总派又在德宣路伏击旗派车队,开枪打死10人。13日,21名无武装的旗派人员乘小艇巡游珠江,突遭总派机枪射击,仅3人生还。因总派屡屡犯下暴行,旗派被迫组织人员冲击军队、抢走一小批枪支,于18日攻下珠江畔的总派据点省总工会大楼。20日,总派展开极度卑鄙的报复。是日,他们放出谣言,称广州北郊石井有一无人看管的海军军火库。旗派缺乏枪支,遂分乘车辆前去抢枪。旗派车队刚一出城,就遭总派伏击,当场死亡150多人,伤者不计其数。31日,总派又在广州西村水厂以枪炮攻击旗派,先后打死12人,后被旗派开火打退。9月1日,总派以“地总”、“主义兵”攻打旗派占据的河南太古仓,以炮击炮、平射炮、加农炮猛烈开火,将太古仓化为一片火海。2日,太古仓旗派不支,在市区旗派掩护下撤退。11日,总派于越秀山体育场集会,随后进入市区游行,被愤怒已极的数万市民拦截于街头。总派当场开枪扫射,打死打伤市民100多人\footnote{刘国凯:《广州红旗派的兴亡》,页45—49}。在此期间,除广州外,广东各地也相继发生总派杀戮旗派的事件。如在海丰县,总派自8月26日起发动长达半个月的“围剿”,先后杀死旗派及无辜平民上百人,其中包括持旗派观点的彭湃家属。彭湃之侄彭科于29日下午被总派活捉,当场砍下头颅。彭湃之子彭洪则被支持总派的军警逮捕,先被游街批斗,后被关进公安局饱受酷刑折磨,于9月1日伤重而死\footnote{杜钧福:《彭湃一家在文革中的悲剧》}。在一片黑暗中,就连魔王彭湃的家人都站在旗派一边,成为相对正义的一方,这真是荒诞至极。

自1966年夏起,因广州市区陷入一片混乱,市民组织街道联防,在各巷口竖栅自卫。起初,市民的防御对象是四处抄家的老红卫兵。每当老红卫兵靠近时,市民便鸣锣示警,一齐呐喊鼓噪。到1967年6月后,市区已成为两派战场,市民精神高度紧张,不少人死于流弹。8月11日夜,因社会上风传有数千被释放劳改犯涌入广州抢劫,市民大举出动,将街头可疑生人及不会粤语者一律吊死。次日晨,吊在街头树上、电线杆上的尸体至少有180具,连市中心的北京路都出现吊尸,其中多有冤死的市民和流浪汉,是为著名的“吊劳改犯事件”。此事虽是一场惨剧,却体现了南粤社会的生命力\footnote{关于此事,可参看谭加洛:《文革中广州街头“吊劳改犯事件”调查》}。它表明,就算在中共已将南粤本土凝结核消灭殆尽时,粤人仍有强大的自组织能力。

当南粤文革暴力进入高潮时,澳门、香港左派大受鼓舞,先后发起暴动。1966年12月3日,澳门“一二·三”事件发生。澳门左派在中共党内激进派的支持下,发动罢工、罢市,同时中共方面切断了对澳门的食水供给。澳葡政府被迫作出妥协,接受澳门左派提出的赔偿、惩凶、不阻挠办学等要求。事后,葡萄牙政府在澳门的统治威信尽失,中共势力实际掌控澳门。左派在澳门的胜利,也促成了此后香港“六·七”暴动的发生。

196*6、1967年之交,部分岭北、广东红卫兵组织在解放军的默许下,越过罗湖边界,窜入香港,他们有的是激进而真诚的毛派分子,有的则是老红卫兵。受澳门“一二·三”事件鼓励,在中共、红卫兵鼓动下,香港左派、工人借新蒲岗塑胶花厂劳资纠纷案,仿照文革做法,自1967年5月起走上街头,打出“反英抗暴”、“解放香港”等口号,采取游行示威、飞行集会、张贴大字报乃至制造、使用“土炸弹”等形式,直接与港英政府对抗。到6月7日,暴动进入高潮,引发大规模骚乱。不少市民死于左派炸弹下,造成严重社会危机。

为应对香港左派暴动,英国政府一方面向中共外交部提出抗议,另一方面派遣航空母舰开赴香港。在英方协助下,港英政府宣布宣布香港进入紧急状态,实行二战结束以来首次宵禁,并出动大批军警平暴。英国与香港政府对待香港左派的强硬态度,激怒了中共外交部的造反派和北京的红卫兵。8月22日,他们以港英政府“疯狂迫害香港爱国新闻事业”为由,聚集在北京英国驻华代办处门前召开“首都无产阶级革命派愤怒声讨英帝反华罪行大会”。会后又以英方逾期不答复最后通牒为由,冲进英国代办处,火烧办公楼,导酿成中共建政以来最大之外交丑闻。

“火烧英国代办处”事发后,中共迫于国际社会压力,不得不作出让步,命香港左派停止暴动。据事后统计,1967年香港“六·七暴动”期间,约有51人丧生,其中中包括11名警察,一名英军拆弹人员及一名消防员。另有802人受伤,包括200名警察。约1936人在暴动中遭到逮捕、检控。暴动发生后,香港股市受到重挫,股指跌至历史最低点。部分市民则变卖财产离开香港,引发香港史上第一轮移民潮。虽然“六·七”暴动中,香港蒙受巨大损失,但此后香港左派却因其暴行为港人所认识,逐渐走向没落。暴动中英勇、忠诚的香港警队,获英女皇赐赠“皇家”封号。而暴动当年11月19日香港TVB开台,全天候播放粤语节目,则为香港此后本土文化的兴盛、南粤文明的复兴预埋了种子\footnote{关于香港“六七暴动”,可参看张家伟:《香港六七暴动内情》}。

1967年夏,广西武斗也进入高潮。由于韦国清支持联指压制四二二,四二二陷入与旗派相似的处境。在武斗中,中共广西当局多次向联指提供武器,南宁四二二唯有铤而走险,于8月18日抢劫军列,夺走中共援越炮弹数千发。但在“中央文革”的严令下,他们不得不于两日后交还大部分炮弹。自9月起,总派、联指开始指责旗派、四二二是受“黑五类”指使的反共组织,将屠刀对准“黑五类”。农村的中共民兵也闻风而动,开始杀害本村的“黑五类”。此后,保派组织在各地城乡不断杀戮无辜的“黑五类”平民。同月14日,因各邦保造武斗不断,社会秩序难以恢复、代替原省、市、县党委、政府的“革命委员会”迟迟不能建立,毛泽东发表讲话,称两派同属“工人阶级”,“没有根本的利害冲突”。这表明,在保派的强力抵抗下,毛不得不给文革降温了。在毛的压力下,南粤的两派唯有暂时“联合”。10月27日,旗、总两派代表在北京达成协议,决定成立联合组织“工人联筹”。11月9日,两派共十万人在越秀山体育场召开大会,宣告“工人联筹”正式成立,双方暂时休战\footnote{“工人联筹”全称“广州工人阶级革命大联合筹备委员会”,见刘国凯:《广州红旗派的兴亡》,页64}。然而,因韦国清不断指使联指攻击四二二,广西两派迟迟不能实现“大联合”。自1968年1月起,联指更调集大批农民进入各中小城市,将四二二成员及其家属指为“土匪”、“反革命”滥加杀害。3月29日,一批四二二领导更率一部投靠韦国清,广西造派分裂为保韦的“老四二二”和继续反韦的“新四二二”,实力大为受损。入夏后,仍被四二二完全控制的大城市仅剩桂林,南宁、柳州亦有部分城区在四二二手中。5月17日,韦国清更向中共中央提交一份报告,信口开河地说广西境内存在受台湾资助的“反共救国团”组织,这一组织的后台就是四二二\footnote{广西文革大事年表编写小组:《广西文革大事年表》,页65—75}。在广东,以黄永胜为主任的“广东省革命委员会”于2月21日成立,20名委员中只有一名旗派。黄因自身地位得到巩固,便指使由总派控制的工厂组织大批头戴钢盔的工人纠察队上街巡逻,基本控制了广东各城市。同月,汕头旗派惨遭总派镇压。自旗派崛起以来,广东仅有广州、肇庆、汕头、韶关及部分乡村出现过成规模的旗派组织,其余各地的中共党政军官僚大都很快运用列宁主义机器粉碎了造反派。至此,旗派几乎陷入绝境。5月22日,总派再次发动袭击,于广州供电公司杀死数名旗派。在随后的混战中,旗派组织广州二十二中“东方红”被围于公司大楼,便焚楼撤退。6月1日,大批军警在总派的配合下以军用云梯攻陷二十二中,消灭“东方红”组织。3日,忍无可忍的“中大红旗”展开反击,以机枪、炸药攻陷总派组织“革造会”在中大校园内的据点物理大楼,双方各死一人。11日,广州警察又在六榕路打死一名旗派学生\footnote{刘国凯:《广州红旗派的兴亡》,页92—93}。一场极度血腥的大屠杀,已近在眼前。

1968年7月,决定南粤造反派命运的时刻到了。由于南粤造反派有着强烈的民族起义军性质,一直保持着旺盛斗争,坚持与中共广东、广西当局对抗,已引起毛泽东的恐惧。在毛看来,造派是他用以打倒列宁主义官僚、建立毛主义新秩序的工具。但如果造反派的斗争性太强,威胁到中共统治的根基,他便决不能容忍。至于周恩来和林彪,更是对冲击列宁主义官僚和军队的南粤造派恨之入骨,极欲杀之而后快。是月3日,在毛的默许下,周恩来以中共中央名义发布文革史上著名的《七三布告》,针对南宁、柳州、桂林的造反派发出如下命令:

\begin{quote}

一、立即停止武斗,拆除工事,撤离据点。首先撤离铁路交通线上的各据点。\\
二、无条件地迅速恢复柳州铁路局全线的铁路交通运输,停止一切干扰和串联,保证运输畅通。\\
三、无条件地交回抢去的援越物资。\\
四、无条件地交回抢去的解放军武器装备。\\
五、一切外地人员和倒流城市的上山下乡青年,应立即返回本地区、本单位。\\
六、对于确有证据的杀人放火、破坏交通运输、冲击监狱、盗窃国家机密、私设电台等现行反革命分子,必须依法惩办\footnote{广西文革大事年表编写小组:《广西文革大事年表》,页99}。

\end{quote}

所谓有“确凿证据”的“反革命分子”,自然指四二二。而“依法惩办”四字更是充满威胁,给韦国清以大开杀戒的“尚方宝剑”。11日,韦国清命令广西各地、市、县召开“群众集会”,强迫民众为《七三布告》“欢呼”。同日,联指集会于南宁火车站,大肆庆祝《七三布告》颁布。韦国清亲临会场,动员“围歼阶级敌人”。15日,由韦国清又在南宁发表长篇讲话,要求“刮一场十二级台风”,“把一切反革命分子统统挖出来。”与此同时,大批联指人员涌上南宁邕江大桥游行,向附近的四二二据点挑衅。四二二忍无可忍,开枪射击,联指2死2伤。韦国清随即宣布,这是“阶级敌人对抗中央《七三布告》”的“反革命事件”。当天下午,大批共军和由各地汇聚而来的联指分子包围了南宁城内的四二二控制区,决战一触即发\footnote{广西文革大事年表编写小组:《广西文革大事年表》,页105—106}。据一名造派幸存者回忆,当时四二二控制区的情形是:

\begin{quote}
四二二一派在南宁控制的地方,除广西大学等高校与因历史原因而成为四二二总部所在地的孤立据点展览馆以外,连片的“解放区”(造派控制区)基本上是南宁市下层市民集中居住的老市区,如解放路、华新街、上国路、西关路等处。这里房屋老旧,好的是古老的骑楼,差的则为砖木结构陋房乃至棚户区,其居民原来多从事传统行业,三教九流,历来被上层社会视为“情况复杂”之地……在上述“解放区”里,居民对意识形态并不感兴趣,满街的大字报多为当事人或与当事人有关。诸如某某领导欺男霸女,某某官员挟私陷害,某某小民冤案莫伸,某某百姓负屈莫诉等等。而他们的群体要求则多有十分明显的利益指向:临时工、合同工要求转正,下乡知青要求返城等等。

“解放区”的社会、经济状况也出乎我的想象。一般都认为造反派是极左的教条主义信徒,然而在造反派控制的这片地区,正规计划经济色彩十分淡漠,“江湖经济”则熙熙攘攘,十分热闹。“解放区”的中心水脚塔地区赫然一片在工棚式临时建筑中开业的私营餐馆,号曰“南疆饭店”,这大概是“三大改造”之后城市中从未有过的经济景观。临近街巷中,摆小摊的、江湖卖药行医的、兜售各种自印奇方秘诀的、甚至算命的与赌博的,林林总总,不一而足\footnote{秦晖:《沉重的浪漫——我的红卫兵时代》}。

\end{quote}

此段生动的描述将南粤造反派的民族起义军性质展现无遗。四二二控制区的景象告诉我们,只要列宁主义机器稍有松动,南粤人便能爆发极强的自组织能力,建立生机勃勃的社会。在联指中的岭北侵略者和列宁主义走狗看来,这一切自然是不可容忍的。7月15日下午起,共军和联指猛烈炮击解放路,南宁之战打响。至17日,经连续猛轰,解放路、汉乐街、上国街、自强街、灭资路、民生路已全部化为火海,大片民房、商铺熊熊燃烧,将无数军民化为焦炭,邕江上的大批船只也被炮弹点燃。与此同时,联指开始在全市范围内挨户搜杀四二二成员,将恐怖与死亡带给南宁城。19日,共军、联指的炮击更为猛烈。中共南宁警备司令部、政治部赶忙发出公告,无耻地说老城区的大火是四二二点燃的。2*1、27日,共军、联指以高射机枪、火箭炮、无后坐力炮两次轰击百货大楼,使二楼、三楼起火燃烧,附近房屋全部被毁。中共南宁革委会、南宁警司和联指又发布广播,说四二二自己爆破了大楼。28日,战斗蔓延至南伦街、华强路、自强路,将三条街道化为灰烬。此战,四二二顽强抵抗,毙伤共军、联指多人。南宁当局、联指便再造谣言,说部队和“无产阶级革命派”在掩护“群众”救火时遭到炮击。至31日,南宁四二二除展览馆外、解放路已丢失全部据点。共军以七个连兵力会同大批联指分子,于当天下午3时总攻展览馆。8月1日上午8时,展览馆陷落,共军、联指又以重兵包围解放路,于3日发起猛攻。8日,解放路失守,南宁之战结束。据中共于1983年发布的不完全统计,共军、联指在仅在展览馆、解放路两地就杀害四二二成员1470人、俘虏9845人,被俘者中有大批平民。共军、联指随后对战俘施以灭绝人性的杀戮、虐待,杀害其中2324人。这还不是遇难人数的全部,因为南宁有三十三条街道在此战中化为废墟,葬身其中的军民总数根本无人统计\footnote{广西文革大事年表编写小组:《广西文革大事年表》,页105—116}。战斗结束后,有3000—7000名四二二成员躲入地下人防工事顽强抵抗,韦国清见强攻不成,竟丧心病狂地下令打开邕江上游左江水电站拦河大坝的水闸,放水灌城,将藏身地下的数千人全部淹死,数千座民房也被淹没\footnote{《韦国清南宁屠城记:广西四二二文革鲜为人知的屠杀》}。这样,仅做最保守估计,被共军和联指屠杀的南宁军民也有数万之多。这不是常规战争,是灭绝人性的屠城!

下一个被血洗的城市是桂林。桂林是由造派控制的城市,支持四二二的组织“老多”在此有相当实力。在1968年7月至8月初的武斗中,“老多”曾顽强抵抗,击毙联指198人,令联指恨之入骨。8月5日,“老多”宣布服从《七三布告》,主动交出枪支,拆毁据点,但中共没有放过他们。17日,在黄永胜的直接命令下,中共桂林军分区召集各县联指、民兵头目开会,军分区副司令吴华高声叫嚣,说“桂林问题肯定要解决,一小撮阶级敌人一定要搞干净。”20日,近万名共军、联指分子、工纠队及武斗人员组成的“毛泽东思想宣传队”开入桂林市区,如打猎一般搜捕造反派。不久后,整个桂林地区的大搜捕全面铺开,处处皆是腥风血雨。至25日,桂林市区的搜捕完毕,共有数千人被捉,下属各县则一直持续到10月。被捕者多遭惨绝人寰的批斗与虐待,被枪杀、砍杀及酷刑折磨死者不计其数。共军、联指分子不但杀戮造反派,也将城批“黑五类”全家杀绝,连三岁的孩子都不放过\footnote{晓明:《青山垂泪 漓水悲咽:文革中桂林武斗及大屠杀的回顾与反思》}。据中共于1983年公布的数据,在1968年8月20日至10月间,桂林地区有9097人被杀。毫无疑问,这一数据也是大大缩小的\footnote{广西文革大事年表编写小组:《广西文革大事年表》,页105—116}。

在柳州,更令人发指的暴行出现了。支持四二二的柳州造反派自称“造反大军”,活跃于武宣县。《七三布告》公布前,柳州联指便已动手,于5月13日大举进攻“造反大军”。造反派节节败退,成批被俘。从6月15日起,穷凶极恶的联指民兵开始食人。他们用刀将俘虏的腹腔剖开,取食心、肝、肉,被食者有男有女、有老有少,甚至连造反派的同情者都不放过。一名中学副校长仅仅因为同情“造反大军”,就被一群联指民兵在学校操场里割去所有内脏和肉,只剩一副骨架。联指分子随后在学校宿舍下烘烤人肉人肝,“大快朵颐”。一个叫陈文留的女民兵队长不但吃了6副人肝,还极度变态地割下5名男俘的阴茎泡酒喝。7—8月间,屠杀更为凶残,大批来自桂平、贵港的联指民兵涌入柳州地区,疯狂捕杀造反派和“黑五类”。据中共于1983年公布的数据,仅柳江、武宣、融安三县就有至少1290人被杀、武宣县至少75人被吃\footnote{小平头、问青天:《广西融安大屠杀》}。一时之间,桂柳大地哀鸿遍野,呈现一片率兽食人的阴森场景!

大屠杀发生时,在南宁市区和桂柳地区之外,广西造反派几乎没有势力,但共军和联指没有放过当地百姓。在他们看来,虽然那里没有四二二,但仍有不少“黑五类”。因有不少“黑五类”参加造反派,他们决定进行一场大屠杀,以防“阶级敌人”再次造反。粤西钦州本属广东,是大英雄陈济棠的故乡。1965年,中共将钦州地区划入广西,当地在1967年后成为联指的天下。1968年9月,钦州大屠杀爆发。联指人员将当地超过2万与四二二有瓜葛者及“黑五类”逮捕,展开灭绝人性的折磨。在灵山县,民兵用刀棍和农具整村地屠杀“黑五类”。在上思县,该县革委会于上思中学召开批斗大会,当场斗死12人,并将死者剖腹取肝,由县、公社领导分食。该县思阳公社武装部长王召腾更下达命令,要求每两三个生产队分食一人肝脏,以致“共同专政”。除食人外,凶手中多有性变态者。在东兴县,他们将木棍、鞭炮放入女俘阴道,进行极度残酷的虐杀。在浦北县,联指分子将造派同情者和“黑五类”几乎全部屠戮,甚至将十七岁的女孩轮奸后割阴割乳、取肝杀食。在吃掉男受害者的肝脏后,凶手还强迫死者妻女改嫁,勒索所谓的“改嫁费”。不用说,许多人都被杀夫、杀父凶手霸占了。将据中共事后公布的数据,在整个钦州地区,被杀者多达10420人\footnote{张立平:《广西文革灭绝人性的人吃人事件》}!除上述地区外,南宁的隆安县、武鸣县、上林县、大新县及玉林地区的贵港都发生过大屠杀和食人事件。在中共的屠刀下,八桂、粤西阴风滚滚,其悲惨程度就连地狱都比不上。如果说那些凶手是禽兽,便是侮辱了“禽兽”二字,因为连禽兽都不会故意取食同类的肝脏、割食同类的生殖器!

在广东,大屠杀也开始了。1968年7月13日,全副武装的军警闯入同情旗派的第二十九中,当场扫射无武装学生。十几人倒在血泊中,军警扬长而去。15日,军队包围旗派据点北京路丝绸商店,附近市民纷纷向坚守其中的旗派学生送水送食,痛骂军队,遭军队开枪射击,一名眼镜店学徒被杀。16日,军队清除长堤路旗派据点轻工大楼,开枪乱射,造成多名市民死亡。17日,军队又清除中山六路旗派据点,在未遇抵抗的情况下wu 乱枪杀市民,造成多人死亡。有个路人仅说了一句“解放军不应开枪”,就被拖出枪杀。至*7、8月之交,大批工人纠察队在居委会的配合下展开全城搜捕,将数以千计的旗派人员和“黑五类”拉上街头游斗。被斗者在烈日下头戴沉重的铁帽,像牲口一样被捆在一起,一路遭工纠队棍击、殴打,被当场打死者不计其数,更多人被打断肋骨、腿骨,留下终身残疾。在反复游斗后,俘虏就被押进称为“牛栏”的各单位私牢,继续遭受残酷的肉体、精神折磨,被打死、自杀者成百上千。由于杀人太多、太随意,广州城内在这段时间的死者人数根本无人统计。7月31日,由3万名军人、工人、农民组成的“毛泽东思想宣传队”涌向广州旗派最后、最主要的据点中山大学,实施“无产阶级专政”。他们全副武装,于上午冲进学校正门。缺乏武器、训练的旗派几乎毫无还手之力,很快溃败。到中午,宣传队已深入学校腹地,占领各主要教学楼,将敢于反抗者全部击倒。到下午,他们又闯进宿舍区,逐屋搜捕旗派。入夜时分,只有中大北门外的码头还在旗派手中。十余名残存旗派学生在此乘上快艇,逃出广东。到天亮时,一切都结束了。在中共的屠刀下,红旗派毁灭了\footnote{刘国凯:《广州红旗派的兴亡》,页6—9}。

在广东各地,针对旗派同情者和“黑五类”的大屠杀也开始了。广东的屠杀规模虽不如广西,但仍使人毛骨悚然。据中共公布的数字,在广东57个县中,有28个县发生了集体屠杀,其中有6个县被害者超过1000人。在情况最严重的阳春县,有2600人在1968年8—10月间被枪杀、木棍打死、推入水中淹死或活埋,遇难者上至75岁的白发老人,下至仅有4个月大的襁褓婴儿\footnote{李文光:《春砂赋》,页41—42}。在粤东澄海县,共军与总派分子于1968年7月武力清除旗派各农村据点,仅在苏南公社就打死打伤430多人,甚至对俘虏先以长矛乱戳、灌硫酸入口,再用斧头砍死\footnote{《论汕头澄海武斗事件与“余林反革命集团”等冤案》}。在五华、蕉岭、怀集、广宁、儋县等地,总派在中共县武装部的指挥下疯狂屠戮旗派同情者和“黑五类”,将死者尸体吊在村口。在粤东北丰顺县,“黑五类”被中共民兵以数百人为一群集中于山凹,遭机枪成批屠杀\footnote{刘国凯:《广州红旗派的兴亡》,页93—94}。在粤北曲江县,“黑五类”被以数十人为一批活活烧死\footnote{苏阳:《文革中的集体屠杀:三省研究》}。

对于这场惨绝人寰的大屠杀,中共高层如何看待?1968年7月25日,周恩来、黄永胜、吴法宪、康生、姚文元等人在北京召见广西两派代表。这些人中,周是列宁主义官僚头目,黄、吴是林彪麾下干将,康、姚则是紧跟毛泽东的文革派。可以说,他们的态度代表了整个中共高层的态度。在会上,各中共头目一致猛烈抨击四二二,说四二二烧了南宁城区,“两广地区”有“反共救国团”,总团在广州,分团在广西。黄永胜还故意点了旗派领袖武传斌的名字,贼喊捉贼地说他“到处煽风点火,挑拨离间”\footnote{刘国凯:《广州红旗派的兴亡》,页5}。至此,南粤造反派便被中共各派一致定性为“反共救国团”,再无出头之日。得到毛、林、周的一致许可后,黄永胜和韦国清遂能肆无忌惮地展开大屠杀,无需担心受任何追责。

在这场惨绝人寰的大屠杀中,究竟有多少南粤人遇难?据中共于1984年公布的数字,广西在文革期间有119700人“非正常死亡”、2万人失踪,其中死于两派武斗者仅3700人,至少有79000万人是被有计划、有组织地屠杀的。这一数字显然被极大地缩小了,因为在中共当局于1980年代初所作调查中,曾有本地干部、民众反映,广西在文革中死了20—50万人\footnote{晏乐斌:《文革广西死了多少人》}。若从中共将“恐怖的红八月”死亡人数由10285改为1722这一点推断,则死于文革的广西人恐怕不止50万。他们中的绝大多数都是造反派和“黑五类”,死于1968年的血腥屠戮。1968年8月11日,韦国清出任广西革委会主任,随后对联指杀人凶手论功行赏。据事后统计,竟有2.7万多人是在杀人后“突击入党”的\footnote{小平头、问青天:《广西融安大屠杀》}。中共政权的食人本质,于此暴露无遗。在广东,死亡人数较广西为少。据美国学者苏阳对中共广东县志所作统计,在发生大屠杀的29个县中,共有7784人死亡\footnote{苏阳:《文革中的集体屠杀:三省研究》}。这必然是个极大缩小的数字,因为据另一美国学者莫里斯·迈斯纳研究,在血腥的1968年,被共军和总派处决的旗派就有4万人\footnote{莫里斯·迈斯纳:《毛泽东的中国及后毛泽东的中国:中华人民共和国史》}。具体的死亡人数恐怕是个谜。但无论如何,在1966—1968年间,数十万南粤人被中共以极度残酷的手段屠杀了。

在文革中,南粤是发生如此大规模屠杀的唯一一国。正因为南粤造反派有着强烈的民族起义军色彩,方令毛泽东、林彪、周恩来感到由衷恐惧,使他们搁置彼此矛盾一致屠杀粤人。红旗派和四二二失败了,但他们的战斗精神永不会消亡,他们利用中共内讧奋起反抗的壮举永不会被人遗忘。他们的前身是南粤史上的无数男女英雄,他们的继承者是今天仍在战斗的千百万南粤同胞。在他们前仆后继的冲击下,列宁主义侵粤机器被严重动摇。此后,南粤造派残部转入地下活动和游击战,继续战斗,向世人宣示南粤人永不屈服。直到1982年,广西的最后一支四二二游击队才在河池地区被共军彻底镇压\footnote{最后坚持抵抗的是名叫韦明乐、韦明成、韦明立的三兄弟。见广西文革大事年表编写小组:《广西文革大事年表》,页123}。这时,距四二二的诞生已有十五年之久。

1968年7月,毛泽东派出由军队、工人组成的“解放军毛泽东思想宣传队”、“工人毛泽东思想宣传队”进驻北京各高校、机关,将北京红卫兵头目逮捕。此后,文革运动迅速降温。9月,“中华人民共和国”的29个直辖市、省、自治区皆建立被称为“新生红色政权”的革委会,文革初期的夺权阶段结束。其中,有包括广东、广西在内的20个省、区革委会主任由军人出任,以林彪为首的军队系统在中共党内迅速崛起。10月,刘少奇被定为“叛徒、内奸、工贼”开除出党。12月22日,毛泽东下达“最高指示”,要求知识青年全部“到农村去,接受贫下中农的再教育”。“全国”196*6、196*7、1968三届学生从此离开校园“上山下乡”,被流放至农村,红卫兵组织的土壤就这样被铲除了。仅在广东,就有34.97万城镇青年在1967—1972年间“下乡”,广西被迫下乡者也有20多万人,其中包括的大批的原造反派。他们有的被派往湛江、海南等地垦荒,还有的则被送到岭北的冰天雪地与荒原中,从此一去不返。他们在农村的人民公社中与农民一同被中共干部残酷地虐待、欺辱,大都过着牛马不如的生活。仅1968—1973年间,全广东被披露出的奸污猥亵女知青案即有675宗、毒打残害知青案有345宗,而这不过是全部案件中的九牛一毛。许多南粤青年就这样丢失了生命,再也无法回家\footnote{罗国建:《广东知识青年上山下乡史》(中)}。

1969年4月,中共九大在北京召开,林彪被指定为毛泽东接班人,周恩来成为中共三号人物,中共党内的新秩序诞生。1970年1月,中共中央发动“一打三反”运动,重点打击所谓的“反革命破坏活动”。林彪视之为打击残余造反派势力、巩固军队权力的绝好机会,提出“杀!杀!杀!”的恐怖口号。中共南粤当局闻风而动,对粤、桂的原旗派、四二二人员展开又一轮残酷迫害。截止到1971年6月运动收尾时,广东已有48721人被打成“反革命”。广西的总体数字则缺乏统计数字,仅在1970年2月,桂林一地就就有602人被打为“历史反革命”、“现行反革命”。他们大都被关押、毒打、判刑,不少人被共军枪杀\footnote{广西文革大事年表编写小组:《广西文革大事年表》,页151—152}。1970年3月,中共中央又发动“清查五一六”运动,意图定点清查一个名叫“首都五一六红卫兵团”的北京反周恩来组织。以林彪、周恩来为首的中共军、党干部遂趁机再次整肃造派。在韦国清授意下,中共广西当局称四二二的后台就是“五一六兵团”,所以“从一开始就要打倒韦国清同志”。直到1974年,“清查五一六”才不了了之。而在此之前,又有成千上万的南粤造反派被枪杀、毒打、判刑\footnote{广西文革大事年表编写小组:《广西文革大事年表》,页153—154}。然而,就在这无边的黑暗中,南粤人也从未失去反抗的勇气。在广西的群山间,四二二残部仍在勇敢地与共军周旋。在广州城中,当过“主义兵”的南下干部后代仍时常与本地青年冲突。本地青年结成帮派,勇敢地教训这些满手血债的家伙。一篇回忆文章曾yeusi 描述:

\begin{quote}

1969年中,广州已逐渐形成几大群体,比如“西关山鹰”、“河南小八路”、“东山海鹰”……(西关)山鹰的名称来源于1969年广州15中学(德政北路,未搬迁)一支篮球队。这篮球队很快就与附近省实中学的“火炬”组织合并,从事着许多和篮球无关的事。接着又兼并芳村一个小团伙和西关宝华路一带的团伙。1969年中,西关山鹰因小事要和东山海鹰“摆场”(打群架)。东山海鹰是以外省籍中学生(军队、铁路子弟)为主,成员身材高大,全部军装。西关山鹰是广州本地中学生(个别是回城知青),衣服破烂参差。当天傍晚双方到达越秀公园五层楼附近,开局前大家亮出武器,东山海鹰的全部是军用刺刀、人称“洪常青”的大刀。西关山鹰亮出的是短水管、菜刀、斧头。在士气上西关山鹰完全占上风,战斗欲望强烈。东山海鹰张姓头目见势不妙,建议“只抽”(头目之间单打独斗)。端着军用刺刀站在阵前。西关山鹰的一个名叫细B(1953年出生,当时未满16岁)的瘦小个子端上斧头面无惧色也站在阵前。在个人气势上也压倒对方。东山海鹰李姓头目再次有点心怯。提议不带武器“只抽”,细B同样答应。双方交手不到一分钟,细B大获全胜把对方打倒在地。东山海鹰完全败下阵来,扔下部分刺刀和50元悻悻而退\footnote{引自《文革后期广州的流氓团伙》}。

\end{quote}
	
南粤社会在黑暗中涌动的生机,由此清晰可见。1971年,就在军队系统不可一世之际,剧变发生。是年9月13日,林彪突然乘飞机逃往苏联,半路因飞机失事死于蒙古温都尔汗。韦国清因紧跟毛泽东未被株连,身为林系干将的黄永胜则难逃一劫。林彪死后不久,黄永胜即被撤除一切职务,后于1973年8月被开除党籍、送入牢房。直到1984年病死为止,双手沾满粤人鲜血的黄永胜都没有恢复自由,在监禁中度过了悲惨的残生,获得了应有的下场。林彪一向亲苏。在排除林彪的干扰后,毛泽东可以放手发动外交革命了。1972年2月下旬,美国总统尼克松访问北京,与中共发表《中美联合公报》,宣言双方将于1979年3月1日正式建交。至此,中共已在毛的骑劫下彻底摆脱苏联,向亲美之路狂奔。在与西方隔绝近四分之一个世纪后,南粤迎来了再次与西方世界交往的机会。













