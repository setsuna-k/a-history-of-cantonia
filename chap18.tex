\chapter{决断时刻:“广东人之广东”还是“中华民族”}

\section{对南粤未来的第一种规划:“中华民族”}

\indent 19、20世纪之交,东亚风云诡谲。自19世纪末起,愚昧腐朽的清帝国日渐落后于世界大潮,显现出不堪一击的面目。在清帝国行将瓦解之时,南粤应当抓住历史机遇寻求自立,还hay 继续被岭北帝国纠缠、与之一同坠入深渊?在此亟需决断的时刻,这一事关南粤命运的问题引发了大批粤人的思考。他们从各种立场、各种角度出发,为南粤在未来的命运提出了许多迥异的方案。他们中最重要的四人,是康有为、梁启超、孙文、欧榘甲。

在论及康有为之前,我们需首先了解他的老师朱次琦。在19世纪的南粤,若论有哪位大儒的学问可与陈澧相提并论,那无疑便是被称为“九江先生”的朱次琦。朱次琦,南海九江镇人,1807年出生于当地一个商人家庭。他的六世祖在17世纪时因抗清殉节,这一光荣的家族传统令朱氏罕有出仕清帝国者。不过,由于朱次琦自幼便聪慧好学、熟读诗书,他仍希望参加科举考试求得官职,从而施展自己的学识。1847年,朱次琦考中进士,其后前往晋地襄陵任知县,在当地积极治水,深受百姓爱戴,谱写了一段粤晋友谊的佳话。然而,清帝国官场的浊乱令朱次琦心灰意冷。仅仅为官半年,他就不顾当地百姓的挽留毅然回乡,从此发誓不再踏入城市一步。他在家乡设立九江书院,招收有志于学术的南粤青年为徒,每日传授学问、居住于竹屋中,数十年如一日地过着简朴的生活,致力于为南粤培育英才,并拒绝了学海堂对他的招揽。他认为教育学生时当兼重德才,且以德性为重。为此,他专门提出被合称“四行”的“惇行孝弟”、“崇尚名节”、“变化气质”、“检摄威仪”做为人的行为准则,希望他的学生先具备“四行”再讲求学问,并以“孔子之学”救“治道之失”。1882年,朱次琦以76岁高龄在家乡去世。离世前夕,他将自己的书稿尽数焚烧,以表明其无意于留名后世、只欲为南粤培养人才的态度\footnote{关于朱次琦的生平和学术主张,参见白红兵、唐棱棱:《学术、政治与意识形态:中国近代社会转型与文学变革研究》,页31—46}。做为与陈澧齐名的大儒,他淡泊名利、一意扎根于乡土,其流风余韵至今仍动人不已。

朱次琦一生弟子颇多,其中最著名者为简朝亮和康有为。简朝亮,顺德简岸乡人,1851年生,世仁称其“简岸先生”。他曾在于1875年师从于朱次琦,后因五赴乡试皆落第,乃绝意于仕进。朱次琦去世后,简朝亮继承其师之学,用数十年时间注释《尚书》、《论语》、《孝经》,宣扬其反对激进政治、社会变革的思想,力图以此“正人心,挽世风”,并在家乡授徒讲学,他的学生中便包括此后在南粤民族发明史上掀起轩然大波的黄节。1933年,简朝亮安然去世,享年83岁\footnote{关于简朝亮的生平,参见王钊宇总纂:《岭南文化百科全书》,页536—537}。若说简朝亮完整继承了朱次琦的学术观点和保守主义态度,那么大名鼎鼎的康有为便无疑是个师门叛徒。1858年3月19日,康有为出生于南海县丹灶苏村。他的家族是当地的著名士绅,其曾祖康式鹏曾为清福建按察使,祖父康赞修曾中举人,父亲康达初则在江右做过候补知县。因出身于这样一个官宦家庭,幼年的康有为自然受到了良好的儒学教育,笃信程朱的祖父康赞修是他的启蒙老师。1875年,十八岁的康有为前往九江师从于朱次琦,继续钻研理学。当时,南粤正处在文明开化的进程中,各种新奇的西方思想大量涌入,令年轻的康有为为之躁动不已。在朱次琦门下学习了四年后,私智超群的他认为理学仅讲“修己”,不讲“救世”,便抛弃了恩师,独自前往西樵山白云洞读书。在安静的山中,康有为阅读了一批明清之际学者的经世学著作,并抽空去了一次香港,为英人治下香港城邦的繁荣深感震惊。此后,他“购地球图,渐收西学之书”,立志要打通东西方学问。1882年,康有为走出西樵山赴北京参加会试,落榜后途径上海,购置了一批关于欧洲国家政治制度、自然科学的书刊。到1885年时,因思想体系已日渐成型,康有为开始动笔规划他心目中的“人类理想社会”,将之命名为《人类公理》。此后经不断完善,该书于1902年最终完成,改名为《大同书》。该书以《春秋公羊传》中提出的“三世”之义为基础,认为人类社会的发展分为“据乱世”、“升平世”、“太平世”三个阶段,而“太平世”正是儒家经典《礼记》之《礼运大同篇》中所提到的“大同”。当然,康有为所规划的“大同”世界绝非《礼运》中所讲的“天下为公”那么简单。事实上,他设计的是一个极度阴森、恐怖的社会,其要点为:

\begin{quote}
	一、无国家。全世界置一总政府,分若干区域。\\
	二、总政府及区政府皆由民选。\\
	三、无家族。男女同栖不得逾一年,届期须易人。\\
	四、妇女有身者入胎教院,儿童出胎者入育婴院。\\
	五、儿童按年入蒙养院,及各级学校。\\
	六、成年后由政府指派分任农工等生产事业。\\
	七、病则入养病院,老则入养老院。\\
	八、胎教、育婴、蒙养、养老诸院,为各区最高之设备,入者得最高之享乐。\\
	九、成年男女,例须以若干年服役于此诸院,若今世之兵役言。\\
	十、设公共宿舍公共食堂,有等差,各以其劳作所入自由享用。\\
	十一、警惰为最严之刑罚。\\
	十二、学术上有新发明者,及在胎教等五院有特别劳绩者,得殊奖。\\
	十三、死则火葬,火葬场比邻为肥料工厂\footnote{梁启超:《清代学术概论》,页76—77}。
\end{quote}


此种规划的激进程度甚至已超越了日后的大部分激进左派,其将死者骨灰作为肥料的设想更是连红色高棉都未曾实施过的。不过也需要指出,康有为并不认为“大同”社会是要立即建成的。在此之前,首先应改革政治制度,建设等同于“升平世”的“小康”社会\footnote{梁启超:《清代学术概论》,页78}。至于他改革的对象绝非则其母邦南粤,反而是残酷压榨南粤的清帝国。1888年,康有为前往北京向光绪帝递交《上清帝第一书》,提出“变成法”、“通下情”、“慎左右”三条纲领,未得任何回应\footnote{黄晶:《中国名人大传:康有为传》,页47—48}。康有为随后返回南粤,在广州的徽州会馆谋得一份教书差事,意图择机再起。在此期间,康有为构思出一套“孔子改制”论,提出儒家六经实为孔子所作,孔子之理想乃是“托古改制”。1890年三月,精通考据学的南海士人陈千秋因慕康有为“上书”光绪帝之名与之相见,对其“孔子改制”说大为折服,遂拜之为师。六月,在陈千秋的介绍下,康有为最重要的门徒梁启超亦投入康门。1891年七月,在陈、梁二人的协助下,康有为完成了讲述其理论的《孔子改制考》一书。同年,经陈、梁二人提议,康有为开始在广州开堂讲学,许多南粤士人纷纷慕名而来。1893年冬,康有为将学堂设于广州府学宫仰高祠,正式悬挂“万木草堂”匾额,是为南粤和东亚历史上著名的“万木草堂”\footnote{李济琛主编:《戊戌风云录》,页225}。

在万木草堂中,康有为刻意模仿孔子,狂妄地自命“圣人”,将康门比作超越孔门的学派,呼陈千秋为“超回”、梁启超为“轶赐”,其他学生亦获得“越伋”、“乘参”之类的称号。1894年春,康、梁师徒二人赴京参加会试。在答卷中,康有为称自己为“大于孔子者”,意态极为狂傲。他奇特的行为终于引起了清廷的注意。在清廷的命令下,康有为的另一著作《新学伪经考》遭到禁毁,粤士亦被禁止从其问学。不得已之下,康有为只得于六月返粤。风波平息后,他又于1895年春与梁启超再赴北京会试。当年4月17日,清帝国在日清战争中战败,日、清代表伊藤博文、李鸿章签订《马关条约》。此消息传到北京,立即引起轩然大波。当时,大批举人正在北京等待考试结果。条约签订后不到一个月内,即有清帝国各地的上千名举人、官员“上书”清廷,要求毁约再战。5月1日,梁启超在康有为的命令下发动80余名广东举人“上书”。与此同时,另一位来自南粤的举人陈景华(香山南屏人)也组织了一次广东举人的集体“上书”,参加者多达289人。官员、举人们的“上书”自然没有得到清廷的正面回应。然而,康有为却利用此事大肆炒作,将自己包装为所谓的“公车上书”领袖。在四年后写下的回忆录《我史》中,康有为大言不惭地称参与“上书”的千余名举人是由自己发动起来的,这些人的集体“上呈”的要求拒和、迁都、变法的“万言书”也是由他撰写的,该书更曾由他本人亲赴清帝国都察院投递,但被都察院拒收。这一历史发明完全漏洞百出,因为都察院官员正是清帝国官僚中最支持再战的群体。现存的清帝国档案也证明,都察院从未拒收过举人们要求再战的“上书”,而这些“上书”中从来没有一封是由康有为撰写的。这表明,康有为本人根本没有投递过所谓的“万言书”,这一切都是他精心制造的骗局\footnote{关于康有为伪造“公车上书”历史的骗局,学界已有深入研究、考证,参见茅海建:《从甲午到戊戌:康有为"我史"鉴注》}。

不久后,康有为中会试第五名,又在殿试中名列二甲,被任为工部主事,成为清帝国的京官。自命“圣人”的他当然不甘心止于此。同年7月,康有为与梁启超在北京创办政治刊物《中外纪闻》。该报为双日刊,面向在北京的清帝国官员,其内容则主要为围绕“公车上书”展开的宣传及对为敦促清帝国“变法”而进行的鼓吹。11月,康、梁又在北京成立政治团体“强学会”,得到清翰林学士文廷式(潮州人,江右移民二代,陈澧弟子)的大力支持。“强学会”的宗旨在于促进清帝国“变法”,为清帝国“救亡图存”。在为该会所写的纲领性文件《强学会序》中,康有为如是说:

\begin{quote}

俄北瞰,英西睒,法南瞵,日东眈,处四强邻之中而为中国,岌岌哉!况磨牙涎舌,思分其余者,尚有十余国。辽台茫茫,回变扰扰,人心惶惶,事势儳儳,不可终日……举世界守旧之国,盖已无一瓦全者矣\footnote{转引自李济琛主编:《戊戌风云录》,页302}!

\end{quote}
	
此段纲领中,康有为绝无半点保卫南粤的意识,所思所想唯有清帝国的存亡,可谓十足的南粤叛徒。强学会成立后,北京的不少官僚、士人慕名加入,在其中议论时政。康有为则前往南京,成功说服清署理两江总督张之洞成立上海强学会,风头一时无两。强学会的活动终于引起清廷震动。1896年1月12日,慈禧太后以光绪帝名义下令禁止强学会议政,张之洞随之要求上海强学会解散,康有为的政治投机又告失败\footnote{白寿彝:《中国通史》第11卷,页1169}。次年11月,德国向清帝国强租胶州湾。康有为借此再掀波澜,于1898年1月向光绪帝“上奏”要求“变法”,并“进呈”其所撰《日本明治变政考》、《俄罗斯大彼得变政记》两书,希望光绪帝效仿日本明治天皇、俄罗斯彼得大帝展开改革,制造一个清帝国版本的明治维新。此外,康有为又积极联络旅居北京的各邦人士,于1898年4月12日在北京成立新政治组织“保国会”,提出“保国”、“保种”、“保教”的政治纲领。至于保国会所保之“国”、“种”,自然是所谓的“中国”、“中国人”,绝非南粤、南粤人。此后一个多月内,保国会先后在北京举行三次集会,要求清帝国迅速“维新变法”。以大学士荣禄、军机大臣刚毅为首的清帝国高官纷纷对保国会口诛笔伐,称其乃“揽权生事”、“形同叛逆”的乱党。压力之下,保国会被迫解散\footnote{张瓊:《中国名人大传:梁启超传》,页88—90}。不过,年轻激进的光绪帝却在ni 时向康有为伸出了援手。同年6月10日,光绪帝发布《明定国是诏》,起用康有为、梁启超及一批主张维新的官员发动“戊戌变法”,在教育、经济、军事、政治等发面发出一连串激进的指令,试图将清帝国变成第二个明治日本。除湖湘外,清帝国各地的封疆大吏皆对来自北京的指令视若无睹,保守官员更是对变法大加挞伐,慈禧太后亦担心变法实为光绪帝削弱其影响力的手段。9月19日,慈禧太后发动政变,软禁光绪帝于中南海瀛台,并下令通缉变法人士。9月28日,包括康有为亲弟康广仁、湖湘名士谭嗣同在内的六人在北京菜市口遭斩首,康有为、梁启超则分别在外国使节的帮助下分别逃至加拿大和日本。在加拿大,康有为不改其善于欺骗的卑鄙做派,伪称自己持有光绪帝的“衣带密诏”,蛊惑了一大批对清帝国严重不满,但又未有与岭北大一统帝国进行身份切割之意识的北美粤侨。1899年7月20日,康有为在英属哥伦比亚省组织“保皇会”,大肆宣传“奉诏”讨伐慈禧、营救光绪帝的主张,依靠招摇撞骗聚敛钱财和名声\footnote{关于戊戌变法的详细经过,学界研究颇多,本书不拟详述,仅述其大略。}。此后,康有为日益沦为一个跳梁小丑似的角色。在剩余的人生中,他又多次在岭北进行政治投机,屡屡失败。在这些投机活动中,他从未为南粤的自由、自立说过一句话。晚年的康有为依靠骗来的钱财过着荒淫无耻的生活。1922年,62岁的他纳了一名19岁的小妾。七年后,深感力不从心的他在上海求助于德国医生,将猴子的睾丸移植到自己身上。这一荒诞的举动很可能是他的直接死因。1927年3月31日,69岁的康有为在胶州七窍流血而死,这一背叛南粤、一生招摇撞骗、为岭北大一统强权出谋划策的游士终于得到了与其自身德性相匹配的下场\footnote{康有为在死前确曾接受过德国医生的“返老还童术”,但他到底是否换过睾丸仍有争议,详见谌旭彬:《康有为有无移植过猴子睾丸?》}。

与康有为相比,梁启超的人生更为复杂。1873年2月23日,梁启超出生于新会县茶坑乡的一户耕读人家。他的祖父梁维清是个秀才,父亲梁宝瑛则因屡试不中而将希望寄托在爱子身上。梁启超自幼聪明好学,是个“八岁学为文,九岁能缀千言”的神童。1884年,年仅十一岁的他考中秀才。三年后,他又入读广州学海堂,在其中保持着“季课大考,四季皆第一”的好成绩。1889年,十六岁的梁启超顺利通过广东乡试,以排名第八的佳绩获得举人头衔。中举之后,他返回学海堂继续学习,并开始备考会试。然而,一路顺利的他却在次年的会试中意外落榜。意兴阑珊的梁启超重返学海堂,陷入迷茫中。这时,他在朋友的介绍下结识了已投入康门的陈千秋,进而与康有为相见。那时的梁启超虽然满腹东亚传统学问,但对世界的概念很模糊,只知地球上有五大洲各国。康有为“学贯‘中’西”的知识结构使梁启超大为震撼,两人很快结为师徒,一同建立万木草堂。康有为最初以陈、梁二人为万木草堂学长。1894—1895年间,梁启超两随康有为入京会试,并曾“上书”清廷反对《马关条约》。1895年,陈千秋英年早逝,梁启超遂成为康有为最为器重的学生\footnote{张瓊:《中国名人大传:梁启超传》}。

1896年初强学会解散后,梁启超应黄遵宪及吴越维新派汪康年之邀来到上海,于同年8月9日创办宣扬“变法图存”的《时务报》,并在当地结识了谭嗣同。年底,他又被有意维新的清湖广总督张之洞邀至武昌,在当地继续宣传变法思想。1897年10月6日,梁启超发表《论君政民政相嬗之理》,将其师康有为的“三世说”和社会达尔文主义结合,提出世界各国的历史发展必经“君主专制”、“君主立宪”、“君民共立”三阶段。此种论调使思想保守的张之洞和汪康年大为不满。不堪压力的梁启超唯有转赴湖湘,参与当地维新派的活动\footnote{张瓊:《中国名人大传:梁启超传》}。

当时,湖湘是清帝国对内维新运动最为兴盛之处。1897年9月24日,在清湖南巡抚陈宝箴的支持下,黄遵宪与湖湘维新派熊希龄于长沙开办同时教授东西方学问的时务学堂,以梁启超为“中文总教习”。经梁启超安排,该校学生必须重点学习《春秋公羊传》、《孟子》,在其中找出所谓宣扬“民权思想”的主张,并将之与西方国家政治、法律互相参证\footnote{张瓊:《中国名人大传:梁启超传》;李济琛:《戊戌风云录》,页379—382}。1898年9月,戊戌变法失败,陈宝箴被革职,时务学堂被迫停办。在短短一年中,该学堂共招生约两百人,其中多有日后湖湘历史上的关键人物,大滇血统的湖湘民族英雄蔡锷即名列其中。

1898年初,因康有为在北京创办保国会急需人手,梁启超北上与其师会面。此后数个月内,他虽未曾见到光绪帝,却参与了戊戌变法的全过程。当年9月,变法失败,梁启超避入北京的日本驻清使馆,其后流亡日本。在日本,梁启超与革命派领袖孙文(关于孙文,详见下一节)会面,其思想日趋激进。当时,主张“勤王”的康有为与主张反清革命的孙文水火不容。为调和两人矛盾,梁启超提出了一个颠覆清帝国、建立共和制度、以光绪为共和国总统的计划。1900年,愚昧荒诞的庚子拳乱在华北爆发。当年6月21日,清帝国可笑地向英、美、法、德、意、日、俄、西、比、荷、奥十一国宣战,随即引发国际社会出兵。8月14日,英、美、法、俄、日、德、意、奥组成的“八国联军”攻陷北京,慈禧太后与光绪帝逃亡西安。在清帝国风雨飘摇之际,原时务学堂教习、梁启超的挚友唐才常(湖湘浏阳人)决定发动“武装勤王”,意图颠覆满洲亲贵的统治,建立由满人皇帝和“汉人”联合执政的君主立宪制。唐才常的计划自然得到康、梁的支持,意图火中取栗的孙文亦对其有所资助,甚至连年迈的容闳亦由美国赶赴上海参与其事。1900年7月26日,唐才常筹办的“中国议会”在上海公共租界愚园召开,与会者包括汪康年、章太炎、严复等吴、闽革命、维新派人士。议会推举容闳为议长、严复为副议长,决定联合各处反清会党组织“自立军”,兵分七路于汉口、汉阳、江淮、湘、赣等处起事。唐才常原本将起兵时间定于8月9日,然在此关键时刻,康有为迟迟不从海外转送军饷,起兵时间只得推后至该月23日。20日,梁启超亲赴上海租界,得知康有为迟不送饷,极感不满。这时,自立军即将起事的消息已经走漏。21日,张之洞破获自立军设于汉口英租界的总机关,逮捕唐才常等二十名重要首领。两天后,原定起事的日子,唐才常等人被斩首于武昌,自立军遂告失败\footnote{关于唐才常与自立军的详情,学界研究甚多,具体研究情况可参看孙桃丽:《近二十年唐才常及自立军起义研究述评》}。

自立军失败后,愤怒至极的梁启超前往槟榔屿,与正在当地进行宣传活动的康有为对质,反遭康有为责难。康有为称,梁启超与孙文合作完全是背叛师门的行为。尽管如此,他仍会给梁一个悔改的机会。被恩师严厉训斥的梁启超只得屈服。不久后,梁启超前往澳大利亚为康有为的保皇会筹款,然而康有为却怀疑他中饱私囊。愤懑、委屈之下,梁启超于1901年初离澳返日,开始宣扬革命\footnote{张瓊:《中国名人大传:梁启超传》}。他要革命的对象自然是清帝国,而这场革命的参与者则是被他发明出来的“中华民族”。1901年,梁启超在保皇会创办于日本横滨的机关刊物《清议报》上发表《中国史叙论》一文。在文中,他这样说:

\begin{quote}

今且勿论他族 ,即吾汉族,果同出於一祖乎?抑各自发生乎?亦一未能断定之问题也。据寻常百家姓谱,无一不祖黄帝。虽然,江南民族,自周初以至战国,常见有特别之发达,其性质习俗颇与河北民族,异其程度。自是黄河沿岸与扬子江沿岸,其文明各自发达,不相承袭。而瓯闽两粤之间,当秦汉时,亦既已繁盛,有独立之姿。若其皆自河北移来,则其移住之岁月,及其陈迹,既不可考见矣。虽然,种界者本难定者也。於难定之中而强定之,则对於白棕红黑诸种,吾辈划然黄种也,对於苗、图伯特、蒙古、匈奴、满洲诸种,吾辈庞然汉种也。号称四万万同胞,谁曰不宜\footnote{梁启超:《中国史叙论》第五节}。

\end{quote}

在此,梁启超虽然意识到上古的百越、华夏分属不同文明系统,却仍然将“汉种”的标签贴在清帝国境内的编户齐民身上,进而捏造出“四万万同胞”的虚假概念。非但如此,他还要将“同胞”的概念进一步扩大:


\begin{quote}

叙述数千年之陈迹,汗漫邈散,而无一纲领以贯之,此著者读者之所苦也,故时代之区分起焉。

第一,上世史。自黄帝以迄秦之一统,是为中国之中国,即中国民族自发达自争竞自团结之时代也。其最主要者,在战胜土著之蛮族,而有力者及其功臣子弟分据各要地,由酋长而变为封建。复次第兼并,力征无已时,卒乃由夏禹涂山之万国,变为周初孟津之八百诸侯,又变而为春秋初年之五十余国,又变而为战国时代之七雄,卒至於一统。此实汉族自经营其内部之事,当时所交涉者,惟苗种诸族类而已。

第二,中世史。自秦一统后至清代乾隆之末年,是为亚洲之中国,即中国民族与亚洲各民族交涉繁赜竞争最烈之时代也。又中央集权之制度,日就完整,君主专制政体全盛之时代也。其内部之主要者,由豪族之帝政,变为崛起之帝政。其外部之主要者,则匈奴种、西藏种、蒙古种、通古斯种次第错杂,与汉种竞争。而自形质上观之,汉种常失败。自精神上观之,汉种常制胜。及此时代之末年,亚洲各种族,渐向於合一之势,为全体一致之运动。以对於外部大别之种族……

第三,近世史。自乾隆末年以至於今日,是为世界之中国,即中国民族合同全亚洲民族,与西人交涉竞争之时代也。又君主专制政体渐就湮灭,而数千年未经发达之国民立宪政体,将嬗代兴起之时代也。此时代今初萌芽,虽阅时甚短,而其内外之变动,实皆为二千年所未有,故不得不自别为一时代。实则近世史者,不过将来史之楔子而已\footnote{梁启超:《中国史叙论》第八节}。

\end{quote}

在这段对所谓“中国历史”的分期中,梁启超第一次提出了“中国民族”这一概念。在他看来,“中国民族”的范围是随着时代的变化日渐扩大的。在秦以前的“上世”,“中国民族”等同于所谓的“汉族”。在秦至清乾隆之间的“中世”,“匈奴种、西藏种、蒙古种、通古斯种”逐渐与“汉种”合一。在乾隆后的“近世”,“中国民族”已包含上述形成“合一之势”的各族,并将联合亚洲各民族与西方交涉竞争。次年,梁启超完成《论中国学术思想变迁之大势》一书,首次提出“中华民族”的概念:

\begin{quote}
齐,海国也。上古时代,我中华民族之有海思想者厥为齐,故于其间产出两种观念焉:一曰国家观、二曰世界观……立于五洲中之最大洲而为其洲之最大国者,谁乎?我中华也。人口之居全地球三分之一者,谁乎?我中华也。四千余年之历史未尝一中断者,谁乎?我中华也。我中华有四百兆人公用之语言文字,世界莫能及。我中华有三十世纪前传来之古书,世界莫能及\footnote{转引自蒋嘉骏:《论梁启超的“中华民族”概念》}。

\end{quote}

至此,梁启超更进一步,将所谓的“中华民族”发明成一个三千年来历史从未中断过的古老民族,全然不顾其间发生的无数次大洪水与人口替换、冒名顶替。1903年,梁启超在《政治学大家伯伦知理之学说》中又提出了更加狂热的观点:

\begin{quote}
吾中国言民族者,当于小民族主义之外,更提倡大民族主义。小民族主义者何?汉族于对国内他族是也。大民族主义者何?合国内本部属部之诸族以对于国外之诸族是也……合汉、合满、合蒙、合回、合苗、合藏,组成一大民族,提全球三分有一之人类,以高掌远跖于五大陆之上,此有志之士所同心醉也\footnote{转引自蒋嘉骏:《论梁启超的“中华民族”概念》}!
\end{quote}

这时,梁启超已公然喊出将“汉”、满、蒙、回、苗、藏“组成一大民族”、让这一“大民族”称雄于全球的主张,并明确表示这是令“有志之士所同心醉”的大好事。到这时,他已经完全忘记了对他有养育之恩的乡邦南粤,将南粤与虚幻的“汉族”乃至“中华民族”捆绑在一起,完全不认为南粤有自立的必要。1903年,梁启超赴美考察共和政体,其后再次改变立场,转而与康有为一同支持保皇。1906年,梁启超完成了他的重要著作《新民说》,在其中详细论述了所谓的“国家思想”:

\begin{quote}
(国民)一曰对于一身而知有国家,二曰对于朝廷而知有国家,三曰对于外族而知有国家,四曰对于世界而知有国家\footnote{转引自王彦君:《梁启超国民态度的演变》}。
\end{quote}

在这种理论下,所谓的“中国”成为所有“中华民族”个体成员必须效忠的“祖国”,东亚大陆各邦绝不会有任何自立和获得自由的机会。此后,梁启超规划出的阴森可怖的图景迅速发酵,“中华民族”这一概念成为20世纪各路大一统分子剥夺东亚大陆各邦自由的理论基础,流毒至今,仍在欺骗着东亚大陆的居民们。梁启超不但和康有为一样无耻地背叛了南粤,还成了谋杀南粤及东亚大陆各邦自由的元凶之一。南粤人和一切热爱自由的人们必须戳穿他编织的画皮,方能打破黑洞般的大一统诅咒,迎来真正的自由、得到真正的尊严。


\section{对南粤未来的第二种规划:“驱逐鞑虏,恢复中华”}

\indent 在康、梁二人致力于推动清帝国改革、发明“中华民族”之际,南粤史乃至世界史上大名鼎鼎的孙文为南粤的未来提出了另一种规划。孙文,号逸仙,香山翠亨村人,于1866年11月12日出生在一个贫农家庭,自幼便熟习农事。孙文五岁时,他的长兄孙眉因生活贫困出海远航,前往夏威夷檀香山打工,在当地开垦了自己的农场。然而,孙文一家的生活仍十分困苦。六七岁时,孙文时常听村中一位名叫冯观夹的太平天国老兵讲反清故事,因而对洪秀全十分仰慕\footnote{《孙中山年谱长编》}。大概从那时起,他的反社会人格便已有萌芽了。

1878年5月,十二岁的孙文受到孙眉接济,随母乘轮船远赴檀香山。当时,夏威夷群岛虽尚未被美国吞并,但英、美、法等西方国家的势力已大举进入夏威夷王国,使该国呈现一派西化面貌。1879年秋,孙文入读英国圣公会在檀香山创办的意奥兰尼学校,学习英语、英国历史、数学、化学、物理、神学等课程。1882年7月,他以优异的成绩毕业,收到夏威夷国王卡拉卡瓦一世的接见。次年春,他升入夏威夷的最高学府、由美国公理会主办的奥阿厚书院,开始了中学生活。这时,受过教会学校多年教育的他已表现出强烈的入教意愿。对孙文的宗教倾向,思想保守的孙眉相当不满,便断绝对他的接济,将他送回家乡翠亨村\footnote{《孙中山年谱长编》}。

因被兄长强行送回家乡,孙文愤愤不平,便将仇恨发泄到家乡的传统秩序上,对幼年穷困生活的记忆加深了他的这种仇恨。回乡不久,他就伙同一起长大的好友陆皓东砸毁了村中北帝庙的神像。这种蔑视乡村共同体信仰的行为当然会引起众怒。在村中乡绅的干预下,孙、陆二人无以立足,只得避走香港,并在香港经美国公理会传教士喜嘉理(Charles Hager)主持受洗成为基督徒。不过,孙文非但没有基督徒应有的虔敬与诚实,其反社会人格反而更加强烈。由于自幼便与南粤的乡村共同体格格不入,游离于社会边缘的他不但缺乏对乡邦的感情,更对其社会秩序有一种刻骨的仇恨。因此,对共同体生活十分饥渴的他便将情感投向了所谓的“中国”。在香港,目睹了粤人罢工示威情形的他开始相信“中国人已经有相当的觉悟”和“种族的团结力”。1885年清帝国在法清战争中落败后,他更“始决颠覆清廷,创立民国之志”\footnote{《孙中山年谱长编》}。至此,他已完全倒向大一统立场,南粤于他而言不过是“中国”的一部分。他对清帝国的态度虽然比康、梁更为激进,但在坚持大一统这一点上,双方实乃一丘之貉。

1886年,孙文考入广州博济医院附设的南华医学堂(今广州孙逸仙纪念医院)习医。次年,他又转入香港西医院学习,重返香港。1892年,孙文毕业,二十六岁的他走到了人生的十字路口:究竟是成为一名普通的南粤医生,服务广大南粤民众,还是投身政治、实现他的宏大理想?做为具有强烈反社会人格的社会边缘人,孙文的选择自然是后者。不过,私智发达的他却策略性地不准备立即与清帝国翻脸。孙文毕业这年,时任清北洋大臣的李鸿章正任香港西医院的名誉赞助人。孙文希望通过李鸿章挤入清帝国的统治机器内,以投机的方式迈出自己从政的第一步。1893年12月,回到翠亨村的孙文完成《上李鸿章书》,提出“人能尽其才,地能尽其用,物能尽其用,货能畅其流”这四项清帝国的“富强之本”。1894年6月,他与陆皓东到达天津意图“上书”,但公务繁忙的李鸿章根本没有接见他们。遭到忽视的孙文因此对清帝国极度仇恨,决计展开“徐图所以倾覆而变更之”的活动。至于他在颠覆清帝国后要创立的国家,自然是一个大一统的“中国”\footnote{《孙中山年谱长编》}。

1894年10月,孙文自上海出发前往檀香山。11月,他在檀香山成立武装反清组织“兴中会”,首批入会者20余人,筹得军费数万元。次年1月,他返回香港与当地反清组织辅仁文社进行接洽。辅仁文社创于1890年,其主张为推翻清帝国、建立“中国”的“合众政府”。为展开反清活动,该社社长杨衢云(东莞人,闽越移民二代)还加入了香港的洪门(天地会)组织。杨衢云是孙文在香港时的旧识。孙文返港后,政治立场相近的两人一见如故,杨衢云表示可以将辅仁文社并入兴中会。1895年2月21日,香港兴中会总部在中环士丹顿街13号成立。该会规定,凡入会之人需举右手对天说出如下誓言:

\begin{quote}

驱逐鞑虏,恢复中华,创立合众政府。倘有二心,神明鉴察\footnote{《孙中山年谱长编》}。

\end{quote}

这一誓言表明,兴中会立志在驱逐满洲人后“恢复”由“汉人”主导的“中华”国家,建立共和政体。兴中会虽未如梁启超那样将“汉”、藏、满、蒙、回发明成统一的“中华民族”,且提出了驱逐满洲侵略者这一正义主张,却仍沉迷于“汉人”、“中国”这些虚幻的概念,对南粤的自立漠不关心。兴中会香港总部成立后,孙文与杨衢云立刻做出进攻广州的决定,定于1895年10月26日,即农历重阳节攻占广州,由孙文主持军事、杨衢云留港筹款、郑士良(惠阳人,天地会首领)联络会党。根据计划,300名香港会党成员组成的突击队将携带装载武器的木桶,在起事前夜乘轮船潜入广州。到起事当天,突击队将首先攻取广州各主要衙门,事先混入城内的千余名受雇而来的北江、顺德、香山会党则应闻声响应,在炸弹队的配合下占领广州。起事军的“青天白日旗”由陆皓东设计,该旗便是日后的国民党党旗。10月26日黎明,除香港突击队外的各路人马均已准备完毕,孙文亦已身在广州。这时,孙文突然收到杨衢云从香港发来电报,内称突击队尚未准备完毕,必须延期两日起兵。无奈之下,孙文只得暂令各路人马返回原地。次日,起事消息走漏,接到密报的清两广总督谭仲麟派兵保卫王家祠、咸虾栏等处,将留守城内的陆皓东等五人捕获。28日,运送武器的香港突击队终于抵达广州码头,立即被清军截获。陆皓东等人受尽酷刑,不屈而死。身为军事总指挥的孙文则溜之大吉,躲入博济医院,后经香港逃亡日本\footnote{《孙中山年谱长编》}。

1895年11月,孙文到达日本,随后剪掉发辫,换上西装,开始打造自己“革命党”领袖的形象。他先在横滨设立了兴中会分会,后又于1896年1月赴檀香山游说当地粤侨捐款。海外粤侨早已对清帝国深恶痛绝,许多人因而被孙文的说辞欺骗,向他捐献了大批钱财。6月,孙文又到达美国西海岸的旧金山,继续展开游说活动。可是,因洪门在当地粤侨民众中影响很大,而孙文又不是洪门成员,信任他的人非常少。9月,深感失望的孙文离美赴英,希望考察欧洲政治以完善自己的政治理论。这时,清帝国驻英使馆已经盯上他,使馆的密探拿到了他的照片。10月11日,孙文在路过清使馆外街道时被突然抓获,失去了人身自由。就在清方准备将他遣送回清处决时,英国外交部采取了行动。英方认为,清帝国在伦敦随便捉人是决不能容忍的。10月22日,英国首相索尔兹伯里(Robert Salisbury)勒令清使馆立刻放人,否则便将清方外交人员全部驱逐。清使馆不堪外交压力,只好于次日将孙文释放\footnote{《孙中山年谱长编》}。

对孙文来说,这次有惊无险的遭遇是个提高其身价的好机会。事后,孙文就此事撰写了《伦敦蒙难记》一书,大肆炒作自己遭受的迫害,使其名声大燥,开始受到世界各国的注意。获释后,孙文在伦敦居住了大半年。在此期间,他每日去大英博物馆看书,其“三民主义”思想开始定型。所谓的三民主义,指“民族”、“民权”、“民生”三个主义,即同时进行民族、政治和社会革命。颠覆满洲政府、建立共和制度乃民族、政治革命的目标,民生主义的核心则在于“平均地权”,即由政府征收土地价格之增值,以达到均贫富的目的\footnote{《孙中山年谱长编》}。这一社会革命理论,无疑是与孙文的反社会人格和仇富心态息息相关的。此种社会蓝图虽不如康有为在《大同书》中描绘的图景可怕,但其左翼乌托邦色彩仍是十分浓烈的。

1897年7月,孙文经加拿大回到日本横滨。此时,他的“革命党”身份已闻名于日本政界。不久后,受日本外务省之命调查反清会党情况的浪人宫崎寅藏求见孙文,两人相谈甚欢。宫崎是个真诚的泛亚主义者,他以联合亚洲国家一同对抗欧美强权为己任,对反清革命者十分同情。他曾阅读过孙文的《伦敦蒙难记》,相当崇拜孙文。同年9月27日,孙文通过宫崎的介绍在东京会晤了日本民党领袖犬养毅。犬养毅乃主张联合亚洲各国对抗日本列强的泛亚主义大佬,他将孙文迎至自己寓所中会面,并为孙文在东京安排了住所。在日方建议下,孙文在进行入住登记时采用了“中山樵”这一日文化名。此后,人们便习惯性地称他为“孙中山”\footnote{钟声:《孙中山:从医生到民主革命家》,页120—121}。

由于攀上了犬养毅这位大佬,孙文在日本的居留权得以解决。在犬养的帮助下,孙文先后拜会日本外相大隈重信、商务大臣大石正己等三十余名日本政要,成为日本泛亚主义者的盟友。此后数年间,孙文一直在犬养毅的卵翼下居住于日本。1900年,孙文的机会再次出现。是年5、6月间,庚子拳乱在华北爆发。英国担心义和团势力侵入长江流域,遂策动清两江总督刘坤一、湖广总督张之洞与西方列强合作。是年6月26日,张之洞、刘坤一以上海道台余联沅为代表,与各国驻上海领事签订《东南互保章程》九条,议定上海租界归各国共同保护,长江中下游地区内各国商民、教士产业。随后,清两广总督李鸿章、铁路大臣盛宣怀、山东巡抚袁世凯、闽浙总督许应骙皆加入“东南互保”,陕西巡抚端方、四川总督奎俊亦对“东南互保”表示支持。李鸿章更对清帝国向西方十一国宣战的“诏书”发出了极其强硬的回应:

\begin{quote}

此乱命也,粤不奉诏\footnote{转引自靳会永:《晚清第一外交官李鸿章传》,页191}!

\end{quote}

太平天国战争后,湘淮军出身的封疆大吏把持着清帝国各省的军政大权。在此历史节点,他们断然与西方列强合作,实行东南互保,使南粤、闽越、吴越、湘楚、江右、江淮、巴蜀、齐鲁、关中躲过了洪水的侵袭,这着实是他们对东亚各邦的历史贡献。除使南粤加入东南互保外,李鸿章还做出了更勇敢的举动,那便是策划“两广独立”。早在5、6月之交,当清廷尚未对各国宣战时,香港立法局的粤人议员何启(祖籍南海)在请示港英政府后找到了留守香港的兴中会党人陈少白,建议兴中会与李鸿章合作,建立两广独立政府。陈少白表示赞同,随即与在日本的孙文联络,香港方面亦开始与李鸿章接触。当时,李鸿章有位名叫刘学询的机要幕僚乃孙文同乡,对此十分热衷。他向李鸿章提议与孙文合作,李鸿章表示同意。随后,刘学询向孙文发出如下电文:


\begin{quote}

因北方拳乱,欲以粤省独立,思得足下为助,请速来粤协同进行\footnote{转引自董丛林:《李鸿章对“两广独立”的态度与庚子政局》}。

\end{quote}

孙文对南粤独立的态度完全是机会主义的。他认为,兴中会的首要任务是颠覆清帝国、“恢复中华”,南粤是否独立并不是很重要的事。不过,如果南粤独立对反清和“恢复中华”有好处,他亦不会反对。他这样说:

\begin{quote}
此举设有成,亦大局之福,故亦不妨一试\footnote{转引自董丛林:《李鸿章对“两广独立”的态度与庚子政局》}。
\end{quote}

6月11日,孙文偕杨衢云、郑士良、宫崎寅藏等人自横滨启程,于17日抵香港海面,遇到了李鸿章派来迎接他的军舰。由于担心此乃清帝国诱捕他的陷阱,孙文不敢贸然上船,便派宫崎等三名享有治外法权的日本人代其赴会。当夜10时许,宫崎等人抵达刘学询在广州的公馆,双方开始秘密谈判。在谈判中,刘学询表示在列强攻陷北京前,李鸿章“不便有所表示”,暗示南粤独立之事需待北京易手后进行。宫崎则提出,李鸿章应保障孙文安全,并向兴中会借款六万元作为双方合作基础。刘学询当即请示李鸿章后表示同意,并当面付款三万元。次日凌晨3时,长达五小时的秘密谈判结束,宫崎等人随即趁夜返回香港。可以说,李鸿章在此次谈判中表达了足够诚意,双方进一步合作的空间很大\footnote{转引自董丛林:《李鸿章对“两广独立”的态度与庚子政局》}。

然而,6月18日天亮后,形势急转直下。就在前一天,八国联军攻陷大沽口。清廷惊慌失措,忙电令李鸿章立刻“遵旨”北上。究竟是遵从清廷,还是留在广州成为南粤国父?在犹豫之间,李鸿章选择加入东南互保进一步观望局势。7月16日,清廷任命李鸿章为直隶总督兼北洋大臣。经过权衡,李鸿章决定从命。次日,他乘船离开广州,在当天抵达香港,随后拜会港督卜力。卜力诚恳地对李鸿章说:

\begin{quote}
刻下是两广脱离清廷独立之良好机会\footnote{转引自董丛林:《李鸿章对“两广独立”的态度与庚子政局》}。
\end{quote}

卜力还说,若南粤独立,则李鸿章当为主权者,而孙文仅任顾问。对卜力的忠告,李鸿章没有做出任何回应。7月18日,李鸿章离港北上,并于次年代表清帝国与西方列强签订《辛丑条约》后去世。这个为清帝国的存续奔波了一辈子的老人最终没有跨出迈过悬崖的一步、没有做出南粤独立的决断,南粤稍纵即逝的机会窗口也随着他的北去关闭了。假如李鸿章果真在英国人的支持下宣布南粤独立,那么他不但将成为赵佗般的英雄,更将成为独立南粤的首任总统。至于孙文则很可能在英国人的压力下放弃他的日本朋友和革命主张,成为李鸿章身边的得力顾问。历史没有如果,李鸿章在一念之间的决断,断送了南粤本应在百余年前获得的自由\footnote{转引自董丛林:《李鸿章对“两广独立”的态度与庚子政局》}。

李鸿章的懦弱和对清廷的愚忠固然是粤独计划流产的主要原因,孙文对粤独的三心二意亦加速了此计划的失败。在宫崎等人结束谈判秘密返港时,深恐被李鸿章设计逮捕的孙文已乘船南下往越南西贡而去。到达西贡后,孙文一面致电刘学询了解谈判情况,一面又命人准备在南粤再次武装起事,并要求留在香港的兴中会党人“分头办事”,即同时策动粤独和起事。6月27日,孙文在西贡召开军事会议,决定由郑士良在惠州发动会党起兵。不久后,郑士良潜回南粤,联络三合会众、日本浪人百余人,在归善县三洲田立下大营。10月6日,郑士良率部起事,于当日进攻沙湾获胜。15日,革命军在佛子坳大破清军,俘清归善县丞杜凤梧以下数十人,缴获700余条枪枝。17日,革命军又于永湖大破清军,清提督邓万林中枪坠马而逃。20日,革命军在崩冈圩获胜。至此,革命军已四战四胜,吸引了大批依附而来的会党成员。22日,革命军进入三多祝,人数已超过两万。郑士良在此整编部队、调集粮饷,声势颇为浩大,准备东进厦门,从海上获得日占台湾方面的接济,粤、闽的清帝国官吏陷入一片恐慌\footnote{《简明广东史》,页496—497}。

不过,革命军的好运到此为止了。当时,为给革命军筹款,孙文正在台湾活动。台湾总督儿玉源太郎本承诺帮助孙文,但日本首相伊藤博文突然改变政策,要求台湾方面停止对孙文的接济。孙文只得将实情电告郑士良,并派日本浪人山田良政经香港潜入郑士良军中,令郑自决进止。失望的郑士良只得将大部分队伍遣散,留下千余精锐回军三洲田,并排入从水路至香港购置弹药、军械,图谋袭取广州。然而在清军的围堵下,革命军弹尽援绝。11月7日,郑士良不得不解散残军,率少数骨干逃至香港。山田良政在逃亡途中迷路,遭清方杀害\footnote{《简明广东史》,页497}。

在三洲田起事同时,潜入广州的兴中会党人史坚如(番禺人)正在城内准备响应。然因缺乏饷械,史坚如不得不采取暗杀手段。10月28日,史坚如以两百磅炸药爆破清两广总督署,署两广总督、广东巡抚德寿正在睡觉,被气浪从床上抛至地下,侥幸未死。次日,史坚如被捕,于11月19日遇害\footnote{《简明广东史》,页497}。山田良政和史坚如是优秀的日本武士和南粤武士,他们的武德和勇敢足以令人敬佩不已。然而,他们的牺牲都被孙文利用,成了koy 实现自己妄想和野心的工具,ni 着实令人叹息不已。
三洲田起事失败后,孙文及兴中会的名声日渐响亮。他们“驱逐鞑虏,恢复中华”的主张欺骗了东亚诸邦一大批反清人士。据孙文称:

\begin{quote}
当初次起义(1895年广州起事)失败也,举国谬论莫不目予辈为乱臣贼子、大逆不道,诅咒谩骂声不绝于耳,吾人足迹所到,凡人识者,几视为毒蛇猛兽,而莫敢予吾人交游也。惟庚子失败(1900年三洲田起事)以后,则鲜闻一般人之恶语相加,而有学之士,且多为吾人扼腕叹息,恨其事之不成也。前后现较,差若天渊\footnote{孙文:《建国方略》}。
\end{quote}

在此前后,孙文一方面利用山田良政和史坚如的牺牲炒作自己的名声,为兴中会筹款,并于1904年初在檀香山加入洪门致公堂;另一方面,他不顾流亡槟榔屿、含辛茹苦地抚养着三个子女的发妻卢慕贞,在日本先后纳十五岁的女佣浅田春为妾、十五岁的中学生大月熏为妻。此外,孙文的活动也引起了日本泛亚主义大佬头山满的注意。1901年,头山满、内田良平等人在东京成立泛亚主义组织黑龙会,内田自任“主干”,聘头山为顾问。黑龙会主张将俄国的势力驱逐出黑龙江流域,并反对日本国内的亲英派,视之为与“西方帝国主义”勾结者。1905年8月20日,以孙文为首、粤人为主的兴中会,以黄兴、宋教仁为首、湘人为主的华兴会,以蔡元培、章炳麟为首、吴越人为主的爱国学会及以河北人张继为首的青年会等反清组织在内田良平的牵线下合并,于东京成立同盟会,推举孙文为同盟会总理。同盟会成员虽来自东亚大陆各邦,但他们大都是支持驱逐满洲人、恢复“汉人”之“中华”的大一统主义者。在同盟会机关刊物《民报》的发刊词中,孙文的“三民主义”得到表述,并成为同盟会的政纲。ni 一政纲的核心在于以下十六个字:

\begin{quote}
驱逐鞑虏,恢复中华,建立民国,平均地权
\end{quote}

同盟会不仅要建立一个大一统的“中国”,甚至还要展开左翼色彩浓厚的“平均地权”运动,推行社会革命。从本质上来说,它和康有为的“大同”理论及梁启超发明的“中华民族”并无不同,皆为禁锢南粤的大一统乌托邦地狱。在同盟会建立前后,以孙文为首的“革命派”与以康、梁为首的“保皇派”通过报刊进行了一系列激烈的论战。此种论战并不能代表当时南粤社会的真实面貌,而仅是两拨粤奸间狗咬狗式的自相残杀。在南粤漫长的历史上,我们的伟大祖先中涌现出过无数为南粤的自由与尊严而战的豪杰。康、梁、孙三人决不能代表南粤历史的主流,他们只是背叛乡邦的叛徒。

同盟会成立后,孙文遣冯自由(日本粤侨,祖籍南海)赴香港发展会员,与陈少白一同将兴中会香港总部扩建为香港同盟分会。其后,香港同盟分会又派许雪秋(祖籍海阳,新加坡粤侨)、邓子瑜(博罗人)、朱执信(番禺人,吴越移民后裔)分赴潮汕、惠州、广州、桂、闽等地发展会员。1907年5月22日,在许雪秋的策划下,同盟会于潮州黄冈发动乡民、会党千余起事。革命军一度攻占黄冈,成立军政府,但被清广东水师提督李准所率援军击败,被迫于28日解散。6月2日,在邓子瑜的策划下,同盟会又发动会党起事于惠州城外之七女湖。革命军与清军周旋十余日,弹尽援绝,亦被迫解散。同年,粤西钦州民众发动由刘思裕领导的大规模反清起义,反抗清帝国的苛捐杂税。孙文意图利用钦州民众的英勇斗争实现其野心,遂命黄兴(湖湘长沙人)赴钦州联络民众、策反清帝国新军(关于清帝国编练新军一事,详见下节)。当时,清两广总督周馥派出新军统制郭人章、标统赵声率军三四千人前往钦州镇压,郭人章系黄兴旧友、赵声则为同盟会地下会员,孙文遂视之为绝好机会。然而,郭人章不顾与黄兴的情谊,仍向起义军进攻,致使刘思裕壮烈战死。孙文只得再派王和顺(广西邕宁人)潜入钦州,仓促发动千余新军于9月3日起事。王和顺自称“中华国民军南军都督”,于5日率部攻陷粤越边境的防城县城,杀清县令宋鼎元等官吏20人。此后,革命军先后进攻钦州府城、灵山县城不克,只得退入十万大山“以图后备”,王和顺则逃亡越南河内\footnote{《简明广东史》,页541}。

这时,孙文仍不死心,希望从越南发动一次新攻势。1908年3月,他命胡汉民(番禺人,江右移民后代)、黄兴留守同盟会河内机关部,以黄兴为总司令,组织一次跨境攻势。黄兴集结越南粤侨青年200余人,称“中华国民军南路军”,于是年3月29日越境攻入钦州东兴。29日、31日及次月2日,革命军三战三胜,占据马笃山,规模扩大到600余人,清军降者40余人。其后,革命军向广西上思推进,大破郭人漳所部3000余名新军,缴获大批清军军旗、战马。然而,因兵力不足,革命军只能在钦州、廉州、上思之间不断游击,接连袭击数十城镇,终因弹尽援绝,于四十余天后解散。黄兴率少数革命军干部逃至河内,余部则退入十万大山\footnote{《简明广东史》,页542}。

连战连败之后,同盟会加紧在广东新军中发展会员,希望以此打开突破口。与此同时,南粤本土的土豪们也积极行动起来,开始致力于争取南粤的真正自立。至于步入末路清帝国,则仍在试图维持其在南粤摇摇欲坠的统治。在清帝国统治南粤的最后几年里,ni 三股力量交织在一起展开激烈的斗争,最终博弈出了一个耐人寻味的结果。

\section{对南粤未来的第三种规划:“广东人之广东”}


\indent 当康、梁与孙文提出不同的大一统方案,并为之争斗不已时,南粤人并未放弃对独立的追求。自李鸿章的粤独计划在1900年流产起,南粤人意识到只有靠自己才能获得真正的自由。1902年,一篇名为《广东人之广东》的文章横空出世,吹响了南粤独立运动的号角。此文作者,便s 著名的南粤英雄欧榘甲。

欧榘甲,淡水客家人,1870年生,曾为康有为在万木草堂的学生,后积极追随康、梁投身清帝国“维新”事业。如前所述,1897年9月,梁启超开始在湖湘维新派开设的长沙时务学堂担任“中文总教习”。次年初,欧榘甲赴长沙协助梁启超,任时务学堂分教习。梁启超任职时务学堂期间曾积极主张湖南广开学堂以“自保”,此种与大一统背道而驰的观念深刻影响了欧榘甲。s 年9月,维新变法失败,欧榘甲随梁启超东渡日本,与孙文有所接触,吸纳了革命推翻清廷的理论,其思想开始偏离康有为。1902年,美洲洪门致公堂在倾向“保皇派”的粤侨的支持下于旧金山创办《大同日报》。该报聘欧榘甲为总编辑,试图宣传“迫朝廷改专制而为立宪政体”的主张\footnote{《中国历史大辞典》,页131}。不过,欧榘甲却另有打算。此时的欧榘甲在名义上虽仍属康有为的“保皇”阵营,其思想却已倒向革命一边。不过,欧榘甲欲通过革命创建的国家却与孙文截然不同:孙文试图通过革命创造一个“汉人”的“中国”,欧榘甲则欲创造独立的“广东人之广东”。同年,欧榘甲在《大同日报》上以笔名“太平洋客”刊登《新广东》一文,此文又名《广东人之广东》。此文开头,欧榘甲开宗明义地表明自己的立场:

\begin{quote}

中国之名,于其身泛而不切,尊而不亲,大而无所属,远而无所见。欲志士舍头、富商舍财、勇士舍命,以图其自立,非仁人杰士,有高瞻远瞩之心,长驾远驭之志者,断乎未有能动者也。夫治公事者不如治私事之勇,救他人 者不如救其家人亲戚之急,爱中国者不如爱其所生省份之亲。人情所趋,未如何也。故窥现今之大势,莫如各省先行自图自立……吾广东人,请言自立自广东始。姑名是议曰“新广东”,以谂我广东人欲享新国之福份者。
\end{quote}

欧榘甲此论虽然只言广东做为清帝国一省而“自立”,未明言南粤独立,但他无疑已暗示了变南粤为独立国家的规划。若他的规划果能付诸实施,那么最低限度亦能将所谓的“中国”便为松散的邦联,而南粤则为此邦联之一国。接下来,欧榘甲又自豪地提出四点广东的“自立特质”:

\begin{quote}
	
一曰人才之出众。广东通商最早,风气最开,其能通外事知内情者,所在而有……若夫在海外者,除福建人外,则皆广东人也。闻有能谈时事、开报馆,遣子弟入外国学堂者,惟广东人为多。而近年又有以一大会以团海外数百万人为一体,讲爱国爱种之策,俨成一外中国新中国焉。于是中国全部之事,几于有广东人则兴,无广东人则废。外人之论中国者,辄谓命脉在于广东,诚非虚语也。兼之英勇通达之士,游学各国,而通其政治、理财、武备、制造之精者,指不胜屈。此人才之超于各省者也。

一曰财力之雄厚。广东以财雄闻于天下,中外所公认也。咸、同以来,政府若有兵事、赈荒、国债、赔款,需大款大饷等项,莫不向广东而搜括,其数常数倍于各省,岁出达数千万万以上,此广东之财耗于政府者也。而贪官污吏,尤以广东为窟穴,其各省无赖之子,人类所不齿者,辄相借贷捐官,以取倍称之息,分省得广东,则亲戚友朋置酒而相贺,到任才数月,莫不满载而归。嗟我广东人,其饱虎狼之吞噬者,岁不知几何矣!此广东之财耗于官吏者也。至于洋货之进口,以广东为大宗,此广东之财耗于外洋者也。然而此稽一县之财,往往比他荒瘠之一省有余,即比之欧洲小国亦未见其不足。固由出外洋善经商之故,而其饮食起居器用,奢丽之程度,各省常为惊羡所未见。盖粤人一月之费,足彼一岁之费者常多,则财力之厚可知。此财力之超于各省者也。

一曰地方之握要……一则北部诸省,或共黄河之流域;中部诸省,或共扬子江之流域,而广东则特受珠江之流域。二则各省或为水陆交通之势,或为背山面海之形,其间风气皆可以相通,而广东则背横长岭万余里焉,与中原风气邈绝,其言语风俗,往往不同。三则外国文明输入中国者,以广东为始,东西两洋轮舶之所必经,海外万岛皆其种族之所流寓,即谓之广东殖民地,亦非过也。四则广东港口纷歧,与海外交通之便利,万物皆可运入,无能留阻,不独南部诸省所独,即北中部诸省亦所不能。此地势之别于各省者也。

一曰户口之繁殖。广东人口滋生之易,世界殆无其比……若夫高原诸省,一旱则饿莩相望;河流江流诸省,大水则淹没无数,而且地方寒瘠,谋食甚难,一经灾殄,有百年而元气未复者。乃广东孳孕力之雄厚,不独充塞其本部,其澎涨海外者,且至数百万焉,与欧美诸大国并矣。彼丁抹、瑞典、葡萄牙、西班牙诸邦,其人口有不及广东之一县一府者。此人口之逾于各省者也。

\end{quote}
欧榘甲指出,南粤与西方交通最早,因而西化人才最多,并有团结的海外粤侨群体。南粤富甲天下,因而成为岭北帝国官吏疯狂压榨钱财之地。南粤地理形势独特,面朝南海、背靠南岭,独立于黄河、长江流域。南粤因资财雄厚、环境优渥,故人口极多、殖民海外者多达数百万,许多西方国家的人口甚至不能当南粤之一府、一县。有此四点,广东自立实乃顺理成章之事。

其后,欧榘甲又为广东自立提出三条策略。首先,是重视渗入侵粤清军中的会党力量,利用其推动自立:


\begin{quote}

中国之兵,大都无业之游民,非私会中人,鲜有作游民者。既为游民,求其可以不织而耕、不耕而食,莫如为兵。故中国兵者,私会之人居其大半……不独哥老会蔓延长江诸省营中已也,广东之兵,而私会亦居其半……法兰西革命之初,其人亦非上等豪杰,不过起于民间之私会耳。日本浮浪子,岂劲忠臣义士哉!只数维新领袖,能运动之以为正用,故一变而为侠士烈夫,然则此秘密社会者,亦言自立之一大关键也。则图自立于广东,又岂可忽乎哉!

\end{quote}

此种重视会党的思想,应是欧榘甲自孙文处得到的。他主张利用会党为广东谋自立,与孙文谋建“汉人”之“中国”有本质区别。至于他提出效仿法国大革命及日本明治维新展开自立运动,虽有激进、背离保守主义之处,但揆诸其时激进思潮横行全球的情形,为南粤独立而呼喊的欧榘甲无足深怪。

其次,欧榘甲提出,南粤的本地(广府)、客家、福佬(潮汕)三族实为同胞:

\begin{quote}

此三者种族,同出一源,不过因声音而异……三者种族,智识心思,脑输角度,形体精神,不相上下。即以其族谱而言,其祖先莫不由中原丧乱,越岭南迁。故本地之族,多由南雄而至广肇;客家之族,多由赣州而至;嘉惠福佬之族,多由江浙而至福潮。其声音之异,亦由所居之地而变迁焉……然则三者同为种族,无可疑也。

\end{quote}

欧榘甲吸收了16世纪以来粤人精英构建南粤“小华夏”的成果,虔诚地相信南粤三族皆为华夏嫡传,因而实为同族。在此基础上,“血脉相连”的南粤三族共建“广东人之广东”自然不成问题。事实上,欧榘甲此论绝非强行将互不相干的三族捏合一处,建构一个“中华民族”式的虚假“国族”。经过16—19世纪间九军之乱、土客战争、发明民族等历史进程,南粤三族已经互相承认了对方的存在,和而不同地居住在南粤大地上。五年后因《广东地理乡土教科书》而发生的轩然大波,更使南粤三族完全融为一个坚实的共同体。欧榘甲的叙述,是完全符合当时南粤的实情的。
其三,欧榘甲特别点出,追求广东自立的义士绝不能重走太平天国的老路:

\begin{quote}

洪杨行动,众叛亲离,手足干戈,旦夕待灭,既无爱我汉人之心,残虐过于满人……故以洪秀全之蹂躏名城,几有中国全部,而所以为敌而摧灭之者,乃反出于汉人。

\end{quote}

由此段可见,欧榘甲十分可惜地仍未摆脱对“汉人”这一虚幻概念的迷信,没有堂堂正正地喊出“南粤民族”四字,这当与孙文对他的影响有关。然而,欧榘甲提出不应重蹈太平天国覆辙、残虐东亚大陆诸邦,则十分值得称道。此种主张下诞生的广东自立军,无疑将是一支以大一统帝国为死敌、对受尽压榨的东亚大陆各邦和平而有礼的军队。

在文章的最后,欧榘甲更是意气风发地指点河山,给“汉人”之地做出了详细的解体方案:

\begin{quote}
广东不得湖南,亦无进取之途,必终身困守岭表,就越汉之故居,无复与江淮豪杰登龙争虎斗之大舞台。虽然广东人有心,而湖南人未必允也。既未必允,则就两粤之土地人民以成自立。广西之人本粤东流寓而生,疆土既相毘连,情义复相亲挚,于联邦最为易事,此地理人种,关于天然界者也。不独两广为然,既就今中国本部总督所辖之地,而分立为国土,亦人种地理天然界之相合者。如直隶总督所辖直隶、山东、陕西、河南四省为一独立国,两江总督所辖江苏、安徽、江西三省为一独立国,两湖总督所辖湖北、湖南为一独立国,云贵总督所辖云南、贵州,陕甘总督所辖陕西、甘肃,闽浙总督所辖浙江、福建(昔兼台湾为三省,今割于日本),皆两省为一独立国,四川总督所辖四川一省为一独立国。置之欧罗巴强国中(除俄不计,除属地不计),疆域之广,未有能及之者也。即不然,而活因河流江流海流,分为北中南三大部分:阴山以南,黄河以北,诸省合为一独立国;黄河以南、扬子江以北,诸省合为一独立国;扬子江南岸、南洋北岸,诸省合为一独立国。三大干并立,固近世非常之雄国也。然而不论南部、中部、北部,亦不论诸省相合为一与否,而苟有独立之一省,起于其间,则南省必归于南部,中省必归于中部,北省必归于北部,可无疑也。即或因声音风俗政体之异,北省不归北部而归中部,中省不归中部而归南部,南省不归南部而归中部,而苟能自立,虽任其意之所向可也\footnote{本节以上各段引文,皆出自欧榘甲:《新广东》}。

\end{quote}

欧榘甲在此提出了两种解体方案。其一为根据清帝国八总督的辖区将“汉人”之地分解为八国,其中广东、广西合为一联邦。其二为以黄河、长江为界,将“汉人”之地分解为南部、中部、北部三国。根据第一种方案,则南粤无疑会自为一国;根据第二种方案,那么长江以南、南洋以北诸邦便会合为一大国,南粤则将从属于这一大国内。不过,欧榘甲在第二种方案中提出了较为灵活的变通手段,提出若有一“省”因“声音风俗政体之异”脱离三大国之一“而苟能自立”,亦无不可。因此,欧榘甲的解体方案并非单纯地以清帝国划出的省份为界,亦不是倡导南方各邦合为一国的小型大一统分子。若仅单纯以省为界,那么两广虽大致与面朝大海、背靠南岭的南粤传统地理范围重合,如江苏这种将中原人、江淮人、吴越人强行捏合在一起的省份则十分缺乏自为一国的理由。欧榘甲的方案,事实上为帝国各省的进一步解体留下了空间。令人注意的是,欧榘甲又特别提出南粤不可违背湘人意愿北上侵略湖湘。这不但能使南粤避免重蹈南汉国的覆辙,更表明了南粤对饱受大一统帝国压榨的东亚各邦和平守礼的姿态。

欧榘甲摧破大一统诅咒的文章掀起了巨大的震动。1903年,在《新广东》问世后仅一年,留学日本的湖湘反清志士杨毓麟(长沙人)以笔名“湖南人之湖南”发表《新湖南》一文。杨毓麟的立论深受欧榘甲影响,亦主张“汉人”之地解体。不过,杨毓麟更进一步,将清帝国的解体比喻为罗马帝国崩溃、将湖湘独立比作欧洲民族国家建构:

\begin{quote}

民族建国主义何由起?起于罗马之末。凡种族不同、言语不同、习惯不同、宗教不同之民,皆有特别之性质。有特别之性质,则必有特别之思想。而人类者,自营之动物也,以特别之性质与特别之思想,各试营其自营之手段,则一种人得有特别之权力者,必对于他一种人生自存之竞争。故异类之民,集于一政府之下者,实人类之危辀仄轨也。罗马政府集异族于一范围,此古世帝国主义之橐约也。政府之势力,不能无类败,而此异思想异性质之民,各自求其托命,异者不得不相离,同者不得不相即。异者相离,同者相离即,集合之力愈庞大而坚实,则与异种相冲突相抵抗之力亦愈牢固而强韧。非此,则异类之民族将利用吾乖散暌离之势,以快其攫博援噬之心,此民族主义所以寝昌侵炽也。日耳曼以独立不羁之民族,服属于罗马之宇下,其反拨之力最盛,久而久之,此义遂由日耳曼民族而倡佯于欧洲大陆。苟为他族所箝束欺压,则必洒国民之颈血以争之,掷国民之颅骨以易之,绵延数十载,以至百年,必得所欲而后止……湖南者,吾湖南人之湖南也。铁血相见,不戁不竦,此吾湖南人,对于湖南公责也\footnote{杨毓麟:《新湖南》}。

\end{quote}

《新广东》与《新湖南》两文很快便并行于世,为粤、湘土豪策动独立运动提供了坚实的理论基础\footnote{据革命派冯自由记载,“《新湖南》一卷,鼓吹湘省脱离满清独立之说甚力,与粤人欧榘甲著之《新广东》并行于世。”见冯自由:《革命逸史》上,页235}。欧榘甲洒下的种子不但为南粤的真正自由开启了大门,还跨越了五岭,为湖湘的自由和大一统理论的瓦解埋下了伏笔。直到今天,欧榘甲的主张仍对南粤的自立和自由有着充分的指导意义。欧榘甲的伟大理论遭到了孙文和康有为这两个大一统分子的一致反对。康有为绝不认同欧榘甲的粤独理论,两人从此分道扬镳,欧榘甲只得转赴新加坡,在那里从事反对孙文的事业。1906年,欧榘甲试图与清帝国驻新加坡领事展开机会主义合作,干涉当地革命党人的活动。孙文视之为眼中钉,命其党徒设法暗杀欧榘甲。无奈之下,欧榘甲只得转投商海,组建振华公司,于1909年回到南粤,准备在广西贵县开矿。然而,无耻的康有为却抢先一步地向清帝国当局通风报信,称欧榘甲准备“借商谋乱”。欧榘甲无以立足,唯有心灰意冷地回到家乡隐居。1911年,欧榘甲在乡里卷入一次因琐事而起的纠纷中,遭人误伤而亡,年仅46岁。在康有为和孙文卑鄙的迫害下,粤独先行者欧榘甲自始至终都没有看到南粤获得自由的那一天,这着实令人唏嘘悲叹不已。不过,欧榘甲的理论已经为一大批南粤土豪提供了理论弹药。在下一节中,我们便能看到这些土豪奋起战斗的身影。

\section{通往独立之路:1904—1911年}

\indent 20世纪初,随着南粤文明开化进程的加快,大批近代企业在南粤如雨后春笋般涌现,土豪与资本家们亦开始自我组织起来,以维护自身及南粤的利益。1904年,广东总商会在广州成立,发出“提倡农工路矿各种实业”、“挽回利权”等口号。当时,南粤本土近代企业不但面临承受着清帝国官吏的沉重盘剥,亦面临西方、日本、香港同类企业的商业竞争\footnote{《简明广东史》,页534}。广东总商会成立后,立即围绕铁路问题与美国企业及清广东当局展开了激烈的斗争。

早在1898年,清廷因谋建连接广州、武昌的粤汉铁路而向西方企业借款。美国合兴公司获得粤汉铁路借款权,遂得以控制粤汉路权。据借款合同规定,粤汉铁路由合兴公司代筑,筑成后由该公司派人管理,直到偿清借款方可收回铁路。又据1900年的《粤汉铁路借款续约》规定,借款金额为4000万美金、借款期为50年、5年筑成全路、不得将合同转让予他国。然而不久之后,合兴公司便违反续约,将粤汉铁路三分之二的股票卖给比利时东方万国公司。到1904年,铁路仅建成了广州至佛山、三水间数十公里的支线。合兴公司缺乏商业契约精神的行为激怒了南粤土豪。是年10月14日,广东总商会联合一批士绅在广州举行会议,“决议力争废约”,随即电告清帝国外务部、商部,提出:

\begin{quote}

先以政争,继以腕力争,不得已当用铁血争……力争一年、十年、百年,不达目的不罢休\footnote{转引自子月:《岭南经济史话》(下),页177—178}。

\end{quote}

除发表强硬声明外,会议还选出绅、商、学各界66人为路权公所办事员,刊印数万份公启向海内外广泛传播。与此同时,湖南、湖北方面的绅商亦加入废约运动。合兴公司随即派员赴广州,提出“以美接美”方案,即合兴公司与清方合同作废后,美方另组织协丰公司承建粤汉铁路,并增筑五条支线。然而,合兴公司在此前的接连违约行为已透支了他们的信誉。南粤土豪现在只想将路权收至粤人手中,便通电公开回绝该方案。接着,美国驻上海领事古诺又提出所谓的“中美合办”方案,提议由清帝国与美方联手投资筑路。11月13日,该方案亦被南粤绅商回绝。其后,美国华尔街巨头摩根出面,以高价从比利时人手中收回1200股,声称合兴公司的大部分股票仍在美国人手中,南粤绅商无理由废约。南粤绅商立即回电,表示若美方不废约,则“三省(广东、湖南、湖北)商民另筑一路,以图抵制”。在如此激烈的抵制下,美方不得不同意废约。1905年8月,清湖广总督张之洞经向英国贷款,以675万美元高价买下粤汉铁路\footnote{子月:《岭南经济史话》(下),页178}。

废约运动胜利后,粤、湘、鄂三方代表议定,路款由三方分摊,粤、湘各出七分之三,鄂出七分之一。此外,铁路由三方商民各自修建本方境内路段。然而,清帝国却阴谋将粤汉铁路广东段据为己有。经张之洞授意,清广东巡抚岑春煊提出将粤汉铁路广东段由商办改为官商合办,并在广东增加多种捐税以为筑路筹款。此令一出,南粤绅商一片哗然,他们决不能允许清帝国将南粤来之不易的路权抢走。1906年1月12日,南粤各界代表在广州广济医院召开大会。会上,绅商代表黎国廉、梁庆桂以激烈的言辞怒斥岑春煊派来的官方代表,表示粤人路权绝不会将铁路交给贪腐的清帝国官府。当夜,岑春煊以“破坏路政”罪名逮捕黎国廉,梁庆桂逃亡香港。为救援黎国廉、保卫铁路,广东总商会于2月2日召开大会,提出招股商办铁路、每股5元。与会者当场认签180余万股,筹得近千万元资金。至6月21日,南粤人团结一致,共集股8817565份、集资4400余万元。参与集股者不仅有豪绅和富商,更有大批普通平民。在土豪们的感召下,无数南粤民众亦纷纷从自己微薄的收入中拿出5元或10元,为南粤的路权而战。此时,南粤人正在做为一个坚实的共同体与清帝国战斗。无数南粤商民踊跃集股的情形使岑春煊十分惊恐,认清了粤人实力的他唯有妥协,同意铁路商办,并将黎国廉释放。同年4月,商办粤汉铁路有限总公司成立。在1906—1911年间,筑路工程稳步进行,郑观应(香山人)、梁诚(黄埔人)、詹天佑先后出任公司主办,公司筑成广州黄沙至黎洞间长106公里的路段\footnote{龚伯洪:《商都广州》,页151}。

粤汉铁路之争是20世纪初南粤人在土豪的带领下首次集体对抗清帝国的行动。南粤豪商的政治势力日渐高涨,于1907年在广州成立了粤商自治会。很快,南粤土豪成立了权力更大的自治组织,那便是谘议局。1908年8月27日,清廷颁布《钦定宪法大纲》,规定于9年之间完成立宪,清末“预备立宪”由此拉开帷幕。据《大纲》规划,清帝国各省需于当年筹办准议会组织谘议局,并于1909年完成谘议局选举。选举分为初选、复选两次。凡有选民资格者均可参加投票,选出若干候选人,是为“初选”;候选人之间再互相选举,产生规定数额之议员,是为“复选”。按照规定,选民除需有本省籍贯及25岁以上年龄,尚需具备以下条件:或办理学务或其他公益事业三年以上者;或中学及以上毕业者;或举贡生员以上出身者;或曾任文七品、武五品以上官职者;或有5000元以上资产者。至于候选人的年龄限制,更提高到30岁以上\footnote{《广东通史》近代下册,页270}。如此严苛的条件,使选民及候选人只能从各邦士绅、豪商中产生,客观上为各邦土豪议政提供了良好条件。而有意参选的土豪,又往往是对清帝国立宪运动十分热心的立宪派。广东有选民资格者为141558人,占全省人口0.43\%;广西有选民资格者为42145人,占全省人口0.42\%。1909年7月2日、14日,广西、广东先后完成初选,广西的投票率约为四分之三,广东各地的投票率则在二分之一至四分之三之间\footnote{《广东通史》近代下册,页271;《广西通史》第二卷,页534}。由于这是南粤历史上的首次西式投票选举,许多土豪尚不甚了解此种制度,这样的投票率不可谓过低。8月25日、30日,广东、广西相继完成复选,各有94人、57人当选为谘议局议员。10月14日,广东、广西谘议局分别在广州、桂林宣布开会。广东谘议局因会堂尚未建成,暂借旧巡抚衙门为会场。至11月初,广东谘议局议堂在广州东门内的番禺县学附近落成(即今广东革命历史博物馆),该建筑造价12.6万两白银,仿美国白宫而建,耸立于苍翠的松柏中,拥有雄壮的圆顶和宽敞的西式回廊。议堂楼下为议席,楼上则可供上千人旁听。至于广西谘议局的地址,则在桂林王城一带\footnote{《广东通史》近代下册,页271;《广西通史》第二卷,页534}。

清廷推行“预备立宪”运动,其本意为维持清帝国摇摇欲坠的腐朽统治,通过适度向各邦土豪放权防止革命运动。然而,“预备立宪”毕竟使各邦土豪有了一个合法的议政、参政平台,其积极作用无需完全否定。广东的94名议员中,固然有57人担任过清帝国各级官吏,但也有至少20人为豪商出身,其中包括广州、汕头的商会领袖区赞森和肖永华。可以说,南粤土豪在谘议局中是有不容小觑的影响力的。值得注意的是,广东议员中还有邹鲁、古应芬、陈炯明三人暗中拥有同盟会会员身份\footnote{《广东通史》近代下册,页273—274}。在即将到来的历史节点中,他们将扮演出人意料的角色。

据清帝国的《谘议局章程》规定,谘议局“为各省采取舆论之地,以指陈通省利病,筹计地方治安为宗旨”,具体职权有议决本省“应兴应革”、“岁出入预算”、“岁出入决算”、“税法及公债”、“担任义务之增加”、“单行章程之增删修改”、“权利之存废”等。不过,若议员“议事有逾越权限不受督抚劝告”、“所决事项违背法律者”、“所决事项有轻蔑朝廷者”、“所决事项有妨害治安者”,各省督抚可谘议局停会乃至“奏请解散”。由此可知,清廷只给各省督抚及谘议局的权利分配划出了一条原则性的模糊边界,议员们在此基础上大有文章可做。在清廷和督抚看来,谘议局也许只是个装点门面的咨询机构,若“不听话”便可任意解散;在议员们看来,谘议局却是个代表本土利益、拥有立法权、并能监督行政权的强大机构\footnote{《广东通史》近代下册,页275}。在此基础上,两广谘议局于1909—1911年间与清帝国督抚展开了激烈的宪制斗争。

1909年10月,广东谘议局召开成立后的首次常年会,做出定期一律禁绝赌博的决议。然而,清两广总督袁树勋却认为博彩业乃帝国重要财源,若不开征“赌饷”抵补,则决不能禁赌。议员们针锋相对,提出“赌饷”只会加重粤民负担,断无开征之理。1910年5月,议员举行临时会,再次做出定期一律禁绝赌博之决议,袁树勋则仍坚持己见。同年10月,广东谘议局举行第二次常年会,议员们通过强硬议案,要求袁树勋在三日内向北京“代奏”禁赌一事,否则谘议局便停议力争,直至全体辞职。11月9日,谘议局讨论议员苏秉枢的安荣公司开设赌铺一事,有议员提议安荣公司立即停止设赌。然而,苏秉枢却买通多数议员将此提案否定。消息传出,社会舆论哗然,主张禁赌的43名议员在议长的带领下联名辞职,于广州府学明伦堂聚集全广东数千选民、民众召开声势浩大的集会,要求清廷即刻在广东禁赌,并开除“庇赌议员”。在激愤的社会舆论下,35名议员被迫辞职,清廷亦只得让步,同意自1911年3月30日起在广东禁赌,空出的议员席位则另行改选。1911年3月,广东开始全面实施禁赌,清广东当局出现200余万两的财政损失。为弥补损失,清广东当局多次要求开征多种苛捐杂税,但谘议局只同意加征屠捐、烟丝捐共60万两,反对征收爆竹、香烛、元宝、桑基、鱼塘等捐。新任两广总督张鸣歧吸取前任教训,不敢得罪南粤土豪,只好另谋他法,靠向日本、英国银行分别借款160万日元和500万日元补足收入,这又使他遭到议员们的激烈攻击\footnote{《广东通史》近代下册,页278}。这样,经过一年多的抗争,南粤土豪在禁赌问题上对清帝国取得完胜,广东百姓免除了上百万的税负。

在广西,议员们也对清帝国的武断之治进行了卓有成效的抗争。1909年11月,广西谘议局召开首次常年会,做出全面禁烟的决议,议定于1910年5月9日禁绝全广西之鸦片土膏店。然而在1910年1月,时为广西巡抚的张鸣歧却擅自更改决议,将禁决期延长至该年9月4日。不久后,广西劝业道胡铭槃又与之勾结,将禁决期再延长五个月。此举一出,议员哗然,纷纷指斥张、胡二人“视决议为儿戏”、“实属专制之极”。1910年10月初,广西谘议局议员宣布集体辞职。时张鸣歧已升任两广总督,新任广西巡抚魏景桐被议员的强硬态度吓得六神无主,忙将禁烟决议“代奏”北京,得到清廷批复,要求禁烟一事按谘议局原议施行。10月12日,广西谘议局在胜利的欢呼声中召开第二次常年会。在和清帝国的宪制斗争中,广西土豪们亦大获全胜\footnote{《广西通史》第二卷,页540}。

两广谘议局议员不但积极维护乡邦利益,亦对清帝国开设国会一事十分热心。据《钦定宪法大纲》规定,清帝国应于1910年开设准国会组织资政院,并在1916年颁布宪法、正式召开国会。各地谘议局议员大都对清帝国的“预备立宪”寄予厚望,他们多认为召开国会的时间太迟了。1909年12月,在清末立宪派领军人物张謇(吴越南通人)的召集下,十六省谘议局代表齐集上海,讨论速开国会问题,广东代表沈秉仁、陈寿崇、广西代表吴锡龄出席了此次会议。会议决定推举沈秉仁率代表至北京请愿,要求清廷“速开国会,建立责任内阁”。1910年1月16日,请愿代表团到达北京,在都察院大门外列队请愿,被清廷拒绝。沈秉仁回到广东后,立即成立广东支部,发动商会、海外粤侨联名请愿。南粤及各邦支持立宪的绅商、民众亦纷纷加入请愿行列,各省议员在短时间内便征集到了超过20万个签名。1910年6月16日,包括两广代表团在内的17个请愿团体共150人在北京再次递交了代表20余万人的请愿书,第二次要求速开国会,清廷仍然置之不理。9月23日,清帝国资政院在北京如期召开。196名议员中,“钦选”议员和民选议员各居其半,除新疆外的帝国各省都有议员身列其中。资政院的开幕使各邦立宪派看到了希望。10月7日,规模空前的第三次国会请愿运动在北京开始,各邦数以千计的代表向资政院进发,直隶、河南、山西、福建、四川等省均亦爆发数千人参与的群众集会,一致要求清廷速开国会。17日,正在举行第二次常年会的广东、广西谘议局均通过速开国会案。迫于议员的压力,两广总督袁树勋等八名督抚亦联名通电清帝国军机处,要求“立即组织责任内阁”、“明年开设国会”。11月3日,清摄政王载沣不得不在声势浩大的请愿下做出妥协,同意于1911年成立责任内阁、1913年成立国会。各邦请愿代表见此,大都见好就收,纷纷解散\footnote{《广东通史》近代下册,页289—292;《广西通史》第二卷,页538}。从表面上看,立宪派的国会请愿已基本成功。可是,载沣却另有打算。

自1908年担任清帝国摄政王起,载沣因其根基不稳,一直对晚清以来权力日渐膨胀的地方督抚进行打压,重用爱新觉罗宗室。1911年5月8日,清帝国如约成立责任内阁。该内阁的13名成员中,竟有7人为皇族,史称“皇族内阁”。如此一来,原本应具有广泛代表性的责任内阁便成了爱新觉罗氏的“家天下”。深感失望的立宪派对清帝国失望透顶。在此倒行逆施下,清帝国彻底失去了东亚大陆各邦土豪的人心。现在,各邦土豪终于认识到清帝国是靠不住的,要获得自由只有靠自己。在南粤,土豪们终于开始筹划脱离清帝国、让家邦独立的伟大事业了。甚至连流亡日本的梁启超亦已看出清帝国即将崩溃。“皇族内阁”成立后,他愤而做出如下预言:

\begin{quote}

将来世界字典上,决无复以‘宣统五年’(1913)四字连成之名词者\footnote{转引自夏于全:《中国名人百传》,页17}。

\end{quote}

事实证明,梁启超之言可谓一语成谶。当南粤土豪对清帝国离心离德时,同盟会亦在加紧活动。如前所述,1908年以后,因在南粤的起事接连失利,同盟会转而在清帝国在南粤编练的新军中发展会员,企图打开局面。自1903年12月起,清帝国开始模仿西方国家军制编练“新军”。1907年8月,清帝国陆军部颁布练兵计划,预定编练新军36镇(师),每镇12512人,下设协(旅)、标(团)、营、队(连)、排、哨(班),其中近畿4镇、四川3镇、直隶、江苏、湖北、广东、云南、甘肃各2镇,其它省各1镇。不过,在此之前,掌握各地实权的督抚们早已开始编练新军。在南粤,新军编练起自1902年。是年,广西巡抚柯逢时率赣军二营入桂,扩为五营,于1906年整编为一协两标,分镇龙州、南宁、桂林。至1911年,广西新军仍未编足一镇,以赵恒锡为协统,兵力仅2500余。广东新军起于1903年岑春煊署理两广总督时,于是年编成常备军十五营。次年,两广总督张人骏将之统编为一协。1909年,袁树勋又扩广东新军为两协。1911年,清陆军部开始将广东新军整编为第二十五镇,以龙济光任统制,兵力3500余\footnote{刘仲敬:《民国纪事本末·新军篇》}。南粤境内的新军兵力虽少,然其装备精良、训练有素,具有强大战斗力。由于新军官兵多为受过近代教育之人,易于接受革命思想,遂成为革命党的重点发展对象。1910年,在同盟会的策动下,广州新军果然爆发了兵变。

早在1908年,新军的广州巡防营中便已成立同盟会外围组织“保亚会”,炮兵排长倪映典成为新军中的同盟会骨干。倪映典为江淮合肥人,曾在安徽新军中任职,后因策划反清兵变逃亡广东,混入广东新军,在赵声的介绍下暗中加入了同盟会。根据同盟会指示,倪映典在广州豪贤街天官里寄园巷五号设立总机关,又于雅荷塘六十七号、高第街宜安里等处设立分机关,在新军中积极发展会员。其发展方式,为利用节假日组织官兵至白云山濂泉寺郊游,再由革命党人从中活动,寻机让他们填写同盟会入盟书。通过这种方法,广州新军中很快就有超过一般人成为同盟会同情者乃至成员。1910年初,同盟会南方支部见时机已成,决定于是年2月24日在广州发动兵变,以赵声为总指挥、倪映典为副总指挥。然而在2月9日,因有新军官兵与广州警察发生冲突,局势陡然紧张,倪映典决定将起事时间提前至15日。在前往香港获得黄兴首肯后,倪映典于11日返回广州东郊燕塘的炮兵驻地,士兵们斗志旺盛,已做好战斗准备,表示随时可以出战。倪映典无法抑制部下情绪,只得同意于次日起事。这时,赵声尚在香港,未及奔赴前线。12日晨,倪映典率一部炮兵起事,枪杀炮兵营管带齐汝汉,各营纷纷响应,革命军膨胀至千余人,共推倪映典为司令,随即兵分三路进攻广州城,在沙河顶茶亭一带与清水师提督李准、防营统领吴宗禹所部两千余名清军展开激烈交火。在清军的阻击下,革命军进展不利,前锋工兵营在清方机枪射击下伤亡惨重,倪映典本人亦中伏受伤、坠马而亡。因总司令战死,革命军余部绕白云山退回燕塘,又遭清军阻击,全部溃散\footnote{《简明广东史》,页544}。同盟会寄予厚望的广州新军兵变,就这样以仓促而起,以惨败而终。

广州新军兵变虽败,但孙文仍未死心,决定在广州再次起事。1910年11月中旬,孙文、黄兴、赵声、胡汉民等同盟会骨干于槟榔屿召开会议,制定了一项庞大的作战计划,准备首先攻取广州,然后由黄兴进攻两湖、赵声经江西进攻南京,长江流域各省响应,而后会师北伐。会后,孙文赴北美洲筹款,黄兴则赴香港主持军事。1911年底,同盟会于香港跑马地成立统筹部,由黄兴、赵声分任正、副部长,选调南洋粤、闽侨及粤、桂、闽、苏、皖、蜀等地革命青年共800人成立“选锋队”,准备在广州新军内同盟会员的接应下再次发难。经数月准备,选锋队齐聚香港,军械弹药亦已运入广州城内外的36处秘密据点。4月8日,统筹部决定于五日后起兵。然而同日,广州城中发生了同盟会员温生才(梅州人)在谘议局前枪杀清广州将军孚琦的惊变,清方随即在广州进行戒严,大举搜捕革命党人。温生才系一激进革命者,其行动出于自发,完全打乱了同盟会的既定部署。因清军开始在城中搜捕革命党人,起事被迫推迟。18日,温生才在谘议局前被清方处决。23日,因局势愈加险恶,黄兴自香港潜入广州,进入设于越华街小东营五号的总指挥部主持大局。为保存实力,黄兴命赵声率300人离开广州避至香港,准备取消行动。这时,他听说有一支新军自顺德入驻广州,其中哨官大都为同盟会员,遂决定以仍留在广州的少数选锋队员展开一场豪赌。27日下午5时半,随着一声螺响,黄兴率约130名臂缠白布的选锋队员跃出潜伏地小东营,沿街枪杀巡警,发动“起义”。革命军攻入两广总督署,打死清军管带金振邦,总督张鸣歧在枪声中慌忙逃往水师行台,命令李准立刻出兵。革命军遍搜张鸣歧不得,便放火焚毁督署,由东辕门冲出,与李准所部大队清军遭遇。激烈的交火中,黄兴右手中弹、两指齐断,唯有下令分散突围。一片混乱中,黄兴率十余人逃至双门底大街。这时,闻枪声而响应的数十名巡防营中的同盟会员赶来增援,因与黄兴缺乏联络,竟不辨敌友地与之爆发枪战。混战中,黄兴的部下溃散,其本人逃入一间小店,易装出城逃至河南(海珠岛),九死一生地潜回香港。至于留在城中的革命军则各自为战,死者、被俘者过百,余者皆逃散。战死者皆曝尸街头,其状惨不忍睹。事后,同盟会员潘达微(番禺人)冒死潜往河南,与广州士绅领袖江孔殷(南海人)会面,请求他协助埋葬“烈士”。江孔殷当时正受清当局之命,以“清乡总办”之衔搜捕革命党,但做为南粤土豪头面人物早已对清帝国失望透顶,对同盟会的反清活动有所同情,乃鼎力相助。经江孔殷庇护,潘达微偕同百余忤工在清当局的眼皮下将林觉民(闽越闽侯人)、喻培伦(巴蜀内江人)等72人之遗体安葬于广州黄花岗。事后,孙文等人将此次规模不大的失败渲染为一部无比壮烈的革命史诗,称此役死者为“黄花岗七十二烈士”,厚颜无耻地利用无数青年的尸骨为自己捞取着名实不符的政治资本\footnote{《佛山文史》第7辑《华侨、港、澳史料专辑》,页103—104}。

在各邦土豪离心离德 、革命活动此起彼伏的形势下,清帝国迎来了末日。1911年5月9日,在“皇族内阁”成立的第二天,清邮传部大臣盛宣怀悍然推出“铁路国有”政策,出尔反尔地将已归商办的粤汉、川汉铁路收归清帝国“国有”。在南粤、两湖、巴蜀,清帝国的倒行逆施激起轩然大波。粤汉铁路商办是南粤人辛苦争来的权利,这条铁路凝结着粤人团结战斗的英勇历史。现在,清帝国却要蛮横地将其夺走,此种行为实与抢劫无异。6月6日,粤汉铁路的千余股东在广州召开大会,要求清廷取消国有令,“以昭大信”。海外粤侨亦积极投身捍卫南粤财产的斗争,越南海防的粤侨勇敢地发出了将“劫夺商路”者“格杀勿论”的声明。张鸣歧却无视南粤人正义的呼声,悍然宣布股东大会的决议无效,从而激发更大规模的反抗。广东百姓纷纷弃用官发纸币,前往银局“持票领银”,每日达数十万两。在南粤人万众一心的斗争下,清帝国在广东的白银储备被搬运一空,清广东当局陷入无银可用的窘境,只能通过向清帝国度支部和日、英、法、德四国银行团借款维持运转。16日,张鸣歧气急败坏,蛮横地逮捕抨击“铁路国有”政策的记者,并表示要将有“不轨言行”的粤人“立予拿办”、“格杀勿论”。至此,南粤人与清帝国的两广总督已经互相发出“格杀勿论”的威胁,双方彻底决裂。铁路股东退至香港继续斗争,于9月3日成立广东保路会,通知谘议局、粤商自治会及各商会派代表参加。开会之时,近万人冒雨齐聚会场,通过了“保路权,维持完全商办”的决议,发誓一定要守护南粤的财产。会后,大会派代表赴南洋进行宣传,将运动向海外粤侨扩散\footnote{《简明广东史》,页551}。

张鸣歧之所以如此有恃无恐地采取高压政策,是因为他相信自己掌握着龙济光的新军第二十五镇这支王牌武力。李准的水师虽曾在4月的战事中救过张鸣歧一命,但张鸣歧却对李准颇为忌惮,令其居于龙济光之下。8月13日中午,李准在双门底大街被革命党人投掷的炸弹炸伤。李准本已对张鸣歧心怀不满,更因此事受惊,便拒绝了张鸣歧搜捕革命党的命令。张鸣歧见此,乃趁机剥夺李准之一部兵权,并派人将水师在虎门要塞的火炮撞针卸去。这时,清帝国的总崩溃已经开始了。9月,巴蜀的保路运动转入武装斗争,四川保路同志会与同盟会结盟,发动二十余万起义军猛攻成都。清廷忙派湖北新军入蜀增援,正导致武昌空虚。1911年10月10日,著名的“武昌起义”爆发,辛亥革命拉开帷幕。两日后,革命军夺取武汉三镇,成立湖北军政府,湖北脱离清帝国独立。同月,湖南、陕西、江西、山西、云南相继独立,东亚各邦纷纷摆脱清帝国的桎梏,迎来了解放。在此局势下,李准既担心张鸣歧对他卸磨杀驴,又怕遭革命党清算,便派幕僚赴香港与胡汉民接洽,决定向同盟会投降。与此同时,老奸巨猾的龙济光亦借口其第二十五镇尚有一标步兵未完成编练,不肯正式出任镇统。因李准已离他而去、龙济光态度暧昧,张鸣歧无将可用,唯有坐困总督署等待命运的裁决。10月25日,革命党人李沛基(汕尾人)在仓前直街以炸弹成功暗杀新任清广州将军凤山,随后成功逃脱,张鸣歧更如惊弓之鸟。在广州城外,革命军则早已遍地开花。10月12日,渗入柳州新军中的革命党人率先发难夺城,打响了南粤1911年独立战争的第一枪。24日,粤西化州同盟会员彭瑞海起兵攻克化州县城。30日,革命党人王兴中起兵夺取新安县城。同日,广西梧州士绅周之济联合新军、革命党起兵响应,宣布梧州脱离清廷统治。11月1日,陈炯明、邓铿(惠阳人)在南洋粤侨的资助下于惠州淡水组织起一支8000余人的革命武装,号称“循军”,以“蓝底井字旗”为军旗,于八日后在新军响应下经激战攻克重镇惠州。陈炯明任循军总司令、邓铿任参谋长。11月7日,清广西巡抚沈秉堃在谘议局及新军的威胁下于桂林宣布广西独立,沈秉堃被举为广西都督、广西提督陆荣廷被举为副都督。三日后,因桂林巡防营发生兵变,沈秉堃惊恐地以“北伐”为名逃出桂林,都督遂为驻于南宁的陆荣廷所得。此外,在南粤的其余州县,当地驻军、绅商、官吏纷纷宣布“反正”,清帝国在南粤的统治已然土崩瓦解。在广州城内,决断的时刻终于到了\footnote{上述战争过程,参见《简明广东史》,页554—556;《广西通史》第二卷,页583—592;郭廷以:《近代中国史纲》,页275}。

10月25日,即凤山遇刺当天,广州城内各绅商自治团体齐聚西关第九甫路之文澜书院开会,商议广东的未来。在江孔殷的主持下,会议推举耆绅邓华熙、梁鼎芬为正副主席。在会上,一场决定南粤命运的辩论开始了。江孔殷提出:

\begin{quote}

救亡之有二策:一、速行联邦政策,廿二省分治,以阻革命风潮;一、政府假权督抚……改良政治,以阻国民独立\footnote{丁身尊:《广东民国史》,页91}。

\end{quote}

江孔殷开出的两个方案中,一为由广东提倡清帝国各省分治,独立后的各省不再重新统一,而是结为松散的联邦。若此方案能够实施,则历史很可能会向欧榘甲理想中的方向发展,南粤将以粤桂联邦的形式赢得真正的独立,东亚大陆各邦亦将赢得真正的解放。在这种情况下,蔓延东亚大陆各邦、以建立大一统国家为目标的革命,自然也就不可能成功了。另一方案则为仍在表面上维持清帝国在广东的统治,并由张鸣歧实行更彻底、更符合南粤土豪利益的政治改革,以此阻断激进的革命。江孔殷在第二种方案中仍对清帝国抱有幻想,这不能不说是他的失察之处。不过,若我们能明了南粤在此后的历史走向,便会惊叹江孔殷所提的第一个方案是多么地有预见性。当时,革命军虽占据了南粤的多座城镇,陈炯明的循军更已逼近广州,但南粤的大部分地域并未被革命党渗透太深。在琼州、廉州、钦州、潮州、梅州等地,脱离清帝国的独立运动是都由保守的土豪甚至原清帝国官吏主导的。若南粤土豪能够坚持自立,不与革命党合作,那么由士绅和豪商主导的保守政治秩序必能长期确保南粤的自立,革命党则将日益边缘化。这一方案所展示的,无疑是一条继承欧榘甲的规划、使南粤获得真正自由的康庄大道。

然而,江孔殷的方案却遭到了其他代表的否定。粤商自治会的代表谭民三、陈惠普表示,江孔殷的方案“断不济事”。29日上午,广州各界团体又在爱育善堂开会,决定“承认共和”,派人赴香港与革命党接洽\footnote{《简明广东史》,页552}。与会的南粤土豪虽然对清帝国深为不满,亦大都有让南粤脱离清帝国的意愿,但做为百年前的人,他们并未意识到南粤民族的存在,无法清楚地分辨出同盟会倡导的各省独立与南粤的真正独立有什么不同。在此种民族认同不明朗的情况下,他们做出了错误的抉择,选择与他似武力强大的革命党联手,从而打开了潘多拉魔盒。绝好的机会便在那个上午如流星般划过,与南粤失之交臂了,这真是个令人抚膺叹息的巨大悲剧。若此刻的他们能够知道此后数十年间南粤的命运,他们定然会为自己的决断追悔莫及。

当日下午,会议转场至文澜书院继续进行。在书院门口,一面写有“广东独立”四字的大旗高高飘扬,令广州民众欢欣鼓舞。大批民众涌上街头,高呼“独立万岁”抒发着自己脱离清帝国桎梏、获得自由的喜悦心情。在广州城内外,处处人山人海、张灯结彩、鞭炮轰鸣,一派喜庆气氛。在这样的情况下,张鸣歧为了自保,只得同意广东“和平独立”。然而不久之后,他突然接到清廷要求他维持地方秩序的电报,遂临时变卦,表示反对独立,再次钻入总督署躲了起来。士气高涨的广州民众随即以全城罢市应对,要求张鸣歧立即表态。11月8日,广州各界代表在广东总商会召开大会,逼迫张鸣歧宣布独立。无计可施的张鸣歧只好派胡鸣槃到会传信,表示他将在次日宣布广东独立。随后,张鸣歧发布一张告示:

\begin{quote}

国势日危,大局岌岌。多数人民,主张独立,现在筹议……定期竖旗,以昭正式\footnote{转引自《简明广东史》,页553}。

\end{quote}

告示一出,广州各界代表立即推举张鸣歧为都督、龙济光为副都督。然而张鸣歧自知已对粤人犯下太多罪孽,不敢就职,当即逃亡香港。龙济光则默不作声,选择继续观望。各界代表乃改选同盟会员胡汉民为都督。11月9日,广州市民一觉醒来,发现南粤的心脏广州城已经光复,广东已然独立。10日,胡汉民到达广州就任都督,宣布广东军政府成立。17日,各界代表又增选陈炯明为副都督。至此,虽然同盟会会员控制了广东军政府的大权,但胡汉民、陈炯明毕竟都是南粤人,此时的广东和广西毕竟是完全独立、不隶属于任何岭北政权的政治实体。也就是说,此时的南粤在法理上已然完全地摆脱清帝国的统治,傲然独立了。南粤走完了发明民族与文明开化的近代史,开始走入风云激荡的现代史。然而,此种由土豪推举出的、由同盟会会员主导的怪异独立究竟能维持多久?很快,在大时代的浪潮下,前所未有的剧变将在南粤大地上演!


