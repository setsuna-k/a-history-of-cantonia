\chapter{何真政权与第六次北属}

\section{失之交臂的机会窗口:何真政权}

公元1351年,红巾流寇战争在黄河流域打响,元末大洪水爆发,流寇入侵的危险再度降临南粤。次年,湖湘流贼窜入粤北,一度攻陷乐昌,并凶残地驱赶当地数万百姓进攻韶州。韶州城虽未被攻破,但此次战事无疑给南粤敲响了警钟\footnote{《广东通史》古代上册,页1018}。如果粤人再不行动,两宋之际湘赣流寇大规模屠掠粤北的惨剧将再次发生。

随着流寇入侵粤北,珠三角亦发生了社会失序的现象。1353年,南海人邵宗愚纠集起一批海寇,自称元帅,开始了他在珠江口的行劫生涯\footnote{阮元:(道光)《广东通志》卷168《前事略六》}。同年,粤北再次发生流寇入侵事件。元廷急调驻广西的两千蒙古精兵赴援,结果被流寇击败于连州\footnote{段雪玉:《 庐江郡何氏家记 所见元末明初的广东社会》同注395}。蒙古精兵的惨败令我们的祖先看清了元帝国的虚弱。他们意识到只有自己行动起来,才能保护一方乡土不受流寇和海寇的侵袭。1354年,在饱受海寇劫掠之苦的东莞县(今东莞、深圳、香港),一批土豪宣布武装割据自保,传奇的南粤骑士何真随之登上了历史舞台。

何真,字邦佐,东莞圆头山村(今东莞茶园镇一带)人。1321年,他出生于当地一个有大片土地与佃户的富裕家庭,从小衣食无忧。何真八岁丧父,此后,他母亲对他进行了严格的教育,使他成为精通骑射、熟读诗书的文武双全之才。何真成年后,他丰厚的家财和优雅的举止使他得以跻身权贵之门,与地方土豪及广州城内的一批元帝国官僚颇有交游。这些人脉关系成为他的社会资源,为他日后的崛起打下了基础。1353年前后,何真获授河源县务官,不久后又被命令前往淡水管理盐政。其时,大洪水的浪头已清晰可见,红巾流寇扫荡中原,邵宗愚则率海寇频频袭击珠江口沿岸。他明智地选择了拒绝赴任,回到自己在东莞的庄园侍养母亲、静候世变\footnote{段雪玉:《 庐江郡何氏家记 所见元末明初的广东社会》}。

变局很快到来。1354年,东莞土豪群起割据自保。至1355年,珠江口东岸已经形成了群雄割据,互相合纵连横的局面,形势一如“十二使君之乱”时的越南红河三角洲。这些土豪共有十九家,他们或控制盐场、或有广阔的田土,各自筑起了城寨,拥有数百、数千不等的私兵。他们的割据情形详见下表\footnote{此表引自段雪玉:《 庐江郡何氏家记 所见元末明初的广东社会》}:

\begin{center}
	\begin{tabular}{ c c c }
		\hline
		序号 & 土豪姓名 & 割据地 \\
		\hline
		*1 & 李确 & 靖康场 \\
		*2 & 文仲举 & 归德场 \\
		*3 & 吴彦明 & 东莞场 \\
		*4 & 郑润卿 & 西乡黄田 \\
		*5 & 杨润德 & 水心镇 \\
		*6 & 梁国瑞 & 官田 \\
		*7 & 刘显卿 & 竹山下萍湖 \\
		*8 & 萧汉明 & 盐田 \\
		*9 & 黎敏德 & 九江水崩江 \\
		*10 & 黄时举 & 江边 \\
		*11 & 封微之 & 枫涌寮步 \\
		*12 & 梁志大 & 板石老洋枰柏地黄漕 \\
		*13 & 袁克贤 & 温塘 \\
		*14 & 陈仲玉 & 吴园 \\
		*15 & 陈子用 & 新塘 \\
		*16 & 王惠卿 & 厚街 \\
		*17 & 张祥卿 & 篁村 \\
		*18 & 张黎昌 & 万家租小亭 \\
		*19 & 曹任拙 & 湛菜
	\end{tabular}
\end{center}



在元帝国的统治衰弱后,仅仅珠三角东岸一隅之地便能出现如此复杂的群雄割据之局,南粤社会结构的复杂多彩由此可见一斑。唯有树大根深、遍布土豪的社会方能形成如此复杂的权力结构,这是南粤文明强大生命力的最好体现。面对群雄并起的局面,何真虽也散财召集乡兵,但弱小的他不得不暂时投靠割据归德场的文仲举。文仲举系一心胸狭窄之人,他无法容下英武善战的何真。不久后,两人决裂,何真率兵回到家乡。此时,何真家中遭遇变故,他的母亲与妻子皆因病过身。1358年,陷入绝境的何真只好率部投靠割据西乡黄田的郑润卿,再次过上寄人篱下的生活\footnote{同注395}。

在郑润卿麾下,何真过得并不如意。郑氏家臣皆嫉妒其才华,图谋暗算他。在家臣劝说下,郑润卿命何真出守黄田场诸要路以对抗文仲举、吴彦明、杨润德、刘显卿等人。倘若何真获胜,则自然能为郑氏开疆拓土。倘若战败,由于这些人都是郑氏姻亲,郑润卿大可将责任全部推到何真身上,任由他们消灭何真。何真看破了郑润卿的毒计,决心背水一战。虽然初战失利,但何军在瓢湖迳一战反败为胜,斩郑军760余级、俘400人。两次激战使何真丧尽资财,他明白东莞并非久留之地,遂于1359年率部东去,进驻惠州\footnote{同注395}。

当时,惠州的元帝国地方官乃何真故交,他好心地收留了何真。1360年,经过一年休整的何真恢复了实力,率部重返东莞,击败郑润卿的盟友黎敏德(割据九江水崩江者),收降被杨润德(割据水心镇者)打败的梁国瑞(割据官田者),兵威大震。次年,惠州的元将黄常据城杀官而叛,大肆蹂躏民众。抓住机会的何真率兵奔袭惠州,在当地父老的迎接下逐走黄常,接着又攻占循州,成为南粤境内一股不可小觑的势力。不甘失败的黄常勾结东莞豪族王成再攻惠州,又被何真击退。因何真稳定了惠州的局势,元廷遂授他惠阳路判官、广东宣慰司都元帅之职\footnote{钱谦益:《国初群雄事略》卷15,页298}。两年后,海寇邵宗愚攻破广州,“据城纵火,杀掠居民”。已获得“官军”名分的何真率其子弟举兵北上,击退邵宗愚,“严号令,禁屠掠,广人大悦。\footnote{仇巨川:《羊城古钞》卷17}”

1365年,更大的危机降临南粤。两年前,割据湘楚的红巾流寇陈友谅在鄱阳湖大战中被朱元璋军杀死,其部将熊天瑞割据赣南成为独立势力。是年,熊天瑞欲掠夺南粤的财富,乃领兵数万南下侵粤。他自恃兵多将广,不可一世,攻陷韶州、梅州、循州等地,其舟师直抵胥江(今广州附近北江芦苞河段),威胁广州。雷雨交加中,何真率战船北上迎战。两军激战时,熊天瑞座舰的桅杆被雷电击中,其军士气大跌,纷纷败退。熊天瑞对南粤的威胁得以解除,何真从红巾流寇手中拯救了南粤\footnote{《东莞历史人物》,页53}。兵败的熊天瑞走投无路,乃举赣南及粤北韶州、南雄之地向朱元璋投降\footnote{《广东通史》古代上册,页1021}。

一波才平,一波又起。同年秋,曾被何真打败的王成与邵宗愚一同向何真展开进攻,东莞土豪李确(割据靖康场者)、张黎昌(割据万家租小亭者)亦发兵 tonk 何真为敌。在多路围攻下,何真只得放弃广州向东撤退,一路在敌军的缠斗下历尽艰险,好不容易才进入惠州城,并击退了尾随而来攻城的王成军。何真随后回师东莞,当地土豪在其声威下纷纷投降,王成被彻底孤立。1366年,何真将王成包围在其据点茶园营,与其宿敌展开了最后的决战\footnote{同注395}。

王成本为东莞小土豪,后率兵袭杀大土豪封微之(割据枫涌寮步者),成为东莞境内的大势力。如前所述,他曾与屠害南粤百姓的黄常、邵宗愚合作,是个政治德性十分低下的人,他的部下亦多有贪财好货者。包围王成后,何真开出“钞十千”的价码征募擒拿王成的勇士。很快,王成的家奴便将其捆绑起来,打开寨门投降。其后发生的戏剧性事件,充分展示了何真朴素的道德观与政治德性:

\begin{quote}
	成奴缚成以出,真予之钞,命具汤镬,趋烹奴,号于众曰:“奴叛主者视此。\footnote{张廷玉:《明史》卷130《列传第十八何真传》}”
\end{quote}

王成败死,岭南大震。1367年,何真平定粤东、再次击败邵宗愚并入主广州,其部将李质则控制了西江上游的肇庆、封州、德庆等地。至此,他已成为无可争议的南粤之主。“时中原大乱,南北阻绝,真练兵据险,保障一隅。\footnote{钱谦益:《国初群雄事略》卷15,页299}”短暂的机会窗口期出现在何真面前:究竟是效南越武帝、南汉高祖故事称帝岭表,还是继续输诚于北廷?他选择了后者,并杀死了劝他称帝的部下\footnote{同注395}。次年正月,朱元璋在南京称帝,建立明帝国。同月,明将汤与率军侵占闽越,闽越的守护者、元福建行省平章事陈友定仰药殉难。至此,湖湘、江右、闽越已全被明帝国侵占,南粤陷入了生死存亡的关头。1368年二月,明帝国侵略军在朱元璋的命令下兵分三路汹涌南下,对南粤发起了全面的侵略。侵略军的编制为:

\begin{itemize}
	\item 东路:征南将军廖永忠、浙江行省参政朱亮祖率水师由福建入广东,此路为主力;
	\item 中路:赣州卫指挥使陆仲亨、副使胡通率赣州、南雄、韶州等地卫军南下,会同主力共取广东;
	\item 西路:湖广行省平章杨璟、左丞相周德兴率军由湖南入广西\footnote{《广东通史》古代下册,页40}。
\end{itemize}

面对大举袭来的明帝国侵略军,何真没有表现出足够的警惕。他大概没有认识到元廷与明廷的不同,正如困守北平的傅作义以为跟谁走都差不多。既然向元帝国输诚便能保境安民,那么向明帝国投降以换取南粤的安定,又有何不可?从这一刻起,他的命运被锁定了:他不但丧失了成为南越武帝的机会,甚至丧失了像陈友定那样荣耀地殉节的机会。短暂窗口期中错误的决断使他丧失了一切翻盘的机会,也使南粤不可避免地落入了洪武社会主义的魔掌。

当年三月,廖永忠率明东路军侵占潮州,何真当即派人赴潮州“奉表迎降”。其后,廖永忠率部由海路直入珠江口,登陆于东莞,何真率部下亲往迎接\footnote{张廷玉:《明史》卷130《列传第十八何真传》}。明中、东路军迅速在广州会师,攻破海寇盘踞的南海三山寨,将邵宗愚俘送广州斩首。至此,明帝国侵略军的中、东两路几乎未经过一场像样的战斗,便轻而易举地侵占了广东\footnote{《广东通史》古代下册,页41}。

然而,侵略军却在广西遇到了坚强的抵抗。1353年,明廷将广西之地从湖广行省分出,设广西行省,以蒙古人也儿吉尼为广西行省平章事。也儿吉尼曾在元廷中担任御史,乃刚正不阿之人。到达广西后,他在驻地靖江(今桂林)很快就如宋末的马堲一样得到了当地人的信任与尊重。1368年初,面对袭来的侵略军,也儿吉尼决定拼死一战。当年三月,明西路军攻陷全州;四月,进抵靖江城下,与自广东来援的朱亮祖部会合。英勇的靖江军民决定追随他们90年前的先辈,宁死不向侵略者投降。当月,明帝国侵略军的第一次强攻失败,仅在靖江城西门外就丢下了300具尸体。此后,惨烈、凄苦的攻防战持续了两个月,侵略军攻城益力,靖江军民竭力防御。六月,城中积储耗尽,侵略军登城而入,在城东伏波门俘虏了也儿吉尼,靖江陷落\footnote{《明太祖实录》卷28,洪武元年六月壬戌条}。广西各地的元帝国官吏随之纷纷对明投降,南粤完全落入了朱元璋的魔掌中\footnote{《广西通史》第1卷,页308}。

也儿吉尼被俘后,明军将他押往南京。在南京,他拒绝了朱元璋的劝降,慨然赴死。据史书记载载,当也儿吉尼被送至南京时,有一颗大如鸡卵的“青赤色”流星在天上划过\footnote{同注411}。这一流星不但呼应着也儿吉尼生命的陨落,亦象征着南粤与自由的失之交臂。随着明帝国侵占南粤,第六次北属时期开始,前所未有的苦难降临到了我们祖先的头上。

\section{洪武社会主义对南粤的蹂躏}

登基后不久,朱元璋便迫不及待地在明帝国境内开始了大规模的社会改造,南粤亦难逃此劫。早在占领南京之前,朱元璋的淮右流寇大军就以“烤人排的精湛技术著称”,渡江后则以发掘吴越人祖坟的围攻战术著称。登基称帝后,流寇出身的他致力于消灭土豪、改造蒙元留下的五彩斑斓、族群复杂的东亚大陆。他渴望高度同质化、扁平化的社会,因为只有那种社会符合他猥缩而卑贱的世界图景。朱元璋所规划的社会,笔者称之为“洪武社会主义”\footnote{刘仲敬:《朱重八的历史使命》}。1370年,朱元璋发布了一道杀气腾腾的“圣旨”,要求军队在全帝国范围内统计人口:

\begin{quote}
我这大军如今便不出征了,都教去各州下着,绕地里去点户、比勘合。比着的便是好百姓,比不着的便拿来做军。比到其间有司管理隐瞒了的,将那有司官吏处斩。百姓每自躲避了的,依律要了罪过,拿来做军。钦此\footnote{转引自刘志伟:《在国家与社会之间:明清广东里甲赋役制度研究》,页39}。
\end{quote}

朱元璋规定,帝国治下的所有人都要被强制编户,以便于进行扁平化管理。在帝国境内,抗拒编户的人便“不是好百姓”,就要被充军。若有地方官不认真执行编户工作,则要被处决。十一年后,朱元璋在另一道“圣旨”中进一步规划了他理想中的社会蓝图:

\begin{quote}
	

诏天下编赋役黄册,以一百一十户为一里,推丁粮多者十户为长。余百户为十甲,甲凡十人。岁役里长一人,甲首一人,董一里一甲之事。先后以丁粮多寡为序,凡十年一周,曰排年。鳏寡孤独不任役者,附十甲后为畸零。儒僧道给度牒,有田者编册如民科,无田者亦为畸零。毎十年有司更定其册,以丁粮增减而升降之。册凡四:一上户部,其三则布政司、府、县各存其一。上户部者册面黄纸,故谓之黄册\footnote{《明史·志第五十三·食货一》}。

\end{quote}

黄册是登记里甲人户的大型户口本。里甲绝非一些毫无格局感的学院派所说的那样,是所谓的人民自治组织,其实是人民公社。被编入里甲的民户不得任意迁徙,只能在原地充当帝国的纳粮赋役者\footnote{冯天瑜:《中国文化生成史下》,页628}。在里甲当中承担徭役的,是“里长”和“甲首”\footnote{关于里长、甲首应役,参见李江主编:《中国财税史》,页104}。只有有田产的人才可以充任“里长”、“甲首”,而那些没田的人被称作“畸零”,不在这个体系里,无需当差。这是一个压榨有产者的制度。在里甲内部,每十年要重新统计一次田产,因为随着时间的推移可能会有土地产权的变化。除此之外,朱元璋还设计了一种“粮长”制度,要求帝国境内的富户负责督促赋税的缴纳。这些富户如果无法上缴足额赋税,就要自己补足,直至倾家荡产\footnote{参见梁方仲:《明代粮长制度》}。朱元璋设计的是一个不断对有产者进行阶级斗争的制度,黄册则是阶级斗争的工具。

在南粤,朱元璋如欲推行这种制度,首先要做的事便是消灭土豪。实施这一行动时,他找到了一个十分“好用”的工具:何真。做为曾经的南粤守护者,何真与南粤各地的土豪有着广泛的联系。他的大批旧部仍然散处南粤各地,担任着南粤社会的军事凝结核。如要将粤人改造为明帝国的顺民,就必须除掉这些土豪、军人,而这些人不会服从除何真外的任何人。因此,朱元璋的第一步行动便是将何真调离南粤。何真投降后,朱元璋很快便将他召至南京,授予他江西参政之职。1370年,何真又被调为山东参政。此后十八年间,朱元璋先后令何真在山东、山西、浙江、湖广等地为官,并在1381年侵略大滇的战役中命他与儿子何贵“规画粮饷,开拓道路”。曾经的南粤保卫者何真便这样脱离了乡土共同体,一步步落入朱元璋的陷阱。在此期间,朱元璋一次次地命令他利用曾经的威望召集南粤土豪、旧部迁往岭北。在利用他挤干南粤这颗甜美的橘子之前,朱元璋并不会对他下手。两人间的关系,像极了一场变态至极的SM。在此特将史籍中何真召集旧部、土豪的记载抄录于下(人数一律写作阿拉伯数字):

\begin{quote}
	
洪武四年(1371)辛亥:命何真还广东,收集旧将士。还京,复归山东。

洪武五年(1372)壬子六月:参政何真收集广东所部旧卒3560人,发青州卫守御。

洪武六年(1373)癸丑六月:真招还广州旧所部兵士20777人,并家属送京师还朝。

洪武十六年(1383)癸亥:是年,何真致仕(退休)。复命真及真子贵往广东收集土豪10623人还朝。

洪武十七年(1384)甲子闰十月:致仕布政使何真复招集广东旧所部兵3423人还京师\footnote{执经生:《死去的南粤骑士:何真》}。

\end{quote}

并非所有何真旧部及南粤土豪都愿意跟随何真北去。他们中有的人选择了反抗。1381年,东莞人苏友兴、广州人曹真、苏文卿等人联合山民起兵反抗。起义军接连攻克东莞、鹿步、清远大罗山等处,一度逼近肇庆、南海,多次打败明军的“进剿”。最后,明南雄候赵庸统军残酷镇压了起义,“克寨十二,擒贼万余人,斩首二千级”,将“降贼”尽数发往泗州。第二年,广东境内爆发更大规模的反抗,赵庸则展开了更血腥的屠杀,“凡获贼党一万七千八百五十一人,贼属一万六千余,斩首八千八百级。”短短两年内,明廷通过大屠杀镇压了南粤的反抗。而在一年后,已致仕的何真便受命入粤招集土豪北迁。不知其时的他,究竟怀着一种怎样的心境踏入他曾拼死守护的乡土,又有何颜面与家乡父老相对?到这时,那个曾戎马倥偬、保卫乡土的南粤骑士,已经死了。

1387年,朱元璋封何真为东莞伯,“赐第”于南京。次年,何真病逝,葬礼备极荣宠,其子何荣袭封东莞伯。1393年,朱元璋将何荣罗织入“蓝党”,族诛何氏。乡居东莞的何真之弟何迪终于聚众发动了迟到的反抗,在击毙三百多名侵略军后被俘送京师杀死。何氏族人承认谋反的口供颇为详实,其可信度大致相当于布哈林在大清洗中承认自己 hày 帝国主义间谍。1394年,朱元璋对社会凝结核已损失惨重的南粤发动社会改造,大批百姓被编入军籍,承担起沉重的世袭军役。还有许多人被编入“画地为牢”的里甲人民公社,丧失了自由迁徙的权利\footnote{执经生:《死去的南粤骑士:何真》}。

除了将粤人编入里甲与军籍外,朱元璋还要斩断南粤的海洋贸易,将粤人彻底变为明帝国的农奴。如前所述,早在南宋治粤时期,我南粤的商船就已能航行到苏伊士地峡,南粤的商人更曾到达西班牙、西西里和非洲西北角。虽然元前期的海禁 tonk “官船”制度严重摧残了南粤的海洋贸易,但在元中后期开禁后,我们的祖先仍垄断了东亚和印度间的航线。对于这一切,朱元璋完全不能容忍,出身内陆流寇的他对海洋怀有天然的恐惧。上台后不久,朱元璋便下达了“寸板不许下海”的命令,切断了南粤、闽越、吴越沿海居民的生计\footnote{胡宗宪:《筹海图编》卷4}。此外,他还禁止海外舶来品在市面流通,甚至连南粤出产的香木亦不准卖往岭北。此后,明帝国又将粤东南澳岛、珠江口三灶岛、新会上川岛、下川岛的居民全部强制迁往内陆,并禁止沿海制造用于航海的二桅以上大船\footnote{《广东通史》古达下册,页176—177}。1374年,因东南三越的海洋经济几乎完全瘫痪,朱元璋干脆撤销了广州、泉州、宁波的市舶使司\footnote{潘义勇:《中国南海经贸文化志》,页126}。 

面对朱元璋的倒行逆施,我们伟大的祖先自然不会坐以待毙,他们发动了轰轰烈烈的反抗。除上述两次大规模起义外,洪武(朱元璋的年号,1368—1398)时期的粤人武装斗争还有很多。1370年,粤北阳山县的十万山民发动反明起义,被明廷调集广西的卫所军镇压。1372年,潮州百姓群起反抗,一度攻克揭阳、潮阳两县。同年,南宁的三千多名百姓为反抗当地明帝国军官强制征兵,亦发动武装斗争。1395年,广西数万瑶人、僮人又发动大规模起义,以更吾、莲花、大藤等寨为据点,向都康、向武、上林等地进军,最后被征南将军杨文佩、广西都指挥使韩观以血腥的大屠杀基本镇压下去\footnote{白寿彝主编:《中国通史》第9卷,页272}。韩观是一个变态至极的杀人魔王,他不但指挥部下屠杀了近两万名起义军,还将起义者的人皮剥下来做成坐褥、当众蒸食战俘的人脑和人眼\footnote{《广西民族与历史文化研究 第一辑》,页260}。可以说,就算是最残忍最变态的杀手,亦难以想像明帝国侵略军对我们祖先犯下的滔天罪行!就算是这样的恐怖屠杀,这批起义军仍然没有放弃战斗。他们化整为零,在广西的群山之间不断袭击明帝国侵略军。直到1420年代,仍有起义军余部在坚持战斗\footnote{潘义勇:《中国南海经贸文化志》,页126}。

1398年,朱元璋病死,明帝国的海禁政策略有松动。早在洪武末年,就已有南粤百姓冒死出海,与南洋、日本等地贸易。当时,珠江口有钟福全、李夫人自称总兵,坐拥二百余艘“倭船”,成为维系南粤与世界联系的重要渠道\footnote{郭棐:《粤大记》卷32}。还有不少沿海之民学习外国语言,暗中与被明帝国称为“番舶”的外国商船进行交易。1426年,更有潮州通事(翻译)刘秀招引日本船至饶平东里港湾,当地各村百姓纷纷“赴船领货”,公开进行外贸。气急败坏的明宣宗乃于1431年重申海禁,下令沿海民众不得“下番贸易,交通夷人”。然而,他根本挡不住粤人对自由的向往\footnote{《广东通史》古代下册,页178}。当时,有很多不甘受明帝国奴役的粤人、闽人逃离了帝国统治,流亡南洋。14世纪后期,爪哇满者伯夷国攻灭三佛齐王朝(又称室利佛逝),旅居当地的南海人梁道明被千余名粤侨、闽侨推为三佛齐国王,领兵守卫三佛齐之北疆、苏门答腊岛南端的旧港(今印尼巴邻旁)。此后十年间,有数万粤人跨过南海来到旧港,将当地建设为南粤的殖民地\footnote{关于梁道明的事迹,参见朱杰勤:《东南亚华侨史》,页24}。1403年前后,梁道明病逝,其副手潮州人施进卿继位。当时,有旧港粤人土豪陈祖义不服施进卿的统治,起兵作乱。1405年,明帝国官方的郑和舰队沿着粤人早在六七百年前就开辟的海路开始了所谓的“七下西洋”。次年,郑和舰队到达旧港。施进卿抓住机会与郑和结成机会主义的同盟,消灭了陈祖义的舰队。事定之后,陈祖义被郑和俘送南京杀害,施进卿则被明廷“封”为旧港宣慰司\footnote{关于施进卿的事迹,参见吴玉成《广东华侨史话》,页101—102}。

施进卿并非向明帝国屈膝投降的粤奸,而是一个善于利用明帝国的马基雅维利主义者。铲除陈祖义后,施进卿继续以三佛齐国王的身份治理旧港这一南粤殖民地。1423年,施进卿病逝,其子施济孙继位,此后旧港对明帝国“朝贡渐稀”。不久后,施济孙之妹施二姐篡位成为三佛齐女王,其姐施大姐(又名俾那智)反对其政变,走入爪哇投奔满者伯夷王朝,被任命为粤人聚集的海港革儿昔的“港主”,充当满者伯夷国王与外商的联络人\footnote{朱杰勤:《东南亚华侨史》,页25}。施济孙则向日本求援,未果。1424年,明廷再派郑和赴旧港调解王位争端,却完全无法干涉当地粤人的内政\footnote{参见吴玉成《广东华侨史话》,页101—102}。施二姐对旧港的统治持续了十七年。1440年,满者伯夷王朝攻占旧港,摧毁了当地的粤人政权\footnote{施忠连:《五缘文化:中华民族的软实力》}。旧港政权是粤人建立的第一个海外独立政权。在其治下,苏门答腊岛上的粤人享受了四十余年的安宁。这一段史事,实为南粤近代大航海的伟大前奏。

15世纪上半页,明帝国的洪武社会主义体制虽然有所松动,但依然对南粤维持着严酷的统治。在此情况下,我们的祖先仍不断地英勇反抗、不断地奔向南洋,演奏了一曲动人的自由壮歌。1448年,一场惊天动地的大乱在珠三角爆发,洪武社会主义在南粤的总崩溃开始了。这场夺取了数万条生命的乱事,便是著名的“黄萧养之乱”。

\section{洪武社会主义在南粤的崩溃:黄萧养之乱与佛山保卫战}

1449年在明帝国史上是个至为关键的年份,有人以之为明朝由盛而衰的时间节点。是年八月,明英宗率领的二十万大军在土木堡被蒙古瓦剌部全歼,英宗本人被俘,是为震惊东亚的“土木之变”。十月,瓦剌军围攻北京,明兵部尚书于谦指挥军民奋战三昼夜,方将瓦剌军击退。在帝国心脏地带发生如此剧烈的波动时,远在南粤的事情相对而言似乎显得“无足轻重”。在大一统史观的教育下,许多对土木之变耳熟能详的粤人根本不知道当时的广东也在进行一场惨烈的战争——黄萧养之乱。这场战争历时一年余,波及几乎整个珠三角地区,导致了数万人的死亡。在这场战争之后,珠三角的社会结构发生了深刻的变化。

为什么会爆发这样一场战争呢?在讨论黄萧养之乱前,我们先来看看当时明帝国的情形。明初,洪武社会主义体制试图对全社会进行极权主义控制。此种体制在20世纪尚且难以维持,遑论14—15世纪。明朝开国不到一百年,该体制便走向崩溃,脱离里甲人民公社的流寇遍地蜂起,较著名者有广东黄萧养之乱、福建邓茂七之乱、郧阳刘千斤李胡子之乱、中原刘六刘七之乱。其中,黄萧养之乱爆发于1448九月,比土木之变早十一个月。

黄萧养是广州府南海县冲鹤堡农民,生得面貌丑陋,还是个独眼龙,性情聪明狡诈。他曾因犯盗贼罪被抓入广州监狱。在狱中,一名同室内监禁的江右人教会了黄萧养“藏利斧饭桶中”的越狱方法\footnote{黄瑜:《双槐岁钞》卷7,页125}。1448年九月,黄萧养在狱外同党接应下与数百名囚犯同时越狱,逃至城外后大肆招揽流民,短短一个月内即纠集了上万人,攻占桂洲、逢简、大良、马齐等地(上述地区相当于今顺德),随之自称“顺民天王”,年号“东阳”,发动反对明帝国的流寇战争\footnote{科大卫:《皇帝与祖宗:华南的国家与宗族》,页96}。贼军凡经过一地,皆胁迫百姓随之造反,不从者皆惨遭贼军屠杀,致使贼军人数迅速恶性膨胀。由他们的凶残暴行来看,与历史上的南粤起义领袖完全不同,绝非反抗明帝国的英雄,而是南粤社会内部的败类。1449年六月,黄萧养率十余万人、船千余艘围攻广州,并声言欲攻佛山。贼军首先击败由粤西高州来援的明军,接着开始围城。在贼军围困下,广州城内粮食断绝,出现了“城中饥死者如叠”的惨剧\footnote{黄瑜:《双槐岁钞》卷7,页125}。于此同时,紧邻广州的佛山亦危在旦夕\footnote{执经生:《南粤土豪的胜利:1449·佛山血战》}。

在珠三角,甚至在整个明帝国,佛山都是一座非常特殊的城市:它并非府治或县治,而是一座自发形成的市镇。佛山位于广州西南16公里处,沿西、北江南下、东下的船只必先经过佛山,方可入珠江、进广州。早在宋帝国治下,佛山即因优厚的地理位置发展成商业聚落。元帝国治粤时,佛山已是一片“骈肩累迹,里巷壅塞”的热闹景象。至元末,佛山本地已产生霍氏、冼氏、陈氏、梁氏、李氏、伦氏、赵氏等豪族。

明帝国侵占南粤后,开始编户籍、立里甲,佛山亦难逃此劫。佛山在行政区划上被定为广州府南海县季华乡,其884户居民共被编成8个图(图即里的别称)、80个甲。然而,明帝国并未过度撼动佛山原有的社会秩序,当地土豪往往举族进入同一甲乃至同一里,里甲、乡村的头面人物“乡老”、“乡判”亦由土豪耆老担任。可以说,在洪武社会主义的冲击下,佛山土豪社会仅仅换上了一套里甲的外衣,其原有社会结构并未大变。事实上,除佛山外,当时的南粤仍有许多土豪采用这种方式进入里甲。他们成为洪武社会主义下南粤仅存的社会凝结核,保留了南粤最后的政治共同体。

由于佛山并非府州县治所,因此没有地方官,城市事务由土豪耆老在城中的“祖庙”聚会商议。“祖庙”是一座同时祭祀道教神袛真武与佛教神袛观音的庙宇,又被称为“北帝庙”、“龙翥祠”,乃佛山的祭祀中心。元末战乱时,曾有海寇逼近佛山,“乡人祷于神”,瞬间风雨大作,贼船倾覆过半。后来海贼贿赂守庙僧人以秽物污染祖庙神像,终于攻破佛山,纵火焚毁祖庙。由此可见,祖庙在佛山人心目中是与保境安民联系在一起的。

1372,佛山乡老赵仲修重建祖庙。1429、1436年,乡老梁文慧、乡判霍佛儿又两次扩建祖庙,凿莲花池,植菠萝、梧桐树以壮其观瞻。佛山土豪之所以有修建祖庙的财力,与其对冶铁业的经营关系关系颇深。自15世纪初起,佛山即形成了大批冶铁点,生产铁锅、农具、钟鼎和军器。佛山铁锅远销吴越、湘赣,每年都有岭北客商携数十万两巨资来佛贸易。此外,佛山生产的火器亦性能良好、精准度高——在后来的佛山保卫战中,这些火器发挥了巨大作用。明帝国治粤时期,冼氏、霍氏、李氏、陈氏皆为冶铁大户,拥有庞大产业。明末有人称“佛山地接省会,向来二三巨族为愚民率,其货利惟铸铁而已”,此即当时佛山的真实写照。

听闻贼军欲攻佛山,佛山土豪二十二人(人称“二十二老”)群聚祖庙商议对策,决定组织民兵抵抗到底。二十二人推举他们中的冶铁大户冼灏通为“乡长”指挥防御。其余二十一人中,最年长的梁广(74岁)平日“处事公平,乡里信服”,在城中威望颇高,另外二十人亦都是“家颇富饶”的“大家巨室”。佛山是一座自发形成的市镇,并无城墙,但流过城北的汾水与环绕城西、南、东的佛山涌为佛山提供了天然护城河。二十二老倾尽家财,组织佛山居民“树木栅,浚沟堑”,日夜打造冷热兵器,迅速建起了一道周长十余里的环形木栅。同时,土豪们各“聚其乡人子弟,自相团结”,编成一个个名为“铺”的战斗单位,“沿栅置铺,凡三十有五。每铺立长一人,统三百余众。”由此推算,佛山民兵仅有一万余人,兵力远远少于贼军。

1449年八月,黄萧养屡攻广州不下,人困马乏,“闻富户多聚于佛山,欲掠之”,乃遣部将彭文俊率数万人攻打佛山。真正的危机到来了。听闻贼军逼近,冼灏通召集所有父老子弟于祖庙誓师,发表了激动人心的演讲:

\begin{quote}

灏通不才,谬辱上命,为若辈保妻子。念今日之事,国事也。分以死图报,不顾私家矣!若辈宜协心力以保厥家,有异心者杀无赦,战阵无勇者杀无赦!人各食其粮,卒有急,灏通愿罄储共食,若辈恭命无忽!

\end{quote}

随后,二十二老一齐在真武像前宣誓:

\begin{quote}

苟有临敌退缩,怀二心者,神必殛之!

\end{quote}

观冼灏通的演讲,虽有论及忠于帝国之处,但更多的则是号召佛山民众保卫他们的家人妻小。对有共同体中的民众而言,保卫家乡、保卫家人、保卫共同体远远比保卫一个帝国重要。冼灏通的演讲因之产生奇效,“军声大振,士气百倍”。

不久,贼军乘数百艘船至,意图破城之后大肆掳掠。佛山血战开始了。贼军包围佛山后,开始了不分昼夜的四面环攻。佛山民兵各铺“首尾联络,互相应援”,依托木栅展开了壮烈的防守战。二十二老多与他们的家人子弟一起站立在木栅之后,一同作战。冼灏通组织锻造了大批可发射大如碗的石弹的大火铳,安放于木栅各处,给贼军造成了惨重的伤亡;其次子冼靖则率乡族子弟溶铁为液,供守军泼向贼军。梁裔坚(二十二老之一)率其诸弟“悉以家货供乡兵食”,其年仅十八岁的末弟梁颛“状貌雄伟”,曾率乡人开栅出击,“持丈二红刃刺贼先锋,大呼陷阵”,击退贼军的猛攻。梁敬亲(二十二老之一)“与诸义士树栅拒之,谋定而后战,呃吭捣虚,所向必克”。中秋之夜,贼军本欲发动夜袭,然梁俊浩(二十二老之一)早已令各铺少年举彩旗、鸣金鼓游行,并燃放大爆竹。贼军见之,以外城中有备,遂不敢攻。每次贼军攻城前,二十二老必聚于祖庙并祷于神,神许出战则开栅出战,神不许则在栅后防守。作为佛山社会的凝结核,祖庙和二十二老通过这种方式加强了民兵的信心和士气。

贼军连日攻城,除遗尸累累外一无所获,遂大造云梯,展开新的攻势。因云梯笨重,持云梯的贼兵往往收阻于栅前壕沟,民兵遂投火炬焚毁云梯。贼军主将彭文俊又集中兵力猛攻木栅南侧的栅下一带。防守该地的霍佛儿(二十二老之一)、霍仲儒、冼光(二十二老之一)率民兵极力防战,战况极为惨烈。霍氏一族父子兄弟并肩作战,并“撤屋为栅,浚田为涌”以加固防御工事,耆民首领霍仲儒亦殉于阵中。激战中,冼光“开栅门出战”,阵斩贼军主将彭文俊。贼兵大怒,攻城益急,结果被民兵用“飞枪巨铳”击退。

无计可施的贼军派使者李某入城劝降,结果被冼灏通三子冼易拔剑斩杀。贼兵只得“退兵二里许,联舟为营”,开始长期围困佛山,意图坐等佛山粮尽、不攻自破。围困期间,幸亏城中“大家巨室藏蓄颇厚,各出粮饷资给”,全城兵民“皆饱食无虑”,士气高昂。贼兵曾“有自恃勇悍、翘足向栅谩骂者”,被民兵以火铳一发击毙——佛山兵民惊奇之余,多认为此乃真武保佑\footnote{执经生:《南粤土豪的胜利:1449·佛山血战》。关于佛山之战详情,还可参看郑广南:《中国海盗史》,页171—172}。

贼军对佛山的围困战持续了半年。在此期间,广州战局发生了重大变化。1450年初,明代宗任命杨信民为广东巡抚,率兵解广州之围。杨信民突入广州城后,遣使召黄萧养谈判。黄萧养应其召,杨信民乃单独出城赴约,两人隔护城河对话。其时,久攻广州、佛山不下的黄萧养已难以为继,遂在谈判中答应投降,并送大鱼一条予杨信民\footnote{科大卫:《皇帝与祖宗:华南的国家与宗族》,页96}。当年三月,正当黄萧养在准备投降事宜时,明都督董兴突率江西、两广援兵杀到广州城下。值此关头,杨信民突然病死,双方合约乃告破裂。四月十七日,董兴率舰队对贼军发起全面进攻,双方舰队在广州沙面、黄沙、洲咀头、芳村之间的白鹅潭江面展开决战。战斗中,黄萧养中箭身亡,贼军全军崩溃、死者万余\footnote{关于此战经过,见郑广南:《中国海盗史》,页172;罗一星:《明清佛山经济发展与社会变迁》}。听闻黄萧养的死讯,围困佛山的贼军一夕溃散。明军随之进驻佛山,以冼靖(二十二老之一)为“乡义”,命其协助明军攻剿贼军余部。董兴是个极度残暴的帝国军官。击败黄萧养后,他率军进入曾被贼军控制的区域,对曾被裹挟入贼的百姓展开了报复性的大屠杀。由于贼军曾编写过胁从者名单,官军得到名单后即按名单上的“姓名乡里”展开不分男女老幼的挨户搜杀,“遂滥及不辜,并乡之民,多横罹锋镝者”。曾惨遭流寇蹂躏的珠三角百姓,至此又横遭明帝国侵略军屠戮。在随军进剿的过程中,冼靖甄别“良莠”、不问胁从,阻止了官军的大量屠杀行为,“存活者百数千人”,充分体现了南粤土豪的政治德性\footnote{执经生:《南粤土豪的胜利:1449·佛山血战》}。

在持续半年的战斗中,佛山民兵杀伤数千贼军,击毙贼将彭文俊、梁升、李观奴,生擒张嘉积。更重要的是,他们以区区万余民兵拖住了数万贼军,极大减轻了广州的压力,为明军对黄萧养的最后一击的成功创造了条件。1452年,明廷叙功,明代宗敕封佛山为“忠义乡”,敕封祖庙为“灵应祠”,并命广州官员春秋致祭;佛山二十二老亦获赐“忠义士”称号。

在黄萧养之乱中,除佛山土豪外,珠三角一带还有龙江、北村、沙头、龙山、九江、大同等地的土豪结寨自保并击退小股贼军的进犯\footnote{科大卫:《皇帝与祖宗:华南的国家与宗族》,页96}。如果说这场战争中有谁是正义者的话,那便是这些保境安民的南粤土豪。除了最后的白鹅潭决战外,明军在战争中几乎没有做出任何贡献。平定黄萧养之乱的光荣属于我们勇武的祖先,而绝不属于残暴的明帝国。事实上,明帝国侵略军在平乱之后对南粤百姓犯下的战争罪行绝不亚于贼军。明帝国与黄萧养的争斗实为一场狗咬狗的肮脏战争,南粤百姓是其直接受害者。

黄萧养之乱给南粤带来了巨大的创伤与灾难,亦极大地改变了南粤的社会格局。明帝国统治者认识到了这样一个事实:在洪武社会主义崩溃后,他们已无力维持对南粤的有效统治。当流寇出现时,他们只能无奈地依靠那些在明初浩劫中幸存下来的南粤土豪。1451年,明廷将曾为贼军核心控制区的地域从南海县划出,设为顺德县。这一行政规划并非出自明廷本意,而是来自大良土豪罗氏家族对明廷的“请求”。在递交给明政府的呈文中,罗氏称其设县目的为将当地人“治以官司,联以户口,齐以科教”。事实上,这一切不过是罗氏的托词。罗氏的真实目的,乃是将其乡土共同体“合法化”,逼迫明帝国承认其自治权。明帝国难以与罗氏的强大势力抗衡,只得答应其“请求”,并将县治设在被罗氏家族控制的大良镇。此后,珠三角的大批土豪纷纷以这种方式向明帝国“效忠”,在名义上被重新编入里甲。既然土豪已经“答应”进入里甲体制了,那么明军便再无任何理由进攻他们。而事实上,里甲只是南粤土豪自治的外衣,虚弱的明帝国对其无可奈何。就这样,珠三角土豪在15世纪中期与明帝国达成了微妙的权力平衡。他们通过英勇的战斗、高超的政治智慧为南粤夺回了相当一部分自由,彻底终结了南粤的洪武社会主义体制,使南粤起死回生。

珠三角土豪与明帝国达成了权力平衡,但珠三角以外的人们暂时还没有。与黄萧养之乱几乎同时,广西东北部也在进行着一场惨烈的战争。这场战争,便是英勇绝伦的大藤峡起义。

\section{蛮族武士的抗争:侯大苟与大藤峡起义}

在广西桂平西北8公里处的黔江(西江支流)下游,坐落着风景绝美、地形险要的山谷大藤峡。大藤峡两侧是峻峭的绝壁、险峻的山石,峡中水流湍急、暗礁密布。在15世纪,峡谷中最令人惊奇的景观是一处巨大的藤条。对这一奇景,史书这样描述:

\begin{quote}
	
峡中有大藤如斗,延亘两崖,势如徒杠。蛮从议度,号大藤峡。峡最险恶,地亦最高,登藤峡岭,数百里皆历历在目\footnote{(民国)《桂平县志》}。
\end{quote}

根据这段记载,大藤横跨江面,连接两崖,可谓异常壮观,大藤峡之得名也由此而来。大藤已不知存在了多少年,它的寿命也许与南粤的古老自由一样长久。然而,今天的我们却已看不到那条大藤了。大藤是如何消失的?它消失的背后有着怎样的故事?欲回答这些问题,我们就必须了解那场发生在15世纪中期的伟大起义、认识南粤瑶民的英雄侯大苟。

早在1395年,大藤峡的瑶民就曾参加过反明起义。起义军虽然遭到了明帝国极度残酷的镇压,但其余部仍坚持战斗到1420年代。为了断绝广西瑶人再次反抗的可能,明帝国不但禁止铁器、大米、食盐、布匹流入瑶区,还于1441年调集大批军队进入大藤峡地区屯垦、肆意霸占瑶人的田土\footnote{《中国少数民族自治地方概况书 第49卷》,页53}。面对侵略者的暴行,大藤峡瑶人的怒火被点燃了。他们已在nï 美丽的山谷中度过了数千年的自由生活,怎会忍受侵略者肆意践踏他们的尊严?1442年,大藤峡瑶人在其首领蓝受贰的率领下发动起义。他们主动联络各地瑶民,四处袭击侵略军、杀死明帝国官吏\footnote{《中国少数民族自治地方概况书 第49卷》,页53}。明广西总兵柳溥见起义军势大,不敢正面对敌,乃想出一条无耻的毒计。他指示部下千户潘智出面佯装与蓝受贰谈判,又暗中设下伏兵,卑鄙地杀害了蓝受贰、覃公崇等十余位中计的起义军将领\footnote{高言弘、姚舜安:《明代广西农民起义史》,页27}。然而,丧失了首领的起义军并未瓦解,他们认清了明帝国的卑鄙嘴脸,追随蓝受贰的部下侯大苟继续坚持战斗。

侯大苟是大藤峡田头村人,以贩蛤蚧为生。由于他平时处事公道、乐于助人,因此深受家乡父老的敬爱。蓝受贰遇害后,起义军在他的指挥下日益发展壮大。到1446年,起义军已有步、骑、水三军共万余人,控制柳州、浔州、梧州三府十余县,使广西东北部完全恢复了自由。此后近二十年间,起义军频繁出击,多次攻克梧州、北流、藤县等地,并曾深入广东境内进攻肇庆、罗定、化州等地,部分义军甚至还进入过湖湘。1463年,起义军更一度进攻新会县城、攻陷粤北清远县城,逼近到距广州仅百余里之处\footnote{高言弘、姚舜安:《明代广西农民起义史》,页32}。应当指出,大藤峡起义军在进入较为陌生的广东境内后,因为补给困难,曾犯下不少劫掠民众的战争罪行。然而时人的记载指出,粤北百姓畏惧明帝国“官军”更甚于畏惧瑶人\footnote{科大卫:《皇帝与祖宗:华南的国家与宗族》,页111}。可见,明帝国侵略军的纪律是远不如起义军的,他们犯下的战争罪行显然更大。

大藤峡起义军的凌厉攻势令明英宗坐立不安。1464,明英宗下令“有能捕侯大苟者予千金、爵一级”,却无人响应\footnote{金鉷:(雍正)《广西通志》卷95}。同年,明英宗病死,其子宪宗继位。明宪宗被后世史家视为暗弱之主,实则极度残忍好杀。甫一上台,他就命左佥都御史韩雍统“南北二京、江西、湖广达(蒙古)汉官兵”十六万人进攻大藤峡,意图以此庞大的兵力一举屠尽起义军。韩雍是个丧心病狂的帝国走狗,他十分崇拜那个曾蒸食广西义军人头的韩观,非常想建立像他那样的“赫赫武功”。1465十月,韩雍率军攻破修仁(今广西荔浦修仁镇),屠杀起义军民7300余人。十一月,侵略军进至峡口,韩雍下令斩杀当地儒生、耆老数十人,将他们剖腹抽肠、分解尸体、悬挂林中,意图制造血腥景象以恐吓起义军\footnote{高言弘、姚舜安:《明代广西农民起义史》,页35}。值此关头,侯大苟临危不乱。如骑士般高贵的他将妇孺疏散到峡外,自己则留下来与部下同生共死。十二月一日,十六万明军齐集于大藤峡南北两侧,展开总攻。起义军以峡中大藤为交通线往来其上,拼死防御。韩雍命部下“以大斧刊木开道”、发射火箭,尽焚起义军之营栅\footnote{凌云翼、刘尧诲:《苍梧总督军门志》卷18,页196}。侯大苟率780余名将士退至最后的据点九层楼山,以石块、滚木、毒箭多次打退侵略军进攻,与敌激战竟夜。十二月二日白天,侵略军从后山抄小路登上山顶,突入起义军阵地。侯大苟及780余名将士无一降者,全部壮烈战死。在保卫大藤峡的战斗中,阵亡的起义军达4000人之多\footnote{高言弘、姚舜安:《明代广西农民起义史》,页37}。

攻陷大藤峡后,韩雍指挥侵略军对当地未及疏散的平民展开了灭绝人性的大屠杀。在韩雍于事后所作诗中,满是“积尸如山血如川”、“诛锄只许留襁褓”这一类血淋淋的句子\footnote{高言弘、姚舜安:《明代广西农民起义史》,页37}。至于那些幼童实际上也没有逃脱侵略军的毒手,他们中的许多人惨遭阉割,被送往北京宫中成为宦官,明中叶著名的权宦“西厂厂公”汪直就是他们中的一员。此外,侵略军还砍断了大藤,将峡谷改名为“断藤峡”,以示对南粤的羞辱\footnote{凌云翼、刘尧诲:《苍梧总督军门志》卷18,页196}。然而,英勇的南粤瑶人是不会屈服的。在侵略军主力撤走后,他们中的幸存者重建了大藤峡中的村落,不断发起抗争。直至16世纪晚期,当地仍有瑶人发动反明起义\footnote{龙小峰:《明代大藤峡之役屡开鲜效的环境因素分析》,《史学月刊》2013年第12期}。

在帝国文人的笔下,粤人总是被描绘成“不服教化”的南蛮。然而,侯大苟疏散妇孺,率男子拼死一战的行为无疑可比最为高贵的西欧骑士。相形之下,屠杀无辜、阉割幼童的明帝国侵略军则是一群禽兽不如的刽子手。大藤峡起义虽然失败了,但那些蛮族武士的精神永远不会消失。他们在历史上书写了南粤近代前夜最为壮烈的一笔,鼓舞南粤人为自由和尊严而战。

\section[南粤小华夏的种子播撒者:陈白沙]{南粤小华夏的种子播撒者:陈白沙\protect\daggerfootnote{本节在Japgenksank:《南粤小华夏的种子播撒者:陈白沙》一文基础上小有改动}}

	

在近代前夜,伪装成里甲的土豪自治共同体与大藤峡起义军代表着南粤的两种路径。一种是不再主动反抗岭北帝国并与之结成机会主义同盟,从而较稳定地保全南粤的部分自由;另一种是彻底排斥华夏文明,完全依靠数万年来的百越习惯法生活,对岭北帝国展开永不止息的武装斗争。前一种路径下,大批粤人的生命固然能在短期内得到保全、许多自治共同体亦能和平地维系下去,但终究失之机会主义,容易使我们南粤人丧失斗志与信仰,最后被岭北帝国潜移默化地彻底同化;后一种路径下,与华夏文明永远对立固然可敬,但不讲策略地拼命出击终究会使粤人承受不必要的牺牲,甚至有可能导致岭北帝国在未来像西班牙人消灭印第安人一样将粤人杀戮殆尽。在此微妙关头,南粤似乎已陷入了困局。就在这时,一位伟大的南粤儒者横空出世,将两种路径结合起来,吹向了发明南粤民族的号角。他的名字,叫做陈献章。

陈献章,字公甫,号石斋,又被人称为“白沙先生”,后世学者因而多呼他为“陈白沙”。陈白沙出生在15世纪南粤新会的一个汉人家庭。那时的新会,与四百年后的梁启超时代决然不同,更类似清代民国汉夷杂居的湘西。当时,在地处珠三角边缘的新会,汉人与蛮族的区别往往只在于是否被帝国编户,双方并无严重的文化隔阂。如果将陈白沙所属的族群与中原流寇划为一族,无疑会得出扭曲的历史图景。对当时的北京朝廷来说,广东仅是一个缺乏重要性的边陲省份。经过洪武社会主义的洗礼,从南越国时代持续到元代的广州国际贸易已变得可有可无。除肥沃的珠三角及西江、北江、东江边的一些帝国神经节点(府城、县城)外,整个南粤都被北京视同化外。

陈白沙是一名遗腹子,幼年由母亲林氏抚养成人。幼年的他身体虚弱、无岁不病,竟直到九岁才断母乳。十岁时,他们母子随白沙的祖父由都会村迁居白沙村。在聚族而居、以男性亲属关系为骨架的南粤乡村中,外来的孤儿寡母不仅很可能遭村人排斥,亦难以在共同体中获得较高的社会地位。身体的孱弱令白沙不可能从事繁重的工、商、农事务,但一次次静养大概给了他充足的时间用来思考。他略通诗赋的父亲留下的一笔不大不小的家业,又正好能使他成为一名脱产读书人。在此种种基础上,让白沙参加科举并入仕、成为帝国官僚,从而让整个家族获得稳定的社会地位,遂成为白沙的长辈们最自然的选择。

此种士大夫型的人生规划,无疑与当时“蛮荒”的南粤社会格格不入,一个汲汲于读书、考试并担任帝国公务员的人无疑很难与南粤社会产生有机联系。白沙虽是一个粤人,当时的他与北京的心理距离却比他离家乡更近。身体条件的限制令白沙能够将主要经历放在“寒窗苦读”上。事实证明,经过多年的学习,他应付科举这种智力游戏的能力并不差。1447年,年仅二十岁的他通过乡试,以广东第九名的身份成为举人。对于当时无数皓首穷经亦无法获得功名的老书生来说,白沙无疑是一极令人羡慕的天才,白沙自己亦对高中进士颇有期待。然而,在次年的会试中,他却意外失利,仅入副榜。据明帝国规定,会试副榜要么入国子监读书、重新备考;要么进入某地的学校,当一名教官。作为一个自我期许颇高的人,白沙当然不会让自己在“地方上”终老一生。然而,在三年后的下一次会试中,他竟再度落第。连续不中令他感到灰心,他遂南归家乡。

在白沙北去的三年里,南粤经历了黄萧养之乱的浩劫。1449年,新会人黄三、温观彩联手举兵,拥众数万,响应黄萧养。次年,黄三被招安,温观彩被讨平,明帝国侵略军更在新会凶残地肆意杀掠,“擒捕贼徒,尽数剿戮,捉获人、畜无算”,犯下了滔天罪行。劫后荒凉惨淡的家乡,想必令刚刚度过了三年都市生活的白沙感到悲凉与不适应。而科举的接连落第,更令他感到由衷烦闷。1454年,二十七岁的白沙收拾独自北上江西崇仁,师从当时名动天下的大儒吴与弼(号康斋)。

十五世纪,东亚思想史朱子学与阳明心学间的暧昧期。宋儒回向三代的理想已如肥皂泡般幻灭,经洪武社会主义洗刷过的大地一片惨白。口称尊朱的明初儒者与其说仍是程朱的门徒,不如说已转向禅宗式的内省。由于对外在世界的探索、对三代之治的追求已在高压下失去意义,儒者只得退缩到内心这一最后的自我阵地,并力图通过身体力行尽量将身边的师友家人团结为一个道德高尚的小团体,精通《易》学康斋即是如此的典型人物。康斋曾两辞明廷招用,汲汲于在家乡的田野间讲学授徒、建设儒学小共同体。其讲学方式亦注重实践,不尚空谈。对于他的教学场景,史籍如是记载:

\begin{quote}

(康斋)雨中被蓑笠,负耒耜,与诸生并耕,谈乾坤及坎离艮震兑巽,于所耕之耒耜可见。归则解犁,饭粝蔬豆共食。陈白沙自广东来学,晨光才辨,先生手自簸谷。白沙未起,先生大声曰:“秀才!若为慵懒,即他日何从到伊川(程颐)门下,又何从到孟子门下?\footnote{黄宗羲:《明儒学案》卷1《崇仁学案一》}”

\end{quote}

吴康斋的形象,全然一副耕读传家的老儒模样。据说,他因日夜思念孔子,以至曾在梦中与孔子相会。万分欣喜的他将这一美梦告知老妻,得到的却是无情的嘲讽。追求意义者眼中所看到的奇迹,在失去意义的末人眼中不过是虚无。康斋此种质朴的追求在其妻看来好似神经分裂,却无疑给年轻的白沙带来了极大震撼。在白沙此后的文字中,我们看不到康斋在《易》学方面对他有多大影响。但康斋那种勤奋耕耘、授课,凭儒者的坚定信仰笃实地追寻自我意义的态度,肯定给白沙带来了极大的鼓舞;而康斋汲汲于建立共同体的态度,无疑给从小缺失共同体生活的白沙非比寻常的震撼与向往。数个月后,白沙回到新会。此时的他已全然脱胎换骨。此后十二年间,他在新会小庐山下闭关读书,穷尽古今典籍、佛老经典,乃至稗官野史。巨量的阅读使他培养出了自己的格局感。他体认到了“学贵自得”这一宗旨,提出典籍之言需与自我的体悟相合方有意义,否则便皆是无用知识的堆砌,“典籍自典籍,而我自我。”

到1466年,三十九岁的白沙已经变得十分强大。也许是为了震撼一下那些缺乏格局感、只知卖弄无用知识的酸文人,他重游北京太学。他提出的学者不应注重词章,而应遵从古圣贤的义利之辨专注于修养德性的主张,得到了祭酒邢让的极大赏识,赞之为真儒复出,亦震动了北京的官僚士人圈子。甚至连李东阳这样的文学大家,亦对白沙的学说表示惊叹。官任给事中的辽东人贺钦更奉他为精神导师,辞官并向这位来自南粤的儒者行跪拜礼。此后,两人虽一在南粤、一在辽东,却终身通信,保持着师徒关系。1469年,白沙第三次参加会试,又告落第,遂再一次南归家乡。在北京的官僚士人给他写的赠别诗中,满是希望他以真儒学救世的期许。同科落第的南粤东莞人林光干脆陪他同归新会,拜他为师、随侍左右。

四十一岁的白沙在回乡之时,已是个名满天下的知名人士。他一面照料年迈的母亲,一面招收门徒、传播学说。大批南粤士子倾慕他的学说,纷纷来到白沙村拜他为师。甚至连新会知县丁积都拜入他门下,向他请教本县的施政方针。经白沙的指点后,丁积开始推行一项庞大的乡礼共同体建设计划。他以平实易懂的文字撰写了《礼式》一书,写明儒家冠婚丧祭之礼的详细程序。该书以《朱子家礼》为宗,实为一本实用性极高的简化版《朱子家礼》使用指南。黄萧养之乱后残破的新会土地上,出现了一个个祸福相忧、居处相乐的儒家共同体。此种共同体使宋儒重建社会、回向三代的理想在新会开始变为现实。它们虽然由百越的苗裔们建造,却是最为正统的华夏文明的嫡传。在华夏文明断绝于中原长达千余年之后,这一文明在十五世纪的南粤开始复活了。

担任新会县丞的广西郁林人陶鲁,则成了白沙的挚友。也许是与新会本地人同为粤人的缘故,陶鲁虽为帝国官员,却与新会父老形成了密切联系,在当地威望极高。1463年,从广西出发的大藤峡起义军进军粤西廉州、雷州、高州等地,攻入肇庆府,直逼新会,一路劫掠不已。危难关头,陶鲁召集当地新会父老,号召他们率乡党子弟“死守城邑、保家族”,得到极其热烈的回应,最终打赢了一场壮烈的守城战,复制了佛山土豪在十四年前对黄萧养流贼的胜利。这场战斗,实为南粤的两种历史路径之间一次激烈的对撞。对陶鲁这位保卫乡土的土豪化官员,白沙推崇备至,他在返乡当年(1469)便与陶鲁密切合作,在宋末崖山古战场附近主持修建了至今尚存的崖山祠,祭祀文天祥、陆秀夫、张世杰三位南宋忠臣以及殉国的宋末帝昺与杨太后。这一举措显著提升了新会人对自身的自豪感,他们中的许多人开始骄傲地声称,自己的家族本是宋帝扈从或南宋宗室,当年随宋廷流亡至此,代表华夏文明进行了抵抗野蛮人的最后战斗。在北方已被流寇与蛮族搞得一团糟时,只有他们还在南粤这片热土上坚守着仅余一脉的华夏文明,做华夏最后的孤忠。这一祖先世系叙述模式,是十六世纪普及于南粤的南雄珠玑巷祖先神话的先声。

1482年,在明广东巡抚朱英(白沙的仰慕者之一,两人虽未曾谋面,但已通信十余年)、布政使彭韶(白沙在北京时所交之友)的交章荐举下,明宪宗“下诏”启用白沙。在此之前,已有许多官员、友人劝白沙出仕,但皆被白沙推辞。可对于一名虔诚的儒者来说,长时间不出仕无疑会在其内心深处制造剧烈的道德危机感。在此压力下,白沙不得不离开家乡、学生、朋友与母亲,北上赴召。赴京途中,白沙一直犹豫着是否应该出仕,一路拜访老友,走走停停。途径江西时,他还拜祭了恩师吴康斋的墓。1483年三月,在经过七个半月的漫长旅程后,白沙终于到达北京。然而,等待他的确是命他先赴吏部考试、再对其量授官职的圣旨。朝廷敷衍的态度令白沙坚定了决心,他随即以母亲年迈为由“疏乞终养”,最后得“授翰林院检讨而归。”据白沙晚年的重要弟子张诩回忆,白沙之所以匆匆归乡,与质疑白沙学说的国子监祭酒、海南人丘濬的排挤有关。若此情况属实,则受到同为粤人的高官算计的白沙想必会对此感到厌烦——在帝国宫廷之中,就算同为粤人者亦不免互相倾轧。如此浊乱的宫廷,又有什么可留恋的呢?

在这之后,直到1500年以七十三岁高龄去世时,白沙再未离开南粤一步。他断绝了出仕的念头,与岭北帝国斩断了联系,致力于建设南粤的儒学共同体。1494年,他因林光在母亲病危的情况下仍决意赴山东做官而极感愤怒,毅然将这位跟随了他二十五年的弟子逐出师门。白沙晚年最信任的学生为番禺人张诩(号东所)和增城人湛若水(号甘泉)。在白沙身后,东所在学术上继承了他的心学格局,专注于自我修行。甘泉则与佛山人霍韬、南海人方献夫在西樵山中建设书院、教育子弟、往来讲学,一同建成了一座“理学名山”。1520年代,随着霍、方二人在“大礼议”中获得胜利,南粤土豪趁机掀起了波澜壮阔的社会改造运动,纷纷将其乡党改造为儒化宗族,以帝国意识形态塑造挖帝国墙角的自治共同体。理学士大夫们继承着白沙的事业,深入南粤的山林、乡村,兢兢业业地建设着宗族、乡约,一如8世纪不列颠群岛洪荒中的传教士。与此同时,葡萄牙人正好到达南粤。他们复活了南粤的外贸,为许多宗族的构建提供了充足的资金。在白沙身后150年,南粤宗族化宣告完成,南雄珠玑巷的民族共同祖先神话构建成功。到白沙身后400年时,潮汕、客家亦已完成了类似的历史过程,共同鄙视岭北的南粤“小华夏”诞生,他是百越的直系后裔,也是华夏的波兰;他是一个有数千年历史的古老共同体,也是一个只有数百年历史的崭新民族。随着大航海时代的到来,这个新民族的春天开始了。无数武德充沛、生命力旺盛的儒化小共同体在南粤大地上星罗棋布,向着无尽的海洋扬帆、启航——建设这种小共同体,正是白沙的理想。其后,无论经历怎样的风霜、怎样的摧残,木棉盛开的粤土上依然挺立着一座座祠堂——那是南粤社会深处的凝结核,那是粤人永远的根。

至于这累累硕果的播种者陈白沙,如今正在新会的白沙祠中继续守护着南粤。作为一名播种者,他将种子精心种入土壤后,飘然而逝。他虽无法预料到自己会创造出怎样的奇迹,但他对种子精心的呵护和培育,使这样的奇迹成为必然出现的东西。他留下的坚实的种子已深深植入南粤的土地,使南粤必能度过洪水的冲刷、风雪的肆虐,迎来春日的朝阳。三万多年来,百越先民、公师隅、南越武帝、二征、士燮、赵妪、吕兴、郭马、李贲、冼夫人、冯宝、冯盎、陈行范、冯璘、刘隐、刘谦、南汉高祖、周思琼、麦汉琼、李接、熊飞、马发、朱光卿、何真、也儿吉尼、苏友兴、梁道明、施进卿、侯大苟等人的英魂从未逝去、所有为南粤的自由与尊严而战的士的英魂从未逝去,他们一直活在粤人心中,早已和南粤的文明土壤融为一体。树大者根必深,白沙播撒下的种子必然结出丰硕的果实。几个世纪后,这种子果然长出了无数的草木花果。这一切波澜壮阔的历史进程,都是本书下编将要展示的内容。在本书下编中,我们将见证奇迹。

只要种子不死,无虑花果凋零。








