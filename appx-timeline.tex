
\chapter{南粤编年简史}

\section*{史前时代与第一次北属}

公元前333年 楚国杀越王无疆,无疆子之侯避处南粤,自立为王。

公元前312年 之侯遣公师隅出使魏国,以图牵制楚国。

公元前257年 古蜀末代王子泮率古蜀遗民流落至今越南北部,自称安阳王,定都古螺,建瓯雒国。

公元前218年 秦始皇遣尉屠睢、赵佗率军入侵Namyeud,西瓯君译吁宋率众反抗秦军,壮烈战死。

公元前217—215年 秦军在南粤遭遇越人的激烈抵抗,“三年不解甲弛弩”。屠睢阵亡,秦军被击毙数十万人。

公元前214年 秦始皇令任嚣、赵佗统五十万谪徙民入南粤,“与越杂处”,置桂林、南海、象郡。

公元前207年 赵佗率军攻灭瓯雒国,安阳王投海殉国。

公元前206年 秦朝崩溃。秦龙川令赵佗据南海,下令封闭五岭各关口,攻占桂林、象郡。

\section*{南越国时代}

公元前203年 赵佗自称南越武王,“和辑百越”。

公元前196年 武王对汉高祖称臣。

公元前183年 武王因汉吕后采取“别异蛮夷”政策、禁“南越关市铁器”,自立为南越武帝,发兵破长沙国数邑。吕后遣隆虑侯周灶率兵攻击南越,汉越两军相持于五岭。

公元前179年 吕后死,汉军北撤。武帝向汉文帝称臣,仍于国中称帝。

公元前137年 南越武帝崩,其孙赵眜继位,是为南越文帝。

公元前135年 闽越军入侵南越,文帝向汉武帝求援。闽越发生政变,停止入侵。文帝遣太子赵婴齐入长安“宿卫”。

公元前122年 文帝崩,太子赵婴齐归国继位,去帝号,是为南越明王。明王立北人樛氏为王后,立樛氏子赵兴为世子。

公元前115年 明王薨,赵兴继位,是为南越哀王。朝政归于太后樛氏之手。

公元前113年 汉武帝遣安国少季使南越,与樛氏达成南越内属、三年一朝、去除边关之协议。

公元前112年 南越丞相、越人吕嘉起兵反汉,诛樛氏、哀王、安国少季,立赵建德(明王长子,越人王妃之子)。汉武帝遣伏波将军路博德统军十万入侵南越。

公元前111年 番禺(广州)陷落,末王赵建德被俘、吕嘉丞相壮烈殉国。南越亡国,历5主,93年。汉以南越国故地置南海、苍梧、九真、郁林、日南、合浦、交趾七郡。

\section*{第二次北属}

公元前110年 汉军渡过琼州海峡,增置儋耳、珠崖两郡。

公元前106年 汉武帝置十三州,其中岭南九郡属交趾。

公元17—23年 新莽末,天下大乱。交趾牧邓让闭关自守。

公元29年 邓让以岭南附东汉。

公元40年 交趾女子征侧、征贰起义,据交交趾、九真、日南、合浦等地65城。

公元42年 光武帝遣伏波将军马援统兵南征交趾。

公元43年 汉军击败交趾起义军,二征壮烈战死。

公元166年 罗马帝国使节到达交趾。

公元187年 苍梧人士燮任交趾太守。岭南九郡中,有七郡被士氏一族控制。

公元210—220年 孙吴以步骘为交州刺史,增筑番禺城。

公元220年 孙吴以吕岱继任交州刺史。

公元226年 士燮去世。吕岱分岭南九郡为两部,以交趾、九真、日南为交州,以其余六郡为广州,自任广州刺史,以戴良为交州刺史。士徽(士燮之子)起兵阻戴良入境。吕岱出兵,屠灭士氏,合并交州、广州。

公元263年 郡吏吕兴起兵据交趾、九真、日南,以三郡归降曹魏。孙吴与曹魏于岭南展开惨烈战争。

公元264年 孙吴再次分离交州、广州。

公元271年 孙吴与曹魏(西晋)争夺交州、广州的八年战争结束,吴军攻占交州。

公元279年 广州人郭马起义,据广州城。孙吴以一万七千军队南下镇压。

公元280年 西晋灭吴,晋军入岭南,继续镇压郭马起义。郭马下落不明。

公元401年 罽宾国(今克什米尔)僧昙摩耶利至广州,立王苑朝延寺(1151年易名光孝寺)。

公元404年 海寇卢循以火攻侵占广州,烧死一万余人。

公元410年 卢循部撤离广州。

公元453年 南朝宋南海太守萧简据广州反,被讨平。

公元466年 南朝宋广州刺史袁昙远据广州反,被讨平。乱兵大掠广州。

公元527年 印度僧人、禅宗始祖达摩至广州,建西来庵(1655年改名华林寺)。

公元535年 南梁罗州刺史冯融以其子冯宝娶高凉俚人大首领冼氏女(冼珍,即冼夫人)为妻,冯冼联姻。

公元537年 沙门昙裕法师(梁武帝萧衍之舅)自扶南(今柬埔寨)求得舍利带至广州,广州刺史萧裕营造宝庄严寺供奉之(1100年改名六榕寺)。

公元544年 交州人李贲起义反抗南梁,据有交州,称帝,改年号为“天德”,是为越南史上之李南帝。五年后,李贲败死。

公元549—552年 侯景之乱爆发。冯宝、冼夫人率南粤兵助陈霸先北伐平乱。

公元557年 梁陈之际,“海隅扰乱”。冼夫人“怀集部落,数州晏然”。

公元570年 陈广州刺史欧阳纥起兵作乱,被冼夫人平定。

\section*{冯冼政权}


公元589年 二月,陈亡,“岭南未有所附,数郡共奉夫人,号为圣母,保障安民”。十一月,冼夫人向隋朝投降。

公元590年 番禺俚帅王仲宣起兵反,被冼夫人平定。冼夫人被封为谯国夫人,开谯国幕府,置长史以下官属、给印信、可调动六州兵马。冼夫人之孙冯盎、冯暄分别被晋封高州刺史、罗州刺史。

公元594年 南海神庙(波罗庙)立于广州东郊。

公元602年 冼夫人去世。

公元618年 隋末大洪水。冯冼氏归附割据江右的楚帝林士弘、合浦豪族宁氏归附梁王萧铣,双方展开南粤内战。

公元620年 冯盎“克平二十余州,地数千里”,占据广州、苍梧、朱崖等地,自号总管。

公元622年 冯盎向唐朝投降。

公元627年 穆罕默德门徒、阿拉伯人艾比*宛葛素于广州建立东亚首座清真寺怀圣寺。同年,唐太宗分天下为十五道,以五岭以南为岭南道。

公元628年 唐太宗向冯盎发出敕文(《前敕》),对其进行威胁。冯盎遣子冯智戴“入朝”。

公元631年 唐太宗第二次向冯盎发出敕文(《后敕》),再次进行威胁。冯盎亲自赴长安“入朝”唐太宗。

公元646年 冯盎去世。

公元676年 六祖慧能在光孝寺剃度。

公元693年 武则天屠杀岭南流人三千四百人。

公元694年 武则天以岭南“獠反”为借口,杀“名士”三十六家。

公元697年 武则天发动“征冯”之役,冯子游(冯宝、冼夫人曾孙)、冯梧(子游之子)、冯君衡(子游之侄)为保卫南粤的自由悉数壮烈战死。武周军入高州城,冯冼氏被“籍没其家,裂冠毁冕”。

\section*{第三次北属}


公元698年 唐岭南讨击使李千里虏获冯君衡之子冯元一,将其作为阉童送入宫中,改名高力士。

公元705年 唐中宗命广州都督周仁轨率兵屠灭合浦豪族宁氏。

公元728年 岭南土豪陈行范、冯璘起义,克四十余城,陈行范自称南越王。唐玄宗遣大军镇压起义,陈、冯壮烈战死,六万余人惨遭唐军屠杀。

公元758年 波斯、阿拉伯(大食)兵围广州,唐广州刺史韦见利弃城而逃。两国兵入城焚掠、浮海而去。

公元862年 唐廷分岭南道为岭南东道、岭南西道。

公元879年 流寇黄巢军一度攻克广州,屠杀阿拉伯、波斯、印度、东南亚“蕃商”十二万人。

公元882年 广州牙将刘谦因截击黄巢军有功,任封州刺史,拥兵万人、战舰百余艘。

公元894年 刘谦去世,其子刘隐继任封州刺史。

公元896年 刘隐讨平叛将卢琚、覃玘,攻取广州、端州(肇庆)。

公元898年 韶州刺史曾衮发兵攻打广州,刘隐讨平之,占领韶州。

公元904年 刘隐因向朱温进献奇药、珠宝,得其欢心,获授清海军节度使。

公元911年 后梁太祖朱温封刘隐为南海王。同年,刘隐去世,其弟刘岩继任清海军节度使。

公元913年 刘岩攻取高州、潮州、邕州,与后梁断交。

\section*{南汉帝国}

公元917年 刘岩于广州称帝,改广州为兴王府,国号“大越”。

公元918年 刘岩改国号为“南汉”,是为南汉高祖。

公元923年 南汉高祖以刘隐之女增城县主嫁予大长和国皇帝郑仁旻。

公元926年 南汉高祖改名刘䶮。

公元928年 封州之战,南汉军击退南楚入侵。

公元930年 南汉高祖遣兵攻占交趾,擒静海节度使曲承美(越南史上之曲后主)。南汉军入侵占城,大掠而还。

公元931年 交趾爱州人杨廷艺起兵反抗南汉。

公元938年 白藤江之战,吴权(杨廷艺之婿)指挥交趾军大破南汉军。

公元939年 吴权称王,定都于古螺,建立吴朝。越南自此脱离帝国。

公元942年 南汉高祖崩,第三子刘玢继位,是为南汉殇帝。同年,博罗小吏张遇贤于循州起事,自称“中天大国王”,年号“永乐”,攻克兴王府以东大部分州县。

公元943年 高祖第四子刘弘熙发动政变弑殇帝、篡位,改名刘晟,是为南汉中宗。此后十二年间,中宗尽杀其兄弟十四人及十四人之子孙,收其女眷入宫。同年,张遇贤率军北伐南唐之虔州,败亡。

公元951年 中宗遣宦官潘崇彻率兵伐南楚,取郴州,复败南唐援兵于宜章。中宗“益得志”,大兴宫室。

公元952年 中宗遣宦官吴怀恩率兵伐南楚,取连州、桂州,尽有岭南地。

公元958年 南汉中宗崩,其子刘继兴继位,改名刘鋹,是为南汉后主。后主在位时游乐于数十离宫之间,广修佛寺。

公元964年 宋军南侵,取南汉之郴州。

公元970年 宋太祖遣潘美入侵南汉,破昭州、桂州、连州。

公元971年 宋军攻广州,后主出降。南汉亡国,历4主,75年。韶州土豪周思琼、南汉春恩道指挥使麦汉琼起兵抗宋,经数月血战被屠灭。宋分天下为十五路,以岭南为广南路。

\section*{第四次北属}

公元972年 原南汉宦官乐范起兵抗宋,被屠灭。

公元997年 宋廷分广南路为广南东路、广南西路。此为“广西”、“广东”称呼之始。此后,元代岭南为广东道。明清岭南为广东布政司、广西布政司。

公元1052年 僮人豪酋侬智高起兵反宋,围攻广州五十七日。曲江士人余靖组织成功组织广州守城战。

公元1075年 越南李朝宦官李常杰率兵北侵,破钦州、廉州、邕州,尽屠邕州五万八千人而还。

公元1087年 宋广州知府蒋之奇扩建原本仅为一所狭窄厅堂的广州府学。

公元1127年 靖康之变。一批北方难民此后经南雄珠玑巷进入南粤。

公元1130—1135年 湖湘爆发种相、杨幺之乱,兵匪流劫、屠戮粤北连、韶、南雄、循、梅、英德六州,粤北人口锐减、田土抛荒严重,大批居民向珠三角移民,珠三角开发加速。

公元1170年 宋韶州知府周舜元支持建相江书院于韶州,祀周敦颐。理学入粤自此始。

公元1211年 朱子弟子廖德明(广州知府)于广州首次刊行《朱子家礼》。

公元1244年 广州举行首次正式的大型乡饮酒礼。

公元1276年 元军入临安、下湘赣,此后又一批北方难民经南雄珠玑巷入粤。元军南侵,入广州。此后一年内,元军与义兵反复拉锯,广州四次易手。南宋朝廷迁入南粤

公元1277年 元军第三次侵占广州,夷平东、西城。

公元1278年 元军攻潮州月余,破城,知州马发自尽。元军屠城,“城中居民无噍类”。

公元1279年 崖山海战,南宋亡。

公元1285—1286年 元世祖两度暂禁“商贾航海者”。

公元1303年 元廷下令“禁商下海”。

公元1314年 元廷下令“仍禁人下蕃”。

公元1323年 元廷开海禁,“听海商贸易,归征其税”。

公元1337年 增城人朱光卿起义,称大金国、改元赤符,被元军屠灭。

公元1352年 湖南流寇南侵,陷乐昌,驱民数万攻韶州。

公元1353年 海寇邵宗愚起兵于南海县,自称元帅。

公元1355年 元末大洪水在南粤爆发。东莞土豪王成、何真等起兵割据自保。

公元1361年 何真袭据惠州,元廷授其惠阳路判官、广东宣慰司都元帅。

公元1363年 邵宗愚军入广州,大肆焚掠,被何真率军击退。

公元1366年 何真攻灭王成,统一东莞。

\section*{何真政权}

公元1367年 何真平定粤东,入主广州,“时中原大乱,南北阻绝,真练兵据险,保障一隅。”

公元1368年 二月,朱元璋遣征南将军廖永忠等率军三路侵粤。三月,廖永忠军至东莞,何真“奉表迎降”。


\section*{第五次北属}

公元1371年 朱元璋“禁濒海民不得出海”。同年,何真首次回乡收集旧部。

公元1372—1373年 朱元璋命何真回乡收集旧部2万余人及其家属,送往北方。

公元1376年 珠江口著名堤围桑园围开始修建。

公元1381年 朱元璋开始在全帝国范围内推行里甲制度。何真旧部起兵反抗,明南雄侯赵庸于是年及次年两次征剿,屠杀万余人。

公元1383—1384年 朱元璋命何真回乡收集土豪、旧部万余人,送往北方。

公元1387年 朱元璋封何真为东莞伯,“赐”第于南京。

公元1388年 何真去世。

公元1393年 何真一族被朱元璋罗织入“蓝党”,族诛。

公元1394年 朱元璋在南粤发动社会改造,大批百姓被编入军籍,造成“民籍者鲜矣”之局。

公元1397年 南海人梁道明于大巽他群岛之三佛齐称王。

公元1405年 明郑和舰队于旧港屠灭南粤“海贼”陈祖义。

公元1426年 潮州通事(翻译)引“倭船”泊于饶平,与民贸易。

公元1448年 南海冲鹤堡人黄萧养聚众万人作乱。

公元1449年 黄萧养率流民、蜑民、裹挟十余万人围攻广州,分兵攻佛山。佛山土豪“二十二老”于祖庙誓师抗贼。

公元1450年 黄萧养之乱被明军及南粤土豪联手讨平。此后,明廷开始默认南粤基层社会的土豪自治状态。

公元1463年 瑶人袭击新会,县丞陶鲁聚集土豪击退之。

公元1465年 明两广总督韩雍屠灭广西大藤峡“瑶乱”。

公元1469年 陈白沙在新会展开推行儒家乡礼之社会实验。

公元1501—1502年 琼州黎人符南蛇发动起义,被屠灭。

\section*{海通时代}

公元1513年 葡萄牙人首次于南粤海岸登陆。

公元1517年 葡萄牙舰队驶入珠江口,首次到达广州。

公元1518年 明南赣巡抚王守仁出兵屠灭粤东北浰头、上陵之畲人起义。同年,南粤理学官员霍韬、方献夫、湛若水始于南海西樵山往来讲学、构建儒学共同体。

公元1520—1522年 明广东提学副使魏校发动“毁淫祠”运动,大批不载于“祀典”之佛寺、庙宇被毁。

公元1525年 佛山石头村霍氏于霍韬主持下购原淫祠之地,构筑宗祠。石头霍氏之宗族建设,为南粤十六世纪宗族化运动之代表。南雄珠玑巷祖先神话加剧扩散。清初,宗族普及于南粤。

公元1552年 耶稣会士、西班牙人方济各*沙勿略欲至明帝国传教,于南粤上川岛荣归主怀。

公元1557年 葡萄牙人于澳门修建永久性房屋。澳门城邦史由是始。

公元1558年 饶平库吏张琏起义。

公元1560年 张琏自称“飞龙人主”,克程乡、大埔、小靖,联合“倭寇”。

公元1562年 明军屠灭张琏起义,屠杀三万人。张琏逃亡南洋,为旧港“蕃舶长”。

公元1565年 明总兵戚继光部于潮州南澳岛屠灭“海贼”吴平部,屠杀一万五千人。

公元1566年 澄海“海贼”林道乾部入据南澳岛,远航至台湾、渤泥(今文莱一带)后返航。

公元1568年 饶平“海贼”林凤克惠来县神泉港。

公元1573年 林道乾舰队袭粤西阳江。

公元1574年 明总兵张元勋部破林道乾栖身之潮州南洋寨,尽屠寨中男女千余,林道乾突围逃亡南洋。林凤舰队远征琼州,杀明军民二万余人,复远征吕宋,攻西班牙帝国治下之吕宋首府马尼拉,不克。

公元1575年 明、西联军于吕宋围剿林凤部,林凤突围,不知所踪。

公元1576年 明两广总督凌云翼屠灭罗旁山“瑶乱”,屠、虏四万人。

公元1578年 林道乾舰队航至大泥,“攘其地以居,号道乾港”。北大年(暹罗属国)女王招林道乾为婿。

公元1583年 耶稣会士、意大利人罗明坚、利玛窦入居肇庆,开始传教。

公元1584年 利玛窦于肇庆建天主教堂仙花寺。

公元1620年 佛山乡居高官李待问主持建立佛山首个制度化议会组织嘉会堂。

公元1625年 郑芝龙大会闽、粤海贼领袖于台湾北港溪,结成“十八芝”同盟。

公元1629年 南粤海贼刘香与郑芝龙决裂。

公元1632年 刘香舰队掠福建南安、同安、海澄。

公元1633年 料罗湾海战,郑芝龙舰队击败荷兰*刘香联合舰队。

公元1635年 田尾洋海战,郑芝龙舰队攻灭刘香舰队,刘香战死。

公元1637年 英国商船队因明帝国拒绝与其通商,袭击虎门炮台,克之而去。

公元1646年 南明绍武政权、永历政权分别建立于广州、肇庆。清提督李成栋部占领广州,绍武政权灭亡。同年,粤东北客家人组织忠于明朝之武装“九军”,与潮州福佬人(潮汕人)械斗。九军破揭阳,尽屠城中士绅、掳走妇女。

公元1647年 “广东三忠”张家玉、陈邦彦、陈子壮组织义兵抗清,围攻广州不克。三人悉数壮烈殉难。

公元1648年 李成栋举南粤之地反清归明。

公元1650年 清平南王尚可喜、靖南王耿继茂率兵围攻广州十一个月,破城。清军血腥屠城,制造广州大屠杀(庚寅之劫),死者10万至70万(各种估计不等)。同年,香山人基督徒郑玛诺随法国神父陆德抵达罗马,就读于耶稣会主办之学校。

公元1653—1654年 南明李定国部围攻新会,不克而退。围城期间,清守军屠居民为食。

公元1658年 清军破新宁县文村,屠灭义兵王兴部。

公元1661年 清军破阳江县永丰寨,屠灭南粤最后的抗清武装。同年,清廷发布“迁海令”。

公元1662年 南粤沿海居民被强制内迁50里。

公元1664年 南粤沿海居民再内迁30里。迁海时,不愿离乡者惨遭清军屠戮,内迁者亦死亡枕籍。

公元1667年 耶稣会、方济各会、多明我会二十三名传教士聚于广州,召开著名的广州会议(四十天会议),决定延续利玛窦尊重东亚传统的传教路线。

公元1668年 清广东巡抚王来任上疏请求复界。

公元1669年 清圣祖下令复1664年之界。

公元1676年 尚之信幽禁其父尚可喜,举兵响应吴三桂。

公元1677年 尚之信降清。

公元1680年 清廷杀尚之信,消灭平南王藩府。

公元1684年 清廷于南粤完全复界。迁海浩劫,南粤死者超过90万。大批客家人自粤东北迁移至粤东、珠三角。

公元1685年 清圣祖下令开海贸易,设粤、闽、浙、江四海关。

公元1686年 广州十三行成立。

公元1698年 第一艘法国商船驶达广州。

公元1699年 英国商船“麦士里菲尔德”号驶至广州黄埔港,展开英国对广州之贸易。

公元1717年 清圣祖禁清国商船至南洋贸易。

公元1723年 清廷默许南粤重开南洋海禁。四年后,清世宗全面解除南洋海禁。

公元1732年 因“礼仪之争”导致禁教,清当局将广州的三十五名传教士放逐至澳门。

公元1734年 潮州人郑信建立泰国吞武里王朝。

公元1738年 佛山出现新议会组织大魁堂。

公元1757年 清高宗下令关闭江、浙、闽海关,实行粤海关一口通商政策。

公元1770年 梅州客家人罗芳伯于西婆罗洲坤甸成立兰芳公司。

公元1777年 罗芳伯改“兰芳公司”为“兰芳共和国”,定都东万律,任总制。

公元1782年 粤语广州话韵书《分韵撮要》出版。

公元1784年 美国商船“中国皇后号”首航广州,粤美贸易开始。

公元1795年 罗芳伯去世。兰芳共和国公民选举江戊伯接任总制。

公元1796年 五名南粤水手乘“路易斯夫人”号抵达美国费城。

公元1802年 越南阮朝灭西山朝。效忠于西山朝之南粤海贼驶返粤洋,形成红、黑、黄、青、蓝、白六大旗帮。

公元1808年 孖州海战,郑一嫂、张保仔之红旗帮大破清广东水师。

公元1809年 广州湾海战,红旗帮大破清广东水师,击毙清总兵许廷桂。郭婆带之黑旗帮克番禺县三善乡。此后,郭婆带、张保仔火拼,郭婆带率黑旗帮降清。

公元1810年 郑一嫂、张保仔率红旗帮降清。张保仔保留部分舰队,助清军剿灭蓝、黄、青、白四帮。

公元1807年 英国新教传教士马礼逊抵达澳门、广州,开始传教。

公元1824年 清两广总督阮元建学海堂于广州越秀山。

公元1839年 清钦差大臣林则徐进行“虎门销烟”。广东水师与英国舰队于新安县穿鼻洋武装冲突,鸦片战争爆发。

公元1841年 广州城向英军投降。广东团练与英军冲突于三元里。英国割占香港岛,香港城邦史由是始。

公元1848年 南粤移民于美国三藩市建立北美首个唐人街。

公元1849年 清两广总督徐广缙煽动起“广州反入城事件”,阻止英商入广州。

公元1850—1852年 信宜拜上帝教众凌十八聚众起事,两年间转战高州、玉林、罗定等处,被清军讨平。

公元1854年 天地会于珠三角发动洪兵之乱,围攻广州。清廷征集客勇围剿洪兵。四邑、珠三角广府、客家互相仇杀,惨烈的土客战争爆发,以四邑战事最惨。

公元1855年 清两广总督叶名琛指挥清军、南粤团练击败广州城外洪兵。事成后,叶名琛滥施暴行,搜捕洪兵,仅在广州城内即屠杀7.5万人。南粤境内死者达10万—100万人(各种估计不等)。洪兵主力西退至广西,破浔州,以之为“秀京”,年号“洪德”,立国号“大成”。

公元1856年 亚罗号事件,第二次鸦片战争爆发。广州暴民焚毁十三行。

公元1858年 英法联军克广州,擒叶名琛,扶植广东巡抚柏贵统治广州。同年,《天津条约》签订,汕头开埠。

公元1859年 英法联军讨平广州郊外反英团练。南粤士绅开始与联军积极合作。英、法广州沙面租界建立。

公元1860年 《北京条约》签订,英国割占九龙司地方一区,香港城邦面积扩大。

公元1861年 英法联军撤离广州。清军破秀京,大成国余部逃走。

公元1862年 粤语版《马太福音》由美国长老会于广州出版。

公元1864年 清军消灭大成国余部。

公元1866年 惨烈的土客战争结束。战争中,广府、客家两族粤胞死难者共计100万人。

公元1872年 南海人陈启沅于南海县西樵简村开办继昌隆缫丝厂,是为南粤首间西式民营工厂。

公元1886年 荷兰攻灭兰芳共和国。

公元1888年 广州石室圣心大教堂建成。

公元1893年 康有为于广州建万木草堂。

公元1894年 粤语全译本《新旧约圣经》由美国圣经公会于上海出版。广州陈氏书院(陈家祠)建成。

公元1895年 孙文首次起事于广州,失败。

公元1898年 戊戌变法失败,清廷查封万木草堂。英国租借深圳河南岸新界地区99年。香港城邦面积再次扩大。

公元1901年 粤汉铁路动工。

公元1902年 欧榘甲于日本出版《新广东》,宣扬“广东人之广东”。

公元1903年 清两广总督岑春煊始练广东新军。

公元1906年 清两广总督岑春煊欲收粤汉铁路粤段为官办,经粤人激烈反对后不果。

公元1907年 顺德人黄节之《广东乡土地理教科书》出版,称广府为“汉种”,福佬(潮汕)、客家非“汉种”。潮汕、客家士子被激怒,于汕头《岭东日报》撰文激烈批驳。

公元1908年 《广东乡土地理教科书》再版,删除称福佬、客家为非“汉种”之文字。

公元1909年 广东谘议局成立。

\section*{民国自立时代}

公元1911年 三月,革命党起事于广州,惨败,其72具遗尸被葬于黄花岗。十月,武昌兵变,清帝国开始瓦解。十一月,南粤议绅于广东谘议局宣布广东和平独立,成立广东军政府,以胡汉民为都督、陈炯明副之。

公元1912年 中华民国成立,广东成为中华民国一省。

公元1913年 六月,民国大总统袁世凯解除都督胡汉民之职,陈炯明代之。七月,“二次革命”爆发,陈炯明宣布广东独立,自任广东讨袁军总司令。八月,粤军一部叛,陈炯明流亡香港,广东取消独立。民国广东镇抚使龙济光入据广州,解散广东议会。

公元1916年 先是,袁世凯于去年(1915年)末称帝。二月,革命党朱执信率部袭取广州。四月,南粤商民迫使龙济光宣布广东独立。五月,龙济光擅自取消独立。七月,滇、桂军(护国军)逼近广州,与龙济光军战于广州郊外。十月,龙济光撤离广州。护国军抚军陆荣廷自任广东都督。

公元1917年 四月,民国大总统黎元洪任陆荣廷为两广巡阅使,陆荣廷占据两广。十月,广州召开中华民国非常国会,选举孙文为大元帅,陆荣廷、唐继尧为元帅,组成护法军政府。护法军北伐入湘。

公元1918年 五月,军政府改组,选举唐绍仪、唐继尧、孙文、伍廷芳、林葆怿、岑春煊七人为政务总裁。孙文与滇、桂系矛盾激化,未就总裁职而离粤。六月,岑春煊任主席总裁。

公元1919年 孙文正式电辞军政府总裁之职。

公元1920年 三月,军政府总裁仅剩岑春煊、陆荣廷、林葆怿三人,大批国会议员离粤,其命令已属无效。八月,陈炯明奉孙文之命率粤军驱桂,是为第一次粤桂战争。十月,岑、陆、林联名通电,解除军政府职务。十一月,孙文重据广州,通电恢复护法军政府。孙文谋求于广州成立取代北京之中华民国政府。

公元1921年 一月,中华民国国会非常会议于广州召开。四月,国会非常会议选举孙文为非常大总统、委陈炯明为陆军部长、内务部长。六月,孙文下令攻打广西之陆荣廷,任陈炯明为援桂军总司令,第二次粤桂战争爆发。八月,粤军陷南宁,陆荣廷离桂流亡。

公元1922年 五月,孙文下令攻打中华民国大总统徐世昌,粤军北伐入赣。六月,陈炯明讨伐孙文,炮击总统府,孙文避入永丰舰。孙文以海军炮击广州,杀害平民百计。八月,孙文逃亡上海。陈炯明任粤军总司令,联络赣、湘、浙、黔、滇诸省,策划联省自治。

公元1923年 一月,孙文收买滇、桂军一部,联合拥孙之粤军许崇智部,分西、东两路合击陈炯明。陈炯明通电下野,由广州撤至惠州。同月,孙文与苏联特使越飞发表联合之宣言。二月,孙文重据广州。十月,粤军反攻,与孙军战于广州近郊。

公元1924年 一月,孙文于广州主持召开国民党一大,与中共合作,第一次国共合作开始。二月,黄埔军校成立,是为党军之始。十月,广东省商团军联防总部总长陈廉伯率广州商团起兵抗孙,国民党军屠灭之,火烧西关、大肆屠掠,平民死难者千计,陈廉伯流亡香港。

公元1925年 二月,国民党以蒋中正为东征总指挥、周恩来为前方政治部主任,进攻陈炯明部,国民党军陷淡水。三月,孙文死于北京。六月,滇、桂军于广州起兵反抗国民党,战败。六月,国共发起省港大罢工,大元帅大本营宣布对英绝交。同月,英、法军于广州沙面租界遭游行队伍枪击,开枪还击,杀52人,是为“沙基事件”。七月,大元帅大本营改组为国民政府。九月,国民党预备再次东征,蒋中正任东征军总指挥。十月,国民党军东征,陷惠州。十一月,国民党军陷潮州,陈炯明流亡香港。

公元1926年 一月,国民党军陷海南岛。三月,“两广统一”,李宗仁率新桂系投靠国民党。七月,国民政府发动北伐战争。

公元1927年 二月,国民党解散佛山大魁堂,佛山六百年自治城市史终结。四月,国共决裂。国民党广东省政府主席李济深主持清党,杀2100余人。九月,三河坝之战,国民党军击败自南昌南下之红军。十二月,共产党起事于广州,失败。

公元1929年 三月,蒋桂战争爆发,李济深因反蒋之故,被扣押于南京。国民政府任陈济棠为广东绥靖委员,入广州主持南粤政局。

公元1931年 四月,陈济棠通电反蒋,驱逐广东省长陈铭枢。五月,汪兆铭于广州另立国民政府,陈济棠任广州国民政府委员、军委会常委。九月,满洲事变爆发,广州国民政府取消。

公元1933年 陈济棠主持建造广州海珠桥。

公元1936年 六月,陈济棠与新桂系联合举兵反蒋,两广事变爆发。七月,两广成立军事委员会、抗日救国军,以陈济棠为委员长兼总司令,以李宗仁为副总司令,进军湖南。七月,粤空军司令黄光锐率飞机70余架叛投蒋中正,陈济棠被迫通电下野。九月,蒋中正、李宗仁会晤于广州,两广事变结束,广东归入南京控制。



\section*{列宁主义时期}

公元1937年 广州爱群大厦(高十五层)落成。

公元1938年 十月,日军攻占广州。

公元1941年 十二月,日军攻占香港。

公元1943年 潮汕大饥荒,饿死11万人、逃荒者300万人。

公元1945年 八月,日本战败,二战结束。九月,国民党新一军进入广州,军纪败坏、大肆“劫收”。

公元1949年 二月,国民党中央党部迁至广州(十月再迁至重庆)。十月,共军入粤,占领广州。弃城前夕,国民党军炸毁海珠桥,平民死伤200余。十二月,共军占领两广。

公元1950年 三月,共军占领海南岛。十月,广东土改试点运动开始。

公元1951年 二月,共军屠灭十万大山“土匪”。同年,陶铸在两广主持大规模土改。至次年两广土改告一段落,仅广西死者即在4万以上。

公元1952年 第一次广东“反地方主义运动”,7000余本土干部受到处分。

公元1957年 第二次广东“反地方主义运动”,10000余本土干部受到处分。广东80\%之主要领导变为北方人。同年,首次逃港潮爆发。

公元1958—1962年 大跃进、大饥荒,两广饿死162万人。1962年,第二次逃港潮爆发,4000人逃亡成功,51000余人被强制遣返。

公元1965年 广东之钦州、廉州被划入广西。同年,以潮汕人、广府人为主体之新加坡独立。

公元1967年 香港左派组织“六七暴动”,被港英政府平定。同年香港TVB开台,全天候提供粤语电视节目。

公元1974年 港英政府宣布对大陆难民实施“抵垒政策”,“内地非法入境者”若能抵达香港市区或接触到在港亲人,即可在港居留。

公元1978年 香港旭日集团于顺德容奇镇投资建厂,是为港资入粤之始。

公元1979年 第三次逃港潮爆发,参与人数10万余人,逃亡成功4万余人。

公元1980年 深圳、珠海、汕头获得特许权,成为“经济特区”。同年,港英政府取消“抵垒政策”,改为遣返所有“非法入境者”之“即捕即解政策”。

公元1988年 海南被建省,与广东分离。

公元1997年 香港“回归”,157年之城邦史结束。

公元1999年 澳门“回归”,442年之城邦史结束。

公元2009年 先是,1980年代以来,大量外来人口入粤,粤语普及率在南粤下降。是年,广州电视台推出普通话节目,引起收视率下降。

公元2010年 北方官员二代纪可光抛出改广州电视台应主要使用普通话之提案,引起市民愤怒并发起“撑粤语运动”。广州江南西之撑粤语集会有上万人参加。

公元2011年 增城新塘、潮州古巷爆发粤人与川、湘务工人员械斗之事件。

公元2016年 历史,正在进行
