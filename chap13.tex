\chapter{保卫华夏文明的百越武士:南粤抗清战争}

\section{绍武、永历政权的并立与桂林之战}


\indent 公元1644年三月,李自成的流寇大军攻陷北京,崇祯帝自尽,明帝国灭亡。四月,李自成在著名的山海关大战中惨败于清军,随后西撤。五月,明福王朱由崧于南京称帝,是为南明弘光帝。十月,清摄政王多尔衮迎时年七岁的顺治帝入关,定都北京。短短不到一年的时间内,北京城三易其主。而在山海远隔的南粤,此时尚无人察觉到即将来临的滔天洪水。明帝国依然维系着对南粤的统治,人们仍一如既往地生活。

岭北局势的变化之快远出粤人意料。1645年夏,清军攻灭弘光政权,一直打到钱塘江北岸,随即颁布野蛮的“剃发令”。吴越大地反剃发义兵蜂起,陷入一片腥风血雨。郑芝龙及闽籍高官黄道周(漳浦铜山人)随即在福州拥立明唐王朱聿键继位,是为南明隆武帝。隆武朝廷之实权由郑芝龙一手掌握,政治德性低下的他毫无抗清意图,所思唯有在明清之间投机。在郑芝龙的挚肘下,一心北伐的黄道周得不到任何军饷,只得召集一支三千余人的义兵入赣作战,后于当年十二月于徽州婺源全军覆没。隆武帝只得寄厚望于郑芝龙之子郑成功,赐予他明帝国“国姓”朱。然而,由于郑成功年少势弱,并不能扭转其父的态度。1646年五月,清军强渡钱塘江,至七月完全占领南吴越。此时,郑芝龙见清军大兵压境,早已与之暗中约降。八月,清军兵不血刃通过天险仙霞关进入闽越,杀隆武帝于汀州。九月,福州在郑军未作任何抵抗的情况下陷落。十一月,郑军十一万三千人一齐放下武器,郑芝龙前往福州降清,试图令清帝国保全其势力,却被清军强制带往北京。清军随即进入郑氏家眷所在的安平镇,大肆淫掠,郑成功之母田川松被辱,羞愤自尽。郑成功愤怒难当,乃召集数千残兵前往南粤东端的南澳岛,起兵抗清。另一方面,又一支清军入侵江右,于1646年三月陷吉安、十月陷赣州,屠杀数十万人。这样,到1646年接近尾声时,清军已从赣、闽两个方向逼近南粤。南粤,处于风暴将至的阴霾中。

早在1646年九月,隆武帝被杀的消息便已传至南粤。在清帝国侵略军日益逼近的情况下,南粤士绅迅速作出反应,开始讨论拥立南明新帝抗清之事。清兵在闽越和江右的杀掠近在眼前,他们绝不想让他们深爱的家邦也遭受同样命运,乃决定与南明帝国结成机会主义的抗清同盟。当时,流落至粤的帝位人选有二:一为在梧州的永明王朱由榔、一为由闽浮海逃至广州的隆武帝之弟朱聿鐭\footnote{顾诚:《南明史》第十三章第一节}。当时,曾任隆武朝大学士的东莞人苏观生正随侍朱聿鐭左右,因而主张拥立朱聿鐭。然而,明帝国在南粤的实权人物两广总督丁魁楚(河南永城人)、广西巡抚瞿式耜(吴越常熟人)却抢先一步,于十月在肇庆立朱由榔为“监国”。苏观生见此,只得维系抗清同盟的团结,派兵部主事陈邦彦至肇庆“朝贺”,却被傲慢的丁魁楚拒绝。盛怒之下,苏观生乃联合广州官绅梁朝钟、关捷先等,于十一月五日于广州立朱聿鐭为帝,是为南明绍武帝。丁、瞿二人不甘示弱,在十三日后针锋相对地立朱由榔为帝,是为南明永历帝\footnote{《广东通史》古代下册,页729}。在岭北帝国官僚的蓄意破坏下,粤明抗清同盟在一开始就蒙上了瓦解的危险。

绍武政权由南粤士绅一手建立,显然更代表粤人的利益。对于这个政权,颟顸的丁、瞿二人完全不放在眼里,派出两名使节赴广州,要求绍武君臣向永历臣服。苏观生视之为耻辱,下令将二使拖出斩首,两政权随之开战\footnote{顾诚:《南明史》第十三章第一节}。永历帝当即命兵部侍郎林佳鼎率军一万乘舟沿西江向广州进攻,于三水县城以西大破绍武军,斩杀八百余人,俘督师陈际泰\footnote{《广东通史》古代下册,页730}。永历军志得意满,进至广州城外的三山海口,遭遇绍武军总兵林察所部阻击。林察军中有一批珠江口的海盗,所用大船远强于永历军的内河小船。十二月二日,两军在三山海口展开决战。在绍武军火器的猛烈射击下,永历军一败涂地,纷纷弃舟登岸,陷入三尺多深的泥淖中,全军覆没,林佳鼎本人陷入水中溺死\footnote{顾诚:《南明史》第十三章第一节}。消息传至肇庆,永历君臣惊恐至极,其攻占广州的图谋宣告彻底破产\footnote{《广东通史》古代下册,页730}。

绍武君臣虽成功打退了永历军的进犯,但其政权内部问题很大。苏观生虽有保卫南粤之决心,却不幸是个妄自尊大、见识不高的人。当年十一月,就在绍武、永历大打出手之际,清将佟养甲、李成栋已率兵自闽南入粤,轻取潮州、惠州。佟养甲出身于辽东汉军八旗,系满清宿将。李成栋原为陕北流寇,后先降明、再降清,曾在吴越制造惨绝人寰的嘉定三屠和昆山大屠杀,是个心狠手辣、阴险狡诈的屠夫。在李成栋的策划下,清兵每进至一地皆利用当地官印伪造“无警”的塘报送往广州,苏观生则信以为真,毫不防备,忙于在广州城内举办庆典。十二月十五日,清帝国侵略军的前锋骑兵突然从东门冲入广州城。当时,绍武君臣在广州武学举行庆典,沉浸在胜利喜悦中的苏观生如在梦中,竟以“妄言惑众”之罪斩杀了报信人。很快,以白帕包头的大批清兵便登上城墙,脱去头帕露出辫子,大肆放箭,城中顿时一片慌乱\footnote{顾诚:《南明史》第十三章第一节}。由于主力部队尚在广州城外防备永历军,苏观生于仓促之间仅能调集百余人迎战,根本无法御敌,广州遂告沦陷,绍武君臣皆被俘虏。被俘后,绍武帝拒绝李成栋送来的饮食,称“吾若饮汝一勺水,何以见先帝于地下”,自缢而死,苏观生、梁朝钟亦为之自缢殉死,关捷先则屈膝投降。清帝国侵略军攻陷广州后,以“放赏”为名大肆淫掠三日,给广州居民带来了巨大的苦难\footnote{《广东通史》古代下册,页731}。

懦弱的永历帝听说广州已陷,慌忙地抛下南粤军民,于十二月二十五日乘小舟逃往广西梧州,身边从行的大臣仅有瞿式耜一人。佟养甲乃率数百清兵留守广州,命李成栋以主力追击永历帝。李成栋紧追不舍,于1647年正月连陷肇庆、梧州、平乐,深入广西境内,永历帝复逃至全州\footnote{《广东通史》古代下册,页731}。这时,无耻的丁魁楚早已带着他在南粤为官多年搜刮、掠夺来的八十余万两家财和包括成群妻妾在内的家人抛弃永历帝跑到广西岑溪,与李成栋暗中约降。李成栋对其财富垂涎三尺,便假意应允。二月,丁魁楚出降,被清兵像囚犯一样押往广东,于半路被杀。其族中男子无论少长,皆遭斩首,女子及大批家财则被贪婪的清将瓜分\footnote{顾诚:《南明史》第十三章第一节}。三月十一日,一支数百人的清军前锋进至桂林城下,被明军击退\footnote{《明清之交的中国葡萄牙雇佣军》}。至此,永历帝已近乎走投无路。就在这时,澳门的葡萄牙人伸出了援手。

在永历朝廷中,有一位名叫毕方济(Francesco Sambiasi)的意大利籍耶稣会士深受器重。走投无路的永历帝只得向他求助,命他前往澳门求援。毕方济果然不负重托,带回了一支以葡将尼古拉·费雷拉(Nicolau Ferreira)为统帅、有火炮数门、兵力三百的葡萄牙雇佣军,进驻桂林。五月二十五日,自湖南南下的清军孔有德部大举进攻桂林城,被葡军以火器射退。次日,葡军开城出战,排成长枪兵居两翼、火枪手居中、纵深十列的“摩里斯横队”正面迎击清军骑兵冲击,大获全胜\footnote{《明清之交的中国葡萄牙雇佣军》}。清军伤亡高达数千,狼狈败退,被追杀数十里\footnote{《广西通史》卷1,页409}。在葡人的帮助下,广西总算转危为安。

然而,广东军民已被卑怯的永历政权彻底丢给了凶残的李成栋。面对此种极度无耻的叛卖行为,广东的士绅、百姓决定不再依靠南明帝国,靠自己的力量保家卫国。作为刚刚完成华夏化的族群,粤人绝不会轻易向清帝国侵略军低头,屈辱地留起胡人的辫发。作为百越武士的直系后裔,南粤小华夏的人们将靠自己的力量守卫华夏文明的最后尊严、守卫延绵千万年的古老自由。1647年初,三位英雄站了出来。他们将领导广东人发起壮绝的英勇抵抗,为自由与尊严而战。这三人,便是名垂青史的“广东三忠”。

\section{“广东三忠”的英勇殉难:陈邦彦、张家玉、陈子壮}

\indent 今天的广州白云山上,有一面巨大的浮雕刻划着十余位南粤历史上的著名人物。其中,陈邦彦、张家玉、陈子壮三人慷慨激昂的形象占据着十分显眼的位置,暗示着三人在南粤史上的重要地位。他们究竟有过何等动人的事迹,方能在今日粤人心中占据如此重要的地位?欲明此,我们便需回到公元1647年的历史现场,与他们一同经历那场英勇绝伦的抗争。

陈邦彦,字令斌,顺德龙山人,于1604年出生在一个儒生家庭中。天资聪颖的他曾于十八岁那年考得秀才,然此后屡试不中,遂绝意于科场,一心钻研经世致用之学,于大良开馆授徒,并时常与明帝国地方官府交涉,维护家乡父老的利益。1646年,陈邦彦被南明隆武帝委任为兵部主事,后与苏观生一同跟随绍武帝逃到广州,并被派往肇庆“朝贺”永历朝廷。“朝贺”失败后,绍武、永历政权很快开战。陈邦彦对此心灰意冷,遂逃入高明山中隐居,决心不问政事\footnote{顾诚:《南明史》第十三章第三节}。

不过,时局的快速变化不容许陈邦彦隐居。1646年底,永历帝无耻地背叛粤明同盟,抛下广东军民独自逃命。在清帝国侵略军的铁蹄下,广东百姓遭受着巨大的苦难。见及于此,陈邦彦明白自己必须站出来保卫南粤了,乃决定出山。当时,侵粤清军主力已在李成栋的率领下西进追击永历帝,留守广州者仅有佟养甲部的数百人。陈邦彦来到西江边,望见一望无际的清帝国侵略军军旗,做出了如下的战局分析:

\begin{quote}

夫若乘其未定,得奇兵径袭广州,此孙膑所以救赵也\footnote{顾诚:《南明史》第十三章第三节}。

\end{quote}

陈邦彦的计划,是趁广东清军兵力薄弱之际迅速袭取广州,迫使李成栋回援,进而解广西之危。当时,顺德县境内有个叫做余龙的豪强,曾参加南明军,在赣南和清军打过仗,目睹了清军制造的赣州大屠杀。余龙绝不愿南粤落入残暴的清兵之手,便来到顺德甘竹滩结寨自保,依附他的农民有二万余人。在陈邦彦的劝说下,余龙同意与其合作抗清。二月初,两人率义军乘船直取广州,于十日大破清军水师于广州城外,击毙清总兵陈虎、焚清船一百多艘。十一日,义军进攻广州城。佟养甲下令关闭城门,以数百名久经沙场的汉军旗兵拼死守城,缺乏作战经验的义军一时难以攻克。但由于双方兵力差距悬殊,收复广州只是时间问题。狡猾的佟养甲一面传檄李成栋迅速回援,一面放出谣言,称清军援兵将直取甘竹滩。爱乡心切的余龙信以为真,于十四日收兵撤回。陈邦彦孤掌难鸣,只得率少数亲兵黯然退回顺德\footnote{顾诚:《南明史》第十三章第三节;《广东通史》古代下册,页732}。此次进攻广州的战役,便这样令人扼腕地功亏一篑。这时,东莞的大英雄张家玉向他伸出了援手。

张家玉,字玄子,东莞县城西北村头村人。1616年,他出生于当地一个贫穷的农户家庭,自幼相貌俊美,好任侠、善用剑,时常结交英雄豪杰,是个极富正义感的勇武美男子。此外,他亦非常聪明,于少年时代凭一己之力学习了大量经史知识,十九岁考中秀才、二十二岁中举人、二十九岁中进士,成为北京的一名翰林官。然而,仅仅在张家玉中进士后不到一年,李自成的流寇大军就攻陷了北京城。据《明史》记载,当时的张家玉表现得十分懦弱:

\begin{quote}

贼以杀其(张家玉)父母恐之,乃跪。其实父母尚在粤也\footnote{张廷玉:《明史·张家玉传》}。

\end{quote}

张家玉之所以如闹剧般地向李自成投降,实则因他实在无爱于明帝国。后来发生的事情表明,他绝非一个懦夫。1644年四月,李自成在山海关被清兵击败,弃北京西遁,张家玉乘机南逃,抵达南京。次年五月,清军占领南京,张家玉复逃至福州投奔隆武政权,先后担任礼、兵部侍郎。1646年清兵入闽时,郑芝龙指使郑军不作抵抗,张家玉则在新城县奋勇出战,中箭落马摔伤手臂,被隆武帝调往粤东潮州、惠州招募义勇。然而,因隆武帝在不久后被清兵杀害,张家玉心灰意冷,遂解散乡勇,返回东莞家乡。他没有加入绍武政权,而是积极联络家乡父老,随时准备迎接即将来临的洪水。

1646年十二月,清兵占领广州,东莞随之沦陷。清帝国侵略军在东莞县境内飞扬跋扈,四处抢掠,激起东莞百姓的巨大愤怒。1647年正月,张家玉联合好友博罗县举人韩如璜起兵誓师抗清,响应者达五千余人。三月十四日,义军攻克东莞县城,俘获清知县郑鋈。这时,李成栋已率清军主力回援广东,对东莞发起猛攻。张家玉指挥义军英勇奋战,击毙清将成升,但最终不支,退守道滘镇。李成栋军随后追至,攻陷道滘。张家玉再次率军后撤,退保新安县之西乡(今深圳宝安区西乡街道)。凶残的李成栋因屡战不能捉到张家玉,恼羞成怒,便丧心病狂地将张家玉留在东莞县境内的三十余名家人全部杀害。虽然家人已全部惨死,但张家玉仍不屈服,而是与西乡土豪陈文豹联合,秣兵厉马,积极准备迎战,与他在北京时的表现形成了鲜明对比\footnote{张廷玉:《明史·张家玉传》}。这表明,张家玉所深爱的无疑是南粤。在南粤有难,需要他站出来保家卫国时,他一定会万死不辞。

四月,李成栋率军进攻西乡,与义军展开极其惨烈的攻防战。西乡百姓一齐上阵,用简陋的武器拼尽全力与清军精锐誓死战斗。双方大战三次,陈文豹在战斗中阵亡,然李成栋仍未能得手。六月,李成栋军再次进攻,双方围绕每一寸土地进行反复争夺。在清军的疯狂进攻下,西乡的数万名英雄男女战斗到了最后一刻,全部壮烈战死,无一人投降。在西乡附近以捕鱼为业的白石村,村民感佩于西乡人的忠义,亦拿起武器拼死搏杀,举村殉难。在壮烈的西乡保卫战中,清帝国侵略军死伤上万,李成栋部下精锐损失过半,已暂时无力再次进攻\footnote{《广东通史》古代下册,页733}。西乡、白石村百姓创造了明帝国军队无法创造的奇迹。他们的英勇奋战和悲壮牺牲,乃是我南粤武德最直观的反映,应被后人永远铭记。

西乡沦陷时,张家玉率少数亲兵冲出包围,向东转移。他利用李成栋军损失过半、无力进攻的机会,接连收复龙门、博罗、长宁、连平、归善等地,屯兵于博罗。在李成栋与张家玉缠斗之时,陈邦彦趁机招集乡勇,恢复实力,高明土豪麦而炫和南海土豪陈子壮“毁家纾难”响应之。陈子壮,字集生,南海沙贝村人。1596年,他出生于当地一个儒化官员家庭,自小便受到良好的儒学教育。陈子壮小时候是个神童,七岁便能作文。1619年,时年二十四岁的他考中探花,进入翰林院任职。此后,他因反对魏忠贤遭罢职,又在明思宗登基后被召回北京,于1632年担任礼部侍郎。因对明思宗增税镇压流寇不满,陈子壮“上疏劝谏”,触怒思宗,被关入锦衣卫诏狱,于次年遭罢职回乡的处分\footnote{张廷玉:《明史·陈子壮传》}。回到南粤后,他在广州白云山开设书院,讲学授徒。1643年,广州及附近州县发生饥荒,许多饥民涌入广州城乞讨。陈子壮捐出巨资,四处奔走筹款设立粥厂,拯救了数千人的生命\footnote{阮元:(道光)《广东通志》}。

南明隆武政权建立后,隆武帝“下诏”召陈子壮赴福州,被其拒绝。1646年底,他前往肇庆投奔永历政权,官拜大学士。不久后,清军入侵南粤,永历帝卑怯地狼狈西窜。为保卫家乡,陈子壮来到南海九江镇,发动当地土豪、百姓起兵抗清。陈子壮的义军不但有淳朴的南粤乡民,还有许多在水上生活、善于水战的疍民,甚至还有被称为“番鬼”的欧洲人、黑人雇佣兵,战斗力非常强大。1647年七月,他率部与陈邦彦会合,二人决定再次进攻广州。他们首先令三千名将士诈降进入广州城,约定于七月七日夜半三鼓时于城内外同时起事\footnote{《广东通史》下册,页733}。七月五日,陈子壮率军乘舟进抵广州城外的珠江江面。不幸的是,清军俘获了其在城外张贴檄文的家僮,经审讯后得知城中有内应。佟养甲遂在城中大肆搜捕,将三千内应全部擒获杀害,接着又突然开炮轰击义军战舰。当时,江面上正刮着强劲的北方,义军处于逆风位置。佟养甲开城出击,在白鹅潭击败义军,陈子壮长子陈上庸壮烈战死。陈子壮只得率军退保九江,婴城固守。就在陈子壮在广州城下与清军交锋之际,陈邦彦已率部收复三水,从西北面逼近广州。李成栋遂率部向北进攻。八月,两军大战于何真抵御陈友谅部将熊天瑞的古战场胥江,陈军先胜一阵,移驻清远。李成栋乃纠集起两万军队,于九月十九日攻打清远县城。陈邦彦平日与将士们同甘共苦,每日只进食一餐,夜中往往枕“假寐不就枕”,其部下人人感动,皆愿为之效死。面对以优势兵力来攻的侵略军,义军极力抵御,矢尽援绝,不幸战败,身中三刃的陈邦彦连同五名部将在城中被清兵俘虏。他们被押往广州,遭受极度残酷的“寸磔”之刑,英勇就义。跟随陈邦彦的义军将士,也大多战斗到最后一刻,为南粤的自由与尊严献出了宝贵的生命\footnote{顾诚:《南明史》第十三章第三节;《广东通史》古代下册,页732}。

李成栋的下一个目标是活跃于广州东侧的张家玉。十月初,张家玉率军从博罗出发,进攻增城县城。十日,李成栋率清军主力突然出现在增城城下,以骑兵发起突击,城中守军亦开城出击。在突如其来的内外夹攻下,义军猝不及防,牺牲6000余人。身负重伤的张家玉不愿做俘虏,投入水中,壮烈殉难\footnote{顾诚:《南明史》第十三章第三节;《广东通史》古代下册,页732;《广东通史》古代下册,页734}。数日后,李成栋挥军攻陷博罗,将韩如璜及其数百名族人全部屠杀,并疯狂屠戮城中百姓,制造了人口“十不存一”的惨剧。当时,有七名韩氏女子不愿受辱,一同投湖自尽。后来,当地人为了纪念她们,便将此湖称作“七女湖”\footnote{《“忠贞竞爽”牌匾见证博罗韩氏赤胆忠臣》}。七女湖之名承载着粤人勇敢抗清、博罗全城军民殉难的伟大历史,激励着一代代粤人为南粤而战。

陈邦彦、张家玉和他们的部下已近乎全部牺牲了,只有陈子壮仍在独自奋战。虽然如此,他仍下定决心决不投降,宁愿牺牲一切也要为牺牲的南粤军民报仇、保卫他深爱的家邦。他率军放弃九江,入守坚固的高明县城。十月二十五日,广东的清军几乎倾巢出动,在李成栋的指挥下向高明发起最后的进攻。在四天时间里,清兵向高明发动了一次又一次冲击,皆被义军打退。二十九日,城墙被清军大炮轰破,清兵蜂拥而入,陈子壮被俘。清兵入城后,杀害了所有未剃发结辫的人,并将陈子壮押回广州。面对侵略者的审讯,陈子壮视死如归,仅要求赦免其幼子陈上壮,表示“愿膏斧锧”。经会商之后,佟养甲、李成栋决定将他“寸磔”于广州校场\footnote{顾诚:《南明史》第十三章第三节;《Guongdonk 通史》古代下册,页732}。临刑之时,陈子壮忍受着巨大的疼痛,毫无呼痛求饶之态,带着粤人的高贵尊严慷慨就义\footnote{屈大均:《广东新语》}。

陈邦彦、张家玉、陈子壮被后世称为“广东三忠”,受到粤人和一切热爱自由的人们的永远怀念。他们依靠南粤土豪与百姓,在明帝国已完全抛弃广东的情况下坚持奋战一年,给李成栋麾下的清军精锐以重创,使到处逃窜的永历帝转危为安。为了保卫南粤不受侵犯、为了南粤小华夏的衣冠不被摧毁,他们及广东军民拼死战斗,谱写了一曲令任何人都会为之落泪的壮歌。他们不愧是百越文明最为优秀的苗裔、也不愧是华夏文明最纯正的嫡传与孤忠。他们的壮烈事迹应当永载史册,被后世子孙写入南粤的史诗代代传唱。“广东三忠”和广东百姓视死如归的奋勇战斗不但感动了所有粤人,亦使李成栋大为震撼。这个双手沾满东亚各族人鲜血的冷血屠夫开始思考,南粤人究竟是怎样的一群人,竟然能够在如此惨烈的屠杀下仍然屡仆屡起、坚持抵抗?1648年四月,李成栋突然做出了惊人的举动:他在广州正式对清帝国竖起叛旗,宣布“反正”,投靠永历政权。一场更大规模的腥风血雨,即将在南粤大地上演!

\section{李成栋之死与广州大屠杀}


\indent 1954年,一部名为《万世流芳张玉乔》的粤剧在香港上演,感动了无数观众。有人称,此剧剧情令人印象深刻,“故事好到无以复加”\footnote{钟哲平:《万世流芳张玉乔》}。剧中,一位名唤张玉乔的南粤女子以侍妾身份对李成栋以死相劝,终于使其回心转意,反正抗清。张玉乔究竟何以沦为杀人魔王李成栋的侍妾,又为何要劝谏李成栋?这一切,皆和17世纪中叶南粤那场惨绝人寰的浩劫有关。

1647年底,李成栋凶残地镇压了“广东三忠”的起义。陈子壮之妾张玉乔因才貌双全留得性命,被李成栋收入府中。张玉乔是明末广州名歌妓张二乔之妹,她们的母亲原为吴越苏州人,后因落难避走广州,沦落风尘,生下了姐妹二人。张二乔“体貌莹洁,性质明慧”,自幼便善于弹琴、作诗、作画。她曾在十三岁时所作的一首诗中明确表示,“离骚怨公子”并不为她所喜,唯有有侠肝义胆的男人才配得上她。1633年,陈子壮罢官回粤,一下子就爱上了小他十九岁的她。那个后来经受酷刑毫无惧色的硬汉陈子壮,曾为她写下“难将公子意,写入美人心”的柔情诗句。然而红颜薄命,年仅十九岁的张二乔在同年不幸去世。陈子壮悲痛不已,便将与张二乔相似的玉乔纳为小妾,以解哀思\footnote{《现实版林黛玉:“生有侠骨”的明末女诗人张二乔》,《广州日报》2015年12月10日}。陈子壮殉难后,她落入李成栋之手,沦为侍妾。

当时,信任旗人的清廷以佟养甲为两广总督兼广东巡抚,战功远更大的李成栋则屈居佟养甲之下,仅为广东提督。李成栋因此愤愤不平,但又无可奈何,对清廷的不满日增。曾任弘光政权内阁首辅、其时正避居于家的香山小榄乡绅何吾驺遂趁机与李成栋密会于广州,坚定了其反抗清帝国的决心。1648年正月,降清明将金声桓、王得仁在南昌起兵“反正”,进一步鼓舞了李成栋的斗志\footnote{顾诚:《南明史》第十三章第三节}。三、四月间,一件令李成栋受到巨大冲击的事件发生,成为促使他起兵抗清的最后一根稻草,那便是张玉乔的以死相谏\footnote{关于张玉乔的去世时间,参见司徒琳:《南明史》,页117}。

据说,正在李成栋游移之际,张玉乔私下怂恿他起兵响应金声桓。李成栋其时早有反清之谋,唯恐消息走露,便大声斥责张玉乔不得干政。张玉乔遂说:

\begin{quote}

公如能举大义者,妾请先死尊前,以成君子之志\footnote{王夫之:《永历实录》下}!

\end{quote}


言毕,她便回到居所写下血书藏于衣衫中,自刎而死。美人的香消玉殆使李成栋大受刺激。1648年四月十日,他毅然在广州宣布以全广东之地反清归明、剪辫易服,并胁迫佟养甲与他一同行事,广州城内的一千余名八旗兵丁则被尽数斩杀\footnote{司徒琳:《南明史》,页117}。正在广西流窜的永历帝大喜过望,决定迁回广东。六月十一日,在李成栋所派代表的迎接下,永历帝由南宁启程,于八月一日至肇庆,旋即封李成栋为“翊明大将军”、其养子李元胤为锦衣卫指挥使。李成栋驻广州,执掌朝中大政。然每有政事,李成栋必向永历帝“请旨后行”,绝不独断专行\footnote{《广东通史》古代下册,页735}。不久后,因李成栋察觉到佟养甲与清廷暗通款曲,便命李元胤率兵诛杀佟养甲,并处决其所有亲信\footnote{顾诚:《南明史》第十六章第二节}。

做为一名在历史大潮中身不由己的弱女子,在家邦有难时,张玉乔以性命促使李成栋反正抗清,展现了我南粤女性十足的刚烈与女德。在南粤悠久的历史上,每当南粤陷入危险时,常有伟大的女性站出来守护南粤,赵妪、冼夫人、张玉乔便是她们的杰出代表。她们的勇敢抗争表明,无论男女,南粤都绝不会有以妾妇之道事敌之人。这些英雄儿女持续不断的勇敢抗争,便是南粤文明不会灭亡的原因。

在李成栋反正前夕,永历帝已再一次陷入绝境。自1645年起,南明帝国便在湖湘境内统战了四十余万李自成流寇余部。这批流寇由李过、郝摇旗率领,御敌无能、残民有术。南明湖广总督何腾蛟为了供养他们,便在湖湘采取竭泽而渔的打土豪政策,在极短的时间内实现了“湖南民死亡过半”的“治绩”\footnote{刘仲敬:《经与史》}。1647年三月,清恭顺王孔有德率军攻陷长沙。同年秋,清兵陷湘西武冈,一度进至全州城下。永历帝时在武冈,慌忙从小路逃往桂林。当时,郝摇旗所率的流寇大军正云集于桂林。1648年二月,他们以何腾蛟欠饷为借口肆意抢杀,屠戮百姓数万人,迫使永历帝窜往南宁\footnote{参见顾诚:《南明史》第十六章}。因此,李成栋在同年四月的突然反正可谓救永历帝于水火。

李成栋反正后,何腾蛟指挥流寇军于湖湘发起反攻,连破衡州、宝庆、常德、湘潭,于十一月进围长沙,不克而退。虽然永历政权夺回了半个湖湘,但江右方面的战局却十分不利。当时,金声桓、王得仁虽已在南昌反正,但赣南的赣州守将高进库却拒不响应,仍向清廷效忠。在1648年三月至五月间,金、王二人统大军围攻赣州,终究无法破城。这时,清帝国南下的满蒙援兵已经逼近江右,二人只得退守南昌。自七月十日起,南昌陷入满洲将领谭泰、何洛会所部八旗兵的重围。八旗军驱赶数十万百姓挖掘深壕,彻底阻断南昌的对外联系,并肆意虐待民夫,造成了十余万人死亡的惨剧。金声桓、王得仁困守孤城,多次出击,均被击退。自九月末起,城中粮尽,陆续有军民零星地从城中逃出投降,皆被八旗兵残酷屠杀,南昌陷入绝境。1649年正月十八日,八旗军向南昌发起总攻,于十九日午后破城。金声桓投水自尽、王得仁被俘杀。两日后,清郑亲王济尔哈朗率八旗军突袭湘潭,杀何腾蛟,屠戮数万人\footnote{顾诚:《南明史》第十六章第一节}。轰动一时的金、王反正,便这样被清帝国残酷镇压。永历政权昙花一现的雄起,也随之化为泡影。

在金、王围攻赣州期间,若李成栋能迅速出兵合作,则战局当不会如此被动。然而,李成栋未对此有所重视,迟至八月才从广州率数万大军北上,骄傲自满、志在必得。九月下旬,李成栋军越过大庾岭。十月一日,进围赣州城,以四十门火炮射击彻夜。守城清军虽仅有五六千人,但他们见李军初来乍到、立足未稳,乃于二日凌晨冒险开城突击,迅速击破李军营垒。李军争相溃逃,死者万余。李成栋本人只得收束败兵,退回广州。此次进攻赣州的行动,就这样以惨败收场\footnote{顾诚:《南明史》第十六章第三节}。对解救江右危局来说,此次作战没有起到任何作用。

李成栋退回广州后进行短暂休整,于1649年正月再次北伐。二月下旬,李军第二次越过大庾岭,驻于信丰,意图先取赣州外围州县,再攻赣州坚城。然而,清赣州守军却采取“利在速战”之策,倾巢出动,于二月二十九日突然出现在信丰城外五六里处。李成栋出城迎战,败而回城。三月一日白天,清军从北、西、南三个方向攻城。李军军心不稳,于当日夜间纷纷涌出东门溃逃。当时,信丰城东的桃江正处在汛期,河水泛滥。李成栋本人在渡河时不慎落马,溺水而死,其部下将士纷纷逃命,一直退到大庾岭才稳住阵脚\footnote{顾诚:《南明史》第十六章第一节}。永历帝闻李成栋败死,只得命其副将杜永和代领其军、任两广总督\footnote{《广东通史》古代下册,页736}。

在南粤历史上,李成栋是一个极度矛盾的人物。他先是血腥镇压“广东三忠”起义的大屠夫,后来又在南粤女子张玉乔的死谏下改弦更张、成为为保卫南明帝国而战至最后一刻的人物。我们很难说最终忠于南明的他的抗清举动是为了保卫南粤,但从客观上来看,他的抗争确实延缓了清帝国吞并南粤的步伐。他的存在,本身就表明东亚大陆历史的诡异和无常。在他死后,流窜岭南的永历朝廷士大夫忙于党争,再也无心为保卫南粤做一点事。以湖北人袁彭年、丁时魁为首的“楚党”内结李元胤、外结驻守桂林的瞿式耜,与吴越人朱天麟、吴贞毓组成的“吴党”势不两立、互相攻击\footnote{顾诚:《南明史》第十九章第二节}。永历帝不敢得罪任何一方,只得整日斡旋于两者间,对军事部署几乎不闻不问\footnote{顾诚:《南明史》第十六章第一节}。杜永和则在繁华的广州城内花天酒地,对防务毫不上心\footnote{顾诚:《南明史》第十六章第二节}。在如此荒诞的情形下,清帝国侵略军的铁蹄已步步逼近。1649年五月,清廷以恭顺王孔有德为定南王、怀顺王耿仲明为靖南王、智顺王尚可喜为平南王,命三人率军南征两广,以孔有德攻广西,耿、尚攻广东。三人随即率部由北京南下,联合吴越、湖湘之兵,于十一月初分别进至湘、赣境内。这时,忽有满洲贵族告发耿仲明在军中窝藏“逃人”(由旗地逃出的农奴)一千余人。耿仲明大为惊恐,于十一月二十七日畏罪自杀。清摄政王多尔衮乃命耿仲明之子耿继茂统其父旧部,担任尚可喜的助手。十二月二十八日,尚可喜率二万精兵越过大庾岭,入侵南粤境内。三十日,正当南雄府城的军民欢庆除夕夜时,清帝国侵略军突然破城而入,将六千余名守军、二万余名居民屠戮殆尽。这场大屠杀,是超级屠夫尚可喜在南粤制造的第一起血案\footnote{顾诚:《南明史》第二十章第三节}。

1650年正月六日,在肇庆的永历帝听说南雄失守,立即登舟向西逃亡梧州。其朝臣也全无党争时的悍勇姿态,纷纷各寻出路、一哄而散。清帝国侵略军连陷韶州、英德、清远、从化,于二月初进至广州城下。广州三面环水、北有越秀山,系南粤的心脏,城防十分坚固。杜永和在越秀山顶五层高的镇海楼上设立指挥所,指挥守军布防。二月九日,尚可喜指挥侵略军持云梯猛攻北城,大败而退,遂不再强攻,而是在城东、北、西三面掘壕,开始围城。四月二十六日,尚可喜招降珠江河盗头目梁标相、刘龙胜、徐国隆,命这三名粤奸率船队封锁广州城外江面,与陆上侵略军遥相呼应。此外,尚可喜又派出使者招降南粤各地明军将领,惠州总兵黄应杰、潮州总兵郝尚久纷纷投降\footnote{顾诚:《南明史》第十六章第二节}。进入夏季后,持续的高温、湿热令来自岭北的清兵饱受疟疾之苦,无心作战,双方陷入漫长的对峙。然而,拥有优势兵力的杜永和却无心出战,只命部下以重炮与清军对轰\footnote{司徒琳:《南明史》,页131},自己则在镇海楼上“张宴设乐无虚日”,还嘲笑镇守西门的部将范承恩是“草包”\footnote{《广东通史》古代下册,页737}。在此期间,时年七十岁的何吾驺未曾逃离南粤。他组织起一支有船百余艘的军队沿西江而下,意图解广州之围,却不幸战败于三水,其本人被俘。这位可敬的老英雄饱受清兵酷刑折磨,仍不投降,最终牺牲于一间陋室内。直到1656年,他的遗骨才被亲人迎回家乡安葬\footnote{王文杰:《广州何吾驺“阁老之墓”》}。

由于杜永和毫不作为,何吾驺白白地牺牲了。入秋之后,尚可喜因广州城屡攻不下,又从江西调来一万援兵。这时,不堪杜永和侮辱的范承恩已与侵略者暗中约降。尚可喜得其内应,遂于十月二十八日挥军攻入西关。十一月二日,在侵略军重炮的猛轰下,广州城西北角被打出宽数丈的缺口,侵略军由此蜂拥而入。在短暂而激烈的巷战中,广州本地穆斯林出身的将领羽凤祺、马承祖、撒之浮三人率部顽强抵抗,尽数殉难,明军被杀6000余人。后来,三人被南粤穆斯林安葬于广州流花侨畔,人称“教门三忠”。杜永和则率残部乘船千余艘突围而出,逃至海南岛\footnote{司徒琳:《南明史》,页131}。攻陷广州后,尚可喜丧心病狂地下令屠城。自公元879年黄巢屠城以来的最大浩劫,降临在广州城中。通过意大利籍耶稣会士卫匡国(Martino Martini)在其历史著作《鞑靼战纪》中的记载,我们可以明白此次大屠杀是何等的残暴:

\begin{quote}

大屠杀从11月24日(按:农历十一月二日)一直进行到12月5日。他们不论男女老幼一律残酷地杀死,他们不说别的,只说:杀!杀死这些反叛的蛮子……最后,他们在12月6日发出布告,宣布封刀。除去攻城期间死掉的人以外,他们已经屠杀了十万人

\end{quote}

广州大屠杀过后,番禺县文人王鸣雷写下了一篇哀痛至极的祭文。其所描绘的屠杀场景,可谓一字一泪:

\begin{quote}
甲申(按:即1644年)更姓,七年讨殛。何辜生民,再遭六极。血溅天街,蝼蚁聚食。饥鸟啄肠,飞上城北。北风牛溲,堆积髑髅。或如宝塔,或如山邱。便房已朽,项门未枯。欲夺其妻,先杀其夫;男多于女,野火模糊。羸老就戮,少者为奴;老多于少,野火辘轳。五行共尽,无智无愚,无贵无贱,同为一区\footnote{转引自顾诚:《南明史》第二十章第三节}。

\end{quote}

在17世纪中期,广州维持着其近两千年来的繁荣,拥有无尽的财富和数十万生活富足的人口。然而,此次大屠杀却使之近乎毁于一旦。浩劫过后,尚可喜命令侵略军将尸体堆放在广州城东门外纵火焚烧。大火持续了数日,由此形成的巨大骨灰、尸骨堆使“行人于二三里外,望之如积雪”。直到19世纪,这一屈辱、残酷的遗迹仍然存在\footnote{魏斐德:《洪业:清朝开国史》,页716}。对此次大屠杀的遇难者数,各种史籍估计不一,最低者称有7万、最高者则认为有70万。然而,大多数历史记载都将死者数定为“十万”或“十数万”,可见这一数字当是比较可信的\footnote{《广东通史》古代下册,页737—738}。无论死者人数是7万、10万还是70万,这场被称为“庚寅之劫”的广州大屠杀都是清帝国对我南粤犯下的不可饶恕的罪行。它将尚可喜这一南粤史上空前凶残的屠夫永远钉上了历史的耻辱柱,并永远提醒粤人决不能忘记岭北帝国的凶残和罪恶。

在广州第二次陷落的四天后,桂林亦告失守。当时,听闻孔有德率领的清帝国侵略军已经逼近,桂林的数万守军早已抛下镇守当地的大学士瞿式耜、总督张同敞(张居正曾孙)逃散一空。瞿、张二人决意赴死,遂在堂中彻夜饮酒,于十一月六日清晨被冲入城中的清兵俘虏。二人坚拒孔有德劝降,于一个月后被杀于桂林城外叠彩山下\footnote{《广西通史》卷1,页413}。次年四月,镇守浔州的明庆国公陈邦傅降清。至十二月,清帝国侵略军攻陷宾州、南宁,永历帝逃亡至广西最西端的濑湍\footnote{《广西通史》卷1,页415}。年底,李元胤在钦州防城被俘。1652年一月,侵略军渡过琼州海峡,杜永和投降,海南岛陷落\footnote{《广东通史》古代下册,页738}。至此,南粤全境已彻底落入清帝国之手。李元胤听说连杜永和都已降清,便日夜请死,被耿继茂下令杀害\footnote{顾诚:《南明史》第二十章第四节}。然而,南粤的苦难还远远没有结束,两股新的侵略者已加入到争夺南粤的战争中,他们便是来自滇黔的大西军与来自闽南沿海的郑成功。


\section{三股侵略者蹂躏南粤与王兴、萧国龙抗清}

\indent 1652年初,正当永历帝在桂西朝不保夕之际,盘踞滇黔的大西军突然向其伸出了援手。早在1646年底,张献忠在蜀地制造了灭绝性的大屠杀后被清军射杀于川北西充凤凰山。其后,张献忠留下的十余万大西军余部在其养子孙可望、李定国等人的带领下南窜,于1648年完全占领滇黔。1649年,孙可望向肇庆派出使节,表示希望臣服于永历帝,要求得到“一字王”的封号。但永历政权中的“吴党”蔑视孙可望,只同意封其为“二字王”。双方往复遣使,虚耗了一年多的时间。其后,广州、桂林相继失守,永历帝失去讨价还价的筹码。孙可望乃于1652年二月派兵将永历帝迎至贵州安隆,自己则驻于贵阳,被封为“秦王”,得到了其朝思暮想的“一字王”头衔。朝中政务完全被孙可望控制,永历帝彻底沦为流寇的统战花瓶\footnote{参见顾诚:《南明史》第二十一章}。

掌控了永历帝后,孙可望即刻对清帝国发起反击,命李定国率部入湘作战。李定国系一悍将,战斗力超群,其进军势如破竹。1652年四月,李定国军进入湖湘,迅速控制除湘北外的大部分地域。六月,李定国率精兵、战象南下入桂,先克全州,又于七月四日攻占桂林,迫使守城的清定南王孔有德自焚而死。广西清军望风奔溃,全桂迅速落入李定国之手。李定国军活捉叛明的陈邦傅父子,将二人送往贵阳,以“剥皮实草”酷刑处决。广西易手的消息传至北京,顺治帝大为震动,忙命其堂兄敬谨亲王尼堪率八旗军南下迎击。十一月二十二日,八旗军在湘南衡州城下被李定国以伏兵之计击败,尼堪当场阵亡\footnote{顾诚:《南明史》第二十三章第一节}。这样,在仅仅半年多的时间里,明清战局便发生重大逆转。清廷十分惊慌,打算放弃湘、粤、桂、赣、川、滇、黔七省,与南明议和\footnote{《广东通史》古代下册,页740}。假如这一议和真能实现,那么此后摆脱了强大岭北帝国威胁的南粤虽然仍会被南明帝国和大西军控制,但很可能会走上一条相对独立自主的道路。然而,嫉贤妒能的孙可望却突然向李定国发难,于1653年二月召其至湘西沅州议事,意图借机杀害李定国。李定国识破其阴谋,率军退屯柳州。三月十七日,孙可望率军于湘西南之宝庆于清军会战,大败逃回贵阳\footnote{顾诚:《南明史》第二十三章第五节}。稍纵即逝的历史机会,遂因孙可望的破坏化为泡影。

1653年三月十四日,退至柳州的李定国发动东征广东的战役,试图在另一方向上打开局面。李定国的突然进攻使尚可喜、耿继茂所部清军猝不及防,连连败退。李军兵分两路,一路破开建、德庆,另一路破广宁、四会。至下旬,两路李军已会于肇庆城下。在粤东潮州,本已降清的潮州总兵郝尚久见李军势大,便于三月二十二日宣布“反正”,下令全城割辫附明。然而,惠州总兵黄应杰仍忠于满清,阻断了郝尚久的西进之路,使其无法与李定国会合\footnote{顾诚:《南明史》第二十五章第一节}。

三月二十六日,李定国开始挥军进攻肇庆城。清肇庆总兵许尔显顽强抵抗,以绳索缒精兵下城反击,夺得李军云梯百余架。李定国见强攻无果,便命将士挖地道透入城中。许尔显识破其计,在城内挖掘一条与城墙平行的深壕,双方在地道和壕沟中展开惨烈肉搏。四月初,就在双方僵持之际,尚可喜、耿继茂已率平南、靖南王府藩兵主力赶到肇庆。四月八日,两藩藩兵发动反攻,李定国大败退去,放弃已攻占的广东州县,回到柳州。李定国第一次进攻广东的战役,至此结束\footnote{顾诚:《南明史》第二十五章第一节}。

郝尚久知悉李定国兵败,急忙向活跃于闽南的郑成功求援。郝尚久系河南人,本是个毫无政治德性的投机军阀。郑成功身为闽人海上武装领袖,亦无爱于南粤,不愿全力支援,两人的合作从一开始便埋下了失败的种子。六月,郑成功率军猛攻澄海县鸥汀寨,意图破寨之后大肆劫掠一番。鸥汀是一座巨寨,呈长条形,连绵七八里,由澄海土豪建造。郑军入侵后,附近乡民纷纷涌入鸥汀寨寻求庇护。在乡民推举下,备受敬爱的本地秀才陈铁虎成为寨长,率民兵竭力防御。郑军百计攻寨,未能攻破,失利而去,郑成功本人亦中炮负伤。七月,郑成功率军驻于揭阳,满足于向附近结寨自保的各路粤东土豪勒索保护费,毫无救援潮州的举动。八月,在勒索到足够的粮饷后,郑成功军便浮海而去,返回厦门\footnote{顾诚:《南明史》第二十五章第一节}。至此,郝尚久被彻底抛弃,潮州沦为孤城。

这时,驻防南京的满洲八旗兵已在靖南将军哈哈木的率领下做为援军进入南粤。八月十三日,耿继茂、哈哈木率靖南王藩兵、八旗兵开始围攻潮州城,惨烈的攻防战持续了一个月之久。九月十四日夜,潮州城破,郝尚久及其子郝尧自尽。清兵入城后大施暴虐,将城中成年男子全部屠戮。在为期三天的屠杀中,死难军民达二万人以上。至于侥幸逃过屠杀的年轻女子,则有数千人被掳至岭北卖作娼妓、奴婢\footnote{潮州屠城的死难人数,有夸张记载认为达十万以上,然有学者经详实考证,认为此系夸大之辞。见贝文贤:《清兵屠潮州五考》}。清帝国侵略军之残暴,于此可见!

1654年四月,李定国经充分准备,动员起号称二十万的庞大军力,携大量战象、枪炮再次东征。五月,李军攻破粤西高州、雷州、廉州三府,夺取罗定、新兴、阳江、阳春、恩平等县。六月,李定国又派一支偏师渡过琼州海峡,攻下海南昌化、临高等县。尚可喜、耿继茂见李定国军力庞大,不敢应战,便集中兵力防守广州府。六月二十九日,李军主力开始围攻坚城新会。然而,李定国这时因患病留驻高州,无法亲临新会城下指挥,使前线士气低落,迟迟未有进展。为维持前线的消耗,李定国在粤西大肆搜刮民间财物,下令 “每米一石纳扉、履及铅、铁等物”,使当地百姓苦不堪言。十月三日,病愈的李定国终于出现在新会城下,指挥大军以挖掘地道、大炮轰城、伐木填濠等手段猛烈攻城,然仍不能得手。十一月十日,尚可喜、耿继茂率援军由广州赶至三水,但不敢与李军交锋,便屯于当地,等候满洲援兵南下\footnote{顾诚:《南明史》第二十五章第一节}。到十二月初,新会城中粮食已尽,守城清兵竟开始杀百姓为食。据《新会县志》载,当时城中的情形是:

\begin{quote}

围城之内,自五月防兵一至,悉处民舍,官给月粮,为其私有;日用供需,责之居停。贫民日设酒馔饷兵,办刍豆饷马,少不丰赡,鞭挞随之,仍以糗粮不给为辞,搜粟民家,子女玉帛,恣其卷掠。自是民皆绝食,掘鼠罗雀,食及浮萍草履。至腊月初,兵又略人为食,残骼委地,不啻万余。举人莫芝莲、贡生李龄昌、生员余浩、鲁鳌、李炅登等皆为砧上肉。知县黄之正莫敢谁何,抚膺大恸而已……盖自被围半载,饥死者半,杀食者半,子女被掠者半。天降丧乱,未有如是之惨者也\footnote{转引自顾诚:《南明史》第二十五章第一节}!

\end{quote}


由此段记载可知,新会守军在短时间内便杀食上万平民,制造了一派地狱图景。十二月十日,清靖南将军朱马喇所统满洲兵赶到城外,于四日后会同两藩兵发起反攻。至十八日,李军全线败退,新会之围遂解。浩劫过后,防守新会的食人恶魔dey 兵肥马壮,城中残民则形销骨立。两藩兵入城后又进行了一番劫掠,使残民陷入更悲惨的境地\footnote{顾诚:《南明史》第二十五章第一节}。然而,这还不是此役中粤人所受苦难的全部。李定国战败后,于二十四日退至高州、二十六日退回广西境内,一路上大肆掳掠人口、强迫男女老幼随军西撤,高州、雷州、廉州等地被其裹挟者竟达六七十万之巨\footnote{《广东通史》古代下册,页741}。在清帝国侵略军与流寇的交相杀掠下,南粤百姓几乎走投无路,在大洪水中承受着难以想像的苦难。李定国第二次进攻广东的战役,便这样血腥地结束了。至1655年,李定国率残军驻南宁,可战之兵仅余数千。次年春,在清军进攻下,李定国军迅速溃入滇黔,广西完全落入清帝国之手\footnote{《广西通史》卷1,页417—418}。

1656年,又一场大屠杀降临在南粤大地,其制造者为“国姓爷”郑成功。是年,曾率乡民打退郑军的陈君谔病逝。郑成功决定趁此机会发动报复,一举踏平鸥汀寨。他派大将甘辉率兵于澄海县登陆,发起第二次鸥汀寨之战。当时,内奸向甘辉献策,称鸥汀寨如蛇,只需破其中门,则此寨首尾必不能相顾。甘辉依计而行,迅速破寨。当时,寨中有六万余名澄海、海阳、揭阳等县的平民。他们大多是逃难而来的妇孺,坚信此寨能像上次一样保护他们躲过郑军的蹂躏。这一次,他们的希望破灭了。郑军大开杀戒,冷血地将这些已沦为瓮中之鳖的难民全部屠戮,之后浮海而去。次年,澄海僧人收集遇难者遗骨进行火化,竟得到骨灰三百余石\footnote{蔡绍彬:《“鸥汀惨案”漫议》}。

在清军、大西军、郑军三股侵略者的轮番蹂躏下,昔日繁华的南粤大地洪水肆虐,处处皆是蔽野的白骨、哀嚎的饥民。在如此逆境之中,我们的伟大祖先仍不放弃抵抗、从未放弃对自由与尊严的向往。在新宁县西南部依山傍海的文村,当地土豪王兴率众“熬海铸山,务农积粟”,庇护了许多南粤抗清义士。1656年春,尚可喜派兵三万进攻文村。在王兴的指挥下,当地军民依托山势坚守三个月,歼敌7000余人,打退侵略军进攻,取得保卫乡土的辉煌胜利。次年,文村爆发饥荒,尚可喜便卑鄙地趁人之危,于当年八月调集十万军队、民夫从水陆两路封锁文村。清军挖深壕、筑高垒,将文村围得水泄不通。至1658年夏秋之际,文村粮食耗尽,连一只老鼠的价格都已达到百文。为使跟随他的军民能够活下去,王兴决定牺牲自己。他对他的弟弟说出了如下一番话:

\begin{quote}

城可恃而食不支,天也。我终不降。弟善抚诸孤以续先祀,我死且不朽\footnote{转引自顾诚:《南明史》第三十章第四节}!

\end{quote}

言毕,他命五个儿子护送年迈老母前往清军大营向尚可喜投降,得到了尚可喜对其部下军民生命安全的保证。八月十七日夜,他宴请部下官员及抗清义士,宣布已与清方达成合议,诸人可各寻出路。随后,他先命妻妾自缢,接着点燃事先备好的火药,在烈焰中壮烈地杀身成仁。尚可喜遵守约定,未进一步加害投降的文村军民\footnote{顾诚:《南明史》第三十章第四节}。王兴用自己的生命为部下和家乡父老赢得了生存机会。这种壮烈豪迈、充满担当的举动,实为南粤武德的绝佳展现。

王兴占据文村时,曾命其部将萧国龙据守阳gong 县永丰寨。王兴殉难后,萧国龙率众降清,但在清军撤走后再举义旗。1661年八月,尚可喜命水师总兵张国勋进攻永丰寨。萧国龙掘壕固守,清兵筑长围以困之。九月二十四日,经近两个月的顽强防御,永丰寨陷落,义军牺牲1400余人,萧国龙之家属集体自焚,其本人投水殉难\footnote{《广东通史》古代下册,页743}。这样,南粤本土的抗清义军便被全部镇压。自1647年清军侵粤至此,南粤的抗清战争已进行了十五年。残暴的清帝国虽已屠杀了数以十万计的粤人,但它仍不满足,还要想方设法彻底打消南粤海上力量反抗的可能。1662年二月,一场史无前例的浩劫降临南粤,那便是极度野蛮残酷的“迁海令”。


\section{史无前例的惊天浩劫:沿海迁界}

\indent 自清平南王尚可喜、靖难王耿继茂入粤起,两藩便大肆屠杀、掠夺,疯狂残害南粤百姓。广州大屠杀后,二人又分别在城东、城西占地八十余亩,强征大量木料、民夫营建穷极奢华的藩府,并索取乳猪、翎毛等二十余项“贡物”\footnote{罗一星:《清初两藩踞粤的横征暴敛及对社会的影响》 }。他们还强迫高要县之民至山势险峻的羚羊峡采集石料,导致死者无数\footnote{《广东通史》古代下册,页752}。二人尚且如此,其部下的行径可想而知。在吴川县,县令高鸿飞巧取豪夺,以酷刑强迫百姓缴纳超额赋税,死于杖下者达六百人\footnote{陈舜:《乱离见闻录》卷下}。两藩藩兵在广州城内肆无忌惮地拆毁房舍,制造用以牧马的空地。他们还在广州、惠州、潮州、琼州等城随意霸占民宅,其中潮州被强占的民舍竟达一半。贪婪的藩兵又时常诬陷粤商为“盗”,任意勒索赎金。若有人拒交,藩兵便上门血洗其居所\footnote{罗一星:《清初两藩踞粤的横征暴敛及对社会的影响》}。在两藩的胡作非为下,刚刚经历浩劫的南粤无法休养生息。1660年,清廷将靖南王藩府移往福建,广东被尚可喜独踞。在杀人魔王尚可喜的盘剥下,我们的祖先继续承受着无尽的苦难。

早在1655年,清廷为防备沿海之民接济郑成功,便对沿海各地发布禁止商民船只出海的“禁海令”。1661年,清廷又进一步下达丧心病狂的“迁界令”,要求山东至广东沿海居民迁往内地。1662年二月,迁界令开始在南粤执行,广东东起饶平、西至钦州等二十四个州县的沿海地区及所有海岛(澳门除外)居民被强令内迁五十里。尚可喜严格执行清廷命令,要求沿海之民在三天内必须内迁。如到期时仍不迁走,就派兵一律屠杀。被划在界外的地域不但禁止平民通行,房屋亦要被全部捣毁,凡有越界耕种、捕鱼者则将被就地斩杀。至于划界方法,更是极度简单粗暴:

\begin{quote}

先画一界,而以绳直之。其间有一宅而半弃者,有一室而中断者。浚以深壕,别为内外。稍逾跬步,死即随之\footnote{转引自顾诚:《南明史》第三十一章第四节}。

\end{quote}

1664年三月,清廷又下达“再迁令”,不但要求南粤所有已迁州县再内迁三十里,更将珠三角最繁华的顺德、番禺、南海县及粤东海阳县的部分区域划为界外。迁界是由军队强制执行的措施。清兵在南粤沿海逐村驱民,强迫百姓立即出发。许多人不及收拾财物,只能携家人草草离开。南粤文明自古以来便有着发达的海上贸易,沿海地区人口密集、村镇棋布,许多人以出海捕鱼、贸易为生,乃南粤最富庶的区域。清帝国强迫他们离开世代居住的家园,等于几乎完全剥夺了他们的生存机会。数百万难民在从沿海涌入内地后生计无着,陷入了难以名状的惨境。据史籍记载:

\begin{quote}
(难民)养生无计,于是父子夫妻相弃,痛哭分携,斗粟一儿,百钱一女……其丁壮者去为兵,老弱者展转沟壑,或合家饮毒,或尽帑投河\footnote{转引自顾诚:《南明史》第三十一章第四节}。

\end{quote}

直到今天,我们仍能透过这简短的文字看到祖先们承受的巨大苦难。在清帝国侵略军的驱赶下,大群百姓妻离子散、家破人亡,化为死亡枕籍的难民,许多人走投无路,只有举家服毒、投河自尽。至于因失去生计而活活饿死的人,更是不计其数。就连朱元璋的海禁亦未曾制造如此规模的惨剧。清帝国及其走狗尚可喜对我南粤沿海居民犯下的滔天罪行,可以说是史无前例的!当时,香山县长连埔一带有万余平民不愿离开他们世居的故土,便躲入深山。尚藩左翼总兵班际盛随即带兵开入当地,声称百姓只要前往军营报名便可不迁。这些善良的人们信以为真,纷纷走出藏身之地进入军营,被全部杀害。在将南粤沿海百姓全部清空后,清兵便在界外以骑兵驰射,用“火箭焚其庐室”,大火累月不熄,数以千计的大小船只亦被付之一炬\footnote{罗一星:《清初两藩踞粤的横征暴敛及对社会的影响》}。南粤曾傲视世界的航海事业,至此遭遇空前挫折。其后,清帝国侵略军在界上修筑大量堡塞,驻以重兵,严密防范百姓越界。在筑塞过程中,被强征为民夫的平民饱受“拷掠鞭捶”,死者不知凡几\footnote{顾诚:《南明史》第三十一章第四节}。

被剥夺了家园的南粤沿海百姓自然不会全都束手待毙。他们中的许多人留在家乡,展开了正义的反抗。1663年十月十五日,番禺市桥疍民周玉、李荣聚起一批渔民扬帆出海,发动起义。周玉自称“恢粤将军”,传檄珠江口各地,分兵袭击番禺、新会、顺德、香山、东莞等地的清军,击毙清江门游击张可久。十七日,义军在石龙大破清广东提督杨遇明部,广州震动。珠江口各地的渔民群起响应,争相向义军提供军需。二十二日,义军船队驶至广州城下,进攻西关炮台不克,佯装败逃。清军水师尾追至缆尾,被义军反身还击,几乎全军覆没。义军焚敌炮舰24艘、毙敌1000余人。二十五日,义军攻克顺德县城,俘清知县王胤,大举赈济民众,百姓争相加入。三十日,周玉、李荣率义军船队在市桥大石江口与大队清军水师决战。义军先胜一阵,毙敌数百人。周玉乃产生轻敌心理,挥军轻进,结果被敌舰分割包围,其本人不幸被俘就义。李荣率残部突围,撤往外洋。此战,义军丧舰131艘、阵亡2600余人,实力大损。决战过后,义军退守珠江入海处之东涌岛休整,实力逐渐恢复百余艘船只。1665年,尚可喜发大兵“进剿”东涌,义军惨败,李荣在海上壮烈战死\footnote{《广东通史》古代下册,页756—757}。珠江口渔民壮烈的反抗,就这样被尚可喜残酷地镇压了。

1664年,惠州碣石卫又发生了规模浩大的苏利起义。苏利系当地土豪,在明末清初大洪水中结寨自保,拥有战舰上千艘,曾被南明永历政权授以总兵官。尚可喜攻陷广州后,他为保全家乡父老安全而降清,仍任总兵,其势力未被撼动。1664年五月,尚可喜勒令苏利立即率众迁入界内。为了家乡父老的生计,苏利宁死不从,毅然举兵反抗,歼灭当地清军守界兵,一连攻克多座县城,收复了潮州府、惠州府间的大片土地。尚可喜命侵粤清军倾巢出动,猖狂进犯。七月,义军在牛塘、官田、梅垅等地失利,损失2000余人,侵略军逼近碣石卫。八月十二日,苏利亲率义军主力一万人出城,与敌会战于南塘浦。双方的血战自晨至午,两军尸体数以千计,横铺四十里。最终,兵力不占优势的义军战败,余部退守碣石城。侵略军随即破城,苏利壮烈殉难,倒在侵略者屠刀下的起义军民达6000以上\footnote{《广东通史》古代下册,页758}。

南粤的沿海居民被剥夺了出海的权利,而尚可喜却借机大发横财。他在广州大修明市舶司故址,任用一批走狗担任“王商”垄断外贸。每年出海的“王商”商船达上千艘,获利四五十万两。这些钱财完全未能惠及南粤百姓,皆被尚可喜及其走狗装进了腰包。在粤人的累累尸骨上,贪婪无耻的尚可喜成为巨富,甚至还时常向资金不足的西方商人放贷\footnote{罗一星:《清初两藩踞粤的横征暴敛及对社会的影响》}。

野蛮残酷的迁界使广东境内近三分之一的田土(465万亩)被抛荒,清帝国每年因此损失田赋31万两。1668年九月,清广东巡抚王来任在病危之际“上疏”康熙帝,请求复界。当时,康熙帝刚刚在宫廷斗争中击败鳌拜集团,有意振作一番。1669年二月,清廷下令准许广东在1664年的迁界地域复界。然而,回迁人口只有16万,百姓仍不得出海,此道复界令未起到多大作用。同年,已赶走荷兰人并占领台湾的郑氏政权遣其将邱辉率部登陆粤东,于次年占据界外之达濠寨。这股郑军长期流劫于海阳、揭阳、澄海、惠来、普宁等地,掳走数万人口,将男子精壮者充作奴隶、女子貌美者送往台湾、貌陋者廉价贩卖、老病者就地屠杀,使沿海百姓的苦难雪上加霜\footnote{陈春声:《从“倭乱”到“迁海”——明末清初潮州地方动乱与乡村社会变迁》}。1673年,清平西王吴三桂据云南而叛,宣布恢复明式衣冠,建立吴周政权,著名的“三藩之乱”拉开序幕。次年二月,清广西将军、原定南王孔有德之婿孙延龄据桂林叛。三月,靖南王耿精忠据福建叛。康熙帝连忙安抚尚可喜,于1675年正月晋封其为平南亲王。尚可喜因年事已高,欲以次子尚之孝继承王位。其长子尚之信深为不满,乃与吴三桂勾结,于1676年二月发动兵变幽禁尚可喜,举广东之地反清。尚可喜忧惧交加,一命呜呼,时年七十三岁,结束了他罪恶滔天的人生。同月,郑军宿将刘国轩率部攻陷惠州,与尚之信互为犄角\footnote{《广东通史》古代下册,页768}。

尚之信继承其父的暴虐性格,也是个残忍嗜杀的恶棍。他终日饮酒,每醉必手刃旁人,被他亲手杀害的部下和百姓不计其数,就连许多近侍也遭遇毒手。此外,他又极度懦弱。1676年五月,清军攻入闽越,耿精忠在福州投降。尚之信大惊失色,乃于1677年五月宣布降清。康熙帝见其仍有利用价值,便命他仍“袭封平南亲王”,照常办事。同年七月,吴军以数万人从湘南入侵南粤,进攻韶州。康熙帝催促尚之信出兵,但尚之信却以需防范惠州的刘国轩为由拒绝。不久后,刘国轩率军浮海而去,但尚之信仍蜗居广州,不肯派出一兵一卒,康熙帝只得命满洲将领莽依图率部驰援韶州,击走吴周军。年底,吴三桂“悉其精锐俱向广西”,杀死有意降清的孙延龄,于1678年二月在荔浦大破清军,占据除梧州之外的广西全境,并攻入粤西。在康熙帝的一再催促下,尚之信不得不于四月十一日率军西征,在电白获胜一场,夺回高州。八月,吴三桂在衡州病死,其孙吴世璠继位。吴周政权人心不稳,颓势毕现。察觉到风向转变的尚之信突然变得积极起来,主动“奏请”进攻广西,被清廷任命为“奋武大将军”。1679年二月,尚之信率藩兵自封川攻入广西,进至衡州。这时,吴周大势已去,其灭亡只是时间问题。尚之信便将前线指挥权交给部下,自回广州,继续过着骄奢淫逸、残暴无耻的腐朽生活。至十月,广西境内的吴周军被清军全部扫除\footnote{关于“三藩之乱”中南粤的情况,参见刘凤云:《清代三藩研究》,页276—278}。次年,长期盘踞达濠寨的邱辉被清军击走。南粤境内的长期战争,至此终于告一段落。

在1676年起兵反清时,尚之信曾假仁假义地宣布“开界”,令百姓可以在沿海地区随意居住、出海谋生。降清之后,他又撕下伪装,于1679年派兵驱民入界。当时,沿海地区“开界”仅有三年,百姓正居住在茅草屋中艰难地开垦荒田。藩兵所到之处,这些好不容易安顿下来的人们逼迫再度流离失所。他们的茅屋都被焚烧、庄稼尽被割走,还被强迫缴纳当年的赋税\footnote{《广东通史》古代下册,页760}。在尚之信的倒行逆施下,南粤又一次出现了难民遍地、哀鸿遍野的地狱景象。

1680年,由于吴周军已完全退出南粤,尚之信失去了被清帝国利用的价值。是年三月,平南亲王藩下都统王国栋、广东巡抚金俊向清廷告发尚之信滥杀无辜的罪行。康熙帝顺水推舟,“下诏”逮捕尚之信。八月,尚之信暗中指使其弟尚之节、心腹李天植等人谋杀王国栋,密谋再次发动兵变。清军立即出击,冲入藩府,逮捕尚之节、李天植等108人,将其全部处死,尚之信则被康熙帝勒令自尽。1681年冬,清军攻陷昆明,吴世璠自尽,“三藩之乱”结束。次年,清廷下令将被尚藩占据的田产、房舍“一一给还于民”\footnote{《广东通史》古代下册,页769}。,南粤历史上空前残暴的尚氏家族,至此烟消云散。

1683年八月,台湾郑氏降清,清帝国“迁界”的理由不复存在。十月二十八日,清廷废止“迁界令”。次年正月二日,南粤境内贴出了第一张通知百姓正式“复界”的告示,渔民出海之禁亦被解除。此次复界共复农田316万亩,相当于迁界损失田土的约四分之三\footnote{《广东通史》古代下册,页761}。历时二十三年、残酷至极的迁界浩劫,至此终于落下帷幕。这一暴行给南粤带来的是永远无法抹去的伤害。许多人倒毙在异乡,再也没能等到“复界”的那一刻。幸存下来的人们回到家乡,看到的也只是一片残破的废墟。据最保守的估计,在执行“迁界令”最严的南粤、闽越,因“迁界”而死者分别达到90万和180万\footnote{葛剑雄、曹树基:《中国人口史》第五卷}。至于南粤航海事业所受到的打击,其严重性和毁灭性更是无法估量。

无论如何,南粤的抗清战争结束了,明末清初大洪水在南粤退潮了。南粤人将牢记清帝国及各路侵略者犯下的罪行,在遥远的将来完成复仇大业。此刻的南粤人正需要重整河山,恢复生产和贸易,在废墟之中复兴伟大的南粤。随着18世纪的到来,南粤文明开始步入最成熟的阶段,向今天的模样加速演化。






