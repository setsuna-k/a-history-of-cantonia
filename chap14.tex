\chapter{辉煌文明与伟大机遇:盛清帝国治下的南粤}

\section{18世纪的奇迹之城:广州与十三行}

\indent 公元1685年,即清帝国废止“迁界令”后两年,康熙帝下令开海贸易,在广州、泉州、宁波、松江设粤、闽、浙、江四海关,自1655年以来断绝了三十年的南粤海洋贸易至此恢复。粤海关衙门设于广州城五仙门内,有专职监督一员,不受总督、巡抚节制,直接向皇帝、户部负责。粤海关下辖省城(广州)大关、澳门总口、潮州庵埠总口、惠州乌坝总口、高州梅菉总口、琼州海口总口、雷州海安总口共七处总关口,各总口又管辖小口岸七十处,其中省城大关、澳门总口最重要,直属省城大关的虎门口、黄埔口又为重中之重。各关口的职能为向来粤贸易的外商征收税。清帝国设立粤海关的举动系岭北帝国盘剥南粤外贸制度的一次重大变革。自763年唐帝国设市舶司以来,愚昧的岭北帝国皆视来粤贸易的外商为对其“朝贡”者。当时,来粤外商需将货物交予市舶司,在市舶司辖地临时招商发卖。粤海关成立后,外商来粤贸易被岭北帝国视为正常的贸易往来,帝国在理论上不再负责管理贸易过程\footnote{中荔:《十三行》,页13—15}。这一变化使南粤外贸的自由度有所提升,对南粤是有一定好处的。

但必须指明的是,清帝国的粤海关仍对粤人与西方国家的贸易进行着疯狂的盘剥。从这一点来看,粤海关与唐、宋、元、明的市舶司并无不同。粤海关的关税分为船钞、货税两项。1698年,清帝国颁布了西洋、东洋商船的船钞额度,将外商的“洋船”按大小分为四等,每艘分别征收1400两、1100两、600两、400两钞银。此外,每艘出海粤船亦要上缴船钞,其额度为以船之长宽相乘、每平方丈征钞银9两。货税则按种类计量征收,分衣物税、食物税、用物税、杂货税四种。无论进口、出口货物,税额皆分别为每件8分、每百斤2钱5分、每百斤5钱(貂皮、虎皮、豹皮每张各一钱)、每百斤2两\footnote{《广东通史》古代下册,页808—809}。

粤海关对南粤外贸的盘剥绝不仅仅来自船钞、货税两项。以粤海关监督为首的海关官员多为极度贪婪之辈,他们不断以种种名义对来粤外商和南粤商人进行疯狂的敲诈勒索。最为常见的勒索方式乃以皇帝大寿、治理黄河、赈灾等稀奇古怪的理由加征各种临时摊派。例如,在1801年一年内加征的税项便有294种之多。此外,商人还不断被迫上缴所谓的“规礼”。这些“规礼”并无定额,全凭海关官员的心情索要。仅1736年一年,粤海关就盘剥了10余万两的“规礼”,远远超过粤海关每年定额为43564两的关税“正额”\footnote{冷东、金峰、肖楚熊:《十三行与岭南社会变迁》,页89—90}。曾于19世纪晚期在清帝国税务机关工作过的美国人马士在其名著《中华帝国对外关系史》中曾ni 样描述粤海关监督一职:

\begin{quote}

典型的,也是最肥的关务官职就是广州的粤海关监督,掌管广东沿海各口和珠江三角洲的航运及征税事宜。自满洲人在广东建立他们的统治后,就用满洲人充任广州海关监督,以便榨取帝国中最富饶市场的商业……粤海关监督为了保官,每年送往北京的银两不下一百万两\footnote{转引自冷东、金峰、肖楚熊:《十三行与岭南社会变迁》,页89}。

\end{quote}

由此段记载可知,粤海关每年由法外盘剥得来的钱财数十倍于关税“正额”。这些财富有一部分落入了粤海关大小官员的腰包,还有许多被清帝国宫廷、国库拿走。清帝国视粤海关为“天子南库”,粤海关监督大多经清帝钦点,由内务府包衣担任,实为受满洲皇帝宠信的私人\footnote{冷东、金峰、肖楚熊:《十三行与岭南社会变迁》,页89}。海关取代市舶司产生的正面效果,近乎被上至清帝、下至海关小吏的疯狂盘剥抵消。

除疯狂盘剥外,清帝国又对粤人的外贸活动进行种种限制。1686年四月,清两广总督吴兴祚、广东巡抚李士祯、粤海关监督宜尔格图经共同商议后决定建立专营外贸的洋货行。这些洋货行的名称沿用明帝国在1556年指定的对葡贸易商行之名,统称“十三行”。十三行商人又被称为“行商”、“洋商”。据清帝国规定,外商在南粤的贸易对象只能是行商,不能与其他粤商发生任何联系\footnote{《广东通史》古代下册,页810}。十三行行商随即珠江北岸(今十三行路以南广州文化公园一带)建起被称为“夷馆”的十三座商馆用以堆放货物、办理交易,并租给来粤外商居住。以下列表展示这些商馆的名字:

\begin{center}
	\begin{tabular}{cc}
		\hline 
		英文名 & 译名 \\
		
		\hline
		
		Danish Factory & 得兴行 \\
		
		Spanish Factory & 大吕宋行 \\
		
		French Factory & 高公行 \\
		
		Chungua Hong & 东生行 \\
		
		American Factory & 广源行 \\
		
		Paou-shun Factory & 宝顺行 \\
		
		Imperial Factory &  孖鹰行 \\
		
		Swedish Factory & 瑞行 \\
		
		Chow-Chow Factory & 丰泰行 \\
		
		New English Factory & 保和行 \\
		
		Dutch Factory & 集义行 \\
		
		Creek Factory & 怡和行
		
	\end{tabular}
\end{center}

所有商馆的格局都几乎一样,其样式为:

\begin{quote}

它(商馆)拥有一个独立的庭院,进门便有一大照壁,挡住从外面投入的视线,这自是“中国”传统格局……绕过照壁可以看到,夷馆里边左右为两栋平行的两层建筑,两楼之间则为长条石板铺出来的方形院落,有自己的排水系统。走过院子,正面当是主楼,造得很是讲究,外墙用的是青砖、龙骨砖,后者是砖中有木棍做有芯子,当为岭南俗称的空心龙骨砖。屋顶上的瓦片,与西洋的相差无几。室内是一色的木板地。令外商叹为观止的是,楼上每每面向江面伸出一个不小的阳台,阳台下方的石柱更直接打入了水面\footnote{谭元亨:《广州十三行——明清300年艰难曲折的外贸之路》,页43}。

\end{quote}


由此可见,夷馆系粤西合璧式建筑。外商入住后,各夷馆皆竖有其本国国旗。据18世纪中期史料记载,住满外商的夷馆着实蔚为壮观:

\begin{quote}
(夷人)在十三行列屋而居,巵楼相望,明树番旗,十字飘扬,一望眩目\footnote{《广东通史》古代下册,页812}。

\end{quote}

凡欲担任十三行行商者必须“身家殷实”,并花费3万—20万两白银向清帝国户部购买“部帖”(营业执照)。“十三行”之名仅为虚指,并不代表行商实数。开海贸易之初,行商仅有数家。随着外贸的发展,至1720年,行商数已增至16家。是年十一月二十六日,16家行商在神前盟誓,宣布成立排他性的商人行会“公行”,规定除扇、漆器、刺绣、图画等零星商品外,所有外贸货物必须由公行参与“共同议价”。行商此种自我组织控制议价权的行为引起了清帝国的恐慌。次年七月,公行在清广东当局的压力下暂停活动\footnote{《广东通史》古代下册,页813—814}。此后,公行的活动时断时续,一直勉强维持至1842年。而行商数则在此期间随着外贸的繁荣继续增长,一度多达20余家\footnote{中荔:《十三行》,页41}。

除遏制广州行商组织公行外,清帝国还在开海后一度试图限制粤人出海。1717年,晚年康熙帝因担心粤人、闽人在菲律宾、巴达维亚等处建立反清基地,竟卑鄙地重颁海禁之令,禁止清帝国治下之民前往南洋贸易。但对于南洋商船至南粤、闽越、吴越贸易之举,康熙帝并未禁止\footnote{《广东通史》古代下册,页817}。当时,东南三越沿海居民“望海谋生者十居五六”,南洋海禁使他们陷入谋生无门的困境。1723年,新继位的雍正帝只得默许南粤解除海禁。四年后,闽吴的海禁亦被解除\footnote{《广东通史》古代下册,页1009}。然而,清广东巡抚杨文乾于1726年悍然宣布实行所谓的“加一征收”制度,无理地要求外商将进口现银的10\%做为附加税上缴,引发外商极大愤慨。在屡次通过行商向粤海关请愿无效后,十一名英法商人于1728年9月16日持剑强行冲破卫兵岗哨、突进广州城门,一路未受任何阻拦,直入总督衙门内院(今石室圣心教堂所在地),当面向清两广总督孔毓珣抗议。孔毓珣虽十分惊慌,却依然摆出一副傲慢自大的嘴脸,未作答复便将这些外商遣送出城,随即要求行商代外商缴纳附加税。行商商总谭康泰(顺德人,西方人称之为“谭康官”)据理力争、拒绝缴纳,一度被投入监狱\footnote{谭元亨:《广州十三行——明清300年艰难曲折的外贸之路》,页94}。此次事件中,戒备森严的两广总督衙门竟被十一名外商轻易突入,事后清帝国亦不敢得罪外商,只敢拿行商出气。清帝国在西方国家面前的外强中干、色厉内荏,可谓暴露无遗。1735年,新上台的乾隆帝下令废除“加一征收”。至此,这场闹剧般的风波总算告一段落\footnote{谭元亨:《广州十三行——明清300年艰难曲折的外贸之路》,页99}。

这一波折并未对南粤与西方国家间的贸易产生太大影响。自17世纪末起,西方各国商船陆续来到广州,与南粤建起了正常的贸易往来渠道。以下列表反映西方各国开启对粤贸易的情形\footnote{此表之作,参考谭元亨:《广州十三行——明清300年艰难曲折的外贸之路》,页44}:

\begin{quote}
	英国:1689年,英国商船“防卫号”到达澳门,但未能获准进入广州贸易。1699年,英国商船“麦士菲里德号”抵达广州黄埔港,正式开启粤英贸易。1714年,英国东印度公司在广州设立商馆。
	
	法国:1698年,第一艘法国商船到达广州。次年,法国在广州设立商馆。
	
	荷兰:1657年,清廷允许荷兰“八年一贡”,不准粤荷自由贸易。1727年,荷兰获准在广州设立商馆。
	
	丹麦:1731年,丹麦在广州设立商馆。
	
	瑞典:1732年,瑞典在广州设立商馆。
	
	美国:1784年,美国商船“中国皇后号”首航广州,粤美贸易开始。1786年,美国首任驻广州领事山茂召(Samuel Shaw)在广州建立商馆。
\end{quote}

在18世纪,除上述国家外,还有普鲁士、西班牙、奥斯坦德(比利时西部城市)、亚美尼亚商船来粤贸易。在雍正时期(1723—1735年),至广州贸易的欧洲商船有107艘(其中62艘为英国船)。乾隆帝统治的头六年(1736—1741年),这一数字为76艘。1750—1756年间,更有154艘西方商船至广州贸易。由是可知,来粤西方商人的数量是在稳步增长的。1754年,清廷确立十三行的“保商制度”,要求行商完全负责对外商征税、为朝廷搜罗珍奇的业务。若有外商“违法”,行商要负连带责任\footnote{谭元亨:《广州十三行——明清300年艰难曲折的外贸之路》,页116—117}。此举巩固了粤海关通过行商控制外商的体制。由于粤海关官吏异常贪婪、时常敲诈外商,西方商人极度不满。1757年十一月,乾隆帝突然命令关闭闽、浙、江三海关,要求除菲律宾之西班牙商船允许赴厦门贸易外,所有外商均只能到广州贸易。其时,英国东印度公司正希望拓展在吴越的丝、茶贸易,清帝国的倒行逆施无疑阻断了吴英间的自由贸易。1759年,英商洪任辉(James Flint)受公司委派前往宁波交涉,被拒绝入港。洪任辉随后直上天津,向清直隶当局递交呈词控告粤海关的种种贪贿行为。乾隆帝命将洪任辉押回广东质讯,经调查后发现其所说情况属实,便一面命将粤海关监督李永标革职抄家,一面又下令将洪任辉押至澳门“圈禁”三年、刑满后驱逐回国,并将替洪任辉撰写汉字呈词的蜀人刘亚匾残酷处死,广州做为清帝国唯一对外口岸的“一口通商”时代由是开始\footnote{《广东通史》古代下册,页1022}。清帝国的此种行为完全避重就轻,因为仅惩处一贪官根本无法改变粤海关的贪婪。至于其无理关押洪任辉、处死刘亚匾的行径更是无理、残忍至极,充分展示了清帝国的暴虐本质。

自17世纪末起,广州十三行的对外贸易保障了南粤与西方国家间规模可观的经济、文化交流。巨量的茶叶、瓷器、广绣出口带动了南粤相关产业的飞速发展。在18世纪,盛产瓷器的江右景德镇已与粤商结成了紧密的合作关系。每年,赴赣购瓷的粤商都会带着大批货物溯赣江而上直抵赣州,再翻越大庾岭至南雄,顺北江而下到达广州。其中,有的瓷器在运抵广州后再经工匠以西式“金胎烧珐琅”加工,变为精美至极的仿欧瓷器“广彩瓷”。这种外销瓷有着精美的西式图案,并吸收了欧洲、阿拉伯国家制瓷、珐琅和玻璃器皿艺术的特点,深受西方人喜爱。经广州出口的货品货品极大地改变了欧洲的社会风尚。上至法国、普鲁士王室、下至一般平民,整个欧陆都为广州外销瓷而痴狂。法国国王路易十四曾在17世纪末以重金构筑被称为“瓷宫”的特里阿农宫,收藏大批青花瓷、五彩瓶。在18—19世纪的英国,以青花瓷茶具饮茶成为绅士和淑女的时髦风尚\footnote{冷东、金峰、肖楚熊:《十三行与岭南社会变迁》,页19—35}。另一方面,西式钟表、鼻烟、银器、花旗参、水獭皮亦大量涌入18—19世纪的南粤,成为粤人家居中的常见用品。十三行更从瑞士、法国引进了不少钟表工匠,极大地推动了粤西间的技术交流。此外,南粤的饮食习俗亦受到了西方世界的巨大影响\footnote{冷东、金峰、肖楚熊:《十三行与岭南社会变迁》,页44—55}。在款待外商的宴会中,行商吸收西餐的烹饪方法和餐具摆放模式,在开餐前于每人面前放置一套碗筷齐全的餐具。这一饮食习惯如今已完全本地化为南粤习俗,并深刻影响了东亚诸邦。在18世纪中期,十三行一带开设了一间名为“碧堂”的大型西餐馆。该餐馆提供正宗的西式肉排、包点和酒水,使粤人品尝到了地道的西方美味。西方人的饭后甜品西米露更被粤人完全接纳,成为地道的粤式糖水\footnote{冷东、金峰、肖楚熊:《十三行与岭南社会变迁》,页200}。

1984年底,瑞典政府在该国哥德堡港外900米处海面下对沉船“哥德堡号”进行了全面的考古发掘。“哥德堡号”系18世纪瑞典的一艘商船,曾于1739—1744年间三次前往广州。1745年9月12日,返航途中的该船在距故乡不足一公里处触礁沉没。这场发掘所打捞出的货物数量使人震惊:沉船中装载着由广州十三行行商代购的700吨货物,包括350吨茶叶、重达100吨的50—70万件瓷器以及大批丝绸、藤器、珍珠母。这批货物若能按当时的价格在哥德堡的市场上拍卖,将值200—250万西班牙银元\footnote{何龙宁:《“哥德堡号”沉船与广州十三行研究》}。仅仅在一艘瑞典商船上便发现了如此巨量的货物,18世纪广州与西方国家间贸易量之巨可见一斑。如此巨量的珍宝自然引起了清帝国皇室的贪欲。自1754年成立“保商制度”以来,行商每年都要向清宫输送四次被称为“采办官物”的紫檀、象牙、鼻烟、钟表、仪器、玻璃器、金银器、毛织品、宠物等西方货物。此外,行商每年还要以“备贡”之名向清宫内务府造办处上缴5.5万两白银,为皇室购买“贡品”提供经费\footnote{中荔:《十三行》,页51—52}。此外,种种法外加派更hay 时而有之,将许多行商逼到家破人亡,以下仅举数例:1795年,行商石中和因对外商欠下数十万两债款,被清廷押往西域伊犁充军,其债务由家属分六年设法偿清;次年,十三行总商蔡世文更因欠银50万两而吞鸦片自杀;1809年,行商沐士方因欠款35万两惨遭抄家,被押往伊犁充军…\footnote{中荔:《十三行》,页91—92}…在清帝国敲骨吸髓的掠夺下,许多行商都对外商欠下了巨额债务。但清帝国非但不采取任何补救措施,反而动辄将之抄家、充军乃至杀害。清帝国对粤商犯下的罪行,可谓罄竹难书!

在如此艰难的环境下,仍有行商坚持经营并获得成功。1757年“一口通商”后,广州的对外贸易额直线上升。潘氏和伍氏在行商中脱颖而出、成为巨富。潘氏原为闽人,于18世纪中期移居广州经营同文行。经潘振承、潘启祥、潘正炜(西方人称之为“潘启官”)祖孙三代极力经营,潘氏在1830年代成为极其富裕的家族。潘氏由八九英尺高的围墙保护的园林中遍布珍禽异兽、奇花异果,曾使到访的美国商人大为惊叹。至于生活在1769—1843年间的怡和行行商伍秉鉴(西方人称之为“伍浩官”)更是行商首富,亦很有可能是19世纪前期的世界首富。他坐拥2600万两白银的家财,不但在清帝国境内拥有大量地产、房产、茶山、店铺,更在美国投资铁路、证券交易和保险业务。据说,他有一次曾当着美国商人的面撕碎了一张价值7.2万两白银的欠条,表示自己毫不在乎这一点钱。在19世纪前期的美洲,伍秉鉴之名无人不晓,美国人甚至曾将一艘商船命名为“浩官”\footnote{中荔:《十三行》,页88—89}。因发达的对外贸易,广州在18世纪成长为一座巨大而富庶的都市。据英商汉密尔顿在18世纪前期的记载,当时广州的情形是:

\begin{quote}
	广州是一个繁荣的市场,城内人口住有90万,在近郊的还有上述人口的1/3。一年中每天城市前面河上(按:即珠江)的船艇,除小艇外,经常见有5000艘的贸易帆船\footnote{《广东通史》古代下册,页1002}。
	
\end{quote}

可见,18世纪前期广州城的人口已达120万之巨。由于此后广州外贸额度一直在不断上涨,19世纪前期的广州人口应多于此数。可以说,18世纪广州的人口规模已经超越了北京,成为与日本江户并列的世界最大城市。至于广州城的国际色彩,更是远远超过锁国中的江户。1793年,著名的英国马戛尔尼(Geor的Marcartney)使团出使清帝国,途径广州。其副使乔治·斯当东(George Staunton)如是记载他所看到的广州城:

\begin{quote}
作为一个海港和边境重镇的广州,显然有很多“华”洋杂处的特色。欧洲各国在城外江边建立了一排他们的洋行,华丽的西式建筑上面悬挂着各国国旗,同对面“中国”建筑相映,增添了许多特殊风趣。货船到港的时候,这一带外国人熙熙攘攘,各穿着不同服装,操着不同语言,表面上使人看不出这块地方究竟是属于哪个国家的\footnote{转引自中荔:《十三行》,页27}。

\end{quote}

由此可知,在18世纪,当岭北愚昧的人们还在对世界大势一无所知时,广州人已对西方人见怪不怪了。值得注意的是,清帝国并不允许外商进入广州城墙内,粤西贸易只能在西关十三行街一带进行。南粤及清帝国境内各地的富商大贾遂争相在西关购置房产,著名的“西关大屋”就是这一时期最为典型的建筑\footnote{中荔:《十三行》,页49}。在西关下九甫一带,丝绸织造厂林立,三四万名织工在大规模生产由珠三角蚕丝和吴越生丝混合而成的“广纱”、“广缎”,其质地“金陵、苏杭皆不及”\footnote{曾新:《明清广州城及方志图研究》,页108}。在十三行以南的江心洲沙面,娱乐场所鳞次栉比,在陆上者称“花林”、在水中者称“花船”,“妓船鳞集以千数”,每日歌舞升平、弦歌妙曼,饮食“珍错毕备,一宴百金”,商贾和市民通宵达旦地游乐其间,其繁华景象令人咋舌\footnote{黄佛颐:《广州城坊志》,页635}。在东关沿江河滩处,珠江和东壕涌的河道边码头林立,商贾货物川流不息,周围乡民大都在此出售农产品\footnote{曾新:《明清广州城及方志图研究》,页109}。进入19世纪后,广州城区进一步扩张到珠江南岸被称为“河南”的巨大江心洲(今海珠区)北缘,经西江、北江、东江运来的各地土产大都在此集散,行商则在此开设成片厂房以加工外销瓷\footnote{黄佛颐:《广州城坊志》,页635}。

在城墙内,繁华亦不输墙外。在城西一带,东起四牌楼(今解放中路)、西至城墙、南至大德街、北至光塔街之间的区域为清帝国驻防八旗兵居住的“满城”。1682年,清廷设驻防广州将军一员,统汉八旗驻防兵3000人。1756年,清廷又裁汰广州的半数汉八旗兵,代之以从北京携眷南下的1500名满八旗兵\footnote{关溪莹、翟麦玲:《清代满族八旗兵驻防广州缘由探析》}。这批人数稀少的驻防军被上百万粤人完全包围,根本无法撼动广州的主体文化。当这些驻防军初入广州时,他们狂妄地以外来征服者自居,时常骚扰百姓,因而饱受粤人歧视,被轻蔑地称为“旗下佬”。然而在数代人之后,他们渐渐融入了绝对强势的本土粤文化,变为讲粤语、生活饮食习俗几乎完全粤化的广州世居满人。今天的世居满人大都以广州人自居,除拥有独特的祖先认同和过年习俗外几乎与粤人毫无二致,亦被人视为广州人,成为南粤社会一块奇妙的拼图\footnote{伍嘉祥:《远去铁骑的留痕——满族人在广州》}。在城中央的双门底大街(今北京路)与惠爱街(今中山四、五、六路)一带坐落着清帝国的各类衙门及祭祀着南汉高祖的城隍庙,大批私立书院亦集中于此。由于帝国官僚、文人墨客多聚于此,这一带有上百座刻书、卖书的书坊,古玩贸易也很兴盛。至于南北向的大道双门底大街更是热闹非凡。自第四次北属时代以来,此街即是广州城的中轴线。每年九月,街上都会举办大型庙会。庙会中,街边搭满高数丈的蔑棚,演剧“观者如堵”,“市门各悬傀儡,造制奇巧,锦绣眩目”,种种珍奇商品琳琅满目,灯火辉煌,歌声通宵达旦\footnote{曾新:《明清广州城及方志图研究》,页106—107}。在城南一带,民居稠密、人烟密集,主要街道濠畔街、清水濠街、高第街聚集着大批银号,乃广州城内最大的金融中心,许多吴越、江淮、巴蜀客商皆居于此处\footnote{曾新:《明清广州城及方志图研究》,页107}。可以说,18世纪至19世纪初的广州功能完备、市民生活丰富,是一座如天堂般美妙的国际化大都会,亦是当时地球上当之无愧的第一大城市。

必须指出,广州在18世纪的空前繁荣在很大程度上是清帝国的“一口通商”政策导致的,潘氏、伍氏等行商的巨富亦与清帝国对南粤外贸的限制密切相关。在漫长的历史上,南粤与世界各地的自由贸易一向发达,粤商甚至还曾在第五次北属时期涉足西西里和西班牙,广州则是一直都是南粤最重要的外贸港口。然而,由于元、明、清帝国的接连打压,南粤的对外自由贸易受到严重限制。在清帝国治下,只有行商有资格与西方人直接交易,普通私商只能以行商为外贸中介人。假如清帝国未曾侵占南粤,那么南粤必然会与进入大航海时代的西方世界产生更广泛的接触,广州城肯定会更加繁荣、富商肯定会更多。若无“一口通商”的限制,闽越、吴越的外贸亦将与南粤一样发达,并很可能与南粤一同融入世界秩序。清帝国之罪恶,由此可见一斑。在如此严酷的限制下,我们的伟大祖先仍然创造出了空前的繁荣,并积极与欧美各国交往,这着实是一个异常伟大的奇迹。使广州在18世纪变得空前繁荣者显然不是清帝国,而是无数放眼世界的南粤商人和平民。在他们的努力下,广州变成了东亚最富庶、全世界最大的城市,在珠江口昂然矗立、傲视全球。


\section{东亚大陆最优秀的自治城市:佛山}

\indent 在18—19世纪,广州是南粤乃至整个东亚最大的城市,佛山则是南粤的第二大城市。广州、佛山这两座大都市虽相距仅16公里,但它们的特点却迥然不同。紧邻广州的佛山何以成为南粤的第二大城市,并展现出与广州完全相异的特性?欲回答这一问题,便需考察佛山悠久的自治史。

如第九章所述,与传统东亚大陆城市由政治中心发展而来不同,佛山是一座自发形成于南宋治粤时期的商业城镇。明初编定里甲时,仅将当地大族集体划入里甲中,并未过度破坏原有社会秩序。该地大族多从事冶铁业,积累了财富与社会威望。1449—1450年抗击黄萧养流寇的佛山血战,是佛山自治城市形成的第一个关键历史节点。此战,佛山本地有威望的土豪22人(二十二老)在城市的祭祀中心祖庙组织民兵、指挥战斗,取得胜利。战后,明廷“封”二十二老为“忠义士”,并“敕封”祖庙为“灵应祠”,承认了佛山土豪聚集于祖庙议事的权力结构\footnote{执经生:《佛山自治城市简史》}。

16世纪的广东宗族化运动是佛山自治史上的第二个关键节点。随着霍氏、冼氏、伦氏等本地士大夫获得大量功名,他们在佛山本地积累财富、建设宗族、扩张势力,代替了远在广州的官府的权威,成为佛山社会、经济、教育、祭祀活动的组织者及佛山本土利益的保护者。1553年,珠三角发生大水,佛山周边发陷入饥荒,数千灾民涌入城中,于白昼四处劫掠,“夺米抢金,撞门拆屋”。对此灾情,明帝国广东当局毫不赈济,致使佛山的乱局愈演愈烈。危难时刻,佛山本地乡居进士冼桂奇通过富户组织民兵护送外地粮食入境,施粥赈济灾民,使城市秩序迅速稳定,仅捕获“最桀骜者”一人即平定乱事。在佛山城市史上,此事被称为“冼子平乱”。佛山士大夫通过此次事件展示了他们强大的自治能力,并在城市平民中获得了巨大威望\footnote{罗一星:《明清佛山经济发展与社会变迁》,页151}。

17世纪前期佛山自卫武装及市政议会组织的建立,是佛山自治史上的第三个节点。其时,佛山已成为冶铁大镇,居住于城北的大族及聚居于城南的工匠矛盾尖锐,常发生冲突。在城北大族中,拥有大量进士、举人的细巷李氏乃当时佛山富户的领袖。1614年,曾在北京明廷中官至郎中的李待问回到家乡,倡议设立城市民兵组织以捍卫乡土,得到全城富户响应。冶铁富户出资170两,组建了一支数十人规模的自卫部队“忠义营”。1628年,曾任明帝国员外郎的李待问之弟李升问又组织起三百名“乡夫”。忠义营为职业军队,乡夫则为业余部队。两支军队互为奥援,一同维持城市治安、承担城市防务,成为佛山的常设武装力量\footnote{罗一星:《明清佛山经济发展与社会变迁》,页155—156}。

设立自卫武装仅是李待问重整佛山秩序的第一步。在建立“忠义营”后,他开始着手调解大族和工匠间的矛盾。他首先对过分盘剥工匠的豪右动手,向把持铁、炭、沙之利的他们征收“佐营”之税以维持城市自卫武装。此外,他还整顿城市的度量衡,设置由自己亲自监管的“公秤”以避免商业交易中的欺诈行为,整顿了佛山的经济秩序。至于他做出的最重要举动,无疑是设立制度化的市政议会组织。此前,佛山士绅、土豪虽有至祖庙商议城市大事的习惯法,但“旋聚旋散,率无成规”。1627年,李待问与其兄李好问及乡绅梁完赤、梁锦湾、陈玉京一同出资,用数个月的时间在祖庙西侧空地上建起了一座名为“乡仕会馆”的建筑。李待问更亲自为乡仕会馆书写了有“嘉会”二字的匾额,使之获得别名“嘉会堂”。嘉会堂系佛山历史上首个正式的城市自治机构,主要功能有“处理乡事”及管理本地公益款项,并负有举行文会以“课乡子弟之俊秀者”及对市民进行道德教化的责任。嘉会堂有定期开会的制度,“岁有会,会有规”。在嘉会堂中乡绅的努力下,佛山通往广州的大道于1634年修通。这一庞大工程极大方便了南粤两座最大城市间的人员、物流往来,为南粤的繁荣做出了不可磨灭的贡献。1642年,李待问又出资于城中建立文昌书院,供市民子弟读书,提高了市民的受教育程度\footnote{罗一星:《明清佛山经济发展与社会变迁》,页157—160}。

建立文昌书院的同年,李待问在佛山病逝,享年61岁。从他辞世的时间来看,他是幸运的。因为再过五年,明末清初大洪水便会席卷南粤。1647年清军入侵南粤后,佛山暂时未受几乎遍及全粤的战火侵袭。然而,在1650年底尚可喜发动惨绝人寰的广州大屠杀后,佛山陷入了巨大危机。当时,佛山两次无视尚可喜派来的使节,拒绝剃发降清。尚可喜的部将皆请其进屠佛山,然而老谋深算的尚可喜却拒绝了部下的提议,称:

\begin{quote}

此地为四方商旅凑集之区,往来贸易百货在是。一经杀戮,市井丘墟,商旅裹足,百货不通,亦非吾等之利\footnote{转引自罗一星:《明清佛山经济发展与社会变迁》,页190}。

\end{quote}

尚可喜之所以不屠佛山,绝非其有爱于南粤。在这个丧心病狂的屠夫看来,佛山是一块供其肆意榨取财富的宝地,他要通过更加“细水长流”的方式盘剥佛山。尚可喜在佛山设置了“铁锅总行”,强迫市民将特产铁锅运往总行发卖。不运者便被视为违禁私商,会被藩兵重加勒索。尚可喜还在佛山任用一批旗人恶棍大肆侵渔民众。这些禽兽无恶不作,“强买害人而反诬抢夺”,甚至对贩卖咸鱼、乌榄的小贩亦疯狂敲诈。在尚可喜的疯狂侵害下,佛山市民虽未遭遇战祸,但仍生活于人间地狱中\footnote{罗一星:《明清佛山经济发展与社会变迁》,页192}。

1680年,清帝国消灭平南藩府,佛山总算获得喘息之机。至18世纪中叶,经大半个世纪的和平发展,佛山成为人口超过50万的巨镇,冶铁、陶瓷、纺织业发达。据时人记载,佛山的商贸繁荣程度甚至高于广州:

\begin{quote}

四方商贾之至粤者,率以是(佛山)为归……桡楫交击,争沸喧腾,声越四五里,有为郡会(广州)之所不及者。沿岸而上,屋宇森覆,弥望莫极。其中若纵若横,为衢为衏,几以千数。阛阓层列,百货山积,凡希觏之物,会城所未备者,无不取给于此。往来驿络,骈踵摩肩\footnote{罗一星:《明清佛山经济发展与社会变迁》,页238}。

\end{quote}

广州是南粤面向世界的窗口,佛山则是南粤内部贸易及清帝国对粤贸易的中心。在18世纪,佛山商人大举开发西北面罗定、肇庆的铁矿,每年产铁量达76.5吨。佛山生产的铁锅、缝纫针、火器、农具、香炉、金属丝质量上乘,甚至远销天津。在城南的工业区,数以万计的冶铁工人奋力生产,保证了巨大的铁制品产量。佛山和附近的石湾镇又有极度发达的陶瓷业,两地的产品一举垄断了广州乃至整个东南亚的瓦器、陶瓷市场\footnote{斯波义信:《中国都市史》,页208}。著名的工艺品“石湾公仔”,即在这一时期遍布南粤乃至东南亚各地。这些精美绝伦、栩栩如生的人物、动物陶瓷雕塑既反映了佛山与石湾高超的制瓷技术,亦骄傲地展示着南粤文明的富庶与精致。此外,佛山商人又用本地特产砂糖从吴越换来棉花,用以生产远销东南亚的优质布匹“长青布”。至于用从吴越进口的湖州生丝生产的佛山纱、佛山缎更hay 广受欢迎,远销清帝国西北。佛山又是广西、湖湘米之集散地,两地所产大米分别经西江、北江运抵佛山,再传往南粤各地。因此,18世纪的佛山垄断了桂、湘的米价,拥有数十家米店。发达的商业带来了金融业的繁盛。在当时的佛山城区,银铺、典当铺随处可见。大批外邦商人常驻佛山,兴建了山陕会馆、浙江会馆、莲峰会馆(福建纸商)、江西会馆、楚北会馆、楚南会馆。相应地,佛山商人在城中修建了28座本地会馆,他们中的许多人经西江、北江、东江而出前往南粤各地,有的人甚至还远赴汉口、汉阳、苏州、天津等地做生意\footnote{斯波义信:《中国都市史》,页208—209}。繁荣的商业区坐落于佛山城区北部,由汾水铺、大基铺、基文铺三个街区构成。这一区域紧靠流过城北的汾水,有25座繁忙的码头、大批餐馆、妓馆、店铺、戏班及佛山的几乎所有商业会馆,每间店铺的高度皆为至少两层\footnote{斯波义信:《中国都市史》,页209}。城市的中部为传统的大族聚居区,有成片的深宅大院、园林和历史悠久的祖庙,商业也很发达。至于城南则为工业区,打铁之声、织机轧声不绝于耳,冶铁产生的火光通宵达旦地点亮着佛山的夜空\footnote{罗一星:《明清佛山经济发展与社会变迁》,页260—265}。在这样一座功能齐全、区域多样的城市中,数十万不同行业、不同阶级的居民形成了各式各样的自治团体,铁商、棉布商、生丝商、药商、盐商、木材商、宝石商、米商皆有自己的行业会馆。1740年代以后,随着各行业的发展壮大,日渐增多的工匠在行会中产生影响力,各行会迅速分化为由作坊主组织的“东家行”和由工匠组织的“西家行”。最早产生此种分化的行业为专门生产花盆、金鱼缸、花垌、建筑部件的陶艺花盆行。1741年,该行东西家一同议定了工价和行规,并对外行人入行做出限制,要求外来学徒必须学满六年、每季交银12.5元方可出师。这一举动不但保护了本行业的利益,亦令行内劳工获得与“东家”一同议定薪资、管理行内事务的权力。此后一个世纪内,佛山各行业的东西行几乎皆以此种方式展开合作。工匠的利益得到了东家行的承认,西家行则相应承认整个行业的利益。此种现象,在清帝国境内是仅见的。例如,在同时代的吴越苏州,手工业虽然十分兴盛,但当地工匠几乎没有任何权力,他们发动的“叫歇”(罢工)活动几乎总是被城内富户勾结清帝国官府镇压\footnote{罗一星:《明清佛山经济发展与社会变迁》,页354}。

另一方面,大批来自省内南海、新会、香山、顺德、高要,及部分来自岭北的侨寓商贾形成了侨寓集团,与佛山土著大族多有冲突,要求在城市事务上取得更大的政治权力。随着佛山城市的不断扩大,阶级、族群间的博弈日渐复杂,拥有更大仲裁能力的议会组织呼之欲出。新的市议会“大魁堂”组织的出现,是佛山自治史上的第四个节点。1738年,佛山土豪、乡绅设大魁堂祖庙东侧的崇正社学内。该社学建于16世纪魏校毁淫祠运动之后,乃佛山士子课文、士绅祭拜文昌帝君的场所。大魁堂系佛山的最高决策机构,“乡事由斯会集议决”,其负责人被称为“值事”,共有八名。充任值事者有商人、乡居官员、生员、耆老。无论土著、侨寓,凡深孚众望者皆可出任值事。全镇数年(不定期)公举大魁堂值事一次,值事不得连任。凡遇大事,大魁堂值事传“阖镇绅士”公议。大魁堂的职能有:设置并管理佛山义仓以应对饥荒、设置义冢、收养弃婴、照料老人、定期疏浚对商业至关重要的河道、仲裁商业纠纷、管理学校、与官府交涉\footnote{罗一星:《明清佛山经济发展与社会变迁》,页364}。至此,佛山的议会政治已经十分成熟,出现了制度化的选举制度。

1757年,侨寓商贾和土著大族对祖庙祭祀权的争夺,成为佛山自治史上的第五个节点。祖庙祭祀自1449年佛山之战后,即由土著垄断。侨寓商贾亦欲分享祭祀权,双方冲突激烈。是年,经清帝国南海县官府裁定,祖庙祭祀应由全镇“绅耆士庶”一同进行。至此,祖庙祭祀事宜收归大魁堂。此后,侨寓因与土著一同分享着祖庙祭祀而对祖庙和佛山本地产生了强烈认同,变得日益本地化。两者日渐融合为一个新共同体,土侨纠纷官司明显减少\footnote{执经生:《佛山自治城市简史》}。

随着大魁堂的建立,佛山各行业、各街区间的关系变得日益融洽。在每年三月三日“北帝诞”这一天,一年一度的“烧大爆”活动在祖庙门前举行,成为全城的盛典。是日,人们用数以万计的大小纸爆、椰爆堆成种种美丽的楼阁、人物模型,最大的大纸爆香车足有2.5米高。燃爆者攀上高架,以祖庙中的“神火”引燃长六七丈的导火索。在“发声如雷,远近震动”的爆声中,全城数十万男女竞相观堵,场面壮观至极。爆声过后,活动最精彩的环节到来了。全城各行会、会馆、街区皆派出自己的代表队,争相抢夺爆竹中被称为“爆首”的铁环。在激烈的争夺中,各队队员互相配合,为了自己行业、街区的荣誉奋力拼搏,优胜的队伍能够得到丰厚的奖品。“烧大爆”活动不但为全城居民提供了一同欢度节日的平台,亦能培养城市内部各行业、街区的共同体意识,可谓寓教于乐\footnote{罗一星:《明清佛山经济发展与社会变迁》,页455—456}。舒适富足的城市生活并未使佛山人忘记自己的英勇祖先。每年中秋,他们都会举办被称为“出秋色”的活动,以纪念1449年佛山保卫战时佛山民兵在中秋之夜举彩旗、鸣金鼓的一幕。在这一活动中,各街区争奇斗艳,出动庞大的花灯队伍通宵游行。这些花灯既有亭台楼阁、花草树木、神仙古人之状者,更有许多兵将、战马、火炮的造型,用以象征那场气壮山河的佛山保卫战。每支队伍中都有大批优伶戏子演奏音乐、表演节目,声势辉宏。更令人称奇的是,每支队伍中皆有习武丁壮组成的舞狮队。这是因为佛山各铺均有自己的武馆,而舞狮正是习武之人展示本铺武力的绝好方式。每次“出秋色”活动皆是各街区间的激烈竞争,亦是对各街区凝聚力的一次考验。正因为佛山拥有坚实的社区组织,此种活动才能每年举行,它是佛山社区自治及武德的最好体现\footnote{罗一星:《明清佛山经济发展与社会变迁》,页409—411}。

自18世纪后期起,大魁堂的组织进一步系统化,在大魁堂值事之下设置了祖庙值事、义仓值事、二十四铺(街区)值事和书院值事,分管全城的祭祀、赈济、民政与教育,各值事下又有一批管理账目收支的“司数”\footnote{罗一星:《明清佛山经济发展与社会变迁》,页376}。大魁堂运作良好,并具有发动市民与官府对抗的能力。1784年,有与清帝国政府勾结的两名外地商人欲在佛山栅下河旁建硝厂。若此厂建成,势必严重污染佛山城市用水。大魁堂值事区宏绪、劳潼(皆为侨寓出身)“上控”南海县衙,反遭吏员刁难,遂发动全镇绅民向清两广总督府申诉,终于引起两广总督孙士毅干预,从而阻止了硝厂的建造。1824年,佛山饥荒,大魁堂开放义仓赈济贫民。时有人谣传士绅侵吞义仓款项,清帝国政府遂趁机将义仓收归官有。佛山市民愤怒了。为了保卫自己的城市和自由,他们勇敢地涌上街头展开暴动,一举砸毁了清政府设于佛山的巡检司(仅三十名象征性驻兵)。其后,在大魁堂值事冼沂交涉下,清政府不得不交出义仓,暴动市民亦迅速自行解散,佛山市民的抗争取得了完胜\footnote{执经生:《佛山自治城市简史》}。

自治城市佛山在近代南粤的出现乃是南粤文明创造出的一项奇迹。虽然时人将佛山与河南朱仙镇、江右景德镇、楚地汉口镇并称为“四大镇”,但后三者显然远远不能与拥有完美的议会制度和市民社会的佛山相提并论。佛山拥有复杂的社会结构,城内各行业、各阶级、各族群皆有自己的组织。这些组织通过旷日持久的博弈,催生出了佛山令人惊叹的议会政治。在整个东亚大陆,佛山这样的城市都是独一无二的。这座城市足以与意大利的威尼斯、弗洛伦撒、德国的汉萨同盟城市及日本的堺港、博多相提并论,乃全世界自治城市史中的优秀一员。这表明,18世纪南粤的文明程度已足已与欧洲、日本比肩。我们不难想像,假若南粤没有被罪恶的清帝国侵占,随着南粤自发秩序的演化,如佛山一般的自治城镇将在南粤大地上星罗棋布,使南粤足以傲立于地球之上,成为与西方携手步入工业革命与立宪政治的伟大文明。

\section{伟大的中南半岛拓殖事业:河仙政权、“龙门将士”与郑信}

\indent 自17世纪末起,我们的祖先不但在南粤本土创造出了空前的文明成果,亦勇敢地拓殖东南亚,在中南半岛建立了一系列政权。创立这些政权的勇者们活跃于其时东南亚的历史舞台上,合纵连横,演出了一段波澜壮阔的历史活剧。这些南粤的海外种子极大地推动了东南亚的历史进程,亦帮助许多粤人同胞渡过南海、摆脱清帝国的残暴统治。ni 段历史虽发生于东南亚,却无疑是南粤史的光荣一页,值得我们为之骄傲、将之永远铭记。

1671年,南粤本土正在清帝国暴虐的“迁界令”下饱受折磨。这一年,一位名叫莫玖的粤人携全族浮海来到柬埔寨(真腊国,越南人蔑称之为“高蛮”)南荣(今金边),得到柬埔寨王室的信任。对于莫玖的早年经历我们几乎一无所知,只知他是雷州海康县东岭村人。粤西的雷州半岛与海南岛隔海相望,本是俚人、僚人世代居住的土地。在冯冼时代,这里由僚人土豪宁氏长期割据。公元705年唐帝国血腥屠灭宁氏政权后,许多闽人移民泛海来到此处定居。至第五次北属时代后期,这里已形成了由闽人、俚僚土著混合而成、讲闽南语系雷州话的南粤雷州民系。在明帝国治粤时期,勇敢的雷州人无视帝国愚昧的海禁政策,积极从事海洋贸易,在“迁界令”颁布后,他们继续进行勇敢的抗争,不断地尝试出海,被气急败坏的清帝国诬称为“西贼”\footnote{李庆新:《16—17世纪粤西“珠贼”海盗与“西贼”》}。可以断定的是,莫玖便是这些伟大抗争者中的一员。

到达柬埔寨后不久,莫玖就以重金贿赂该国国王,被任命为该国东南边陲港口城市河仙的“屋牙”(地方官)。莫玖之所以选择河仙,是因为该地居住着大批粤侨,且有银矿,故商贸发达、十分富庶。莫玖到达河仙后,立即将大批越南拓荒者召至附近农村定居,史载:

\begin{quote}

(莫玖)招越南流民于富国、陇棋、芹渤、淎渀、沥架、哥毛等处,立七村社\footnote{李庆新:《莫氏河仙政权“港口国”与18世纪中南半岛局势》}。

\end{quote}

“七村社”的位置大体相当于今日柬埔寨磅逊湾至越南金瓯湾间200公里的区域,包括越南西南端的岛屿富国岛。此地处在越柬边境,为两国政府力量所不及,遂使莫玖得以在夹缝间一展身手。自1592年起,越南即陷入皇室黎氏大权旁落,军事政权阮主、郑主分据南北的局面。郑主居于北方之升龙(今河内),挟持黎皇;阮主居于南方顺化,不断向南拓土,双方以越南中部的灵山为界。在1627—1673年间,郑阮两军在边界一带展开七次大战,无法决出胜负,遂默认了对方的存在。在莫玖到达河仙前后,阮主正积极向南扩张,于1692年完全征服占婆,并向柬埔寨的嘉定地区(今湄公河三角洲西贡一带)扩张。大批平民随阮军大举南下,为阮主积极开垦田土\footnote{}。莫玖招徕大批越南拓荒者之事,便是在这一背景下发生的。

1679年,泰国(暹罗)入侵柬埔寨,进攻河仙,莫玖兵败被俘。此后九年间,莫玖被囚居于泰国万岁山海津,无时无刻不在等待东山再起。1688年,泰国发生内乱,莫玖遂趁机逃回柬埔寨,于其越南妻子裴氏廪生下了长子莫天赐。1700年,莫玖重返河仙,继续统治当地,四方商贾来附者益众。1708年,因见柬埔寨孱弱、无法保护自己,莫玖遂转而向阮主称臣,被阮主授为河仙镇总兵,形同阮主政权属国国君。为避免与曾于1527年篡夺黎氏帝位的越南逆臣莫登庸同姓,莫玖改姓为“鄚”\footnote{李庆新:《莫氏河仙政权“港口国”与18世纪中南半岛局势》}。

依附阮主后,河仙政权获得平稳的发展环境。鄚玖将境内土地分配给治下民众,为他们提供农具、兴修水利、伐开大片森林,且不向民众征税,使河仙变为富庶的农耕区。他又大举招揽外国商船前来贸易,不向他们征收商税,制造了“帆樯连络而来”,粤人、越南人、柬埔寨人、爪哇人遍布城中的局面。此外,他“开赌博场,征课本,谓之‘花枝’”,通过博彩税获得了巨额收入。依靠这些收入,他大举兴建河仙的城防工事,亲赴西班牙人治下的菲律宾和荷兰人治下的巴达维亚学习军事、施政技术,筑起了坚固的堡垒、宽阔的城壕,并训练了强大的炮兵部队。在1728、1729年,鄚玖两次遣使赴锁国中的日本,获得了派船前往日本贸易的许可。这表明,河仙政权已完全成为中南半岛南端的一处独立王国,甚至拥有不受越南人控制的自主外交权。事实上,时人已视河仙为独立国家,称之为“港口国”。1735年,鄚玖以81岁高龄去世于河仙,子鄚天赐继位。鄚天赐延续其父独立自主的政策,从阮主手中获得了独自铸币的许可。河仙所铸货币称“安法元宝”,质量颇为优良\footnote{李庆新:《莫氏河仙政权“港口国”与18世纪中南半岛局势》}。特别值得一提的是,河仙政权一直保持着明式发式、衣冠,以示不忘华夏衣冠。可见,鄚氏父子在维持着粤人自古以来的航海传统的同时,亦以华夏文明的最后保卫者自居\footnote{李庆新:《鄚氏河仙政权(“港口国”)及其对外关系——兼谈东南亚史上的“非经典政权”》}。

在17世纪末,另一批由粤人指挥的军队出现在中南半岛南部,他们便是被越南人称为“龙门将士”的郑氏残军,其首领系茂名人杨彦迪。杨彦迪本为粤西海盗,于1655年向郑氏“投诚”,后随郑成功前往台湾。1677年,杨彦迪率船80艘从台湾出发前往南粤策应吴三桂反清,一举攻占粤西龙门。1681年三月,杨彦迪率部在海南登陆,攻克海口所城及澄迈、定安两县,但不幸在海战中被清军击败,遂会同其副将黄进、郑氏高雷廉总兵陈上川率残部3000余人、船50余艘退往越南顺化,向阮主表示臣服\footnote{李庆新:《16—17世纪粤西“珠贼”海盗与“西贼”》}。阮主十分忌惮这支强大的武装,又希望对其加以利用,乃命龙门将士前往东浦地区(今西贡河流域)垦荒。当时,东浦基本由柬埔寨控制,阮主只在那里设有几个据点,急欲吞并之。龙门将士到达东浦地区后分居于鹿野(今属越南边和省)、美萩(属定祥省)、盘粦(属边和),垦荒造屋,强迫柬埔寨将此地划归阮主,建起嘉定(又称“柴棍”,今西贡)、美耐大铺(今边和)、美湫大铺(今美湫)三座繁荣的城市,前往其地贸易的西方、日本、爪哇商人甚众\footnote{陈荆和:《鄚天赐与郑信——政治立场、冲突及时代背景之研究》}。1688年,黄进发动兵变杀害杨彦迪,随即被阮主派兵击杀\footnote{陈重金:《越南通史》,页241—242}。1700年,因西贡河流域已被龙门将士大规模开发,阮主乃于当地设嘉定府,派出官员统治\footnote{陈荆和:《鄚天赐与郑信——政治立场、冲突及时代背景之研究》}。

阮主设置嘉定府后,龙门将士仍维持自主,不受越南军官节制。因杨彦迪、黄进已死,陈上川乃递补为龙门将士的新领袖,被阮主任命为“藩镇都督”,掌管嘉定兵权。当地的实际统治权不在阮主派出的流官手中,而是归于陈上川。陈上川系粤西高州吴川人,原本家境富裕,曾于1641年考中秀才,入高州府学读书。后来,他投笔从戎,追随杨彦迪参加抗清战争。出任藩镇都督后,陈上川坐镇嘉定,积极招徕客商,并坚定地站在阮主一边。1714年,柬埔寨发生内战,国王匿螉深在泰国支持下起兵讨伐亲越南的副王匿螉淹,匿螉淹其逃至嘉定求援。陈上川乃率军深入柬埔寨境内,围匿螉深于罗碧城,迫使其弃城流亡泰国,陈上川遂立匿螉淹为王而还\footnote{陈重金:《越南通史》,页242}。

此次热带远征大概拖垮了陈上川的身体。一年后,陈上川病逝,其子陈大定继领龙门将士。陈大定与河仙联姻,娶鄚天赐之妹鄚金定为妻,生子陈文方,从而与河仙结成了牢固的粤人联盟。阮主政权将这支能够左右柬埔寨国政的“外籍军团”视为眼中钉,急欲除之而后快。1732年,阮主令龙门将士配合越军进攻柬埔寨。在战争中,越军统帅张福永忽然诬陷陈大定“行兵滞留,与高蛮私相结纳”。蒙此不白之冤的陈大定极为愤怒,亲赴顺化申辩,反遭逮捕,惨死于狱。陈文方闻讯后,急忙携其母赴河仙投奔舅父鄚天赐。至于失去统帅的龙门将士则被阮主重新整编,划入越南军队中\footnote{许肇林:《越南阮氏政权如何侵吞下柬埔寨》}。这些盛极一时的南粤海外种子,从此令人扼腕地烟消云散。

在龙门将士被越南人背信弃义地消灭时,河仙政权正如日中天。1747年,柬埔寨内战又起,匿螉深之子匿原(Chey Chettha V)自泰国引兵回柬,驱逐国王匿螉他(匿螉淹之子)而篡位。匿原采取敌视阮主的外交政策,与北面的郑主通使,图谋合击阮主。此事败露后,阮主于1753年派兵进攻柬埔寨。经两年战争,匿原兵败,逃往河仙寻求庇护。1756年,鄚天赐“上书”阮主,称匿原欲献地求和,阮主许之,匿原亦得以归国。三年后,匿原病死,其叔匿润监国,却被其婿匿馨发动政变杀害。阮军趁机再次进攻柬埔寨,匿馨败死,其子匿螉尊流亡河仙,认鄚天赐为养父。鄚天赐乃为其“养子”向阮主“上书”,极言可立匿螉尊为王,阮主从之。在越南军队的护送下回国继位后,匿螉尊感念其“养父”的恩德,便将真森、柴末、灵琼、芹渤、淎渀五府之地献给河仙。至此,河仙政权的领土大大扩展,进入全盛时期。鄚天赐对外自称“高棉王”或“真腊王”,形同柬埔寨的太上皇\footnote{李庆新:《鄚氏河仙政权(“港口国”)及其对外关系——兼谈东南亚史上的“非经典政权”》}。

正当鄚天赐坐拥河仙,遥控柬埔寨,几乎不将越南人放在眼里时,他生命中的最大宿敌却在泰国悄然崛起,此人就是被泰国人敬称为“吞武里大帝”的郑信。1734年,侨居泰国大城府的潮州澄海县商人郑镛与其泰族妻子诺罗央生下了一名健康的男孩,他便是日后的风云人物郑信。幼年时代的郑信被泰国高官收养为义子,接受了完整的小乘佛教教育。14岁那年,他被送入宫廷担任侍从。此后,他受到泰王信任,被任命为西部与缅甸接壤的重镇来兴府之副府尹,担负起边防重任。

由于从小在泰文化中长大,郑信自认为是泰国人,并对自己的南粤父系血统很是敏感。因此,他狂热地试图证明自己是一个合格的泰国人。在危难到来时,他对祖国的忠贞比其他泰国人更为坚定。1765年,缅甸军队大举入侵泰国。经一年余战斗,缅军包围了泰国首都大城府。1767年,在首都即将陷落时,郑信率五百名残兵杀出重围,寻机东山再起。不久后,大城府失守,泰国国王厄伽陀殉国,持续了四百余年的阿瑜陀耶王朝灭亡。逃出大城府后,郑信遣使持其亲笔信赴河仙向鄚天赐求援,这是此对宿敌的第一次交往。是年四月十四日,郑信的使节抵达河仙,受到鄚天赐的隆重接待。出于同为粤人之谊,鄚天赐非常重视深陷危难的郑信,允诺将在八九月间派出水师参加抗缅作战\footnote{陈荆和:《鄚天赐与郑信——政治立场、冲突及时代背景之研究》}。然而很快,两人的关系却出人意料地迅速破裂了。

1767年夏季开始后,缅甸人忙于与清帝国的战争,无暇顾及泰国方面的战事。因此,郑信的部队于年底顺利击败缅军、光复大城府。次年初,郑信在吞武里(今曼谷以西)登基为泰国国王,开创吞武里王朝。开创新朝的郑信自然视前朝王族为眼中钉。不巧的是,在1767年底,一位名叫昭翠的阿瑜陀耶王朝王子流亡至河仙,被鄚天赐保护起来。鄚天赐希望施其在柬埔寨所行之故伎,将昭翠护送回国登上王位,从而使自己成为泰国的太上皇。这一野心勃勃的计划,无疑是和郑信针锋相对的\footnote{陈荆和:《鄚天赐与郑信——政治立场、冲突及时代背景之研究》}。

1768年,两人进行了第一次较量。是年,郑信遣使河仙,提出以两门欧式火炮和一部分土地为交换物,要求鄚天赐交出昭翠。鄚天赐假意答应郑信,命其婿徐湧率一支满载大米的船队以赈济战争流民为名驶至曼谷,意图发动突袭擒拿郑信。然而,郑信识破了鄚天赐的阴谋。在泰军的攻击下,徐湧败死。同年,郑信遣使至柬埔寨,要求匿螉尊称臣,被其以“郑信非暹罗王种”为由拒绝。匿螉尊的傲慢回答触碰到了郑信最为敏感的血统问题,令其恼羞成怒。1769年3月,泰军大举进攻柬埔寨,败于炉熰。7月,在郑信的策划下,侨居河仙、泰国边界之白马山的潮州人陈太与鄚氏族人鄚崇、鄚宽勾结,未起兵而事泄,鄚崇、鄚宽被杀,陈太只身逃入泰国。鄚天赐对郑信进攻自己“养子”、煽动叛乱之举十分震惊,决定出兵报复。同年9月,鄚天赐组织起五万大军,以昭翠、陈文方为统帅,浩浩荡荡地杀入泰国境内\footnote{陈荆和:《鄚天赐与郑信——政治立场、冲突及时代背景之研究》}。

兵力庞大的河仙军初战告捷,在泰国东南重镇真奔城下击溃郑信部将陈联(潮州人)所率三千泰军的阻击,随即攻陷该城。然此后两个月内,河仙军深受疫病困扰,死者日以百计。此外,他们虽然打出了拥立昭翠恢复阿瑜陀耶王朝的旗号,但根本得不到泰国百姓的支持。年底,河仙军减员严重、无力再进,不得不退返河仙。此次军事行动,鄚天赐损失了超过三万名兵员,元气大伤。1770年,郑信策动河仙治下的柬埔寨人及逃军发动叛乱。叛军一度进攻河仙城,但在激战后被鄚天赐讨平。此次乱事进一步削弱了河仙的实力,使郑信得以趁虚而入\footnote{李庆新:《鄚氏河仙政权(“港口国”)及其对外关系——兼谈东南亚史上的“非经典政权”》}。1771年,郑信亲率东征大军包围河仙城,鄚天赐率全城军民极力抵御,“城内一人挟作十人之役”。双方以西式火器猛烈对射,相持十余日,均伤亡惨重。11月19日,泰军攻破河仙,俘获鄚天赐之子女及昭翠,鄚天赐弃城逃至朱笃。因泰军紧追不舍,鄚天赐又逃至龙湖(今越南永隆市),得到阮主庇护\footnote{陈荆和:《鄚天赐与郑信——政治立场、冲突及时代背景之研究》}。 

获胜之后,郑信留陈联镇守河仙,携俘虏回国,将昭翠杀害,却未加害被俘的鄚氏族人。郑信留有余地的做法为双方的和谈创造了空间。1773年初,鄚天赐在阮主的命令下遣使携礼品赴泰国议和。郑信认为自己已好好地教训了鄚天赐,遂同意归还河仙城及俘虏。回到河仙后,鄚天赐见到的只有一片残破,几乎丧失了一切的他乃不再自称“真腊王”\footnote{陈荆和:《鄚天赐与郑信——政治立场、冲突及时代背景之研究》}。

这时,阮主政权已处在风雨飘摇之中。1771年,震动全越南的“西山之乱”爆发。六年后,西山军攻陷嘉定,末代阮主阮福淳被杀。随着阮主政权的覆灭,河仙亦告陷落。被西山军追赶的鄚天赐流亡海上,在逊磅诸岛湾口的罗腔岛(今柬埔寨西哈努克市附近)苟延残喘。就在鄚天赐走投无路时,对其感情复杂的郑信意外地伸出了援手。当年10月,郑信遣使至罗湾岛,提出可为鄚天赐提供庇护。鄚天赐别无选择,只得同意。次年初,鄚天赐抵达吞武里,被引入王宫金殿,受到郑信的亲自款待。郑信视鄚天赐为国宾,为之设宴五日,并推心置腹地说:

\begin{quote}

愿以诚好而相结,幸无念旧恶\footnote{转引自陈荆和:《鄚天赐与郑信——政治立场、冲突及时代背景之研究》}。

\end{quote}

此后一段时间里,鄚天赐做为政治难民居于曼谷,与郑信关系亲密。但很快,两人的关系就再度破裂。1777年末,阮氏残军推阮福淳之侄阮福映为大元帅,克复嘉定。阮福映通好泰国,据嘉定对抗西山军,于三年后称王。见阮氏与泰国关系紧密,西山政权决定实施反间计。1780年夏,西山政权伪造了一封阮将杜清仁致鄚天赐的密信,内称越南船队将不日造访吞武里,届时鄚天赐当与之里应外合、一举拿下曼谷。当时,步入晚年的郑信已因自己血统和认同间的矛盾陷入精神疾病,变得暴躁易怒、多疑好杀。盛怒之下,郑信将无辜的鄚天赐投入监狱,于11月1日迫其饮毒药自尽。11月20日,鄚天赐的数十名亲属惨遭处决\footnote{陈荆和:《鄚天赐与郑信——政治立场、冲突及时代背景之研究》}。盛极一时的河仙政权,就这样惨淡收场了。这时距离郑信的死,只有不到两年。

次年,已陷入癫狂状态的郑信下令严加堤防外商,粤商在泰国的活动几乎完全停止。1782年初,为了证明自己绝对有资格担任泰国人的国王,郑信以总体战的方式动员20万大军,任命他的泰族老部下通銮(母系有部分潮州血统)为元帅大举入侵柬埔寨,使臣民怨声载道。自复国战争时起,通銮便一直追随郑信,二人情同手足。然而,多疑的郑信却将通銮的两名家属扣押于曼谷充当人质,使通銮非常寒心。当通銮的大军正在前线与敌对峙时,吞武里的禁军将领披耶讪发动政变废黜了郑信。通銮乃火速回师平息乱局,于4月7日以“精神错乱、残忍、独断”的罪名将郑信斩首,随即迁都曼谷并登基,自称“拉玛一世”,开创了延续至今的却克里王朝\footnote{陈荆和:《鄚天赐与郑信——政治立场、冲突及时代背景之研究》}。潮商家庭出身的泰国英雄郑信战胜了强大的缅甸侵略者,但最终受困于血统与认同间的矛盾,没有战胜自己。

鄚天赐和郑信这对源自南粤的宿敌就这样消失在历史的浊流中,但粤人在中南半岛南部的政治活动尚未结束。1782年,阮福映被西山军击败,流亡泰国,通过法国传教士百多禄(Pigneau de Behaine)与法国展开军事合作。1787年,拉玛一世将鄚天赐幸存的第四子鄚子生派往河仙,使河仙成为泰国的附庸。不久后,鄚子生病逝,其侄鄚公柄继位。1799年,鄚公柄病逝,拉玛一世又以鄚天赐的幸存幼子鄚子添出镇河仙。此后经长年作战,阮福映于1802年成功消灭西山政权,于顺化登基称帝,建立一统越南的阮朝,是为嘉隆帝。嘉隆帝希望将河仙收归己有,便任命鄚子添为河仙镇镇守。此时的河仙早已无昔日之威,只得俯首听命,在两属于越南、泰国的状态下苟延残喘。1809年,鄚子添逝世。当时,泰国正忙于与缅甸交战,无暇东顾。嘉隆帝便趁机废黜由泰国任命的鄚公瑜(鄚子生之侄),派出流官镇守河仙。泰国为避免两面受敌,只得于1810年无可奈何地接受现状\footnote{许肇林:《越南阮氏政权如何侵吞下柬埔寨》}。至此,延续了一个多世纪的河仙鄚氏政权便在越南阮朝的吏治帝国机器下彻底消失。今天,河仙早已成为越南西南边陲的坚江省。我们只能通过鄚氏家族留在当地的庞大墓葬群凭吊当年河仙政权的伟大\footnote{今日越南坚江省之屏山有鄚氏家族墓葬群,计墓葬45处,系鄚天赐时代之物。详细研究,可参看李庆新:《“海上明朝”:鄚氏河仙政权的中华特色》}。在18世纪,是南粤人推动着中南半岛南部的历史进程。然而在错综复杂的国际环境下,当地由粤人开创的政权不幸未能经受考验,先后消失了。假如他们可以多撑半个世纪,那么当西方人大举进入中南半岛时,我南粤在当地结出的果实定然会更为丰硕。历史无法重来,今天的我们面对这一结局,唯有在慨叹之后吸取教训、自强不息。在清帝国侵占南粤的情况下,我们的伟大祖先依旧能够在东南亚创造出如此璀璨的文明成果。那么假如南粤拥有真正的自立,便一定会在大航海时代纵横全球,成为足以与西方相提并论的伟大文明。

\section{旗帮海盗的兴亡与海上英雄张保仔}

\indent 18世纪末,正当西山军横扫越南时,一群粤人海上英雄加入了他们的队伍,从而在越南、南粤沿海掀起了滔天巨浪,并为南粤创造出一种充满活力的崭新社会秩序。这一秩序最终影响到南粤在19世纪历史走向,在南粤史上有着重要的意义。

1780年,一个名叫陈添保的粤西廉州渔民和他的妻儿在出海捕鱼时遭遇风暴,被吹到越南北部海岸。他们在升龙(河内)一带居住下来,继续从事捕鱼业。三年后,西山军对越北郑主政权发动攻势,俘获了陈添保。由于陈添保有充足的航海经验,西山军以之为总兵。在消灭郑主的战争中,陈添保出力颇多。1787年,郑主政权灭亡,末代黎皇昭统帝黎维祁向清帝国求援。次年,乾隆帝命两广总督孙士毅发粤、桂、滇、黔二十万大军自水陆两路进攻越南,清军入侵越南后,很快占领升龙,拥昭统帝复位。西山主阮惠乃于当年十一月二十五日于御屏山登基称帝以收拾人心,是为光中帝,随即率十万军队、战象百余北上迎敌\footnote{陈重金:《越南通史》,页273}。值此危急关头,光中帝又封陈添保为“总兵保德侯”,令其招募水师以抗水路之敌\footnote{穆黛安:《华南海盗:一七九〇——一八一〇》,页38}。

光中帝分配给陈添保的军队,只有可怜的6艘战船和200名越南兵。不过,陈添保利用其身为粤人的优势迅速从南粤招募到了梁文庚(新会渔民)、樊文才(陆水渔民)这两名精通航海的人才。不久后,光中帝又向陈添保增拨16艘战船,令他招募更多士兵。陈添保遂将当时活跃于粤西雷州的海盗头目莫观扶、郑七招入麾下,使他的水师实力大增,拥船百艘,上述四人则都被任命为总兵\footnote{穆黛安:《华南海盗:一七九〇——一八一〇》,页38}。这支水师名义上是西山政权的军队,其实大部分人都是南粤海盗。

1789年正月五日黎明,光中帝对升龙一带的清帝国侵略军发动奇袭,取得了史诗般的胜利。清军望风奔溃,尸横遍野,孙士毅与黎昭统帝仅率数骑狼狈逃回清帝国境内\footnote{陈重金:《越南通史》,页275}。1790年,昭统帝到达北京,做为失国之君被编入汉军八旗,并于三年后孤苦地病死在那里\footnote{陈重金:《越南通史》,页279}。清帝国经此惨败,无力再战,只得承认西山政权为其“属国”,不再介入越南内战,西山军遂得以全力对付南面的阮福映。

此后十余年间,西山政权与得到法国、泰国支持的阮福映展开了漫长而惨烈的战争。1792年,为筹措军饷,光中帝命陈添保的水师劫掠清帝国治下的粤、闽、吴越沿海。此后数年间,陈添保将其水师分为三队,由麾下的四名总兵轮番率领出击。当时,已立国百余年的清帝国文恬武嬉,对海防毫不在意。ni 支以粤人为主的越南水师纵横于广东、福建、浙江沿海,一面劫掠乡村、一面招募人员,令清帝国官僚束手无策,只得恐惧地呼之为“艇匪”\footnote{穆黛安:《华南海盗:一七九〇——一八一〇》,页39 }。1794年,粤西吴川海盗唐德被陈添保收为部将。次年,陈添保升任都督\footnote{穆黛安:《华南海盗:一七九〇——一八一〇》,页40}。在1797年6月沱㶞港(今岘港)洋面的海战中,陈添保舰队奋勇迎战阮军,折舰三十艘,成功破坏了阮福映进攻归仁的计划。陈添保因而受到西山政权嘉奖,被任命为“统善艚道各支大都督”,有权指挥越南境内的所有海盗\footnote{穆黛安:《华南海盗:一七九〇——一八一〇》,页40}。

获得如此权威后,陈添保开始整编其部队。陈添保的部队本由各股互不统属的南粤海盗聚合而成。这些海盗帮派漫无纪律,只听命于被称为“老板”的帮派大佬。这些老板仅在名义上服从陈添保,实则经常自行其事。现在,每个老板都被授予“乌艚总兵”(所谓“乌艚”,指陈添保水师所使用的越南黑色战船),成了陈添保的正式部将。1797年7月,这支新军首次出击,以陆海夹攻的方式攻击庆和一带的港口,击毙大批阮军\footnote{穆黛安:《华南海盗:一七九〇——一八一〇》,页41}。然而,他们最终仍然不是受过法式军事训练的阮军的对手。1799年,阮军攻克归仁。次年,陈添保舰队协同西山陆军反攻归仁,结果水陆两军同遭失败。1801年,阮福映亲率大军进攻顺化,于2月21在陆海两面与西山军主力展开决战。据参战的法国教官称,此役系“交趾支那历史上所知的最为惨烈的一战”,西山军全线败北,损失了5万军队和超过6000支枪炮。陈添保的舰队也遭遇毁灭性打击,樊文才、梁文庚、莫观扶皆被阮军俘虏。战后,西山景盛帝阮光缵(光中帝之子)自顺化逃往升龙。陈添保见西山政权大势已去,乃于同年11月率家属及三十名随从回到南粤,向清帝国上缴了自己的越南官印,获得嘉庆帝赦免,被安置于远离海洋的粤北南雄\footnote{穆黛安:《华南海盗:一七九〇——一八一〇》,页49}。

虽然陈添保逃跑了,但仍有一批南粤海盗继续效忠西山政权。当时,郑七拥战船200艘,实力强大,被景盛帝封为大司马。然而,他们终究无法抵御阮福映的法式军队。1802年7月,阮军攻克升龙,对西山皇室施以凌迟处死、五象分尸酷刑。9月,郑七的最后据点、位于越粤边境的港口江坪被阮军攻陷,郑七被俘,遭斩首示众。此外,樊文才、梁文庚、莫观扶则被阮福映移交清帝国,惨遭凌迟处死。因越南全境已被阮军占领,群龙无首的郑七部众纷纷驶返南粤洋面,变回互不统属的海盗。这些海盗分为十二股,为争夺财富和领导权互相攻击,陷入了血腥的自相残杀\footnote{穆黛安:《华南海盗:一七九〇——一八一〇》,页59}。

到1805年,经过长达三年的火拼和分化组合,这些海盗形成了六个集团。是年夏,这六个集团的老板举行会盟,签订了一份合约,规定各集团以不同颜色为旗号,称“旗帮”,各自划定活动区域,不再互相交战。以下列表反映各旗帮的基本情况:

\begin{center}
	\begin{tabular}{ c | c | c}
		\hline
		帮派名 & 活动范围 & 帮主 \\
		\hline
		
		红旗帮 & 粤洋东路(惠州、潮州洋面)、中路(广州、肇庆洋面) & 郑一(郑七之弟)\\
		
		黑旗帮 &  & 郭婆带(番禺人)\\
		
		黄旗帮 & 粤洋西路(高州、廉州洋面) & 吴知青(顺德人,号“东海八”)\\
		
		青旗帮 &  & 李尚青(号“虾蟆养”)\\
		
		蓝旗帮 &  & 麦金有(号“乌石二”,雷州人)\\
		
		白旗帮 &  & 梁宝(号“总兵宝”)
	\end{tabular}
\end{center}

六个帮派中,以红旗、黑旗两帮势力最大,分别拥船三百余艘和百余艘,其余四帮亦各有船数十艘。黑旗帮帮主郭婆带本为郑一部下,后积极招募私人部属,发展出一独立帮派。令人称奇的是,郭婆带并非一个粗野武夫,反而非常喜欢读书,望之如一书生。他的座舰中装满了书籍,每日供他手不释卷地阅读。他还在船头悬挂锦幔,上书“道不行,乘槎浮于海;人之患,束带立于朝”,以黑色幽默的方式说明自己何以不出仕腐朽的清帝国,而是选择当一个海盗老板。郑一在世时,黑旗帮一直与红旗帮维持着同盟关系。然而在郑一于1807年十二月死于台风后,两者即分道扬镳\footnote{郑广南:《中国海盗史》,页304—306}。

郑一死后,红旗帮归入其妻石氏之手。石氏,人称“郑一嫂”、“石香姑”,是个骁勇善战、风姿绰约的美女。当时,石氏坐拥三百余艘战船、一万六千余部众,乃肇庆、广州、惠州、潮州洋面无可争议的霸主。她最为宠信的部将,则是南粤史上大名鼎鼎的张保。张保,人称“张保仔”,生于1786年,本为新会县江门的渔家少年。1802年,他在跟随父亲出海捕渔时被郑一的船队俘获,加入了海盗队伍。据说,因张保仔年轻俊秀,竟同时成为郑一夫妇的情人,三人组成了一个关系复杂的“三角家庭”。郑一去世后,张保仔一跃而为红旗帮的二把手,以郑一嫂的助手兼情人的身份获得了巨大的权威\footnote{郑广南:《中国海盗史》,页306}。

郑一嫂乃一有远见卓识的巾帼英雄。成为红旗帮帮主后,她一改过去掳掠沿海百姓的政策,制定了严格的帮规,要求部众不得任意杀掠百姓、不得强奸妇女,违者格杀勿论。如有人未经其同意私自上岸,一犯者当中刺穿双耳、再犯者立即处死。此外,红旗帮还对“海盗舰队驻地的居民给以各种帮助”,并要求百姓以钱物补偿。如此一来,“盗亦有道”的红旗帮大受南粤沿海百姓欢迎,人们视他们为南粤子弟兵,源源不断地以各种物品资助他们,支援他们袭击清帝国侵略军的斗争。郑一嫂将主力舰队布置于珠江口,控制住广州的外贸航线。至此,红旗帮已成为南粤沿海的实际政府,从而成了清帝国的心腹大患\footnote{郑广南:《中国海盗史》,页306—307}。

当时,活跃于南粤沿海的海盗们大都有与越南阮氏的法式军队交手的经验,装备有大批西式枪炮,武器陈旧、思想愚昧的清帝国侵略军完全不是他们的对手。1808年七月二十一日,狂妄自大的清广东右翼总兵林国良率25艘战船出海,意图一举“剿灭”红旗帮。郑一嫂早已侦知此事,命张保仔率舰队迎敌,两军遇于珠江口孖洲岛(今属深圳)附近海面。张保仔仅以数舰迎敌,而将主力舰队埋伏于孖洲岛一带。战斗一开始,正面应击的数艘红旗帮战舰便佯装败退。林国良不知是计,率全军猛追,在孖洲洋陷入重重包围。张保仔立于船头,率其座舰冲锋在前,直向侵略军冲去。久疏战阵的清军炮手一片慌乱,在张保仔进入射程前便纷纷开炮,炮弹自然都落入海中。愚昧的侵略军见此,以为张保仔是能够躲避炮弹的下凡天神,彻底吓破了胆,舰队阵型大乱,各舰撞作一团。这时,张保仔下令全军开炮齐射,顷刻间便将十艘敌舰打得燃烧起来。至此,清帝国侵略军的士气已彻底崩溃,纷纷跳海逃生。红旗军将士跳上林国良的座舰,当场击毙总兵林国良、游击林道材以下数名敌将,以极微代价取得完胜\footnote{郑广南:《中国海盗史》,页308}。

林国良败死的消息传至北京,嘉庆帝大震,要求侵粤清军务必全力消灭红旗帮。1809年二月,清广东提督孙全谋统率百余艘船只浩浩荡荡自广州南下,图谋与红旗帮决战,两军相遇于万山群岛洋面。面对侵略军大举来袭,郑一嫂临危不乱,将舰队分为四部,以张保仔率船十余艘居中,于左右两翼各置船数十艘侯命出击,自率主力居后。开战后,侵略军先与张保仔的舰队交火。郑一嫂当即发出信号,左右两翼立即包抄而上,合围敌军。清军舰队遭到前、左、右三个方向的攻击,一时疲于应付。这时,郑一嫂亲率大队后军出击,一举击溃敌军。侵略军纷纷弃船逃生,死者无数。此战,孙全谋侥幸逃出战场,红旗帮夺船14艘,取得了第二次大捷\footnote{郑广南:《中国海盗史》,页308}。

同年七月二十一日,红旗帮又与清广东水师展开第三次大战。是月,清广东左翼总兵许廷桂率船六十余艘“进剿”红旗帮,遭遇大风雨,乃泊于香山县桅夹川暂避。郑一嫂侦知敌情,命张保仔率船二百艘进攻。在风雨中,张保仔的舰队对清舰发起突袭,红旗军将士纷纷跃上敌舰,如砍瓜切菜般斩杀清兵。毫无防备的侵略军乱作一团,各自跳海逃生。许廷桂见大势已去,乃于绝望中拔剑自刎。侥幸逃上岸的清兵向石岐方向逃窜,一路大肆掠夺百姓的粮食、财物,激起百姓的巨大仇恨,使红旗帮的民望更为高涨\footnote{郑广南:《中国海盗史》,页309}。

至此,红旗帮三战皆捷,连续斩杀清帝国大将,已摧毁了清广东水师的一半船只\footnote{穆黛安:《华南海盗:一七九〇——一八一〇》,页129}。恼羞成怒的嘉庆帝将两广总督吴熊光撤职,代之以旗人百龄。百龄到任后,下令实行海禁以“断‘贼’粮食”,但根本无法阻断南粤沿海百姓对红旗帮的接济。百龄的海禁令犹如一纸空文,香山、东莞、新会、番禺、顺德等地之民纷纷与红旗帮进行贸易,热情地鼓励他们继续战斗\footnote{郑广南:《中国海盗史》,页309}。正在这时,郭婆带的黑旗帮对珠三角沿岸发动了一连串攻势,使整个局势为之一变。

与郑一嫂不同,郭婆带虽然饱读诗书,却是个对百姓毫无怜悯的恶棍。他的黑旗帮既与清帝国为敌,又残酷地蹂躏南粤百姓。1809年七月二十一日,在红旗帮取得桅夹川大捷后一天,郭婆带率一百艘船只溯珠江口而上,进抵香山县城附近,向当地百姓勒索了一大笔保护费。随后,他的部队在附近的横档村登陆,凶残地屠杀了该村村民。八月十一日,黑旗帮进攻香山县三善村,该村村民在宗族士绅的率领下修筑栅栏、架设火炮,奋起抵抗,但仍告失败。黑旗帮将三善村抢掠一空,并尽杀男子,随后纵火焚村,将妇女儿童全部掳走。十三日,黑旗帮攻陷顺德县马洲村;十六日,陷三山村。两村百姓皆曾在宗族共同体的组织下奋勇抵抗,但都不幸战败,惨遭毒手。十八日,黑旗帮进攻顺德沙湾镇,被乡勇击退。两天后,他们转攻黄连镇,当地乡勇在举人温汝能的率领下死守六日,总算击退黑旗帮,保住了一镇百姓的生命。由于两战皆败,黑旗帮无力再攻,遂向南退走。在为时约一个月的战斗中,倒在黑旗帮屠刀下的百姓多达一万人\footnote{穆黛安:《华南海盗:一七九〇——一八一〇》,页131—133}。

当黑旗帮狼狈败退时,红旗帮在珠江口的活动却越发活跃。九月九日,束手无策的清广东当局向澳门葡萄牙人递交了一份公函,要求葡人出兵协助。由于担心红旗帮威胁到澳门的安全,葡人欣然允诺,于九月十二日派出两艘军舰参战。十五日,葡舰在广州黄埔江面遭遇张保仔的舰队,双方爆发激战。战斗一开始,葡舰强大的火力给红旗帮造成了不小的伤亡,鲜血染红了江水。然而,红旗帮的大批战船死战不退,终于以排炮重创一艘葡舰,迫使其退出战场。另一艘葡舰见势不妙,亦向澳门退去\footnote{穆黛安:《华南海盗:一七九〇——一八一〇》,页138}。澳门葡军的第一次出击,就这样虎头蛇尾地结束了。此后,红旗帮甚至已不将西方人放在眼里。九月十七日,张保仔的舰队劫掠了一艘英国东印度公司的商船,勒索到了一万银元的赎金和两箱鸦片、两箱火药\footnote{戴胜德:《中国南海海洋文化传》,页233}。十一月初,澳门葡人组织了一支有6艘军舰、730名水手和110门火炮的舰队,会同孙全谋指挥的清广东水师对红旗军在珠江口的基地大屿山发动进攻。十一月八日,双方发生首次交火。由于葡人对上次的惨败心有余悸,因此不敢将军舰开得离敌太近,只在远处遥遥射击,结果收效甚微。此后约半个月内,双方进行了数次交火,但并未发生决定性的大战。十一月二十九日,红旗帮不再恋战,放弃大屿山撤往外洋。事后,清帝国官僚大肆吹嘘,称清葡联合舰队在此次“大捷”中击沉“海盗”船20艘、俘获6艘、打死“海盗”1400人。然而据参战的西方人估计,红旗帮在此战中不但一船未失,且只损失了40人\footnote{穆黛安:《华南海盗:一七九〇——一八一〇》,页140—143}。在大屿山狠狠地戏弄了清帝国和葡人后,张保仔向澳门送去了这样一封威胁信:

\begin{quote}

眼下,我有众多的船只和充足的粮草,这些物质足够用于我等日常生活与行动的开支。我现在什么都不缺……此刻,我请求你们给我4艘由你们的人指挥的军舰,当我需要的时候由我调动……(当我占领整个清帝国时)我将赠给你们——我的兄弟……两到三个省的回报\footnote{穆黛安:《华南海盗:一七九〇——一八一〇》,页143}。

\end{quote}

从此封信的语气来看,张保仔认为清帝国已不配做自己的对手,只配被自己消灭。只有葡萄牙人才是有资格和自己对等谈判的人。从他声明要消灭残暴愚昧的清帝国并给葡人“两到三个省的回报”的话语中,我们能感受到粤人一种昂然挺立、视岭北帝国如无物的雄豪气魄。这一声明的气势,足可与16世纪时张琏“尘埃亦知天子哉”一语相提并论(关于张琏,详见第十二章第二节),堪称19世纪初南粤发出的最强音。假若张保仔能以这样的气势继续与清帝国战斗下去,那么南粤将很可能获得自由,被清帝国压迫的东亚诸邦亦很可能获得解放。然而,局势却很快变得急转直下,黑旗帮和红旗帮间爆发了激烈的战斗。

当红旗帮在大屿山奋战时,张保仔曾向黑旗帮求援。毫无政治德性的郭婆带按兵不动,对红旗帮与清葡舰队的战斗采取作壁上观的态度。在郭婆带眼中,反清不过是他捞取政治资本的方式。只要清帝国向他开出合适的价码,他就愿意接受“招安”。1809年底,郭婆带将自己的五十名家属送往故乡雷州海康,令他们将自己的投降意图告知清政府。同时,他又派出百余名部下分头前往新安、阳江,与清帝国官僚讨论投降事宜。十二月十二日,郭婆带率大批战船在虎门洋面突然袭击了张保仔的舰队。张保仔猝不及防,大败而退,红旗帮伤亡惨重,被俘321人,还有16艘战舰被掳。这样一来,郭婆带便通过杀戮自己的同胞获得了向清帝国投降的“投名状”。1810年正月十三日,郭婆带率黑旗帮全军,连同被他引诱的部分黄旗帮部众在归善县向百龄投降,向清帝国交出了5578名兵士、800名妇女儿童、113艘帆船和500门火炮\footnote{穆黛安:《华南海盗:一七九〇——一八一〇》,页146}。大喜过望的百龄欲授郭婆带把总之职,令其率旧部“进剿”其余海上武装。然而,郭婆带却拒绝了百龄的要求,表示自己无意继续作战。无奈之下,百龄只好同意。此后,郭婆带改名郭学显,以平民身份生活在广州城中,每日读书治学,悠然终老\footnote{郑广南:《中国海盗史》,页305}。他的这种平静生活无疑是通过屠杀同胞、屈身事敌换来的。而他不但对此毫无悔过之意,反而乐在其中,可以说是个极度无耻的粤奸。

黑旗帮投降后,清帝国得以全力“攻剿”红旗帮。虎门惨败后,红旗帮将士士气低落,无心迎敌。这时,郑一嫂因自己年事已高,精力大不如前,也丧失了继续战斗下去的信心。为了保住将士们的生命,郑一嫂决定与清广东当局商谈投降事宜。1810年初,在与红旗帮有来往的澳门医生周飞熊的奔走下,双方开始联络。郑一嫂、张保仔率舰泊于虎门沙角,准备与清帝国官员面谈。百龄先命紫泥道巡检章予之前往沙角,但郑一嫂以章予之官职太低为由拒不相见,表示必须由百龄亲自出面,展现了虽然欲降而不屈的态度。百龄见红旗帮实力强大,只得于正月十八日硬着头皮单舟赴会。郑一嫂下令全军全副武装,树旗鸣炮以迎百龄。百龄被红旗帮盛大的军容和轰鸣的炮声吓得瑟瑟发抖,直到被郑一嫂以礼迎上船才镇定下来。正当双方谈判之时,忽有十艘装载火炮英国商船驶至虎门。郑一嫂以为此乃被清军招来偷袭红旗帮的战船,果断下令中止谈判,将百龄放走,率舰队扬长而去\footnote{郑广南:《中国海盗史》,页316}。

不久后,郑一嫂明白发生了误会,遂于三月十四日亲赴广州会见百龄。双方在谈判桌上唇枪舌剑地争吵了三天,郑一嫂坚持要求必须给张保留下一支足够强大的舰队用于贩盐。对于百龄的反驳,她一概无视,并暗示如果百龄不同意她的要求,她就会率红旗帮继续战斗。百龄无计可施,只得同意\footnote{穆黛安:《华南海盗:一七九〇——一八一〇》,页150}。三月十七日,红旗帮全军在香山县芙蓉沙降清,交出17318名兵士(绝大部分为广东人,另有220名闽人、43名越南人、13名吴越人、7名广西人、1名赣人、1名湘人、1名蜀人)、226艘帆船和1315门火炮,大部分降者随即被打散建制编入清军各营。不过,郑一嫂被允许连同四千余名部属居住于广州城内,张保仔则得以二三十艘战船和部分部众,摇身一变成为清军参将。此外,郑一嫂和张保仔还获准正式结婚,这标志着他们两人的爱情终于修成正果\footnote{穆黛安:《华南海盗:一七九〇——一八一〇》,页151;郑广南:《中国海盗史》,页316}。

黑、红旗两帮相继降清后,南粤沿海的反清武装仅剩粤西的蓝、黄、青、白旗四帮,它们的实力都远不如黑、红旗帮。1810年四月,百龄及清广东巡抚韩崶组织起一支有130艘战船、一万兵力的西征舰队,张保仔的二三十艘船被编入其中。同月,该舰队在七星洋、放鸡洋先后轻易地消灭了黄旗帮一部和青旗帮全军\footnote{郑广南:《中国海盗史》,页317—318}。五月十二日,该舰队在海南儋州洋面与四帮中实力最强的蓝旗帮展开决战。蓝旗帮帮主麦金有原为西山政权重臣,其帮内有不少越南人。该帮极度残忍,经常在粤西沿海及海上屠杀平民、抢掠财物。对于这伙蹂躏同胞的歹徒,张保仔十分反感。因此,在决战中担任先锋的他进攻得异常卖力。此次海战一直持续到第二天。经清军各部“连环攻击”,蓝旗帮伤亡惨重,有十余艘战船被焚。在战斗的最后时刻,张保仔跃上麦金有座舰,亲自擒获了这个魔头。此战,清军俘获兵士362名、帆船86艘、火炮291门,并救出128名饱受奸淫虐待的妇女\footnote{穆黛安:《华南海盗:一七九〇——一八一〇》,页152}。麦金有等八名头目被押往雷州双溪港凌迟处死,另有119名俘虏被斩首。至此,凶残的蓝旗帮宣告覆灭。见蓝旗帮灭亡,黄旗帮帮主吴知青自知无力抵抗,乃率残部3400余人降清,白旗帮则远走菲律宾\footnote{郑广南:《中国海盗史》,页318}。曾经控制整个南粤沿海的旗帮海盗,就这样或降、或死、或逃,全部停止了反清活动。

“平定”旗帮海盗后,清廷依然忌惮张保仔的威势,便卑鄙地对其名升暗降,令他以闽安协副将的身份移驻澎湖\footnote{郑广南:《中国海盗史》,页318}。从此,他离开了南粤,沦为失去根基和乡土共同体的清帝国流官。唯一令人欣慰的是,与他深深相爱的郑一嫂也随他一同前往澎湖,使他不至太过寂寞,并为他生下一子一女。1822年,年仅36岁的张保仔在郁郁寡欢中病逝于澎湖。郑一嫂多活了二十二年,于1844年去世。此对纵横海上的南粤英雄,便这样落寞地谢世了。假如他们当年没有选择投降,或许南粤早已重获自由,他们亦绝不会落到如此悲凉的下场。他们为保全部下生命而做出的投降举动固然不应被过分苛责。然读史至此,我们依然会止不住地惆怅,深切地惋惜又一次错失机会窗口的南粤。

旗帮海盗虽然皆是反清武装,但他们当中鱼龙混杂,既有红旗帮这种军纪严明的队伍,亦有如黑旗、蓝旗帮般蹂躏百姓的恶棍。因此,我们不应对他们做出完全肯定的评价。但必须指出的是,在他们控制南粤制海权的1810年代,南粤沿海出现了一种崭新的社会秩序。当时,沿海各地出现了许多与他们合作的港口。这些港口中布满供海盗“销赃”的商号,亦有大批出售粮食、武器和维修船只的店铺,从而在陆地与海洋之间形成了一个庞大的地下经济体系,吸引了大批从业人口。旗帮海盗消失后,这一经济体系遂转变为一张庞大的走私网络,使粤人夺回了被清帝国通过广州“一口通商”霸占的南粤外贸权。由于数以万计的旗帮海盗被编入了清广东水师,南粤的清军亦加入到这一经济体系中,包庇乃至从事走私贸易。对于这一现象,清帝国束手无策,只得听之任之\footnote{徐承恩:《躁郁的城邦:香港民族源流史》,页86}。随着走私活动日渐高涨,一种特殊的商品大量涌入南粤市场,并将引发一场震撼南粤乃至整个东亚的碰撞。这一商品便是鸦片。这场战争,便是人尽皆知的鸦片战争。













