\chapter{文明开化:南粤的大航海与西化}

\section{香港城邦的建立}

1841年1月25日,英国皇家海军“硫磺号”舰长拜尔秋(E. Belcher)率队在香港岛西北角的上环水坑口登陆,将之命名为“占领角”。次日上午8时15分,英军在“占领角”升起英国国旗,为维多利亚的女王三次欢呼。英国对清贸易商务总监查理·义律随即在“威里士厘号”军舰甲板上宣布,香港已是女王陛下的领地\footnote{弗兰克·韦尔什:《香港史》,页128}。英治香港城邦伟大的历史,由此拉开序幕。

香港所处的位置得天独厚,正在南粤各族群势力范围的交界点上。港岛北面、西面为珠三角,为广府人聚居区。东北则为粤东丘陵地带,乃客家人聚居区。此外,香港东北沿岸则靠近福佬人聚居的海丰、陆丰。自18世纪起,港岛东北面的长洲岛上便居住着一批福佬移民。至于沿海及珠江口的疍民亦十分青睐香港岛,时常在此避风。从一开始,香港城邦便为南粤各族提供了一处自由的场所,并命中注定成为多族共存之地\footnote{徐承恩:《躁郁的城邦:香港民族源流史》,页30}。

当英人占领港岛时,岛上仅有一些零星的疍民和靠采石为生的客家村落,居民共计3650人。英国人在岛上的最初定居点位于其登陆点略东的地方,其后扩展至西环、上环、中环、金钟、湾仔、铜锣湾一线,称“域多利城”。岛上的客家居民成为香港城邦最早的建筑工,为城邦修建了最初一批西式建筑。1841年6月7日,义律宣布香港为自由港,无需向英国政府支付任何款项\footnote{弗兰克·韦尔什:《香港史》,页165}。这表明从一开始,香港就是自由之地。1841年8月,义律被召回国,英方以砵甸乍(又译“璞鼎查”,Henry Pottinger)代之。1842年8月29日,英清签订《南京条约》,清帝国正式将香港岛割让予英国。砵甸乍随即于10月27日发出如下告示:

\begin{quote}

香港乃不抽税之埠,准各国贸易,并尊重“华人”习惯\footnote{《广东文史资料》,页2}。

\end{quote}

此通告示表明,香港从此将成为零税率的自由贸易港,来此定居的粤人的生活方式也将受到尊重。1843年4月5日,维多利亚女王发布任命砵甸查为香港总督的制诰。该制诰于6月抵港,砵甸查遂成为首任港督。他当即组织行政局、立法局、最高法院,按照西方三权分立的原则建立起香港的最高行政、立法、司法机构。港督皆由英国政府委派,行政局、立法局议员则由香港政府委任。城邦建立之初,港岛上有300余名英国人。他们中地位最高者为港督和行政局中的三名官守议员,其下则有43人任“太平绅士”(Justice of the Peace, JP)\footnote{弗兰克·韦尔什:《香港史》,页172}。“太平绅士”系一种源自英国的制度。出任太平绅士者皆为民间德高望重之人。他们坐镇地方法庭,肩负起维护社区治安、防止法外刑罚的责任,亦要处理一些简单的法律程序。此时,香港的粤人尚无任何政治权力,亦无政治义务。

1844年,香港的粤人第一次在城邦政治中发出了自己的声音。是年8月21日,立法局通过法例,要求所有居民每人每月缴纳1元的“人口登记税”。对于居住在香港的英国人来说,这笔钱不算多。但对月入仅2至3元的粤人苦力、建筑工来说,这样的税负实在是太重了。11月1日,香港的粤人发起浩大的罢工、罢市活动。十二天后,第二任港督德庇时只得宣布取消“人口登记税”。在粤人的奋斗下,香港的全体居民无分种族从此只需向政府登记,不再此税\footnote{徐承恩:《躁郁的城邦:香港民族源流史》,页96}。因此受益者不单只有粤人,亦有定居香港的西方人。相应地,英国人亦在法律上给予粤人较高的地位。1845年,一个美国人曾指出:

\begin{quote}

只有在香港和澳门殖民地,欧洲人才会因为谋杀了“华人”而被处死\footnote{转引自弗兰克·韦尔什:《香港史》,页195}。
\end{quote}

1850年代,太平天国之乱、洪兵战争、土客战争相继爆发,大批南粤难民涌入香港,使这座城邦的人口迅速膨胀。1853年,香港人口膨胀至39017人\footnote{《广东文史资料》,页3}。1856年10月,第二次鸦片战争爆发,香港的安全受到了首次考验。开战后,杀人魔王叶名琛下令封锁港岛,煽动南粤民众向香港停运食物,并派人潜入香港发动恐怖袭击。1857年1月15日,叶名琛派出的细作在全港唯一一间面包房裕盛馆投放砒霜,导致许多西方人大作呕吐。2月28日,英商在湾仔新设的面包房又被人纵火焚毁。此外,有两艘英国商船分别在1月13日和2月23日遭恐怖分子焚毁,船上西方人全部惨遭杀害。当时,香港街上出现了许多来路不明的反英标语,身份不明的暴徒更频频袭击西方人的产业,甚至曾试图行刺港英政府官员。一连串恐怖事件下,港岛陷入恐慌中\footnote{徐承恩:《躁郁的城邦:香港民族源流史》,页132—133}。

然而,到4月之后,恐怖袭击便逐渐消失了,对香港的封锁也停止了,这是因为港岛北面的新安县居民已经厌倦了清帝国的煽动。在深圳河北岸,皇岗、深圳等村镇的居民纷纷无视清帝国煽动,继续向港岛运送补给。在深圳河南岸的荃湾,更有村民绑架拒绝向港岛提供物资的士绅,要求其对港恢复供应。在港岛上,粤人全力支持英军,许多人都入伍充当民夫。至于受雇于西方人的粤人,亦不断向雇主通风报信,令清帝国恐怖分子谋害西方人的大部分阴谋无法得逞。在南粤人与英国人令人感动的全面合作下,港岛变为清帝国无法渗透的自由孤岛,解除了危机\footnote{徐承恩:《躁郁的城邦:香港民族源流史》,页133}。1860年10月,第二次鸦片战争结束,清帝国在《北京条约》中将九龙半岛南部及邻近之昂船洲割予英国,香港城邦的范围由此扩展至大陆之上,其北界相当于今日的界限街。香港城邦由此获得更大的地理纵深,其地缘形势更加安全。

第二次鸦片战争后,第五任港督罗便臣(Hercules Robinson)开始着手进行政治改革。此前,香港城邦缺乏一批了解南粤民情的专职公务人员,港英政府中弊案丛生。在罗便臣的推动下,自1862年起,香港城邦开始推行“官学生制度”,规定唯有通过公开考试选拔的优秀英国大学生方能至港英政府工作。这些被选中者在从政之前,需先在香港进行为期两年的汉文学习。这批“官学生”熟知南粤文化,很快便成长为能够跟粤人顺利沟通的专业政务官。但对一个城邦而言,仅有专业的行政官员是远远不够的。至1865年,香港人口已膨胀至12万以上,其中大部分是南粤人。这些人口必须自己组织起来进行社区自治,方能为自己的共同体在城邦生活中谋求更多权利,并降低港英政府的治理成本。为躲避战乱涌入香港的大批南粤士绅、豪商成为组织粤人自治的凝结核。1865年,从事沿海及南洋贸易的在港粤商组建南北行公所,用于裁定成员的商业纠纷。两年后,在南北行公所的要求下,第六任港督麦当奴(Richard Macdonnell)批准其组织“团防局”并招募一批全职更练自行维持粤人社区治安。1868年,南北行公所在文咸西街获得一块由港英政府批出的土地,于此建立会所。此后,该会所便成为在港粤商与民众商议公共事务的场所,形同粤人自己的议会\footnote{徐承恩:《躁郁的城邦:香港民族源流史》,页148—149}。至此,南北行公所已不再是一个简单的商人行会,其成员也不只有商人,更有各行业工会的领袖、洋行买办和街坊领袖\footnote{徐承恩:《躁郁的城邦:香港民族源流史》,页152}。1870年,南北行公所又开始筹办医院,使香港粤人在的自治进程再进一步。

香港开埠以来,在港粤人并无入医院就医的习惯,患病者往往被遗弃在街头。1851年,商人谭亚才在太平山街设立广福义祠,用于供奉济公活佛及客死于香港者。许多被遗弃在街头的病患遂聚集至义祠,领取基本的茶水和饮食,静候死亡,祠内臭气熏天,遍布死尸。1869年4月,有港英官员造访广福义祠,对其中恶劣的环境深感震惊,乃督促立法局尽快设立“华人”医院。1870年,立法局通过《东华医院条例》,要求南北行公所成立一个20人的筹备委员会筹办医院。香港的粤人精英立即翻修广福义祠,将其做为委员会会址,并把委员会人数扩充至125人。1872年2月14日,完全由“华人”运营的东华医院在上环普仁街开张。该院为病人提供住院服务,其大部分医疗服务为中医,亦有为民众接种牛痘的西医服务。医院总理共11人,既有买办,亦有各行业豪商。由于担心医院像广府义祠那样变成收容街头患病贫民的场所,东华总理随即介入济贫事业,出资协助流落香港的无业游民返乡,并设立东华义庄收容死者遗体。此外,东华医院还开设孤儿院,并收容精神疾病患者。这样一来,东华医院便不仅是一所普通的医院,更变成为在港粤人提供社会服务的中心,因而在粤人中获得巨大威望。根据规定,东华总理每年进行一次选举,凡能捐资10元者皆可参选。换言之,香港各行业中等收入以上的粤人皆有参选资格。当选者不但要对医院、义庄进行管理,还经常召集特别会议处理社区内的突发事件、审理民事纠纷。至此,东华医院已成为在港粤人的议会。1877年,第八任港督轩尼诗(John Hennessy)就职。轩尼诗系爱尔兰人,在英国属弱势族群。这一出身使他颇同情在港粤人,并十分尊重东华医院,东华医院遂成为粤人与港英政府接触的唯一渠道。粤人民众的诉求多由东华总理向港督当面转达,港督亦时常就城邦政务问题咨询东华总理。1878年,东华总理成立打击人口贩卖的机构“保良局”。该局设于东华医院内,其董事会与医院在人事上多有重叠,人称“东保一家”。此后,一种新的习惯法形成了:刚刚步入政坛的粤人需先在保良局任职,方能参选东华总理\footnote{徐承恩:《躁郁的城邦:香港民族源流史》,页154—157}。

东华医院和保良局使粤人在香港的政治地位极大提升。港英政府意识到,南粤人是完全有能力进行自我组织、社区自治的优秀族群。1878年,港英政府开始委任粤人为太平绅士。同年,港英政府开始允许“华人”宣誓加入英国国籍。1880年,轩尼诗又任命出生于英属马六甲、由英国留学归来的粤人伍廷芳(祖籍新会)为立法会议员,是为香港的首个粤人议员。虽然在轩尼诗于1882年离职之后,港英政府不再准许东华总理接受民众诉讼。但粤人已通过东华医院为跳板进入香港城邦的主流宪制中,东华医院的历史使命已告完成\footnote{徐承恩:《躁郁的城邦:香港民族源流史》,页158}。

对于南粤本土的粤人来说,此后的在港粤人便属于另外一个共同体中的重要组成部分,南粤和香港沿着不同的道路分离了。然而,香港和澳门一样都是南粤与西方文明结合而成的优秀结晶。若无英国人带来伟大的西方秩序,香港城邦定然不会存在;若无广府、客家、福佬、疍家等南粤主要族群聚集在香港这片自由的土地上做出优秀表现,这座以粤人为主要人口的城邦也绝不可能有今日的伟大与荣光。香港虽和南粤分离了,但它和澳门一样永远是南粤最亲密的亲邦。在20世纪,香港城邦将一次次为南粤文明保留种子,并在关键的历史节点中反哺南粤,在惊涛骇浪中推动南粤文明向前迈进。

\section{英军占领新界与广州湾租借地的诞生}

1895年,清帝国在日清战争中惨败于日本之手,西方各国掀起瓜分清帝国的狂潮。在南粤,英法两国皆展开行动。1898年6月9日,英清在北京签订《展拓香港界址专条》,议定英国向清帝国租借香港九龙界限街以北、深圳河以南地方及附近200余个离岛,租期99年,至1997年6月30日期满。此处租借地被称为“新界”,极大地向北扩展了香港城邦的面积。1899年4月中旬,英军进驻新界,随即遭到当地宗族乡勇的激烈抵抗。这是因为英人直到该年4月7日至9日才在新界各村张贴公布接收详情的告示,新界本土的宗族深觉受到侮辱。4月14日至19日间,英军与乡勇在大埔、沙田、粉岭、屏山、石头围、吉庆围等地展开了一场短暂而猛烈的战役,史称“新界六日战争”。此战,南粤乡勇以简陋的武器视死如归地向英军阵地冲击,遭炮火重大杀伤,阵亡500余人,不得不向英军投降,而英军的损失不过是轻伤1人。经过此番战争,英军认识进一步认识到了南粤人的武德,将战死乡勇郑重地集体埋葬于沙埔公墓。至此,香港城邦延续至今的地理范围宣告定型\footnote{关于新界六日战争的详细起因、经过、影响,可参看夏思义:《被遗忘的六日战争:1899年新界乡民与英军之战》}。

直到此时,清军仍在深圳河南岸留有一支武装,那便是与香港岛隔维多利亚港相望的九龙城寨中的300名守军。5月14日,第十二任港督卜力(Henry Black)派兵进占九龙城寨,将守军全部缴械。当时,清军正在深圳河北岸集结兵力,蠢蠢欲动,英军遂进一步北上,于5月16日分三路跨过深圳河,在下午5时45分攻占有两千人口的深圳镇。一名英军将领如是描述深圳:

\begin{quote}

深圳周围的村庄是肥沃富庶的,它比其他市镇好,比九龙城好\footnote{转引自鲁言等:《香港史话(第六集)》,页97}。

\end{quote}

卜力十分希望将深圳划入香港城邦中,他认为只有这样方能确保香港北面山区的安全。5月26日,他计划攻占新安县治所在地南头城,但以兵力不足作罢。在英国政府看来,卜力的行为已然违犯条约。6月13日,卜力接到英国政府要求其将深圳交还清帝国的通知。无奈之下,他再三拖延,只得于11月2日下令英军放弃深圳。11月13日,英军整队撤离深圳,退回深圳河南岸\footnote{鲁言等:《香港史话(第六集)》,页97—99}。就这样,深圳黯然失去了被并入香港城邦的机会。

与英国夺取新界、进攻深圳同时,法国亦在粤西发起攻势。1898年4月9日,在法国的强大压力下,清总理衙门同意将雷州湾租予法国,租期99年。此后,清帝国欲与法方会勘界址,法方却抛开清帝国独自行动,先派出海军在雷州、高州两府沿岸骚扰,不断攻击清方炮台,又于同年10月攻占洋洲、东海二岛,封锁了高州、雷州、廉州、琼州、钦州五府的出海要道。1899年3月13日,法国驻清公使毕盛(Stephen Piehon)向清总理衙门单方面提交分界地图一幅,要求将雷州湾东西120里、南北100里之地,即包括洋洲、东海二岛及吴川、遂溪两县大半辖区的地域划为法界,统称之为“广州湾租借地”。逼迫之下,清帝国只好屈从。1899年11月19日,法远东军司令高礼睿与清广西提督苏元春在法舰上签订《广州湾租界条约》,不但规定陆上面积518平方公里、海上面积2130平方公里的广州湾租借地全归全归法方管理,更允许法国在租界内驻兵、设炮台、铺设电线\footnote{司徒尚纪:《雷州文化概论》,页117}。

广州湾租借地中有一良港,即今日的湛江港,其南侧海面称广州湾。当时,广州湾租借地内已有一赤坎港,然不具城镇形态,其它区域亦无城镇。法国人占领广州湾后,随即在海滨规划现代城市西营(今湛江市霞山区)。仅二十余年,西营便发展为繁荣的港口城市。当时的报刊曾如是说:


\begin{quote}

西营地方虽小,但那街道之整洁雅致可就足以令你惊叹不已。那些街道是那样的宽宏和雅静,短的红墙,院内院外的花木是那样的栽植得恰到好处\footnote{此为民初《申报》之报导,转引自司徒尚纪:《雷州文化概论》,页118}。

\end{quote}

法人在广州湾租借地建设了西式医院、邮局、银行、电台、海关、教堂、商店、学校等设施,对当地的现代化贡献良多\footnote{司徒尚纪:《雷州文化概论》,页119}。租借地被划归法属印度支那邻邦,归法国越南总督管辖,其最高行政长官为法国正副公使各一人。公使署内设有约二十名俗称“法国师爷”的工作人员,多为越南人,亦有少量南粤人。公使署初驻于东营(麻斜),后迁至西营。租借地的驻兵则有“红带兵”、“蓝带兵”之分。其中红带兵系法国国防军,驻防西营市区;蓝带兵则为由越南骨干指挥的粤人,负责地方防卫。必须指出,广州湾租借地不似香港城邦那般重视粤人权利,本地粤人时常受越南军官殴打、欺压。此外,法人亦未能很好地维持租借地治安。因法人的无能,1920、1921年曾发生过土匪接连洗劫遂溪县城、雷州城,杀害千余民众的惨剧\footnote{李剑连:《法国殖民时期的广州湾社会》}。英、法两国的差距,由香港城邦、广州湾租借地的差距上即可一览无余。

无论如何,广州湾租借地毕竟是西方世界向南粤输出秩序的产物,极大地推动了粤西的现代化进程。至1900年,南海之滨、珠江之畔已有香港、澳门、沙面、广州湾这四块由西方人完全控制的土地。它们象征着南粤与西方世界的水乳交融,亦象征着南粤在国际舞台上的文明开化。然而,南粤人绝不仅仅被动地接受西方秩序输入。在19世纪,粤人乘风破浪地向世界各地展开了伟大的航海,将南粤的种子撒向世界的每个角落。


\section{大航海:粤人走向全球}

在悠久的南粤历史上,我们的伟大祖先一直在海洋中进行着壮阔的探险,曾创造出惊人的航海成就。如前所述,在14至18世纪间,南粤人曾先后在苏门答腊、中南半岛等地建立过旧港、港口国等独立政权,深刻影响了东南亚历史的走向。南粤人在近代的航海成就当然远不止此。除建立海外政权外,南粤人还向世界各地大举移民,使南粤文明向世界的每一个角落传播。

自19世纪中期起,随着广东人口的膨胀,大批珠三角的广府人沿着西江大举西进,前往广西定居。他们多从事粤桂间的西江贸易,拥有丰厚的资金,将桂东南的贵港、南宁、藤县、北流、崇左等地和桂西的百色一带变为粤语的世界。在短短数十年内,广西便形成了桂东南广府、桂东北桂柳及桂西北人三族鼎立,客家人、平话族群点缀其间的格局。至于海洋,则是远更宽广的天地。在南海,东南亚各国与南粤人一直共享着ni 片海域及其上的美丽岛屿,西沙群岛和南海群岛一直是各国商船的重要停靠站,还有南粤船民居住在南沙群岛上。在泰国,粤人移民一直颇多。1581年,曾有粤东海盗林道乾到达泰国南部的穆斯林城邦北大年,被该城苏丹召为驸马。林道乾之所以能顺利浮海至北大年,与当时粤人对粤泰间航路之熟悉度颇有关系。在16世纪,当时的泰国首都阿瑜陀耶已有众多粤、闽移民居住,出现了“华人街”。此外,在泰国湄南河下游有一岛屿被西方人称作“唐人岛”,岛上满是粤人移民。到17—18世纪,粤人移民已遍布泰国各主要城市。除阿瑜陀耶和北大年外,北览波、万佛岁、北柳、柴真、万岑、六坤、普吉岛等地皆有粤侨居住,其中广府、客家、福佬三族皆有\footnote{《广东通史》古代下册,页595}。18世纪泰国伟大的民族英雄郑信,便出身于福佬商人家庭。粤侨在当地既入乡随俗、学习泰语、并与泰国人通婚,又保持着自身的风俗和语言。1782年,拉玛一世弑郑信登基,建立延续至今的却克里王朝。拉玛一世在位时,为营建新都曼谷的宫殿、庙宇,并在战乱后发展农业手工业,泰方在南粤大力招工,每年仅在粤东便有数万人移民泰国。至1850年前后,泰国粤侨、闽侨已超过100万。到1830年代,曼谷的40万居民中竟有20万为粤人或闽人。史载,当时泰国的人口构成情况是:

\begin{quote}

暹罗流寓,闽、粤人皆有之,而粤为多,约居土人六分之一\footnote{徐继畬:《瀛寰志略》}。

\end{quote}

可见,到19世纪中期,粤人和闽人已遍布泰国各地,其中又以粤人为多,达到泰国人口的六分之一。泰国人与粤人关系十分友好,互相视为“同胞兄弟”。勤劳、聪慧的粤人之中不但富商辈出,亦有许多人进入泰国宫廷,身居高位并参与朝政。此外,南粤的航海技术亦深刻影响了泰国。泰国人大量雇佣南粤水手,并仿照南粤的样式成批地建造出海商船\footnote{《广东通史》古代下册,页1061—1062}。

在婆罗洲(加里曼丹),南粤客家人自18世纪中叶起成批涌入当地,大力从事金矿开采,每年迁至婆罗洲西部者达3000人\footnote{《广东通史》古代下册,页1062}。18世纪后期,嘉应客家人吴元盛、罗芳伯到达婆罗洲,使当地局势为之一变。吴元盛系天地会会众,曾于家乡策划反清活动,后因事泄,遂逃至婆罗洲西北部的土邦戴燕王国。当时,戴燕国王为政残暴,百姓怨之。正直的吴元盛便联络民众发动起义,于1783击毙暴君,被起义军民推上王位。戴燕王国境内既有土著,又有粤人,两者皆感激吴元盛“为民复仇”的义举,对其十分敬服\footnote{梁启超:《中国八大殖民伟人传》}。吴元盛病逝后,其妻、子吴德奎、孙吴广淮相继继位,直至19世纪中期王国被荷兰人灭亡。

与吴元盛相比,他的下属罗芳伯开创的伟业更为惊人。罗芳伯系嘉应程乡客家人,亦为天地会会众。他原名罗芳柏,因性格豪爽、重义气,被友人尊称为“罗芳伯”、“罗大哥”。罗芳伯幼时饱读诗书,又精通剑术,还是个种田、放牧的好手,可谓文武全才。1772年,时年三十五岁的他在乡试中落第,遂投入天地会,率百余名族中兄弟及会众怀着昂扬的“壮游之志”从虎门出海,经海南岛、西沙群岛、菲律宾、赤道,一路航海至婆罗洲西岸的坤甸,在此居住下来,并脱离吴元盛的管辖。当时,坤甸一带的粤人组织了多个采矿团体,时常互相争斗。有胆有识的罗芳伯依靠其高尚的道德获得了当地粤人的尊重,被众人一致推举为领袖。此外,罗芳伯又用其军事才能赢取了当地穆斯林土著的信任。当时,坤甸一带的苏丹正饱受叛乱部下的困扰,只得出资请求罗芳伯出兵相助。罗芳伯采用“明修栈道,暗度陈仓”之计杀敌甚众,大破叛军,因而被苏丹视为救命恩人,得以出入其宫廷。在宫中的宴席上,苏丹曾对罗芳伯这样说:

\begin{quote}

居有大动我族,顾约为兄弟,世世子孙毋相忘也\footnote{温雄飞:《南洋华侨通史·罗芳伯传》}。
\end{quote}

经此一役,当地土著酋长多视罗芳伯为领袖,请求受其保护,苏丹权威日渐衰落。罗芳伯对土著与粤人一视同仁,因而受到两者的一致敬爱。1775年,在粤人和土著的一致推戴下,罗芳伯以坤甸附近的小镇东万律为首府,组建兰芳公司。1777年,他又将兰芳公司改名为“兰芳大总制共和国”,定当年为“兰芳元年”,采取总统制,自称“大唐总长”、“大总制”。亚洲历史上的第一个共和国,就这样被南粤人建立起来,在南半球靠近赤道的地方诞生了。

兰芳共和国采取民主制度,“国之大事皆众议而行”,总长及重要官员皆由选民推举产生。若选民认为他们无能,可以发起弹劾。在总长竞选活动开始前,总长有权向国民推荐数位候选人。在新总长被选出之前,由副总制代行大总制职权。值得一提的是,兰芳共和国没有成文宪法,完全依靠南粤和婆罗洲土著的习惯法处理国内事务,其对自发秩序的尊重着实令人赞叹\footnote{James R.Hipkins:《婆罗洲华人史》}。

兰芳共和国的疆域西、北至海岸线,南至坤江,东至万劳,与东面的戴燕王国接壤。吴元盛夺取戴燕王位后向兰芳共和国称臣,成为其藩属国。在国内,罗芳伯大力改进农耕技术,发展农业和采矿业,并创办学校,组织青壮年进行制度化的军事训练。国内丁壮平时为民,战时为兵。该国还设有能够生产西式武器的兵工厂,制造大批枪炮用于国土防御。兰芳十九年(1795),深受爱戴的罗芳伯在担任总长的第十九年去世,享年58岁。临终时,他推荐文武全才江戊伯为继承人,得到选民的同意,使国内的第一次政权交接平稳过渡。此后,江戊伯、阙四伯、宋插伯相继担任总长,维持着蒸蒸日上的国势。兰芳四十七年(1823),第五任总长刘台二就职,兰芳共和国开始走向衰落。

自兰芳四十五年(1821)起,荷兰人开始大举进入婆罗洲,占据该岛东南部地区。兰芳四十八年(1824)初,刘台二赴荷兰东印度公司总部巴达维亚进行外交访问,随后在荷兰人的利诱下签订了一项互不侵犯条约,承认荷兰东印度公司对兰芳共和国享有“宗主权”。次年,荷人又策动兰芳共和国内的马来土著发动两次叛乱,使共和国蒙受惨重损失。此后,共和国内的土著叛乱频频,政府权威日渐衰落。19世纪中期,随着与共和国接壤的戴燕王国亡于荷人之手,共和国已面临着事关生死存亡的威胁。兰芳一百零八年(1884),第十二任总长刘阿生病逝,荷军随即开入兰芳共和国将其政权推翻,将共和国的土地分割给了几名土著酋长。共和国的一批残民逃至苏门答腊,其后向北迁移,定居于马来半岛。1912年,荷人宣布吞并西婆罗洲\footnote{James R.Hipkins:《婆罗洲华人史》}。至此,曾在南半球大放异彩的兰芳共和国烟消云散。

兰芳共和国虽然灭亡了,但这个经历了108年历史、12任总长的共和国却在南粤史上留下了无比壮阔的印记。它是南粤人建立的首个共和制政权,在南粤史和亚洲史上都有深远的意义。它表明,在18世纪后期,当岭北的帝国降虏们还不知近代文明为何物时,南粤人已有能力组织起总统制共和国了。这一伟大共和国不但是南粤人永远的骄傲,亦是我南粤史上一座伟大的丰碑,更是南粤人在大航海时代创造出的文明奇迹。

在马来亚,英国人自1786年夺取槟榔屿后便广事招徕南粤移民。1794年,当地有3000名粤人,至1830年已膨胀至8963名,当地“所有木匠、铁匠和鞋匠都是广东人”。1795年,英人自荷兰手中夺取马六甲,当地原本仅有千人左右的粤人、闽人数量迅速增加,至1834年已达4143人。至于马来 半岛南端的小渔村新加坡则更成为一个奇迹。在1819年英国人夺取新加坡之前,当地只有居民百余人。1819年后,在英人的招徕下,大批粤人、闽人航海来到新加坡从事开发事业,人数逐渐超过了当地的马来人。到1840年,新加坡已成为一座繁荣的城市,其3.5万人口中有1.7万系粤人、闽人,又以南粤四邑(新宁、开平、恩平、新会)、香山人为多。在19世纪中期,粤、闽流入南洋各地的移民共计100多万人,其中有60—70万是南粤人。称当时粤侨已遍布东南亚各地,是一点也不夸张的\footnote{《广东通史》古代下册,页1062—1063}。在亚洲,日本亦hay 粤人移民的一大目的地。自1871年《清日修好条约》签订后,大量粤人便落籍日本,以横滨最多,东京、神户次之。1898年,在日粤侨成立统合性组织“宗仁会”,其下又有“三邑公所”(南海、番禺、顺德)、“四邑公所”(新宁、开平、恩平、新会)、“要明公所”(高要、高明)\footnote{《广东通史》近代下册,页844}。直到今天,日本横滨的“中华街”中仍居住着许多粤人后裔。

除向亚洲各地移民外,许许多多的南粤人还在19世纪跨过太平洋和印度洋,向美洲、大洋洲和非洲移民。早在1796年,即有五名南粤水手乘坐商船“路易斯夫人号”抵达美国东海岸费城,是为最早踏足美国土地的粤人。自1820年起,即不断有粤人乘船横渡太平洋,在美国西海岸加利福尼亚州的三藩市(旧金山)定居下来。1848年,加利福尼亚发现金矿,大批粤人随之来到美国。到1890年,在美粤人已达10.75万人,其中以四邑、香山人为多。在三藩市,他们聚集在一起,形成了当地最早的“唐人街”。南粤人在美国西部开采金矿、兴建农牧场,为当地的开发做出了不可磨灭的贡献\footnote{《广东通史》近代下册,页844}。至于粤人对美国最伟大的贡献,则无疑是修建著名的太平洋铁路。

1862年7月1日,美国国会通过《太平洋铁路法案》,授权联合太平洋铁路公司和中央太平洋铁路公司修建一条东起内布拉斯加、西至加利福尼亚、横贯美国东西的铁路。1862年7日2日,林肯总统正式签署该法案,铁路随之动工。联合太平洋铁路公司承建的东段多在大平原地区,因而工程难度颇低,由白人工人修建。中央太平洋铁路公司承建的西段则途径遍布崇山峻岭的加利福尼亚、内华达等州,平均每隔160.9千米便要穿越一座海拔两三千米的高山,工程难度极高。开工两年后,铁路西段仅铺设了不足80.45千米,许多白人工人因施工条件恶劣纷纷离开,工程几至停顿。无奈之下,中央太平洋铁路公司开始雇佣成千上万的粤人参与施工。勤劳、正直、朴实的南粤工人很快便震撼了美国。他们在高山深谷中用简单的镐、锹、铁锤、铁钎挖开一条条隧道、架起一座座高桥。1866年春,在旧金山合恩角工地,面对陡峭的石壁,南粤工人们灵巧地运用东亚传统的栈道技术在天险中修成了跨越合恩角的铁路,创造出人间奇迹。在开凿内华达州塞拉岭隧道的工程中,南粤工人在严冬里冒着雪崩的危险舍生忘死地作业,牺牲者成百上千,赢得了美国人的一致尊敬。1865年10月10日,加州州长利兰·斯坦福在给总统安德鲁·约翰逊的报告中曾高度评价南粤工人:

\begin{quote}

就劳工阶级而言,他们非常勤劳,热爱和平,耐力也比其他民族强得多。这些“华人”的学习能力令人惊讶,他们很快就学会了未来铁路建设工作中所需要具有的专业技术,而且无论哪一种工作都能在最短的时间内熟练;另外,以工资而言,也是最经济的,虽然目前我们已雇佣了千名以上的“华工”,但是我们仍打算以最优厚的条件,通过介绍业者的协助,再增加“华工”的人数。

\end{quote}

直到今天,无数南粤工人在美国土地上辛勤工作的身影依然能通过这段文字传达到我们眼前,令我们为之感动落泪。1869年5月10日,太平洋铁路全线贯通。在通车庆典上,最早倡议雇佣南粤工人的中央太平洋铁路公司巨头查尔斯·克劳克曾说过如下一番感人肺腑的话:

\begin{quote}

我愿意提请各位注意,我们建造的这条铁路能及时完成,在很大程度上,要归功于贫穷而受鄙视的,被称为“中国”的劳动阶级——归功于他们表现卓异的忠诚和勤劳\footnote{以上两段引文及关于太平洋铁路的内容,参见潜堂编著:《美国》,页84—86}。

\end{quote}

自19世纪后期起,随着粤美间商业和人员往来的日渐频密,美国人对南粤产生了巨大好感。在美国各州,陆续有一批以“广州”(Canton)命名的市镇出现。1795年,马萨诸塞州东部诺福克郡的广州镇成为美国的首个“广州”。俄亥俄州北部的广州市则是美国最大的“广州”。在今天的美国境内,数以十计名为“广州”的城镇和乡村分布在二十三个州内\footnote{《美国的“广州”:“中国皇后号”掀起的地名热》}。南粤对美国进行的文化输出之强力,于此可见一斑。对南粤人和美国人来说,横亘在两者之间者唯有浩瀚的太平洋,只需乘上海船展开远航便能踏足对方的国土。从这一点上来看,南粤与美国虽然相隔半个地球,但两者却通过海洋紧紧联系在一起,在某种意义上可以说是比邻而居的邻邦。

除美国外,19世纪的加拿大与墨西哥也迎来了粤人移民的浪潮。1880年代,因加拿大修筑太平洋铁路,先后有17万南粤工人移居该国,形成了上百个粤人社区。1890年代,许多契约粤工通过香港、澳门转赴墨西哥,从事煤矿开采和棉花种植。在南美洲和加勒比海诸国,秘鲁、巴西、巴哈马、智利、古巴等近20个国家和地区也分布着数以万计的南粤移民,多为广府人,亦有部分客家人。特别值得一提的是,粤侨曾积极参与智利独立战争(1810—1818)和古巴独立战争(1895—1898),为南美各国的独立与西班牙殖民帝国浴血奋战。在古巴独立军中,至少有20名南粤人成为校级、尉级军官,甚至有位名叫胡德的粤侨成为古巴开国元勋马克西莫·戈麦斯(Maximo Gomez)将军的密友和心腹。此外,还有一位名居住在比亚克拉拉地区的南粤富商吴潘倾尽家财支援独立军。一位古巴独立军将军曾深情地说,参加独立战争的粤人没有一个叛徒,也没有一个逃兵。在古巴的战后重建中,南粤人也发挥了重大作用。许多南粤商人组建了种种人道救援组织,向灾民提供医疗和食物\footnote{袁艳:《融入与梳离:华侨华人在古巴(1847—1970)》}。在20世纪,许多粤侨继续投身于南美各国的独立事业。其中最著名者,便hay 祖籍梅州大埔县的圭亚那国父、于1970年任该国首任总统的钟亚瑟。

在大洋洲,澳大利亚成为南粤人的首选移民目的地。自1830年代起,粤人开始移入澳大利亚。1850年代当地发现矿山后,大批粤人、闽人纷纷涌入,至1890年已达8—10万人,其中以粤人为主。他们多经香港、新加坡进入澳大利亚,其中定居新威尔士州和悉尼者多为香山、东莞、高要、增城人,定居维多利亚州及墨尔本者则多是四邑人和南海、番禺、顺德人。他们是勤劳的农民、手工业者和小商人,许多都是契约工人,一同在这片南方热土上勤恳地建设着新家园。在非洲,英国人自1860年占领南非后便从南粤招募了一批客家劳工用以开发当地。1904年,因南非发现金矿,大量广府四邑人又前往南非淘金,至1906年已达7万人。由于与南非白人发生了激烈冲突,他们于1910年被英方几乎全部遣返。除此之外,还有一些来自顺德、南海、四邑、嘉应州的粤侨居住在马达加斯加、毛利求斯、留尼汪岛等地\footnote{《广东通史》近代下册,页845—846}。

南粤先辈们向全球各地航海的路程是十分艰辛的。他们中有许多人是为了摆脱岭北帝国的残暴统治、追求自由与尊严而出海的,还有很多人则或是为人所迫,或是遭人口贩子欺骗,在不愿背井离乡的情况下离开故土的。在土客战争中,甚至有数万名客家战俘被土勇交给蛇头,以被称作“卖猪仔”的方式如牲畜般被卖往海外的\footnote{关于土客战争与“卖猪仔”活动的关系,详见刘平:刘平:《被遗忘的战争——咸丰同治年间广东土客大械斗研究》}。在航海之路上,他们不但要经历风涛、疾病的困扰,还有许多人饱受蛇头折磨,惨死在浩瀚无边的大洋中。不少契约工人在到达目的地后还有服长时间的苦役,方能恢复自由之身。在回顾南粤伟大的大航海时代的同时,我们决不能忘记先辈们曾遭受的种种艰辛。经过先辈们胼手胝足的开创活动,南粤的种子终于撒向了世界各地,在五洲七海焕发着生机。无数南粤人脱离了岭北帝国的残暴统治,在海外建立起粤人的社会,保留着纯正的南粤文明。在许多国家,南粤人还曾积极投身当地的独立事业。这些激昂壮烈的事实表明,南粤撒向海外的种子不但象征着南粤人对自由的追求,亦在为太平洋彼岸的异国争取着尊严与自由。正因为先辈们的这些努力,今天的我们才能在世界的每一个大陆上见到粤人、吃到粤菜、听到粤剧。

18、19、20世纪的伟大航海使南粤人深刻地认识了世界。1782年,一个名叫谢清高的嘉应客家商人在海南乘上外国商船,开始了他长达十四年的周游世界之旅。在此期间,他曾到过东南亚、土耳其、英国、法国、西班牙、葡萄牙、荷兰、比利时、瑞士、普鲁士、奥地利等国,学会了多种西方语言。1802年,三十八岁的他回到南粤,在澳门谋得一份葡语通事(翻译)的工作。此后,他埋首于撰写《海录》一书,详细记述自己在海外的种种见闻。1820年,《海录》在广州出版,其中记载的种种奇情异事引起强烈轰动。次年,南粤史上首个世界周游者谢清高去世,将世界的完整图景留给了南粤,他本身亦成为南粤大航海时代最具代表性的人物。而这时,距离鸦片战争的爆发尚有18年,愚昧的清帝国君臣仍对世界大势一无所知。1842年,著名的湖湘经世学者魏源受鸦片战争刺激愤而撰成介绍世界详情的《海国图志》一书,《海录》正是其重要资料来源之一\footnote{程美宝:《澳门作为飞地的“危”与“机”——16—19世纪华洋交往中的小人物》}。魏源与林则徐被大一统史观塑造为所谓的首批“开眼看世界”之人。然而,早在他们之前,南粤人便早已开眼看清世界的面目,向世界敞开怀抱,扬帆起航。

\section{开化之路:教会、留学生、科技、近代工业}

自16世纪后期罗明坚、利玛窦由澳门进入南粤以来,基督教便与南粤结下不解之缘。然而如前所述,在明帝国的疯狂迫害下,西方传教士在南粤的传教事业于17世纪中期陷于停顿。1720年,清帝国因与罗马教会发生著名的“礼仪之争”而下令禁教。此后约一百年里,传教士和南粤残存基督徒的活动转入地下,声势颇为不振\footnote{《简明广东史》,页373}。

1807年9月8日,英国传教士马礼逊(Robert Morrison)到达广州,为基督教在南粤的历史翻开了新的一页。马礼逊于1782年出生在一个苏格兰贫民家庭中,他的父亲是个虔诚的教徒。受此影响,马礼逊自幼攻读《圣经》,于16岁加入苏格兰长老会、20岁进入神学院。1807年,年仅25岁的他被伦敦会派往广州秘密传教。到达广州后,马礼逊居住于美国商馆,进行了为期一年的粤语和“官话”学习,随即开始其传播福音的伟大工作。1809年,他被英国东印度公司聘为翻译,从此得以“合法”地往来于澳门与广州之间,并投入到以汉文翻译《圣经》的艰辛工作中,于1813年完成《新约》的全部翻译工作,在广州秘密出版。1818年,伦敦会又派传教士米怜(William Milne)赴粤担任马礼逊的助手。然而,由于米怜缺乏在广州居留的“合法”身份,马礼逊只好派他前往有不少粤侨的马六甲展开工作。在马六甲,米怜将一位名叫梁发的高明人发展为教徒,此人将在日后的传教事业中发挥重要作用。在马礼逊和米怜的合作下,南粤史上的首座教会学校英华书院于同年在马六甲落成。该校学生一般有一二十人,主要招收粤人贫民子弟。该校不但教授神学和英文,还开设历史、地理、算学、几何、天文、代数、机械、逻辑、道德、哲学等课程,于当时西方最先进的科学知识无所不包\footnote{顾卫星:《马礼逊与中西文化交流》}。英华书院学生遂得以成长为首批英语流利、几乎彻底西化的南粤知识分子,他们中有不少人在毕业后回粤工作,成为沟通南粤与世界的重要桥梁。至1844年,英华书院迁往香港,继续源源不断地为南粤培养西化人才\footnote{《简明广东史》,页375}。

1819年,马礼逊在米怜的协助下终于完成了对《旧约圣经》的翻译,该书得以在马六甲公开出版。然而仅仅三年后,米怜便不幸病逝。1824年,马礼逊受召回国。临行前,他于1823年12月将梁发按立为南粤史上的首个新教牧师,把在广州和马六甲传教的重任全部交给了梁发。然而,由于清帝国仍在严厉禁教,梁发在广州的工作很难展开。在马礼逊于1826年再次受派遣赴粤后,他也只能与梁发一同秘密传教,收获不多。当时,正式受洗的南粤新教徒仅有十余人,其中包括梁发的妻子黎氏\footnote{《简明广东史》,页374}。这时,梁发另辟蹊径,使传教事业产生了转机。

梁发系高明县古劳村人,生于1789年,曾读过四年私塾,有一定的文化知识。1804年,年仅15岁的他前往广州打工,在十三行学习印刷,后与马礼逊相识。其后,他又前往马六甲协助米怜工作,在那里受洗成为教徒,并成为英华书院的教师。梁发不但十分虔诚、勇敢,亦很聪明。1828年,他回到家乡高明为好友古天青施洗,两人随之合作开设了一所教育幼童的私塾。这间私塾虽然教授四书五经,有着传统的包装,却实为一间教会学校。它不但教授东亚传统学问,还教授西方的科学、地理知识及英文,因而当之无愧地成为由南粤人自主创办的第一所西式学校。不幸的是,该校遭到一些保守乡民的反对,最终引来清帝国政府的查封,梁发只好逃至澳门。令人感动的是,梁发虽然经此挫折,却依然心怀将福音传遍南粤的信念。很快,他便出发前往粤西高州,在当地写下《灵魂篇》、《真道寻源》等教义小册子向民众秘密散发。1832年,梁发回到广州,出版了其多达九卷、通俗易懂的著名教义著作《劝世良言》。受此书影响的南粤民众难以计数,其中甚至包括那位日后掀起惊天巨变的野心家洪秀全。对于梁发深入民众、不顾艰险传播福音的壮举,马礼逊十分感动,曾发表过如下一番评价:

\begin{quote}

在此充满偶像及拜偶者之国土中,上至王公,下至乡愚,皆反对及窘迫基督之门徒。即使有许多尼底哥母或如作《教会历史》之米怜氏所谓“异教之基督徒”(意即未有完全之宗教知识及怯懦而不敢直认为基督徒之人),亦不足为奇也。但于此有一人焉,彼之基督徒系出于己愿,而又能公然承认其信仰。此人为谁?则梁发是也\footnote{转引自《近代史资料 总39号》,页160}。

\end{quote}

在人生的最后几年中,受梁发鼓舞的马礼逊继续着伟大的传教事业。在他的接待下,一批美国、德国传教士得以在广州扎根。1830年,美国传教士裨治文(Elijah Bridgman)抵达广州。次年底,德国传教士郭实腊(又译“郭士立”,Karl August)到达澳门。1833年,又有美国传教士卫三畏(Samuel Williams)及医疗传教士伯驾(Peter Parker)来到广州。他们都得到马礼逊的热情接待,因而得以顺利开展工作。1834年,马礼逊因染上急病不幸在广州去世,年仅45岁。此后,梁发全盘接手了他的工作。不久后,他再次被愚昧的清帝国政府盯上,遭到通缉,只得在裨治文的帮助下流亡新加坡、马六甲。纵然遭此困厄,梁发仍旧没有屈服。拥有高尚美德的他在1839年重返广州,一直工作到他于1854年去世之时\footnote{《简明广东史》,页374}。为了南粤,他奉献了自己的一生。令人唏嘘的是,直到去世时为止,梁发都未能看到教禁在南粤解除的那一天,而这无疑是他毕生最大的心愿之一。

梁发去世后仅仅四年,他的心愿便达成了。1858年6月26日,英法两国与清帝国签订《天津条约》,西方人从此获得在清帝国境内自由传教的权利,桎梏南粤达138年之久的教禁终于解除。自19世纪下半页起,基督教的在南粤的传教事业如雨后春笋般发展起来。到1900年,广东已有分设于广州、汕头、肇庆、英德、北海等地的传教总会30余处,各县分会基址百余处,基督徒1.3万余人。在20世纪早期,这一数字呈几何级数膨胀,至1926年达到78516人\footnote{《广东基督教历史》}。

传教士们不但将基督教的福音传向南粤大众,亦给南粤带来了先进的西方科学、医疗技术。1829年,英国东印度公司医生郭雷枢在广州开设了一家小医院,是为南粤西医院之始。1835年,伯驾又在广州创设“眼科医局”,是为南粤最早的大型西医院。伯驾不但亲自出诊,更招收数名南粤学生担任助手,为南粤培养了最早的一批本土西医。“眼科医局”大受广州市民欢迎,在开业头六周便接收了450名病人。在鸦片战争中,“眼科医局”一度停办,后于1842年11月重新开业,并在1859年改名“博济医院”,存续至今\footnote{《简明广东史》,页376}。

对于南粤的民族语言,传教士更做出过一番令人惊叹的伟业,极大推动了19世纪南粤的民族发明工作。马礼逊刚刚到达广州时便发现,广大南粤民众中懂“官话”者寥寥无几。若要展开传教工作,他必须学会粤语。koy 指出:

\begin{quote}
这里大部分的“中国人”不会说“官话”,也不识中国字。中国的穷人太多,但他们必须听得懂我讲的“官话”和所写的中文,我才能将极度的福音传给他们\footnote{转引自李婉薇:《清末民初的粤语书写》,页28}。
\end{quote}

马礼逊尚未察觉到南粤百姓不通“官话”的真正原因,依然将之归因于贫富问题,而不是民族差异。不过,马礼逊依然认识到传教士学习粤语的重要性。1828年,由他为英国东印度公司编著的《广东土话字汇》一书出版,是为西方人学习粤语的最初教材\footnote{李婉薇:《清末民初的粤语书写》,页28}。马礼逊去世后,裨治文对粤语进行了更细的研究。1841年,他的著作《广府话文选》在澳门出版,是为又一本面向西方人的粤语教材。在书中,他提到了一个耐人寻味的现象:

\begin{quote}
在这帝国的每一个地方,字体一般是通用和统一的。唯一不同的是,有些字会被稍稍更动以表示地方的用法。在这些情况下,所改变的往往只是读音;但有些时候,人们会在通用字体的左侧加上一个“口”字,以表示该字体已经改变了。例如,“喊唪唥”这三个字是用来代表“hampalang”(所有、全部的意思)这个发音的,它们本身是没有意思的。只有将它们并置起来读出,它们作为一个词组的意思才能被识别\footnote{转引自程美宝:《地域文化与国家认同:晚清以来“广东文化”观的形成》,页150}。
\end{quote}

裨治文所说的“没有意思”的字,实为流传已久的粤字。事实上,“喊唪唥”一词在吴语中有发音相近、含义类似的对应词汇,应为百越词汇,与汉语没有渊源。对于这种词汇,粤人自然只能用粤字书写。这一点无疑在暗示,粤语是一门独立于“官话”的语言。此后半个世纪内,西方人编纂的粤语教材、字典纷纷涌现,多达数十种。至19世纪末,居住在香港的美国学者波乃耶(James Ball)终于认识到了这一点。在1883年出版于香港的粤语教材《广府话易学》中,他点明粤语绝非简单的“方言”(dialect),而是一门历史悠久的“语言”(language)\footnote{李婉薇:《清末民初的粤语书写》,页28}。在1901年出版的《顺德方言》一书中,他又进一步指出,粤语存在着公认的标准语,那便是广州的西关话\footnote{程美宝:《地域文化与国家认同:晚清以来“广东文化”观的形成》,页153}。

19世纪后期,随着西方传教士对粤语日渐重视,koy 们纷纷开始用粤语编写传道书籍、以粤文翻译《圣经》和宗教著作。1870年,英国循道会传教士俾士(George Piercy)完成了以粤文翻译宗教读物《天路历程》的工作,该书以《天路历程土话》之名在广州出版。此书乃17世纪英国布道家约翰·班扬(John Bunyan)的名著,讲述了一个受迫害的基督徒的心路历程。俾士在粤译本中大量运用通俗的粤文和图像,使该书十分易于阅读,因而在南粤社会中流传甚广。在1915、1923和1930年代,该书曾三次再版,一时供不应求\footnote{李婉薇:《清末民初的粤语书写》,页222—224}。因阅读该书而了解基督教义、皈依极度的南粤民众,更是不知凡几。

在《天路历程》粤译本出版前后,聚集在广州的英美传教士也在进行着用粤文翻译《圣经》的工作。1862年,美国长老会在广州出版粤译本《马太福音》,是为粤译《圣经》之始。至1884年,在英国、美国圣经公会的参与下,《新约圣经》粤译工作全部完成。1873—1894年间,《旧约圣经》的粤译工作也由英美圣经公会完成。最后,《新旧约圣经》粤语全译本于1894年由上海美国圣经公会出版,粤译《圣经》工作历时三十二载,终于大功告成\footnote{梁慧敏:《十九世纪"圣经"粤语译本的研究价值》}。对南粤人而言,粤译《圣经》无疑比“官话”本《圣经》远更平易近人。只消我们比较《旧约·创世纪》第一章前五节在粤译《圣经》和“官话”和合本《圣经》中的区别,便可一目了然:

\begin{quote}

始初上帝造天造地。地系混沌、空虚,深渊面上黑暗;上帝之神埔紧在水面。上帝话:“爱有光”,就有光。上帝看倒光系好,就分光隔暗。上帝安光做“日辰头”,安暗做“夜晡辰”。有晚辰,有朝辰,系第一日。

起初,神创造天地。地是空虚混沌,渊面黑暗;神的灵运行在水面上。神说:“要有光”,就有了光。神看光是好的,就把光暗分开了。神称光为“昼”,称暗为“夜”。有晚上,有早晨,这是头一日。
\end{quote}


在20世纪,《新旧约圣经》粤语全译本曾多次出版,其重要性无论怎样高估都不为过。使用粤译《圣经》的南粤教会随之涌现,数以万计的粤人通过该书皈依了基督,并能用自己的母语朗诵《圣经》。19世纪至20世纪初,西方传教士、学者与南粤精英一同致力于粤语研究和粤文写作,共同促成了南粤民族共同语的诞生。南粤的民族觉醒不但要归功于广大的南粤精英与大众,也要归功于那di 深爱南粤的西方传教士。对于他们的丰功伟绩,今天的我们一定要铭记在心。

当西方传教士纷纷进入南粤时,许多渴求了解世界文明的粤人远赴海外求学,成为南粤最早的一批留学生。出生于马来西亚的粤侨曾恒忠(祖籍潮州海阳)是第一个赴西方留学的粤人。他于1843年受教会资助前往美国求学,于三年后入读纽约汉密尔顿学院,后因资金问题被迫辍学。1847年1月4日,一位颇具色彩的留学生从香港出发,登上前往美国的航船,他便是深刻影响了南粤历史走向的容闳\footnote{《那些被遗忘的背影:近代广东留学生》}。

1828年,容闳出生于与澳门一水之隔的香山县南屏村,他的父母都是贫农。因紧邻澳门的缘故,香山人对西方人和西方文化并不陌生,多有赴澳门打工者。1835年,七岁的容闳随父前往澳门,就读于英国人开办的马立逊预备学堂,受到传教士郭实腊之夫人的亲自教导。也就是说,他的启蒙教育是由西方人完成的,英语几乎是他的母语。五年后,容闳的父亲去世,他则继续在澳门读书。1842年,他又赴香港马立逊纪念学校继续学业。因成绩优异,他和同乡黄胜、黄宽于1847年得到留美求学的资格,随美籍校长布朗(Samuel Brown)牧师远赴美国。当年4月12日,他们到达纽约,入读马萨诸塞州之孟松预备学校。1850年,容闳出色地完成了预备学校的学业,凭自己的实力考入名校耶鲁大学,成为首个就读于耶鲁的南粤留学生。1852年,他获得美国国籍,并于两年后出色地完成大学学业,获文学士学位\footnote{刘成禺:《容闳生平大事年表》}。这时,二十六岁的容闳已是个彻底西化、学识渊博、彬彬有礼的绅士了。

从耶鲁毕业后,难掩思乡之情的容闳回到阔别十九年的南粤,在广州的美国公使馆工作。1855年夏,他在广州亲眼目睹了叶名琛制造的血腥屠杀,对清帝国的愚昧残暴深恶痛绝,一度寄希望于盘踞吴越的太平天国。1860年,他来到太平天国首都南京,向干王洪仁玕提出革新军事、政治、教育制度的改革方案,却未得到积极回应。心灰意冷之下,他又试图与投身于“洋务洋务”的湘淮官员合作,意图革新清帝国。1863年,容闳前往安庆与曾国藩会面。当时,曾国藩正在筹建洋务运动的首座工厂江南制造总局,便委托容闳赴美采购机器。一年后,容闳自美携百余种机器来到上海,江南制造总局乃得以于1865年落成。曾国藩对容闳的辛劳投桃报李,保举其以五品候补同知衔,令其任江苏巡抚丁日昌的译员\footnote{刘成禺:《容闳生平大事年表》}。

容闳虽然以一介游士的身份效力于湘淮官员,但他从来没有忘记南粤。1870年,他向曾国藩提议派遣幼童赴美留学。曾国藩经与李鸿章商议后将该提议“上奏”清廷,获得批准,随即组建留学肄业局。1872年8月16日,30名留着发辫、穿戴整齐的幼童在留学肄业局正监督陈兰彬(高州吴川人)、副监督容闳的带领下从上海登上客轮,开始赴美之旅。他们的年龄在10—16岁之间,有24位是南粤人,其中包括13名容闳的香山同乡,以及日后主持修建京张铁路的著名工程师詹天佑(广州人,吴越徽州第三代移民)。9月12日,客轮抵达旧金山,留学生们随即登上蒸汽火车,沿太平洋铁路展开横贯北美大陆之旅,最终到达美国东部的新英格兰地区,被分配到康涅狄格、马萨诸塞两州数十户美国家庭中居住,并入读美国的中、小学校。容闳负责监督留学生们的学业,陈兰彬则负责教授他们汉文。此后三年内,又有90名幼童分三批从上海出发到达新英格兰,使留美幼童数量达到120人,其中有84位来自南粤\footnote{《那些被遗忘的背影:近代广东留学生》}。对留学生们来说,美国的一切都让他们倍感新奇,他们学会了流利的英语和西方的礼仪,穿着西式服装与白人孩子一同在学校里读书、在运动场上玩棒球、在住所举办party、和白人家长朝夕相处,许多人还俘获了美国女孩的芳心。就连容闳也在找到了他的异国挚爱,他于1874年与美国女子玛丽·凯洛克结为夫妻\footnote{刘成禺:《容闳生平大事年表》}。更为重要的是,留学生们接受了西方的价值观。他们不再对人下跪磕头,许多人剪掉了象征着带有耻辱印记的清式发辫,还有不少人皈依了基督教。到1880年,已有50多名留学生完成中学学业,进入美国大学,其中有22人进入耶鲁大学、8人进入麻省理工、3人进入哥伦比亚大学、1人进入哈佛大学。次年,詹天佑和欧阳庚(象山人)从耶鲁大学毕业,是为留学生中最快完成大学学业者\footnote{王芳:《耶鲁中国缘:跨越三个世纪的耶鲁大学与中国关系史1850—2013》,页21}。然而就在这一年,愚昧的清帝国却突然下令撤回全部留学生,粗暴地中断了他们的学业。造成此种局面的一大罪魁祸首,便是陈兰彬。

陈兰彬虽是粤人,但清帝国翰林官出身的他却和南粤缺乏有机联系。他虽略通西学,但不会讲英语,骨子里是个十分保守愚昧的人。见到留学生们剪去发辫、皈依基督教、抛弃下跪磕头等礼仪的行为,陈兰彬十分愤怒。1874年,陈兰彬调任清帝国驻美公使。不久后,留学监督由吴嘉善(江右南丰人)接替。吴嘉善虽通晓西方文史、数学知识,却也是个守旧之人,与陈兰彬臭味相投。经两人商议后,陈兰彬于1881年初对清廷“上奏”道:

\begin{quote}

外洋风俗流弊多端,各学生腹少儒书,德性未坚,尚未究彼技能,实易沾其恶习。即使竭力整饬,亦觉防范难周,亟应将局裁撤\footnote{谌旭彬:《中国1864—1911》,页122}。

\end{quote}


清廷遣留学生赴美之举的目的,本为为其培养忠于帝国的西学人才。但假若这些留学生抛弃了对清帝国的认同,那么愚昧的帝国必然要中断他们的学业。1881年9月6日,清廷裁撤留洋肄业局,全体留学生被勒令“归国”\footnote{珠海市委宣传部选编:《容闳与留美幼童研究:容闳与中国近代化》,页159}。除两人坚持继续留在耶鲁大学外,所有人都选择服从命运的安排\footnote{王芳:《耶鲁中国缘:跨越三个世纪的耶鲁大学与中国关系史1850—2013》,页24}。清帝国的武断之治打碎了容闳为南粤培养西化人才的梦想。回到清帝国后不久,他意兴阑珊地选择离开,于1883年起长期定居于美国\footnote{刘成禺:《容闳生平大事年表》}。在19、20世纪之交,他又积极投身于维新变法、辛亥革命,不过那是后话了。

留美计划虽然夭折了,但它毕竟为南粤培养出一大批通晓西方知识、有多年西方生活经验的留学人才。这批最了解西方世界的南粤青年才俊很快便将投入历史的洪流中,成为推动南粤以及东亚大陆文明开化的中坚力量,在军事、教育、工业、政治等领域大显身手,詹天佑及日后担任中华民国首任总理的唐绍仪(香山人)就是他们中的优秀代表。

19世纪中后期,随着西方先进的技术、知识传入南粤,不少南粤知识分子投入了对西学的研究。1844年,陈澧的好友南海人邹伯奇(1857年任学海堂学长)凭着他对西方光学、数学多年钻研,制造出了南粤史上的第一部照相机,仅比世界上首台照相机的问世晚三年。1847年,他又根据哥白尼的“日心说”制造出精美的太阳系模型“七政仪”\footnote{王钊宇总纂:《岭南文化百科全书》,页481}。更值得注意的是,随着西方先进的科学知识传入南粤,一批南粤土豪怀着振兴乡邦的理想投身实业,在南粤大地上建起了一座座西式工厂。1872年,曾在南洋经商的南海西樵乡简村商人陈启沅在家乡设立继昌隆缫丝厂,是为南粤史上首座西式工厂。该厂由陈启沅之兄陈启枢投资白银7000两而成,最初招收300名女工,其中简村人居半,其余半数亦来自临近村庄。陈启沅大量引进先进的西式设备和管理制度,厂中以鸣汽笛为上下班之号,实行计件工资制,且设有全勤奖和超产奖。因该厂“出丝精美,行销于欧美两洲”,获利巨大,厂中女工数量很快增至六七百人。此外,陈启沅又以缫丝厂的利润在家乡开办永生米机厂和副食杂货店,每月对村中鳏夫寡妇施米20斤,赢得了乡民的尊敬与热爱。受其影响,举人陈植渠、陈植恕也于1877年在简村对岸的学堂村开办了裕厚昌丝厂\footnote{《简明广东史》,页458—459}。

继缫丝业而起的是火柴业。1879年,肇庆旅日粤侨卫省轩在佛山文昌沙创办巧明火柴厂,日产千余盒。1908年,该厂又接受日商注资,成为粤日合资企业。这不唯是南粤的首家火柴厂,也是清帝国境内的第一家。至1890年代,大批火柴厂已在广州府境内纷纷涌现,其中建于广州石围塘的义和火柴厂竟有6万箱的年产量。火柴造价低廉,又满足了人们日常的生火需求,因而销量巨大,改变了南粤人的生火方式。紧跟其后的行业则为造纸业。1882年,南海盐步水藤乡商人钟星溪成立造纸股份公司,集资15万两,于1889年在家乡创办占地三十余亩的宏远堂造纸厂。该厂大量引进英国设备,聘请英国专家为工程师,工人中多有技术熟练的旅美归粤者。此厂每日产纸62担,其纸张坚韧耐用,行销南粤各地,被称为“宏远堂纸”、“西洋纸”。在汕头,因粤东农民喜用豆饼做肥料,当地在1879—1893年兴起数座豆饼厂,其出产的优质豆饼不但足以供应整个潮汕地区,甚至还远销台湾。至于最令人自豪的,则毫无疑问是电灯业的兴起。自1879年美国人爱迪生发明电灯后,这一依靠电力、无比方便快捷的照明技术便迅速向全球扩散。在电灯问世后仅仅十一年,台山旅美粤侨黄秉常便通过向旅美粤侨招股的方式集资40万元,成立了广州电灯公司。该公司聘请美国专家为总工程师,发电量可供1500盏电灯之用。短短两年内,公司便向广州的商号和街道提供了700余盏电灯、路灯,使夜晚的广州城笼罩在一片电灯的光明之下。除此之外,南粤的近代化采矿业也在19世纪末得到长足发展。至1890年代,南粤境内已出现50余家采矿公司,从事铁矿、铅矿、金矿、银矿开采\footnote{《简明广东史》,页461—464}。到1900年代,南粤人自己的铁路也出现了。1903—1906年间,梅州籍南洋粤侨张榕轩、张耀轩兄弟引进日资,修成南起汕头、北至潮州意溪,全长42公里的潮汕铁路。1906—1910年间,又有新宁旅美粤侨陈宜禧通过在美粤侨集资270万元,筑成联通阳江、新宁,全长120余公里的新宁铁路\footnote{《简明广东史》,页469—471}。随着铁路的铺设,蒸汽火车轰鸣着驶过南粤的田野、乡村与城市,和无数工厂、学校、电灯一同构成了一副文明开化的图景。如果说这幅图画里还缺点什么的话,那便是翱翔于南粤天空中的飞机和威风凛凛的空中骑士。

1903年,美国莱特兄弟发明飞机,人类第一次实现了飞上蓝天的梦想。一位年仅20岁的南粤少年听闻此消息,立志造出粤人自己的飞机。冯如,恩平人,于12岁时随父漂洋过海谋生,在旧金山成为一名童工。幼时的冯如在家乡已感受过日渐文明开化的气氛。到达美国后,他更是为繁荣的西方文明深感震惊,对日新月异的机器产生了极其浓厚的兴趣。不久后,他前往高楼林立的纽约求学,十年如一日地钻研机械学和电力学,用自己的双手造出了抽水机、打桩机、小型发电机和无线电收发机。得知莱特兄弟已经乘飞机飞上天空后,他全力转向对载人飞行器的研究,发誓一定要造出比莱特兄弟的飞机更先进的机型。1906年,他回到旧金山,联络当地粤侨集资,于两年后在旧金山以东的奥克兰建造了自己的飞机制造厂。1909年初,他又成立广东飞行器公司,自任总工程师,全力投入飞机制造中。冯如根据自己掌握的空气动力学知识手绘图纸、监督建造,终于完成了自己一手打造的飞机“冯如一号”。1909年9月21日傍晚,冯如驾驶着“冯如一号”从奥克兰郊外的一座园丘上腾空而起,在天空中翱翔了800米。此刻,这个26岁的南粤青年是全世界飞得最高的人。而他的滑翔距离,更是六年前莱特兄弟航程的三倍。飞机的螺旋桨虽然断了,但冯如凭着高超的驾驶技术使飞机安然着地。他的壮举震惊了世界,《旧金山考察者报》在头版刊登了他的照片,称他为“东方莱特”。试飞成功后,冯如大力改造机型,于1910年7月研制出“冯如二号”。1911年1月18日,他驾驶“冯如二号”在奥克兰进行飞行表演,在200多米高的空中飞行了32公里,再次使世界震惊。不久后,南粤摆脱清帝国统治的独立战争爆发,冯如拒绝了美国人的挽留,毅然回到南粤,在广东军政府中担任飞行队长,成为一名保卫南粤天空、为南粤的自由而战的骁勇斗士。1912年8月15日,他在广州燕塘的一次飞行表演中意外失事,不幸遇难。弥留之际,冯如留下如是遗言:

\begin{quote}

吾死后,尔等勿失其进取之心\footnote{关于冯如生平,参见恩平市政协学习和文史委员会:《中国航空之父冯如研究》}。

\end{quote}

冯如的遗言所反映的,正是那个时代南粤人的精神面貌,它正如明治时代的日本人迸发出的激昂风貌。笔者只需对日本著名历史作家司马辽太郎在其名作《坂上之云》中所说的一段话略加改写,便可展示出那个时代的南粤人追求文明开化的伟大精神:

\begin{quote}
海滨古老的南粤,正在迎来开化期。说到产业,它虽有农业、手工业和商业,却饱受岭北帝国摧残;说到人才,似乎也多是一些耻于用母语交谈的士大夫。经过19世纪和20世纪初的发明民族,南粤人有了自己的民族共同语、形成了由精英和民众组成的坚实共同体。经过19世纪和20世纪初的文明开化,南粤人首次有了西式的工厂、铁路、学校和飞机。那个时代的南粤人正是这种经历的最初体验者,因而昂扬不已。如果不去体验这种令人感动落泪的昂扬,也就无法理解这段历史。说来有点滑稽,一个远离西方世界的海滨邦国,也想拥有欧美国家的先进文明。可是不管怎么说,这便是那个时代南粤人的目标,也是他们有些孩子气般的希望之所在。他们以那个时代的南粤人所独有的劲头,目视前方大步行进、一直向前。如果在山坡之上的蓝天中,有一朵亮丽的白云,那么登上坡道,就是为了去亲眼目睹吧。
\end{quote}







