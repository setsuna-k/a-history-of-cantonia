
\chapter*{序}

\begin{flushright}
	刘仲敬
\end{flushright}

\begin{quoted}
	上帝保佑南粤,南粤自由万岁。
\end{quoted}

南粤先民从记忆无法追溯的远古时期开始,就在韩江、珠江流域生生不息,为人类贡献了芋头、香蕉、西米和众多块茎食物,尤其是丰富多彩的亚热带水果。粤人得天独厚,享受伊甸园一样的美好生活,不太需要为觅食而终岁操劳,可以将大部分时间用于艺术创造。相形之下,黄河中游的鼠尾草食物就显得既乏味又单调。粤人陶醉于印纹陶器的几何形花样,在纹身艺术上精益求精,只能属于宽裕悠游的民族。黍米民族的艺术作品讲究单调实用,想象力的天花板甚低,折射出物主的艰难窘迫。

西南亚贸易线和南洋贸易线的发展,给南粤的富实锦上添花。犀角、象牙、玳瑁、明珠,养育了合浦的土豪和商人。粤人身为海上民族,是全世界的邻邦。他们的技术水平,远远高于闭塞的东亚内地。南粤先民以高超的冶炼技术、锐利的短剑和好勇轻生的武士精神著称。他们的原始丰饶比东亚内地各民族开始得更早,结束得更晚,仿佛天将降大任于斯人,保存了丰沛未凿的元气。粤人勇敢、骄傲、富裕,对自己的部族共同体非常满意,不屑于放弃自己的优越生活,模仿蚂蚁一样卑微的北佬官吏和士兵。

装腔作势的帝国朝廷往往声称普天之下莫非王土,在地图和文书上管理南粤。然而在南粤的土豪和武士眼中,朝廷的官吏无异于打秋风的浪人。他们长途跋涉的主要动机,不外乎觊觎黄金海角的奇珍异宝,渴望满载而归的幸福时刻。他们如果知趣,南粤土豪通常不会吝惜珍珠。他们如果不知趣,就会将南粤武士的吼声留在断简残篇当中。南粤豪杰辈出,一再教育无理取闹的支那人,让他们懂得怎样尊重东道主的自由和习俗。

大航海时代为粤人打开了更加广阔的窗口,将东亚的伟大民族变成了世界的伟大民族。海洋民族一向胸怀宽广,擅长吸纳全世界的思想资源,不亚于经营东西洋的物资资源。儒家文化、印度文化、日本文化、伊斯兰文化和基督教文化都在南粤共同体成长的各个阶段,贡献了宝贵的组织模式和文化财富。随着世界体系的延伸,南粤民族也开始构建自己的国族。共铲国际在帝国解体的间隙期入侵东亚,同时造成了真空的解放效应和真空的填补效应。

共同体如果已经成熟,帝国的瓦解就是自我解放的机会窗口。波兰和芬兰就是这一类共同体的典范,已经具备承担责任的能力,一旦获得解放,就能顺利填补帝国解体留下的真空。列强只要能够找到负责任的交涉对象,就无需费力拼凑断肢残体。共同体如果尚未成熟,帝国的瓦解就会导致弗兰肯斯坦式假共同体的产生。弗兰肯斯坦是各种尸体碎块拼凑而成的假人,自然没有真人的生命力,但在真共同体发育不全的情况下,列强就只有用临时拼凑的假共同体填补帝国解体留下的真空。

塞尔维亚-克罗地亚-斯洛文尼亚联合王国(南斯拉夫王国)和汉满蒙回藏联省共和国(中华民国)就是典型的弗兰肯斯坦,当然没有真正生命体的自我保护能力。这就是为什么广土众民的南斯拉夫民族和中华民族在国际恐怖组织和颠覆势力面前不堪一击,而小小的克罗地亚民族和南粤民族却能奋斗自卫的根本原因。共同体无论多么弱小,都比原材料强大。生命无论多么弱小,都比死肉强大。

天命为腐肉准备了秃鹰,让communism为诸夏驱除。南粤永远得风气之先,《南粤史》自然应运而生。粤人每逢历史的节点,都不会缺乏天命的担纲者。国族通过民族史的编撰,发现了自己。民族史既是历史的纪录者,又是历史的驱动者。古老而又年轻的南粤民族一脚踢翻了楚门的世界,大开门户迎接久违的阳光和空气。

