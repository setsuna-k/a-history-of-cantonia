\chapter{第五次北属}

\section{夹缝中的战场:北宋统治前期的南粤}

大宝十四年(971)二月十五日,南汉后主向宋帝国侵略军投降,南汉灭亡,南粤史进入第五次北属时期。在宋帝国统治初期,我们不甘做奴隶的伟大祖先发动了一场又一场起义,与侵略者展开了殊死搏斗。

同年十月,南汉宦官邓存忠发动二万余名百姓起义,围攻邕州城达七十日之久,击伤宋守将范旻。直至宋军援兵从广州赶来,起义才告失败。邓存忠逃脱宋帝国追捕,继续与侵略者周旋。次年八月,邓存忠联合南汉宦官乐范、春恩道都指挥使麦汉琼、韶州土豪周思琼发动了更大规模的起义,一连收复五州。此外,崖州牙校陆昌图亦于海南岛起兵反宋,烧劫官署。对于遍布南粤各地的起义军,赵宋侵略者采取了血腥屠杀的政策。经三个月浴血奋战,各路起义军全部战败\footnote{陈欣:《南汉国史》,页164—165}。南粤人反抗宋帝国、复兴南汉的英勇斗争,就这样被侵略者残暴地镇压了。

粤人失去了自己的国家与自由,被绑上了宋帝国的战车,成为宋帝国与越南间纠纷的牺牲品。北宋初期,宋帝国与刚独立不久的越南关系微妙。公元1009年,李公蕴(李太祖)建立越南李朝,并于当年向汴京派出朝贡使,建立了宋越之间的朝贡关系\footnote{郭振铎、张笑梅:《越南通史》,页284}。1028年,李太祖崩,其继承人李太宗(李佛玛,1028—1054年在位)虽继续保持对宋朝贡,但颇为敌视宋帝国。1036年,李太宗发兵进攻广西,攻至邕州附近,焚庐舍而还\footnote{郭振铎、张笑梅:《越南通史》,页287}。此外,两国又有关于广源州(今越南高平)的领土争端。广源州之居民系西瓯族的直系后裔僮人。自第四次北属时起,该地即由僮人土豪侬氏统治。宋帝国以广源州为邕州管辖下的“羁縻州”,仅对当地拥有名义上的统治权。为在宋越的夹缝间生存,侬氏一面对宋帝国称臣,一面向越南李朝“岁输金货”,采取左右逢源的外交策略。然而,李太宗对侬氏十分苛刻、“赋敛无度”,给广源州百姓带来了沉重的负担\footnote{郭振铎、张笑梅:《越南通史》,页290}。公元1038年,忍无可忍的侬氏首领侬存福乃起兵反越,自称昭圣皇帝,立其妻阿侬为皇后,建国号“长生”,宣布不再对越进贡\footnote{陈重金:《越南通史》,页67}。

越南人对侬存福的反抗进行了无情的镇压。1039年,李太宗御驾亲征北伐广源州,俘杀侬存福及其子侬智聪。阿侬与年仅十四岁的幼子侬智高逃至安德州(今广西靖西境内),在此重整旗鼓。两年后,母子二人率军占领傥犹州(近广源州),建“大历国”,再度成为越南的威胁。在越南军的征讨下,侬智高兵败,被俘送越都升龙(今越南河内)。李太宗因已杀其父兄,不忍杀之,乃“赦其罪,授广源州牧”\footnote{《钦定越史通鉴纲目》卷3}。次年,侬智高被越南人加封为“太保”,侬氏遂由对宋朝贡的土豪转变为越南官员。

侬智高并未忘记与越南人的国恨家仇,他暂时的隐忍不过是为了日后的报复。1049年,侬智高大举招兵买马,再度向越南竖起反旗,重占安德州,建立“南天国”,改元“景瑞”,自立为帝。李太宗命太尉郭盛溢率军讨伐,被侬智高击退。为增强自己对抗越南的实力,侬智高向宋帝国进贡求官,却被宋廷以如下理由拒绝:

\begin{quote}
智高叛交趾而来,恐疆场生事\footnote{司马光《涑水纪闻》,转引自范宏贵主编:《侬智高研究资料集》,页5}。
\end{quote}

宋帝国胆怯而傲慢的态度彻底激怒了侬智高。1052年,侬智高在安德州聚集七千兵力开始了对宋帝国的进攻,意图夺取整个南粤\footnote{同注322}。由侬智高所用兵力之少来看,他的这次军事行动无疑是十分冒险的。然而,其时的宋帝国已深陷于“冗官”、“冗兵”、“冗费”问题,乃一泥足巨人。侬军很快攻占邕州,杀知府陈珙。侬智高在此改国号为“大南”,改元“启历”,自称“仁圣皇帝”,建立了与宋帝国分庭抗礼的政权\footnote{郭振铎、张笑梅:《越南通史》,页292}。当时,驻粤的宋帝国官僚多为只知压榨民众的无能之辈。面对侬军的凌厉攻势,他们望风奔逃。许多不堪忍受赵宋暴政的南粤百姓亦揭竿而起,纷纷加入侬军,侬智高麾下的主要谋士黄纬、黄师宓便都为广州人。一个多月内,横、贵、浔、龚、藤、梧、康、封、端诸州皆被侬军占领,侬军兵力膨胀至二万,于五月进围广州城\footnote{贺喜:《亦神亦祖:粤西南信仰构建的社会史》,页42}。

宋广州知州仲简系一愚蠢狠毒之人,不相信侬军会打到广州。侬军围城前,仲简不但囚禁了前来告急的人,还下令将百姓“相煽动欲逃窜”者斩首。侬军兵临城下时,仲简紧闭城门,不准躲避战乱的百姓入城,称:


\begin{quote}

我城中无物,犹恐贼来,况聚金宝于城中邪\footnote{黄振南:《侬智高》,转引自范宏贵主编:《侬智高研究资料集》,页429}?
\end{quote}

此后,走投无路的城外百姓纷纷加入侬军,仲简可谓自食其果。围城战开始后,胆怯的仲简不但坚守不出,还丧心病狂地放纵兵士割取城中民众首级以邀功\footnote{同注326}。然而由于广州城防坚固,侬军以楼车、石炮力攻五十七天仍不能克\footnote{同注324}。九月,困于坚城之下的侬智高担心宋军援兵来攻,遂引兵西去,克昭州、宾州,于十一月返回邕州。在昭州城外,侬军曾屠杀数百平民,犯下了战争罪行\footnote{同注326}。

自侬智高反宋起,宋廷即深为恐慌。在侬军横扫南粤、如入无人之境时,宋仁宗急忙命枢密副使狄青统军南下,与湖南江西安抚使孙沔、广南西路安抚使余靖所部会合,总兵力多达三万一千人。狄青乃北宋名将,用兵诡诈。他一面下令调集十日军粮、按兵不动,一面于下达命令的次日急速行军,到达天险昆仑关下,时为1053年正月十五日。为进一步麻痹侬智高,狄青大张灯烛、分宴诸将,其本人却率兵冒风雨夜度昆仑关,列阵于邕州城外之归仁铺,并派人传令诸将过关会合。直到这时,侬智高方知狄青已兵临城下,显然为时已晚。十六日白天,侬智高亲率全军出城迎战。侬军穿红衣,持大盾、标枪,列阵三行,望之如赤火。战斗开始后,侬军以标枪杀死宋军先锋王简子、破其前军,惊得狄青汗如雨下。然而,宋军毕竟人数占优,侬军无论如何无法击溃其阵。随着战局的发展,宋军渐渐压制了侬军,侬智高不得不下令全军撤回邕州城。归仁铺之战,侬军伤亡惨重,被斩首二千二百级、被俘五百余人,余部已无心再战。次日,宋军攻入邕州城,屠杀侬军军师黄师宓以下三千余人。狄青更将邕州城内外的5341具军民尸体聚拢一处,在邕州城北筑成“京观”以夸耀赵宋的“赫赫武功”,充分展现了宋帝国的暴虐本质\footnote{黄振南:《侬智高》,转引自范宏贵主编:《侬智高研究资料集》,页430}。

城陷前夕,自知大势已去的侬智高率其家人弃城西逃,前往粤滇边境的特磨寨落脚。特磨寨酋长侬夏卿系智高之友,友善地收留了侬智高一家。智高之母阿侬与侬夏卿合力练兵,很快收集到了余部三千余人,试图东山再起。可是在不久之后,一员侬军裨将被宋军俘获,泄露了侬军军情。侬智高至此完全心灰意冷,竟抛下了家人,率兵五百投奔滇地的大理国。当年十二月,宋军攻破特磨寨,俘获年过六旬的阿侬、侬智高的亲弟及两个儿子,将他们押往汴京。擒获阿侬的宋将杨元卿本想与人分食其肉以泄愤,却被宋仁宗惺惺作态地阻止,要求杨元卿如“孝子以养亲”地侍奉阿侬。宋仁宗此举不过是为了诱降侬智高。1055年四月,大理国因担心得罪宋帝国而捕杀了侬智高,他的亲人至此完全失去了统战价值。当年六月,侬智高之母、弟、子皆被斩杀于汴京,其中最小的孩子只有十岁\footnote{黄振南:《侬智高》,转引自范宏贵主编:《侬智高研究资料集》,页431}。在宋越夹缝间的广源州侬氏,至此退出了历史舞台。

侬智高的传奇生涯,是南粤与越南分离后未能划分明确边界的历史产物。摇摆于宋越之间的侬智高自视为第三方势力。在这场由宋帝国、越南李朝、侬氏共同上演的“三国志”中,已被侵占的南粤完全失去了话语权,沦为各方势力角逐的战场。由侬军入粤时曾受到粤人的热烈响应来看,侬智高实为我南粤之友。虽然侬军亦曾在南粤犯下过屠杀无辜的战争罪行,但其残暴程度远不及宋军。假如侬智高真的能攻下广州,长久地据有南粤,那么他很可能会建立起又一个南粤本土王朝。可惜的是,在狄青的诡计下,南粤重获自由的历史路径被堵死了。随着宋侬战争的结束,许多南粤城市、乡村已经化为废墟。然而,只要宋越间的边境争端尚未解决,南粤便会继续沦为战场。二十余年后,一场更加残酷的战争向南粤袭来,那便是宋帝国与越南李朝间的“宋交战争”。

公元1068年,宋神宗登基,随即任用宰相王安石推行所谓的“新法”。新法中的青苗、市易、保甲、保马等法不但加强了国家对社会的经济、军事控制,亦十分扰民,使帝国境内苦不堪言。为建立边功以证明新法有效,王安石积极准备进攻越南(交趾)之事,命人于边境修缮兵器、战船,做好南侵准备,并禁止南粤边民与越南人贸易,严重影响了无数粤人的生计。越南李朝因此深感不安,派出宋帝国询问的使节又反被扣留,乃决定先下手为强。1075年十二月,李仁宗命辅国太尉宦官李常杰、大将宗亶率十万大军北伐。越南军打出的旗号是拯救被王安石新法残害的百姓,但对粤人而言,越南军赶走宋军完全是前门拒虎、后门引狼。由于宋军无能,越南军迅速攻陷钦州、廉州,屠杀八千余人。次年初,越南军经四十余日战斗攻下邕州,灭绝人性地将全城自知府苏缄以下的五万八千人全部屠杀\footnote{陈重金:《越南通史》,页79}。当年十二月,宋军发动反攻,将越南军逐出南粤。次年底,李常杰指挥越南军在富良江成功阻击宋军,双方遂罢兵议和\footnote{李常杰虽是屠杀粤人的大屠夫,但无疑是越南的民族英雄。在富良江之战中,他作诗一首,伪称其为天书以激励守军。这首诗气势如虹,充满了越南爱国主义情操,对于我南粤抵抗岭北帝国亦很有激励作用。诗曰:“南国山河南帝居,截然定分在天书。如何逆虏来侵犯,汝等行看取败虚!”见陈重金:《越南通史》,页74}。在越南军占领钦、廉、邕三州的一年中,死于其手的粤人多达十万以上。至于被掳到越南去的人,当亦为数不少。1078年,宋交达成和议,两国恢复朝贡关系,宋帝国要求越南将俘虏的人口全部交还,然后方从越南撤军。次年,越南将221名俘虏放还,其中男子十五岁以上、二十岁以上者分别刺“天子兵”、“投南朝”字样于额,妇女则刺“官客”于臂,极尽羞辱之能事。由越南归还俘虏人数之少来看,大部分被掳粤人很可能已被折磨致死。越南军之残酷,于此可见一斑\footnote{陈重金:《越南通史》,页75}。

公元1081年,宋帝国将广源州割予越南,宋越争端至此落下帷幕。在宋交战争中,我们的祖先沦为宋帝国对抗越南的人肉盾牌,遭到了越南军疯狂的屠戮。越南人奋起反抗暴宋的压迫固然是正义之举,但他们对南粤百姓的屠杀则显然是极度卑劣的行径。失去国际话语权的南粤别无选择,只能任由战争双方蹂躏,成为越南人宣泄对宋帝国仇恨的对象,遭受了比宋侬战争更大的破坏。一旦失去了来之不易的自由,处在宋越夹缝间的南粤便陷入了这样悲惨的境地,可见自由对于我南粤而言是多么地珍贵。宋侬战争和宋交战争表明,只有在能够决定自己命运的情况下,粤人才不会沦为任人宰割的对象。

\section{“教化”与大洪水:北宋帝国在南粤的儒化尝试、湘赣流寇}

当宋帝国侵占南粤时,北方侵略者面对的是一个对他们而言完全陌生的文明。在nï 个文明中,儒学只在帝国官僚聚集的城市中存在。虽然汉代以来中原持续不断的战乱使许多北人入粤避难,但这些人中的大部分完全融入了南粤,并未给本土思想带来太大冲击。当时,除了由印度和阿拉伯输入的佛教、伊斯兰教外,南粤还存在着各种本土信仰。我们虔诚的祖先敬拜着形形色色本土神祗,对帝国的统治思想儒学没有表现出什么兴趣。在漫长的历史上,如陈元(约公元25年前后在世,苍梧人)、杨孚(约公元77年前后在世,南海番禺人)、张九龄(678—740,韶州曲江人)般的粤人儒者虽不时出现,但他们并不代表当时南粤社会的主流。

本土化的道教是重要的南粤宗教之一。早在4世纪,道教史上的著名人物抱朴子葛洪(吴越丹阳句容人)便与南粤结下了不解之缘。公元327年,葛洪听闻南粤出产丹砂,遂向东晋朝廷请求出任勾漏(今广西北流)县令。晋廷批准后,葛洪南下赴任,首先在广州会晤了刺史邓岳。邓岳告诉葛他,粤东罗浮山乃一神仙洞府,据说秦时方术家安期生曾在此服食九节昌蒲、羽化登仙。葛洪闻之十分神往,遂抛弃官职,于330年携全家入居罗浮山,于山上炼丹烧药、建庵授徒。葛洪先于朱明洞前筑南庵,后因从学者日众,乃增建东庵九天观、西庵黄龙观、北安酥繆观\footnote{《岭南中医药名家1》,页11—12}。葛洪在云烟缭绕、风景优美的罗浮山上度过了十一年的悠游岁月,于341年病逝,时年61岁\footnote{葛洪之死系年有两说,一为343年,一为363年。据胡守为考证,当以343年为是。详见胡守为:《岭南古史》,页327}。至第四次北属时期,南庵香火日盛,成为道教十大洞天之一,被称为葛仙祠。1088年,该庵又被宋哲宗“赐”名“冲虚观”,延续至今\footnote{《岭南中医药名家1》,页12}。朱明洞并非山洞,乃一遍布圆石的山坡,这一地形很可能是某种“本地信仰的崇拜对象的所在”\footnote{科大卫:《皇帝与祖宗:华南的国家与宗族》,页27。南粤史上,所谓的“洞”往往指山间小盆地,而非山洞。}。在罗浮山上海流传着葛洪之妻鲍姑治病救人的传说。鲍姑是曾任南海太守的葛洪之师鲍靓的女儿。据说,精通艾灸之术的她“善灸赘疣”,造福了一方百姓\footnote{胡守为:《岭南古史》,页330}。鲍姑去世后化为南粤本土道教的神祗,被供奉于始建于4世纪并延续至今的广州著名道观三元宫\footnote{余信昌、黄诚通:《鲍姑与三元宫》,《中国道教》1984年15期}。

另一个重要的女性本土神是悦城(今广东德庆)龙母。相传,龙母姓温,系第一次北属时期之人。她从小便能预知祸福,乐善好助,被人呼为“神女”。一日,龙母在西江边濯洗时拾得大卵一枚,孵出五条神龙,因而被称为“龙母”。龙母常令五龙呼风唤雨,造福一方农耕。龙母逝世后,当地人感念她和五龙的恩德,将她葬于西江北岸珠山下,建庙以祀之,是为至今仍存的龙母祖庙\footnote{郑靖山:《宗教文化看岭南》,《精细品鉴》,页211}。后来,无耻的帝国文人编造出了龙母嫁给秦始皇为妃的故事,意图将悦城龙母信仰整合到帝国体系中\footnote{科大卫:《皇帝与祖宗:华南的国家与宗族》,页27}。然而对粤人来说,她始终是个保境安民的神灵。至第四次北属时期,悦城龙母信仰已传遍整个南粤各地,龙母受到了无数粤人的敬奉\footnote{蒋明智:《悦城龙母祖庙:龙母信用的核心所在》,《中国南海民俗风情文化辨岭南沿海篇》,页218}。

南粤与岭北帝国的神祗争夺战亦体现在著名的波罗神身上。波罗神信仰的前身是珠江口扶胥镇(今广州黄埔)的一个粤人神坛\footnote{同注342}。公元594年,隋文帝突然“下诏”,将这一神坛指定为祭祀南海神的场所,于此建立了延续至今的南海神庙。所谓的南海神即是华夏世界信仰体系中的火神祝融,兼有掌管水的能力\footnote{颜志图:《南海神与波罗神》,《羊城讲古》,页28}。隋帝国希望塑造出一个为帝国镇守南海的神祗,从而在南粤的本土信仰体系中插入忠于帝国的元素。此后的历代岭北帝国皆十分重视南海神,唐、宋帝国曾分别册封其为“广利王”、“洪圣”,将其抬到无以复加的高度\footnote{同注342},希望以此提高其影响力,进而在粤人的思想中植入大一统木马。帝国的阴谋无疑落空了。在第四次北属时期,我们充满智慧的祖先找到了对抗方法。当时,在扶胥镇流传着这样一个故事:某年,一位名叫达奚司空的印度人乘商船来到扶胥。下船之后,达奚司空沉醉于岸上美丽的南粤风光,独自四处观赏,将两颗波罗(即至今仍十分常见的岭南佳果菠萝蜜)树种撒在南海神庙中。由于游兴太浓,达奚司空错过了商船返航的时间。待他反应过来时,商船早已远去。思乡心切的达奚司空悲伤欲绝,终日手搭凉棚眺望南海、放声痛哭,最后化为一块石头,被水淹没。这一故事的发明者今已不可考,应当是由无数南粤先民口耳相传保留下来的,颇能反映其实的南粤作为国际贸易重要参与者的特点。能够确定的是,我们的祖先巧妙地利用了这个故事,在南海神庙中塑造了一尊黑皮肤、左手举于额前作望海状的达奚司空雕像,将他作为南粤本土神袛加以供奉。由于达奚司空曾播种过波罗树,因此也被叫做“波罗神”。

我们的祖先对波罗神十分崇敬,南海神则被抛在一边,大多时候只接受帝国官方例行公事的敬拜。我们的祖先不但将南海神庙称作“波罗庙”,甚至还把每年二月十三日的南海神诞称作“波罗诞”。许多粤人都认为,南海神就是波罗神,而不是那个由帝国强加的祝融。到12世纪,宋帝国朝廷也不得不承认波罗神的“合法”地位,“册封”其为“助利侯”,并自欺欺人地继续宣称南海神是祝融\footnote{颜志图:《南海神 tonk 波罗神》,《羊城讲古》,页28—29}。经过数百年的较量,我们的祖先终于用本土的波罗神替换了帝国的南海神,在这场神祗争夺战中胜出。

在粤东潮州,人们敬奉着被称为“三山国王”的神灵。所谓“三山”,乃当地的巾山、明山、独山(在今揭西县河婆镇),“三山国王”则是这三座高山的山神。早在漫长的百越时代,我们的祖先便有崇拜山神的习俗。至迟到冯冼时代,三山国王信仰的雏形已在粤东出现。据说,当时三山产生了神迹,感到由衷敬畏的百姓遂于巾山下建庙奉祀三山神。在7—8世纪,潮州人对三山神的信仰已十分虔诚,每年都会定期祭祀之,以求禳灾纳福。公元819年,唐帝国名臣韩愈被贬至潮州为官。当时潮州秋雨连绵、农业歉收,韩愈乃向三山神祈祷,收到了雨过天晴、五谷丰登的奇效。后来,岭北帝国见此三山神法力强大,遂决定将其占有。无耻的帝国文人编造了这样一个故事:在宋太宗侵略三晋攻打太原城时,得到了金甲神人挥戈驰马突阵相助。战后,神人重返天界,于云中对宋军说了“潮州三山神”五字。宋太宗非常感激,乃将三山神“册封”为国王,称之为“三山国王”。这一故事无疑是想表明,三山神有匡扶帝国大一统之功,乃岭北帝国在南粤的帮手。岭北帝国的阴谋又一次失败了。对粤东人来说,“三山国王”仍然只是保佑乡土风调雨顺的地方神。虔诚的潮州人在向三山神顶礼膜拜时,心中所想的绝非保佑帝国国祚延绵万世,而是祈求家人、乡土的平安\footnote{卢颐:《粤东三山国王崇拜的起源与演变》,载陈景熙主编:《潮青学刊》第1辑,页556—561}。

在粤西南的雷州半岛,人们则崇拜着石狗与雷祖神。早在百越时代,当地的越人部落就以狗为图腾,并雕刻了形形色色的石狗。至冯冼时代,当地更产生了关于雷祖的传说,使石狗崇拜大为发展。雷祖名叫陈文玉,在历史上确有其人,于公元570年在雷州城附近的榜山村出生。公元630年,他开始担任东合州(今雷州)刺史,施行宽仁的统治,受到了家乡父老的由衷敬爱。他曾“奏请”唐廷将东合州改名为雷州,获得批准,使雷州获得了延续至今的名字。638年,陈文玉在任上去世,雷州百姓悲痛万分。四年后,唐太宗突然追封陈文玉为“雷震王”,命人于雷州城西南部建庙祀之\footnote{曾国富:《广东地方史·古代部分》,页95—96},以“表彰”他对唐帝国的忠诚以及在家乡的德政\footnote{唐太宗称陈文玉“恶事非君,受职父母邦,德政彰明”,转引自高诚苗主编:《雷祖活动文化纪实·资料集》,页347}。唐太宗的这一举动是十分阴险的,他妄图利用雷州人对陈文玉的怀念将其打造为忠君爱乡之人,进而向他们潜移默化地灌输忠于帝国的思想。然而,我们的伟大祖先绝不会受帝国的蛊惑。在雷州人看来,陈文玉永远都是那个泽被一方的同乡。到第五次北属时期,一则关于陈文玉的神话开始在雷州流传。该神话称,陈文玉未出生时,其父以捕猎为生,曾养过一只长有九耳的怪狗。陈父每次打猎前,都会数一数有几只狗耳在动,其后所获猎物数量皆与之相合。一日,怪狗九耳齐动,陈父大喜,知此番出猎必大有所获。其后,陈父果然在丛林中拾获一枚巨蛋,带回家中。良久,天空中忽然涌起无数云朵,一道雷电直劈陈家将蛋炸开,一个小男孩从中跃出,左右手各有“雷”字、“州”字。乡人视之为雷电的子孙,呼之“雷种”。这个男孩,正是大名鼎鼎的陈文玉\footnote{贺喜:《亦神亦祖:粤西南信仰构建的社会史》,页111}。

这样一来,陈文玉便与雷电结合起来,被人视为雷神。人们亲切地称他为“雷公”、“雷祖”,认为他是上天带给雷州的礼物。至此,人们不再认为陈文玉是个忠于帝国的官僚,而把他当做了应当敬奉的雷祖神。由于九耳怪狗在这则神话中占据着重要地位,雷州人对于古时百越部落的图腾狗亦更加尊崇,石狗崇拜这一百越传统宗教活动随之日益昌盛。在今天的雷州半岛,形形色色的石狗雕像分布在城市、乡村的各个角落,最大者高达2.5米,最小者仅高10公分\footnote{王增权:《雷州石狗简介》,《广东民族研究论丛(第六辑)》,页296}。这些可爱的精灵守护着粤人的家门、村口,为人们驱魔镇妖、去邪消灾\footnote{牧野主编:《雷州历史文化大观》,页278}。

除了上述几个著名的神祗外,当时的南粤各地还存在着数不胜数的种种大小神灵。他们或源自百越自然宗教传统、或是被神格化的历史名人。他们一同构成了丰富多彩的南粤神界,用无边的法力守护着南粤的平安。

对于南粤此种与岭北格格不入的社会面貌,宋帝国以前的士人漠不关心。在他们看来,南粤仅仅是个海外贸易发达、光怪陆离的地方,有着可怕的“瘴气”与令人垂涎的财宝, 是帝国在遥远的南方开拓出的一块殖民地。至于粤人的生活方式、精神世界,他们没有了解的兴趣。然而,宋人的态度却为之一变。自晚唐起,帝国的儒者日益感受到来自佛教的挑战。由于儒学不能解决人们对于生命与宇宙的终极疑问、不能保佑人们避免灾祸,因此无法贴近社会,在底层民众中缺乏信徒。为应对这一挑战,宋儒产生了一种狂热的传教精神。他们吸收了佛道两教的宇宙观和本体论,试图让儒学扎根于社会基层。

公元1072年,宋儒开始了儒化南粤社会的初步尝试。是年,广州狭小的府学(广州府教授儒学、培养儒者官员的学校)在知府程师孟的主持下得到扩建。然而,扩建后的府学仍然很小。1087年,新到任的广州知府蒋之奇惊讶于广州儒学的简陋,下令重修。数年后,他的继任者章楶不但将府学再次扩建,还“痛心”地描述了当时广州的社会风俗:

\begin{quote}

盖水陆之道四达,而蕃商海舶之所凑也。群象珠玉,异香灵药,珍丽瑰怪之物之所聚也。四方之人,杂居于市井,轻身涉利,出没于波涛之间,冒不测之险,死且无悔。彼既货殖浩博,而其效且速,好义之心,不能胜于欲利,岂其势之使然欤?俗喜游乐,不耻争斗。妇代其夫诉讼,足蹑公廷,如其在家室,诡辞巧辩,喧啧诞谩,被鞭笞而去者无日无之。巨室父子或异居焉,兄弟骨肉急难不相救;少犯长、老欺幼而不知以为非也。嫁娶间无媒妁者,而父母弗之禁也。丧葬送终之礼,犯分过厚,当然无制。朝富暮贵,常甘心焉。岂习俗之积久,而朝廷之教化未孚欤\footnote{《元大德南海志残本*附辑佚》,转引自科大卫:《皇帝与祖宗:华南的国家与宗族》,页38}?

\end{quote}

此段文字是11世纪后期北人观察南粤社会的一手资料,值得做一番讨论。章楶首先指出了广州海外贸易发达、“珍丽瑰怪之物”、“四方之人”众多的事实,接着笔锋一转,认为粤人轻义好利、喜游乐、爱争斗,缺乏儒家伦理中的父子、夫妇、老幼观念,丧葬习俗亦完全不遵从儒家礼仪。章楶得出的结论是:南粤的社会风貌之所以如此“不堪”,都是因为缺乏“朝廷之教化”。

如果我们站在南粤的立场上分析这段文字,便能得出完全不同的历史图景。粤人好利、喜游乐、爱争斗,正是商贸发达、热爱生活、充满血性的表现。至于章楶说粤人“轻义”则无疑是污蔑之词。如前所述,我们伟大的祖先对帝国暴政的反抗是一直没有停止的。反抗岭北帝国、为南粤的自由而战,便是粤人的大义。至于粤人的伦理关系与儒家的规定完全不同,则说明南粤拥有与岭北完全不同的社会结构,粤文明自成一体。在这样一个社会中,宋儒若想推行儒学,可谓是种植无根之木,一定会面临巨大的阻力。广州府学所在的位置即很能说明问题。由于建设府学需要富人出资支持,宋帝国官员不得不将府学设在穆斯林富商聚集的蕃坊。程师孟于1072年扩建府学之举,就是在一位名叫卒押陀罗的阿拉伯人的资助下完成的。然而,这些穆斯林并没有“归化”儒学\footnote{科大卫:《皇帝与祖宗:华南的国家与宗族》,页39}。在缤纷多彩的广州城中,府学设在异域风情最为浓厚的蕃坊,就如同水上的一叶浮萍,在南粤社会的背景下显得格格不入。

在北宋帝国治下,南粤本土的儒化精英颇少,曾在宋侬战争中积极作战的余靖是其代表。余靖乃粤北韶州曲江人,系张九龄同乡。当时,由于粤北距离岭北较近,因此受到帝国影响的程度远较南粤其他地域(广州除外)为大。宋帝国1080年的户口统计数据显示,在南粤境内,粤北南雄、潮州、连州三府的人口密度仅次于广州、潮州\footnote{马立博:《虎、米、丝、泥:帝制晚期华南的环境与经济》,页62}。这一数据并不能反映真实的人口数字,仅表明被帝国编户者的多少。粤北被编户者众多,正表明当地受帝国控制程度颇深。换言之,若有大洪水自岭北来袭,当地受波及的可能性也更大。两宋之际,这一可能不幸成为现实。公元1127年,靖康之变爆发,北宋帝国灭亡。1129年,金军南下追捕宋高宗,经吴越一直打到江西南昌。次年,金军又攻陷湖湘长沙并屠城。金军虽然很快北返而去,但湘、赣的社会却完全失控,变为兵匪横行、流贼肆虐之地。在1130—1135年间,这些流寇垂涎于南粤的富饶,不断南下,一次又一次地蹂躏粤北。以下列表反映这六年间流贼入粤的情形\footnote{此表之作,参考《广东通史》古代上册,页840—842}:

\begin{quote}

1130年 湖湘茶陵流贼入掠韶州,被当地宋帝国官员“招抚”

1131年 湖湘宜章流贼李冬率部两次进犯英州、连州、韶州,最后亦被“招抚”

1132年 江右虔州流贼陈颙、谢达、谢宝部入寇,攻围循、惠、南雄、梅等州县

1133年 江右流贼陈颙、罗闲十等四百余股共十万余人大举入寇,掠循、梅、潮、惠、英、韶、南雄等州县,一度威胁广州

1134年 湖湘流寇入掠韶州、连州,一度攻至广州、惠州、循州郊外

1135年 江右流贼周十隆部攻围循、梅、潮等州

\end{quote}

由上表可见,湘赣流寇在1130—1135年间每年都会入侵南粤。在流寇攻势最猛烈时,不仅粤北受害,连粤东和广州都遭到波及。这一乱局直至1135年南宋名将岳飞平定湖湘钟相、杨幺之乱方告一段落,然粤北已在流寇的反复杀戮下遭受到不可挽回的损失,陷入了“人不聊生”、“数里无人烟”的悲惨境地。直至14世纪末,粤北的人口都远未能恢复到流寇杀戮前的规模\footnote{同注356。这一数字,是对比帝国于1080年、1391年的统计数据得出的结论。1391年的统计系朱元璋所为,相当严格,较能反映真实的人口数字。1080年南雄府、韶州府人口密度分别为每平方公里4.57户、3.29户,1391年两府人口密度分别为2.00户、1.02户。考虑到1080年的统计数字距实际情况较1391年有更大距离,则流寇给粤北带来的人口损失定是十分骇人的,称之为灭绝性屠杀亦不为过。}。

湘赣流寇对粤北的杀掠是南粤史上极为悲惨的一幕。面对骤然降临的大洪水,粤北土豪中未能有人站出来领导大家封锁南岭、合力抗贼,结果使无数粤人惨遭屠戮。这一历史教训表明,南岭实为我南粤隔绝岭北最重要的地理屏障。今后为保卫南粤而战的义士必须要牢记这一点,以免类似的惨剧再度上演。

金兵南侵与流寇入掠岭北给南粤历史带来了两个重要影响。首先,一批北人为逃避岭北战乱而翻越大庾岭、经南雄珠玑巷移入南粤。这些人大多数都完全归化了南粤,但与南雄珠玑巷有关的共同历史记忆却流传下来,最后在16世纪结出绚烂的果实,为南粤的民族发明埋下了伏笔。第二,许多粤北百姓为躲避流贼而向南迁徙,使广州附近的人口大为增长,从而开启了如史诗般壮阔、长达近千年的珠三角大开发\footnote{马立博:《虎、米、丝、泥:帝制晚期华南的环境与经济》,页64。马立博点出当时了粤北人口下降、广州人口上升的事实,但归因于蒙古入侵,完全没有考虑湘赣流寇对粤北的破坏,显然是很不准确的。}。这两个问题将在本书近代史部分进行详细讨论。现在,让我们将目光放回12世纪,看看南宋帝国治下的南粤发生了何种变化。

\section{南宋帝国治下的南粤}

宋儒的传教狂热在理学家身上表现得最为明显。在理学未兴的北宋,士人在南粤建立的儒学书院仅有寥寥数所。至南宋帝国治粤期间,南粤大地上先后共出现了33所书院。随着大量书院的兴建,理学得以入粤。

公元1132年,二程的再传弟子、闽越人罗从彦担任宋博罗县主簿,在县境内修建了六所书院。罗从彦虽未在南粤境内培养亲传弟子,但无疑使粤人首次认识到了理学。1146—1150年间,南宋名臣张浚(巴蜀人)被贬至粤北连州为官,其子张栻即在当地完成了对《周易》、《论语》的学习。1161年,张栻前往湖湘衡山师从二程三传弟子胡宏修习理学,后在长沙主讲城南、岳麓两书院,成为理学“湖湘学派”的领军人物,亦与罗从彦的再传弟子朱熹(朱子)成了好友。张栻的名声使帝国各地的学者纷纷前往长沙拜他为师,这其中就包括南海人简克己。简克己师事张栻数年,得到了张栻学问的真传。回粤后,他杜门不出,绝意于科举,一意追寻“真知实践”与“性理之学”\footnote{郭棐:《粤大记》卷14}。

简克己的例子表明,我们的祖先往往并不热衷于入仕帝国。他们中若有人接受了儒学,亦很难变成大一统的鼓吹手,而是会将之视为一套修身养性、为人处世的工夫。此后,理学进一步在南粤传播。1170年,宋韶州知府周舜元主持修建相江书院于韶州,祀理学开山祖师周敦颐\footnote{(同治)《韶州府志》卷18《建置略》}。1211年,担任宋广州知府的朱子弟子廖德明(闽越人)于广州刊刻出版《朱子家礼》,是为这部重要理学著作的首次面世\footnote{周鑫 朱子家礼 研究回顾与展望》,《中国社会历史评论》2001年}。

在南宋帝国治下,随着儒学(尤其是理学)的不断传播,南粤出现了崔与之(增城人)、李昴英这两位本土大儒。崔与之于1193年考中进士,此后长期担任帝国官僚,于1232年辞官回到广州,成为南粤士人的领袖,声望很高\footnote{郭棐:《粤大记》卷16}。李昴英是崔与之的学生,于1225年中进士,后官至宋吏部侍郎。1239年崔与之去世后,李昴英成为南粤士林的新领袖。他不但积极钻研理学,更于致力于在粤传播儒家礼乐。1244年,李昴英的好友、广州知府方大琮(闽越人)在广州城内举行了隆重的儒家礼仪乡饮酒礼。当时,二百余名广州耆老被方大琮召集起来齐聚一堂,宾主双方在肃穆的音乐中一同饮酒,讲习礼仪\footnote{科大卫:《皇帝与祖宗:华南的国家与宗族》,页45—46}。李昴英亦获邀出席此次礼仪,深感兴奋的他详细记载了nï 次盛典\footnote{杨正华:《李昴英》,页48}。

虽然南宋帝国治下的南粤出现了自己的理学精英群体,但终宋一朝,理学始终未能深入南粤社会基层。李昴英于1257年去世,他没能再在广州看到第二次乡饮酒礼。事实上,这一礼仪仅是昙花一现,直到宋亡时都再未举行。在朱子的规划下,社会基层应当遍布以宗族祠堂、义仓、社学为核心的小共同体。然而,当时的南粤到处都是自百越时代延续下来的各种乡土共同体,土豪、佛寺及种种本土神祗是我们祖先在生活中的核心,他们并无接受理学的必要。因此,崔与之、李昴英等人虽然名声显赫,却始终未能与乡邦共同体融为一体。

1135年后,虽然湘赣流寇的连续入掠已告一段落,但直至13世纪,湘赣流寇仍不时侵略粤北、粤东。在1141、1165、1175、1179、1208、1233年皆有流寇入粤的记录,连、韶、英、循、梅、潮等州大受其害,被流寇与宋军屠杀的南粤百姓更是不知凡几\footnote{《广东通史》古代上册,页842;华山:《南宋绍定、端平年间的江、闽、广农民大起义》,《文史哲》1956年}。与此同时,我们伟大祖先对宋帝国的武装抵抗亦从来没有停止。1142年,宋廷开始推行丈量田亩的“经界法”以扩大税源。在南粤,宋帝国官僚借推广“经界法”的机会大肆打击土豪、横征暴敛、欺辱百姓。1149年,忍无可忍的海南文昌、琼山之民在当地人陈集成的带领下发动起义。宋廷震慑于义军的战斗力,不得不做出妥协,下令在海南停止“经界”\footnote{《广东通史》古代上册,页842—843}。1170年代,广南西路(广西)境内发生饥荒,宋帝国官僚不但不积极救灾,还继续征收高额赋税,造成了“连年荒歉,饿殍满路”的惨景。1179年五月,陆川(今广西陆川)弓手李接率当地军民发动了轰轰烈烈的反宋大起义。他们专杀宋帝国官吏,不屠百姓,每到一处都开仓放粮,受到沿途粤人的热烈欢迎,很快就攻克了容州、雷州、高州、郁林等地。在起义军占领区,人们尊敬地称李接为“李王”,呼宋帝国侵略军为“贼”。至十一月,起义在宋帝国的大力镇压下失败,李接及义军官兵265人被侵略军俘送广西静江酷刑处死,兽性大发的侵略者还吃掉了他们心肝\footnote{《中国农民战争史(宋辽金元卷 四)》,页182—183}。岭北帝国对待粤人的残暴程度,可以说已经超过了食人野兽。

1197年,宋帝国侵略军又在南粤犯下了更疯狂的罪行。当时,在珠江入海口处的岛屿大奚山(今香港大屿山)上,生活着许多以捕鱼、晒盐为生的和平居民。是年,宋帝国为垄断南粤食盐贸易,突然指责大奚山岛民贩卖“私盐”,派兵上岛抓捕从事盐业者。对岛民来说,贩盐是他们的生计来源。若不想活活饿死,他们只能反抗。于是,武装起义发生了。对待反抗的岛民,宋广州知府钱之望采取了恐怖屠杀的政策,命令登岛镇压的宋兵将岛民一个不留地全部屠戮。经宋帝国侵略军的疯狂屠杀,昔日繁荣的大奚山成了一片满是尸体的废墟,大奚山的原住民几乎被完全灭绝了\footnote{韩振华:《试析福建水上疍民(白水郎)的历史来源》,《民族研究论文集 第1集》}。

1235年,一场新的反抗在广州发生,是为规模浩大的“催锋军兵变”。“催锋军”系宋帝国在广南东路(广东)一支地方部队的名号,始创于南宋初期。据宋廷规定,广东的军队需要前往岭北“戍守”,每两年更换一次人员。由于南宋与金、蒙古帝国连年交战,这一规定被彻底破坏。是年,已远戍建康四年之久的催锋军终于接到撤戍命令。然而他们刚走到南岭便接到了不得回粤、留守江西的命令。这批归乡心切的南粤子弟兵彻底愤怒了。他们为宋帝国做了多年的炮灰,到头来却仍不能回家休整。在一位名叫曾忠的军人的率领下,催锋军决定用武力打回家乡。他们越过南岭,首先攻下惠州,接着又于二月七日进围广州城,声言要擒获城中的宋帝国官僚才能甘心。宋帝国广州守将曾治被催锋军吓破了胆,立刻出海逃之夭夭。这时,正在城中的崔与之、李昴英不忍这些南粤子弟兵横遭杀戮,亦担心广州城遭遇战火蹂躏,遂挺身而出担任调解人。时年七十八岁的崔与之登上城头,对催锋军“晓以祸福”,并命李昴英出城劝告官兵停战。在李昴英的劝解下,大部分官兵皆散去,余部于两日后撤围,经肇庆退往四会、怀集一带。这时,宋帝国又令崔与之任广东经略使镇压催锋军余部。崔与之不愿投入这场同胞相残的内战,但宋廷强迫却这位老人上任。无奈之下,崔与之只得违心地投入战事,于七月间调集循、连、南雄三州之兵镇压了这次起义\footnote{杨正华:《李昴英》,页41—42}。宋帝国挑动粤人自相残杀,可谓卑鄙、阴险至极。

正当宋帝国疯狂蹂躏南粤之际,蒙古侵略者又接踵而来,给我们的祖先带来了更大的苦难。1253年,蒙古军侵占大滇,大理国灭亡。1258年,蒙古蒙哥汗命制定南北对进的攻宋计划,命蒙将兀良合台率偏师自滇地经广西进攻湖湘,自领主力南下,约定两军于第二年会师于长沙。兀良合台之军连破宾州、象州、静江(今广西桂林),直抵长沙城下,却为守军的激烈抵抗所阻,只得渡长江北上与蒙哥会合\footnote{胡泊主编:《蒙古族全史上》,页437—438}。次年,蒙哥汗在巴蜀合州钓鱼城下被宋军打死,宋蒙战争暂时告一段落。1276年,宋蒙战火再次降临南粤。是年二月,宋恭帝对元投降,元军占领宋都临安。三月,元军迅速夺取湖湘、江右,又逾岭南下入侵南粤。面对元军的进攻,驻粤的宋帝国将领皆畏缩不前,只有南粤子弟兵催锋军在将领黄俊的率领下奋起反击,于南越国曾激烈抵抗汉军的古战场石门与元军展开鏖战,不幸失败。六月十三日,元帝国侵略军攻陷广州,黄俊英勇就义\footnote{《广东通史》古代上册,页928}。

在南宋官军望风溃败、元帝国侵略军横行肆虐的危局下,一位土豪站了出来,担负起保卫南粤的责任,他便是东莞人熊飞。熊飞曾随文天祥在岭北作战,“有武略,善骑射”,是个顶天立地的南粤男儿。元军入粤后,他在东莞聚集民兵,于八、九月间两次反攻,一举击败立足未稳的侵略军,收复了广州城。这时,宋制置使赵溍突然来到广州,将熊飞的义军收归麾下,夺取了义军的胜利果实。随后,英勇善战的熊飞又率军一举收复韶州、南雄,直抵大庾岭下\footnote{《广东通史》古代上册,页929}。一时之间,元帝国侵略军似乎已被成功击退了。

不幸的是,元军主力此时尚聚集于大庾岭,未有损失。十月,元军再次南下,攻陷南雄,攻至韶州城下。此时,熊飞正与宋将刘自立同守韶州城。关键时刻,刘自立突然打开城门投降,无耻地出卖了熊飞的义军。熊飞坚决不向侵略军低头,率领全城军民激烈巷战,最后投水自尽。至于韶州城的百姓,更是遭到了元军的疯狂屠戮。十二月初,贪婪怯懦的赵溍见大势已去,弃广州城而逃,元帝国侵略军第二次侵占广州\footnote{《广东通史》古代上册,页929}。

同年,元将阿里海牙率领另一支侵略军攻破越城岭上的天险严关,进入广西,直抵静江城下。守城宋将马塈虽非粤人(为甘肃人),但曾在广西为官,对于南粤文化颇有了解,深受静江百姓信任。在马塈的指挥下,全城军民同仇敌忾,与元帝国侵略军展开了长达三个月的苦战。十一月,元军破城,滥施屠戮。马塈被俘杀,其部将娄钤率残部250人以火药集体自焚。静江城中的百姓亦表现出了极大的气节,纷纷点燃居室,投漓江自杀,宁死不做元帝国的降虏,捍卫了南粤的尊严,谱写了南粤史上无比悲壮的一幕\footnote{胡泊主编:《蒙古族全史上》,页517—518}。静江失陷后,广西各地亦相继被侵占,南粤全境基本落入元帝国之手。

同年十二月,走投无路的宋廷经潮州流亡至惠州甲子门,南粤境内战火再燃。在粤东的潮州,由于宋帝国官僚早已逃之夭夭,全城军民推举催锋军将领马发(潮州本地人)担任知州,坚守城池。1277年初,南海、番禺、新会、香山等地的土豪、百姓不甘降元,纷纷起兵,于当年四月收复广州城。十月,马发率潮州军民击退元将唆都的进攻。在此有利局面下,南粤境内的宋帝国官军几乎完全无所作为,张世杰、文天祥两军皆告失利。十一月,元帝国侵略军第三次侵占广州城。为了报复粤人的不屈抵抗,他们凶残地夷平了广州的东、西二城\footnote{宋帝国统治时期,广州分为西城、子城(中城)、东城三部分。元帝国侵略军夷平东西二城,已将广州破坏过半。}。次年正月,元将唆都率军第二次进攻潮州。马发与全城军民竭力死战,防守月余。二月底,南门守将黄子虎打开城门放元军入城,侵略军大开杀戒,几乎屠杀了所有他们看到的百姓。三月一日,仍在城中率百余残兵坚持战斗马发不愿投降,率全家壮烈自尽,以生命履行了对乡邦的最后忠诚\footnote{《广东通史》古代上册,页931}。

1278年初,宋军一度夺回广州,不久后被元军逐出。四月,宋端宗病死于碙洲(今湛江以南),末帝昰登基。至五月,只有珠江口以西的香山、新会一带、粤东潮阳及海南岛仍在坚持抗元斗争。宋廷于六月迁往新会崖山海滨,率号称二十万人的残军苟延残喘。十月,元军招降海南岛的南宋官僚。十二月,文天祥于潮阳城外的五岭坡被俘。1279年二月六日晨,元将张弘范、李恒率大军对宋廷展开了最后的总攻,是为著名的崖山海战。当日黄昏,宋军惨败,宰相陆秀夫负末帝昰投海自尽,海上浮尸达十余万具。五月,南宋最后的抵抗者张世杰于南恩州平章港口(今阳江以南)舟覆身亡。至此,南宋帝国便彻底灭亡了\footnote{《广东通史》古代上册,页932—934。南宋遗臣在南粤的零星抗元战争持续到1283年。是年,宋臣林获之子林桂芳(吴越人)于新会拥宋宗室起兵,称罗平国,年号延康,旋被镇压。见《广东通史》古代上册,页946}。

在南粤的抗元战争中,南粤义兵是当之无愧的主力。无能的宋帝国官军坐拥十余万大军,只知抢夺义兵的胜利果实,或在关键时刻出卖义兵。在最后的决战中,他们困守崖山一隅,除纷纷投海自尽外毫无可敬之处。与此相反的是,保卫乡土的义兵表现出了极强的战斗力。熊飞以数百东莞乡民起兵,就能一举克复广州、韶州、南雄。马发凭城固守,亦能一度击退元军。在大一统史观的荼毒下,今人只知对张世杰、文天祥等人的事迹津津乐道,还有谁能记得大英雄熊飞、马发等人的英勇事迹,记得静江百姓的集体殉难?他们壮烈的故事就连石人也会为之感动落泪,更不用说有血有肉的我们了。热爱南粤的人们、为南粤的自由而战的人们,请一定不要忘记他们的故事、他们的牺牲。他们是南粤永远的英雄、是粤人永远的骄傲。


\section{元帝国治下的南粤}


南宋虽亡,但我们祖先的抗元斗争仍未停止。1283年,广州人路黎德、欧南喜起兵反元,攻克新会、增城等地,势力最盛时拥众二十万、有船数千条。直至1285年,这次起义才被镇压下去。而就在这一年,循州、梅州、潮州等地又发生了万人规模的起义,使侵略军疲于奔命\footnote{《广东通史》古代上册,页945—947}。1291—1293年,海南黎人(俚人后裔)展开了时长三年的抗元战争。元帝国不得不出兵渡海远征,方勉强镇压了起义。侵略军更在海南岛中央的五指山上刊刻石碑,大肆吹嘘其屠杀粤人的“赫赫武功”\footnote{贺喜:《亦神亦祖:粤西南信仰构建的社会史》,页61}。这样,直至元军入侵南粤后的第十五个年头,遍及全粤的反抗方在侵略者的屠刀下暂告中止。

在元帝国治下,粤人被视为低于蒙古人、色目人、汉人(北方“汉人”)的“南人”,受到了严密的监控和管制。元帝国首先在行政区规划上彻底拆散南粤,将广东、广西称为广东道、海南海北道,分别归于江西、湖广行省管辖。如此一来,一旦南粤出现反元斗争,驻江右、湖湘的元帝国侵略军便能迅速南下镇压。在1290年之前,由于担心南人反抗,元廷甚至不准南粤、江右、湖湘、闽越等处官府负责捕盗的县尉、弓兵携带武器\footnote{刘基:《宋史》卷16《世祖纪》}。此外,元廷又废除科举,排斥汉人、南人为官。除元初及其末年外,无一粤人担任元帝国的高级官职。直到1313年,元廷方重开科举,然此后亦时有停办。这一政策导致粤人完全决意于儒学,宋儒在南粤所取得的有限儒化成果毁于一旦。在元治南粤时期,只有广州一带出过少量进士,广州以外更是一个都没有\footnote{科大卫:《皇帝与祖宗:华南的国家与宗族》,页49}。

在儒学完全式微的情况下,佛教在南粤城市中变得日渐强势,乡村则是本土信仰与佛教共存的局面。在广州,佛寺与市民社会融为一体,接受市民的布施,为他们做法事、看护坟墓。现存最早的珠三角地契即是两位佛教徒向光孝寺捐献土地的记录,记载了一位名叫郑八娘的广州女子与她丈夫一同购买近百亩田地捐给光孝寺的故事\footnote{科大卫:《皇帝与祖宗:华南的国家与宗族》,页76—77}。此外,由于元廷视阿拉伯人、波斯人为高于南人的“色目人”,穆斯林在当时的南粤颇为强势,许多来自中亚、北非的穆斯林皆沿着历史悠久的航海路线来到南粤定居、旅行,广州蕃坊变得更为繁荣。1347年,世界史上著名的摩洛哥旅行家伊本·巴图塔来到广州,记录下了他所看到的景象:

\begin{quote}
(广州)世界大城市中之一也。市场优美,为世界各大城所不能及。其间最大者,莫过于陶器场。由此,商人转运瓷器至“中国”各省及印度、也门……转运出口至印度诸国,以达吾故乡摩洛哥。此种陶器,真世界最佳者也\footnote{转引自《广东通史》古代上册,页973}。
\end{quote}

巴图塔的记录得到了考古方面的证实。20世纪,埃及首都开罗以南的尼罗河畔出土了一批精美的瓷器,它们被考古学家认出是元治南粤期间的南粤褐釉陶瓷。可见,当时的南粤陶瓷确曾经海路远销北非。此外,巴图塔还记载了一个伟大的事实:当时,元帝国与印度间的海上交通皆操于粤人、闽人之手。粤、闽制造的远洋商船分为大中小三种,其中大者有三至十二张帆,每艘可载千人,内有带厕所的客房。水手更于船上种植花草,极富生活情调。在印度西海岸的港口喀里克脱,巴图塔见到了十三艘闽粤商船,皆由广州、泉州制造。由是可知,我们的祖先在那时早已对驾船驶往印度轻车熟路。他们航海技术先进、资本雄厚,已经垄断了东亚与印度间的航线\footnote{转引自《广东通史》古代上册,页972—973}。事实上,早在南宋治粤时期,粤人、闽人装备着指南针的商船便时常横跨印度洋,直抵东非的甘眉(今莫桑比克东北科摩罗群岛)。许多商人还跨过苏伊士地峡进入地中海,前往北非的默加腊(今北非摩洛哥一带)贸易,部分商人甚至曾到达过意大利的西西里岛、西班牙南部海岸及非洲西北角。这表明早在13世纪,粤、闽商船便已具备从欧亚大陆的东端抵达西端的能力\footnote{曲金良主编:《中国海洋文化史长编 宋元卷》,页302}。这样看来,粤闽商船垄断元印航路,实为题中应有之意。我们祖先在海上横贯欧亚大陆的气魄,着实令人神往不已。要知道,直到20世纪初,清帝国的部分顽固官僚仍不相信世界上存在欧洲国家,曾可笑地声称“西班有牙,葡萄有牙,牙而成国,史所未闻,籍所未载,荒诞不经,无过于此”。与南粤相比,岭北帝国可谓是世界文明中彻头彻尾的外围。读史至此,我们不能不再一次掩卷肃立,对七百多年前那些横贯欧亚大陆的无名南粤英雄致以最伟大的敬意。 

此外,值得注意的是,元帝国治下的海南岛亦成为了伊斯兰教盛行之地。当时,迁居海南的穆斯林有两种,一为经贸易路线而来的阿拉伯人,一为占婆人。1281年,潮州屠城的刽子手唆都率军由海路入侵占婆,兵败被杀。大批占婆人为躲避战乱,纷纷东渡海南岛定居。在海南岛西部的儋州、南部的万州、崖州,许多穆斯林社区出现了。至于海南岛南端的城市三亚,更发展为一座穆斯林城市。后来,这些穆斯林中的大部分都融入了南粤文明,改信佛教,但仍保留着不吃猪肉的习惯\footnote{贺喜:《亦神亦祖:粤西南信仰构建的社会史》,页57}。少数人则一直保持对伊斯兰教的信仰,成为今日海南岛上的“回辉人”。他们都是当之无愧的南粤人。

由于海外贸易发达,当时南粤能够一直与世界的中心保持紧密联系,掌握了许多项领先于东亚其他区域的科学技术,南粤纺织业对吴越即是最为突出的例子。早在宋帝国统治南粤时,我们的祖先便已掌握了先进的纺织技术,能够以木棉织出名为“吉贝”的优质布匹。海南岛的布匹尤为精良,远销东南亚的爪哇、加里曼丹等地。13世纪后期,海南女子黄道婆发明了一种纺车。史载,此车的特点为:

\begin{quote}

以足助手,一引三纱。错布为织,粲如文绮\footnote{包世臣:《安吴四种》卷29《齐民四术》}。
\end{quote}

由此记载可知,黄道婆发明的是可同时转动三个纱锭的纺车。那时,吴越的织工尚只能使用单锭手摇纺车,效率仅为黄道婆纺车的三分之一。1295—1296年间,黄道婆乘海船来到吴越松江以东五十余里的乌泥泾,将纺车技术传授给当地人,使当地纺织技术大为提高,变得名动天下,从业者达数千人之多\footnote{陶宗仪:《南村辍耕录》卷24。黄道婆之籍贯有两说。一说其为吴越乌泥泾人,少时流落海南发明纺车,后回乡传授技术。一说其为海南人。今从后说。}。14世纪以后,松江发展成为东亚大陆的纺织重镇,有“衣被天下”之美称。这一切成就,无疑要归于技术输出者黄道婆。她是南粤的詹姆士·哈格里夫斯,不但革新了南粤的纺织技术,亦引发了吴越的纺织业革命,为粤吴友谊写下了又一段佳话。

对于技术先进、海外贸易发达的南粤,元帝国自然视之为利薮,进行敲骨吸髓的盘剥。元帝国延续唐、宋帝国之制,于广州设市舶使司,对南粤的海外贸易进行制度化的盘剥。此外,元帝国还在广州组织“官船”贸易,命令海商为官府牟利,抽取其收入的十分之七。更令人发指的是,由于担心粤商的资本过于雄厚,进而“勾结”海外势力反元,元帝国屡下海禁之令。1285、1286年,元世祖忽必烈两度暂禁“商贾航海”。1303年,元成宗又发布“禁商下海”之令。九年后,元仁宗在重申海禁的同时,宣布允许进行“官船”贸易。直到1323年,元英宗才终于解除海禁,“听海商贸易,归征其税”\footnote{《广东通史》古代上册,页983}。在此期间,南粤的海外贸易几乎完全被“官船”垄断。海商若想出海贸易,只得将自己的船只变为“官船”,承受巨额盘剥,向元帝国缴纳百分之七十的利润。在元帝国治下,南粤的海外贸易虽仍保持着强劲的势头,但无疑遭到了沉重打击。元帝国的此种行径,可谓罪恶滔天。然而,仅仅在元帝国解除海禁后约二十年,南粤海商便已垄断了东亚与印度间的航路。我们伟大祖先的聪慧、坚韧、勤劳,着实令人惊叹!

公元1333年,元顺帝登基,元帝国走向了末日。当时,元帝国虽然解除了海禁,但仍不断提高赋税,其税额较元初已达二十余倍。在南粤,元帝国为了敛财而向无赖出售官职,制造了一批狐假虎威、横行乡里的粤奸。他们掠人财物、淫人妻女,恶贯满盈。在残酷的压榨下,南粤打响了东亚诸邦元末抗争的第一枪。1337年正月,增城人朱光卿发动起义,建国号“大金”,改元“赤符”。四月,惠州人聂秀卿、谭景山率众打造军器,奉戴甲为“定光佛”,起兵响应朱光卿。七月,两支起义军皆被元江西行省残酷镇压\footnote{毕沅:《续资治通鉴》卷207《元纪二十五》。定光佛信仰乃佛教一派,该派信众相信“世且乱,定光佛再出世”。参见俞樾:《茶香室丛钞》卷13},但反抗的烽火已经点燃,元帝国已然时日无多,一场岭北帝国崩溃后必然出现的大洪水亦即将出现。公元1351年,红巾流寇起于黄河流域。在“莫道石人一只眼,挑动黄河天下反”的谶语声中,大洪水迅速吞没了岭北。南粤,又一次站在了关键的十字路口上。


