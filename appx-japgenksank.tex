

\chapter{死去的南粤骑士·何真}

何真(1321—1388)是南粤历史上罕见的悲剧人物。他人生的前后部分截然两分,如同六百年后的傅作义。在前半生,他如一名骑士般奋勇战斗,保卫着南粤的父老与河山;在后半生,他竟被迫成为明帝国控制南粤的工具,在朱元璋的驱赶下亲手一点点地毁掉了自己曾拼命守护的一切。在短暂的机会窗口期,他未能作出正确抉择,洪武社会主义因而迅速吞没了他自己,也吞没了他曾经守护的南粤。

何真的家族是元末东莞大户,与当地官员颇有交游。早年,他曾任河源县务官,后被命令前往淡水管理盐政。其时,大洪水的浪头已清晰可见,红巾流寇扫荡中原,珠三角出现了“聚兵抗官”的海寇邵宗愚。他明智地选择了拒绝赴任,回到自己在东莞的庄园静候世变。

至正十五年(1355),元末大洪水在南粤爆发。东莞境内的局势如同日本战国中期的尾张,冒出了十余家拥兵数百、上千不等的豪族。大乱之中,何真亦散财召集乡兵,成为割据一方的势力。他先后依附于强族文氏、郑氏,又先后与两者决裂,曾一度率部逃往惠州境内。至正二十一年(1361),叛将黄常作乱于惠州,杀官踞城,大施暴虐。抓住机会的何真率兵奔袭惠州,在当地父老的迎接下逐走黄常,复击退前来攻城的流贼,稳定了惠州的局势,元廷乃授其惠阳路判官、广东宣慰司都元帅之职。两年后,海寇邵宗愚攻破广州,焚掠不已。已获得官军名分的何真率其子弟举兵北上,击退邵宗愚,“号令严肃,广人大悦。”不久后,盘踞赣州的陈友谅旧将熊天瑞率舟师数万南下攻粤。何真在雷雨中迎战于胥江,大破其众。惠州、广州百姓的生命,南粤境内的安定,皆因何真的义举而被保护。

其后三年内,何真着手统一东莞。至正二十六年(1366),“东莞战国史”上的“关原合战”在该县境内的茶园营打响。何真率大军包围大豪族王成的城寨,并开出“钞十千”的价码征募擒拿王成的勇士。其后发生的戏剧性事件,充分展示了南粤土豪朴素的道德观与政治德性:

\begin{quote}
成奴缚成以出,真予之钞,命具汤镬,趋烹奴,号于众曰:“奴叛主者视此。”
\end{quote}

王成败死,岭南大震。何真奄有东莞、惠州之地,于次年(1367)平定粤东、入主广州。至此,他已成为无可争议的南粤之主。“时中原大乱,南北阻绝,真练兵据险,保障一隅。”短暂的机会窗口期出现在何真面前:究竟是效赵佗故事称帝岭表,还是继续输诚于北廷?他选择了后者,并杀死了劝他称帝的臣子。次年(1368),朱元璋在南京称帝,展开北伐元廷、南征两广的战役。何真大概没有认识到元廷与明廷的不同,正如困守北平的傅作义以为跟谁走都差不多。既然向元廷输诚便能保境安民,那么向明廷投降以换取南粤的安定,又有何不可?从这一刻起,他的命运被锁定了:他不但丧失了成为赵佗的机会,甚至丧失了像陈友定那样荣耀地殉节的机会。短暂窗口期中错误的决断使他丧失了一切翻盘的机会,也使南粤不可避免地落入了洪武社会主义的魔掌。

此后,何真的人生变得乏善可陈。身为南粤土豪的他被剥离了乡土,成为朱元璋改造南粤社会的工具。在洪武元年至二十一年(1368—1388)间,朱元璋令他在山东、山西、浙江、湖广等地为官,并在洪武十四年征伐云南的战役中命他与他的儿子何贵“规画粮饷,开拓道路”。此外,朱元璋一次次地命令他利用曾经的威望召集广东旧部、土豪迁往北方。在利用他挤干南粤这颗甜美的橘子之前,朱元璋并不会对他下手。两人间的关系,像极了一场变态至极的SM。在此,特将史籍中何真召集旧部、土豪的记载抄录于下(关于人数的数字一律写作阿拉伯数字):

\begin{quote}
洪武四年辛亥:命何真还广东,收集旧将士。还京,复归山东。\\
洪武五年壬子六月:参政何真收集广东所部旧卒3560人,发青州卫守御。\\
洪武六年癸丑六月:真招还广州旧所部兵士20777人,并家属送京师还朝。\\
洪武十六年癸亥:是年,何真致仕(退休)。复命真及真子贵往广东收集土豪10623人还朝。\\
洪武十七年甲子闰十月:致仕布政使何真复招集广东旧所部兵3423人还京师。
\end{quote}

并非所有何真旧部及南粤土豪都愿意跟随何真北去。他们中有的人选择了反抗。洪武十四年(1381),东莞人苏友兴起兵反抗。明南雄候赵庸统军讨平,“克寨十二,擒贼万余人,斩首二千级”,将“降贼”尽数发往泗州。第二年,广东境内爆发更大规模的反抗,赵庸则展开了更血腥的屠杀,“凡获贼党一万七千八百五十一人,贼属一万六千余,斩首八千八百级。”短短两年内,明廷通过杀戮万人的方式镇压了南粤的反抗。而在一年后,已致仕的何真便受命入粤招集土豪北迁。不知其时的他,究竟怀着一种怎样的心境踏入他曾拼死守护的乡土,又有何颜面与家乡父老相对?洪武四年,何真首次返乡招集旧部时,曾在梅关拜谒夫人庙、赋诗一首。经由该诗,或可模拟他痛苦的心境:

\begin{quote}
奉命重过庾岭梅,伤心马迹旧苍苔。\\
山僧唤起烹茶急,父老惊传策杖来。\\
古树阴森张相庙,飞烟远杂粤王台。\\
自知桃李为春令,全仗东风巧剪裁。
\end{quote}

面对惊惶的家乡父老,他唯有黯然伤心、执行皇命。那个曾戎马倥偬、保卫乡土的南粤骑士,已经死了。

洪武二十年,朱元璋封何真为东莞伯,赐第于南京。次年,何真病逝,葬礼备极荣宠,其子何荣袭封东莞伯。二十六年,朱元璋将何荣罗织入“蓝党”,族诛何氏。乡居东莞的何真之弟何迪终于聚众发动了迟到的反抗,在击毙三百多名官军后被俘送京师杀死。何氏族人承认谋反的口供颇为详实,其可信度大致相当于布哈林在大清洗中承认自己是帝国主义间谍。二十七年,朱元璋对社会凝结核已损失惨重的南粤发动社会改造,大批百姓被编入军籍,承担起沉重的军役,南粤出现了“民籍者鲜矣”的局面。洪武三十一年(1398),朱元璋病死,建文帝朱允炆继位,大赦天下。此时,何氏幸存的族人,仅剩何真第五子何崇祖一支而已。作为帝国的顺民,他们终于能够活下去了。

\section*{附录:何真之子何荣对明廷的口供原文}

(摘自钱谦益《国初群雄事略》卷14,中华书局1982年点校本)

\begin{quote}
何荣招云:

荣,广东惠州府归善县人,洪武二十五年九月钦差往山西抽丁,本家置酒,与何贵、何宏弟闲话。

弟何贵说:“大哥想李太师(李善长)、延安侯(唐胜宗)众人,为交结胡丞相(胡惟庸),都结果了。我每(通“们”)老官人(何真)在时,也曾去交结。如今胡党不绝,只怕不饶我这一家儿。”

荣说:“我心里也常为这事烦恼,又没躲避处。由他,看久后何如?”

次日,起程前去。不期弟何贵在家,怕前事发露,又与龙虎卫指挥法古私通蓝玉谋逆,伏诛了当。

又招云:

二十六年二月十八日,东平侯韩勋到镇朔卫,在荣下处吃酒。酒间,密说:“前日宣宁侯(曹泰)使人来说,凉国公(蓝玉)收拾四川人马,与陈义指挥等商议摆布,要下京来做一手,著我这里听候接应。如今全宁(孙恪)、会宁(张温)、宣宁、怀远(曹兴)等侯、刘真都督比先都是胡党,已商量接应他,你心里如何?”

荣回说:“我先父亦曾交结胡丞相,因见延安侯众人废了,我与弟何贵常常烦恼,久后不知何如?既官人每都从了时,我也和你每做。”

酒毕,各散在家,一向与本官潜谋听候。不期奸党败露,今蒙提问罪犯。

\end{quote}

执经生注:除何氏一族外,何荣口供提到的蓝玉、曹泰、孙恪、张温、曹兴皆在蓝党案中被杀。

\chapter{南粤土豪的胜利:1449·佛山血战}

\section*{1449年前的明帝国与广东}

正统十四年(1449)在明朝史上是个至为关键的年份,有人以之为明朝由盛而衰的时间节点。是年八月,明英宗率领的二十万大军在土木堡被蒙古瓦剌部全歼,英宗本人被俘,是为震惊朝野的“土木之变”。十月,瓦剌军围攻北京,明兵部尚书于谦指挥军民奋战三昼夜,方将瓦剌军击退。在帝国心脏地带发生如此剧烈的波动时,远在南粤的事情相对而言似乎显得“无足轻重”。在大一统史观的教育下,许多对土木之变耳熟能详的广东人根本不知道当时的广东也在进行一场惨烈的战争——黄萧养之乱。这场战争历时十个月,波及几乎整个珠三角地区,导致了数万人的死亡。在这场战争之后,珠三角的社会结构发生了深刻的变化。

为什么会爆发这样一场战争呢?在讨论黄萧养之乱前,我们先看看当时明帝国治下的珠三角社会有着怎样的形态。元末大乱时,珠三角各地呈土豪割据混战的局势,混战的胜利者为东莞人何真。洪武元年(1368),何真以广东全境归顺明朝,后被朱元璋封为东莞伯。洪武二十一年(1388),何真去世,葬礼备极荣宠。但仅仅五年后,朱元璋便把何真的两个儿子打成所谓的“蓝党”,将何氏满门抄斩,并大加株连。广东土豪因此次清洗损失惨重,帝国的触角向基层插入得更深了。

事实上,此次对广东土豪的清洗只是朱元璋社会改造计划的冰山一角。此种社会改造计划,笔者命名为“洪武社会主义”,其要点为:(1)以亚述式大流放破坏原有共同体的完整性。(2)以里甲人民公社严密控制社会基层的生产与流动。(3)以军、民、匠、灶的世袭身份划分万民,建立生产建设兵团和手工业、盐业国企。(4)通过打击逆臣集团开展政治运动,打击土豪,造成“百姓中产之家大抵皆破”的局面。(5)在社会原子化的前提下开展全民学《大诰》运动,给予众费拉抓捕胥吏的“大民主”权力,发动农民阶级文化大革命。在洪武社会主义下,珠三角的大批百姓被编入里甲,丧失了迁徙自由,被迫承担帝国的赋役。然而在佛山,情况却显得颇为不同。

\section*{土豪自治:1449年前的佛山}

在珠三角,甚至在整个明帝国,佛山都是一座非常特殊的城市:它并非府治或县治,而是一座自发形成的市镇。佛山位于广州西南16公里处,沿西、北江南下、东下的船只必先经过佛山,方可入珠江、进广州。早在宋代,佛山即因优厚的地理位置发展成商业聚落。到元代,佛山已是一片“骈肩累迹,里巷壅塞”的热闹景象。至元末,佛山本地已产生霍氏、冼氏、陈氏、梁氏、李氏、伦氏、赵氏等豪族。

洪武三年,明帝国开始编户籍、立里甲,佛山亦难逃此劫。佛山在行政区划上被定为广州府南海县季华乡,其居民884户居民共被编成8个图(图即里的别称)、80个甲。然而,明帝国并未过度撼动佛山原有的社会秩序,当地土豪往往举族进入同一甲乃至同一里,里甲、乡村的头面人物“乡老”、“乡判”亦由土豪耆老担任。可以说,在洪武社会主义的冲击下,佛山土豪社会仅仅换上了一套里甲的外衣,其原有社会结构并未大变。

由于佛山并非府州县治所,因此没有地方官,城市事务由土豪耆老在城中的“祖庙”聚会商议。“祖庙”是一座同时祭祀道教神袛真武和佛教神袛观音的庙宇,又被称为“北帝庙”、“龙翥祠”,乃佛山的祭祀中心。元末战乱时,曾有海贼逼近佛山,“乡人祷于神”,瞬间风雨大作,贼船倾覆过半。后来海贼贿赂守庙僧人以秽物污染祖庙神像,终于攻破佛山,纵火焚毁祖庙。由此,可见祖庙在佛山人心目中是与保境安民联系在一起的。

洪武五年,佛山乡老赵仲修重建祖庙。宣德四年(1429)、正统元年(1436),乡老梁文慧、乡判霍佛儿又两次扩建祖庙,凿莲花池,植菠萝、梧桐树以壮其观瞻。佛山土豪之所以有修建祖庙的财力,与其对冶铁业的经营关系颇深。自永乐年间起,佛山即形成了大批冶铁点,生产铁锅、农具、钟鼎和军器。佛山铁锅远销吴越、湘赣,每年都有外省客商携数十万两巨资来佛贸易。此外,佛山生产的火器亦性能良好、精准度高——在后来的佛山血战中,这些火器发挥了巨大作用。有明一朝,冼氏、霍氏、李氏、陈氏皆为冶铁大户,拥有庞大产业。明末有人称“佛山地接省会,向来二三巨族为愚民率,其货利惟铸铁而已”。此即明代佛山的真实写照。

\section*{小洪水来临:黄萧养之乱爆发}

洪武社会主义体制试图对全社会进行极权主义控制。此种体制在20世纪尚且难以维持,遑论14—15世纪。明朝开国不到一百年,该体制便走向崩溃。正统—正德间,脱离里甲人民公社的流寇遍地蜂起,较著名者有广东黄萧养之乱、福建邓茂七之乱、郧阳刘千斤李胡子之乱、华北刘六刘七之乱。其中,黄萧养之乱爆发于正统十三年(1448)九月,比土木之变早十一个月。

黄萧养是广州府南海县冲鹤堡农民,曾因犯盗贼罪被广州官府抓入监狱。正统十三年九月,黄萧养在狱外同党接应下与数百名囚犯同时越狱,逃至城外后大肆招揽流民,短短一个月内即纠集了上万人,随之自称“顺民天王”,年号“东阳”,发动“农民起义”。贼军凡经过一地,皆胁迫百姓随之造反,不从者皆惨遭贼军屠杀。至正统十四年(1449),东莞、新会、高要等地皆落入贼军之手。是年六月,黄萧养率十余万人、船千余艘围攻省城广州,并声言欲攻佛山。紧邻广州的佛山危在旦夕。

\section*{保卫佛山与妻小:土豪与民兵的行动}

听闻贼军欲攻佛山,佛山土豪二十二人(人称“二十二老”)群聚祖庙商议对策,决定组织民兵抵抗到底。二十二人推举他们中的冶铁大户冼灏通为“乡长”指挥防御。其余二十一人中,最年长的梁广(74岁)平日“处事公平,乡里信服”,在城中威望颇高,另外二十人亦都是“家颇富饶”的“大家巨室”。佛山是一座自发形成的市镇,并无城墙,但流过城北的汾水和环绕城西、南、东的佛山涌为佛山提供了天然护城河。二十二老倾尽家财,组织佛山居民“树木栅,浚沟堑”,日夜打造冷热兵器,迅速建起了一道周长十余里的环形木栅。同时,土豪们各“聚其乡人子弟,自相团结”,编成一个个名为“铺”的战斗单位,“沿栅置铺,凡三十有五。每铺立长一人,统三百余众。”由此推算,佛山民兵仅有一万余人,兵力远远少于贼军。

正统十四年八月,黄萧养屡攻广州不下,人困马乏,“闻富户多聚于佛山,欲掠之”,乃遣部将彭文俊率数万人攻打佛山。真正的危机到来了。听闻贼军逼近,冼灏通召集所有父老子弟于祖庙誓师,发表了激动人心的演讲:

灏通不才,谬辱上命,为若辈保妻子。念今日之事,国事也。分以死图报,不顾私家矣!若辈宜协心力以保厥家,有异心者杀无赦,战阵无勇者杀无赦!人各食其粮,卒有急,灏通愿罄储共食,若辈恭命无忽!

随后,二十二老一齐在真武像前宣誓:

苟有临敌退缩,怀二心者,神必殛之!

观冼灏通的演讲,虽有论及国事之处,但更多的则是号召佛山民众保卫他们的家人妻小。对有共同体中的民众而言,保卫家乡、保卫家人、保卫共同体远远比保卫一个帝国重要。冼灏通的演讲因之产生奇效,“军声大振,士气百倍”。

不久,贼军乘数百艘船至,意图破城之后大肆掳掠。佛山血战开始了。

\section*{为家园而战:持续半年的激烈攻防}

贼军包围佛山后,开始了不分昼夜的四面环攻。佛山民兵各铺“首尾联络,互相应援”,依托木栅展开了壮烈的防守战。二十二老多与他们的家人子弟一起站立在木栅之后,一同作战。冼灏通组织锻造了大批可发射大如碗的石弹的大火铳,安放于木栅各处,给贼军造成了惨重的伤亡;其次子冼靖则率乡族子弟熔铁为液,供守军泼向贼军。梁裔坚(二十二老之一)率其诸弟“悉以家货供乡兵食”,其年仅十八岁的末弟梁颛“状貌雄伟”,曾率乡人开栅出击,“持丈二红刃刺贼先锋,大呼陷阵”,击退贼军的猛攻。梁敬亲(二十二老之一)“与诸义士树栅拒之,谋定而后战,扼亢捣虚,所向必克”。中秋之夜,贼军本欲发动夜袭,然梁俊浩(二十二老之一)早已令各铺少年举彩旗、鸣金鼓游行,并燃放大爆竹。贼军见之,以外城中有备,遂不敢攻。每次贼军攻城前,二十二老必聚于祖庙并祷于神,神许出战则开栅出战,神不许则在栅后防守。作为佛山社会的凝结核,祖庙和二十二老通过这种方式加强了民兵的信心和士气。

贼军连日攻城,除遗尸累累外一无所获,遂大造云梯,展开新的攻势。因云梯笨重,持云梯的贼兵往往受阻于栅前壕沟,民兵遂投火炬焚毁云梯。贼军主将彭文俊又集中兵力猛攻木栅南侧的栅下一带。防守该地的霍佛儿(二十二老之一)、霍仲儒、冼光(二十二老之一)率民兵极力防战,战况极为惨烈。霍氏一族父子兄弟并肩作战,并“撤屋为栅,浚田为涌”以加固防御工事,霍仲儒亦殉于阵中。激战中,冼光“开栅门出战”,阵斩贼军主将彭文俊。贼兵大怒,攻城益急,结果被民兵用“飞枪巨铳”击退。

无计可施的贼军派使者李某入城劝降,结果被冼灏通三子冼易拔剑斩杀。贼兵只得“退兵二里许,联舟为营”,开始长期围困佛山,意图坐等佛山粮尽、不攻自破。围困期间,幸亏城中“大家巨室藏蓄颇厚,各出粮饷资给”,全城兵民“皆饱食无虑”,士气高昂。贼兵曾“有自恃勇悍、翘足向栅谩骂者”,被民兵以火铳一发击毙——佛山兵民惊奇之余,多认为此乃真武保佑。

\section*{土豪的胜利:解围与佛山共同体的确立}

漫长的围困战持续到次年(景泰元年,1450)二月。在历时半年的防御战中,佛山民兵依靠土豪的领导、自产的武器、对真武神的信仰、大户提供的粮饷和保卫家园的热情一直保持着高昂的士气,使人数占绝对优势的贼军无法踏进他们的家园一步。当月,明都督同知董兴率江西、两广军对贼军展开进攻,双方在广州沙面、黄沙、洲咀头、芳村之间的江面展开决战,贼军大败,阵亡一万余人,黄萧养亦中箭身亡。听闻黄萧养的死讯,围困佛山的贼军一夕溃散。官军随之进驻佛山,以冼靖(二十二老之一)为“乡义”,命其协助官军攻剿贼军余部。

在曾被贼军控制的乡村地区,明朝官军对曾被裹挟入贼的百姓展开了报复性的大屠杀。例如,在龙江镇,由于贼军曾编写过胁从者名单,官军得到名单后即按名单上的“姓名乡里”展开不分男女老幼的挨户搜杀,“遂滥及不辜,并乡之民,多横罹锋镝者”。曾惨遭贼军蹂躏的珠三角百姓,至此又横遭官军屠戮。在随军进剿的过程中,冼靖甄别“良莠”、不问胁从,阻止了官军的大量屠杀行为,“存活者百数千人”,充分体现了佛山土豪的政治德性。

在持续半年的战斗中,佛山民兵杀伤数千贼军,击毙贼将彭文俊、梁升、李观奴,生擒张嘉积。更重要的是,他们以区区万余民兵拖住了数万贼军,极大减轻了广州的压力,为官军对黄萧养的最后一击的成功创造了条件。景泰三年,明廷叙功,明代宗敕封佛山为“忠义乡”,敕封祖庙为“灵应祠”,并命广州官员春秋致祭;佛山二十二老亦获赐“忠义士”称号。

1449—1450年的佛山之战是佛山城市史上的重大事件。佛山之战告诉我们,拥有凝结核的共同体在费拉大潮的进攻下是何等地坚固、何等地不可战胜;为保卫家园、保卫共同体而战的平民是何等英勇、何等顽强。在本地土豪的率领下,各铺在战斗中的互相应援打破了战前各族相对隔阂的局面,佛山居民第一次以一个整体共同作战。为防御而建立的栅栏则明确了共同体的边界,将佛山城市居民与周围乡民分隔开来。真武在一次次战斗中的不断“显灵”则加强了居民对祖庙的认同,土豪议事的祖庙成为了佛山坚实的政治、信仰凝结核。正因为佛山土豪和民兵的奋战,后世的佛山方有机会成长为“天下四大镇”之一。整个明清时期,保持着土豪自治传统的佛山一直屹立在珠三角上,成为庞大的费拉帝国中一抹罕见的亮色。

最后,笔者在此列出领导佛山百姓奋起战斗的二十二老的名字,以供我们缅怀:

梁广、梁懋善、霍伯仓、梁厚积、霍佛儿、伦逸森、梁浚浩、冼灏通、梁存庆、何焘凯、冼胜禄、梁敬亲、梁裔坚、伦逸安、谭熙、梁裔诚、梁颛、梁彝頫、冼光、何文鉴、霍宗礼、陈靖。

热爱自由的人们、南粤的公民们,若有机会去佛山的话,请一定要拜访一次祖庙。在那里,为自由和共同体而战的佛山土豪曾率领居民发出过保卫家园的呐喊。请记住他们的名字,不要让他们的英勇与牺牲被历史遗忘。


\chapter{佛山自治城市简史}

在15到20世纪早期,佛山是一座自治城市,其逼格不输日本战国时代的堺港。1927年,佛山的自治机构被国民党领导的群众运动推翻,佛山自此转入黑暗。

与传统桂枝大城市由政治中心发展而来不同,佛山是一座自发形成于南宋的商业城镇。明初编定里甲时,仅将当地大族集体划入里甲中,并未过度破坏原有社会秩序。该地大族多从事冶铁业,积累了财富与社会威望。

1449-1450年抗击黄萧养流寇的佛山血战,是佛山自治城市形成的第一个关键历史节点。此战,佛山本地有威望的土豪22人(二十二老)在城市的祭祀中心祖庙组织民兵、指挥战斗,取得胜利。战后,明廷封二十二老为“忠义士”,并敕封祖庙为“灵应祠”,承认了佛山土豪聚集于祖庙议事的权力结构。

16世纪的广东宗族化运动是佛山自治史上的第二个关键节点。随着霍氏、冼氏等本地士大夫获得大量功名,他们在佛山本地积累财富、建设宗族、扩张势力,代替了远在广州的官府的权威,成为佛山社会、经济、教育、祭祀活动的组织者,及佛山本土利益的保护者。1553年,岭南大水,佛山周边发生饥荒,数千人聚众于城中劫掠。乡居官员冼桂奇通过富户组织民兵护送外地粮食入境、赈济灾民,仅捕获一人即平定乱事。

17世纪前期乡居高官李待问建设市政议会组织“嘉会堂”,是佛山自治史上的第三个节点。其时,佛山已成为冶铁大镇,居住于城北的大族及聚居于城南的工匠矛盾尖锐,常发生冲突。1620年代,李待问依靠本人威望,设立常设民兵组织“忠义营”、打击苛待工匠的冶铁大族、重修祖庙、兴建学校、请求官府免除佛山的大量赋税,并设置了地方豪右“处理乡事”、管理城市公益款项的机构“嘉会堂”。佛山土豪的议会政治,由此进入制度化时期。

清初,清兵虽屠广州,但未屠佛山。至18世纪中期的乾隆年间,佛山发展为人口超过50万的巨镇,冶铁、陶瓷、纺织业发达,形成了城北商业区、中部大族聚集区、南部工匠区三个区域。各行业皆形成行业会馆,每个行会都分为作坊主组织的东家行、工匠组织的西家行,协调双方利益。大批来自省内南海、新会、香山、顺德、高要,及部分来自山西、陕西的侨寓商贾形成了侨寓集团,与 佛山土著大族多有冲突,要求在城市事务上取得更大的政治权力。随着佛山城市复杂化,拥有更大仲裁能力的议会组织——大魁堂,随之出现。

1738年,新的市议会“大魁堂”组织的出现,是佛山自治史上的第四个节点。大魁堂中的负责人被称为“值事”,由全镇士绅公举产生。充任值事者,有商人、乡居官员、生员、耆老。全镇数年(似不定期)公举大魁堂值事一次,值事不得连任。凡遇大事,大魁堂值事传“阖镇绅士”公议。大魁堂的职能有:设置并管理佛山义仓以应对饥荒设置义冢、收养弃婴、照料老人、定期疏浚对商业至关重要的河道、仲裁商业纠纷、管理学校、与官府交涉。

1757年,侨寓商贾和土著大族对祖庙祭祀权的争夺,成为佛山自治史上的第五个节点。祖庙祭祀自1449年佛山之战后,即由土著垄断。侨寓商贾亦欲分享祭祀权,双方冲突激烈。最后经南海县官府裁定,祖庙祭祀应由全镇“绅耆士庶”一同进行。至此,祖庙祭祀事宜收归大魁堂。此后,土著与侨寓因一同分享着祖庙祭祀、及对祖庙和佛山本地的认同,日渐融合为一个新共同体。乾隆之后,土侨纠纷官司明显减少。

在19世纪,大魁堂运作良好,并具有与官府对抗的能力。例如,1784年,有与官府勾结的两名商人欲在佛山栅下河旁建硝厂。若此厂建成,势必严重污染佛山城市用水。大魁堂值事区宏绪、劳潼上控南海县衙,反遭吏员刁难,遂发动全镇绅民向两广总督府申诉,终于引起两广总督孙士毅干预,从而阻止了硝厂的建造。又如1824年佛山饥荒,义仓赈济贫民。时有人谣传士绅侵吞义仓款项,官府遂趁机将义仓收归官有,引发镇民暴乱,砸毁清政府设于佛山的巡检司(仅三十名象征性驻兵)。其后,在大魁堂值事冼沂交涉下,官府不得不交出义仓。

20世纪20年代革命起,大魁堂的生存受到威胁。1924年广东商团起义,佛山土豪亦随之起兵反抗国民党孙文暴政,惨遭镇压。1927年2月,国民党组织群众运动,借口大魁堂贪污义仓款项(和1824年的清政府一样的理由),逮捕了大魁堂的士绅们。此后,蓝党党化兴,佛山的600年自治走到了它的历史尽头。
