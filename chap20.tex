\chapter{从回光返照到深渊:1926—1949年的南粤}

\section{癫狂革命中的屠宰场:1926—1931年}

\indent 1926年2月中旬,随着邓本殷兵败出走,南粤完全落入国共两党的魔掌中。国共两党立即加紧整编军队,将南粤变为向东亚大陆输出革命的基地。3月,广州国民政府正式收编新桂系,将桂军改编为国民革命军第七军,以李宗仁为军长、黄绍竑为党代表、白崇禧为参谋长。5月1日,国民政府以第四军叶挺独立团及第七军为先遣队,拉开国民党“北伐战争”序幕。独立团系国民党军中唯一由中共直接控制的军队,俱为亡命之徒,战斗力强悍,于当月底独立团侵占湘南汝城。6月4日,国民党中执委通过“北伐”决议,于次日任命蒋中正为国民革命军总司令。7月9日,国民革命军5万余人在广州东校场举行誓师大会,蒋在会上正式就任总司令。在苏联顾问的指挥下,国民党军进展迅速,于11月陷长沙、22日陷岳阳,侵占全湘,将湖湘的自治彻底扼杀。随后,国民党军于10月10日攻陷汉口,消灭直系吴佩孚军主力2万余人,两湖从此陷入“国民革命”的狂潮中。另一方面,国民党军一部又侵占闽越、侵入江右,于11月上旬攻陷九江、南昌,赣、闽两邦亦告沦陷。1927年初,国民党军自鄂、赣、闽三路入侵吴越,进攻占据长江下游的直系孙传芳部,于2月18日陷杭州。3月21日,周恩来等中共骨干发动上海暴动,经两天一夜战斗占领市区。国民党军随后开入上海,又3于24日攻陷南京。至此,吴越也沦为牺牲品。

在沦陷各邦,国共两党展开癫狂的“国民革命”,肆意践踏国际条约体系、撕裂各民族社会。早在1926年1月,两党组织的农团军、工代会便以贪污义仓款项为名将佛山大魁堂的议绅逮捕,佛山长达299年的自治市历史宣告终结。1927年1月5日,中共骨干李立三、刘少奇煽动30万暴民侵占汉口英租界,将其收入国民政府治下。次日,两党又以同样方式侵占了九江英租界。3月24日,在共产国际策划下,国民党军又大举袭击南京的外国侨民,造成英、美、法、日、意侨6人死亡,招致英美军舰开炮报复。在湖湘,由中共操纵的农会大举出动,游斗、杀戮“土豪劣绅”,早已背叛自己乡邦的毛泽东则大呼“好得很”。至此,粤、闽、赣、吴、湘、鄂可以说已无一片宁静的寸土了。

就在这时,国共两党间的矛盾已日渐明显。共产国际是比国民党远更激进的组织,其无产阶级专政的马列主义理念不能为国民党全盘接受。“北伐战争”前,双方便已开始明争暗斗。1925年11月23日,邹鲁、戴季陶、谢持等国民党右派大佬在北京西山碧云寺集会,自行决定开除中共党员的国民党籍,反对国民党赤化,而后于上海成立国民党中央党部,是为“西山会议派”。1926年3月20日,深恐苏联顾问与党内左派大佬汪兆铭合谋夺权的蒋中正以涉嫌暴动为名,在广州下令逮捕“中山”舰(原“永丰”舰)舰长李之龙等中共党员50余人,是为“中山舰事件”。“北伐”开始后,中共农会的激烈手段亦引起许多国民党人的强烈不满。1927年4月6日,控制北京政府的张作霖以军警突袭苏联使馆,逮捕中共骨干李大钊,搜获大量文件,使苏联赤化远东的阴谋昭然于世。其中一封苏联政府寄其使馆武官的训令,值得全文引用:

\begin{quote}
兹特附送国际共产党执行委员会全体大会通过之关于中国问题议决案,并将根据该项议决案所拟定之训令寄发,仰即遵照办理可也。①现时应全力注意增长中国革命运动之国民性质,为达到此种目的起见,必须以国民党为中国国民独立党,而为有利国民党之宣传,应扩大利用汉口各种事件,及英国对于各该事件之态度,资为证据。第一可以证明国民党国民工作之进步。第二可以证明欧洲各国对于中国革命战鬪力之显然的薄弱。②必须于张作霖军队所占领之地域内,造成排欧之混乱。③破坏张作霖之威信,宣传张氏为国际间各资本主义及帝国主义妨害中国国民党自由工作之受雇者。④激动反抗欧洲暴行之风潮及英国计划(以下被焚)。⑤必须设定一切方法,激动国民群众,排斥外国人。为达到此种目的起见,必须设法获得各国对于国民群众之适用武力战斗。为引起各国之干涉,应贯彻到底,不惜任何方法,甚至抢掠及多数惨杀,亦可实行。遇有与欧洲军队冲突事件发生时,更应利用此种机会,实行激动。⑥现时应暂缓实行共产党纲,因此时实行可使张作霖之地位巩固,并加重国民党之分裂。吾人已向鲍罗廷严重训令,暂时停止对于资本阶级之过激手段。张作霖失败以前,应抱定自己之宗旨,即在国民党内,暂行保留国民之各种阶级,资本阶级亦应保留。⑦实行此种排斥欧人之运动时,保存各国间之不协调,非常重要。日本能于最短期间派多数军队来华,故令日本与各国隔离,尤为特别重要。为达到此种目的起见,于一切运动之中,必须严加监视,务使日本侨民无被害之人。但于激动排外风潮之时,将日本除外,殊足以引起不愉快之观感。故实行激动排外风潮时,必须假托反对不列颠(英国)运动之名义也。本件抄本,迅速分送各分部及指导人员(以下被焚)\footnote{转引自刘仲敬:《民国纪事本末·革命编年史》}。
\end{quote}

另有七份文件事关苏联在远东的间谍组织。对于苏联间谍在南粤的活动,文件如是记录:


\begin{quote}

国民党方面,设广州分机关部,管辖广州、汕头、梧州、云南等分团。分团之下,于重要地点,设密探员及递信员,而核编各分机关之报告文件,则为北京中央密探总部\footnote{转引自刘仲敬:《民国纪事本末·革命编年史》}。
\end{quote}

以上文件中,苏联赤化南粤、赤化远东的缜密计划暴露无遗,足以使人不寒而栗,深刻认识到共产国际的本质。共产国际一向视国民党为白手套。在利用国民党将其革命发展到一定程度之前,它并不会和国民党翻脸。但到1927年4月,随着中共组织在远东迅速扩散、工会组织控制上海,斯大林认为抛弃国民党的时机已经成熟。4月6日,斯大林在莫斯科发表讲话,对党的积极分子如是说:


\begin{quote}

我们要充分利用他们(国民党),就像挤柠檬汁那样,挤干以后再扔掉\footnote{转引自刘仲敬:《大棋局》(八)}。
\end{quote}


在此关头,蒋中正先一步动手,于4月12日在上海发动“清党”,处决300余名中共党员、逮捕上千人,中共称之为“四一二反革命政变”。同日,留守广西的新桂系第四旅旅长遵从黄绍竑的指示,在广西发动清党\footnote{《广西通史》第三卷,页93}。14日,留守广州的国民政府广东省主席李济深遵从蒋中正命令,召开秘密会议,下令清党。当夜,国民党军开始在广州戒严。15日凌晨2时,大批国民党军警统一出动,在广州逮捕2000余名中共党员和国民党左派,是为“四一五事件”。接着,佛山、江门、海口、肇庆、雷州、韶关、梅州等重要城市亦开始实行“清党”。至17日,广州解除戒严,全广东共有5000余人被捕、2100余人被杀,国共两党在南粤的合作从此破裂\footnote{《广东通史》现代上册,页324—325}。

蒋中正开始清党后,汪兆铭一度继续与共产国际合作,在武汉另立国民政府,向河南进攻,是为国民党史上的“宁汉分裂”。至7月15日,汪兆铭亦发动清党,宁汉双方政治立场遂趋向一致,开始连同上海的西山会议派进行三方谈判。由于汪兆铭坚持反蒋立场,蒋中正乃以退为进,于8月13日宣布下野,退居家乡宁波奉化。9月20日,经艰难谈判,三方总算达成统一,于南京共组国民政府,是为“宁汉合流”。

在南粤,桂人李济深及新桂系因积极清党被国民党委以重任。5月20日,南京方面以李济深、黄绍竑、李宗仁等十六人为广州政治分会委员。6月14日,南京方面又将粤、桂、闽、滇四省政务、军事交给广州政治分会。8月1日,李济深出任广州政治分会主席。10日,李又被蒋中正任命为第八路军总指挥。这样,李济深已隐然为国民党内割据南粤的一方藩镇\footnote{《广东通史》现代上册,页629—630}。

1927年8月1日,在“宁汉合流”前夕,忠于汪兆铭的粤籍将领张发奎(韶关人)正率其第二方面军驻扎南昌一带。是日,该方面军一部2万余人在中共骨干周恩来、朱德、叶挺、贺龙、刘伯承策动下发动兵变,中共视之为其建军之始,称之“南昌起义”。兵变发生后,张发奎连忙逃至香港,命其麾下的第四军军长黄琪翔(梅州人)代为指挥。共军起事后,计划侵入南粤,首先攻占东江地区、争取苏援,而后夺下广州。3日,共军撤离南昌南下,然因粤籍将领蔡廷锴(罗定人)率第十师脱队入闽,其兵力骤降至1.3万。见共军南下,李济深急调部队布防于揭阳、潮州一带。9月,共军经闽西长汀、上杭沿汀江、韩江南下进入南粤,于18日陷大埔,随后于20日在此分兵,由朱德率3000人留守大埔、周恩来、贺龙、叶挺率主力进攻潮汕。23日,共军陷潮州、汕头,潮汕已完全落入中共之手。李济深忙命第三十二军军长钱大钧(吴越吴县人)率四个师共2万人进攻三河坝、黄绍竑率桂军经丰顺进攻潮汕,粤籍师长陈济棠(钦州人)、薛岳(韶关人)、徐景唐(东莞人)分率第十一师、新四师、第十三师共1.6万人组成“东路军”由河源东进、寻找共军主力决战。东路军官兵多为粤人,具有守护乡土的高昂热情。他们虽是国民党军,但不愿南粤被更为疯狂的中共控制,遂奋勇应敌。28日,双方首次交火,共军主力6500余人在揭阳县山湖击溃东路军一部,随后进犯至汾水村,与东路军主力展开鏖战。至30日,共军主力伤亡近半,无力再战,乃向揭阳方向逃亡。四日后,他们又在普宁新安的莲花山遭东路军截击,大部溃散。10月1日,钱大钧部开始以优势兵力进攻三河坝。至3日下午,三河坝共军已三面被围,朱德只得率部突围。4日凌晨,钱大钧部攻占三河坝,全歼殿后共军一个营。此战,国民党军损失1000余人,共军伤亡1800余人。5日,两支共军的幸存者在饶平县境内碰头。此后,他们在朱德的率领下流窜于闽粤湘赣边境,最后仅剩约800人,于1928年4月到达赣南井冈山,与毛泽东率领的湘赣边境“秋收起义”部队会合,组成著名的“中国工农红军第四军”\footnote{关于南昌兵变及其后续战斗,相关研究论述极多,在此不赘注}。

汾水村、三河坝战役是南粤史上的重要战事之一。这次战役将初生的共军逐出南粤,粉碎了他们通过海路获得苏援的计划,从而将南粤的完全赤化推迟了22年。若无陈济棠、薛岳等粤籍将士的英勇战斗,此战不可能获胜。陈济棠等人虽然忠于国民党,但他们的奋战的确守护了南粤的安宁。对他们的这一历史功绩,今天的我们绝不能遗忘。

共军刚被击退,广州便发生惊变。“宁汉合流”后,因南京政务被西山会议派把持,汪兆铭深感不满,于9月13日通电下野。9月下旬,身在香港的张发奎以追击共军为名,命黄琪翔率第二方面军4万余人沿赣江南下,进抵粤北南雄、韶关,这支部队的官兵多为南粤人。当时,李济深的部队正在粤东迎战共军,根本无力防守广州,张军遂得以顺利入驻广州、惠州、石龙。共军败走后,李济深忙命部队回防。10月10日,两军在广九铁路爆发冲突,李军一个营被缴械。24日,张军一个师将驻惠州的李军第十八师全部缴械,枪杀师长胡谦\footnote{《广东通史》近代上册,页639}。这样,李、张两军已濒临全面开战。

与此同时,为架空李济深,张发奎在香港通电支持汪兆铭、否认南京中央的合法性。10月11日,张发奎又强迫李济深将广州政治会议改组为省政府,以黄琪翔为军事厅厅长,其余要职亦都由张派出任。29日,汪兆铭(三水人,吴越移民后代)抵达广州,准备联张驱李,在他的乡邦另立中央。这时,国民党正准备召一届四中全会预备会议以分配各派利益。在蒋中正的电邀下,汪、李二人于11月15日离粤赴沪参会。17日凌晨3时,黄琪翔遂趁机指挥张军兵变。经六小时战斗,张军完全控制广州,将李军及黄绍竑的桂军几乎全部缴械,黄绍竑本人化装逃走,桂军残部沿广九、粤汉铁路逃亡,李派将领薛岳则向张军倒戈,是为“张黄事变”\footnote{《广东通史》近代上册,页638—639}。事成后,张军贴出布告,称其起兵目的为驱赶“迫走汪主席、挟制李主席”的黄绍竑,乃“正义”的“护党运动”。18日,张发奎在广州出任临时军委主席,广州的党、政、军大权完全落入张、黄之手,国民党陷入“宁粤对立”之局。不久后,黄绍竑经香港、越南逃至广西梧州,立即部署攻粤,将桂军主力集结于梧州。张发奎则于12月初调主力部队沿西江而上,粤桂大战一触即发\footnote{《广东通史》近代上册,页643}。

此时,留守广州市区的张军仅有第四军教导团、警卫团,河南(海珠岛)则由第五军控制。对中共而言,这是个千载难逢的起事良机,因为指挥教导团的第四军参谋长叶剑英(梅州人)正是中共骨干之一,该团1500名官兵有200多人是中共党员。11月28日,两名共产国际代表到达广州,将200余万美元的活动经费交给中共。中共广东省委书记张太雷亦积极活动,秘密组织起一支有2000人的“工人赤卫队”。12月4日,教导团的中共党员召开大会,决定调查各连的“反动分子”。11日凌晨2时30分,张太雷、叶挺等人来到教导团驻地北校场,命叶剑英率部起事。3时30分,教导团、工人赤卫队同时起事,仅用两小时即控制广州大部分市区。由于第四军军部等数个张军的最后据点难以攻下,共军便放火焚烧,造成大批平民死伤。控制全城后,共军随即制造红色恐怖,大举屠杀四月清党时镇压他们的人士。张发奎、黄琪翔则逃往第五军控制的河南,张随后前往沙面英租界,电令城外军队火速回援。12日上午,两军在广州北郊激战,当地的土豪、绅商纷纷组织民团对抗共军。越秀山一度被张军夺回,又被共军抢走。下午,中共操纵暴动兵民在西瓜园操场集会,宣告成立“广州苏维埃政府”。但这时,他们的末日已然降临,张军已开始自南北两路攻击广州市区,以第五军渡珠江北上、以薛岳师攻击越秀山、突破大北门。情急之下,张太雷与共产国际代表德国人纽曼乘车赶赴大北门指挥,于大北直街附近中伏,张当场丧命。不久后,张军从四面攻入市区,双方展开巷战。至当日夜,共军支撑不住,决定弃城。13日凌晨,共军残部1000余人逃出广州,经花县窜至海、陆丰地区。工人赤卫队留下殿后,成了革命的牺牲品。13日下午,张军粉碎赤卫队的抵抗,完全控制广州,随后将苏联驻广州副领事郝史等五人逮捕并游街枪毙。14—19日,张军在城中大举搜杀中共党员及与起事有关的工人、市民,被处决者多达5700余人,其中包括上百名听命于共产国际的朝鲜人。加上此前死于共方屠杀、纵火及两军交火的军民,共有超过两万人在12月11—19日间死于广州,他们大部分都是南粤人\footnote{《广东通史》现代上册,页669—675;《简明广东史》,页687—690}。继西关大屠杀后,广州又一次沦为修罗场。上一次,是国共两党联手镇压南粤土豪起义。这一次,则是被共产国际撕裂的南粤社会陷入了血腥的阶级战争。

在国民党内部,汪兆铭和张发奎属左派。广州暴动发生后,右翼西山会议派和仇张的新桂系便结成同盟,称汪、张二人是纵容中共暴动的“附逆”分子。12月14日,南京国民政府下令将张发奎、黄琪翔解职,以缪培南(五华人)、薛岳为第四军正副军长。17日,汪兆铭见大势已去,不得不在上海发表引退宣言,随后经香港赴法国。18日,张、黄二人将军队交给缪、薛,接着出洋游历,缪、薛则于26日率四个师的张军撤出广州,沿东江撤往赣南,“宁粤对立”就此告终\footnote{《广东通史》现代上册,页646—647}。

不过,报仇心切的李济深没有放过他们。张军一撤离广州,李济深便在南京反汪派的配合下拼凑起东西两路大军,企图在广州附近围歼张军。西路军由黄绍竑指挥,为桂军第十五军共四个师,其中包括由广东人组成的第十三师(师长徐景唐);东路军则为驻闽粤籍将领陈铭枢指挥的第十一军共两个师,分由陈济棠、钱大钧指挥。1928年1月1日,西路的桂军大举开进广州,沿广九铁路经石龙、惠州直取河源,东路军则进抵增城、花县。12日,西路军在紫金以东追上张军,击溃其殿后的一个团,缴获大批辎重弹药。22,张军与桂军主力在五华县城以西的潭下墟展开决战,双方在300米间的近距离内交火,战况空前惨烈。至次日晨,张军全线败北,逃向赣南。28日白天,第十三师在黎咀墟追上张军,与其一个殿后师激战竟日。双方官兵皆是讲粤语的粤人、穿着同样的草青色制服,却在战场上不顾生死地肉搏,仿佛有着不共戴天的仇恨。至夜10时,两军各自伤亡惨重,张军主动撤退,随大部逃入赣南,第十三师亦无力穷追\footnote{《广东通史》现代上册,页648—649}。数以千计的南粤青年,就这样沦为国民党内路线、派系斗争的棋子,在南粤的土地上疯狂地自相残杀,上演了一出荒唐的悲剧!

当国民党各派陷入混战时,中共已在南粤各地发起暴动。1927年4月下旬,中共广东省委组织东江特别委员会,由彭湃等七人领导。自与陈炯明决裂后,彭湃一度逃亡广州投奔孙文。国民党侵占全粤后,他被派回海丰组织农民运动,在当地有深厚的组织资源。清党开始后,粤东各地的中共党员及国民党左派多遭捕杀,仅澄海县城就有数十人被处决,农村地区被杀的农会成员更多,亦有不少农民遭牵连而死。彭湃乃组织起4000号称“赤卫军”的农军,于10月1日发起暴动。在暴动前制定的《革命纲领》中,彭湃杀气腾腾地宣布,他要将二十种人全部杀掉:

\begin{quote}
一、籍国民党者,杀!\\
二、反土地革命者,杀!\\
三、曾任文武官员者,杀!\\
四、曾充民团警兵者,杀!\\
五、曾充反动政府机关差役伙夫者,杀!\\
六、一切土豪地主者,杀!\\
七、讨租讨债者,杀!\\
八、还租还债者,杀!\\
九、藏匿契据者,杀!\\
十、立妾蓄婢者,杀!\\
十一、不服兵役者,杀!\\
十二、当堪舆命卜者,杀!\\
十三、当巫婆、媒婆者,杀!\\
十四、吸鸦片者,杀!\\
十五、惯作盗窃者,杀!\\
十六、盲目者,杀!\\
十七、麻风者,杀!\\
十八、残废者,杀!\\
十九、老朽不能操作者,杀!\\
二十、信仰一切宗教者,杀\footnote{周康燮主编:《1927—1945年国共斗争史料汇辑·第二集》,页24}!
\end{quote}


由于李、张两军正忙于争夺地盘,根本无力镇压赤卫军,彭湃的暴动得以迅速进展。到11月1日,赤卫军攻陷海丰县城。彭湃乃于21日成立“海陆丰工兵苏维埃”,颁布《土地革命法规》,提出“一切土地归农民”,将恐怖的大屠杀撒向整个海陆丰地区。大英雄陈炯明的家乡,就此遭受惨绝人寰的惊天浩劫。在不到一个月的时间内,数以万计的生命消失了,他们在死前多遭到灭绝人性的虐待,更多的人则逃往广州、香港。海陆丰地区的40万人口锐减了超过八分之一,彭湃在旬月之间便创造出波尔布特耗时四年才达成的“伟业”。与他的赤卫军相比,红色高棉简直像纯洁的天使。

彭湃的“人间天堂”并未维持太久。1928年初,重新控制广东的李济深、黄绍竑派出2万余桂军进剿海陆丰。彭湃不敌,于2月29日率残部撤入潮阳、普宁、惠来交界的大南山区,留下了被他彻底毁灭的家乡。10月,彭湃被上海的中共中央调离广东,此后他一直在上海活动,于1929年8月被国民党捕杀。此后,共军残部在大南山区进行游击活动,一度发展至闽南、赣南,拥有数万人的武装。但从1931年起,他们因深陷以“肃反”为名的自相残杀,很快衰落。至1935年底,粤东的中共武装仅剩饶澄县委统辖的数百人。这些人不久后也转移至闽南,只有骨干古大存率17名残部逃入大埔县山区\footnote{《简明广东史》,页693—694}。

在海南岛,中共亦于清党后掀起大暴动。1927年6月,中共广东省委特派员杨善集在乐会宝堆村召开会议,提出“以红色恐怖镇压反革命的白色恐怖”,组建指导暴动的琼崖特别委员会。9月23日,中共策划的“全琼武装暴动”揭幕,大批武装农军攻陷加积外围的椰子寨墟,随后被国民党军逐出,杨善集战死。由于岛上的国民党军仅有800人,根本无力镇压,共军得以迅速扩张。至1928年1月,岛上共军已攻陷临高、儋县、琼山、文昌、文东县城,拥有正规红军1400余人、赤卫队上万人。当年低,因不敌国民党军的重兵围剿,海南共军退入定安县瑞母山,于次年改称“中国工农红军第二独立师”。1932年7月,该师又遭围剿,全师解体,残部在琼崖特委书记冯白驹指挥下潜伏深山进行游击战,直至对日战争在南粤爆发\footnote{《简明广东史》,页697—699}。除粤东、海南外,在1928—1935年间,在盘踞赣南的共军主力支援下,粤北的南雄、仁化等地亦一直有小股共军活动。在粤西合浦县北海的斜阳岛上,也有一股共军一直盘踞到1933年方被剿灭\footnote{《简明广东史》,页700—703}。

1928年初,蒋中正因宿敌汪兆铭下野得以复出,于1月9日在南京复任国民革命军总司令。4月,国民党军发动“二次北伐”,将“国民革命”的狂潮撒向华北。此时,国共虽已分裂,但国民党仍以打倒“帝国主义”为己任,在“革命外交”的口号下疯狂破坏国际条约体系。5月1日,国民党军攻陷济南,随后于3日蓄意蹂躏日侨,虐杀12人。日军出兵报复,于10日将国民党军逐出济南,并四处纵火、滥杀,遭成数千市民死亡。事后,在国民政府的强硬立场下,日方妥协,日军不得不于十个月后撤离山东半岛。6月4日,张作霖放弃北京退出山海关,因其专列在沈阳皇姑屯被炸身亡,策划此次谋杀的真凶究竟系苏联、系日本,至今仍众说纷纭。8日,国民党军开入北京。国民党军北伐期间,三晋、巴蜀、大滇、满洲的阎锡山、龙云、刘湘张学良纷纷升起“青天白日满地红”,宣布“易帜”归顺“中央”。1928年10月,蒋中正在南京出任国民政府主席。至此,国民党便在名义上统一了东亚各邦,蒋中正成为东亚大陆的统治者\footnote{关于二次北伐、济南事件过程,相关论述极多,此不赘注}。

1929年初,蒋中正见大局底定,乃于南京召开编遣会议以统一军政。当时,国民党军分为一、二、三、四集团军,分别为蒋中正的中央军、冯玉祥的西北军、阎锡山的晋军和李宗仁的桂军。在北伐进程中,新桂系控制了经广西至两湖、河南、河北的广大地盘,并有盟友李济深坐镇广东,拥兵20万,成为蒋中正的心腹大患。编遣会议开始后,蒋桂互不相让,陷入僵局。老谋深算的蒋中正遂于3月12日以调停蒋桂关系为名将李济深召至南京,于次日发动突然袭击,将李囚禁于南京汤山。26日,秉承蒋意的国民政府宣布李宗仁、白崇禧、李济深“抗命称兵,谋叛党国”,将三人免职查办。31日,蒋中正下令各部自江右、皖西、豫南总攻桂系,“蒋桂战争”爆发。仅过五日,桂军即全线败北,李宗仁、白崇禧被迫放弃武汉,逃回广西梧州。这时,广东的政局已发生剧变。早在蒋桂关系恶化时,蒋中正便已派遣国民政府文官长古应棻赴粤,以“考察”为名策动粤籍干部反桂。古应棻在广州与广东省主席陈铭枢、江防司令陈策及陈济棠三人频频接触,政界一时流传起“三陈倒李(济深)”的说法。因深恐陈铭枢上位后成为尾大不掉的藩镇,古应棻向蒋中正极力举荐仅为师长的陈济棠出掌兵权,获蒋同意。李济深被囚后,粤籍诸将陷入一片惊愕,有人主张联桂讨蒋。在此关键时刻,陈济棠于3月30日在白鹅潭的军舰上召开记者招待会,宣布自己遵从蒋中正的命令,正式出任“广东编遣区特派员”,同时要求驻广州的三个桂军团立即撤走。桂军不敢强抗,乖乖撤走。次日,陈济棠离舰上岸,在广州城内设特派员办公室,随后出任国民党中央候补执行委员,升任第四军军长、第八集团军司令。至此,他兵不血刃地结束了新桂系对广东的主宰,成为执掌广东军权的人物\footnote{《广东通史》现代上册,页720}。

败回梧州后,李、白、黄三位新桂系大佬决定出兵攻粤,以消灭投蒋的陈济棠,蒋桂战争的战火烧到了南粤,“第三次粤桂战争”爆发。5月4日,蒋中正免去黄绍竑的广西省主席之职,命广东陈济棠、湖南何键组织“讨逆军”分两路攻桂。次日,新桂系倾其主力三个师共十六个团,以李宗仁为总司令、白崇禧为前敌总指挥,自称“护党救国军”大举东进。陈济棠则将其嫡系第十一师分防清远、芦苞等地,以其余部队拱卫广州。桂军入粤后,一路排除粤军节节抵抗,连陷怀集、广宁、四会,而后分兵进攻清远、芦苞。这时,清远守将第一旅旅长余汉谋(高要人)突被人诬告通敌,被陈济棠下令逮至广州。清远守军因长官无端被捕,士气大跌,仅略作抵抗便于14日弃城。桂军得手后,向花县白泥、军田一带进攻,广州危在旦夕\footnote{《广东通史》现代上册,页727}。

关键时刻,陈济棠为稳定军心,便采取“用人不疑”的态度释放余汉谋,命其以参谋长名义督率援军火速反攻。19日,余汉谋率部急行军至白泥前线,与桂军展开惨烈搏杀。粤军虽兵力不占优势,但有守卫乡土的信念,因而寸土不让。至次日凌晨,桂军终于不支,狼狈西逃,经四会、广宁、怀集败回梧州。此时,早已投蒋的新桂系将领俞作柏(北流人)、李明瑞(北流人)亦趁势攻桂,率所部自汉口乘船经上海至广东,而后与粤军溯西江攻击,于6月2日陷梧州。12日,南京政府下令,任俞作柏为广西省主席、李明瑞为广西“编遣特派员”。在粤军海空力量的支援下,俞、李之军节节推进,于18日陷桂平,而后向南宁推进。24日,李宗仁、白崇禧、黄绍竑见败局已定,遂经龙州逃往越南。27日,南宁易手,由俞、李主导的蒋系广西省政府控制全桂\footnote{《广西通史》第三卷,页153}。

俞、李二人虽忠于蒋中正,实为国民党内的激进左派,与中共极为亲近。1927年4月清党后,中共在广西的活动转入地下,于5月成立广西地委。此后两年内,中共相继在武宣、平南、桂平、左右江各县、东兰、凤山等地掀起农民暴动,广西各处都活跃着小股共军\footnote{《广西通史》第三卷,页97—107}。俞、李系表兄弟,而俞作柏的胞弟俞作豫正是中共党员。通过这层关系,他们得以和中共进行密切的沟通。俞、李主桂后立即释放了在清党中被捕的中共党员和左派分子,并在各地组织工会、农协,开展工农运动。7月,因俞、李主动请求中共方面派干部到广西“协助工作”,中共广东省委派出以中央代表邓小平为首的40多人潜入广西,在俞、李政府中出任要职。8月,俞、李公然向东兰、凤山一带的共军运送武器。到9月,俞、李更将一批中共党员和左派青年派往左右江地区担任县长和农协干部。在俞、李的胡作非为下,广西几成赤色世界,李宗仁曾如是评价二人的行为:

\begin{quote}
为虎附翼,共祸始炽,桂省几成为共产党之西南根据地\footnote{转移自《广西通史》第三卷,页155}。
\end{quote}

俞、李二人一面勾结中共,一面策划反蒋。很快,他们的机会就来了。自被赶出广东后,张发奎即投靠蒋中正,其第四军残部被蒋改变为整编第四师。蒋桂战争中,他积极作战,率部占领鄂西宜昌。1929年9月,蒋中正突然下令张部移防陇海路,张发奎认为蒋欲消灭异己,遂于当月17日在宜昌喊出“护党救国”的反蒋口号,宣布欢迎汪兆铭主政。27日,俞、李宣布广西“独立”,响应张发奎反蒋。10月1日,二人在南宁举行誓师大会,随即率军沿西江而下,向“附蒋”的陈济棠发起进攻,“第四次粤桂战争”爆发。红色恐怖的阴云,似乎马上就要吞没整个南粤了。

然而,出乎二人意料的事很快发生。梧州、柳州守将吕焕炎、杨腾辉早已被蒋买通。二人刚离南宁,吕焕炎便在梧州宣布拥蒋。10月4日,杨腾辉亦在柳州宣布拥蒋。与此同时,陈济棠派出三个师的兵力沿西江而上,准备进攻南宁。四面楚歌之下,俞、李二人及俞作豫等中共要员只得于13日放弃南宁,率少数警卫部队逃往龙州。俞作柏旋即逃往越南,李明瑞则在中共劝说下留守龙州,不久后成为中共党员。这时,南宁的3000余名守军早已脱离二人控制,被邓小平及中共广西前委书记陈豪人、常委张云逸带往桂西,与东兰、凤山的1000余名农军会合,一同开往百色,准备将这支部队改编为共军。但就在起事前夕,邓小平忽于11月初逃往龙州与李明瑞会合,抛下了这支部队。仓促之间,陈豪人、张云逸仍坚持原计划,于12月11日在百色升起红旗,将部队改称“红七军”。1930年2月1日,在邓小平、俞作豫、李明瑞的策划下,退守龙州的约2000名败军亦升起红旗,改称“红八军”,李明瑞以中共党员的身份出任红七军、红八军总指挥\footnote{《广西通史》第三卷,页154—156}。就这样,中共通过狡诈的手段得到了五千正规军。

俞、李失败,广西陷入群龙无首的局面。桂军官兵纷纷思念李、白、黄,希望三人尽快返桂主政。与此同时,汪兆铭响应张发奎的拥戴,自法国回到香港,开始着手组织联合新桂系、张发奎的反蒋同盟。在此情形下,黄绍竑于11月上旬经香港返桂,与桂军诸将会于宾阳,决定与张发奎联兵攻粤。下旬,李宗人、白崇禧自越南河内返桂,与黄绍竑会于南宁。这样,仅仅过了半年,广西便重回新桂系之手。三人当即打出“护党救国军”的旗号,由李出任总司令、黄任广西省主席、白任前敌总指挥。12月初,张发奎率军突破蒋军层层阻截,进抵桂北上岳。12月6日,桂张联军集结5万兵力东进,向广东发起进攻\footnote{《广西通史》第三卷,页158}。

蒋中正对桂张攻粤早有准备。开战前,他已派朱绍良率国民党中央军三个师入粤增援。自取得广东军权后,陈济棠积极扩军,将所部编为五个师,以余汉谋、蔡廷锴、蒋光鼐、香翰屏、李扬敬分任师长,使粤军兵力高达5万人。这样,陈济棠便有八个师约8万人的兵力应敌,兵力远多于桂张联军\footnote{《广东通史》现代上册,页723}。联军跨过粤桂边境后,以桂军为左路、张军为右路沿西江进攻。陈济棠则以余汉谋、香翰屏两师为左翼防御军田、白泥,其余部队为右翼防御花县东北的两龙一带,构筑了坚固的阵地。10日,两军全线交火,战线长达40公里。陈济棠本想以优势兵力以逸待劳地轻松击溃敌军,但因张军战斗力特别强悍,陈军竟支撑不住,节节败退,张军前锋已逼近广州近郊的人和墟。12日,陈济棠出动空军猛烈轰炸张军阵地。张军死伤惨重,陷入困境。同时,桂军亦受阻于军田,伤亡惨重。14日,联军终于支撑不住,向西经四会、广宁、怀集退却。19日,粤军抢先攻占梧州,截断了联军的退路,联军只得绕道退往平乐、荔浦一带休整。经此惨败,联军士气低落,遂撤销“护党救国军”称号。1930年3月,粤军自梧州出发,在空军掩护下攻占藤县、北流。此后,粤桂两军于桂北形成对峙,双方小冲突不断\footnote{《广西通史》第三卷,页158;《广东通史》现代上册,页724—725}。

正当蒋桂激斗不已时,新桂系迎来了新盟友。因编遣军队的谈判破裂,蒋中正与冯玉祥、阎锡山之间的战争已一触即发。1930年2月,阎锡山致电蒋中正,要求双方共同下野,此后一个月,两人以电报往来争吵,时人称之“电报战”。3月15日,57名桂、晋、西北军将领齐聚太原,推阎锡山为“中华民国陆海空军总司令”,李宗仁、冯玉祥、张学良为副总司令,组成空前庞大的反蒋同盟。4月1日,阎、冯、李三人分别在太原、潼关、桂平宣誓就职,然张学良未表态。5日,南京国民政府下令通缉阎锡山\footnote{关于“中原大战”进程,相关研究甚多,此不赘注}。规模空前的“中原大战”,就此爆发。

根据反蒋联盟的作战计划,桂张联军应沿粤汉铁路北上,与冯玉祥的西北军南北对进,夹攻湖北蒋军,会师武汉,然后连同晋军沿陇海线攻占济南、徐州,进而拿下南京。李、白、黄、张四人因此野心膨胀,产生逐鹿天下的野心,决定放弃广西,以全军入湘作战。5月中旬,桂张联军主力集中于桂北,在李、白、张的率领下分三路入湘,于5月18日陷永州、28日陷衡阳、6月4日陷长沙、8日陷岳阳。何键的湘军难以抵挡,退守湘西,朱绍良的中央军则由粤入湘试图阻击,战败北逃。这时,西北军、晋军已分别攻陷许昌、济南,三军会师中原的目标即将达成\footnote{《广西通史》第三卷,页161}。

就在桂张联军准备进攻武汉时,局势骤然转变。自重返广西后,黄绍竑便率军开赴桂西与共军战斗。1930年3月20日,红八军因不敌桂军,遂按邓小平的指示放弃龙州,向红七军靠拢。俞作豫则前往香港寻找党组织汇报,在深圳关口被捕,后被处决于广州。至10月,红八军残部仅剩100多人,终于在凌云县找到红七军主力,被编入红七军。至11月,红七军也坚持不住,遂留下小股部队在原地游击,主力则向赣南的中央苏区转移。至1931年7月,他们终于九死一生地抵达赣南,正好遇上大肃反,李明瑞遭处决,高级干部亦多被整肃。广西境内成建制的共军活动,从此不复存在。1930年6月中旬,当桂张联军准备进攻武汉时,黄绍竑部本该按计划作为后续部队入湘参战、进驻衡阳。但由于他正与共军缠斗,无法抽身北上,遂贻误了战机。与此同时,何键自湘西发起反攻,一举夺回长沙,将入湘联军拦腰截为两段。陈济棠亦趁机派粤军进入湘南,抢占衡阳。因缺少后续部队增援,李宗仁只好无奈地下令退兵。19日,联军退至安仁、茶陵、耒阳一带,决定退守湘江西岸的宝庆、祁阳、永州等地,并夺回衡阳,待秋凉之后再图进取。6月29日,桂张联军大举进攻衡阳,与粤军蔡廷锴、蒋光鼐、李扬敬三师激战于城西回龙寺、弹子山一带。因天气暑热,师老兵疲的联军非战斗减员极多、士气低下,被严阵以待的粤军打得伤亡惨重。7月1日,联军全线崩溃。粤军紧追不舍,于4日攻占祁阳、永州,将联军全部赶回广西\footnote{《广西通史》第三卷,页162}。反蒋联盟三军会师中原的计划,由此落空。

联军惨败后,忠于蒋中正的粤、湘、滇军已对广西形成三面包围之势。6月,龙云命滇军名将卢汉率三个师趁虚攻桂,“第二次桂滇战争”爆发。7月19日,滇军包围南宁,守将韦云淞率少量部队死守不退,双方陷入僵持。在此危局下,黄绍竑、张发奎均心灰意冷。8月21日,黄宣布辞去广西省主席职务,向蒋中正呼吁和谈。李、白二人虽欲与蒋决一死战,但受困于局势,只得默许。9月,李、白、张三人在柳州制定作战计划,决定以一部坚守桂林、柳州,防备粤、湘,并尽全力南下迎战滇军,解南宁之围。9月28日,白崇禧、张发奎率联军主力离开柳州,于10月13日进抵南宁郊外,随后对围城滇军展开迅猛反击。已经苦守近三个月的韦云淞亦挥师出击,与主力内外夹击围城之敌。经整日激战,滇军不敌,向右江溃退,南宁之围终告解除。联军紧追不舍,于23日在百色以东的平马镇赶上滇军主力。经五日血战,滇军再告战败,逃回滇境\footnote{《广西通史》第三卷,页163}。入侵广西的外敌总算被逐出,新桂系取得第二次桂滇战争的胜利。当时,粤军余汉谋师已进占广西宾阳,见滇军败走,遂撤往宾阳。至此,经过一系列复杂的交战,历时年余的第四次粤桂战争亦落下帷幕。战争结束后,黄绍竑随李宗仁来到南宁,新桂系三巨头重聚一堂。然而,这是三人的最后一次共事了。因李、白极力坚持反蒋,黄绍竑无法说服他们,只得于12月2日悄然离开自己的乡邦,经香港前往南京投蒋。曾经亲若兄弟、两次击退滇系侵略军、为八桂英勇奋战的李、白、黄铁三角,从此永远失去了一角\footnote{《广西通史》第三卷,页164}。

桂张联军败退后,蒋中正得以全力对付冯、阎。1930年8月,西北军、晋军在河南相继惨败,转入守势。9月18日,满洲的张学良突然通电拥蒋,于两日后率30万奉军入关参战。10月,奉军占领河北,晋军退回三晋,蒋军则攻下潼关,进入陕西。11月4日,冯玉祥、阎锡山在绝望中通电下野,向蒋屈服。规模空前的“中原大战”,就此以蒋中正、张学良的完胜告终。此次战争,蒋奉联军出动60万兵力,伤亡9.5万余人;李冯阎联军出动80万兵力,损失20万以上。至于因战事流离失所的东亚各邦百姓,更是不计其数。自清帝国崩溃以来,东亚大陆还从未有过如此规模的战争。在“国民革命”的癫狂口号下,南粤和东亚各邦已化为巨型屠宰场。

“中原大战”结束后,蒋中正志得意满,开始大踏步地走上独裁僭主之路。1930年11月12日,国民党三届四中全会在南京召开。蒋中正提议召开御用机构“国民会议”,以之“推举”自己为总统,遭党内元老胡汉民的激烈反对。蒋中正恼羞成怒,于1931年2月28日以晚宴议事为名卑鄙地诱捕胡汉民,将其囚于南京汤山。正在广州休养的古应棻系胡汉民好友,决定不惜一切代价“反蒋救胡”,遂通电辞去国民政府文官长之职。胡汉民系南粤人,曾于民初任广东都督,在粤人中有一定声望。3月4日,广州、香港报刊纷纷报导胡汉民被囚消息,致使广州群情激奋、气氛紧张,分掌广东军政大权的陈济棠、陈铭枢不得不做出抉择。早在第四次粤桂战争结束时,为架空陈铭枢,陈济棠已将陈铭枢老部下蒋光鼐、蔡廷锴的两个师调往中原津浦路前线支援蒋中正。此时,两师已被编为第十九路军,正在赣南剿共。陈铭枢虽为粤人,却是亲蒋派。手无兵权的他感到广州反蒋空气日浓,遂于4月初避走江右。在国民党各派的历次反蒋战争中,陈济棠虽一直站在蒋中正一边,却绝非蒋的走狗。他与蒋的联盟,更似南粤史上许多本土政权与岭北强权的机会主义联盟,真正目的是保住南粤的自由。对陈济棠的想法,老奸巨猾的蒋中正并非没有察觉。在“胡案”发生前一个月,蒋曾要求陈裁减军队和军费。因陈济棠不从,蒋便在3月初令陈率粤军入赣剿共,欲利用共军消耗粤军。陈济棠明白,决断的时刻到了,他可以利用反蒋空气为南粤争取更大的自主权。他虽曾与桂军多次交手,但粤桂毕竟是血浓于水的亲邦,新桂系又是反蒋重镇。4月22日,陈济棠遣使至南宁,向李宗仁、白崇禧提出冰释前嫌、一致反蒋。25日,桂方使节回访广州,与陈济棠洽谈合作事宜。30日,在粤的古应棻等四名国民党中央监察委员联名通电弹劾蒋中正,指责蒋“扣押胡汉民,躬身毁法”、纵容“贵戚”大法横财、剿共不利。5月初,陈济棠令粤军撤出广西,将梧州、桂平交还新桂系。3日,以陈济棠为首的粤军海、陆、空三军将领联名通电,严厉声讨蒋中正的独裁野心:

\begin{quote}
(蒋中正)掊克聚敛,诛锄异己,直至毁灭我革命军之总体,以遂其一人之私\footnote{转引自钟卓安:《陈济棠》,页77}!
\end{quote}

粤军诸将的反蒋通电虽仍运用“国民革命”的话语体系,但其对蒋中正欺压粤人、肆行僭越独裁的行径可谓揭露无遗,足以使人拍手称快。其后,国民党内的左右两派反蒋大佬汪兆铭、邹鲁及孙文之子孙科、元老许崇智、唐绍仪齐聚广州,谋求另组国民政府,白崇禧、张发奎亦赶到广州参与此事。5月26日,汪兆铭、孙科、张发奎、白崇禧等联名发出最后通牒,要求蒋中正在四十八小时内下野。28日,因通牒时间已过,他们遂成立广州国民政府,以孙科、汪兆铭、古应棻、唐绍仪、许崇智为常委,国民党的第二次“宁粤对立”形成。6月2日,广州国民政府任命陈济棠为第一集团军司令,下辖余汉谋、香翰屏、李扬敬的第一、二、三军,李宗仁为第四集团军司令。第一、四集团军分别由粤、桂军编成,各辖150、72个团,前者还有粤军的海、空军配合。粤桂将领随后议定攻蒋计划,决定以粤军第二、三军分别留守东江、北江,其余部队北上进攻湘南衡阳。9月1日,广州国民政府下达入湘动员令。蒋中正则亲赴江右督战,命中央军顾祝同等部入湘。13日,粤桂联军共5万人侵入湘南,与顾祝同部展开激战,于次日攻陷衡阳\footnote{《广东通史》现代上册,页731—732}。就在联军准备继续北进时,影响世界历史走向惊变发生了:1931年9月18日,日本关东军制造满洲事变,向沈阳发起进攻。不到半年,日军即占领满洲,于1932年3月1日拥立前清宣统帝溥仪成立满洲国。对标榜“国民革命”、致力于打倒“帝国主义”、维持大一统的国民党各派来说,满洲事变都是生死存亡的危机。因此,宁粤双方立即停战,于10月27日在上海召“和平会议”。至11月7日,经七次谈判,决定宁方释放胡汉民,双方各在南京、广州召开国民党四大,选出中央执监委员,然后在南京合并举行四届一中全会,改组国民政府,并取消广州国民政府。12、18日,南京、广州相继召开四大。获释的胡汉民专程赶赴广州主持大会,向南京方面提出以蒋中正下野为宁粤合流的前提。在广州四大召开同日,因古应棻病死,胡汉民遂成为粤方名义上的领袖,但南粤的实际控制权完全掌握在拥有军队的陈济棠和新桂系手中。12月15日,蒋中正以退为进,宣布下野,退居家乡奉化遥控南京政局。22日,宁粤代表齐集南京召开四届一中全会,成立以孙科、汪兆铭为首的南京国民政府,宣告国民党统一。至此,因宁粤合流,国民党大佬纷纷离粤北上就职,南粤已无人能挑战陈济棠和新桂系的地位。31日,陈济棠、李宗仁等通电取消广州国民政府,另设 “国民政府西南政务委员会”及国民党中执委“西南执行部”于广州,并称“西南两机关”,分别以唐绍仪、陈济棠、李宗仁、邓泽如、萧佛成等五人与陈济棠、李宗仁、白崇禧、刘纪文、李扬敬等五人常务委员,尊胡汉民为“领袖”\footnote{《广东通史》现代上册,页733—736}。

上述人员中,胡、唐为国民党元老,邓、萧、刘为国民党中央监察委员,皆无实权,唯有陈、李、白等粤桂军将领各自实际统治着粤、桂。这样一来,粤人陈济棠和桂人李宗仁、白崇禧便联合在一起完全取得粤、桂的控制权,成为名义上服从南京,实则有极大自主性的国民党内藩镇。也就是说,经过一连串痛苦的战争与复杂的博弈,南粤终于争取到了来之不易的自立!可是,这种自立毕竟不是真正的独立,陈济棠和新桂系只能打着“国民革命”的旗号维持它。在蒋中正贪婪的目光下,南粤脆弱的自立究竟能维持多久?南粤的前路,一时扑朔迷离。


\section{最后的自立:1931—1936年}


\indent “西南两机关”成立后,南京方面一度于1932年1月7日决议否认其合法性。同日,以退为进的粤桂方面要员在广州召开会议,宣布西南执行部隶属于国民党南京中央党部、西南政委会隶属于南京国民政府,就近管理粤、桂、黔、闽党政军务,而其能实际控制者仅为由粤、桂两邦构成的南粤之地。南京方面只得妥协,默许“西南两机关”的存在\footnote{《广东通史》现代上册,页747}。

获得半独立状态后,南粤自1911年以来持续不断的战争总算告一段落,南粤人总算能喘一口气、享受一段宁静、和平的生活了。陈济棠和新桂系趁此来之不易的机会,积极投身于建设乡邦、保卫乡邦的事业中。在讲述他们的功绩之前,我们需先对他们的背景进行一番了解。

1890年,陈济棠出生于粤西防城县的一个客家农家。他的家族是世代耕读传家的小土豪,有着淳朴的美德。自六岁起,陈济棠便入读村塾,接受了完整的东亚传统儒家教育。17岁那年,他考入广州黄埔陆军小学,投身清帝国新军,很快在其中加入同盟会,参加了1911年独立战争。此后,他曾跟随孙文、陈炯明参加护国战争和前两次粤桂战争,由连长一路升至团长。陈孙决裂后,他站在孙文的伪军一边,参加了攻击陈炯明、邓本殷的历次战役,于1925年7月出任国民党第四军第十一师师长。“北伐战争”开始后,他随李济深留守南粤,从此开始了他守护南粤的奋斗。他虽曾有过屠杀同胞的罪恶,但他终于在上述的一系列斗争中做出正确选择,使南粤赢得来之不易的半独立地位。由于他一手掌握广东的军政大权,因而被人称为“南天王”。他虽不如陈炯明那般对法统和政治有深刻的理解,却拥有朴实的性格和良好的健全常识。

相形之下,李宗仁和白崇禧的经历则略显复杂。李宗仁系广西桂林人,于1891年出生于临桂西乡村的一个贫农家庭,他的父亲是名教师。白崇禧小李宗仁两岁,他的父亲是桂林城中经营粮油的商人,母亲马氏则出自归化南粤的穆斯林世家。在母亲的影响下,白崇禧虽笃信伊斯兰教,却更认同同盟会的观点,以“汉人”自居。1908年,李、白两人一同考入广西陆军小学,在此与日后的新桂系三号人物黄绍竑(玉林广府人)成为同学。李宗仁更于1910年加入同盟会,成为新军中的革命分子。他们三人都曾参加过1911年独立战争。此后,他们被编入旧桂系的军队,多次立下战功。李、白二人分别升任团长、营长,黄绍竑亦成为基层军官。在第二次粤桂战争中,李宗仁率部向陈炯明投降,被派入十万大山中招降桂军残部。如前所述,李宗仁遂与旧日同窗白崇禧、黄绍竑合作,一手创立了新桂系。黄绍竑出走后,新桂系由李、白两人一手掌控。与陈济棠相比,两人反蒋更为积极,对大一统观念更为认同,思想也更为左倾\footnote{关于陈济棠、李宗仁、白崇禧的早年经历,详见钟卓安:《陈济棠》;覃卫国:《桂系战史》}。

由于南京方面仍对南粤虎视眈眈,南粤必须建立一支足以自卫的强大武装力量。自1931年起,陈济棠便着手扩充其被称为“粤军”的第一集团军。至1934年,粤军兵力已高达15万,拥有大量先进武器。陆军方面,粤军兵种齐全,各师皆有炮兵、工兵、战车兵、高射炮兵通讯兵、医疗兵。为提高陆军战斗力,陈济棠全力更新装备,向捷克、德国大批进口大批机枪、步枪、高射炮、和弹药,使各步兵连都逐步换上了新式步枪,并配有9挺机枪。空军、则是陈济棠最为重视的军种。国民党军在广东本有一支飞行大队。在抵抗新桂系的战争中,陈济棠曾多次指挥这支部队,取得显著战果。在“西南两机关”成立前夕,广东空军操纵于孙科之手。“西南两机关”成立后,陈济棠于1932年4月成立广东空军司令部,任命黄光锐(台山)为空军司令,将广东编入第一集团军。1933年,黄光锐赴美考察航空,发动粤侨捐购飞机。此后两年内,粤空军共购入美式战机38架、教练机20余架。此外,设于广州、韶关的制机厂亦日夜自主研制飞机,造出“复兴号”、“羊城号”轻轰炸机各4架。到1936年,粤空军已有4个飞行大队、1各教导、飞机130余架及一批英、美顾问,成为东亚大陆上实力仅次于蒋中正嫡系空军的空中武装\footnote{《广东通史》现代上册,页740}。

为建立粤海军,陈济棠则颇费了一番周折。1932年4月,陈济棠成立广东海军司令部,然原国民党广东舰队司令陈策拒不合作,反将“中山”等十余艘艘舰艇开至虎门外的唐家湾和伶仃洋示威,并令“飞鹰”、“福安”、“海瑞”、“海强”等舰开至海口,设海军行营,摆出武装对抗姿态。在谈判失败后,陈济棠决定武力进攻。6月26日,陈济棠命黄光锐派空军轰炸虎门外的敌舰,又命陆军渡海进攻海口。因陈策舰队缺乏防空火力,粤陆军得以在空军掩护下顺利渡过琼州海峡。7月5日,粤空军猛烈轰炸海口各军事目标,击沉“飞鹰”舰,“福安”等舰避走香港,陆军随后登陆攻占海口。10日,经蔡廷锴调停,陈济棠、陈策达成协议,决定将海军陆战队及“中山”等三舰交付正在岭北的第十九路军、其余十一舰移交陈济棠的第一集团军。获得舰队后,陈济棠不但又在广州造舰一艘,还在1934—1935年间向英、法、意等国购买5艘鱼雷艇及多艘军舰,于黄埔建立雷舰基地,并派员赴英、意学习鱼雷艇技术,建起一支颇有实力的海军\footnote{《广东通史》现代上册,页742}。

陈济棠虽积极扩军,却绝非只知养兵、不顾民生的军阀。1932年9月14日,陈济棠对广东省政府各机关长官、僚属发表讲话,要求他们为建设“模范新广东”一齐努力。27日,他又向西南政委会提出《广东省三年施政计划提议书》,于次年1月1日获通过。根据该计划,广东应在第一年上半年完成剿共,第一年内整顿各机关积弊、废除有碍民生的苛捐杂税,第二年内筹抵占省库收入五分之二的烟税赌饷,第三年内全面禁烟禁赌、发展省立银行、公开县级地方财政、筹备县、市长选举。自1911年以来,粤人仅在陈炯明解放广东的1920—1923年间享受过一段太平。连年战事使当时的广东积欠债务及军费2.2亿元,每月赤字达70万元\footnote{《广东通史》现代上册,页752—753}。陈济棠的“三年施政计划”可谓一声春雷,使甘霖重降南粤。他曾指出,在财政、民生举步维艰的广东,当务之急乃是:

\begin{quote}
速谋建设,以回复地方之气\footnote{《广东通史》现代上册,页752}。
\end{quote}

为实行“三年施政计划”,陈济棠首先着手整顿吏治,要求各县县长“克尽职守”、各公务员必须做到“清、慎、勤、明”四字。在陈济棠的监督下,“三年施政计划”得以顺利进行。1933年,广州的金融业率先稳定,银号由1929年的447家增至298家,其营业规模超过香港。同年,陈济棠将国民党的“广东中央银行”改组为“广东银行”,于香港、汕头、海口、韶关、北海等地设支行,统一回收原中行发行的旧币,推广新币“粤币”。粤币价值稳定,流通于全广东,稳定了物价与市场秩序。直至1935年冬南京国民政府发行统一货币“法币”,粤币方逐渐历史舞台。此外,陈济棠又积极鼓励粤侨回乡投资。1933—1936年间,粤侨汇入广东的侨汇高达粤币11.57亿粤元,解除了广东的财政危机。在广州太平路和东山,一座座由侨汇建起的旅馆、酒家、戏院、房屋、高楼拔地而起,其中最有名者便是高15层、64米的爱群大厦。该大厦于1934年10月动工、1937年4月落成、存续至今,是当时广州的第一高楼。为发展经济,陈济棠又动用重金聘请数十名有英美等国留学背景的管理人才,令他们分任局长、处长、教授、校长、厂长、经理等职。在他们和全体粤人的努力下,至1936年,广东已形成以广州为中心、辐射全粤的经济格局,广州出现了分别以化学工业、纺织工业为主的西村、河南两大工业区。粤汉铁路沿线出现一批军工企业,以韶关飞机厂最为著名。此外,乐昌、曲江的煤矿和英德的钨矿得到大力开采。在新桂系的配合下,粤桂间的交通枢纽梧州则成长为桂货入粤的集散地和工业城市。在各种工业企业中,以制糖业、水泥业最为发达,纺织、造纸饮料等行业亦蒸蒸日上\footnote{《广东通史》现代上册,页770—771}。

在交通方面,陈济棠也大力建设。在他治下,联络全广东的陆、水、空交通网已初步形成。1936年,粤汉铁路全线通车、海南岛环岛公路干线竣工。是年全广东公路干线总长4574.9公里,有官营长途汽车1278辆。水运方面,陈济棠成立广州港务局、筹建洲咀头码头等内港55座,建成拥有上百艘船只的内河客货船队。至1935年,广东已形成以广州为中心的内河、沿海交通网络,仅江河航线就多达6269条。航空方面,陈济棠与新桂系于1933年6月合作成立西南航空公司,开辟两条民航航线,一条由广州至梧州、南宁、龙州、柳州、桂林等地、另一条由广州至茂名、梅菉、海口等地,每周有一至两个班次\footnote{《广东通史》现代上册,页774}。

在陈济棠时代,广州的市容市貌也焕然一新。1933—1934年间,广州增筑马路14万米,拓宽了数以百计的内街。1933年2月,连接江北城区和河南(海珠)的广州海珠桥完工。该桥采取钢结构、全长356.67米,由美国专家设计,是首座连接广州两岸城区的桥梁,极大改善了广州的交通情况。同年,西堤码头等过江轮渡码头完工,这些码头直到今天仍在造福广州。至1936年,广州已有市内公交线路19条、长途公交线路5条、公交客车532辆,新建的西村火力发电站和增埗水厂则基本满足了全城的电力和自来水需求。此外,中山图书馆、中山纪念堂等延续至今的地标性建筑也在这一时期落成\footnote{《广东通史》现代上册,页776—777}。

同一时期,新桂系对广西的建设也未曾落后。他们一面对滇、黔推行“亲仁善邻居”的友好政策,一面积极投身于本土建设。李宗仁曾特别提出,“苦干、硬干”是广西的出路。1932年4月,李宗仁、白崇禧提出推行“自治、自卫、自给”的“三自政策”,以之为施政纲领。白崇禧如是阐释该政策:

\begin{quote}
自卫是用以抵抗敌人军事的侵略,自治是用以巩固我们下层政治的组织,造成真正民主政治的基础,自给是用以抵抗外来的经济侵略……如果不能自卫,便谈不上自治和自给。反过来说,自治和自给,对于自卫也有影响。自治和自给的能力增进一分,自卫的能力也就加强一分\footnote{转引自《广西通史》第三卷,页213}。
\end{quote}

1934年3月,新桂系颁布《广西建设纲领》,在“三自政策”的基础上提出“四大建设”的口号,即政治、经济、文化、军事建设。其中,军事建设为重中之重。根据“自卫”原则,新桂系采取“寓兵于团”的做法,大量裁撤常备军,以民众组成的民团担负主要国防任务。新桂系将桂军裁至不足3万人,以省下的军费向德、日、英、法等国大量购买先进军火,建立了一支兵少而精锐的常备军。至1935年,新桂系又设南宁航空学校,向英、日购机,开始建设桂空军。在全桂范围内,新桂系设南宁、武鸣、桂平、玉林、梧州、平乐、桂林、柳州、庆远、百色、天保、龙州共十二个民团区,每区设民团指挥部指挥官、副指挥官一名。各县设民团司令部,以县长兼任民团司令,副司令员由民团指挥部委派。县以下的民团组织,则区编为联队、乡(镇)编为大队、村(街)编为中队,由区长、乡(镇)长、村(街)长兼任联队长、大队长、中队长。凡在桂省居住超过两年以上的18至45岁男子均要出任团兵,其中18至30岁者编为甲级队、31至45岁者编为乙级队。团兵又分为常备队、预备队、后备队三种,常备队由各县轮流征调的甲级队编成、预备队由常备队甲级队训练期满者编成、后备队由常备队、后备队之外的所用适龄男丁组成。常备队训练在区民团指挥部进行,每期六个月。预备队由县民团司令部选定地点,每年训练一星期。乡村后备队由县民团司令部派督练官巡回训练,在农闲时间进行,以练满180小时为限。除军事学科、术科外,训练内容还包括公民常识、自治概要、卫生概要、实业常识、识字等。除承担军事义务外,民团还有维持地方治安、筑路、造林、垦荒、举办成人教育等职责。白崇禧对民团训练极其重视,他在数年内几乎走遍全桂各区、县。在他的监督下,至1936年,广西已练成团兵40万,社会风气为之一振。据时人观察,当时广西社会的情形是:

\begin{quote}

到了梧州的街上,第一印象是满街巷都军人化了。在这行走着的人,流着汗的人,买着东西的人,人力车上的人,笑、叫、喊……一切一切的壮男们,除了老弱妇孺的人们外,都穿着灰色的、黑色的服装——军服,都带着灰色或黑色的帽——军帽,而拥挤着、动着、行着\footnote{转引自《广西通史》第三卷,页218}。
\end{quote}

这样,新桂系便在数年内将广西变为全民皆兵的尚武社会。此种社会变革影响深远,极大影响了广西日后的历史走向。在政治建设方面,新桂系以防止蒋中正的分化瓦解为宗旨,强调干部的团结和忠诚。按照规定,广西的所有干部都应进行政治训练,内容为“三自政策”、《广西建设纲领》及地方自治,训练不合格者免职。其中中层以上干部在南宁轮训,县以下乡村干部则设民团干校轮训。1934年11月1日,李宗仁、白崇禧在南宁西乡塘召集秘密会议,宣布成立“中国国民党革命同志会”,两人分任正副会长。该会设本部于南宁,对外绝对保密,入会者需向正副会长宣誓效忠。举凡全桂政治、军事、经济、人事任免问题,都需经该会讨论通过。新桂系由此控制了国民党在广西的党务系统,将其变为桂人的禁脔。直到1937年对日战争爆发,该会才宣布解散\footnote{谭肇毅:《“三自政策”与新桂系的军阀政治》}。

经济建设方面,广西省政府于1933年制定工业发展计划,在南宁、桂柳、柳州、贵县、八步等地建立自来水厂、发电厂。柳州更成为以军工企业著称的城市,有轻重机枪厂、步枪长、迫击炮厂、手榴弹厂。此外,新桂系还对民营工业进行扶植,使电池业、纺织业发展迅速。到1936年,全桂已有大小民营工厂62家。交通亦为新桂系的建设重点。在1931至1937年间,新桂系共建成省道、县道2359公里,使全桂公路增至约5700公里,其中可实际通车者约3300公里,使大多数县都通上汽车,内河航运亦航线亦增至3100公里以上,极大方便了乡村民众的出行。1935年,由迁江至来宾的铁路动工,是为广西史上的首条铁路。在对日战争中,该路工程亦未停止,终于1941年建成通车、全长64.2公里。自1933年6月粤桂双方合办西南航空公司后,新桂系又开通桂黔之间南宁至柳州、鹿寨、独山、贵阳的民航航线,甚至还开通了与法国间的邮件空运。1934年后,自南宁寄出的信件仅用数日便可经越南抵达法国\footnote{《广西通史》第三卷,页241—242}。

教育建设方面,新桂系也取得显著成就。自1929年蒋桂战争爆发以来,广西无年不战,直到1931年才稍获喘息。因战争的破坏,广西大批学校停课、停办。1933年,新桂系开始推行国民基础教育普及运动,要求所有适龄儿童及16至45岁的失学者一律入学,并对苗人、瑶人乡村推行“特种部族教育”。至1938年,全桂基础教育入学者已由1933年的705853人激增至2975650人,超过120万大龄失学者得到了接受基础教育的机会,而当时全桂人口不过是1200万人。此外,中等、高等教育亦有所恢复,以南宁的广西大学最为有名。为发展广西大学,新桂系在1931—1936年间还特别聘请原省长马君武出任校长。马君武提出,学校不但要养成“科学的智识”和“工作的技能”,更要培养学生“战斗的本领”。根据“自卫”原则,广西的大、中学生不但要学习文化知识、进行社会实践,亦要经受严格的军事训练,成长为八桂社会的军事凝结核\footnote{《广西通史》第三卷,页259—264}。

在外交方面,粤桂双方均采取灵活的手段,其核心目的都是为了维持南粤的半独立地位。在广东,陈济棠于1932年3月1日发表《告第一集团军全体官兵书》,表达剿共决心,于1932年重创大南山、海南岛的共军,并派兵深入闽、赣剿共。此外,他又大举清剿广东境内之敌,于1933年11月以海、陆、空联合攻势消灭盘踞斜阳岛多年的共军。1932年秋,在对赣南“中央苏区”的进剿中,蒋中正故意命令粤军与共军进行消耗战,使粤军在粤赣边境的水口损失3000余人。1934年春,蒋中正发动对中央苏区的第五次围剿,强令陈济棠出兵,并将中央军部署于闽西南进行军事威胁。6月,陈济棠命李敬扬率十团兵力出兵赣南寻乌,在空军掩护下击败共军一个师,攻克筠门岭上的天险盘古隘。此后,陈济棠看出蒋消耗粤军的诡计,命部队暂停进攻。这时,共方狡猾地利用陈、蒋矛盾,于10月6日派代表何长工、潘汉年赴寻乌罗塘镇与粤方谈判。7日—9日,双方展开为期三天的密谈,议定两军停战,互通情报、通商。会后,粤军还向共方提供了一批食盐和数百箱弹药。10日,自称“中央红军”的共军主力放弃中央苏区,开始踏上史称“长征”的逃亡之旅。陈济棠于粤赣边境划出一条特别通道,使共军主力顺利通过,并命令前线将领不得主动攻击共军,得从而得到共方不侵犯粤境的保证\footnote{《广东通史》现代上册,页779—784}。

共军主力开始“长征”后,蒋中正急令白崇禧、何键派桂、湘军围堵。桂、湘两军及国民党中央军随后沿湘江布置四道封锁线,其中桂军皆为战意高昂的民团部队,布防于全州、兴安、灌阳一带。11月26日,共军主力约8万人经湘南由永安关、雷口关窜入桂境,以彭德怀的红三军团占领界首以南的光华铺、新圩、马桥渡、林彪的红一军团占领自屏山渡至界首的所有渡口,分别阻击南北对进的桂、湘军,掩护中共中央军委纵队渡湘江西逃。28日,桂、湘两军在飞机掩护下发动猛烈攻势,“湘江战役”爆发。因主力行军缓慢,共军阻击部队迟迟不能下撤,伤亡惨重。至12月2日夜,共军军委纵队终于全部通过渡口,其余各部已折损过半。在桂军的英勇战斗下,共方红五军团第三十四师和红三军团第十八团被阻于湘江东岸,全军覆没。此战过后,共军主力由8万余人锐减至3万余人,只得窜入越城岭苗、瑶地区,于13日完全退出广西,逃入湘南\footnote{《广西通史》第三卷,页292—293}。在湘江战役中,身着灰、黑军装的八桂民兵奋勇作战,成建制地歼灭大批共军,重创共军主力元气大伤,在南粤史上写下了一段光辉的篇章。

至1935年上半年,随着共军主力窜入黔、蜀、滇,蒋中正认为共军已不足为患,遂将工作重点转回颠覆国际秩序,积极对日备战。在蒋中正的命令下,国民党中央军开入黔、蜀,开始进行对日战争的“大后方”建设。1935年1月,国民党中央军的参谋团进驻重庆,并向川军各部派出督察员,成为巴蜀本土军阀刘湘的太上皇。同月,中央军占领重庆,其后强迫夜郎本土军阀王家烈辞职,将黔军全部调往汉口。3月2日,蒋中正飞抵重庆,叫嚣四川是“我们革命的一个重要地方”、是“中华民族”的“立国之本”。4月10日,蒋中正亲赴昆明,龙云向其表示效忠\footnote{《抗战前国民政府如何控制四川》}。10月,共军主力窜至陕北,完成“长征”。此后,蒋中正非但不积极剿共,反而甘当共产国际颠覆世界秩序的工具,于1935年11月提出调整对苏政策,“打通与共产党的关系”,并于年底派陈立夫、张冲赴欧洲与斯大林接触。当时,共产国际指使各国共产党与“社会党乃至民主党派”结成“反法西斯”统一战线。在远东,中共需与国民党联手抗日,为苏联减轻来自满洲方面的压力。1936年秋,周恩来与共产国际代表先后在上海、南京与张冲、陈立夫相继会谈,双方于12月初初步达成共军改组为国民革命军、国共两党“共同抗日”的共识\footnote{郭廷以:《近代中国史纲》,页448—449}。至此,蒋中正及国民党再度完全成为苏联在远东的白手套,无耻地投靠了共产国际。此后张学良于同年12月12日为“逼蒋抗日”发动的“西安事变”,实已无关历史大势。

在南京国民政府投靠苏联的情况下,南粤面临着被绑上苏联战车的巨大危险。在1931—1936年间,陈济棠为反蒋故意做出“抗日”姿态,以此攻击蒋中正消极抗日。1932年1月28日—3月2日,日军与国民党军在上海开战,由蔡廷锴、蒋光鼐指挥的第十九路军积极应战日军,是为“上海事变”。陈济棠虽曾派出7架战机赴上海前线助战,但并未认真抗日。此后,在中共的策动下,蔡廷锴、蒋光鼐与陈铭枢于1933年11月20日以十九路军为主力发动“福建事变”,于福州成立以反蒋抗日为宗旨的“中华共和国人民革命政府”,直至次年1月21日方被国民党中央军消灭,陈济棠自然没有对他们施以援手。自1932年初起,日本军部即不断派代表赴广州与陈济棠接洽。1934年3月22日、1935年2月13日,日本第三舰队两次访粤,陈济棠曾分别与其司令今村信次郎中将、百武源吉中将会谈。1935年3月2—5日,日本奉天特务机关长土肥原贤二到达香港、广州、桂林,分访胡汉民、陈济棠、李宗仁、白崇禧等粤桂要员。1936年2月,松井石根大将又在广州访问陈济棠,向粤方赠送枪枝约1000支、子弹20万发及数门山炮。日本军方与粤桂要员接触目的,是希望策动南粤联日反苏反蒋\footnote{《广东通史》现代上册,页750—751}。然而,陈济棠与新桂系却各有打算。

数年以来,陈济棠为守护南粤的半独立状态可谓殚精竭虑。为此,他认定蒋中正是最危险的敌人,不惜一边标榜抗日、一边同时与中共和日本进行机会主义的合作。李宗仁和白崇禧则更为左倾,头脑中有更多大一统毒素。击退共军主力后,新桂系于1935年冬派代表赴西安、天津与中共建立联系,李、白两人更公开提出与中共合作抗日的主张。另一方面,中共投桃报李,于次年5月31日发表《我党在两广的任务》一文,煽动民众拥护李宗仁的抗日主张。新桂系一面敷衍地与日方要员接触,一面在广西煽动反蒋仇日风潮。1936年5月12日,胡汉民病死。次日,蒋中正认为“解决”南粤半独立状态的条件已成熟,要求陈济棠协助“中央”出兵广西。不久后,前往广州吊丧的孙科等人又告知陈济棠,蒋中正的真实目的乃取消“西南两机关”,彻底控制南粤。在此情况下,忍无可忍的陈济棠于19日与白崇禧会于广州,两日决定借“抗日”为名公开反蒋。19日,陈、白二人召集数十名粤桂高级将领开会,研究反蒋大计。会上,白崇禧信誓旦旦地表示,桂方将全力支持粤方反蒋,使陈最终下定开战决心。6月1日,陈济棠、李宗仁在广州召开“西南两机关”联席会议,决定以粤桂两军打出“抗日”旗号,北上讨蒋\footnote{《广西通史》第三卷,页168}。决定南粤历史走向的“两广事变”,就此爆发。

举起反蒋抗日的旗号后,粤桂两军迅速改编为“国民革命军抗日救国军”,由陈济棠、李宗仁分任正副总司令。6月9日,在粤空军的掩护下,粤军三个师、桂军四个师分道攻入湘南,各自占领郴州、永州。蒋中正忙将中央军两个军由湖北调入湘南,于18日抢占衡阳,与粤桂联军对峙于郴州、祁阳一线。这时,粤桂双方已出现裂痕。对陈济棠来说,抗日只是他为维护南粤自立打出的反蒋旗号,但新桂系却是真心倾向抗日的。6月下旬,100余名日本军官抵达广州,入住沙面及新亚酒店,准备协助粤军抗蒋。与此同时,蒋中正开动其国民党学自苏联的情报机关,以重金相继收买粤空军司令黄光锐和第一军军长余汉谋。30日,粤空军飞行员黄志刚率7架战机叛粤投蒋。7月4日,又有48架战机离粤,在南昌降落,宣誓效忠南京政府。同日,余汉谋于广州以其第一军发动兵变,宣布拥蒋,并率部向南雄、韶关推进,对陈济棠实施“兵谏”。6日,又有粤军副军长李汉魂、邓龙光、虎门要塞司令李洁之通电反陈。至18日,黄光锐更亲率72架飞机自广州飞至粤北,宣布投蒋。在蒋中正的收买之下,强大的粤军分崩离析,纷纷投入国民党和共产国际的怀抱,空军更是全体叛逃。陈济棠只得于18日通电下野,奔赴香港,离开了他曾拼命守护的广东。24日,余汉谋全面接管广州防务\footnote{《广西通史》第三卷,页169—170;《广东通史》现代上册,页779—801}。曾经傲立于南粤的“南天王”,就这样令人扼腕地倒下了。

与此同时,新桂系虽仍标榜反蒋,却正与蒋共双方积极勾结。7月,中共秘使云广英抵达南宁,对李宗仁表明中共建立“抗日民族统一战线”的主张,希望新桂系停止反蒋。同月14日,蒋中正下令撤销“西南两机关”,免去陈济棠职务,任命余汉谋为广东绥靖主任、李宗仁、白崇禧为广西绥靖正、副主任。其后,蒋中正亲赴广州,自8月10日起与桂方代表接触。桂方提出,蒋中正必须准备“抗战”,桂方则将全力协助。同月,新桂系宣布恢复第十九路军建制,派一部进占广东北海,表明强硬的抗日立场。9月2日,蒋方代表抵达南宁与李、白正式谈判。3日,新桂系以军队煽动暴民,于北海杀害日商中野顺三。6日,蒋方接受桂方要求,双方达成和平协议,南京国民政府正式任命李、白为广西绥靖正、副主任。17日,李宗仁亲赴广州,当面向蒋中正表示服从\footnote{《广西通史》第三卷,页171—172}。在蒋桂勾结之下,持续三个月的“两广事变”以南粤的全面失败告终。

随着“两广事变”落幕,陈济棠治粤的时代结束了。新桂系虽得以延续,却已投靠蒋中正。在两者以“西南两机关”的名义联合治理南粤的五年里,南粤曾获得一段难得的和平、自立岁月,取得长足进展。陈济棠虽不似陈炯明那般坚毅果决,但他的所作所为绝对无愧于他的祖国。然而,朴实的本土军人陈济棠终不能抵挡国民党列宁主义机器的侵蚀,新桂系则公开背叛南粤父老、投入蒋中正和共产国际的怀抱。自此,南粤史上的最后一次自立终结。南粤被绑在列宁主义的疯狂战车上,开始滑向无边的黑暗。

\section{二战深渊中的南粤:1936—1945年}

\indent 1936年9月6日,新桂系向蒋中正屈服。新桂系投蒋后,首先于10月将广西省政府迁至桂林备战,又进一步与中共勾结,于1937年1月派代表赴延安与毛泽东面谈合作事宜。6月12日,负责中共华南统战工作的张云逸与李宗仁、白崇禧及刘湘的代表张斯可在桂林举行会谈。下旬,三方拟定《红、桂、川联合抗日纲领草案》,结成反日同盟\footnote{《广西通史》第三卷,页302—303}。7月7日,由中共北方局策动的“卢沟桥事变”爆发,日军与国民党第二十九军在北平郊外爆发激战。17日,蒋中正发表《庐山谈话》,宣称“地无分南北,人无分老幼”,皆有“抗战守土”之责。29日,日军克北平。8月8日,日本内阁拟定停战条件,准备与蒋中正和谈。然而,蒋中正却调集大批中央军,于13日大举进攻上海日租界,“淞沪会战”爆发。15日,日本正式下达动员令,编成上海及北支那派遣军。至此,双方再无回旋余地,日蒋间爆发全面战争。二战的战火,在东亚点燃了。

8月20日,蒋中正向国民革命军各部下达命令,划出五个战区,以河北及鲁北为第一战区,晋、察、绥为第二战区,苏南、浙江为第三战区,粤、闽为第四战区,苏北及鲁省为第五战区。其中,第四战区司令长官由其亲信何应钦出任,副司令长官则为余汉谋。至于人称“小诸葛”的白崇禧则因足智多谋,被调往南京出任副参谋总长。22日,陕北共军主力接受国民政府改编,成为国民革命军第八路军\footnote{王辅:《日本侵华战争》}。至此,国共两党绑架了东亚各邦,全面建立“抗战”体制。

战争爆发后,粤、桂两军分别受南京方面改编。粤军被编为第十一集团军,以余汉谋为总司令,下辖第六十二、六十三、六十四、六十五、六十六、八十三共六个军。桂军则被编为第十一集团军,以李宗仁、白崇禧分任正副司令长官,下辖第七、三十一、四十八共三个军。由于广西拥有大批训练有素的民团做为后备兵员,新桂系很快将正规军扩充至10万之众。9月中旬,在蒋中正的命令下,桂第七军、四十八军出桂奔赴上海前线,成为二战中首批沦为共产国际炮灰的南粤子弟兵\footnote{《广西通史》第三卷,页}。

自9月起,淞沪会战陷入胶着。日蒋双方在上海北郊陆续增兵,将此战变为一战以来世界上规模最大的战役。到10月初,国民党军兵力已达六个集团军共40万人、日上海派遣军兵力达五个师团共20万人,双方隔蕴藻浜对峙,均伤亡惨重。10月15日,由廖磊(陆川人)指挥的桂第七、四十八两军赶到上海,被编为第二十一集团军,投入蕴藻浜南岸阵地。21日夜8时,桂军在蒋中正的严令下向谈家头、陈家行一带的日第十三、第九师团发起大规模反攻。这些勇敢而淳朴的八桂子弟第一次领会到了二战的恐怖。日军的猛烈炮火,是他们在此前的战争中从未遇到过的,他们只能以血肉之躯组成人海向日军发起决死冲锋,成片倒下。至22日晨,战斗仍在继续,一名国民党将领曾如是描述惨烈的战况:

\begin{quote}

是日晨,风向不利,烟雾反吹向我方,炮兵看不清目标,无法支援。日军炮多威力大,视界清楚,我炮一发射即可被制压,似知道我反攻部署,预先将坦克及机枪等火力布置。桂军官兵不知厉害,挺直身体毫无掩蔽地向敌阵猛进,拿起步枪向坦克冲锋。敌人放桂军官兵进到阵地前,即用火力前后封锁,猛烈射击。肉体挡不住子弹,又无藏身之地,桂军纷纷壮烈牺牲\footnote{毕洪:《1937淞沪会战》,页111—112}。

\end{quote}

“肉体挡不住子弹”,桂军很快耗尽了鲜血,攻击宣告失败。经战后清点,仅第四十八军就有4名旅长、10名团长伤亡,营级以下官兵伤亡过半。数万八桂子弟兵,就这样倒在上海北郊阴冷的深秋中,成为永远无法回乡的亡魂。25日晨,日军突破蕴藻浜,国民党军被迫退至苏州河南岸。11月5日,由四个师团组成的日本第十军在杭州湾金山卫登陆,从南面逼近国民党军。南北夹攻之下,数十万国民党军全军崩溃,蜂拥西逃。12日,日军占领上海,淞沪会战结束。此战,东亚各国军人都付出了无比沉重的代价,日军伤亡9万余人,国民党军的伤亡则达33万人以上。20日,国民政府惊慌失措地宣布迁都重庆,留下8万炮灰死守南京\footnote{王辅:《日本侵华战争》}。

攻占上海前夕,日军大本营将上海派遣军与第十军合并为中支那派遣军,于11月28日命其发动南京攻略。当时,粤第六十六、八十三军正在开赴上海途中,因上海失守,遂进驻南京。30日,蒋中正致电斯大林,奴颜婢膝地哀求其火速出动“义师”支援,被斯大林一口回绝。12月8日,日军合围南京,向南京城发起猛攻。12日下午,因南京卫戍司令唐生智带头逃跑,守军崩溃,数以万计的中央军乱兵一面烧掠平民、一面涌向城北江边争抢逃命船只,在自相残杀中尸积如山,还有许多人脱下军装躲进平民区。一片混乱里,只有粤军临危不乱,维持着军纪。六十六军军长叶肇(新兴人)与八十三军军长邓龙光(茂南人)当机立断,整队由太平门出城突围。粤军将士高呼粤语口号发起决死冲击,一举突破日军阻截,经紫金山麓成功进入皖南国民党军防区。13日,日军占领南京,随后在清剿国民党残兵时军纪失控,屠戮了相当数量的战俘与平民\footnote{王辅:《日本侵华战争》}。在南京陷落前夕,粤军非但未如中央军一样抢掠平民、自相残杀,反而保持着军纪突围而出,这真是我南粤武德的极好证明。

日军攻占南京后,于1938年初再开攻势,以中支那派遣军、北支那派遣军沿津浦铁路南北对进,合攻徐州,试图摧毁国民党第五战区,“徐州会战”爆发。蒋中正亦决心在徐州与日军决一死战,任命李宗仁为第五战区司令长官,调集60万大军防御徐州,其中包括桂第七、三十一、四十八军。三十一军是ngamngam 出桂参战的部队,军长为李品仙(苍梧人)。李宗人将桂军部署于南线,命李品仙负责迎击北上日军。2月1日,桂三十一军在凤阳附近主动攻击日第十三师团之搜索警戒部队,因受阻于日军强大火力,于激战两日后败退。6日,桂七、四十八军抵达津浦路南线战场,进驻合肥。李品仙乃于8日集中三个军兵力反攻,一举击退日军,击毁多辆坦克,占领凤阳、考城。日第十三师团乃停止北进,与桂军转入对峙。4月7日,李宗仁在台儿庄以第五战区主力成功击退日军南下部队。然日军于5月5日再次南北对进,对徐州形成合围之势。15日,蒋中正下令放弃徐州。次日,李宗仁命第五战区各部分头向豫、皖交界山区突围。19日,日军占领空城徐州,未能捕捉到成建制的国民党军。在此期间,为掩护第五战区主力撤退,桂军奉命在南线死守阵地数日,伤亡近万,其后退守大别山区\footnote{《广西通史》第三卷,页330—332}。

与徐州会战同时,日北支那派遣军又以第十四师团约3万人进攻河南之国民党第一战区,试图切断徐州守军退路。当时,李汉魂指挥的粤第六十四军ngamngam 赶到河南前线。第一战区遂以包括六十四军在内的五个军共15万兵力迎战,于5月下旬围第十四师团于兰封。日十四师团是一支极其精锐的部队。在日后的太平洋贝里琉战役中,该师团曾战至最后一人,给美陆战一师造成超过三分之一的伤亡。此战,该师团由名将土肥原贤二指挥,志在必得。5月25日,国民党军各部猛攻第十四师团。28日,粤六十四军旗开得胜,在重炮掩护下攻占兰封以西的重要据点罗王车站。当时,土肥原正在车站中督战。仓促之间,他抛下大批辎重慌忙逃走,连他的武士刀也被粤军缴获,成了李汉魂的佩刀。然而,因日军于30日发起反攻一举击溃中央军,导致围攻部队全线败退,蒋中正只得于31日命第一战区撤围。为阻止日军沿陇海线西进郑州、威胁武汉,蒋中正竟在6月9日丧心病狂地下令掘开郑州北郊的花园口黄河大坝,使河南、安徽、苏北的89万平民葬身鱼腹、390万人流离失所。国民党之残暴冷血,一至于斯\footnote{郭汝瑰、黄玉章:《中国抗日战争正面战场作战记》}。

攻占徐州、兰封后,日军沿长江西进,直指武汉。蒋中正于1938年6月设第九战区统筹湖北南部及湘赣战事,以其亲信陈诚及粤人名将薛岳指挥。7月26日,日军克九江,随即分五路向武汉推进,“武汉会战”爆发。7—9月间,桂军据守大别山区的坚固阵地,在鄂东广济、黄梅一带与日军反复激战。在长江南岸,薛岳以第九战区的10万大军迎战日第一〇六师团,粤第六十六军身列其中。10月1日,九战区合围日一〇六师团于九江附近的万家岭。该师团战斗力较弱,被逐步分割包围。8日,粤六十六军以决死的白刃突击攻占万家岭主峰,于入夜时分突至日军师团司令部。师团长松浦淳六郎中将率败兵拼死突围,险些被俘,于9日被一〇一、二十七师团的援军救出。此战,日一〇六师团折损近半,16281名战斗兵员仅剩9187人,国民党军阵亡者超过2万。赣北的群山间,成千上万的南粤青年就这样悲惨地死去,被国民党和共产国际的疯狂理念吞噬。战后,薛岳视察战场,见到满山遍野的粤军尸体,同为粤人的他唯有放声痛哭\footnote{郭汝瑰、黄玉章:《中国抗日战争正面战场作战记》}!

日军虽受阻于赣北,却在长江北岸得手。10月12日,日军绕过大别山,经北麓攻占信阳。国民党军战线动摇,全线撤退。25日,日军占领武汉\footnote{王辅:《日本侵华战争》}。此后,桂军一分为三,以廖磊率第二十一集团军留守大别山、李品仙率第十一集团军防御鄂西北,广西本土则组建由夏威、韦云淞分任正副总司令的第十六集团军\footnote{《广西通史》第三卷,页336}。与此同时,日军发动广州战役,二战打到了南粤本土。

自1937年9月15日日机首次空袭广州起,南粤本土便初步品尝了二战的恐怖滋味,连续不断的防空警报与轰炸成为南粤平民对二战的最初记忆。1938年1月17日,日军登陆香山县三灶岛,又于*8、9月间先后夺取南澳岛、涠洲岛。9月7日,日本经御前会议决定进攻广州,切断国民党从香港、澳门进口物资的通道。17日,日本以第五、九、一〇四师团共7万人编成第二十一军,由原台湾军司令官古庄干郎中将指挥,令其发动广州攻略。当时,因粤军大量北调参战,广东本土的部队仅有六十二、六十三、六十五军。六十四军在开战前紧急回防广东,使本土守军兵力增至四个军共6万人。兵力对比如此,战役结果不问可知。10月12日凌晨,日第十八、一〇四师团在大亚湾登陆,排除守军抵抗迅速突击,于当日午夜占领淡水。接着,日军以第十八师团为左翼、一〇四师团为右翼快速推进,于15日陷惠阳、惠州、16日陷博罗、19日陷增城。20日傍晚,日军逼近广州城郊,余汉谋召集第四战区司令部人员集会,决定采取“焦土政策”,弃城北逃。21日凌晨4时,第四战区司令部沿广花公路逃向清远。天亮后,广东省主席吴铁城宣布实施“焦土抗战”,将广州的所有机关公署、重要工厂、公共设施一律爆破,随后溜之大吉。在一连串的爆炸声中,广州市民人心惶惶、纷纷逃难,遭日机扫射。据一名国民党军官回忆,当时的情形惨不忍睹:

\begin{quote}

那时市面上一片混乱,市民们多半向西北方向奔跑。我们到达荔枝湾,看到有一百多部汽车等待着渡江。那里的渡船能力有限,每小时只能渡4部车子。于是我们立即折返黄沙码头,找到一艘自备的电船,撤往清远……当我道经黄沙时,回头看到市面行人已经不多,珠江河面小艇已向西走避一空。远望河南士敏土(水泥)厂附近、东山天河机场附近、三元里白云机场附近,都冒出了浓黑的火烟,还传来一阵阵的爆炸声,大概是在烧毁一些搬不走的军用物资了。在佛山西南,沿途看到无数扶老携幼、拖男带女的难民。他们沿着广三铁路向西奔跑,不断受到敌人飞机分批袭击。死者暴尸,伤者喊救,生者抢路,惨状难言\footnote{转引自《广东通史》现代下册,页163}。

\end{quote}

无数南粤民众便这样在二战的地狱中煎熬,成为国民党与共产国际的牺牲品。21日下午1时30分,日第十八师团冲入广州市区,广州失守。22日下午,日第五师团在珠江口登陆,于23日攻占虎门要塞所有炮台、25日陷龙江、26日陷佛山。29日进抵广州市区以南之货仓,广州战役结束。此战,日军付出的代价不过是173死493伤。粤军主力未作顽强抵抗即撤往粤北,损失不算太大。据日方战报,日军此战共缴获步枪2371支、轻重机枪214挺、火炮188门、战车21辆、汽车151辆、俘虏1340人\footnote{《广东通史》现代下册,页165}。战后不久,古庄中将因战功升任参谋本部附,回国述职,日第五师团长安藤利吉中将出任第二十一军司令。安藤中将命第十八师团扩大战果,于11月22日陷石龙、23日陷东莞、26日陷深圳。1938年底,日本大本营又将台湾混成旅团编入第二十一军,令其进行海南攻略。1939年2月10日,日军于澄迈登陆。海南岛上国共守军仅有2000人,无力抵抗。14日,日军攻占三亚,海南攻略结束。6月中旬,安藤中将又命第十八师团第一三二旅团发动潮汕攻略。是月14日,该旅团于广州黄埔港乘船出发,于20日在汕头以东地区登陆、22日陷汕头、27日陷潮州。至此,广东最富庶的珠江、韩江三角洲已全为日军所据\footnote{王辅:《日本侵华战争》}。因广州轻易失守,蒋中正勃然大怒,改任张发奎为第四战区司令长官、降余汉谋为副司令长官、免去吴铁城的省主席职务,并命李汉魂率岭北粤军回援。1939年1月1日,张发奎、李汉魂分别就任第四战区长官及广东省主席。2月,第四战区司令部、广东省政府正式迁至韶关办公\footnote{《广东通史》现代下册,页255}。

日军攻占珠三角后,国民政府仍能通过越桂公路和滇越铁路输入外援物资。英、法物资在越南海防港上陆后,即可由上述两路输入桂、滇。因此,日本大本营自1939年2月起便策划广西攻略,计划切断南宁—龙州公路。10月16日,大本营命二十一军以台湾旅团及第五师团之第九、二十一旅团发动桂南攻略,攻占钦州湾、南宁、镇南关、龙州。11月15日晨,日军分乘70余艘舰艇于钦州湾登陆,经激战巩固滩头阵地,而后以第九旅团在西、第二十一旅团在中、台湾旅团在东,分三路向南宁推进。蒋中正对此次作战非常重视。为守住国际交通线,他急命中央军第五、三十六、九十九军援桂。21日,国民党中央军抵达邕江北岸布防,与日军反复激战,终因不敌日本海军航空兵的轰炸败退。24日,日军攻占南宁。日军随后于南宁修筑机场、修复被国民党军破坏的公路,完工后以第九旅团继续进击,于12月21日攻占龙州、镇南关,但因国民党军已组织全桂兵力大举反攻南宁,只得撤回。国民党一方对此次反攻志在必得,白崇禧由重庆飞抵桂林亲自督战。12月18日,国民党军以四个集团军共15万兵力猛攻南宁,与日第二十一旅团在距南宁约四十公里的要隘昆仑关爆发激战。主攻昆仑关的国民党第五军是中央军王牌部队,该部装备了大量苏援T26坦克,并配有苏联顾问,军长杜聿明是蒋中正的心腹爱将。经连日血战,双方伤亡惨重,日二十一旅团旅团长中村正雄少将于25日战死,两个联队长亦相继阵亡。31日,第五军终于攻占昆仑关。此战,日二十一旅团战死4000人,国民党军损失达1.6万余人,双方都暂时无力再战,日方乃命第十八师团增援桂南战场,并派出长驻本土的近卫师团第一旅团参战。1940年1月22日,日军援兵在南宁上陆,于28日投入反攻、2月3日重夺昆仑关、8日陷武鸣。国民党军全线败退,遗弃大量尸体、物资。13日,日军主力撤回南宁,“桂南会战”以日军的胜利结束。据日方统计,在1940年1月28日至2月13日间,日军清点到国民党军尸体27041具、俘虏1167人。战后,入桂中央军被调回后方休整,双方转入对峙\footnote{《广西通史》第三卷,页349—366}。桂南会战期间,广东的日军又于1939年12月17—30日间对粤北展开牵制性进攻,一度占领翁源、英德,随后撤出,是为“第一次粤北战役”。事后,国民党方面大肆宣传,称他们收复了粤北“失地”。1940年5月14日,为阻止粤军援桂,日军又自从化、花县北上,发动“第二次粤北战役”。因粤军及民团在山地中顽强抵抗,日军进展不利,遂于6月5日退回原战线\footnote{《简明广东史》,页758}。 

1939—1941年间,世界局势发生了惊人的变化。1939年9月,苏德两国瓜分波兰,英、法对德宣战,二战在欧洲爆发。1940年3月30日,日本扶持国民党大佬汪兆铭成立“中华民国国民政府”,定都南京,南粤日占区在理论上又成为“中华民国”的疆土。5—6月间,德国闪击西欧,荷兰、比利时、卢森堡、法国相继投降。9月,日本乘法国之危,以第五师团越过桂越边境进占法属印度支那。这样,日军已从根本上切断桂越公路,维持桂南占领区已无必要,便于11月17日乘船撤离桂南。日本的突然南进使美、英及控制印尼的荷兰流亡政府空前恐慌。美国于7月25日冻结日本在美资产,又于8月1日与英、荷联手对日实施石油禁运与经济制裁,日本与英美的矛盾迅速激化。日本首相近卫文磨更进一步,于同月提出“大东亚共荣圈”的概念,号召东南亚各地在泛亚主义的旗帜下独立,在日本的帮助下打倒英美。9月27日,德国、日本、意大利三国正式结盟,二战的“轴心国”阵营形成。在欧洲,苏联与纳粹德国于1941年初完成对东欧的瓜分,两国大战一触即发。1941年4月13日,日本、苏联签订《日苏中立条约》,苏联得以专心备战,不再有东顾之忧,乃停止向国民政府输送军援。对斯大林而言,蒋中正已完成其做为人肉盾牌的使命,被无情地抛弃了。6月22日,苏德战争爆发,斯大林再也无暇关照蒋中正。在失去共产国际金主的情况下,蒋中正唯有采取机会主义姿态,与他立志要打倒的西方帝国主义国家靠近。1941年12月7日,日本偷袭夏威夷美国海军基地,制造“珍珠港事件”,太平洋战争爆发。8日,美、英、荷对日宣战。9日,蒋中正正式对日宣战。11日,德国、意大利对美英宣战,蒋随即对德、意宣战。荒谬的阵营划分就这样定型了:标榜反对列强、甘当共产国际白手套的蒋中正,这次居然加入了由英、美主导的“同盟国”阵营。

太平洋战争爆发后,日军迅速南进。早在战前,日本大本营便将驻粤日军改组为第二十三军,以酒井隆中将为司令,第二十一军则被编入负责东南亚作战的“南方军”系统。战争爆发第二天,酒井中将即以第三十八师团越过粤港边境,发动“香港战役”。12月26日,香港英军投降。同时,东亚大陆上的西方租界也几乎全被日军占领,日占区的西方侨民则都被投入集中营。至1942年5月,日军已攻占缅甸、马来亚、新加坡、菲律宾、荷属东印度、新几内亚和所罗门群岛,控制了惊人的领土与海域\footnote{关于二战进程,相关研究甚多,此不赘注}。转瞬之间,日本的泛亚主义战争机器就将西方世界带给南粤的秩序破坏殆尽。

在太平洋爆发前,国民党军是共产国际赤化南粤的白手套,日军则是当时有实力阻止赤化的唯一力量。在国民党的卵翼下,中共在粤、桂的党组织纷纷公开露面,中共的游击队“东江纵队”、“珠江纵队”、“韩江纵队”、“琼崖纵队”在东莞大岭山、宝安阳台山、香山五桂山、潮汕、海陆丰地区及海南五指山积极活动,一些中共干部更堂而皇之地被编入国民党军。自1940年起,因国共武装在岭北冲突不断,双方关系转恶。1941年1月,国民党军围歼新四军军部的“皖南事变”发生。此后,粤军开始限制中共活动,于当年4月捕杀中共驻第十二集团军的政工干部,新桂系更将非桂籍中共干部统统礼送出境,又在1942年逮捕一批中共党员。自1943年春起,中共党组织在广西的活动已转入地下\footnote{《广西通史》第三卷,页376—380;《简明广东史》,页762—766}。另一方面,自蒋中正与英美结成机会主义的同盟后,国民党出于盟友义务,不得不暂时放弃打倒西方列强的幻想、与共产国际疏远,在一定程度上维护西方在粤利益,而鼓吹泛亚主义的日本则成为西方秩序的破坏者。对粤人来说,英美历来是伟大的盟邦。英国不但在南粤身边创造了伟大的香港城邦,更曾用炮舰打破清帝国对南粤的桎梏,而美国则是与南粤隔太平洋相望的邻国。英美等国不但为航海粤侨提供了避风港,更曾促使南粤文明开化、雄踞于世界。因此,在太平洋爆发后,与英美结盟的国民党降为南粤的次要敌人,挑战西方秩序的日本暂时成为南粤的主要敌人。

太平洋战争开始后,日本为满足其“大东亚战争”需求,开始对南粤日占区进行严酷的经济统制。当时,日军占有广东珠江、韩江三角洲地区,控制了广东3200万人口中的1200万。自1942年起,广州的外贸便被日本独霸,南粤与西方世界的传统贸易往来被彻底切断,西方粤侨向家乡汇款汇物的渠道也被封锁,甚至南洋粤侨的侨汇也受到严格管制。为筹措战争经费,日本还在日占区加征财产税、人头税、衣服税、皇军慰劳金、国防献金等百余种苛捐杂税。1943年2月16日,日第二十三军又发动广州湾攻略,于21日完全占领雷州半岛上的法国租借地,当地法国行政机构沦为日军傀儡,西方国家在广东的租借地遂告全灭。此后,日本加紧对日占区的掠夺,仅在1943年7—10月间就从广州、香港、海南捕走1.7万余壮丁,送往南洋服苦役。同年8月12日起,同盟国飞机开始空袭广州,日方又勒购民谷、逼抽田捐,使广州市民人心惶惶、纷纷逃至乡下避难,市区人口骤减至50万人\footnote{《广东通史》现代下册,页763}。更为雪上加霜的是,自1942年冬至至次年春分,富庶的珠江三角洲和粤东滴雨未落,田中待熟的稻苗纷纷枯死。1943年春分后不久虽有一场小降雨,但此后直到立夏又是滴雨未降,立夏后更发生大规模涝灾。到1943年夏,灾情蔓延到粤东北国统区,灾区许多百姓已经断粮,但他们不但几乎得不到海外亲友的侨汇,更被逼交种种苛捐杂税。仅在当年5月,潮汕、梅州、兴宁即有30余万灾民逃向闽、赣。到下半年,粮食紧缺导致米价腾贵,饿死人的事件开始到处发生。而日本和国民党当局不但未加认真赈济,反而继续征税,终于导致南粤史上空前的大饥荒爆发。在灾区,农民以草根、树皮为生,如苍蝇般成片死去,更多人则变为四处逃荒的流民。宗族长老和各地土豪虽尽力赈灾、维持社会秩序,却无法阻止灾难的发生。饿极的人们吃尽了一切可吃的东西,不少人以死尸为食,极个别地区甚至发生了掳走幼童杀食的惨剧。南粤历来富庶、粤人历来质朴有礼,这种在岭北随处可见的大规模灾荒食人现象还从未发生过。在大灾难的逼迫下,连南粤社会中最基本的人伦都已动摇。到年底饥荒结束时,无数饿殍已化为白骨,他们中的许多人甚至在死后也无人领尸,或倒在田间街头、或被埋入乱葬岗、或被抛入海中。据一名国民党官员的调查报告,死于饥荒的人数达到了骇人的规模:

\begin{quote}
今年广东灾害严重,台山饿死20余万人,开平饿死10余万人。又闻汕头饿死近百人,甚至有人吃人者\footnote{转引自《广东通史》现代下册,页803}。
\end{quote}

上述数字绝不是完整的死者数。据现存史料记载,在1943年,还有潮汕地区饿死50万人、新会县城饿死3万人、电白县饿死3万人。掩藏在冰冷数字下的,是一副惨绝人寰的地狱图景。至少80多万南粤同胞没能活过1943年,永远地消失了。“民国三十二年(1943)”成了粤人心中永远的恐怖记忆,一首流传于灾后的童谣便是最好的佐证:

\begin{quote}
千记住,万记住,记住民国三十二!大雨绵绵水浸地,围田被冲崩几处。谷粒无收,人情失意,一元买米只得三钱二。饿到肚皮薄过纸,手拎钵头乞米去。穷人卖仔又卖女,脚烂梅梅无钱治,求死不得生存无意义。死尸遍野!荒田遍地\footnote{转引自《1943珠海饥荒的记忆》}!
\end{quote}

日本泛亚主义者挑战世界秩序的疯狂举动不但给可爱的日本带去沉重的灾难,也使空前惨烈的悲剧降临南粤。今天的日本固然是守卫世界秩序的排头兵,亦是我南粤的可靠友邦,但我们绝不能忘记这段历史,不能忘记粤日两国间曾发生的悲剧。只有铭记它,由粤日民众一同纪念它,这种悲剧才能永远不再发生。这场惨绝人寰的悲剧也告诉我们,在任何情况下,一个国家都不应该为疯狂的理念破坏他国自由。对一些幻想建立三越统一国家或以南粤攻占云贵的小型大一统爱好者来说,这场悲剧无疑是最好的警钟。在追求自由的道路上,我们一定不能自我膨胀,不能制造相似的悲剧。

1944年,国民党将广东自第四战区划出,设为独立的第七战区,以余汉谋任司令长官。同年初,日军开始施行一次空前的行动。因在太平洋战场不断失利,日本谋求在东亚大陆挽回败局。日本人的目光瞄上了贯通大陆南北的粤汉、平汉铁路。当时,日军虽早已攻占北平和武汉,但铁路河南段、湖南段、广东段和广西支线仍在国民党军手上。此外,自1942年起,美国陆军航空兵进驻东亚大陆国统区,其轰炸范围远及台湾。日本遂决定发动一次大规模攻势打通大陆交通线,将美国飞机赶进内陆。为了这次进攻,大批关东军被调入关内,日军集结了十八个师团又十二个混成旅团共41万大军,由支那派遣军司令官冈村宁次大将指挥。在二战史上,日军的这次用兵规模是前所未有的。日本大本营对其极端重视,称之为“一号作战”。4月18日,大战揭幕。日第十二军以15万兵力渡过黄河,向河南的国民党第一战区发起进攻。第一战区虽有40万军队,但作战无方、残民有术,很快崩溃。到5月下旬,日军已占领郑州、许昌、洛阳,打通平汉线。同月,日第十一军司令官横山勇中将集结超过20万兵力由北向南攻入湖湘,于6月18日攻陷长沙,十日后兵临衡阳城下。衡阳守军以劣势兵力死守月余,然因援军未至,衡阳仍于8月8日失陷,国民党第九战区土崩瓦解。9月10日,冈村大将指挥十一、二十三军共16万人,在约150架飞机的掩护下攻向桂北,“桂柳会战”爆发。当时,广西守军除张发奎麾下的桂第十六集团军外,尚有来自一、六战区的援兵,总兵力为约20万人。但因桂军大批北调,留守者多为弱兵,日军得以分两路迅速推进。14日,自湘入桂的日十一军攻陷全州,国民党军于撤退前纵火焚城,大火十余日不熄。日二十三军溯西江而上,排除粤军在四会、广宁的抵抗,于22日攻陷梧州,又于11月12日陷桂平。其后,两路日军分进合击,于10月底逼近桂林。这时,原驻桂林的广西省政府早已逃至百色,守将第十六集团军总司令夏威采取疯狂的“焦土抗战”政策,以“扫清视界”为名下令在北郊纵火。由于西北风强劲,火势迅速向南蔓延,将桂林城化为一片火海,吞噬了大批军民。11月1日,日二十三军一〇四师团及十一军三、十三师团近10万人开始攻城。守城的桂三十一军和民团虽仅有2万兵力,许多团兵的武器只有土枪。但他们大都是生长于斯的八桂子弟,绝不愿家乡沦丧,仍依托废墟顽强抵抗,民团更组织起身绑手榴弹的敢死队,前仆后继地扑向日军坦克和登陆艇。10日,经惨烈巷战,桂林陷落,第三十一军除少数残兵外全军覆没,城防参谋长陈济桓、第一三一师师长阚维雍举枪自尽,军参谋长吕旃蒙战死,漓江江水化为血红色。据战后日军公布的战果,此战桂军有5665人战死、13151人被俘。至于葬身火海的平民人数,更是无人统计。同日,柳州亦被日十一军攻陷,柳州机场上的30架美军飞机被日军摧毁。至此,桂军主力已败,冈村宁次遂命第二十三军南下,于24日攻陷南宁。28日,日南方军第二十一军一部自越南入桂,于12月10日在广西绥淥与二十三军会师,日军彻底打通广西支线。1945年1月3日,日二十三军又与驻湘南的第二十军联手发动打通粤汉铁路南段的作战。二十军一路排除国民党中央军抵抗,于1月25日陷郴州、2月6日陷赣州。二十三军则发动“第三次粤北战役”,在粤军的顽强防御下节节推进,于1月26日陷韶关、2月3日陷南雄、5日越梅岭、7日越大庾岭、9日与二十军一部会师于赣南新城,广东省政府被迫迁往连山。至此,日军打通“大陆交通线”的战略目的完全达成。在“一号作战”中,日军以41万兵力击溃国民党第一、九战区,重创粤桂两军,连陷三十八城\footnote{王辅:《日本侵华战争》},但自身亦付出沉重代价,减员达10万以上,国民党军的伤亡则高达50—70万之众。此战过后,“中华民国”的国际形象一落千丈,不再被同盟国阵营视为与可英美相提并论的大国。

不过,日本人的好运也就到此为止了。1945年1月,美蒋联军打通滇缅公路,奄奄一息的国民政府总算获得源源不断的美援。3月,美军攻占硫磺岛,逼近日本本土。日本军部高叫“本土决战”,提出狂热的“一亿玉碎”口号,并命东亚大陆上的日军收缩战线,为防御美军登陆向海岸线靠拢。4月14日,日军开始逐步放弃广西。在国民党军的尾随攻击和民团的袭扰下,日军的撤离行动进行得异常缓慢。趁此机会,中共在广西各地组织起多股游击队,一边扩充实力、一边袭击日军,进一步拖慢了日军的撤退速度。在撤退时,日军为不留资敌物资,竟残酷地将沿途所有房屋付之一炬。5月26日,国民党军重返南宁。29日,重占柳州。7月27日,国民党军进占桂林。极目望去,全城99\%的房屋已经烧毁,所见者唯有一片废墟。8月15日,国民党军开入梧州,又在全州与日军发生激战。同日,日本宣布无条件投降,双方停止一切敌对行动。9月7日,全副美械的国民党新一军进驻广州。16日,日第二十三军司令田中久一中将与张发奎在广州举行受降仪式,田中中将代表驻南粤的所有日军向国民政府正式投降,二战在南粤结束。在1944—1945年的大规模战役中,有21.5万广西平民死于双方军队的交火和暴行,给南粤的二战记忆留下了惨烈的最后一笔\footnote{《广西通史》第三卷,页398—400}。经过八年艰苦岁月,二战终于在南粤结束了。当战争结束时,成百万的南粤人已经死去。然而,恐怖的浩劫没有停止。第二次国共战争已一触即发,即将使南粤坠入更黑暗的深渊!

\section{铁幕降临:1945—1950年}

\indent 1945年9月16日,二战在南粤正式结束。19日,国民政府新任广东省主席罗卓英(大埔人)自重庆飞抵广州就职。国民党军政官员随之趾高气昂地进入原日占区各地,以“接收”为名四处劫掠,动辄将绅商的财产指为“敌产”没收。在广州,国民党新一军自视为“解放者”,横行霸道,四处强买商品、抢劫民财、奸淫日军女眷,打死打伤市民之事时有发生,使广州大部分商铺只得一近黄昏就关门大吉。新一军是曾在印度接受美式训练的部队,其军纪在国民党军中数一数二。该军如此,“接收”南粤各地的国民党军行为如何则不问可知。当时,广东民间流传起这样一首歌谣:

\begin{quote}

睇错老蒋(中正),迎错老张(发奎)。搭错牌楼,烧错炮仗\footnote{转引自《简明广东史》,页800}!

\end{quote}

国民党在广东的横行霸道使刚经历二战的南粤人不得丝毫喘息,严重的自然灾害又使情况雪上加霜。1945年下半年,粤、桂、湘接连发生水、旱、风、虫灾,粮食收成大减,导致饥荒发生。此次饥荒虽不如1943年猛烈,但仍触目惊心。仅1945年9月20—12月26日间,广州街头就出现1979具路尸,平均每日有20具以上。到1946年,春旱、夏涝在粤、桂接连发生,使灾情更为严重,灾民纷纷食用杂粮、草根、树皮。粮食紧缺致使米价腾贵,国民党当局滥印法币又使物价暴涨,加之广州爆发霍乱,死者日益增多。至该年冬,广州街头已先后统计到7000具路尸,全广东等待救济的灾民超过500万,占广东人口七分之一以上,病饿死者至少有13万人\footnote{《广东通史》现代下册,页1050}。在广西,可怕的瘟疫随饥荒一同降临。1946年,广西染疫者超过50万人、灾民多达340万人,占全桂人口四分之一以上,死者数很可能超过50万\footnote{转移自《广西通史》第三卷,页454}。

在对待灾疫的态度上,广东国民党当局与新桂系形成鲜明对比。罗卓英虽为粤人,却是蒋中正的心腹走狗。张发奎虽曾参与反蒋,此时也早已投蒋。到达广东后,罗、张二人忠实执行蒋中正的命令,大肆压榨、搜刮。1946年7月1日,广东当局开始征粮,要求粤民在年内交足450万石粮食。但因民间疲敝已极,病饿而死者众多,广东当局直到11月才征到47万石。蒋中正大为光火,要求罗卓英不顾百姓死活强制征粮。罗遂派出大批“督征团”、“催征队”下乡强征,被逼到投水、自缢的民众不计其数。然而,直到年底,广东当局也仅征得120万石。同年9月3日,国民党开始在广东征兵,要求在年内征足3.4万人、凡聚众持械抗拒者一律处以死刑。由于无人愿意当兵,广东当局便出动genkcad 、宪兵到处强拉壮丁,将反抗者活活打死。到1947年3月,兵数仍未征足,而新一年度的征兵任务又已下达,广东当局遂下令将逃兵役者的财产一律充公、拍卖。此外,在灾疫四起、物价飞涨的情况下,广东当局非但不认真救济,反而增收五花八门的苛捐杂税,设置了“筵席税”、“汽车捐”、“冥镪捐”、“牛皮捐”、“特种娱乐捐”等种种匪夷所思的名目。在征粮、征兵、征税的“三征”暴政下,广东经济凋敝,民生困苦至极。1947年后,因联合国救济总署大力赈济,灾情总算缓解,但广东民众仍承受着沉重负担,生活在巨大的痛苦中\footnote{《广东通史》现代下册,页1045—1063}。

在灾情更为严重的广西,新桂系大力赈济,使局势得以逐渐好转。1945年9月15日,广西省政府迁回桂林。广西省主席黄旭初(容县人)为新桂系重要人物,自1930年起便一直在任,对乡邦有很深的感情。当时,广西经连年兵燹一片残破,桂北更已化作一片焦土。在1945—1946年的大饥荒中,桂北一带受灾最重,死亡人数达到惊人规模,有的村庄甚至几乎死绝。据一名记者报道,当时桂北的情形是:

\begin{quote}

我们所到过的桂北乡村,差不多每一家都有一个病人,每一家都有一个作为死亡标识的新灵位。幸而还活着的,除少部分壮年人还有点精神外,老的、少的、幼的大多数气息奄奄,使人有置身地狱之感\footnote{这是笔者根据新桂系的赈济统计估算的。据新桂系统计,全广西共有283万余人得到赈济粮食。由此与灾民数相比较,便可估计死者数量。参见《广西通史》第三卷,页460}。

\end{quote}

在此危局下,黄旭初于1946年1月召开省政府会议,提出“在建设上讲生养,从安定中求进步”的政纲。由于李宗仁、白崇禧皆在岭北,黄旭初的省政府成为广西的实际主政者。同年春,广西省政府提出应对春旱措施,决定在全桂推广良种水稻20万亩、陆稻8万亩、杂粮18万亩、冬季作物200万亩。同年5月,省政府又决定用“以工代赈”的方式修复毁于二战的公路。至1947年11月低,工赈业务结束,全桂共修复公路2981公里,数以万计的灾民得到救济。到1947年,广西灾情已有效缓解,粮食产量大幅上升\footnote{《广西通史》第三卷,页464}。

1945—1947年间,国共矛盾步步升级,终于爆发全面战争。1945年8月,苏联出兵侵占满洲。二战甫一结束,中共军队即大举进入满洲,在苏军的庇护和援助下抢占要点。国民党军只得于11月15日强行攻破山海关,开入南满锦州。同时,双方就日军向谁投降的问题争执不休,于华北爆发一系列冲突。在美国特使马歇尔将军的调停下,蒋中正与毛泽东虽于10月10日在重庆签订《双十协定》,达成和平解决争端之共识,但双方心怀鬼胎,协定形同废纸。1946年4月6日,苏军撤离满洲,将哈尔滨移交中共、长春移交国民党。8日,共军袭取长春。蒋中正随即谴责中共违犯协定,于5月命国民党军在满洲发起进攻。6月26日,蒋中正调集160万正规军全面进攻各地共区,“第二次国共战争”正式爆发。此后,共军各部纷纷抛弃国民革命军番号,改称“中国人民解放军”。至1947年3月,因国民党军在中原、三晋、齐鲁战场接连失利,蒋中正又调整战略,集中兵力重点进攻陕北、齐鲁共区。到9月,国民党的重点进攻也告失败,共军转入反攻。共军的积极活动与苏联的全球扩张息息相关。二战结束后不久,苏联即将其占领下的东欧各国一一变为卫星国,在其中推行恐怖的政治清洗和计划经济。1946年9月,英国前首相丘吉尔发表著名的“铁幕演说”,揭露苏联在“从波罗的海什切青到亚德里亚海的第里雅斯特”间建立“一道铁幕”的行径。然而,由于苏联间谍大量渗入美国高层,美国国策严重左倾,对苏联和中共采取一味姑息的态度。1946年7月29日,美国开始对国民党实施武器禁运,使国际社会失去了抑制中共在苏联支持下逐步壮大的机会。在岭北共区,中共疯狂煽动阶级仇恨,实行“土地改革”,大肆屠杀地主、富农,将其财产分给农民。由失地者组建的还乡团亦随国民党军一起行动,对中共土改干部和分地农民实行以牙还牙的残忍报复。为对抗中共土改、增强自身的合法性,国民政府于1947年12月25日宣布结束国民党一党“训政”,颁布《中华民国宪法》。1948年3月29日,行宪国民大会在南京召开,各省代表于4月20日投票选举蒋中正、李宗仁为所谓的“中华民国第一任总统、副总统”。5月20日,蒋、李在南京正式就职,国民政府改组为“中华民国政府”、国民革命军更名为“中华民国国军”。此次行宪无非是国民党一手操办的政治闹剧。国民党做为破坏法统的革命政党,根本没有资格窃据政权。标榜“国民革命”的国民政府虽已解散,新成立的“中华民国政府”也不过是国民党用于伪装成正常国家的遮羞布。只消看看这一政府在台湾犯下的滔天罪行,此次行宪的实质便一目了然。仅仅是在更为疯狂的中共的衬托下,这次行宪才显得好像有那么一点合法性。

在此期间,南粤的国共两军也进行着激烈的交战。二战结束后不久,广东的国民党军便积极剿共。到1946年初,共军东江纵队主力已被压缩至粤湘赣边境,韩江纵队、珠江纵队被分别赶入大北山区和广宁五指山。1月25日,国共双方就广东问题展开谈判,于5月21日议定将中共武装北撤到山东。6月30日,东江、珠江、韩江纵队骨干2583人在大鹏半岛登上美舰离粤。但随着第二次国共战争全面爆发,共方又于7月决定在南粤恢复“武装斗争”。1947年4月7—14日,中共广西党组织各级代表在横县六秀村集会,议定在全桂发动“武装起义”。5月6日,直属中共中央的中共香港分局成立,以原在粤游击骨干方方(普宁人)、尹平林(江右兴国人)分任正副书记,专管南粤、南洋中共党组织,利用南粤民众对国民党的不满策划暴动。至1947年底,广东的中共游击队已发展至上万人,拥有“惠(阳)东(莞)宝(安)护乡团”、“增(城)龙(门)从(化)博(罗)人民自卫队”、“西江人民自卫队”、“粤东支队”、“潮汕人民抗征队”、“琼崖纵队”等武装\footnote{《简明广东史》,页808}。广西的中共游击队则蔓延至24个县,遍及桂西、桂东、桂中、桂北,攻陷县城两座\footnote{《广西通史》第三卷,页516}。

因罗卓英、张发奎一味聚敛、剿共不利,蒋中正将二人撤职,于1947年11月14日任命粤人宋子文(文昌人)为广州行辕主任,由于18日命其兼任粤省军管区司令,令其掌管广东军政。宋子文系蒋妻宋美龄之兄,在国民党内以亲美著称。宋子文到任后,立即提出亲美政纲:

\begin{quote}

以广州为中心,开发粤省和华南经济,俾使华南农工业能支持国府全面肃清各地共军\footnote{转引自《广东通史》现代下册,页1335}。

\end{quote}
	
此后,宋子文立即与美商洽谈引进美资开发工商、矿业事宜,先后邀请魏德迈、司徒雷登、巴大维等美国亲蒋政要访问广州。1948年3月,在宋子文的努力下,美国宣布实行“华南复兴计划”,开始向广东输送资金。同年下半年,美国经济合作署官员接连赴粤考察粤美经济合作情况,并继续提供美援。宋子文利用美援、侨汇大力投资发展基础设施,尤其注重水利建设。1947年下半年,广东再次发生大规模洪涝灾害。但因宋子文大修堤围、积极疏浚珠三角河道,广东平安度过危机,未再发生大饥荒。1948年10月,宋子文自豪地评价自己主粤一年来的成就:

\begin{quote}
目前省营事业的收入已较一年前增加两倍,实业如堤围工程、增加警保力量、滃江水利建设、南岭煤业支线的兴建、黄埔港等巨大事业需用款项,均系省方筹措\footnote{转引自张维缜、崔占龙:《试论宋子文主粤时期的财政政策——以增加收入与引进资金为中心》}。

\end{quote}

然而,当广东正在美国的援助下逐渐复苏时,来自岭北的空前危机已降临南粤。1948年11月9—11月,中共发动“辽沈战役”,消灭国民党军47万。11月至次年1月,中共又发动“淮海战役”消灭国民党军52万,粤第六十三、六十四军,身在其中,数以万计的南粤子弟在淮河流域的荒野上被共军的人海淹没,成了国民党的陪葬品。1948年11月底,林彪率满洲共军入关参战,与华北共军联合发动“平津战役”。1949年1月13日,共军以绝对优势兵力猛攻天津。次日,天津失守,参与守城的粤第六十二军随13万守军一同烟消云散。1月31日,国民党华北剿匪总司令傅作义打开北平城门,公然投共,华北的52万国民党军全军覆没。至此,长江江北已几乎全部沦为赤土\footnote{关于中共“三大战役”进程,相关研究甚多,此不赘注}。

自1948年起,因法币疯狂通胀,已然形同废纸,“民国政府”于是年8月19日发行新币“金圆券”,要求国统区民众于9月30日前以持有的黄金、白银、外币兑换新币,逾期不兑者严办。根据官定兑换率,1元金圆券可兑300万法币或黄金0.005市两、白银0.33市两。如此严苛的兑换率实与明抢无异。在南粤,金圆券的出现引起巨大恐慌,导致市场更加混乱。在1948年9月,广州物价普遍上涨30\%,每市担丝苗米更涨到令人咋舌的5050万元。因新币不受信任,南粤各地民众千方百计地兑换金银、外币,抢购生活物资而国民党当局又视推行金圆券为头等大事,于10月3日在广州召集1200余名复员军人举行动员大会,杀气腾腾地表示要“毫不客气”地对待囤积金银、外币、物资的绅商,“向豪门开刀”。至年底,在国民党的疯狂劫掠下,广东经济已经崩溃,各地百业萧条,工厂纷纷倒闭\footnote{《广东通史》现代上册,页1181—1185}。广东的中共游击队则趁机发展壮大,开辟“粤赣湘”、“闽粤赣”、“粤桂边”、“粤中”等共区,兵力增至4.6万人\footnote{《简明广东史》,页810}。当年12月,野心膨胀的新桂系趁国民党趁机逼迫蒋中正下野。时在武汉出任国民党华中剿匪总司令的白崇禧两次电蒋,要求停止军事行动,在美、英、苏的共同斡旋下与中共谈判。白崇禧更将李宗仁接到武汉,对其合盘托出自己的计划:

\begin{quote}

这一仗老蒋的老本丢得差不多,再搞不下去了。我们要老蒋下台,德公(李宗仁)上台,和共产党谈和,以长江为界,长江以北让共产党去搞,长江以南由我们来搞……蒋介石(中正)如果仍旧在位,各处都不服,战也难,和也难,所以非蒋让位不可。蒋让位之后就会六军用命,可望南北分治\footnote{转引自《广西通史》第三卷,页544}。

\end{quote}

值此危难关头,新桂系所思者仍非保卫南粤、保卫八桂,反而汲汲于与蒋中正争夺天下,建立由新桂系主导的长江以南大一统国家,与中共划江而治。如此荒谬的幻想自然没有可能成功,因为苏联与中共已决心赤化整个东亚大陆。1949年1月21日,内外交困的蒋中正被迫宣告下野、退居家乡宁波奉化,副总统李宗仁在南京代行总统职务。24日,宋子文随蒋辞职,将广东军权交付与余汉谋、政权交付新任省主席薛岳\footnote{《广州百年大事记》,页633}。2月5日,“民国政府”行政院迁往广州办公,南京仅保留李宗仁的代总统办公室。3月,李宗仁一面写信逼迫蒋中正出国,一面可笑地阻止代表团北上议和。4月13—15日,国共和谈在北平举行。中共首席代表周恩来故意提出国方不可能接受的条件,要求将国民党高层列为战犯,限国方于20日前答复。20日,南京方面回绝中共条件,毛泽东、朱德遂于21日发出“向全国进军”的指令。百万共军发起“渡江战役”横渡长江,于23日攻陷南京。同日,李宗仁在共军进城前夕抛弃职责,乘飞机赶回桂林\footnote{《广西通史》第三卷,页547—549}。

这时,长江以南诸邦已面临生死存亡的关头。国民党尚掌握着约100万残兵,其中粤军已所剩无几,桂系则控制着约30万人,以经历过二战的桂第七、四十八军约5万人为核心武力。4月底,李宗仁、白崇禧、黄旭初、李品仙等新桂系要员在李宗仁公馆集会。李宗仁、黄旭初心灰意冷,在会上一言不发,白崇禧、李品仙则高呼与共军死战到底。会后,白崇禧宣布将逮捕所有言和者,李、白二人至此分道扬镳。5月8日,李宗仁赴广州行使代总统职权,一面高唱要领导国民党与中共“作战到底”,一面开始安排自己的退路。*5、6月间,共军相继攻陷上海、杭州、武汉、南昌、西安。6月17日,白崇禧将其华中军政长官署迁至衡阳,准备与余汉谋的粤军互为犄角,死守粤、桂、湘。7月,由满洲共军编成的“第四野战军”开始猛攻湖湘。8月4日,国民党湖南省主席程潜及中央军陈明仁兵团在长沙投共。至此,尚能防守湖湘的力量就只有宝庆、衡阳一带的桂军了。不战而下长沙后,共军四野在林彪的命令下向湘南推进,其先头部队第四十九军一四六师自恃无敌,猖狂冒进,于15日进至双峰县西部的青树坪镇。青树坪系湘中通往湘南的必经之地,位置险要。白崇禧当机立断,于16日命桂四十八军一七一师、一七二师两翼迂回,围攻孤军深入的共军一四六师。经一日激战,共一四六师弹尽援绝、伤亡惨重,已退缩至以塔子山为中心的数个相邻高地。17日白天,桂军将士奋勇冲击,共军拼死反抗、进行困兽之地。至下午4时,桂军终于攻占共军核心阵地塔子山。然而,在共军疯狂的白刃反冲击下,塔子山又在黄昏时分丢失。午夜时分,被围共军残部趁暗狼狈出逃,在共第一四五师的接应下逃至永丰。据中共在战后公布的数据,在青树坪之战中,共一四六师伤亡877人、一四五师伤亡400余人。这无疑是缩小了许多倍的数字。因为在此战之后,原有上万兵力的共一四六师已丧失战斗力,仅能充作预备队。共第四十九军军长钟伟更在战后深刻检讨,称自己犯了“麻痹轻敌的错误”\footnote{《广西通史》第三卷,页552;《中国人民解放军全国解放战争史》第五卷,页343}。

青树坪之战是南粤军人在铁幕降临前的最后辉煌。它沉重打击了共军的嚣张气焰,使共军最精锐的四野饮下失败的苦酒。此战,八桂子弟兵依靠远不如共军的武器装备英勇战斗,在共军势如破竹之际重挫其凶焰,表明了南粤人誓死反抗的伟大精神。然而,淳朴的桂军终不能抵挡由苏联武装的中共。9月13日,林彪集结四野主力发动“衡宝战役”。在共军如泰山压顶般的猛攻下,桂军顽强抵抗,节节败退。10月9日,桂军全线败退,第七、四十八军的四个师被围,于11日全军覆没,损失4.7万人。16日,桂军残部退入广西,战役结束。与此同时,共军已从江右冲入粤北,南粤危在旦夕\footnote{《广西通史》第三卷,页552}。

1949年8月下旬,中共第二野战军已侵占江右。中共中央命方方自香港赴赣州与南下部队会合,与叶商讨“解放”广东之事。9月11日,在叶剑英的主持下,中共华南分局在赣州举行扩大会议,决定以二野之四兵团、四野之十五兵团及广东游击队共22万人联合发动“广东战役”,由陈赓指挥作战,彻底侵占南粤。当时,广东守军除余汉谋的粤第三十九、一〇九军外,还有从赣、闽逃入南粤的中央军败卒,总兵力约15万人,根本无力防守。10月1日,中共在北平举行“开国大典”,宣布成立“中华人民共和国”。次日,共军兵分三路,开始入侵广东。右路共军第四兵团自湘南汝城出发,攻向粤北曲江、英德、翁源地区,随后在粤湘边境的“粤湘防线”前遭粤军顽强阻击,迟迟不能推进。中路共军第十五兵团则绕过粤军防线,一路击溃中央军抵抗,于11日陷佛冈、12日陷从化。与此同时,左路共军游击队也相继占领虎门、香山、顺德,与中路军合围广州。10月11日,李宗仁见大势已去,再次弃职逃跑、奔赴香港,后出亡美国纽约。“民国政府”要员亦纷纷飞离广州,作鸟兽散。大批民众为躲避共军惊恐地涌上街头,纷纷携老扶幼地试图逃离这座即将落入铁幕的城市。然而,由于赴香港的轮船、客机已被国民党官员征订一空,人们只能与国民党溃兵一同步行逃亡。13日,共军进至离广州市区仅1.5公里处,守军抵抗完全瓦解,残兵败将纷纷向西南撤退。下午6时后,全城所有交通中断、商店关门、对外通讯断绝,仿若鬼城。14日凌晨4时,余汉谋、薛岳逃至城外,在黄埔登船逃往海南。天亮后,残留城中的国民党军开始大举炸毁公共设施,于下午1时20分破坏白云、天河机场。5时50分,难以计数的难民和车辆正拥挤在海珠桥上逃命。这时,国民党军引燃早已放好的黄色炸药,将大桥与大批无辜平民化为粉末。一位曾目睹这一惨剧的广州市民回忆称:

\begin{quote}

“解放”前,我在泰康路靠近海珠广场那边的安记电器店做学徒,1949年10月14日的傍晚,我一个人在看着店铺,大概6点,突然听到“砰砰砰”的几声巨响,店里的玻璃窗都给震碎了,我的耳朵也差点震聋了,接着有很多碎沙石飞进我们的店铺,把我们的木门和饰柜都给打烂了。我赶紧跑到路边,躲在柱子后面看看到底发生什么事情,一看吓一跳,海珠桥被炸断了,又是火光又是浓烟,江面的小艇有的沉了有的翻了,海珠广场那个地方的店铺的门窗都被炸烂了,质量差一点的房子就倒塌了。到处都是受伤的人,有的被压在房子下,有的倒在马路上,到处都能听到恐怖声音,喊声、哭声、救命声……那个场面太触目惊心了,至今仍然历历在目\footnote{转引自南方都市报2014年8月14日报导:《一声爆炸 海珠桥碎 广州陷入黑暗》}。

\end{quote}
	
冲天的爆炸声中,上千平民或死或伤,陈济棠时代的遗产海珠桥化作废铁。与此同时,电路也被爆破,广州城陷入黑暗,在一片末日景象中等待着铁幕的降临。下午6时30分,共军自北郊经大北路进入广州。血红的铁幕,完全笼住了南粤的心脏广州城\footnote{《广州百年大事记》下,页767}。

随着广州陷落,广东境内的国民党中央军完全崩溃,只有粤军还在粤湘边境勇敢地坚守着防线,但余汉谋已经抛弃了他们。24日,共第十五兵团突破粤湘边境,攻陷南雄,粤湘防线瓦解。这时,驻守潮汕的国民党中央军已乘船逃走,潮汕落入中共游击队之手,共军主力则向粤西猛进。26日,国民党军残部4万余人被围歼于阳江。11月4日,共军攻陷罗定、信宜、廉江,“广东战役”结束\footnote{《简明广东史》,页822—823}。 

11月6日,中共四野以精锐的第十三兵团第三十八、三十九军自湖湘洞口、武冈出发,发动“广西战役”。当时,白崇禧手中的能战之兵尚有十二个军共15万余人,粤桂边境及湛江还有余汉谋统帅的4万残军策应。而他们要迎战的共军,则有四野、二野的九个军共40余万人,以及已蔓延到广西90多个县、人数超过4万的中共游击队。开战前夕,白崇禧虽下令在全桂实施总体战动员,发动起数十万民团武装,但这些淳朴的乡土军人如何挡得住列宁主义残暴的战争机器?共军入侵后,桂军与民兵虽英勇迎战,但仍节节败退。10日,共第十二兵团四十、四十一、四十五军越过桂湘边境,袭占全县。15日,共十三兵团经黔南侵入广西,向河池、百色推进。同日,共二野第五兵团攻陷贵阳,封锁了桂军向夜郎撤退的道路。23日,共十二兵团侵占桂林白崇禧为挽回危局,急调五个军自玉林、岑溪发动“南线攻势”,于24日越过粤桂边境进击粤西廉江、茂名、信宜,试图在余汉谋部的配合下打通撤往海南的道路。27日,广东的共军第四兵团发起反击,“南线攻势”溃败。共四兵团遂攻入广西境内,于28日陷容县、北流、30日陷玉林、博白。29日,余汉谋残军一度夺回廉江,但共军随之反击,于12月1日消灭该部。因兵力损耗严重,白崇禧只得将残余桂军主力收缩至南宁至钦州一线。2日,白崇禧自南宁飞赴海南,怯懦地抛弃了自己的部队和乡邦。3日,共军各部发动强大攻势,于4日陷南宁、6日陷钦州、7日陷小董圩、11日陷镇南关。至此,除2万余残部逃入越南外,广西守军全军覆没\footnote{《广西通史》第三卷,页600—609}。尚未落入中共之手的南粤土地,便只剩海南等沿海岛屿了。

自1946年起,国民党便起用赋闲已久的陈济棠,令其出任海南特区行政长官兼警备司令。年近耳顺的老英雄陈济棠虽已久不从政,但仍有一腔保卫南粤的热情。到任后,他以私人资财支援地方建设,立志改变岛上的落后面貌,并组织保安部队积极进剿中共游击队。然而在1949年10月,薛岳、余汉谋率大批败兵涌入海南岛,分任琼崖防御正、副总司令,一意排挤陈济棠,不准他插手军事。心灰意冷之下,陈济棠遂率所部两个师移防榆林港,不再过问军政。1950年初,海南岛上仍有薛岳、余汉谋指挥的五个军共10万国民党军。他们多为自广东败退的部队,士气非常低落。为防御海南,薛岳调集52艘舰艇、飞机40架封锁琼州海峡,试图利用海空优势阻止共军渡海。薛岳对自己的海空防御体系颇有信心,称之为“伯陵(薛岳的字)防线”。然而,中共却已做好偷渡准备。1950年1月2日,中共华南分局提出“一切为着争取海南岛战役的胜利”的口号,迅速征集帆船2666艘、船工12000余人、民工966300余人。2月23日,共军侵占南澳岛,获得了在南粤沿海登岛作战的初步经验。3月5日夜,共四十军一个营分乘14艘帆船从徐闻出发,从南侧偷偷登上海南岛。当时,岛上正活跃着由冯白驹指挥、兵力达1.5万人的中共琼崖纵队。在这支游击队的接应下,该营得以顺利上岸。26日夜,共四十、四十三军各一个加强团再次偷渡,一度被守军发现,后在游击队接应下顺利上陆。4月16日夜7时30分,中共四十、四十三军主力共10万人分乘2000余艘帆船,在近百万民工、船工的配合下铺天盖地地涌向海南岛。守军有限的海空力量无力阻拦多如蝗虫的帆船,只能眼看着他们登岛。17日凌晨3时,共四十军在临高角一带成功登陆。不久后,四十三军亦自东路上岛。两路共军随后会攻美亭,薛岳亦命守军主力增援美亭,双方遂于21日爆发决战。22日,薛岳见共军攻势猛烈,命各部立即南撤,随后乘飞机逃往台湾。共军乘胜追击,于23日侵占海口,又分三路南下,东路于30日陷琼南榆林港、三亚,西、中两路于5月1日陷琼西八所、北黎。在榆林港陷落前,先后有包括余汉谋、陈济棠在内的7万余军民从该港登船撤往台湾。驻守榆林的陈济棠军掩护了这次撤退,使数万南粤百姓成功逃离铁幕。这是陈济棠对南粤的最后一次守护。5月25日,共军又出动上万兵力,进攻珠江口上扼守进出港、澳航线的万山群岛。经逐岛争夺,共军于8月7日击败4000余国民党守军,完全侵占万山群岛各岛屿\footnote{《简明广东史》,页823—825}。就这样,南粤沿海各岛屿全为共军所据,赤旗插遍了南粤的每个角落。铁幕,降临了。

在本章最后,让我们花一点时间,看看陈济棠及新桂系大佬们的结局。1954年11月3日,陈济棠病逝于台北,享年64岁。在有生之年,他再未能看到南粤重获自由。黄绍竑于1949年投共,后在文革初期受到冲击,以剃刀自尽。白崇禧则留在台湾死心塌地地跟随蒋中正,成为国民党高官。李宗仁因曾对蒋逼宫,无法再与蒋共事。因得不到美国支持,他只能坐视蒋中正在台湾拿回国民党大权,心中愤愤不平。1965年7月,在中共统战下,李宗仁飞抵北京公开投共。而此时,南粤百姓正生活在中国治下的无尽黑暗中。背叛八桂父老,投靠国共两党的两人都难逃卸磨杀驴的命运。1966、1969年,白崇禧、李宗仁相继暴死。据说,他们是分别被蒋中正和中共毒死的。这对曾经驰骋疆场的战友,最终为他们的背叛行为付出了代价,这不愧是历史的隐秘公正。

在黑暗中,只有黄旭初散发着微弱的光芒,成为那段苦难历史中的一座自由灯塔。1949年底,黄旭初经海口抵达香港,定居于尖沙咀,此后长期从事反蒋反共事业。因新桂系资金已然散尽,黄旭初几乎只能靠一己之力战斗。他曾于1951年移居日本横滨,后因资财散尽,于1958年回到香港,居住于石硖尾,以在杂志上卖文为生。1975年11月18日,在孤苦贫寒中,黄旭初步陈炯明后尘,在香港九龙浸会医院去世\footnote{关于诸人结局,参见张磊:《孙中山辞典》、王俯民:《民国军人志》相关条目}。盛极一时的新桂系,从此消失在历史的滚滚尘埃中。











