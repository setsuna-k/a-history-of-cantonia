\chapter{百越时代和第一次北属}

\section{百越人:自由的南粤先民}

\indent 三万年前,当现代东亚居民的祖先经缅甸进入东亚时,他们并未想到以后会发生如何的历史活剧。这些人分为两支,其中一支沿长江东下,占据了长江至黄河间广阔的土地。这一批人,是日后华夏的祖先。另一支人则沿西江东进,占据着西起南粤、东至吴越的沿海月牙形土地。他们提供了现代南粤人的大部分遗传基因,是我们南粤的伟大祖先:越人\footnote{徐承恩:《躁郁的城邦》,页30。}。

越人内部分为很多不同的部族,统称“百越”。越人先民拓殖这片沿海月牙形土地的历史无疑是漫长的,亦一定是伟大的。由于缺乏文字记载,我们没有办法了解这一伟大拓殖史的具体细节,但能通过分子人类学的研究勾勒出大致框架:越人首先在相当于今天广东、广西、越南北部的地域形成了南越、西瓯、雒越三族。同时,他们向着空旷无人的东北、正东两个方向不断迁移。东北一路,越人进入江右,形成干越族。其后,他们中的一部分人继续东进到吴越,形成于越族,在宁绍平原和太湖流域创造了辉煌的河姆渡文化\footnote{陈国强:《于越在历史上的贡献与地位》}。正东一路,越人入闽,形成了闽越各部。最后,在这两路越人的交汇点上,形成了位于今日温州一带的东瓯族\footnote{李辉:《百越遗传结构的一元二分迹象》;林蔚文:《中国百越民族经济史》}。在长江以南的地域,百越和湖湘的三苗、滇地的百濮一起构成了族群复杂的多彩世界。

在无文的上古,当中原的华夏先民正处在混沌蒙昧之中的时候,诸越已经创造了灿烂的文明。早在公元前一万年前后,在地中海东岸的人们开始栽培大麦和小麦的时候,三苗和南粤先民开始栽培水稻,形成了东亚最早的农业\footnote{徐承恩:《躁郁的城邦》,页32}。五千年前,当吴越的土地上兴起伟大的良渚文明的时候,粤北山区出现了成熟的农耕社会,南粤的海岸线上就出现了很多以渔猎和采集为生的居民。温暖潮湿的海岸线被这些先民提供了舒适的生活环境和充足美味的海鲜,他们驾驶着原始的木舟出没于海岛之间,用鱼网捞取食物。今天,我们只能通过分布在西起雷州、东至汕尾的漫长海岸线上的无数贝丘遗址凭吊他们曾经的故事\footnote{《广东通史》古代上册,页59}。一个个贝丘遗址中数量庞大的贝壳残留,表示他们的饮食相当健康,能够摄入充足的蛋白质。

与史前时代物质简陋、暴力嗜血的豫中小邦不同,百越人的文明有着低烈度的战争和高度精致的物质、精神生活。南粤各地史前遗址中出土的精致玉器,暗示着南粤先民同良渚文明间的相似\footnote{关于南粤史前遗址中出土的玉器,参见杨少祥等:《广东海丰县发现玉琮和青铜器》}。种类繁多的骨角牙雕器和玉石、水晶饰物则不但表明南粤先民有着远远高于北人的审美水平,亦暗示着先民们时常从事复杂的原始宗教活动\footnote{《广东通史》,页81、93}。爱美、爱和平和敬畏神明是他们的美好特点。这一切,无疑是建基于丰富的物质基础上的。

我们的祖先虽爱好和平,但绝不柔弱。约三千年前,南粤进入青铜时代,铜鼓成为了我们祖先的文明中十分重要的元素。在青铜时代,南粤先民们往往在部落的中心位置放置一面铜鼓。每当危急情况来临,铜鼓就会被敲响,骁勇的部落战士们随即被召集起来,头戴绚丽精致的羽冠,乘坐战船出征\footnote{可参看徐承恩《郁躁的城邦》,页32}。部落间的交战虽不似上古中原小邦间的战争那么残酷,却依然是战士们展现勇武的场所。各部落中没有强大的君权,战士们因为部落共同体而战的使命感和荣誉感将凝结在一起,一如塔西佗笔下的日耳曼初民武士\footnote{对于上古南粤之缺乏君权,参见《吕氏春秋》:“缚娄、阳禺、驩兜之国多无君。皆南粤之夷无君者。”见是书卷20}。

铜鼓文化的范围不仅涵盖了南粤,亦遍及整个东南亚\footnote{梁志明:《东南亚的青铜时代文化与古代铜鼓综述》,《南洋问题研究》2007年第4期}。这一点表明,南粤人从史前时代开始便通过南海和中南半岛、马来半岛、印度尼西亚、菲律宾进行交往。珠海宝境湾的出土岩画表明,在两千至三千年前,南粤先民已能建造船首高翘、有桅杆和船帆的海船。这表明早在无文的时代,自由的南粤先民便已驾驶着帆船航行在广阔的南海上航行\footnote{《广东通史》页135-137}。我们可以想像,在原始落后的技术条件下,他们的航行会有多么艰难。他们是南粤历史上最早的航海者,是无数不被史册记载的达伽马和麦哲伦。

热爱生活、敬畏神明、自由勇武、善于航海的南粤先民有着迥异于上古诸夏的习俗。和所有百越人一样,南粤先民喜食鱼、鳖、蛇、蚌、蛤,还有生食的习惯。在今天的粤菜中,这一特点依旧十分明显。在多雨潮湿的南粤,我们的祖先居住着流行于东南亚的“干阑”式建筑,搭木为屋,楼上居人,楼下饲养牲畜。上古华夏世界的居民们不能理解这种建筑形式,遂荒谬地称南粤人“巢居”。百越人断发椎髻和“雕题凿齿(文身拔牙)”的习俗,亦使南粤先民在外观上和华夏迥异。在遵行周礼的上古华夏世界看来,崇信巫鬼、“越巫”大行其道的百越世界更是令人难以理解的异域。最令北人畏惧的,则无疑是越人“习于水斗,便于用舟”的特性。华夏世界的车战在水网纵横的越土毫无用无之地,丧失了军事技术优势\footnote{《广东通史》页160—162}。很快,入侵南粤的暴秦侵略军将因此付出沉重的代价。

\section{最早的南粤国家:之侯、公师隅、高固}

\indent 公元前333年,楚威王的军队消灭越国,杀害了越君无疆。彼时,诸夏短暂的青春期已一去不返,处在文明盛夏的战国七雄相继变法,纷纷抛弃封建秩序、拥抱集权和总体战。而当时的南粤,却依然是一片自由的越人乐土。越国灭亡后,同为越人的无疆之子之侯逃到南粤,建立起了一个世外桃源般的国度。之侯的具体逃亡路线今已不可考,但能够确定的是,数千年以来吴越、闽越和南粤的越人之间一直维持着广泛的文化乃至人口交流,之侯入粤路线很可能早已有无数先民走过\footnote{《广东通史》页100}。之侯在南粤积极任用本地人处理政事,越人公师隅是他最信任的大臣之一。公元前314年,公师隅在珠江口建造了今日广州城的前身——南武城。随着广州的出现,一则动人的民族神话诞生了:据说,在广州筑城之前,珠江口一带发生了严重的饥荒,民不聊生。危急时刻,南海的空中突然传来优美的仙乐,在五彩祥云中,五名身着五色彩衣的仙人分别骑着五色仙羊从天而降,每只羊口中都衔着“一茎六出”的优质稻穗。仙人将稻穗赠与当地人后,腾空而去。从此,珠江口风调雨顺,成为南粤最富庶的地方,广州城在这里诞生。

这一和广州筑城有关的神话虽有着后世加工的道教色彩,但亦反映了南粤人的一些文明特征和观念。仙羊衔稻穗的故事情节,无疑和百越人先进的稻作文明有关\footnote{刘付靖:《百越民族稻谷起源神话与广州五羊传说新街》,《中南民族大学学报》第23卷第2期}。当上古诸夏的居民还在食用难以下咽的黍、粟时,我们的祖先已经能够享用美味的米饭了。而这一神话将广州称为南粤最富庶之地,又反映了我们的祖先在上古时代即已将广州当做南粤文明的中心。

对于和自己有亡国、杀父之仇的楚国,之侯怀有强烈的仇恨和警惕心理。公元前312年,之侯命公师隅出使魏国,以结成南北夹击楚国的同盟。在魏都大梁,公师隅的使团向魏襄王献上了由三百只船、大批箭支及犀角、象牙组成的丰厚礼物\footnote{“(隐王元年)四月,越使公师隅来,献舟三百、箭五百万及犀角、象牙。”见《竹书纪年注》卷下}。这批礼物不但表明当时的南粤不但有着不弱的军事实力,亦是一片物产丰饶、盛产珍宝的乐土。此后,我们未能在史籍中见到关于之侯政权的记载,亦未见到粤魏反楚同盟采取了何种联合行动。然而,公师隅的这次出使壮举表明,公元前4世纪的南粤已经拥有了和战国强国魏平起平坐的实力,实为当时东亚大陆上一股不容小觑的势力。

在之侯的王国于史籍中消失后,南粤又回到了没有君权的状态。百余年后秦军入侵南粤时,南粤并无一个强大的政权统领全粤进行抗战。事实上,之侯政权本身的集权程度很可能亦不会很强。据说在之侯入粤时,一位名叫高固的南粤人甚至去到之侯的敌国楚国担任丞相。在相楚的五年中,高固曾因楚威王不能尽读《左氏春秋》而进呈简短的《铎氏微》一书予楚王\footnote{胡守为:《岭南古史》,页23-24。关于《铎氏微》一书,司马迁有云:“铎椒为楚威王傅,为王不能尽观《春秋》,采取成败,卒四十章,为《铎氏微》。”见《史记》卷14《十二诸侯年表第二》。}。高固的经历表明,之侯对于南粤的控制力度似属有限,他甚至不能阻止粤人出仕楚国。另一方面,高固助楚王读《春秋》的故事则说明,彼时的南粤已有人能够阅读诸夏的典籍,南粤和华夏已产生了文化交流。

\section{越秦战争和第一次北属}


\indent 公元前221年,暴秦消灭六国,华夏世界进入历史的终结。此时,秦始皇的欲壑却仍未满足,他还想将他的帝国扩张得更大、征服疆界推进得更远。公元前215年,经过六年的准备,秦始皇派遣蒙恬进攻匈奴,令尉屠睢、赵佗入侵南粤\footnote{秦军发动入侵南粤战争的时间,学界说法不一,此处采用胡守为的说法。见胡守为:《岭南古史》,页25。}。五十万暴秦侵略军兵分五路,越五岭(都庞岭、萌渚岭、骑田岭、大庾岭、越城岭)而下,其最西两路首先击溃了西瓯族的抵抗,杀害西瓯君译吁宋。第二年,秦军已全面深入南粤腹地,在珠江口南武城之处、广西北部、越南和广西交界处分别设置了南海、桂林、象郡这三个殖民点。

这批秦军由“逋亡者、赘婿、贾人”组成,乃诸夏封建秩序解体的产物,带有浓烈的流氓无产者气息。秦始皇以这批人侵略南粤,性质和汉武帝令囚徒、寇盗和恶少入侵大宛相同,皆属于大一统帝国将社会矛盾向外转移的秘传心法。流氓无产者大军的纪律一定不会非常好,抢掠很可能是他们士气的主要来源。很快,他们的恶性便彻底激怒了南粤先民们。两万余年来,我们的祖先一直在自己的土地上享受着越人部落固有的古老自由。现在,当暴秦的战争机器在南粤的土地上肆虐时,我们的祖先怎会甘心成为新兴专制的奴隶?于是,一场捍卫家园和自由的伟大抗争爆发了。

在南粤的每一寸山河上,先民们躲进丛林中和秦军周旋。他们宁肯和野兽待在一起,亦要坚持抵抗,绝不做秦人的降虏。为了镇压这些伟大的抵抗者,秦军用尽了一切手段,“三年不解甲弛弩”,仍旧一筹莫展。在河流和群山之间,善于水战的先民们将一处处粤土变成了侵略军的坟场。为了补给日渐吃紧的南粤前线,暴秦不得不开凿了沟通长江流域和珠江流域的灵渠以运送补给,却仍不能挽回败局。经过三年的浴血战斗,南粤先民们终于迎来了胜利。对于这场胜利,史书上的记载惜墨如金。然而,从简短的文字中,我们依然能感受到这场胜利的辉煌、体会到这场胜利带给我们的荣耀:

\begin{quote}
	夜攻秦,大破之,杀尉屠睢,伏尸流血数十万\footnote{司马迁:《史记》卷52《平津侯主父列传第五十二》}。
\end{quote}

\indent 这是一场战果异常巨大的胜利,亦是南粤历史上第一次在抗击外敌的战争中获胜。无论我们用怎样美好的言辞赞美这场胜利,都无法描绘那些南粤战士们的伟大。侵略军主将屠睢被他们击毙了、数十万侵略者命丧南粤。和这场战争时隔两千余年的我们在重温这段历史时,依然能够无比地感动、无比地骄傲。这是古老自由对新兴专制的胜利、是南粤对帝国的胜利。它将永载史册,成为南粤不朽的光荣。

屠睢死后,他的副手赵佗接替了他的位置。暴秦明白,单靠屠杀和镇压是绝无可能成功统治南粤的。于是,大规模的殖民运动开始了,五十万“谪徙民”被暴秦派往南粤“和越杂处”。由于许多在粤殖民者无妻,赵佗向秦始皇请求派遣三万女子南下,而秦始皇却仅派遣了一万五千人\footnote{司马迁:《史记》卷118《淮南衡山列传第五十》}。殖民者中严重失调的男女比例使他们中的许多人只得娶越人女子为妻。在赵佗的主持下,屠杀政策停止了,南粤先民们和殖民者间维持着和平的关系。赵佗由一个侵略军将领转变成了哥伦布式的殖民地经营者。随着通婚规模的扩大、越人女子及殖民者男子的子女不断出生,殖民者们日渐浸淫在南粤本土文化中,变得日益越化,融入了南粤本土的共同体当中。今天,这批殖民者为现代10\%的南粤男性提供了Y染色体\footnote{徐舜杰、李辉:《岭南民族源流史》}。

然而,此时的赵佗仍未忘记自己是一名秦军将领,他仍要为暴秦开疆拓土。他将阴鸷的目光瞄向了更南面。在那里,有一个名叫雒越国的国家。

\section{瓯雒国的灭亡}

\indent 瓯雒国位于今天的越南北部,是一个和古蜀文明关系极深的国度。公元前316年,暴秦灭蜀,古蜀末代君主开明十二世殉国\footnote{秦灭蜀之时间,史家有不同说法,此处采取流传最广的公元前316年说。}。古蜀遗民逃亡滇中,又经红河东下。公元前257年,他们在古蜀王族泮的带领下到达越南北部的红河平原。据说,当时红河平原上的雒越族已有了一个名叫“文郎国”的国家,其首领的称号为“雄王”。雄王绝非集权君主,而是一个典型的部落联盟首领。在雄王之下,被称为“雒侯”、“雒将”的酋长依靠百越的自由传统统率着各部落\footnote{郭振铎、张笑梅:《越南通史》,页122。}。据传说,蛮勇并耽于酒食的雄王因“不修武备”而被泮的军队击败、赴井而死。泮自称安阳王,筑古螺城,在红河平原建立了由古蜀人统治的瓯雒国\footnote{陈重金:《越南通史》,页18。}。

古蜀遗民之所以能顺利到达红河平原,系拜上古巴蜀夜郎——安南印度间的贸易路线所赐。巴蜀夜郎被帝国吞灭后,史后之人已无法相像这条路线的存在。日后的越南阮朝史臣修史时,看到这段古史觉得难以置信,只好认为这些古蜀遗民其实是越南北部一个由蜀姓酋长率领的部落\footnote{阮朝史臣称:“蜀自周慎靓王五年已为秦所灭,安得复有王者……相隔二三千里,蜀安得远跨诸国而跨交郎乎……或者西北徼外与文郎邻有姓蜀者,遂以为蜀王,亦未可知。谓蜀王、又巴蜀人,则非矣。”引自《钦定越史通鉴纲目》卷1,页8。此段文字,实为史后之人史观的绝佳案例。}。然而,对越南北部红河流域东山文化遗址的考古发掘已经证实,该文化所用的陶器、青铜器和古蜀文明极为相似\footnote{张弘:《先秦时期古蜀与东南亚、南亚的经济文化交流》,《中华文化论坛》2009年第1期,页129—131。}。由此可见,古蜀遗民在越南北部建立瓯雒国的历史记载是颇有可信度的。

瓯雒国的国名暗示着安阳王治下不但生活着雒越人,亦有部分西瓯人,该国的控制区很可能深入今日的广西境内。安阳王手下的“强兵勇将”有三万之众,其军事实力无疑高于文郎国\footnote{郭振铎、张笑梅:《越南通史》,页132。}。于此同时,红河平原上“雒侯”、“雒将”的权力亦未遭到破坏。古蜀遗民作为征服者,很有可能将红河平原变为了混合蜀越宪制的区域,一如丹麦人入侵英国后制造的格局。 

公元前207年,赵佗发动了对瓯雒国的入侵。对于这场战争的进程,史书进行了神话式的记载:传说,早在安阳王筑古螺城时,曾获得一只金龟的帮助。金龟离去时,脱爪一只赠予安阳王,以助其抵御外侮。安阳王以龟爪为弩机,造出了一部名为“灵光金爪神弩”的武器。赵佗军来攻时,安阳王以神弩抵抗,“三放杀三万人”。赵佗见强攻失利,遂和安阳王议和,送其子赵仲始于雒越国中为人质。很快,安阳王的女儿媚珠和“风姿闲美”的赵仲始便坠入了爱河\footnote{《钦定越史通鉴纲目》卷1,页18。}。在赵仲始的诱惑下,媚珠鬼迷心窍地带他查看了神弩。其后,赵仲始偷走了金龟神爪,并托以省亲为名北归。失去了神弩庇护的安阳王再也不能保护自己的国家。在赵佗军的又一次入侵下,瓯雒国灭亡了\footnote{陈重金:《越南通史》,页133。}。

赵仲始和媚珠的爱情故事,则有一个凄美的结局。赵仲始北归之前曾问媚珠:“异日我再来,万一两国失和,当作何等验质可见?”媚珠答道:“妾有鹅毛绵褥,常以附身。所至歧路,拔鹅毛识之,可知妾所在。”古螺城破后,安阳王乘马载媚珠南逃至海滨,走投无路,遂在拔剑斩杀媚珠后投海殉国。稍后,跟随鹅毛踪迹领追兵赶到海边的赵仲始只见到了媚珠的尸体,伤心欲绝,便带着媚珠之尸回到古螺城,投井自尽\footnote{《钦定越史通鉴纲目》卷1。}。

这一神话中,安阳王筑城得金龟之助的故事,和秦军灭蜀后筑成都时得到神龟帮助的传说非常像,无疑暗示了瓯雒国的古蜀渊源\footnote{关于成都筑城与龟之间的关系,可参看王文才:《成都城坊考(上)》,《四川师范学报》(社会科学版)1981年第1期,页58—66。}。安阳王拥有令赵佗畏惧的神弩,则似在暗示古蜀人的技术水平要比赵佗麾下的秦军高一个档次,而这种技术差距极有可能是由印度安南——夜郎巴蜀这条通道输入的。赵佗以卑鄙手段破坏神弩,无疑象征着暴秦的凶残和狡诈。媚珠和赵仲始的悲惨结局,更给瓯雒国和古蜀文明的最后灭亡抹上了浓重的悲剧色彩。虎狼之秦惩其凶焰,在南粤和越南的土地上演了其第二次灭亡大蜀的暴行。

然而,赵佗毕竟不是屠睢。消灭瓯雒国后,雒侯、雒将们固有的地位、雒越人固有的自由和共同体并未遭到破坏。此时秦始皇已死,由陈胜吴广掀起的滔天洪水在秦帝国的核心领土上肆虐。赵佗就如同英国历史上那些失去欧陆领土的诺曼武士后代一样,坚定地走着本土化的道路。一年后,他将在南粤大地上创造惊人的伟业、开创一个伟大的南粤国家。
