\chapter{从独立到沦亡:1911—1926年的南粤}

\section{三次独立的流产:1911—1916年}

\indent 1911年11月7日、9日,广西、广东相继独立。同月,有上海、贵州、浙江、江苏、安徽宣布独立。同月,清廷颁布《十九信条》,承诺推行英式君主立宪制,并以袁世凯为责任内阁总理,试图收回人心,但已无济于事。统领着北洋军的袁世凯视清廷为奇货,欲以之为筹码与独立各省讨价还价。12月1日,江浙联军攻克南京,清江苏巡抚张勋率败兵弃城北逃。至此,清帝国已彻底失去对长江以南的控制权,灭亡只是时间问题。由于各省独立军多由同盟会、光复会、华兴会一类的革命派把持,立宪派亦对乡邦的真正独立缺乏兴趣,建立大一统的共和政体已成定局。12月21日,孙文自美国抵香港,广东都督胡汉民迎之。胡汉民希望孙文留粤练兵,孙文却急于北上建立新的大一统政府。24日,胡汉民随孙文北上,陈炯明成为广东代理都督。29日,粤、桂、奉、直、豫、鲁、晋、陕、苏、皖、赣、闽、浙、湘、鄂、川、滇等十七省代表齐聚南京,选举孙文为临时大总统。1912年1月1日,孙文于南京就任中华民国临时大总统,中华民国成立,以当年为民国元年,广东、广西成为中华民国之两省,南粤在民国时代的首次短暂独立宣告终结。2月12日,经袁世凯的北洋军逼宫,清宣统帝逊位。孙文投桃报李,对民国临时参议院辞职,荐袁自代。15日,临时参议院选袁为临时总统。3月11日,参议院在南京通过《中华民国临时约法》,规定建立模仿法国的责任内阁制,欲以此限制总统袁世凯的权力。4月1日,孙文正式解除总统职务。次日,民国临时政府迁往北京,局势暂告稳定\footnote{《广州百年大事记》,页131—133;郭廷以:《近代中国史纲》,页282—287}。虽然东亚大陆各邦合并为大一统的中华民国,错失了各自独立的机会,但《中华民国临时约法》毕竟hay 独立各省代表共同认可的宪法性文件。因此,初立的中华民国虽是禁锢各邦自由的大一统国家,但对各邦而言仍具有一定程度的合法性。

广东、广西军政府成立初期,颇有一番刷新振作的朝气。军政府下令革除清帝国的剪辫、跪拜习俗,废除官吏的“大人”、“老爷”等称呼,一律改称“先生”。因张鸣歧逃亡香港时将广东库银席卷一空,广州城内人心不稳。广东军政府乃宣布承认一切原有银票,并发放库存纸币1200万元,稳定了经济和物价。然而,广州城内尚有一更令军政府头疼的问题,那便是四处横行的各路军队\footnote{《简明广东史》,页562—564}。广州光复后,城中不但有各路被称为“民军”的革命军,尚有被称为“反正军”的前清八旗兵、绿营兵、巡防营、水师、新军等。民军自视为革命功臣,视反正军为降虏,反正军则视民军为“草泽绿林”,双方冲突不断。胡汉民就任都督后,令旗兵负责城内治安,新军、水师负责城外治安。至于龙济光所部新军则改称“济军”,被调往粤西钦州、廉州一带清剿土匪。1911年11月下旬,广东军政府组织起一支由同盟会员、会党成员、新军、粤侨青年组成、拥有8000兵力的北伐军赴南京前线作战。广东北伐军于12月8日开拔,兼程北上,一路攻到徐州,将清军张勋部逼退至济南\footnote{《简明广东史》,页564}。

胡汉民虽已将对民军威胁最大的济军远调,但他对鱼龙混杂的各路民军着实无计可施。涌入广州的民军中多有被同盟会发动的会党及投机分子,他们人数多达十万,分为五六十股,在城内整日寻衅滋事、劫掠商民、虚耗粮饷,成为财政和治安的大患。胡汉民虽任命黄世仲(番禺人)为民团总局局长统领各路民军,然黄世仲乃文学家出身,不通军略,难以约束各部。1911年12月24日,陈炯明出任代理都督,开始以强硬手段整编军队\footnote{《简明广东史》,页565}。

陈炯明,字竞存,福佬人,于1878年1月13日出生于惠州府海丰县白町乡的一个士绅家庭。他四岁丧父、八岁丧祖父,经历了家道中落。但年幼的他依然刻苦读书,于1898年考中清帝国秀才。1905年,清帝国废除科举,陈炯明亦将眼光转向新知识,于1906年入读广东法政学堂,为该校第一届学生。1908年7月,陈炯明以“最优等生”成绩毕业。年届而立的他颇有一番振兴家乡的抱负,遂回乡一年,积极投身于倡办海丰地方自治会、戒(鸦片)烟局等社会工作,赢得了家乡父老的信任与尊重。他在家乡目击到清帝国官吏严酷扰民、贪赃枉法的种种弊政,因而开始萌生反清革命思想。1909年8月,陈炯明当选广东谘议局议员,随即在广州与革命党人取得联系,正式加入同盟会。在1910年的广州新军起事中,他以议员身份为掩护积极参与密谋,事后未被清帝国当局察觉。在1911年4月27日的“广州起义”中,他更随黄兴一同冲杀街头,失败后逃亡香港。不久后,1911年独立战争爆发,陈炯明获得南洋粤侨资助,回到海陆丰组织了自己的武装循军,一举攻克惠州。广东独立后,各路民军群聚广州,以陈炯明的循军为最强,广州绅商代表亦认为陈炯明有平息广州乱局的能力。因此,陈炯明是在绅商们的期待声中上任的\footnote{关于陈炯明早期生平,详见吴相湘:《民国政治人物》,页90—93}。

陈炯明的整编计划,是以循军和新军为骨干组建广东陆军,遣散大部分民军及反正军。1912年2月,陈炯明设广东陆军司令部,自任军统。2、3月间,他开始行动。当时,诸路民军以“石字营”纪律最为败坏。石字营统领石锦泉本为一石匠,兼营理发,曾以头发包运军械接济革命党,自认有功于革命。在独立战争中,他纠集起一批无赖和贫民组成石字营,自称“民军”,于开入广州后四处横行,杀害昔日与他有私怨者。听闻陈炯明要整编军队,石锦泉携炸弹闯入都督府索饷,被陈炯明的卫兵捆起,当即枪毙,其石字营旋被解散。数日之内,被裁撤官兵已超过万人,引起诸军强烈不满。3月9日,统领“惠军”的同盟会员王和顺在广州城内起兵反抗,陈炯明派兵攻败之,广州一度戒严。到3月底,整编已在陈炯明的强硬态度下基本完成。新编的广东陆军以陈炯明为军统,下设第一师、第二师及混成旅。遭裁撤者则多被编为警卫军和工兵,分防广东各地\footnote{宋其蕤、冯粤松编著:《广州军事史》,页154;《广州百年大事记》,页132—133}。

陈炯明的裁军行动固然属保境安民的义举,但他不问各军战功一律裁撤的行动未免失之武断。黄世仲对陈炯明的武断行动大加反对,认为应“裁弱留强”,合理整编。然而,陈炯明为竖立自己的威信,居然痛下杀手,于4月9日以“图谋不轨”的罪名逮捕黄世仲,未经合法审判即于5月初草草枪决\footnote{《广州百年大事记》,页133}。陈炯明无视政治规则的草菅人命无疑开了一个极坏的恶例,这是他政治生涯中的巨大的污点。不久之后,他将咽下自己酿造的苦酒\footnote{《中国历代著名文学家评传》第九卷,页620}。

4月25日,因“南北一统”告成,胡汉民回到广州重任都督,陈炯明仍任军统,掌握广东军权。此时,广东的议会政治已步上轨道。在1911年12月初举行的临时议会选举中,广东军政府规定凡年满21岁、有广东籍或外省籍在广东住满五年者皆有选举权。同年12月5日,由全省各团体选出的140名代议士群聚广东总商会,临时省议会开幕。1912年3月4日,临时省议会解散,广东开始筹备正式的省议会选举\footnote{叶曙明:《国会现场:1911—1928》,页16—17}。自袁世凯出任总统后,同盟会骨干宋教仁力主进行议会斗争。同盟会旋即进行改组,放弃了“平均地权”这一左翼色彩浓厚的政纲,力图吸引党员,并以孙文为总理、宋教仁为干事,其中宋教仁为实际负责人。1912年8月,同盟会改组为国民党,成为名义上的议会政党。做为国民党骨干,胡汉民独断地采取“非同党不用”的原则委任干部,国民党员几乎占据了广东各厅、司、局的全部领导职位,使广东军政府沦为国民党的白手套。1912年底,广东举行正式议会选举。胡汉民动员国民党员广泛参选,并卑鄙地采取指派选举监督人、指定候选人的方法操纵选举,使国民党获得大胜。1913年初,拥有120名议员的广东省第一届议会成立,其中约半数为胡汉民一派的人。随后,省议会选出10人为中华民国国会参议员,其中8人为国民党员\footnote{《广东通史》近代下册,页373—377;王鸿鉴:《清末民初的广东议会政治》}。这些不正常的现象表明,此次议会选举虽在名义上具有全民性,实为国民党玩弄的政治游戏,这自然引起南粤绅商的强烈不满,使军政府和国民党日渐丧失人心。

在广西,议会政治的运行情况更加糟糕。独立伊始,广西谘议局即自动转化为省议会。1911年11月8日,即广西独立的第二天,省议会颁布《广西临时约法》,规定根据三权分立原则组织广西政府。1912年2月,广西省议会在桂林正式开会。然而,《广西临时约法》不过是一纸空文,对都督陆荣廷毫无约束力。陆荣廷虽为广西武鸣壮人,但他对家邦的宪法毫无维护之意。他本为天地会成员,后被清军招安,乃一以南宁为老巢、野心勃勃的军阀。他不顾三权分立原则,肆行武断之治,于4月10日在南宁召集部分议员擅自召开“广西省临时议会”,使广西出现了两个议会并立的情形。7月25日,陆荣廷在南宁正式召开省议会。10月17日,他又将省会由桂林迁至南宁\footnote{《广西通史》第二卷,页598—603}。在民国成立后不到一年,广西的议会政治便已被破坏得体无完肤。

在1913年初的中华民国国会大选中,国民党大获全胜,远远超过黎元洪、梁启超领导的第二大党进步党,于国会获压倒性多数席位。宋教仁意气风发,准备以党魁身份组阁。同年3月20日,民初政治史上的惊天巨变发生,宋教仁在上海遇刺,于两日后身亡。主持刺杀宋教仁者究竟是谁?对这一难题,史家历来聚讼不已,主张为袁世凯、孙文者皆有之。无论如何,孙文通过宋案获得了重新掌控国民党、武装夺权的机会。4月16日,汕头国民党机关报《大风日报》发表《万恶政府》一文,激烈攻击袁世凯破坏共和。5月1日,胡汉民于广州通电反袁。6月14日,袁世凯针锋相对地免去胡汉民的都督之职,以陈炯明代之。同月,国民党籍江西、安徽都督李烈钧、柏文蔚亦被罢免。孙文终于获得开战理由,遂指示李烈钧起兵。7月12日,李烈钧于江西湖口举兵反袁,江西宣布独立,所谓的“二次革命”爆发\footnote{《简明广东史》,页508—509}。随后,南京、安徽、上海在国民党的策划下纷纷独立。7月18日,陈炯明来到广东省议会,要求议员立即宣布独立,结果不但绅商出身的议员表示反对,就连国民党议员也无人赞同。议员们一致认为,民国的议会政治刚刚步入正轨,国民党岂能在此时起兵破坏宪政?陈炯明见此,便杀气腾腾地拔刀怒喝道:

\begin{quote}

本都督一人亦要独立!

\end{quote}

议员闻此,皆相顾失色,只有表示顺从,陈炯明遂于当夜8时宣布广东独立,自任广东讨袁军总司令。此次行动虽有“独立”之名,然实为国民党武装篡夺中华民国政权的一步棋,与南粤的独立和自由毫不相干,因此注定无法得到粤人的真正支持,从一开始就注定了悲剧的结局。陈炯明以武力绑架议会的举动,实为对南粤宪制的蛮横破坏,应予以彻底否定。当时,广东陆军中的师、旅级军官皆对独立不抱热心,许多人已在贿赂下与袁世凯暗通款曲。7月22日,袁世凯下令进攻独立各省。26日,袁世凯任龙济光为广东镇抚使,令其进攻广州,南粤大地战祸再起。8月3日,龙济光又被袁世凯任命为广东都督、陆军上将。同日,济军顺江而下占领三水。至此,广州已危在旦夕\footnote{刘仲敬:《民国纪事本末·癸丑篇》;《广州百年大事记》,页141}。

早在1913年初,袁世凯为牵制广东的国民党势力,便曾命龙济光的济军移驻西江上游的广西梧州,并指使陆荣廷向济军提供饷械。济军在梧州扩编为十三营,对广州虎视眈眈。龙济光系大滇蒙自人,乃一无爱于南粤的外来军阀,野心极大,其部下亦多为滇人\footnote{王爱云:《龙济光政府与民初广东社会研究(1913—1916)》}。然而,陈炯明却大意轻敌,对龙济光的威胁不以为意。8月4日,陈炯明将广东陆军编为三个支队,准备出师援赣。同日,早已里通袁世凯的第二师师长苏慎初于广州东郊燕塘起兵叛陈,炮击都督府,要求取消独立。慌乱之中,陈炯明逃往香港、转赴新加坡,粤军遂公举苏慎初为临时都督。次日,军方又举混成旅旅长张我权为都督,宣布取消独立。然而,袁世凯没有放过南粤。8月7日,袁世凯宣布广东省议会追随陈炯明“倡乱”,要求龙济光出兵解散议会。粤军至此才明白,他们刚刚赶走了宪制破坏者陈炯明,现在即将迎来远更残暴的龙济光。8月11日,龙济光率6000余济军兵临广州城下,要求粤军退出。万余粤军誓死不从,决定武装反抗。然而,当时苏慎初与张我权正陷入严重内斗,粤军的指挥系统颇不统一。13日晨,龙济光下令进攻。至下午2时,,城中军民死伤枕籍,粤军除溃散者外皆向龙济光投降\footnote{王爱云:《龙济光政府与民初广东社会研究(1913—1916)》;《广州百年大事记》,页142}。就这样,龙济光用血腥的屠杀控制了南粤的心脏广州城。随着南粤沦入外敌之手,粤人的灾难降临了。

9月1日,袁军张勋部攻占南京,国民党闹剧般的“二次革命”宣告失败。11月4日,袁世凯发动政变,取消国会中的国民党籍议员资格。至此,国会因人数不足无法议事,袁世凯成为实行无国会统治的僭主,中华民国脆弱的法统就此中断\footnote{刘仲敬:《民国纪事本末·癸丑篇》}。在广东,龙济光开始为所欲为,犯下了远比胡汉民和陈炯明更为深重的罪行。龙济光进入广州后立即解散了省议会,大批捕杀国民党籍议员,并将广州27家报纸中的23家封禁,其主编“或饮枪弹,或陷囹圄”。为维护自己的残暴统治,龙济光将其所部扩编为四个旅共38600人,占中华民国总兵力的8\%。仅1914年,济军的全年军费即高达1102万元,广东百姓为此不得不负担沉重的税赋。在扩编军队的同时,龙济光又大肆清洗粤军,不分青红皂白地将有“党人”嫌疑的官兵全部关押杀害。此外,他还在广州越秀山上耗资3000万元建造华丽的住宅,在粤人的尸骨上过着穷奢极欲的生活\footnote{《简明广东史》,页593—594}。

面对龙济光的残暴统治,粤人自然不会坐以待毙。1914年3月10日,粤东饶平县黄岗驻军营长吴文华起兵宣布独立。4月28日,梅州驻军营长王国柱自称“广东讨龙军潮梅总司令”,发动一团人马反抗。这两次英勇的起义都惨遭济军血腥镇压。同年3月,龙济光以搜捕“党人”为名,派济军统领李嘉品率部到顺德上淇、敦教等村大举“清乡”。济军一路烧杀淫掠,使当地十室九空,受辱妇女纷纷投河自尽,情形惨不忍睹。在广州,“济军来咗”成为恐吓哭闹幼童的口头禅,佛山市民则传唱着“期来龙者却来蛇”的民谣以表达对龙济光的切齿痛恨。英国驻广州领事在对伦敦方面的报告中用一句简短有力的话概括了龙济光的黑暗统治:

\begin{quote}

龙济光治下的粤省情势是革命以来最坏的\footnote{转引自《广东通史》近代下册,页428}。

\end{quote}

“二次革命”失败后,孙文逃至日本东京,于1914年7月将国民党改组为“中华革命党”。该党以会党模式重建组织,强调对“领袖”个人的忠诚,要求党员以按手印的方式向孙文效忠,以推翻袁世凯、“实行民权、民生主义”为宗旨,实为孙文策动武装夺权、实现其个人野心的工具。黄兴、陈炯明、李烈钧等理想主义色彩浓厚的党内骨干认为孙文之举已背离民主,拒绝加入中华革命党,另组“欧事研究会”,许多国民党员亦与孙文脱离关系。至当月底,中华革命党仅有党员692人,其中广东籍者78人,沦为一小型秘密团体。10月,孙文遣邓铿、朱执信(番禺人,吴越移民后代)分赴香港、澳门设立机关,策划反袁讨龙行动。邓、龙二人经会商后,决定由邓铿负责策反惠州、潮州、韶关、增城、龙门、香山、江门等地的军队,朱执信负责发动珠三角及粤西的会党、土匪起事。该月下旬,增城、龙门、惠州驻军经邓铿策动相继起事,皆被济军迅速镇压。11月上旬,新会、顺德、香山、阳江等地在朱执信策动下有数千人起事。11月10日,朱执信亲临前线,指挥3000余人连攻佛山四日而不胜,深陷重围,只得将革命军解散。16日,高州、电白又有3000余人起事,于次日攻破电白县城,坚持十余日而败。广州城内准备响应的部队则未及起事便遭龙济光查获,数十人被捕杀。至此,革命党精心策划的暴动全盘失败。次年7月17日,又有革命党人钟明光以炸弹袭击龙济光,炸死其卫队17人、伤龙济光左足,被捕身死\footnote{《广东通史》近代下册,页431—434}。在革命党接连行动的刺激下,龙济光接连发动屠杀,成百上千的无辜者被指为“革命党”惨遭毒手。据龙济光的部下称:

\begin{quote}

我们大帅最恨革命党。只要被他抓着,总不会有活命的,并且要受尽酷刑而死\footnote{转引自《广东通史》近代下册,页427}。

\end{quote}

与此同时,袁世凯正在僭主的道路上越走越远。1914年5月1日,袁世凯推出所谓的《中华民国约法》,取消《临时约法》中规定的内阁制,代之以总统制,将内政、外交大权收归总统之手。1915年8月14日,杨度、严复等曾在清末时积极主张立宪的学者组织“筹安会”,支持袁世凯公开恢复帝制、实行君主立宪。9月3日,曾以其进步党鼎力支持袁世凯的梁启超发表《异哉所谓国体问题者》一文,激烈抨击帝制。然而,袁世凯仍执迷不悟。12月12日,袁世凯在其御用机构“参政院”的“推戴”下悍然接受帝位。21日,龙济光被袁世凯“册封”为“一等公”。31日,袁世凯命改1916年为“洪宪元年”\footnote{刘仲敬:《民国纪事本末·僭主篇》}。至此,袁世凯已彻底抛弃中华民国,成为千夫所指的独夫。 

1915年12月25日,在梁启超的策划下,蔡锷、唐继尧与李烈钧联手,宣布云南独立,组织“护国军”讨袁以捍卫中华民国国体。唐继尧出任云南都督,“护国战争”爆发。与此同时,陈炯明自南洋潜回东江,积极联络旧部,于30日打出“广东共和军”旗号,下令通缉龙济光。1916年1月5日,共和军攻克河源、龙川。次日,攻克博罗。同日,陈炯明在惠州淡水誓师讨袁。接着,共和军进围惠州,与济军展开激战。与此同时,孙文的中华革命党亦在粤西展开行动。孙文任命朱执信为“中华革命军广东司令长官”,于1月6日起兵于电白。中华革命军“或沟通军队,或联络民众”,很快发展到上万人,于2月8日由朱执信亲率4000人进至广州近郊,遭济军突袭,坚持至次日败退\footnote{《广东通史》近代下册,页438—440}。当时,大批济军正在袁世凯的命令下西进“征滇”,广东境内防守空虚,故革命军能迅速发展。

1916年1月4日,袁世凯下达“征滇”令,组织北、中、南三路共十万大军分别自四川、湖南、广西进攻云南。其中,南路军司令系时任广惠镇守使的龙济光之兄龙觐光,有兵力万余。济军欲入滇,必然假道广西。因陆荣廷与龙觐光为儿女亲家,济军得以横穿广西,设司令部于桂西百色\footnote{《广东通史》近代下册,页435}。1月16日,蔡锷率护国军主力北进,攻入贵州。27日,贵州倒向护国军,宣布独立\footnote{刘仲敬:《民国纪事本末·僭主篇》}。2月中旬,龙觐光趁护国军主力北伐之机攻入滇南,攻占重要关隘广南。3月9日,护国军发动反攻,夺回广南。11日,正在两军于滇南激战时,陆荣廷突然指挥桂军反戈一击,局势为之大变\footnote{《广东通史》近代下册,页436}。

在复杂的政治斗争中,陆荣廷一贯善于见风使舵。1912、1913年间,当国民党成为国会第一大党时,koy 曾带着投机心理加入国民党。1913年9月12日,国民党员刘古香、刘震寰等响应“二次革命”,在柳州实行独立,宣布讨袁,于两日后被陆荣廷镇压,刘震寰化妆出逃,刘古香被俘杀。其后,陆荣廷又将《广西临时约法》作废,下令解散广西省议会,并积极配合袁世凯的帝制自为,完全背离了国民党。1916年1月,陆荣廷听说袁世凯与北洋宿将冯国璋、段祺瑞等产生了严重裂痕,乃决定再次转换立场。3月7日,koy 率军从南宁北上,做出一副要进攻贵州的样子,实则谋划着广西的独立反袁之事。此时,梁启超已间道入桂,表示将对陆荣廷鼎力相助。11日,陆荣廷进至柳州,突然下令桂军从后方进攻龙觐光部,将其全部缴械。15日,陆荣廷、梁启超通电宣布广西独立讨袁,陆任广西都督兼广西护国军总司令、梁任都参谋。很快,滇、桂护国军兵分两路,经梧州各自攻至封川、清远,于清远大破济军。4月初,又有进步党人徐勤(三水人)于港澳组织“广东护国军”,自任总司令。很快,因停泊于珠江上的十余艘粤军军舰加入护国军,使之实力大增,护国军得以攻克顺德。4月5日,护国军军舰驶至广州白鹅潭,声言开炮攻城。此时,广东的94个县中已有54个脱离龙济光的控制,龙济光拥万余残兵困守广州,政令无法出城。面对护国军的军舰,龙济光只有妥协,于次日以“保护人民治安”为由宣布独立\footnote{《广东通史》近代下册,页443}。自民国建立以来,这已是广东的第三次独立了。

做为一个残忍的外来侵略者,龙济光宣布广东独立绝非为南粤谋福,仅为自保。当时,袁世凯因护国军日渐壮大,已于3月23日取消洪宪年号、恢复民国五年。因袁世凯已自身难保,龙济光唯有自谋出路。宣布独立后,龙济光保持着暧昧态度以观望局势,只字不提讨袁,还设计了一个诱杀护国军高层将领的鸿门宴。4月12日,徐勤率护国军代表与龙方代表于海珠岛水警署展开谈判,讨论双方的进一步合作事宜。会议开始后,龙方代表无理地要求将护国军编入广东警卫军,被护国军代表汤觉顿(番禺人)拒绝,双方陷入激烈争执。此时,龙军突然在会场开枪,当场杀害汤觉顿等护国军代表四人,徐勤仅以身免,是为“海珠惨案”\footnote{《广东通史》近代下册,页443—444;刘仲敬:《民国纪事本末·僭主篇》}。

惨案发生后,护国军及各路革命军纷纷谴责龙济光,要求将其武力驱逐,龙济光连忙遣使赴梧州向陆荣廷、梁启超求情。陆、梁二人急欲联龙抗袁,遂向龙提出惩凶、警卫军调离广州、整顿济军、解散侦探等条件,龙济光全部接受。4月15日,陆、梁率军进抵肇庆,要求龙济光即刻出兵北伐,以岑春煊代其为广东都督。龙济光仍欲盘踞广州,遂亲赴肇庆谈判。4月19日,双方达成妥协,议定于肇庆成立两广都司令部,推岑春煊为都司令统领两广军务,龙济光仍为广东都督。5月1日,两广都司令部成立,岑春煊如约出任都司令、梁启超任都参谋,下统粤、桂、滇护国军共七个师又三个混成旅。岑春煊系广西西林壮人,本为前清高官,清亡后避走上海,于二次革命中支持国民党。陆、梁以岑资格较老,故推其为统帅,然实际军权仍操于陆荣廷之手。8日,在梁启超的主持下,中华民国军务院于肇庆成立。军务院系独立讨袁各省的最高权力机构,以云南都督唐继尧为抚军长、岑春煊副之,下设抚军八人,包括陆荣廷、梁启超、龙济光、蔡锷等。因唐继尧其时尚在云南,乃由岑春煊摄抚军长之职。军务院以恢复《中华民国临时约法》为宗旨,否认袁世凯政权之合法性,以合议制度“裁决庶政”,实为中华民国的另一政府。军务院成立后,当即决定出兵北伐,向武汉进攻\footnote{《广东通史》近代下册,页439—440}。

6月6日,僭主袁世凯在一片怒骂声中病死于北京,遗命遵1914年之《中华民国约法》,举副总统黎元洪代其为大总统。黎元洪就职后,宣布支持恢复《临时约法》,然国务总理段祺瑞却通电支持《约法》、反对《临时约法》。段祺瑞的暧昧态度令龙济光自以为找到了靠山。9日,他突然宣布取消广东独立,并指示粤北济军堵截护国军北上。18日,云南护国军路经韶关,遭济军枪击,随即开火还击。次日,济军败北,护国军占领韶关\footnote{《广东通史》近代下册,页448}。

韶关之战彻底激怒了护国军和南粤人。广东各界人士纷纷致电北京,要求立即罢免龙济光。7月6日,段祺瑞只得在压力下表态,任陆荣廷为广东督军,改命龙济光督办两广矿务,试图在讨好护国军的同时保住龙济光。护国军已经丧失了耐心,旋即于7月下旬自东、北、南三面进攻广州。龙济光困守孤城,负隅顽抗,两军于西郊泮塘、米埠一带激烈交火,当地居民逃散一空。8月1日,支撑不住的龙济光表示他同意率部撤往海南岛,但必须要先取得250万元的善后款。护国军不给他喘息的机会,继续进攻。至11日,滇、桂护国军已控制江门、佛山、顺德、新会等重镇,龙济光的战败只是时间问题。同日,北京政府为保住龙济光的性命,下令双方停火,并派员南下监视龙济光与陆荣廷的交接,广州城外响彻了半个多月的枪炮声终于平息下来\footnote{《广东通史》近代下册,页449}。10月6日,在向广州商民搜刮了90万元的资财、完成最后一次掠夺后,龙济光带着残部垂头丧气地离开他霸占了三年多的广州,乘船前往海南。陆荣廷率领的滇、桂护国军随即以胜利者的姿态开进广州,取得对南粤的控制权\footnote{《广州百年大事记》,页152}。龙济光对广东的残酷蹂躏,终于结束了。

在民国建立的头五年,南粤曾有过三次短暂的独立。然除1911年的两广独立有南粤人摆脱清帝国统治、争取自由的正义性质外,其余两者均非真正的独立。1913的广东独立和1916年的两广独立分别是国民党和龙济光、陆荣廷为满足其政治目的玩弄的手段。1913年的广东独立是由陈炯明违背南粤绅商意愿、拔刀威胁省议会而促成的、1916年的两广独立则是龙、陆二人实现其个人野心的工具。中华民国系一大一统国家,做为该国两省的广东、广西是不可能在拥护它的前提下获得自由的。国民党的“二次革命”和袁世凯在1913年11月发动的政变中断了中华民国的法统,该国政府在东亚各邦曾一度存在过的有限合法性荡然无存。在这样的情况下,1912、1916年的两次独立到底有着怎样的实质,也就可想而知了。

\section{桂系据粤:1916—1920年}

\indent 民国史上,所谓桂系指广西军阀,分为陆荣廷的“旧桂系”与李宗仁的“新桂系”。自1916年10月陆荣廷率桂、滇护国军进入广州起,旧桂系盘踞广东达五年之久。在漫长的历史上,粤、桂一同构成了南粤。然而,在19世纪发明民族的历史进程中,广东形成了由广府、客家、潮汕构成的三族共同体,并由此产生独特的广东认同,日渐演化为与由广府、壮人、桂柳三族构成的广西不同的共同体。桂东南的广府人虽然在心理上与广东十分亲近,但他们仍对“桂人”这一身份有认同感,自视为与“粤人”有所不同的群体。粤桂虽如欧榘甲所说,乃最为亲近的、应组成联邦的兄弟之邦,但毕竟各有自我认同。清帝国崩溃时,粤桂各自独立,成为两个并立于南粤大地的政治实体。从道义上来说,共享着19世纪前的历史与传统的粤、桂两邦,是应该互相友爱的。然而,旧桂系却视广东人为被征服者,犯下了不亚于龙济光的罪行。

陆荣廷进入广州伊始即倡言裁兵。但事实上,他大量裁撤粤军,并以3万多人的桂军全面控制广东各地。在陆荣廷的安排下,桂军将领林虎、沈鸿英、莫荣新、姜登选、谭浩明分任高雷、钦廉、广惠镇守使、虎门要塞司令及广东第一师师长,1.3万名入粤滇军亦大部分由桂系指挥。驻粤桂军的军费远超龙济光。1914年,桂系在广东支出军费1444.7万元。至1918年,则已激增至2725.9万元。如此庞大的军费,是桂系通过巧取豪夺抢掠来的。桂军强行提取广东的银行现款,将“造币厂所出洋毫概行运回”,甚至将广东各兵工厂生产的所有武器、弹药及广州广雅书局的图书悉数运回广西。此外,桂系又在广东增设多如牛毛的苛捐杂税,如有商贾拖欠,便常以酷刑逼缴。更令人发指的是,桂军将领放任士兵以抢劫的方式自筹军饷,并公开拍卖从百姓手中抢来的耕牛和财物。在抢劫时,桂军时常屠杀无辜、奸淫妇女,军纪极端败坏。例如,1920年1月,桂军在三水芦苞一带将一村庄40岁以下的妇女全部强奸;同年5月,桂军又在兴宁县天堂、河头等处“任意抢劫财物,焚杀居民”,将百余座村镇劫掠一空。在禁锢言论方面,桂系也与龙济光没什么区别。桂系据粤期间,广州先后有12家报刊遭查封,主笔、记者、印刷工50余人被捕。粤人好不容易摆脱了龙济光,又迎来了横行霸道的桂军,继续生活在痛苦之中\footnote{《广东通史》近代下册,页451—455}。

在对外关系上,陆荣廷以“再造共和”的功臣自居,大力讨好民国总统黎元洪,自我标榜为《临时约法》的拥护者。黎元洪于1916年8月1日重开国会、10月1日重开各省议会,中华民国法统重光\footnote{刘仲敬:《民国纪事本末·重光篇》}。梁启超离粤重返北京,在原进步党的基础上组织“研究系”参与国会政治。然而,此次法统重光从一开始便蒙上了不祥的阴影。黎元洪受制于掌控皖系军阀的总理段祺瑞,双方发生激烈的“府院之争”。当时,鲁、奉、吉、黑、豫、直、浙、苏、鄂、赣、绥、察、热十三省督军以盘踞徐州的长江巡阅使、自命清室忠臣的张勋为盟主,成立了反对国会政治的组织“督军团”。段祺瑞以之为外援对抗黎元洪,民国法统命悬一线。在南粤,两广的省议会在桂军的刺刀下饱受威胁,并无自主能力,民国政府脆弱的合法性在南粤几乎荡然无存。为进一步拉拢陆荣廷,黎元洪在1917年3月将其召至北京,并于4月10日任命其为两广巡阅使,又以陆荣廷的亲信陈炳焜、谭浩明分任广东、广西都督,从而承认了陆荣廷对整个南粤的统治权。回到南粤后,陆荣廷不入广州城,而是坐镇自己的家乡广西武鸣,以两广巡阅使的身份遥制广东。由黎、段二人结盟一事来看,黎元洪虽为民国法统的维护者,但着实缺乏诚意,仅以之为巩固自己政治地位的手段,并不比段祺瑞好多少。5月23日,府院之争达到白热化。黎元洪于是日将段祺瑞免职,段祺瑞则通电各省拒不承认免职令。很快,秦、豫、浙、奉、鲁、黑、直、闽、沪、晋相继宣布独立,黎元洪大惊失色,唯有于6月1日召张勋至北京共商国是。2日,张勋复电黎元洪,称只有黎元洪解散国会,他才会进北京。8日,张勋率兵进至天津,摆出武力威胁的姿态。黎元洪只得于13日同意解散国会,弃职逃入日本使馆,中华民国法统再告中断。张勋于次日率其“辫子兵”进占北京,并于7月1日拥立前清宣统帝溥仪登基,宣布恢复清帝国。同日,黎元洪在日本使馆发布命令,要求各省“讨逆”恢复民国。2日,黎元洪又电令副总统冯国璋代行大总统职权、复任段祺瑞为总理。段祺瑞见张勋已失去利用价值,便摆出“三造共和”的立场,于3日誓师于天津马厂,自任讨逆军总司令,以梁启超为参赞。12日,讨逆军攻入北京,张勋避入荷兰使馆。13日,溥仪宣告退位。21日,梁启超因法统中断,建议不必恢复国会,而应召集临时参议院、修改国会组织法以选出新国会,得到段祺瑞同意\footnote{刘仲敬:《民国纪事本末·重光篇》}。这样一来,段祺瑞虽然“讨逆”成功,却仍没有恢复法统。

中华民国法统的第二次中断在南粤掀起了轩然大波。6月2日,粤、桂、滇、黔、蜀联合通电宣布拥戴“中央”\footnote{刘仲敬:《民国纪事本末·重光篇》}。20日,在陆荣廷授意下,陈炳琨、谭浩明宣布“两广自主”,并于22日发表如下声明:

\begin{quote}

总统已被武力胁迫,国会解散。于国会未恢复以前,法律即失效用,所有两广军务暂由两省自主,遇有重大事件径行秉承总统训示,不受非法内阁干涉\footnote{转引自《广东通史》近代下册,页456}。

\end{quote}

陆荣廷支持黎元洪、维护《临时约法》的姿态可谓冠冕堂皇。事实上,这不过是他为绑架南粤参加争霸战争玩弄的手段。桂系摆出护法姿态后,立即制定了兵分三路北伐鄂、赣、闽的计划,并由陈炳琨、莫荣新、李烈钧两次联名通电,宣布联合粤、桂、滇、黔、湘、蜀六省起兵讨逆,拥戴陆荣廷为盟主。陆荣廷的姿态欺骗了许多国会议员,不少人纷纷南下广州,准备重组国会。为壮大声势,桂系还向孙文伸出橄榄枝,邀其参与“护法”\footnote{《广东通史》近代下册,页456}。自讨龙之役结束后,孙文因广东大权为桂系所得,即携其党徒远走上海,立“上海证券交易所股份有限公司”以筹集经费。陆荣廷的举动给孙文提供了重返南粤、篡夺大权的良机。7月2日,民国海军总长程壁光(香山人)、第一舰队司令林葆怿(闽越侯官人)于上海率十余艘军舰通电护法。17日,孙文与陈炯明、廖仲恺(美国粤侨,祖籍惠州归善)、许崇智(汕头人)、朱执信等人乘军舰抵达广州黄埔,受到陈炳琨和在粤国会议员的热烈欢迎。孙文当即发表了看似义正言辞的演说:



\begin{quote}
鄙人今日所望于诸君者,即日联电请海军全体舰队来粤,然后即在粤召集国会,请黎大总统来粤执行职务\footnote{叶曙明:《国会现场:1911—1928》,页127}。
\end{quote}

7月22日,上海的海军舰队受到孙文蛊惑,集体南下广州。事实上,孙文所谓请黎元洪来粤任总统,不过是掩饰其野心的借口。8月9日,国会议长吴景濂(满洲宁远人)抵达广州,与孙文会于黄埔。孙文表示,他欲效1912年故事,自任“临时大总统”。吴景濂当即大怒,双方大吵一场,不欢而散,吴更表示他从此与孙“誓不相见”。13日,滇督唐继尧宣布组织“靖国军”,加入“护法”阵营。25日,流亡广州的120余名国会议员于广州组织“非常国会”,召开第一次会议。31日,非常国会通过《中华民国军政府组织大纲》,决定成立以“勘定叛乱,恢复《临时约法》”为宗旨的军政府。9月1日,被孙文蛊惑的议员们以81票的绝对优势选举其为军政府大元帅。10日,孙文就职,于河南士敏土厂设大元帅府\footnote{叶曙明:《国会现场:1911—1928》,页129}。标榜“护法”的中华民国军政府,就这样在桂军和滇军庇护下产生了。

从法统的角度来看,非常国会完全为一不合法组织。首先,非常国会虽标榜恢复《临时约法》,但却未采取《临时约法》规定的责任内阁制,而是在孙文的要求下授予大元帅孙文军事、内政、外交大权。其次,中华民国国会议员总数为870人。按照规定,需有半数议员即435人到场方能开议,而非常国会仅有百余人\footnote{叶曙明:《国会现场:1911—1928》,页130—132}。因此,非常国会虽高喊“护法”,可它本身就是法统的破坏者。段祺瑞与军政府的南北对立,实为两非法组织的对立。

9月18日,湘南零陵、衡州驻军通电脱离北京政府独立,与入湘北洋军展开激战,南北大战在湖湘境内首先打响。10月,陆荣廷、孙文组织起粤、桂、湘联军,以桂督谭浩明为联军总司令,于11月北伐援湘。联军虽一度攻下长沙、岳阳,但因桂军为自保擅自与北洋军停战,湖湘战局陷入停顿。在广州,孙文虽名为统领粤、桂、滇三省的中华民国军政府大元帅,然实权皆在桂系之手,孙文之政令不出大元帅府,其所能控制的军队只有程壁光、陈炯明麾下的2000余名海军陆战队。10月9日,孙文召集作战会议,商讨援湘、攻闽事宜,要求陈炯明、朱执信加紧招募人员、编练军队。12日,孙文任命陈炯明为海军陆战队司令。在1914年中华革命党成立后,陈炯明曾因不满于孙文的集权脱党而去。然而在孙文参加护法后,对民国法统仍十分尊重的他转向孙文输诚,与其一同来到广州。1913年时,陈炯明曾拔刀威胁广东省议会、破坏法统。现在,与对法统毫无敬畏之心的孙文、陆荣廷相比,陈炯明已成了真诚捍卫法统的人\footnote{赵涛:《北洋时期的入粤客军与广东政局:1913—1925》}。

同月,南粤遭到了两股外敌的威胁。1917年10月,段祺瑞免去陆荣廷两广巡阅使之职,命已盘踞海南一年的龙济光代之。龙济光于11月22日宣布就职,拼凑了一支5000人的乌合之众渡过琼州海峡在徐闻登陆,迅速侵占阳江、阳春、高州、雷州等地,妄图卷土重来。桂、滇、粤军迅速结成“讨龙军”分五路还击,护法海军亦出航封锁琼州海峡。1918年春,退路被截的龙军败北,龙济光率残部狼狈逃回海南。这时,岛上黎民纷纷英勇起义,群起攻杀龙军,使龙济光无法立足,只得率千余残兵败将乘船逃往天津\footnote{《广东通史》近代下册,页477}。10月23日,控制粤东的桂系将领潮梅镇守使莫擎宇宣布离粤独立,投靠忠于段祺瑞的闽督李厚基。11月7日,李厚基遣一旅闽军入侵粤东,然次日即在潮州郊外当地粤军击败,莫擎宇被迫随闽军逃回福建。这样一来,李厚基遂成为南粤的巨大威胁。12月,陈炯明于惠州召集其在讨龙战争时的旧部十余营,连同海军陆战队编为4000余人的“援闽粤军”。但因缺乏桂系支持,他们甚至没有充足的武器。1918年1月12日,陈炯明为保卫南粤的安宁与民国的法统毅然于广州誓师,率部东进,陈兵于粤、闽边境。陈炯明的部队受到潮州民众的热烈欢迎,人们争相向他们提供物资、武器和弹药。至5月上旬,援闽粤军已筹集到2000余支枪械,队伍亦扩大到近万人。5月10日,陈炯明下令对闽发起全线进攻。援闽粤军迅速推进,连克闽西南数县。然而,因闽军得到浙军支援,战局很快陷入胶着。被粤人呼为“北军”的浙闽联军纠集两师兵力发动反攻,一度侵入大埔、饶平、连平等县境内。8月初,因一团浙军倒戈,其余浙军陷入混乱,仓皇撤往厦门。陈炯明乃趁势挥军反击,一举突破北军防线,于8月30日攻克延平,31日克漳州、同安。9月1日,陈炯明进入漳州,于当地设司令部,建立以漳州为中心、辖25个县的闽南护法区。至此,北军已然战败,双方展开谈判,议定于11月1日停火。12月6日,陈炯明与李厚基达成协议,规定双方控制区以闽中三元、闽南江东桥为界,粤军、北军各自后撤10公里,互不侵犯\footnote{张慧卿:《闽南护法区研究》}。陈炯明的征闽之役,遂以大获全胜告终。

当陈炯明在前线英勇奋战时,桂系和孙文却在广州争权夺利。自孙文担任大元帅起,粤督陈炳焜即拒不与之合作,宣称军政府以后“无论发生何种问题,炳焜概不负责。”1917年11月,陈炳焜去职,陆荣廷以莫荣新代之。莫荣新更加防备孙文,不向军政府和非常国会拨款分文。12月17日,孙文首开杀戒,命朱执信派人于长堤码头暗杀了驻粤滇军第四师师长方声涛。方声涛系老同盟会员,但并不支持孙文,遂被孙文视为叛徒,对其痛下杀手。1918年1月2日,莫荣新忽称大元帅府卫兵为“土匪”,派兵逮捕六十余人,枪杀其中数人。3日夜,孙文率护法海军两舰报复,二沙岛一带江面向越秀山上的粤督署开炮五十余发,试图武装政变,但海军、滇军将领皆不响应。莫荣新不欲全面开战,遂禁止部下还击,于次日赴大元帅府向孙文“谢罪”,局势由此缓和。事后,孙文深怪程壁光不肯以海军配合政变,便卑鄙地对其采取暗杀手段。2月26日夜8时30分,程壁光在赴宴途中于长堤码头遭枪手枪击身亡。事后,孙文与莫荣新亲临凶案现场,贼喊捉贼地发布了缉凶令。但在孙文的庇护下,枪手自然没有被捉到\footnote{亲国民党立场之研究,多将程案真凶定为桂系。然近年研究揭示,此案主使者实为孙文。相关讨论及学界研究情况,参见叶曙明:《国会现场:1911—1928》,页153;《广东通史》近代下册,页478}。

莫荣新虽有枪杀大元帅府卫兵之举,但孙文接连刺杀方声涛、程壁光、炮击粤督署,手段远更下作。非常国会议员为之心寒,大都倒向桂系一边。桂系乃因势利导,决定利用非常国会打击孙文。4月初,陆荣廷、唐继尧等与非常国会议长吴景濂联名通电,要求改组军政府,改大元帅制为七总裁合议制,从而限制孙文的权力。4月10日,非常国会开始讨论军政府改组案。5月4日,在武装桂军的保护下,非常国会通过军政府改组案。孙文勃然大怒,然因可用之兵已几乎全数交给陈炯明征闽,亦无计可施,只能愤恨地表示陆荣廷、唐继尧与段祺瑞实为“一丘之貉”,并向非常国会递交辞程。20日,非常国会选出唐绍仪、唐继尧、孙文、伍廷芳、林葆怿、陆荣廷、岑春煊七人为主席总裁。孙文见自己篡夺大权的阴谋失败,便于次日率其党徒乘船离开广州,于26日经汕头抵达位于大埔县三河坝的援闽粤军总司令部,要求陈炯明积极攻闽扩军。随后,孙文等离开南粤前往上海\footnote{《广东通史》近代下册,页481}。孙文的野心,便这样又一次落空了。此时,东亚大陆上仅余陈炯明的闽南护法区仍归中华革命党控制。

1918月1月12日,在未经任何立法机构的批准下,民国代总统冯国璋宣布将举行新一届国会大选。2月17日,北京临时参议院通过《国会组织法修正案》,国会选举开始。为控制新国会,段祺瑞遣其亲信徐树铮、王揖唐在北京安福wu 同组织“安福俱乐部”操纵选举。8月12日,被称为“安福国会”的新国会诞生,其参众两院议长俱为段派人物。9月4日,安福国会选举徐世昌为大总统。安福国会系冯、段政府以行政命令组织起来的,亦不遵从《临时约法》,不具备任何法统\footnote{刘仲敬:《民国纪事本末·残照篇》}。这样,中华民国便在北京、广州各出现了一个国会,两者均攻击对方为叛逆,事实上都是不合法组织。但无论如何,双方至少还在表面上标榜国会政治,未可以单纯的僭主或革命政权视之。11月16日,北京政府下令前线各军停火。22日,广州军政府亦下达停火令。1919年2月20日,双方代表开始在上海和谈,然因对法统问题各执一词,谈判迟迟不决\footnote{刘仲敬:《民国纪事本末·残照篇》}。讽刺的是,就在双方于上海唇枪舌战之时,两者的国会都已走向覆灭。自孙文离粤后,大批非常国会议员认识到桂系与孙文不过是一丘之貉,纷纷出走。军政府七总裁中,陆荣廷手握军权,成为实际统治者。1920年3月,驻于韶关的滇、桂两军发生冲突,总裁岑春煊北上调停,另一总裁伍廷芳趁机将军政府外交部、财政部印章带往香港,发表脱离军政府的声明。军政府至此无法提取关税余款,岌岌可危。4月3日,为保住军政府的法统地位,岑春煊派军警严密监视非常国会,不准议员离开,由此彻底激怒了议员。9日,非常国会参、众两院议长林森、吴景濂通电反对军政府,率部分议员出走\footnote{《广东通史》现代上册,页51}。至此,受非常国会杯葛的广州军政府已丧失了最后一丝法统,形同非法叛乱集团。驻粤桂、滇军亦已失去了最后一丝留在广东的合法性,成为毫无合法性的外来侵略军。5月5日,自广州出走的议员于上海组成“国会非常会议”,议决于迁往云南昆明。8月17日,因不见容于滇督唐继尧,非常国会又迁至正被川、滇、黔三军激烈争夺的重庆。9月19日,这批民国法统的最后守护者在林森、吴景濂进入狼烟四起的重庆,在枪炮声中一夕数惊。10月15日,川军攻逐黔军占领重庆,议员逃散一空。在华北,直系军阀首领曹锟、吴佩孚于1919年12月开始密谋夺权,矛头直指段祺瑞的皖系和受其控制的安福国会。1920年7月14日,直皖战争爆发。直军于23日攻入北京,解散安福国会,段祺瑞被迫下台。随着南北两个国会的消亡,曾被南粤承认的中华民国法统,至此烟消云散\footnote{刘仲敬:《民国纪事本末·残照篇》}。在广东,饱受滇、桂侵略者蹂躏的粤人发起了声势浩大的“粤人治粤”运动。1918年,邹鲁在致唐绍仪的书信中明确指出,广东本为富裕之地,然因“粤人治粤主义不能实行”,桂系盘踞驻扎广东,遂使广东人饱受痛苦。1919年5月,广东省议员岑涛慷慨激昂地发表通电,主张“粤人治粤”:

\begin{quote}
粤人治粤,为三千万人心理所同\footnote{转引自杨美雅:《“粤人治粤”主义论述:1902—1922》}。
\end{quote}

同年7月,为推举伍廷芳为省长,达成“粤人治粤”之主张,广州各界纷纷罢市,在学生的带领下组织起4000多人的游行队伍浩浩荡荡地奔赴省议会请愿\footnote{杨美雅:《“粤人治粤”主义论述:1902—1922》}。躲在上海的孙文看到了利用粤人的抗争实现其野心的机会。1919年10月10日,中华革命党在上海法租界改组为“中国国民党”。1920年6月29日,孙文派朱执信、廖仲恺前往漳州,敦促陈炯明率军由闽返粤,驱逐桂系,并承诺国民党本部将从上海全力进行经济支援\footnote{《广东通史》现代上册,页53}。粤桂两军的大战,已然一触即发。

做为真诚的法统维护者,陈炯明天真地相信孙文和他一样,都是《临时约法》的支持者。因此,他在护法运动开始后摈弃前嫌,重投孙文麾下。率军入闽前,他便发表声明,明确指出:

\begin{quote}
宪法未公布以前,《临时约法》就是国体的保障,中华民国主权寄在国会……政府要照国会的意思施行政治,国会要对政府的政治实行监督\footnote{张慧卿:《闽南护法区研究》}。
\end{quote}

闽南护法区成立后,陈炯明力主“闽人治闽”,从未将闽南视为被南粤征服的地域。1918年10月21日,护法军政府任命陈炯明为福建宣抚使,陈炯明辞之,指出他将“俟闽局底定,当以还付闽人”。12月15日,他再辞军政府任命的福建省长之职,并发表了一份感人至深的声明:

\begin{quote}
福建省长之虚位,饥不可以为食,寒不可以为衣,敝军不可以为战守。况以福建而论,则炯明主张以闽治闽;以省长论,则炯明主张以民治民,非惟今日宣言,抑亦平生之宗旨\footnote{转引自张慧卿:《闽南护法区研究》}。
\end{quote}

与陈炯明相比,一心谋求总统地位的孙文可谓跳梁小丑。在闽南护法区,陈炯明设置了廉勤简约的政府机构,积极维护境内由闽人土豪主导的社会秩序,并大力建设漳州的城市基础设施,并开垦荒田、发展近代教育和实业。在护法区期间,漳州商贾云集、市场兴旺,成为乱世中的一片乐土,谱写了粤闽友谊的一段佳话\footnote{张慧卿:《闽南护法区研究》}。自孙文下令陈炯明出兵驱桂后,陈炯明秣兵厉马,随时准备开战。至8月初,驻闽南护法区之粤军已达102个营,约2.5万人。与此同时,广东境内的侵略军之数则高达7万,其中不但包括桂军、滇军,还有一支由吕公望(吴越永康人)指挥的浙军。吕公望曾任浙江督军,后因在1918年与护法军政府合作,不敌皖系,于1920年奔至广东\footnote{相关兵力对比,参见粤海神僧:《第一次粤桂战争作战示意图》}。由兵力对比来看,陈炯明是几乎没有胜算的。然而,三千万广东人却与他站在一边,随时准备迎接自己的子弟兵打回家乡。从这一点来看,桂、滇、浙侵略军的力量又是无比渺小的。

1920年8月11日,护法军政府下达攻闽动员令。侵略军将驻湘桂军调回,以沈鸿英为左路司令、刘志陆为中路司令、吕公望为右路司令,兵分三路扑向闽南护法区。12日,陈炯明在漳州公园慷慨誓师,发表《声讨莫荣新电》,历数桂系对粤人犯下的罪行:

\begin{quote}
我粤自莫荣新等窃据以来,残害吾民,无所不至。举凡民意所属望之事,民意所归附之人,莫不被其破坏之、杀害之、驱逐之。番摊义会,流毒最烈,年中因而丧命倾家、卖妻鬻子者,以数万计。鸦片烟勒种遍地,乡人求免,反遭其祸。是民意所欲禁绝者,而彼偏弛之。兵工厂所出枪械子弹,悉数发给桂军,或运回广西,粤人军队丝毫不与。近复妙想天开,将机器大部运去南宁。广雅图书乃吾粤文化之命脉,劫掠殆尽。始皇坑焚,无此毒计,是民意所欲保存者,而彼偏夺之。造币厂所出毫洋概运回桂,以致吾粤银根短绌,毫币低落。粤人损失千万,是民意欲留粤以流通市面者,而彼偏掠之。吾粤连年兵燹水旱,民不聊生,莫荣新等不但不设法救济,反巧立名目,大肆敲剥,一县委员,有多至二十人者。虽前清豁免之粮,亦勒令缴纳,否则拿人酷诈,至于妇人,凄楚之状,不忍目睹,是民意所欲求安宁者,而彼偏扰之……至其纵容强盗乞丐之兵,骚扰阎闾,更为罄竹难书。吾民所受亡省之痛,较之高丽、安南、波兰亡国之痛,尤加百倍。彼辈天赋贼性,谋财害命,是其惯技。近复将驻湘桂贼移师入闽,压迫我军,其居心无非仇视粤人,俨同敌国。虽我军在外,仅有一线,亦必欲诛锄之,以为灭种之计,俾得长居吾粤,荼毒吾民。全军同人忍无可忍,乃不得已全体宣誓,誓死杀敌,救ngo-yeud 人,粉身碎骨,实有荣光。英哲有言:“人必爱其乡,而后始能爱国。”粤军今日系为乡为国而战,其一切党派及其他问题,均非所知。是用饮泣誓师,掬诚相告。同胞诸君,尚祈鉴之!粤军全体将士同叩\footnote{《陈炯明集》上卷,页459—460}。
\end{quote}

此篇檄文发自肺腑,将粤人的反抗精神表达得淋漓尽致。笔者行文至此,已难抑盈眶热泪,感动得哭泣出声。必须指出,认同民国《临时约法》的陈炯明并未主张南粤独立,仍视广东为民国之一省。但他将被侵略者霸占的广东比作亡国的越南、朝鲜和波兰,无疑表明广东在他心目中是一必须自立的政治实体。他虽未能完全清除大一统观念的影响,但仍是为南粤自由与尊严而战的伟大英雄。

8月16日,在侵略军主力尚未到达粤闽边境之际,粤军总司令陈炯明先发制人,以20个营留守闽南,亲自与参谋处处长叶举(惠阳人)指挥中路、以参谋长邓铿指挥左路、以第二军军长许崇智(广州人,祖籍汕头)指挥右路,指挥82个营的兵力发动全线攻势,解放广东的第一次粤桂战争打响。左路军自诏安出发越过边境,于黄冈、澄海大破敌军,逼近汕头。19日,侵略军中路司令刘志陆部下炮兵营长余鹰扬据汕头独立,加入粤军。刘志陆慌不择路,狼狈逃往广州。右路军由永定越过边境,于16日当天解放大埔、17日克复蕉岭、19日克梅县、20日克兴宁。中路军则迅速克复潮安、饶平,进至高坡、丰顺一带。至26日,粤东潮梅地区已全部解放。粤军顺东江而下,直取惠州。粤军是一支信奉着“粤人治粤主义”的军队,他们怀着解放乡邦的信念奋勇而战,所到之处受到南粤父老的热烈欢迎。桂、滇、浙侵略者兵力虽众,却多是些只知蹂躏无辜的乌合之众,纷纷一触即溃,向西逃窜。莫荣新惊闻粤东易手,急忙收缩战线,命侵略军各部死守河源、博罗、惠阳一线。粤军乘胜西进,右路军于9月2日收复龙川县重镇老隆,向河源挺近。河源乃惠州门户,为保住惠州,桂军在此负隅顽抗,与粤军展开了长达一个月的血战。这时,孙文派遣朱执信前往珠江口,策动虎门要塞驻军反桂。16日,经朱执信策反,虎门要塞司令丘渭南宣布离桂独立。然在不久后的战斗中,朱执信中流弹阵亡。10月16日,经过苦战,右路军终于攻克河源。与此同时,中、左两路已抵达逼近惠州外围,三路粤军遂一同向惠州发起总攻。莫荣新亦在惠州集结40个营的重兵,力图最后一搏。22日,粤军经激战解放惠州城。23日,陈炯明在惠州召开军事会议,决定分三路进攻增城、石龙、东莞,而后会攻广州。此时,三水、宝安、开平等地的百姓已发动英勇的起义,群起攻击侵略者。粤汉铁路的工人展开总罢工,以手枪、炸弹对抗强迫他们通车的侵略军。在广州城内,全民一同反抗,发动罢市、罢工、罢课浪潮。莫荣新焦头烂额,无兵可调,败局已经注定。24日,桂系以岑春煊、陆荣廷等人的名义取消护法军政府,宣布两广“独立”。同日,岑春煊逃离广州,经香港前往上海。25日,石龙光复。26日,东莞解放。莫荣新见大势已去,又匆忙宣布取消广东“独立”。27日,粤军克复增城,三路军队齐聚广州城郊,于29日发动总攻。同日,莫荣新率万余军队狼狈弃城西逃,粤军在各界民众的支持下解放广州。桂军败兵沿西江撤退时,一路烧杀抢掠,大肆蹂躏西江两岸的墟镇,犯下了他们在广东的最后一起暴行。粤军乘胜追击,相继解放清远、四会、肇庆。11月21日,陆荣廷下令仍在广东境内顽抗的部队全数退回广西,并道貌岸然地自称“顾念邻交,不忍地方糜烂”\footnote{第一次粤桂战争经过,参见《广东通史》现代上册,页54—60;宋其蕤、冯粤松:《广州军事史》下,页195—197}。至此,第一次粤桂战争以粤军的全面胜利告终,陈炯明和粤军将士在民众的支持下解放了自己的乡邦。长达五年的桂系据粤噩梦,终于过去了,粤军取得了史诗般的胜利,创造了伟大的军事奇迹。然而,此后的南粤能够迎来真正的自由吗?在孙文阴骘的目光下,南粤的命运立即变得捉摸不定起来。

\section{陈孙决裂:1920—1922年}

\indent 1920年10月30日,即粤军解放广州的第二天,残留于广州的非常国会议员电邀孙文、伍廷芳、唐绍仪赴粤,试图恢复护法军政府。11月2日,陈炯明在民众的欢呼声中进入广州城,于10日就任广东省省长。20日,广东各界人士在广州高师学校召开欢迎粤军凯旋大会,陈炯明偕邓铿等将领出席,发表演说,提出改造广东、实行禁赌、禁烟,整顿军政、民政、财政等主张。自1913年龙济光侵占广东后,曾被广东谘议局禁止的赌博死灰复燃。到1919年,广东赌税高达财政收入的三分之一,赌博业严重败坏了社会风气。1920年6月,广东省议会曾通过“禁赌案”,但莫荣新拒不执行。陈炯明雷厉风行,于12月1日颁布禁赌令,使广州的赌博业不见踪迹。另一方面,在桂系据粤时期,广东的罂粟种植业在其纵容下十分兴盛,引起社会各界不满。陈炯明出任省长后厉行禁烟,于1921年11月15日联合粤海关税务司、基督教青年会成立广东万国禁烟会,并在12月15日颁布了严格的禁烟令。对广州的市政建设,陈炯明亦十分上心。1921年,广州展开大规模道路施工,修建了相当于两倍的马路及大批公园、体育场、图书馆、市民大学、教育博物展览会,使交通状况和市容市貌焕然一新。此外,陈炯明又取消报禁,使广东人重获久违的言论自由。他还令广东各地民众自办警察、税收,极大减轻了粤人的税负\footnote{《广东通史》现代上册,页61—62}。

陈炯明采取上述政策,与他的“联省自治”思想大有关系。1920年初,因东亚大陆战乱不休,各邦相继出现“联省自治”思潮,希望联合自治的民国各省建立联邦制以消弭战争。1920年6月,湘督谭延闿驱逐皖系将领张敬尧、攻克长沙,在湖南率先启动自治进程。谭延闿虽为国民党员,却力主“湘人自治”,并非孙文一类的大一统爱好者。1920年下半年,湖湘各界热情地投入制定省宪的运动,以彭璜、毛泽东为代表的一批热血青年更是积极主张湖湘完全独立,建立“湖南共和国”,受到谭延闿资助。当年11月,谭延闿被赵恒锡起兵驱逐。赵恒锡亦为湘人,与谭同样热衷于制宪运动,并向湖南省议会表达了“联省自治”的主张。这时,湖湘方兴未艾的民族主义运动引起了苏俄的关注。1917年11月,十月革命爆发,列宁指挥布尔什维克夺取俄国政权,成立苏俄。1919年,列宁用于推动世界革命的共产国际在莫斯科成立。1920年11月,彭璜与毛泽东的政治思想日趋左转,创立湖南共产主义小组。次年1月,毛泽东提出要以“激烈方法和共产主义”改造民国,从此抛弃了湘独立场。彭璜被其说服,但仍在数个月后神秘失踪,其死因至今仍众说纷纭。7月,中共一大在上海召开,毛泽东受共产国际资助,做为湖南代表参会。虽然湘独运动在共产国际的颠覆下遭受重创,但《湖南省宪法》仍于1922年1月1日通过\footnote{彭璜之死因,有人主张为被毛泽东暗杀,但缺乏足够证据。关于湖南省宪进程,详见裴士锋:《湖南人与现代中国》}。此外,在1920年10月,浙江省议会亦在知名律师阮性存的倡导下启动省宪制定进程。1921年9月9日,在浙督卢永祥的支持下,被称为“九九宪法”的《浙江省宪法》通过。其后,受湘、浙影响,福建、四川亦各自颁布省宪\footnote{关于浙、闽、蜀的省宪制定过程,参见林孝文:《浙江省宪研究》、刘沧海:《福建省宪法之文本分析》、龙长安:《民初的四川省宪法述论》}。联省自治运动和一批省宪的出台虽未完全解散中华民国,却是一股与大一统逆向而动的清流,曾遭受许多大一统分子的攻击。陈炯明即是联省自治运动在南粤的积极倡导者。1921年2月,陈炯明发表《建设方略》一文,提出如下观点:

\begin{quote}

盲论之士,往往以主张分治,即为破坏统一。曾不知分治与集权,实为对待之名词,与统一何与?北美合众国成例俱在,岂容指鹿为马。民国以来之祸乱,正生自论者误解集权为统一,于是野心者遂假统一以集权\footnote{转引自任玥:《陈炯明“联省自治”思想浅析》}。

\end{quote}

陈炯明虽未能摆脱“统一”观念对他的干扰,但他反对大一统集权、支持建设美利坚合众国式的国家、将民国各省类比于美国各州的主张倘若实行,仍能给南粤带来相当程度的自由。在此思想指导下,陈炯明不但力主“粤人治粤”,亦将闽南护法区还给了闽人。第一次粤桂战争胜利后,驻闽粤军即撤离闽境,将闽南交还闽督李厚基,陈炯明践行了自己在攻闽前的承诺\footnote{张慧卿:《闽南护法区研究》}。此后,陈炯明又投身于广东立宪工作。1921年12月29日,在他的支持下,广东省议会通过《广东省宪法草案》。该草案模仿湘宪、浙宪,共有15章135条,规定实行三权分立原则,现役军人不得干政、非依法律命令不得召集会议。因宪法尚未经全广东公民投票表决,故只为草案\footnote{刘国新:《中国政治制度辞典》,页323—324}。这样,广东也启动了省宪制定进程。

与陈炯明不同,孙文是个狂热的大一统主义革命家。1920年11月28日,孙文重返广州主政,于次日宣布恢复护法军政府,以孙文、唐绍仪、伍廷芳、唐继尧为总裁、以陈炯明为陆军部长,是为国民党宣称的“二次护法”。然而,此时的广州非常国会早已瓦解,孙文其实无法可护,所谓“护法”仅为幌子。1920年12月12日,孙文召集残留广州的非常国会议员开会,但因法定人数不足无法开会,遂召开座谈会。与会者多对孙文仍有幻想,主张在广州建立民国中央政府、另立总统。1921年1月1日,孙文在军政府发表演说称:



\begin{quote}
此次军政府回粤,其责任固在继续护法。但余观察现在大势,护法断不能解决根本问题\footnote{《广东通史》现代上册,页64}。
\end{quote}

没有法统的议员欲推举孙文做总统,孙文也对护法三心二意。所谓的“二次护法”究竟有着怎样的实质,也就不问可知了。4月7日,由广州残留议员组织的“非常国会”选举孙文为所谓的“非常大总统”,北京政府、直系、奉系军阀及湘、桂、鄂等省均不表示承认,广东省议会亦开会反对。陈炯明对孙文的行径大为愤慨,因为广东刚刚被从桂系手中解放出来,孙文却要把广东当成大一统的新策源地,消耗南粤的民力与北方争霸。陈炯明当即命部下向孙文提出如下质询:

\begin{quote}
一、南方选举总统,无异自树目标。一旦北方来攻,何以御之?

二、广东现时军实是否充实,饷项是否丰饶?果有战事发生,究竟能支持几时\footnote{转引自《广东通史》现代上册,页64}?
\end{quote}

陈炯明的质询可谓掷地有声,但孙文置之不理。为维护国民党内团结,陈炯明只得于14日率全体粤军将领电贺孙文当选总统。孙文乃于5月5日就任“非常大总统”,设总统府于越秀山南麓,随即开始策划攻桂事宜。当时,陈炯明因广东已经解放,不欲再兴兵事,亦因主张“桂人治桂”,不愿入侵广西,遂与陆荣廷“信使往来,约以各守边圄,勿相侵扰”。但在孙文的严命下,他还是不得不于4月在西江地区部署军队。5月20日,北京政府下达“南方讨伐令”,以陆荣廷为“粤桂边防督办”,令其进攻广东。陆荣廷为献媚于北京政府,宣布取消广西独立,以陈炳琨为广西护军使,设署于梧州,并叫嚣“先入粤者任粤督”,妄图卷土重来\footnote{《广东通史》现代上册,页66}。与此同时,孙文亦积极备战,将许崇智所部粤军4000余人编为军政府直辖,并大肆招募新兵\footnote{宋其蕤、冯粤松:《广州军事史》下,页198}。6月13日,陆荣廷首先动手,“第二次粤桂战争”爆发。桂军2万余人兵分三路,以陈炳琨部出梧州进犯郁南、罗定,以申葆藩部进犯高州、廉州,以沈鸿英部出怀集进犯粤北。6月20日,孙文任命陈炯明为讨桂军司令。陈炯明于是日在肇庆誓师,表明战争目的:


\begin{quote}
此次用兵,亦即所以拯拔桂人,尤望敌忾同仇,俾贼巢早荡,民治早覆\footnote{转引自《广东通史》现代上册,页67}。
\end{quote}

在陈炯明心目中,此次战事绝非对广西的征服战争,而是解放广西民众的义战。在陈炯明的指挥下,粤军135个营4万余人兵分三路反击桂军,以许崇智指挥右路出四会、广宁,进攻贺县、平乐,以叶举指挥中路沿西江进攻梧州,以翁式亮指挥左路迎战入侵高州、雷州之敌。粤军势如破竹,中路因桂军一部倒戈,于26日顺利攻克梧州。28日,陈炯明入驻梧州,命三路大军深入桂境,右路出平乐以窥桂林中路、左路分道会攻浔州。中路粤军排除桂军之顽强抵抗节节推进,于7月12日克藤县、16日克浔州、8月5日克省会南宁,右路粤军则于13日克桂林。桂军残部逃入桂西南靠近越南的重镇龙州,据周边山隘死守。9月11日,陈炯明下令全军进攻龙州。粤军兵分数路,奋勇进攻,于26日攻克龙州,第二次粤桂战争遂以粤军占领广西告终。此战,粤桂两军各自伤亡约万人\footnote{《广东通史》现代上册,页67—68、宋其蕤、冯粤松:《广州军事史》下,页199}。龙州陷落前夕,陆荣廷通电下野,取道越南逃亡上海。陈炯明则于8月上旬进驻南宁,协助孙文于7月28日任命的广西桂人省长马君武进行战后建设。陈炯明兑现了自己在出兵前的承诺,宣言“桂人治桂”,恢复了被桂系解散的广西省议会。11月3日,陈炯明见大势已定,率主力部队凯旋回粤。离开南宁前,因当时尚有不少桂系残军逃入山谷、落草为寇,陈炯明应马君武之请,命叶举率50个营留守广西清剿残敌\footnote{宋其蕤、冯粤松:《广州军事史》下,页199}。陈炯明扫荡桂系、恢复广西省议会的义战是他人 生的重要闪光点。但在孙文看来,陈炯明不过是帮他统一南粤、建立北伐根据地的棋子。讨桂军凯旋后,孙文对陈炯明等将领训话称:


\begin{quote}
北伐之举,吾等不得不行。粤处偏安,只能苟且图存,而非久安长治,能出兵则可统一中国\footnote{《广东通史》现代上册,页68}。
\end{quote}

陈炯明的一腔热情又一次被孙文利用了。10月8日,“非常国会”通过孙文递交的“北伐出师案”。孙文随即将粤军编为约3万人的“北伐军”,于15日向广西进发。12月4日,孙文抵达桂林,在此设立“北伐大本营”。对孙文的疯狂举动,陈炯明进行了抵制,拒不离开广州。1922年3月,在陈炯明的劝说下,滇督唐继尧表示不会支持北伐,湘督赵恒锡亦拒绝北伐军取道全州入境\footnote{《广东通史》现代上册,页68—70}。孙文乃再次祭出令人不齿的暗杀手段,于3月21日派刺客枪击陈炯明的得力助手参谋长邓铿。两日后,邓铿因伤重逝世。孙文则假仁假义地于24日追赠邓铿为陆军上将,并在事后贼喊捉贼地声称杀邓案主使者为陈炯明\footnote{国民党宣称邓案元凶为陈炯明。然据汪荣祖考证,此案元凶实系孙文无疑。参见汪荣祖:《邓铿被暗杀真相探索》}。4月16日,孙文在梧州召开军事会议,决定将北伐大本营迁至韶关。21日,他又下令免去陈炯明广东省长、粤军总司令之职。同日,陈炯明率部离开广州,返回惠州,布防于石龙、虎门一带。同时,他命令叶举率部火速由桂返粤。5月8日,孙文要求叶举部驻扎于肇庆、罗定、阳春、高州、雷州、钦州、廉州等地,但叶举拒不从命,于18日率部开入广州。20日,叶举又与粤军诸将致电孙文,要求恢复陈炯明广东省长、粤军总司令之职,遭孙文拒绝\footnote{《广东通史》现代上册,页70—71}。至此,陈孙关系已濒临破裂。

这时,为对抗直系军阀曹锟、吴佩孚,孙文已与段祺瑞的皖系、张作霖的奉系军阀结成反直三角同盟。1922年4月,奉军开入山海关,于29日与直军全面开战,是为“第一次直奉战争”。5月5日,奉军败还满洲,战争以直系的胜利结束。战后,曹、吴二人发生激烈矛盾。曹锟欲谋总统之位,吴佩孚极力阻止,遂力主恢复国会,请黎元洪复任总统。经与吴佩孚密探后,原国会议长吴景濂于6月1日召集旧议员集会于天津,宣布国会恢复职权。6月2日,总统徐世昌在直系的逼迫下被迫辞职。11日,黎元洪进入北京总统府暂行大总统职权。8月1日,拥有足够法定人数的国会于北京正式召开,中华民国法统第二次重光\footnote{《国会现场:1911—1928》,页257—266}。

随着民国法统再次重光,孙文的“二次护法”完全失去合法性,他已没有任何理由继续担任所谓的“非常大总统”。然而,孙文本就没有护法诚意,“护法”不过是他推行北伐大一统的遮羞布。5月9日,北伐军自韶关出发,兵分三路入侵江右,于6月5日侵占赣州\footnote{《国会现场:1911—1928》,页248}。为支撑北伐军的庞大开销,孙文屡次向广州商人勒索军费,使他们疲于奔命。6月1日,为说服陈炯明参加北伐,孙文带两营警卫自韶关返回广州,命令陈炯明到广州面谈,陈炯明则不做回复。3日,叶举宣布广州戒严,城中各街道遍布岗哨,气氛十分紧张。12日,气急败坏的孙文召集广州报界人士举行茶话会,对陈炯明发表了杀气腾腾的声明:

\begin{quote}

我下令要粤军全数退出省城30里之外。他若不服从命令,我就以武力压服他!人家说我孙文是车大炮,但这回大炮更是厉害,不是用实心弹,而是用开化弹,或用八英寸口径的大炮的毒气弹,不难于三小时内把他六十余营陈家军变为泥粉\footnote{《国会现场:1911—1928》,页250}!

\end{quote}

至此,陈炯明终于明白,孙文其实与龙济光、陆荣廷一样,都是疯狂压榨南粤以实现其野心的恶魔,而且是比两者远更残忍的恶魔。为了镇压粤人对他的反抗,他甚至不惜使用毒气。在南粤悠久的历史上,每当面临不义和残暴的统治时,总有一批英雄儿女站出来,为了南粤的自由和尊严英勇战斗。面对孙文这样一个穷凶极恶的魔鬼,陈炯明所能做的也只有像祖先一样拿起武器、拼死一战。尽管陈炯明曾与孙文有过一同战斗的情谊,但为了南粤,他别无选择。13日,陈炯明对叶举下达攻击孙文的密令。15日下午,察觉到粤军异动的孙文离开总统府,抵达海珠的海军司令部,登上“楚豫”舰,又于次日天明后转至“永丰舰”,战斗即将开始\footnote{《广东通史》现代上册,页73—74}。

1922年6月16日凌晨3时,伟大的时刻到来了,叶举命令4000余名南粤子弟兵包围越秀山南麓的总统府,开炮三发以威慑之,但此时孙文已逃上了军舰。清晨,广州市民一觉醒来,发现广州已被自己的子弟兵解放。登上“永丰”舰后,孙文首先下令入赣北伐军回师攻粤,而后决定通过屠杀无辜泄愤。17日下午1时,孙文率“永丰”等七舰从黄埔出动,在广州内河海珠、天字码头、士敏土厂前各处炮击市区,又向白云山、越秀山、大沙头、沙河等处遥射。因有舰队官兵不忍向平民开火,孙文竟亲自操炮射击。至下午3时,第一轮炮击结束。5时,第二轮炮击开始,至7时结束。两轮炮击共开炮约100发,着弹点多为人烟密集的城区。炮击发生时,海珠一带居民纷纷呼救,街上行人被成片炸倒;在东堤一带,几乎所有商铺全被炸毁,宾客满座的冠月茶楼更是直接中弹,30余人或死火伤;在河南、越秀山、沙河、大沙头,民房多被炮击损毁。在两轮炮击中,广州房屋损失总计不下500万元,无辜平民死难者多达数百人。而在国民党则在他们撰写的史书中无耻地宣称,“叛军死于炮火者约数百人”\footnote{陈定炎:《陈炯明与孙中山、蒋介石的恩怨真相》}。无辜的广州市民不但横遭孙文屠杀,更被污蔑为“叛军”。国民党的无耻下作,于此表现得淋漓尽致!

孙文虽控制了几艘军舰,但广州城已完全掌控在粤军手中,内河亦被各炮台的火力封锁。孙文唯有漂泊江上,在重围中度日如年,形同丧家之犬。他的忠实党徒蒋中正其时正在家乡吴越宁波休养。听闻“领袖”遇险,忠心护主的蒋中正急忙乘船赶到广州,于29日登上“永丰”舰随侍。此前,因深恐孙文再次开炮屠民,广州总商会于27日派出二十五名代表前往“永丰”舰与孙文交涉。在听完代表们讲完炮击造成的惨状后,孙文不但毫无同情之心,反而面露凶相,颐气指使地威胁道:

\begin{quote}
若果陈氏今早抵省,我即今早开炮;今晚返省,即今晚开炮。汝等既赞成之,则自负责任。如其不来,我亦何必开炮……总之,陈氏一日在省,则省城地方一日不宁。汝等既系来请我勿再开炮,这又何难。若能本汝等之良心,主张不欢迎陈氏,作消极的抵制,则彼自讨没趣,自然不来省,则我亦可日内离去省城。否则彼来,我亦可来耳\footnote{转引自陈定炎:《陈炯明与孙中山、蒋介石的恩怨真相》}。

\end{quote}

身为屠杀广州市民的凶手,孙文不但毫无悔过之意,反而宣称欢迎陈炯明的广州市民是罪魁祸首。只要广州人不服从他,他便会继续开炮屠民。换言之,全广州的百姓已被孙文绑架为人质。代表们大惊失色,双方不欢而散。不过,孙文的威胁不过是他的又一次“车大炮”。此时,因珠江江面已基本被粤军火力封锁,“永丰”等舰已走投无路。7月8日,孙文率舰靠近英法租界沙面,泊于白鹅潭。白鹅潭系广州对外通商口岸,由英法军队控制,孙文遂能在此躲避粤军火力。10日,英籍海关税务官员登上“永丰”舰,称孙文若不退走,沙面恐受战火牵连。在英人的压力下,孙文只得于8月9日离开“永丰”舰,乘英船逃往香港,而后转搭俄国游轮奔赴上海。15日,陈炯明重返广州,再次出任粤军总司令\footnote{《广东通史》现代上册,页74—76}。在一片欢呼声中,陈炯明又一次解放了广东。

\section{卫粤战争与新桂系崛起:1922—1926年}

\indent 1922年6月16日,粤军再次解放广州。其后,陈炯明承认民国法统第二次重光,派人游说残留广州的议员北上参加国会,非常国会遂告瓦解。陈炯明于8月15日重返广州后,陈炯明又向广东省议会举荐香山人陈席儒为省长,获议会批准,显示了对省议会的尊重。当时,广东虽有增加财政收入以备战的需求,但陈炯明不欲向民众增税,便通过向香港联华公司借贷的方式筹款。饱受孙文横征暴敛之苦的粤人,总算稍得苏息\footnote{《广东通史》现代上册,页76—77}。

8月初,侵赣北伐军接到孙文命令,开始掉头南下侵粤。北伐军除有许崇智、李福林、黄大伟所部的3万余粤军外,尚有朱培德的滇军、李烈钧的赣军、陈嘉佑的湘军共万余人。根据孙文指示,许崇智率粤军入闽开辟基地;滇、赣、湘军则经湘入桂,与广西境内支持北伐的桂军刘震寰部、滇军张开儒部合流\footnote{刘为曾反抗陆荣廷的广西老国民党员,故于第二次粤桂战争后被孙文委任统率投降桂军。张系不听唐继尧指挥、擅自率兵参加北伐的将领,与刘皆受到孙文的信任。参见《广东通史》现代上册,页79}。当时,福建延平镇守使王永泉正与闽督李厚基不和,遂与许崇智达成交易,由许军助王逐李,而后王军助许侵粤。10月初,许军侵闽,与王军配合,于10月12日攻陷福州,李厚基败走南京,王永泉出任闽军总司令。孙文随即于18日将许军改编为“讨贼”东路军,以许崇智为总司令、蒋中正为参谋长。10月下旬,桂、滇、湘、赣诸军于桂平江口合流,总兵力计3万余人,形成“讨贼”西路军。12月6日,孙文以滇将杨希闵为滇军总司令、刘寰寰为桂军总司令\footnote{《广东通史》现代上册,页80}。这样,侵略军便自东西两面对广东形成了夹击之势。

12月25日,由桂、滇、湘、赣军组成的“讨贼”西路军开始向东进兵。就在两年前,孙文还在利用陈炯明的粤军驱逐盘踞广东的桂、滇侵略者。现在,他反而要无耻地用桂、滇军队去侵略被陈炯明解放的广东了。28日,“讨贼”西路军进至梧州,于31日越过粤桂边境,沿西江猖狂东犯,艰苦卓绝的卫粤战争拉开序幕。陈炯明在封开、肇庆、河源布下三道防线,准备节节抵抗。不幸的是,在驻守封开的粤军第一师中,孙文早已卑鄙地安插了叶挺、蒋光鼐、蔡廷锴、陈诚、薛岳、余汉谋等忠实党徒,该师参谋长李济深(广西苍梧人)亦是孙文一派的人。侵略军入境后,他们立即策动粤军第一师倒戈,掉头进攻自己的乡邦和父老。1923年1月4日,孙文又命“讨贼”东路军进犯潮汕。在两路敌人大举来犯、封开守军叛变的情况下,陈炯明兵力不足,很快支撑不住。9日,肇庆沦陷。10日,三水失守,“讨贼”西路军兵临广州城下。12日,广州守军一部及越秀山上的炮兵受孙文策动发动哗变,城内一片大乱。见广州已无法再守,陈炯明只得于15日率部忍痛退往惠州,随后与叶举前往香港继续指挥战争,命粤军第二师师长杨坤如(博罗)死守惠州。次日,广州陷落。20日,孙文改任胡汉民为广东省长、许崇智为粤军总司令、魏邦平为广州卫戍司令\footnote{《广东通史》现代上册,页82}。29日,汕头失守\footnote{宋其蕤、冯粤松:《广州军事史》下,页204}。就ni 样,仅仅过了七个月,广州再次落入孙文之手。

就在侵略军大小头目为侵占广州弹冠相庆,准备向惠州进犯时,他们之间却发生了内讧。1月26日夜,桂军将领沈鸿英串通滇军,以召开江防会议为名,将胡汉民、魏邦平等粤籍政要诱至滇军驻地,拔枪相向,试图一网打尽。胡、魏二人察觉不妙,狼狈逃走。沈鸿英遂一举控制广州,以里通陈炯明之名,将魏邦平所部伪粤军缴械。由于沈鸿英并未公开反孙,孙文仍于2月21日自上海回到广州,重建大元帅府,自任毫无合法性的陆海军大元帅,并于24日任命沈鸿英为桂军总司令,承认了政变制造的既成事实。沈鸿英本为陆荣廷部下,曾做为侵略军的头目在广东活动多年。在第一次粤桂战争中,他的军队纪律极坏,于沿西江撤退时烧杀抢掠。第二次粤桂战争后,他又向孙文投降,系一反复无常的小人和野心家。政变成功后,他仍不满足,试图赶走孙文,由自己统治广东。孙文亦对沈多有防备,于2月23日命其率军移驻肇庆。沈鸿英见时机未成,遂暂时从命,在肇庆秣兵厉马。4月10日,沈鸿英突然率部移驻广州北郊白云山一带,又于15日宣布接受北京政府的广东军务之职,对孙文开战。16日,沈军猛攻越秀山,与滇军、伪军及桂军刘震寰部激战四日,未能得手。19日,滇军夺取白云山,沈鸿英只得率残部退往从化。孙文军紧追不舍,于5月18日将其驱赶至广西梧州。6月初,沈鸿英再次发动进攻,连占韶关、英德、南雄等粤北重镇。孙文乃率滇军沿北江出动与其决战,于7月初将其驱至赣南,沈鸿英随后绕道逃回广西。此次内讧,侵略军伤亡惨重,其中孙军伤亡2000余人、沈军伤亡4000余人\footnote{宋其蕤、冯粤松:《广州军事史》下,页206—207}。

因受沈鸿英影响,孙文迟至5月下旬方命杨希闵、刘震寰的滇、桂军进攻惠州。5月20日,桂军侵占博罗,大施暴虐,将全城屠掠一空。据香港报刊报导,劫后的博罗惨不忍睹:

\begin{quote}
全城仅余老妇二人、中年妇女四五人、小孩数人,余人闻多被捉,或遭杀害,惠州军民有鉴于此,均愿出而助粤军\footnote{转引自宋其蕤、冯粤松:《广州军事史》下,页203}。
\end{quote}

与此同时,粤军林虎部在粤东发起反击,于5月25日克复汕头,许崇智率部逃亡揭阳,侵略军东西夹攻惠州的可能性化为泡影。26日,孙文对惠州下达总攻击令,上万滇、桂军向惠州城猛扑而去。保卫惠州的粤军第二师仅有两旅6000余兵力,师长杨坤如以下的两位旅长为林烈(高要人)、钟子廷。杨坤如是博罗人,他曾眼见自己的家乡被侵略军烧杀成一片废墟。由于敌人极度残暴,附近的绅商和百姓纷纷向他求救、尽全力帮助他。因此,杨坤如和全体官兵下定决心死守惠州,决不辜负乡邦父老。杨坤如命钟子廷率2000余人防守惠阳县城,亲自与林烈率主力保卫惠州城。他又命令附近九县各组一由警察、民兵构成的独立团,以游击战袭扰敌后。惠州是一座三面环水的坚城,守城军民构筑了坚固的工事,严阵以待,随时准备痛击来犯之敌。5月28日,滇、桂侵略军包围惠州城,抢占城外制高点飞鹅岭。6月5日,孙文偕蒋中正视察前线,要求各部于三日后展开总攻。8日,在孙文的亲自督战下,侵略军各部蜂拥而上,在飞机的掩护下向惠州城展开了疯狂的进攻,壮烈的第一次惠州保卫战拉开序幕\footnote{宋其蕤、冯粤松:《广州军事史》下,页202—203}。

此后二十余日内,敌我双方展开惨烈搏杀。侵略军每次进攻前皆以飞机投弹,炸死大批平民,随后乘船渡河。每当敌船渡河时,粤军即集中火力猛射冲锋在前的船只,毙伤敌人甚众,一次次挫败了侵略者的图谋。侵略军随即狡猾地以空船为先导,试图偷渡,很快就被识破。粤军专射后船,再次挫败敌人的图谋。气急败坏的侵略军只好命人攀附船缘泅渡,但还是被粤军察觉,死于水中者亦甚众。经过日复一日的进攻,侵略军付出了惨重的代价,总算摸到城墙下。他们首先掘地道、埋地雷,试图炸塌城墙,但因城墙坚厚,进展不大。接着,他们又丧心病狂地驱赶附近乡民抬云梯冲锋,试图缘梯而上,但还是被粤军击退。到6月21日,英雄的惠州城仍岿然不动。孙文暴跳如雷,遂令各部休整数日,于四日后再攻,务必于三日内拿下惠州。他还灭绝人性地提出,倘若三日内无法攻下,他便要烧掉全城。25日,侵略军再次攻城。次日,孙文为进攻沈鸿英离开惠州前线。至7月中旬,侵略军除在城下遗尸外一无所获,乃于13日起以炮击毁城。为达成毁城目的,孙文下令将两门巨炮从虎门炮台运至惠州前线。至17日,经连日炮击,惠州城内小东街一带燃起熊熊大火,“商民店铺同付一炬”,成百上千的平民葬身火海。纵然如此,守城军民仍誓死战斗,绝不向敌人低头。侵略军无计可施,只得暂停攻击,开始围城。因城内绅商提供了充足的粮食,守军士气高昂,毫无动摇之色。7月31日和9月11日,孙文又两赴前线督促攻城,仍一无所获。在9月11日这天,孙文因损兵折将,甚至亲自向城内开炮六发泄愤\footnote{宋其蕤、冯粤松:《广州军事史》下,页203}。

7月下旬,粤东战场传来捷报。粤军林虎等部乘胜追击,攻入闽南。8月初,粤军攻克漳州,逼近厦门,一举拿下“讨贼东路军”的老巢。解决掉后方威胁后,粤军于8月下旬回师粤东休整。10月下旬,叶举亲临粤东,与洪兆麟率主力部队分两路西进救援惠州,林虎则率部经河源进攻回龙、埔前、古竹一带。在叶举的指挥下,援军一路经白花芒、一路经三多祝、惠阳包夹围城之敌。侵略军支持不住,于10月27日惨败,纷纷向博罗、樟木头狼狈溃逃,杨坤如趁势杀出城外追击,惠州之围遂解,陈炯明随即进入惠州,在此设立司令部\footnote{宋其蕤、冯粤松:《广州军事史》下,页203—204}。历时四个多月的第一次惠州保卫战,便以南粤军民的光荣胜利告终。

惠州解围后,粤军分三路乘胜西进,准备一举解放广州。叶举指挥中路于11月3日解放博罗,而后向石龙节节推进。洪兆麟指挥左路收复樟木头,绕出广九铁路,进击东莞茶山等处。林虎指挥右路克复增城,进袭石滩。至此,三路粤军已包围石龙。12日,石龙光复。粤军诸将乃于17日在南冈召开军事会议,决定仍分左、中、右三路,由洪兆麟、陈炯光(陈炯明堂弟)、刘志陆分别指挥,于次日进攻广州。18日,广州近郊之战打响。是日,粤军左、中两路猛攻石牌,一举击溃刘震寰、许崇智麾下桂军、伪军,左路距离广州市区已不足五公里,右路则在白云山附近与滇军展开鏖战。19日晨,右路进至白云山脚,但中路忽因弹药告罄被迫后退,左路亦未协同进攻,右路侧翼彻底暴露,不得已退往太和市休整。此时,滇、桂军和伪军皆已濒临溃败,而粤军左、右两路仍有战斗力。广州近郊之战,似乎马上就要分出胜负了\footnote{宋其蕤、冯粤松:《广州军事史》下,页205}。令人扼腕叹息的是,侵略军中的湘军此时尚有战力。原湘督谭延闿因为赵恒锡所迫,已率残部投靠孙文。他当即率湘军绕至粤军后方展开突袭,迫使刘志陆率疲惫已极的右路撤出战场。其余粤军见右路撤退,遂全体退走\footnote{宋其蕤、冯粤松:《广州军事史》下,页203—204}。广州近郊之战的走势,就这样在最后关头反转,粤军遗憾地与胜利失之交臂。我们不妨畅想,假如谭延闿的湘军没有出动,那么粤军必然能够解放广州,甚至活捉孙文,结束笼罩在南粤大地上的噩梦。在这之后,国民党的北伐战争也不会存在,谭延闿曾拼死守护的湖湘也随之不会沦亡。谭延闿拯救了孙文,也因此锁定了他最为珍视的乡邦的命运,这不得不说是深藏于历史中的隐秘公正。

经过惠州、广州近郊两场大战,两军都已精疲力竭,短时间内无力再战。此后直至1924年底,双方形成对峙之势,均未发动大攻势。就在这时,共产国际的魔掌已伸入南粤。早在1921年7月中共成立前,共产国际便已开始在粤人中寻找白手套,而第一个与他们接触的人恰恰是陈炯明。在政治光谱中,陈炯明的思想位于中偏左,带有强烈的社会民主主义粉红色彩。在1911年独立战争中,他的循军所用的井字旗正象征着儒家“耕者有其田”的理念。他虽认同土豪自治,但亦十分关注社会平等问题,因而很早就对无政府主义和社会主义产生浓厚兴趣。1917年十月革命后,陈炯明和那时的许多理想主义者一样,并未认清布尔什维克的实质,因而对列宁赞誉有加,认为列宁是领导俄国人民反抗沙俄黑暗统治、为人类解放事业奋斗的英雄。1919年12月1日,陈炯明在闽南护法区创立《闽星》杂志,在发刊词中倡言“全人类社会主义”,这自然引起了列宁的注意。1920年春,列宁的特使路博来到漳州,将列宁的亲笔信交给陈炯明。在信中,列宁对陈称赞有加。5月10日,陈炯明回信列宁称:

\begin{quote}
布尔什维主义建立的新俄已开辟世界革命之新时代……布尔什维带给人类的是福音\footnote{刘仲敬:《陈炯明:庇护赤化的联邦主义土豪》}。
\end{quote}

第一次粤桂战争胜利后,陈炯明开始庇护共产主义者在南粤的活动。1920年12月,他将中共创始人陈独秀邀至广州,命其主管广东省教育达十个月之久。此外,他又和激进的共产主义者彭湃结为至交。彭湃系陈炯明同乡,比陈炯明小18岁。1918年1月,陈炯明曾资助83名男女学生留学英、法、美、日,彭湃即身在其列。1921年5月,彭湃自日本学成归粤,于故乡海丰成立“社会主义研究社”与“劳动者同情会”。8月,他前往广州向陈炯明说明情况,得到陈的全力支持,并被任命为海丰县劝学所所长。彭湃在任上斥退大批保守士绅,提拔激进左翼青年,并于1922年5月1日于海丰组织“五一节”大游行。游行队伍举着“赤化”旗帜高唱《国际歌》、高呼“劳工神圣”等口号,引起保守绅商恐慌,纷纷在报纸上攻击彭湃。迫于压力,陈炯明只得免去彭湃官职,但仍十分重用他。同年6月陈炯明起兵驱孙后,彭湃仍与他十分要好。7月,彭湃开始投身农民运动,在海丰创立了组织农民对抗田主的“农会”。1923年1月1日,拥有十万会员的海丰县总农会成立,彭湃出任会长。随后,农会运动向惠州地区和全广东扩散。5月,彭湃赴香港与陈炯明会面,得到了陈对农会的捐款。7月,广东省总农会成立,彭湃出任执行长,拥有会员13.4万员。同月下旬,海陆丰地区遭强台风袭击,彭湃乃于8月16日召开全县农民大会,以农会的名义通过了减租70\%的决议。次日,海丰县长王作新在士绅的要求下派军警突袭总农会,抓捕干部25人。彭湃当即于23日向陈炯明寻求帮助。陈则对农会的立场表示支持,并立即电令王作新释放被捕农会干部。至10月底,彭湃干脆搬进陈炯明在惠州的司令部居住。陈炯明更公开表示,他对农会“非常赞同”。至11月,农会在陈炯明的支持下迅猛发展,蔓延至潮汕地区,拥有10万会员的惠潮梅农会宣告成立。,县级总农会更发展到10个。12月底,彭湃自惠州返回海丰。因他是受陈炯明庇护之人,王作新和当地粤军将领只得对他表示欢迎。1924年1月下旬,陈炯明回到海丰,彭湃组织了数百名农会会员前往欢迎。陈炯明十分开心,当场表示:

\begin{quote}
工商学都有会,农民那可无会\footnote{转引自沈晓敏:《1924年前的彭湃与陈炯明关系探析》}。
\end{quote}

至此,陈炯明和马列主义者的关系已进入蜜月期。做为持社民立场的左派,此时的陈炯明尚未认清马列主义的实质,以为农会是和商会类似的阶级自治团体。由于同情乡邦底层民众的处境,他对农会大力支持,希望他们积极争取自己的利益。此时的他尚未意识到,马列主义者只以阶级为真共同体,而视阶级共治为修正主义、视民族主义为统治阶级欺骗本国工农送死的工具。为了进行阶级斗争,他们不惜彻底撕裂各个民族,以“无产阶级专政”压倒“土豪劣绅”和资产阶级。2月10日,陈炯明将彭湃、王作新与海丰的豪绅代表邀至私宅会面。彭湃激烈地表明了自己的所有观点,豪绅随即向陈炯明施压,王作新则称若陈炯明继续庇护农会,他将罢职。陈炯明终于醒悟了。他意识到,假如自己继续纵容彭湃,那么他深爱的南粤将被彻底撕裂,成为挣扎在阶级战争中的巨大屠宰场。3月21日,王作新奉陈炯明之命布告解散农会,海丰农会转入地下状态。彭湃因此对陈炯明深为愤恨,恶狠狠地称“此人非杀不可”。4月上旬,彭湃转赴广州加入中共\footnote{沈晓敏:《1924年前的彭湃与陈炯明关系探析》}。而此时,孙文的国民党已与中共合流,变成了共产国际的白手套。

自1922年6月被逐出广州后,孙文便计划寻求海外势力的帮助。然而,他的日本泛亚主义者老朋友此时已对他逐渐丧失信心,他便将目光转向了苏联。马列主义的极左翼立场正与他的反社会人格契合,而布尔什维克严密的组织又给他提供了将国民党改组为更彻底的暴力革命集团的范式。他虽不完全认同过于激进的马列主义,但却对其有极强的好感。1923年1月16日,孙文与苏联政府全权代表越飞在上海会面,讨论两党合作问题。26日,二人发表《孙越宣言》,声明国民党与共产国际将以“民国的统一之成功”为目标展开合作。2月,孙文重返广州,苏联的援助很快便源源到来。6月12—20日,在共产国际的指示下,身为共产国际支部的中共于广州召开三大,确立中共全体党员以个人名义加入国民党、国共建立“革命统一战线”的方针。10月,共产国际代表鲍罗廷来到广州出任顾问。在鲍罗廷的主持下,国民党开始改组为类似布尔什维克的政党。1924年1月20日,国民党一大在广州国立高等师范学院礼堂召开,李大钊、毛泽东、李立三等中共要员均出席会议。孙文遵从鲍罗廷的提议,于会上提出“联俄、容共、扶助农工”三大政策及打倒“帝国主义”、军阀的目标。28日,会议议决开办军校,模仿苏联建立党军。30日,会议选出以总理孙文为首的25人组成国民党第一届中央执行委员,旋告闭幕。“第一次国共合作”由是展开,国民党沦为共产国际颠覆远东国际秩序的白手套\footnote{《简明广东史》,页638—640}。6月16日,“中华民国陆军军官学校”在广州黄埔长洲岛成立,是为著名的“黄埔军校”。孙文任命蒋中正为校长,并以周恩来、叶剑英、恽代英等中共骨干担任军校要职。至于军校的教官和顾问,则是以苏军高级将领布柳赫尔(化名“加伦将军”)为首的数十名苏军军官。该校第一期学生共499人,系由广州、上海的两个招考点招来,多非南粤人。入学后,所有学生连同中共党员在内均集体加入国民党。1925年2月,该校更名为“中国国民党党立陆军军官学校”,赤裸裸地表明了该校乃党军军官的培训机构。经一年时间,该校共培养新军官约2000人,这些遂成为帮助苏联和孙文祸乱南粤、破坏国际体系的马前卒\footnote{《简明广东史》,页640—642}。

黄埔学生军很快就找到了练兵的对象,那便是饱受压榨的广州商民。早在1912年初,因城中民军横行,广州商人成立了自卫组织广州商团,以陈廉伯为团长。陈廉伯系继昌隆缫丝厂创始人陈启沅之孙,生于1884年。12岁时,他被祖父送往香港英式教育,在那里获得英国国籍。16岁毕业后,他进入汇丰银行广州沙面分行工作,后于1905年担任自家企业“昌栈丝庄”司理,并于1908年创立广东保险公司。因善于经营,年轻的陈廉伯很快成为巨富,于1910年加入广东总商会,成为广东商界的名人。出任团长后,陈廉伯亲自支垫商团经费,购置枪械,使商团有了自卫能力,因而能在龙济光、陆荣廷的暴政下维持广州商业的正常运转。1919年3月,商团仿照西式议会政治完成改组,设立了相当于议会的最高机构评议会及十个分团,陈廉伯当选新一任团长。评议会有23名评议员,任期一年、可连任,期满后需重新选举。商团的一切重要事宜均需经评议会同意方可执行。评议会分常会、临时会议两种,其中常会每月一次,遇要事则召开临时会议。商团团长、分团团总及团友满三人以上,即可向评议会提出议案。须有三分之二以上评议员出席会议及三分之二以上人数通过,议案方可通过。团长仅有执行评议会议决事项之权,如此事实属窒碍难行,团长可将议案交评议会复议。如复议仍维持原状,便要召开全体团友大会解决。如遇紧急事项,团长可紧急处置,但应尽快交评议会追认。改组之后,商团的评议会制度运作良好,各评议员皆庄重履行职权,从无故意缺席者。同年8月,广东省商团军成立,陈廉伯当选商团军总长。商团有了自己的正规军队,遂能执行救火、缉捕盗贼、维持城市治安等任务,因此大受市民欢迎。他们甚至还有专业的西式军乐队,每逢春节便在城中举办音乐会,与民同乐。1920年11月,因第一次粤桂战争后广东民生艰辛,商团向东江地区拨赈款25万元。1921年夏,商团又成立“两粤兵灾救济会”,陈廉伯当选会长,随即创办得志婴院。1922年6月,在陈炯明起兵驱孙的短暂混乱中,商团军积极支持粤军,在炮火中主动走上街头维持治安,极大地提高了声望。至1924年,广州商团人数已达1.3万,商团军则有6000余人\footnote{叶曙明:《国会现场:1911—1928》,页276—277}。当时,一位商团团友曾指出:

\begin{quote}
今日商界之要务,不在于资本不集,不在于商业知识之不充;所当速起以求者,即为团体之结合,参政之能力,政治实权之操握。务使政与商联为一气,若臂使指,运棹从心\footnote{叶曙明:《国会现场:1911—1928》,页278}。
\end{quote}

此段论述所反映的,实为1920年代初流行于东亚各邦的“商人政府”思潮。乱世之中,各邦土豪纷纷将资本化为政治力量,力图主导政局,而南粤的广州商团正是其中的佼佼者。南粤有着悠久的自治传统,佛山的自治史便是明证。在文明开化之后,粤人便能自然地组织起广州商团这样的伟大组织,用自己的双手守护家邦。1923年初孙文重返广州后,城市治安再次混乱起来。孙文下令解除赌禁、烟禁,使社会乌烟瘴气。滇、桂军和伪军则在城中横行霸道,时常当街掳人当兵。是年4月7日,有士兵、便衣十余人手持绳索,在云来茶居门口拉夫,凡出入茶楼饮茶者尽被绑走。商团立即派兵驱赶士兵、便衣,使更多无辜市民免于受害\footnote{宋其蕤、冯粤松:《广州军事史》下,页208}。

1924年1月国共勾结后,国民党露出了更为凶残的面目。是年5月下旬,广州市政厅忽然发布“统一马路业权”法案,擅自增收“铺底捐”,引发广州商民强烈抗议。5月至8月,广州商界接连发动罢市,国民党毫不退让。忍无可忍之下,陈廉伯于8月初开始向英商购置枪械,准备武力自卫。8月10日,商团的一批枪械被蒋中正所乘军舰扣押于天字码头。此批枪械已在事前向国民党报告,但孙文仍下令扣押。12、15日,商团军代表千余人两赴大元帅府请愿,要求发还被扣军火,孙文置之不理。22日,在商团的号召下,南海、番禺、顺德、台山、东莞、增城、新会、清远、高要、曲江、罗定等20余县属内的138个城镇发动联合罢市,广州城内商铺闭门者达十分之九。面对民众的怒火,有苏联人撑腰的孙文毫不忌惮,反于24日下令广州戒严。商团则针锋相对,继续做出对抗姿态。9月1日,孙文发表宣言,称商团hay 受“英帝国主义”煽动、准备颠覆革命政府的团体。15日,陈廉伯通电反驳,否认商团军图谋推翻政府。孙文则置之不理,开始制定武力镇压计划,购置了三百箱用于焚城的煤火油。至10月4日,因军火仍未发还,商团联络广州百余城镇代表于佛山集会,决定进行第二次罢市。7日,苏联援助国民党的首批武器弹药抵达广州,孙文已做进攻商团的准备。9日,为麻痹商团,他故意发还部分被扣军火。10日,在共产国际的指使下,中共广州地委组织起五六千人的队伍高呼“打倒商团”、“杀陈廉伯”、“拥护革命政府”等口号沿街游行。下午3时,游行队伍行进至太平南路西濠口,见商团军正在河边起卸发还的军火,激进学生遂一拥而上,强抢枪械。商团军被迫开枪还击,当场击毙数人\footnote{叶曙明:《国会现场:1911—1928》,页278}。这些共产国际第一次在南粤使用以游行“群众”为人肉盾牌的卑鄙战术。在此情形下,若商团军不开枪,便必然会成为俎上鱼肉;若开枪,则会被国共两党顺理成章地抹黑为“帝国主义血腥镇压人民的走狗”。面对此种无赖手段,除奋起战斗外,商团军别无选择。

入夜时分,陈廉伯下令各分团团军于14日下午5时至西关集结,计划在15日拂晓出兵攻下广州城内各重要机关。次日拂晓,商团军封锁西关城区、构筑街垒、散发反孙传单,伟大而悲壮的广州商团起义拉开序幕。同日,孙文开始与鲍罗廷制定镇压计划。12日,孙文仿照苏联模式成立所谓的“革命委员会”做为镇压商团的指挥机关,自任会长,以蒋中正、汪兆铭、廖仲恺、许崇智、陈友仁、谭平山六人为全权委员。13日,大批湘军、伪军开入广州城。14日,孙文发出“平叛”手令,任命蒋中正为总指挥,限于24小时内结束战斗,蒋中正随即将黄埔学生调入市区。这时,商团军已在西关完成集结,大战即将爆发。15日凌晨4时,“革命委员会”发布总攻击令,蒋中正指挥黄埔学生军、滇军、湘军、伪军、工团军、农团军一齐发动进攻。商团军在暗夜中奋起还击,战斗打响。在太平门、普济桥一带,双方激烈交火,反复拉锯,战况异常惨烈\footnote{叶曙明:《国会现场:1911—1928》,页278}。至天明时分,商团军退守西关内围,依托街栅顽强抗敌。上午11时,蒋中正因屡攻不下,遂命理发工人抬出早已准备好的三百箱煤火油,灭绝人性地放火焚城。西关人烟辐辏、房屋密集、建筑多用木材,火势遂迅速蔓延,将西关笼罩在一片烈焰中。下午2时许,商团军在大火中逐渐停止抵抗,大批军民惊恐地四处逃生。侵略军则冷血地向街上的逃生者射击,将他们不分军民地成批杀害。在太平马路,一大群难民被侵略军逼退回火海中,全部惨遭焚死。混乱之中,陈廉伯突出重围,乘船忍痛流亡香港。商团副团长陈受恭为保住部下的生命,下令全军投降。然而,灾难还在继续,深夜时分,西关的大火仍在燃烧,甚至连16公里外的佛山都能看到火光\footnote{《西关屠城事件》}。无数南粤军民悲呼着,在熊熊大火中失去了生命。如此惨绝人寰的景象,大概只会出现在地狱中。但人们的惨叫声却表明,这里是人间,是国民党和共产国际制造的人间地狱!

17日晨,大火终于自行烧尽。据香港报刊统计,西关遭焚烧的大小街道达30余条、商户达2000余家、损失达5000万元以上。此外,还有上千家商户遭到侵略军洗劫。然而,侵略军的暴行仍未停止。当日白天,侵略军仍在惠爱西路一带滥杀无辜,将貌似商人的路人一律屠戮。至18日,一切终于结束了。侵略军们在城中大肆庆祝,数以千计的尸体则扭曲着倒在西关闷烧的废墟中。据孙文政府发布的统计数字,此次“平叛”,“政府军”死伤仅100人、商团军死伤300人、平民死亡1700—1800人\footnote{叶曙明:《国会现场:1911—1928》,页278}。这一数字无疑是他们为掩盖自己的罪行故意大大缩小的,因为西关是广州城最繁华、人口最密集的区域,真实的平民遇害数恐怕是个令人毛骨悚然的巨大数字。西关大屠杀是国共两党对南粤人犯下的不可饶恕的罪行。它如同一个巨型标志,不但象征着南粤“商人政府”的消逝,更象征着南粤人在两党治下遭受的苦难,鞭策着南粤人为自由和尊严不停奋战。

制造西关大屠杀后,孙文狂妄得不可一世,自恃有苏联支持的他已不把广东人放在眼里。在广西,由国民党扶植的新桂系也在逐步取得控制权。自1922年5月粤军叶举部撤离广西后,各地的残余桂军、土匪纷纷打着“自治军”蜂拥而出,互相火并,全桂一时大乱。5月12日,曾被陈炯明寄予厚望的省长马君武逃出南宁,于十日后在梧州宣布辞职,旋即逃亡上海。原桂军沈鸿英部本曾在第二次粤桂战争中倒戈。广西大乱后,他趁机起兵,于11月占据柳州、桂林、梧州等地。1923年11月,不甘寂寞的陆荣廷趁乱回到广西,被北京政府任命为“广西全省善后督办”,在旧部的拥立下重入南宁,控制龙州、庆远等地。与此同时,在十万大山中,一股崭新的势力正在崛起。在第二次粤桂战争中,有个名叫李宗仁的桂军军官统领率千余部下向粤军投降,被陈炯明派往十万大山招降桂系残军。李宗仁遂邀其旧时同窗白崇禧、黄绍竑共事,很快将军力扩展到6000余人。到1922年5月,李宗仁见广西大乱,便自任为“广西自自治军第二路总司令”,控制了玉林、容县等七县市。1923年4月,沈鸿英在广州对孙文开战。白崇禧乃于6月赴广东与孙文会面,表示愿与国民党共击沈军。孙文大喜过望,将“广西讨贼军”的番号给了他们。黄绍竑当即自任总司令、白崇禧自任参谋长,以“广西讨贼军”的名义对沈开战,于7月下旬攻占梧州。接着,李宗仁将所部改称“定桂军”,配合白、黄二人作战。9月,孙文向李、白、黄三人拨款2万元、接济2万发子弹。10月,三人率部相继攻占平南、江口、贵县、桂平,并宣誓加入国民党\footnote{《广西通史》第三卷,页40—42}。这样,由国民党一手扶持的“新桂系”便在乱局中诞生了。

到1924年3月,新桂系已控制广西最富庶的梧州、浔州、玉林三府之地。他们自视为新兴革命势力,意图一统广西。此时,陆荣廷、沈鸿英间爆发激战。陆荣廷自南宁率部北上,攻取桂林,旋被沈军合围。李宗仁遂趁势于5月23日发布倒陆宣言,于6月25日率部以沈鸿英友军的身份一举攻占防守空虚的南宁。8月,陆荣廷自桂林突围而出,率残部逃往全州。在沈军追击下,他又逃到湖南永州,于10月9日通电下野,而后流亡上海。曾经强横一时的旧桂系,至此完全退出历史舞台。与此同时,新桂系则大举接收旧桂系在桂中、桂西、桂南的大片地盘,几乎没有遇到有效抵抗。沈鸿英见其发展迅猛,乃与滇督唐继尧结盟,于12月1日对新桂系开战。1925年1月30日,双方各自兵分三路全面开战,沈军迅速败北。2月7日,新桂系取桂林。14日,李宗仁、白崇禧进驻桂林。15日,新桂系取贺县。16日,沈鸿英率残部逃至广东连山,通电下野,而后只身逃亡香港,从此退出军政界\footnote{《广西通史》第三卷,页42—48}。

然而,广西的战事仍未结束。1925年2月初,野心膨胀的滇督唐继尧倾全滇之力,动员近7万大军进攻广西,图谋占据全桂。滇军兵分两路,一路由第五军总司令龙云为总指挥,兵力3万人,出广南、富州进犯百色;一路由唐继尧自率,兵力3.8万人,出贵州容江、湖南洪江、靖州入桂,沿融江直取桂林,“第一次桂滇战争”爆发。2月28日,龙云一路趁虚侵占南宁。新桂系仅有2万兵力,形势异常危急。新桂系虽已投靠国民党,但他们毕竟属于广西本土势力。他们虽然认同孙文的“革命”,但仍保持着对乡土的热爱,因而能得到全桂民众的支持。3月7日,李宗仁发表通电,号召全桂“助拒滇唐”,得到桂人的热烈响应。人们视新桂系为自己的子弟兵,纷纷援助他们抗敌。20日,国民党亦通电讨唐,派出一部滇军支援新桂系,孙文与唐继尧遂正式决裂。28日,李、白、黄在桂平誓师,随后分左右两路进攻南宁。4月初,右路桂军旗开得胜,于八塘、二塘歼滇军千余人。与此同时,左路迅速兵临南宁城下,龙云据城死守,双方在城郊展开激战。至5月中旬,桂军仍未拿下。而这时,唐继尧一路滇军已经入境侵占柳州,沈鸿英残军亦趁乱夺取平乐、桂林。桂军三面受敌,遂在南宁城下分兵,以李宗仁继续围攻南宁、黄绍竑率主力救援柳州、白崇禧堵截沈军。5月底,黄绍竑以奇袭收复柳州、白崇禧亦已剿灭沈军。6月4日,黄、白合兵一处,于滇军大战于沙浦。至8日,滇军全线败退,折损兵力过半,狼狈逃回滇境。黄、白乃反身南下,与李宗仁会攻南宁。7月7日,龙云率部放弃南宁突围,沿左江退回滇境。至此,新桂系终于统一广西,黄绍竑于9月5日在南宁就任广西民政公署民政长,从而确立了新桂系对广西的支配权\footnote{《广西通史》第三卷,页48—50}。在抗击滇军的战争中,新桂系得到全桂民众的热情支持,遂能以内线作战三面迎敌,以寡胜众。第一次桂滇战争固然是八桂南粤抗击外敌的一次伟大胜利、是南粤史上的一次光荣战争,但它亦为国民党提供了一个安全的后方,使其能专心动向进攻陈炯明。我们虽应承认新桂系的本土性,但也要认识到它毕竟只是国民党中的一股地方集团。它虽然能为桂人保留一定利益,但不会为桂人乃至整个南粤赢得真正的自由。在对陈炯明的态度上,它与孙文、蒋中正之流并无太大区别。

1924年下半年,随着国民党赤化日深,其实力迅速膨胀。与此同时,岭北政局也在朝着有利于孙文的方向发展。1923年10月6日,直系头目曹锟以贿赂方式当选总统。10日,国会通过《中华民国宪法》,是为民国立国以来的首部正式宪法。曹锟虽以贿当选,但他毕竟是由合法国会选出的,国会颁布的宪法也是具有法统性的。然而,这部宪法没过多久就夭折了。1924年9月15日,张作霖集结15万奉军入关讨直,吴佩孚以20万人应战,第二次直奉战争爆发。23日,直军将领冯玉祥开入北京发动“首都革命”,逼迫政府将吴佩孚解职,并软禁曹锟,自立所谓的“摄政内阁”,拥段祺瑞主持军事。30日,奉军攻占秦皇岛,直军全线溃败。11月24日,段祺瑞自任“中华民国临时执政”,又于12月13日下令取消《中华民国宪法》、宣布《临时约法》无效\footnote{刘仲敬:《民国纪事本末·虞渊篇》}。自此,民国法统彻底断绝,由东亚独立各邦共建的中华民国在事实上灭亡。11月10日,孙文应冯玉祥电邀北上,讨论法统断绝后的分赃问题。1925年1月1日,孙文抵达北京,随即因肝癌一病不起。3月12日,这个恶贯满盈的南粤叛徒和野心家在北京的病床上结束了他罪恶的一生\footnote{刘仲敬:《民国纪事本末·革命编年史》}。

孙文虽死,但国共两党仍盘踞着南粤,南粤的苦难还远未结束,陈炯明的抗争仍在继续。1924年12月7日,陈炯明在汕头被广东各团体代表推举为“救粤军总司令”。24日,在苏军顾问加伦的要求下,国民党中执委成立由胡汉民、廖仲恺、许崇智、蒋中正、杨希闵组成的“军事委员会”,开始准备东犯。30日,“军事委员会”在加伦的主持下做出东犯计划,决定以杨希闵指挥滇军为右路、刘震寰指挥桂军为中路、许崇智、蒋中正指挥伪军、黄埔校军为左路分头进攻,总兵力达4.4万人。1925年1月7日,陈炯明见敌人即将大举来犯,遂向粤军下达总动员令,要求分左、中、右路,由洪兆麟、叶举、林虎指挥向广州推进,总兵力为4万人。陈炯明的计划,是先以左路击破敌军兵力最多的右路,之后“一战而下广州”,但他严重低估了黄埔校军的实力。黄埔校军虽然只有4000兵力,却是一支由苏联军官训练、装备先进苏式武器的劲敌。该军由苏联教官和黄埔学生组成,由加伦将军及校长蒋中正、党代表廖仲恺、政治部主任周恩来指挥,下设两个教导团,并配备了齐全的炮兵、工兵和辎重兵。侵略军的作战计划,正是以左路滇军吸引粤军中、右两路的进攻,而以右路校军击破粤军左路,沿海岸线推进至潮汕。2月1日,所谓的“东征革命军”开始行动,然因杨希闵、刘震寰的左、中路皆缺乏战意,只有右路伪军和校军在向前推进,于9日陷东莞、10日陷平湖、11日陷深圳,控制了广九铁路沿线。13日,伪军、校军兵临淡水城下。同日,粤军左路前锋数百人进驻淡水。淡水系惠阳县重镇,城高壕深,周围地形开阔,可谓易守难攻。14日,蒋中正召集军事会议,决定以校军组织敢死队担负责攻城,狂热的校军官兵随即踊跃报名。校军选出了10名军官(其中8人为中共党员)和210名士兵,他们都是些具有革命狂热的亡命徒。周恩来亲自向他们做政治报告,要求他们为革命献身。15日拂晓,在炮兵的掩护下,状若疯魔的敢死队员一拥而上,踏着同伴的尸体冲至城墙下。由于携带的竹梯未能跟上,他们便以人梯搭上城墙,用不到两小时就攻下了淡水。伪军、校军继续推进,于19日与洪兆麟所部左路粤军主力交火。不到三小时,粤军便不支败退。25日,粤军发动反攻,在三多祝击败伪军。但次日,校军、伪军联合反攻,粤军再次战败。27日,许崇智、蒋中正分率伪军、校军分两路逼近海丰。粤军腹背受敌,只得于29日弃守海丰。伪军、校军乘胜东犯,于3月7日攻陷汕头。接着,他们又向北进攻,于13日大破右路粤军于棉湖、17日陷五华、21日陷兴宁\footnote{宋其蕤、冯粤松:《广州军事史》下,页209—210}。至此,在短短一个多月的时间里,不足万人的右路侵略军便击败了击溃了三路粤军。

粤军之所以败得如此迅速,并非因为官兵贪生怕死。粤军是一支由陈炯明亲手带出来的南粤子弟兵,他们曾在1920年解放广东、在1921年攻下广西、在1922年赶走孙文、在1923年取得第一次惠州保卫战的胜利,是一支光荣而顽强、信奉“粤人治粤”主义的军队。然而,全副苏械、由苏联教官指挥、拥有革命狂热的校军却是极为可怕的对手,可以被看作一支苏军。粤军的敌人不仅是蒋中正和许崇智,更是共产国际与赤化的国民党。每次战斗,粤军将士都被敌人的强大火力压得抬不起头。敌人不但装备着苏式铁甲火车和机枪,更有远强于粤军的火炮。当时的香港报刊曾如是说:

\begin{quote}
此次东江粤孙两军之战,换言之,即粤军与中俄两国共产党之战也……(孙军装备的)此种大炮共有二十八尊,所配弹药由夥,口径为九生半,且属快炮一类。而粤军所用者为七生半,口径相差二生,故威力较逊,每次战争均为炮火所迫,不得不退\footnote{宋其蕤、冯粤松:《广州军事史》下,页210}。
\end{quote}

至3月下旬,粤军林虎部退入赣南,叶举、洪兆麟部退入闽南,杨坤如则仍在坚守惠州孤城。由于陈炯明提倡“联省自治”,赣督方本仁、闽督周荫人皆愿与他合作,遂为粤军提供休整之处。面对空前强大、空前疯狂的敌人,粤军将士仍在坚持战斗,绝不投降,这不能不使我们感到由衷钦佩。5月21日,见粤军主力已退出广东,许崇智、蒋中正便率兵返回广州。这时,野心膨胀的杨希闵、刘震寰二人已在策动兵变。6月3日,侵略军中再内讧再起,杨、刘二人率滇、桂军起兵,于次日控制广州各主要机关。5日,大元帅府免去二人的滇、桂军司令之职,下令各部重夺广州。滇、桂军自然不是校军的对手。11日,两军惨败,杨、刘于12日逃亡香港。15日,滇、桂军全数被歼,杨、刘政变告终。同日,国民党中执委决定将大元帅府改组为“国民政府”、将麾下各军编为“国民革命军”。7月1日,以实行“打倒列强、除军阀”、实行所谓的“国民革命”为目标的国民政府建立,汪兆铭出任主席,与胡汉民、谭延闿、许崇智、林森同任常务委员。6日,国民政府军事委员会成立,汪兆铭任主席,胡汉民、伍朝枢、廖仲恺、谭延闿、许崇智、蒋中正任委员。8月26日,军事委员会将各部整编为国民革命军第一、二、三、四、五军,由蒋中正、谭延闿、朱培德、李济深、李福林分任军长。各军仿照苏联红军,均设立政治部和党代表。国民革命军是国民党实现其革命目标的党军,实为国民党的专职打手\footnote{《简明广东史》,页658}。日后横行于东亚各邦的恶魔军队国民党军,至此诞生。与此同时,蒋中正在国民党内的地位迅速上升。8月20日,国民党左派大佬廖仲恺在中央党部门外遇刺身亡,蒋中正、汪兆铭乃借机打击立场较右的胡汉民、许崇智。25日,胡汉民被蒋派兵逮捕。9月20日,蒋又命黄埔学生军围搜伪军司令部,逮捕其大批亲信,迫使许狼狈逃亡上海,从此淡出政坛。23日,国民党又命胡汉民离粤赴苏联考察,将其驱出权力核心\footnote{《广州百年大事记》,页323}。至此,蒋中正不但借助廖案接连搬倒党内大佬、清洗伪军高层,更使其成为一颗受苏联顾问赏识的左派新星。

在此前后,国共两党开始着手撕裂南粤社会。1925年5月30日,上海发生工潮,被日本厂方和英国巡捕镇压,死亡十余人。国共两党立即大做文章,通过中共组织的中华全国总工会和香港海员工会发动规模空前的“省港大罢工”,矛头直指英国。在香港,中共党员苏兆征(香山人)组织起15万人参加罢工。在广州,沙面租界的工人随之响应。6月23日,在国共两党的策划下,粤港澳的各界团体共10余万人在广州举行了声势浩大的游行示威,胡汉民、汪兆铭、廖仲恺等国民党要员均在现场进行煽动。苏联顾问和黄埔生混在游行队伍中,随时准备挑起事端。下午3时,当队伍行进至沙面对岸的沙基西桥口时,苏联顾问指使黄埔生向沙面岛上的英法军队开枪。英军、法军当场以机枪还击,当场打死52人、打伤170余人,死者中包括两名苏联人、黄埔师生27人。共产国际的“人肉盾牌”战术再次奏效了。事后,国民政府贼喊捉贼地谴责英方,称之为“沙基惨案”,大肆煽动粤人的仇英情绪,并于7月6日成立省港罢工委员会,组织起有2000余人的工人纠察队拦截所有流向香港的货物。至7月8日,已有十三、四万人因恐惧逃离香港。至年底,香港已有半数商户倒闭,港英政府只得向伦敦借款三百万英镑以渡过难关。国共两党操纵的工纠队横行街头,以武力逼迫工人罢工。值此危急关头,部分港商在陈炯明旧部下梁永燊的主持下成立了工业维持会,通过以暴制暴的方式反击工纠队,保护上工工人,使香港社会秩序不致崩溃。次年9月22日,国民政府终于下令停止罢工,历时年余的“省港大罢工”结束。此次罢工虽然未能动摇香港社会的根基,却严重撕裂了南粤社会。工纠队与工业维持会在香港街头的激烈冲突表明,南粤社会各阶级已处在全面开战的边缘\footnote{徐承恩:《躁郁的城邦:香港民族源流史》,页236}。很快,这场战争便会以异常血腥的方式上演。

在此期间,退入赣、闽的粤军补充饷械,逐渐恢复了元气。他们趁侵略者忙于内讧的机会重新入粤,收复了梅州、潮州、惠州一带的大片土地。陈炯明更得到港英政府帮助,获300万发子弹支援。1925年9月16日,陈炯明坐镇香港,开始策划反攻事宜。21日,国民政府军委会任命蒋中正为“东征”军总司令、汪兆铭为党代表、周恩来为政治部主任,将第一、第四军近3万人编为三个纵队,分由何应钦、李济深、程潜指挥。10月1日,在粤军未及反击之际,“东征军”开始进攻,以第二、一、三纵队分为左、中、右路,向惠州扑去。其中,第一纵队实力最强,系由国民党第一军编成\footnote{《广东通史》现代上册,页218}。该军有三个师1.5万人,大部分部队由军衔为少尉至上校的苏联军官指挥,第一师的指挥官甚至是加伦将军本人。此外,该军机关枪团有机枪120挺,更有14架苏军飞机助战\footnote{宋其蕤、冯粤松:《广州军事史》下,页211}。为阻击敌人,陈炯明将粤军主力1万余人集中于惠州。杨坤如率全城军民同仇敌忾,下定了死守的决心。10日,敌人三路纵队合围惠州城,于次日占领飞鹅岭高地。蒋中正决定以第一纵队执行攻城任务。13日上午9时,攻城开始,第二次惠州保卫战打响。在苏军战机和苏式火炮的猛烈轰击下,守军伤亡惨重。下午,国民党军在苏联军官的指挥下向惠州城北门、西门发起冲锋,以北门为主攻方向。粤军凭城死守,打退了敌人一次又一次的疯狂进攻,将进攻北门的敌第四团杀伤过半,击毙其团长刘尧辰。当夜,蒋中正因部队伤亡惨重大发雷霆,准备放弃攻城,绕道进攻,被苏联顾问否决。14日下午2时,敌人继续攻城。在敌炮兵的狂轰滥炸下,我方机枪阵地接连被毁,官兵损失惨重。4时30分,敌第四团、第一团分别向北门、西门发起冲锋,北门很快失守。西门方面,敌军中最为狂热的中共党员冲锋在前,引领各部突破了城防,惠州遂告陷落。此战,粤军主力全军覆没,被俘5000余人。敌人亦付出沉重代价,仅黄埔生便有700余人阵亡。在战斗的最后关头,杨坤如身负重伤,在残部的保护下拼死突围,向河源撤退,总算没有落入敌手\footnote{宋其蕤、冯粤松:《广州军事史》下,页211—212;《广东通史》现代上册,页219—220}。 

惠州失守后,粤军大势已去。国民党军快速推进,于20日陷河源、22日陷海丰。27日,志得意满的蒋中正率第一军第三师猖狂冒进,在五华县华阳附近与粤军林虎部突然遭遇。粤军早已在此构筑了坚固的工事,虽奋勇攻击,将猝不及防的敌第三师彻底击溃,击毙其团长、副团长、连长等军官8人、党代表8人。在一片慌乱中,敌第三师代理师长陈赓(中共党员)背着蒋中正仓皇渡河逃命,丑态毕现。这是粤军在卫粤战争中的最后一场胜利。11月1日,重整旗鼓的敌人再次进攻,10月30日击溃林虎部,俘虏4000余人。11月1日,洪兆麟部惨败于河婆,洪腿部中弹,部队丧失战斗力。11月6日,汕头失守。7日,粤闽边境的饶平沦陷,洪兆麟、林虎率残余粤军全部退往闽南\footnote{宋其蕤、冯粤松:《广州军事史》下,页212}。至此,国民党军的第二次“东征”结束,粤东完全沦入敌手。
11月7日,侵略军与苏联顾问在汕头举行了庆祝俄国十月革命五周年纪念日的集会。会上,蒋中正发表演说,向苏联大肆献媚:

\begin{quote}

我们应该庆祝这个纪念日,因为俄国革命的成功,即是中国革命的成功。如果俄国革命失败了,我们今日便不可能有这个革命。让我们欢呼:“俄国革命成功万岁!”“中国革命成功万岁!”“世界革命成功万岁!”“汕头人民万岁!\footnote{转引自宋其蕤、冯粤松:《广州军事史》下,页212}”

\end{quote}

粤东的战事已经全部结束。但在海南岛和粤西,还有一支粤军仍在奋战,他们的指挥官是粤军将领邓本殷。邓本是陈炯明的部属。1922年6月陈炯明驱逐孙文后,委任邓为琼崖善后处长。1923年1月,广州沦陷,邓本殷遂趁势联合高州、雷州、罗定、阳江、钦州、廉州、琼崖诸军,自任“南路八属联军总司令”,成为一独立于陈炯明之外的势力。1925年10月24日,国民党军与李宗仁、白崇禧的桂军联手攻邓,邓军节节败退。30日,开平失守。31日,台山沦陷。11月2日,恩平失陷。7日,阳江陷落。12日,高州沦陷。15日,罗定不守。至21日,化州沦陷。30日,廉州陷落。12月7日,桂军攻陷钦州。至此,邓军全线崩溃,残部撤往海南岛。国民政府以李济深为“南征军总指挥”,命其率国民党第四军两个师渡海攻岛。1926年1月17日,国民党军乘军舰横渡琼州海峡,于琼东北海岸的新榄港登陆。邓军反登陆作战失败,在敌方猛烈的炮火下纷纷溃退。22日,琼山县城失守。邓本殷见大势已去,遂抛下军队,乘日本军舰流亡越南。至2月中旬,海南岛的残余邓军彻底覆灭,南粤大地上最后一支抵抗国共两党的粤军不复存在,国民党宣告“两广统一”,三年零八个月的卫粤战争迎来了黑暗的结局\footnote{《广东通史》现代上册,页227}。3月13日,李宗仁致电广州,表示广西将完全服从国民党和国民政府领导。6月1日,国民党任命黄绍竑为广西省主席\footnote{《广西通史》第三卷,页56—57}。南粤,彻底落入恶魔血红的手中。

在本章最后,笔者将花费一些笔墨,向大家介绍粤军诸将的结局。南粤沦陷后,死守惠州城的英雄杨坤如发誓绝不投降。在部队被打散后,他经河源、横沥潜至香港,于1936年在平静中病逝。叶举亦自闽南转赴香港,此后一直生活在那里,直到1934年去世。与他们二人相比,洪兆麟和林虎的命运则远更令人唏嘘。洪兆麟系湘人,曾为清军军官,于1911年独立战争后归附陈炯明。他在兵败后万念俱灰,不再对南粤有任何留恋,遂乘船前往上海,准备投奔北京政府。1925年12月,在赴上海的船上,他被国民党派出的刺客枪杀,结束了曲折的一生。林虎系桂人,他在战败后避走上海,后又移居法国。1929年,心灰意冷的他回到家乡广西陆川,埋头经营起自己的农场,不再过问政事。此后,他在1938年和1956年分别参加国民党的国民参政会和中共的广西政协,沦为国共两党的统战花瓶。1960年,他以73岁高寿寿终正寝。而那时,他的乡邦正饱受饥荒折磨。

最后,让我们将目光集中到陈炯明身上。1925年10月10日,由美洲粤侨主导的“致公党”在美国旧金山成立,陈炯明被该党推举为总理。该党前身为“美洲洪门致公堂”,本出自天地会,但随着粤人的大航海变成了维护海外粤人社区利益的组织。早在1904年时,孙文曾为在粤侨中争取支持加入这个组织。但现在,粤侨们发现孙文已成了叛徒,陈炯明才是真正的英雄。陈炯明并没有去美国过逍遥自在的党魁生活。他实在太热爱自己的乡邦南粤,不忍离开它,只能与自己的同胞居住在一起。此后,他一直在香港过着拮据的生活,每晚睡在一张行军床上,甚至经常三餐不继。他是个物欲不强的人,也对女色兴趣不大,一生所追求的只有理想。在理想幻灭后,他已对残生没有多大兴趣,心中所思唯有对孙文与蒋中正的仇恨。只有在与叶举、杨坤如等老部下聚会、回忆昔日的戎马岁月时,他才会感到一丝快乐。1933年9月22日,陈炯明在遗憾与悲凉中病逝于香港,享年55岁\footnote{关于粤军诸将结局,参见张磊:《孙中山辞典》、王俯民:《民国军人志》相关条目}。我相信,在去世前的弥留之际,他的眼前一定浮现着故土的乡野风光。那里有他热爱的南粤土地、有他曾拼命守护的无数南粤百姓。至于致公党在他死后因对国民党的仇恨而转身投入中共的政协,则是他始料未及的事了。不知他的在天之灵听闻此消息,还能否安息?

陈炯明是非常坚定的社民党式粉红左派民族主义者,具有左派所特有的天真、虔诚与理想主义,因而既仇视孙文,也仇视随意破坏宪法的北洋军阀。他相信左派口中的“人民”概念,认为只有人民自治才能达到真正的自由,而联省自治、东亚各邦停火就是人民自治要达成的目标。为此,他做为一个粤人,必然首先维护故乡广东的自治,并力图帮助邻邦桂、闽自治。他对孙文的反抗,与毕苏斯基对红军的反抗类似,都是民族主义粉红左对普世主义赤左的战斗。左派对人类普世精神的信奉和自我牺牲精神,使他一次次放弃权力的诱惑,为守护南粤进行了最为勇敢的战斗。他一生都没有打过侵占邻国的不义战争,正如上杉谦信从未觊觎过京都的土地。他一生都没有背叛自己的祖国,正如罗伯特·李将军没有背叛弗吉尼亚。他一生都没有背叛自己的左翼信仰,正如约翰·布朗为自己的理念殉道。如刘仲敬先生所说,“左派发明民族,右派继承国家。”陈炯明的意义早已超脱了左右,他是南粤共同体的伟大象征。









