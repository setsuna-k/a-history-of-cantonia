\chapter{第三次北属}

\section{连绵不绝的反抗:赵妪、吕兴、郭马}

公元226年,士氏灭亡,南粤第三次沦入帝国之手。在第三次北属初期,南粤归孙吴统治。对孙吴而言,新被征服、出产各种珍宝的南粤系一绝佳的榨取对象。南粤的明珠、香药、孔雀等珍宝皆受到孙吴权贵d 喜爱。这些东西或为南粤土产,或由海外进口而来,皆为粤人神圣不可侵犯的财产。然而,孙吴却强迫粤人大量“贡纳”这些珍宝,以供皇室与官员享乐。此外,孙权还曾以南粤的“珠玑、翡翠、玳瑁”与魏使交换战马。孙吴帝国对南粤人力的征调亦非常残酷。吴景帝孙休曾一次性将交趾郡上千工匠抽调至建康劳作。我们骁勇善战的祖先甚至还被强迫服孙吴的兵役,为孙吴的争霸战争充当炮灰,成批战死于与家乡相隔数千里的岭北之地\footnote{《广东通史》上册,页312—313}。

面对暴政,我们的伟大祖先当然不会引颈受戮。公元248年,交趾、九真及粤西之高凉发生了大规模反吴起义,起义领导者系九真安县女子赵妪。赵妪起事时,她的兄长曾经劝她嫁人、不要“作乱”。赵妪对此的答复是:

\begin{quote}
	吾欲乘劲风,踏恶浪,斩决东海长鲸,荡平域内,救民出水火,决不效法俯首屈膝之辈做人妪妾\footnote{陈重金:《越南通史》,页33。}!
\end{quote}

在战场上,赵妪乘大象、披金甲,所向披靡。孙吴交州刺史陆胤见起义军势大,不敢与之正面交锋,遂采取阴毒的“招纳”政策,首先导致“高凉渠帅黄吴等支党三千余人皆出降。”接着,陆胤又率军向南推进,剿“抚”兼施\footnote{《广东通史》上册}。经五六个月的战斗,兵败的赵妪不愿做帝国的降虏,壮烈自尽。这场悲壮的起义,就在样被孙吴镇压了下去。直到今日,赵妪仍是越南著名的巾帼英雄,亦应被南粤人永远敬仰。

赵妪战死了,但粤人的反抗绝不会停止。很快,我们勇敢的祖先就发动了一场更大规模的起义。当时,在位的孙吴皇帝系吴后主孙皓,乃一奢侈残暴之人。为满足物欲,孙皓大兴宫苑,“功役之费,以亿万计”\footnote{陈寿:《三国志·吴书·孙皓传》}。公元263年,不堪忍受的交趾郡人在郡吏吕兴的带领下发动了新的起义。九真、日南两郡之民立即响应,三郡一同遣使向魏国投降,宣布脱离孙吴的统治,由此展开了孙吴与魏晋间长达九年的南粤争夺战(266年,西晋取代曹魏)。为便于指挥战争,孙吴于264年第二次分治交州、广州,南粤与越南在行政关系上就此永远脱离,然而两者之间的真正分离尚要等到六百多年后\footnote{《广东通史》上册,页313}。

孙吴与魏晋在南粤的争霸战争是空前惨烈的。九年之间,十余万外族军队在南粤的土地上疯狂厮杀,导致无数南粤百姓惨死于两军的屠刀下。公元271年七月,吴军经长期围城终于攻破交趾郡城,城中军民早已饿死、病死大半。其后,日南、九真亦相继被吴军攻下\footnote{胡守为:《岭南古史》,页95}。两条恶犬争夺南粤的战争终于告一段落了。在两军身后,是无数燃烧的南粤城市和乡村、无数粤人的男女老幼尸骨。这场战争告诉粤人,如果粤人想反抗、想脱离帝国的统治,便决不能信任任何岭北帝国,无论这些帝国之间有着怎样的矛盾。战争结束八年后,公元279年,英勇的反抗又一次展开。是年,孙皓下令在广州清查户口,意在征调南粤民力。中级将领郭马系“累世旧军”,可能是交州人,亦可能是南下驻防之荆州兵的后代。无论如何,他都是一个对于南粤有高度认同的勇士。面对孙皓的暴政,郭马与一批下级军官积极联络民众,发动兵变,击毙吴广州督虞授、南海太守刘略,赶走广州刺史徐旗。郭马自号都督交、广二州诸军事、安南将军,遣兵进攻苍梧郡、始兴郡\footnote{始兴郡设于公元265年,位于粤北。参见胡守为:《岭南古史》,页97}。闻知郭马起兵,孙皓大惊失色,忙遣一万七千大军南下镇压。公元280年,正当吴军与起义军交战时,晋军已攻灭孙吴,镇压起义的吴军亦全部降晋,以晋军的身份继续进行镇压。此后,郭马在史籍中彻底消失了,我们不知道他的下落究竟如何\footnote{胡守为:《岭南古史》,页98}。可以确定的是,在场起义经孙吴、西晋两大帝国的接力镇压,被彻底扼杀了。

\section{两晋帝国治下的南粤}

公元280年,西晋取代孙吴成为了南粤的新支配者。这决不意味着南粤的解放,只意味着南粤落入了更残暴的统治者之手。如果说孙吴政权仍带有很强的土豪色彩,那么西晋便是一个由僭主建立的大一统帝国。在征服南粤的当年,西晋便颁布征税法则“户调式”,将汲取机器套在了南粤身上。据“户调式”,南海、始兴、苍梧一带“凡民丁课田,夫五十亩,收租四斛,绢三匹,绵三斤”,甚至至未被帝国编户的百越部落亦要缴纳所谓的“义米”\footnote{《广东通史》上册,页319}。这些数字绝非南粤人税负的全部,因为贪腐的帝国官僚必然会在官定税额外任意加征。公元312年六月,愤怒的广州百姓发动起义,占据广州(番禺)城达两个月之久,方被晋广州刺史陶侃镇压。

在广州百姓起义前后,西晋帝国的核心区域已经深陷大洪水中不能自拔。八王之乱、永嘉之乱的战火摧毁了华北,迫使晋廷南迁至吴越,于公元317年建立了拜占庭式的东晋帝国。由于未曾受到两晋大洪水的波及,物产丰饶、海外贸易发达的南粤一跃之间成为乱世中的乐土。两块近年来出土于广州的晋砖上所刻的文字,即反映了时人对这一现象的认识:

\begin{quote}
	永嘉世,九州岛荒,如广州,平且康。\\
	永嘉世,九州岛凶,如广州,平且丰\footnote{《广东通史》古代上册,页320}。
\end{quote}

如此一来,东晋治下的南粤虽仍受到帝国的残酷压榨,但因远离岭北战乱,遂能在我们勤劳祖先的努力下保持繁荣的面目。其时,广州城成为东亚最为繁荣的港口城市之一。如前所述,第二次北属时期的南粤海船主要由粤西之合浦、徐闻及今日越南中部之日南出港。在孙吴与魏晋争夺南粤的罪恶战争利,这些港口在战火中不幸衰落。随着广州港口的崛起,印度、西亚、罗马的商人遂更改航线,由马六甲海峡直线航向广州。至于东南亚商人,更是能利用信风迅速到达广州。据史籍记载,4世纪时自爪哇岛出航的船只仅需五十日便可到达广州港\footnote{释慧皎:《高僧传》卷3 }。至于由广州启航前往东南亚、印度的南粤船只,更是不计其数。如此便利的航程使广州成为了异常繁华的港口都市。许多躲避东晋重税的吴越百姓为躲避帝国压迫,纷纷泛海进入珠江口于番禺上岸,融入南粤当中、建设南粤\footnote{《广东通史》古代上册,页332}。由此可见,孙吴帝国对南粤的屠杀迫害决不能代表一般吴越人的态度。同为百越苗裔的吴越人与南粤人自之侯入粤时便紧紧联结起来,同心协力地抵御帝国的压迫。

如此富饶的南粤自然逃不过帝国官僚的疯狂压榨。东晋帝国的广州刺史多为贪黩之辈。对此,《晋书》称:

\begin{quote}
广州包山带海,珍异所出,一箧之赀,可资数世,然多瘴疫,人情惮焉。唯贫窭不能自给者,求补长吏,故前后刺史皆多贪黩\footnote{阮元:(道光)《广东通志》卷232《宦绩·吴隐之传》}。
\end{quote}

除刺史皆为贪官外,基层地方官亦无比腐败。例如,曾有一名唤羊嗣的粤北浈阳县令因贪污渎职过甚,被“县功曹吏”驱逐出官府,关押于羊栏中\footnote{《广东通史》古代上册,页321}。出于对东晋官僚的痛恨,粤人将广州城西北二十里处之石门(公元前111年南越军与汉侵略军激战之处)的江水称为“贪泉”——但凡帝国官僚赴广州就职,都会经过石门。我们的祖先通过在种方式,表达了对东晋帝国的切齿痛恨。

事实上,当时南粤的“贪泉”不止一处。在桂阳郡(今连州)南岭山下,有一名为“横流溪”的小溪亦被称为“贪泉”。帝国文人描述此“贪泉”称:

\begin{quote}
	(横流溪)溪水甚小,冬夏不干,俗亦谓之为贪泉,饮者辄冒于财贿,同于广州石门贪流矣\footnote{胡守为:《岭南古史》,页163}。
\end{quote}

此段记载竟将帝国在粤官僚的贪婪归因于引用了“贪泉”之水。如此拙劣而荒诞的辩解,正暴露了帝国对其罪行欲盖弥彰的丑态。在东晋帝国无耻的压迫下,广州城已丧失了组织起有效自卫的能力。公元5世纪初,在新的大洪水袭来时,广州陷入了自南越国灭亡以来最惨烈的灾难。

公元399年,“五斗米道”教徒孙恩聚众数万起事于吴越。三年后,孙恩阵亡,余众推其妹夫卢循为主。403年,晋末权臣、日后的南朝宋武帝刘裕率军先后击败卢循军于东阳(今金华)、永嘉(温州)、晋安(泉州)。卢循只得率部浮海南遁,流窜至南粤沿海,将目标指向富庶的广州城。

作为一支有流寇色彩的宗教大军,卢循麾下的“五斗米道”信徒们有着强烈的宗教狂热。公元404年夏,卢循军对广州展开了猛攻。其时,防御广州城者乃晋广州刺史吴隐之。此人在帝国官场中以“清俭”着称,实则自私阴狠,惯于惺惺作态。在前往广州赴任的路上,吴隐之曾路过石门“贪泉”。为了表现自己的“清俭”,他称自己不相信“贪泉”使人贪婪之说,饮用了“贪泉”之水,并赋诗一首:“古人云此水,一歃怀千古。试使夷齐饮,终当不易心。\footnote{阮元:(道光)《广东通志》卷232《宦绩·吴隐之传》}”吴隐之在此竟将自己比拟殷商的忠臣伯夷、叔齐,可谓恬不知耻。事实证明,此人实为一无能之辈。当年十月,经百余日的围攻,卢循军发动夜袭,攻入广州城中,放火焚城。火起时,城中百姓纷纷躲避,填塞于路。吴隐之因害怕城中有人响应卢循,竟残酷地下令禁止救火。结果,大火导致了“焚烧三千余家,死者万余人”的惨剧,广州城遂告陷落\footnote{胡守为:《岭南古史》,页118}。

城陷时,吴隐之未能逃走,被卢循军俘获,不久后又被释放。在携妻北归途中,吴隐之发现其妻携有沉香一斤。为保持自己的“清俭”,他当即将香投入水中\footnote{胡守为:《岭南古史》,页167}。此种作秀式的表演固然保持了吴隐之在帝国的官声,却对那些在大火中或惨死、或失去家人与家园的粤人毫无帮助。对于帝国官僚来说,粤人的生命与财产本就不是他们该考虑的事。一万名粤人的生命,远不如所谓“清俭”的官声重要。

卢循军对广州的占领长达六年之久。公元410年二月,卢循趁刘裕北伐南燕之机率军大举北进,大肆蹂躏江右、吴越、湖湘。十一月,刘裕趁卢循北上、广州空虚之际,遣兵三千自海路袭陷广州城。次年二月,屡吃败仗的卢循引军南返,再次围攻广州,不克而南去\footnote{陆树庆:《试论东晋末年孙恩卢循起义》,《学术研究辑刊》1980年02期}。六月,交州刺史杜慧度与卢循军在龙编进行水战,卢循中箭身亡,其众溃散。历时八年,使东晋帝国境内编户人口锐减47\%的孙恩卢循之乱,至此落下帷幕。在这场异常惨烈的大洪水中,南粤因饱受帝国压榨,未能出现如赵佗、士燮一般的人物,因而无法置身事外。在盘踞广州期间,卢循曾利用粤人的反晋心理征集数万兵员。然而,这些南粤子弟并无机会投入保卫南粤的战斗,他们成为了被卢循利用的棋子,跟随着流寇大军悲惨地战死在异国他乡。我们善良朴实的祖先,就这样再一次被岭北卑鄙的野心家利用了。

孙恩卢循之乱是第一个将南粤卷入其中的帝国崩溃式大洪水,也为东晋在南粤的统治上演了一个血淋淋的结局。在平定卢循之乱时,刘裕在东晋的地位已然极高。公元420年,刘裕废晋恭帝司马德文,自立为帝,建立了刘宋帝国。

\section{刘宋、齐、梁治下的南粤}

公元420年,东晋帝国灭亡。继东晋而起的刘宋(420—479)、齐(479—502)、梁(502—560)三朝皆为北人压迫南人的政权,并无本质不同。在三朝统治下,粤人继续承受着帝国的压榨。更有许多帝国野心家因觊觎南粤的财富,上演了一次又一次武力争夺南粤的丑剧,给我们的祖先带来了极大的苦难。

公元453年三月,以开创所谓“元嘉之治”而着称的宋文帝刘义隆被太子刘劭弑杀。四月,诸臣于京外拥立宋文帝第三子宋孝武帝刘骏讨杀刘劭\footnote{仇巨川:《羊城古钞》卷17《事纪·六朝广州杂乱》}。九月,刘劭死党、南海太守萧简据广州反,旋被宋军讨平\footnote{沈约:《宋书》卷6《本纪第六·孝武帝》}。十余年后,帝国内战再次爆发于南粤:公元465年,宋明帝刘彧弑宋废帝刘子业而登基。次年,明帝侄晋安王刘子勋称帝于寻阳(今湖北黄梅西南),广州刺史袁昙远据广州城响应之。然而,粤北晋康太守刘绍祖却不服袁昙远,起兵反之\footnote{《广东通史》古代上册,页362}。粤北始兴郡士人刘嗣祖亦起兵杀死当地官员,据城反袁\footnote{沈约:《宋书》卷84《列传第四十四邓琬传》}。袁昙远忙派部将李万周北伐始兴,与刘嗣祖军筑垒对峙于浈阳。李万周乃一介莽夫,心思简单。刘嗣祖遂遣军使诳骗李万周,称刘子勋已被平定,朝廷新派的广州刺史马上就到\footnote{沈约:《宋书》卷84《列传第四十四邓琬传》}。李万周果然信以为真,遂回师夜袭广州,其军乘长梯入城,斩杀故主袁昙远,并大肆蹂躏城中百姓,“劫掠公私银帛”\footnote{沈约:《宋书》卷84《列传第四十四邓琬传》}。公元467年,新任广州刺史羊希到任,诛李万周、刘嗣祖,广州的局势终于暂时平稳了下来\footnote{广东通史》古代上册,页362}。

在刘宋帝国的内战中,我南粤接连两次沦为战场,无数百姓变成了帝国军队的刀下亡魂。袁昙远之乱结束后仅仅一年,再也无法忍耐的南粤百姓终于发起了反抗。公元468年三月,交州土著李长仁发动起义,自称交州刺史,攻据州治龙编城,严厉惩办当地的流寓北人,将他们全部诛杀\footnote{沈约:《宋书》卷94《列传第五十四恩幸》}。接着,起义军进攻广州,击毙刺史羊希。晋康太守刘思道意图火中取栗,乃趁乱起兵反宋,与起义军合流,占领广州城\footnote{仇巨川:《羊城古钞》卷17《事纪·六朝广州杂乱》}。可是,宋军很快便发动反攻,夺回广州城,擒杀刘思道\footnote{胡守为:《岭南古史》,页148}。然而对于占领交州的李长仁,宋廷却不敢贸然进攻。当年十一月,李长仁自去交州刺史称号,遣使向宋廷妥协,要求“自贬”为“行州事”,宋廷只得无奈地同意。此后,李长仁执掌交州大权达十一年之久。公元479年,宋灭齐兴,李长仁亦于是年与世长辞,其堂弟李叔献代之。直至公元485年,李叔献才在齐武帝的威胁下离开南粤、迁居建康。李氏兄弟对交州近二十年的统治,至此告一段落。

作为土著,李长仁起兵反宋是值得永世纪念与称赞的壮举。他不加甄别地尽屠龙编北人之举虽有过激之嫌、不应全盘肯定,然揆诸当时岭北帝国长期欺压我们祖先的情形,则此种行为无疑是可以理解的。因帝国不断榨取南粤的资源、掠夺粤人的财产、屠戮粤人的老幼妇孺,粤人对北人的仇恨早已深入骨髓。我们淳朴勇武的祖先本在故土过着自由的生活,然而帝国的武断之治却将他们的自由尽数剥夺,逼使他们发动起义向北人复仇。应当为龙编屠杀这一惨剧负首要责任的无疑不是李长仁,而是岭北帝国。至于李氏兄弟雄踞交州,使当地在事实上脱离帝国统治近二十年,则更是一项丰功伟绩。而起义军在解放交州后不忘进攻广州,则表明当时的越南与南粤仍未分离,两者的抵抗运动是联为一体的。

在南齐治粤的二十四年里,南粤未发生重大战事。至萧梁治下,战乱又起。公元511年,被流放至广州的北人王贞秀勾结参军杜景谋袭广州城,被梁军讨平\footnote{(光绪)《广州府志》卷75}。535年,部落酋长“俚帅”陈文彻兄弟率部起义,进攻广州,被梁军残酷镇压\footnote{姚思廉:《梁书》卷32《列传第二十六兰钦》}。至542年,又有土豪卢子略起兵攻广州,惨遭梁军镇压\footnote{仇巨川:《羊城古钞》卷17《事纪·六朝广州杂乱》}。粤人对于刘宋与萧梁的反抗可谓屡仆屡起,其次数之多、之频繁足以令人动容。在此期间,我们勤劳的祖先亦从未忘记探索海洋、从未忘记建设南粤。随着对外贸易规模的持续扩大,广州成为了国际贸易体系中十分重要的节点,呈现出“海舶继路”的繁盛局面,波斯萨珊王朝的银币则成为南粤境内流通的货币之一。据伊斯兰世界著名史书《黄金草原》的记载,这一时期的南粤商船已能航行至波斯湾北端、幼发拉底河下游的Hira一带(今伊拉克之纳杰夫或巴士拉)。与此同时,西亚、印度、东南亚等地十余国的商船亦每年来粤贸易\footnote{《广东通史》古代上册,页379}。在广州港,进口物包括“金银宝器、犀象、棉布、斑布、金刚石、琉璃、玳瑁、珠玑、槟榔、兜鍪、珊瑚、沉香、杂香药等”,出口物则有“陶瓷、铠仗、袍、袄、马等”\footnote{《广东通史》古代上册,页383}。巨额国际商贸为广州的帝国官僚带来了大量的盘剥机会。对此,《南齐书》中有一句形象的记录:

\begin{quote}
	南土沃实,在任者常致巨富。世云:“广州刺史城门一过,便得三千万也。\footnote{萧子显:《南齐书》卷32《列传第十三王琨》}”
\end{quote}

如在段记载所述,帝国官僚常能通过掠夺粤人由国际贸易得来的财产而“致巨富”。在官僚的无耻抢劫下,许多南粤商人绕开官府,与外商在广州以外的偏僻港口进行私下交易。此种被帝国视为叛逆的行为正表明,我们的祖先一直向往着自由,与海洋的关系远比与岭北帝国的关系更亲近。

只要岭北帝国仍然统治着南粤,南粤的苦难便不会终止。只要岭北帝国仍压迫着南粤,我们祖先的反抗就不会停止。梁陈之际,当帝国陷入崩溃时,有两位英雄人物在南粤站了出来,恢复了南粤的自由、恢复了南人的自由,他们便是著名的陈霸先和冼夫人。

