\chapter{发明民族:南粤民族的定型}

\section{三族共同体的定型:客家与潮汕的分隔与共存}

\indent 自15—16世纪客家人进入粤东北以来,他们便和粤东的福佬民系(今潮汕民系)发生接触。粤东的韩江hay 南粤境内仅次于珠江的第二大河流,其上游分汀江、梅江两支。汀江发源于闽越宁化、长汀两县交界处,梅江则发源于粤东北之紫金县。两江在大埔县三河坝合流,形成韩江。韩江向南流经潮州城下,再分多条支流出海。在潮州城及其以南的韩江三角洲平原上聚居着讲福佬话的居民。至于潮州以北的韩江中、上游地区则为山地,聚居着讲客家话的居民。客家山民生活贫困,加之两族语言不同,双方仇恨颇深。1645年,一场两族间的大战爆发了,是为震撼粤东的“九军之乱”\footnote{ 陈春声:《族群分类与地域认同:1640—1940年韩江流域民众“客家观念”的演变》}。

1645年六月,即清军入侵南粤之前五个月,揭阳县武生刘公显聚蓝田(今属丰顺)、霖田(今属揭西)二都之客家山民作乱,建国号“后汉”,改元“大升”。乱民分为九部,自称“九军”,呼韩江平原上的福佬人为“平洋人”,提出了“尽杀平洋人”的凶残口号。福佬人对其十分惊恐,称之为“(反犬旁+客)贼”。九军甫一起兵,便在揭阳郊外大肆杀掠,与福佬乡勇展开连绵不断的血战。七月至八月间,koy 们围攻揭阳县城月余,未能得手。清军侵占南粤后,刘公显向郑成功输诚,以反清义军的身份掩护其烧杀抢掠的罪行。1646年九月十一日,郑军水师与九军分水陆两路大举进攻揭阳县城,一举陷之,随即展开屠城,杀害万人以上,城中士绅更是被杀戮殆尽。史载,九军和郑军在此次大屠杀中使用的手段特别残酷:

\begin{quote}

钉锁于天中,以猛火燃迫,至于皮开肉绽。掘坎于地下,以滚汤灌渍,至于体无完肤。多以纸浸油,男烧其阳,女焚其阴,异刑不能阐述。各贼分宅镇营,杀戮乡绅士庶……抢掠妇女,尽驱入山。所至破棺碎木主,贼名之曰“辟死鬼”。贼毁文庙,劫城隍,开库狱,焚黄册\footnote{雍正《揭阳县志》卷3}。

\end{quote}

屠杀过后,九军驻于揭阳,改其名为“都督府”,随即一边张挂清帝国旗号,一边继续屠掠韩江三角洲上的乡村,一度进攻潮州府城和潮阳县城,并于1647年将郑成功之叔郑鸿逵迎入城中居住,成为一股同时勾结清、郑侵略者的罪恶势力。1648年四月李成栋反正降明后,九军又打起南明旗号,与同样降明的潮州总兵郝尚久部激战三个月。至1650年初,因尚可喜、耿继茂率清兵包围广州,见势不妙的郑鸿逵遂逃回闽南。同年五月,失去郑军支持的九军被清军击败,刘公显被杀。次年,清军相继攻破九军在揭阳县境内的三大据点郑厝仓、许厝寨、洪厝寨,将其中的两三千人不分男女全部屠杀。1654年八月,清饶镇总兵吴六奇率兵进攻九军大本营蓝田、霖田,于野战中大破其众,包围其最后的据点马头营。经两个月围攻,马头营告破。除突围逃跑者外,寨中之人不分老幼全被清兵凶残屠杀。吴六奇系丰顺客家人,其部下军队亦多为客家人。他们对客家人聚居区的风土人情十分了解,故能在此次作战中顺利获胜\footnote{郭伟川:《岭南古史与潮汕历史文化》,页480—481}。至此,延宕近十年的九军之乱终于结束。

九军之乱是南粤客家、福佬两族间的一场大规模战争。在清帝国的介入下,客家一方的九军落败。此次战争中,两族俱付出了相当沉重的代价,其中福佬人多死于九军的屠掠,客家人多死于清军的无差别屠杀。值得注意的是,并非所有客家人都认同九军“尽杀平洋人”的激进主张,有许多如吴六奇这样的客家将领、土豪都在战争中站在九军的对立面。1662年,清帝国开始对南粤实行丧心病狂的“迁界令”,大批韩江三角洲沿海地区的福佬人被迁至韩江中上游地区,与客家人同住。在此浩劫中,惨死的沿海福佬居民不计其数。复界之后,大部分幸存的沿海居民陆续回到家园,一些客家人也随之来到韩江三角洲垦荒。这样一来,两族的主要聚居区虽仍维持原状,但双方已不再如从前那样互相敌视、渭泾分明,在两族交界的韩江中游更出现了“半山客”这一族群。半山客系一亦客亦潮的族群,他们的自我认同为客家,但许多生活习俗却接近福佬人。他们所讲的语言虽属客家话,但深受福佬话影响,与其他客家人的语言非常不同。有的半山客甚至已在文化上完全福佬化,以福佬话为母语。在整个18世纪,客家、福佬两族虽偶有小规模械斗,但此种械斗甚至不及两族内部各宗族、村落间的械斗频繁。同遭清帝国残酷迫害的两族都深刻吸取了九军之乱的教训,不愿再让此种南粤内战重演,以免再次发生令亲者痛仇者快的惨剧。在17世纪末和18世纪粤东的地方志中,我们几乎看不到关于族群分类的记载。编纂方志的福佬、客家士大夫仅以地域来划分人群(如称潮州府城居民为“郡人”、程乡县人为“程人”),却从不用“客”、“土”之类的称呼将人群按语言、习俗划分开来。这表明,当时粤东的客家人和福佬人相处得相当和睦,双方都没有将对方与自己的不同视为多么了不得的事。在两族精英眼中,族群差异甚至还不如地域差异重要\footnote{李默:《多情曲》,页26—27}。如前所述,1733年,清帝国将客家人聚集的粤东北五县划为由广东省直隶的嘉应州。1807年,又升嘉应州为嘉应府。然而,此种行政区划的变更并未对两族的关系产生多大影响。

1850年代,当广府、客家两族在珠三角西部进行惨烈的土客战争时,粤东的客家人与福佬人相安无事。1858年,清帝国与英法两国签订《天津条约》,将潮州府境内的小港口汕头辟为通商口岸。开埠以后,汕头成为整个韩江流域唯一一座可以停泊机器轮船的港口,迅速发展为一座大城市。至19世纪末,汕头逐渐取代潮州成为韩江流域的经济、文化中心,潮州、嘉应州的大批福佬、客家精英纷纷到此居住。在19、20世纪之交,他们积极响应清帝国的维新变法及新政,在汕头建西式学堂、办报纸。1900年3月,著名客家学者温仲和与丘逢甲在汕头创办岭东同文学堂,丘逢甲出任该校总办(校长),温仲和则任总教习。丘逢甲祖籍大埔,系在台湾出生的南粤客家移民。1895年,清帝国在日清战争中战败,日本占领台湾。丘逢甲不愿留台,遂西渡南粤,与温仲和结为好友。两人创办的这所学校面向潮州、嘉应州两地招生,学生中福佬、客家人皆有。该校以教授西学为主,兼亦教授东亚传统经史知识,设有文学、史学、算学、格致、化学、生理卫生、日文等课程\footnote{徐博东、黄志平:《丘逢甲传增订本》,页71—72}。受此影响,潮州城内紧随汕头步伐,于1902年将金山书院改组为潮州中学堂(今金山中学),其招生、授课制度与岭东同文学堂颇为相似。在这些新式学堂中,客家、福佬师生朝夕相处,很快便因语言不通产生了隔阂,并进而影响教学质量。1902年5月,粤东学者何士果(大埔客家人)、温丹铭(汕头人,祖籍大埔客家人)在汕头创办《岭东日报》,是为南粤境内的首份现代报纸。该报除报导世界各地最新消息外,还留有一名为“嘉潮新闻”的大版块,专门刊登两地来稿及地方新闻。对于学堂中的此种现象,该报相当注意。例如,在1904年1月1日,该报便用两版版面报导了潮州中学堂内两族师生因语言不通产生的严重问题。该报导称:

\begin{quote}
友人来函云,潮州中学堂分教某君尝对人云:本年中学堂之学生,惟澄海最有进步,若大埔学生则多不上课堂,殊少进益。按:大埔学生之多不上课堂,以堂中分教二人皆操潮州土语,语言不通之故。然业已负学生之任,纵格于语音,不能上堂听讲,何不借学堂为自修之地。大埔学生勉乎哉\footnote{转引自陈春声:《族群分类与地域认同:1640—1940年韩江流域民众“客家观念”的演变》}。
\end{quote}

当时的粤东有这样一句俗语:“大埔无潮,澄海无客”。位于潮州府南部的澄海县系纯福佬县,北部的大埔、丰顺则为客家县。根据这段新闻的描述,潮州中学堂中来自大埔的学生因听不懂福佬老师讲的潮州话,因此无法像澄海学生那样安心听讲。此外,《岭东日报》还注意到,汕头城中的客家、福佬商人分别形成了自己的组织,“潮属有万年丰会馆,嘉属有八属会馆”。耐人寻味的是,大埔、丰顺两县虽在行政区划上属潮州府,但来自两地商人却纷纷加入客家人的八属会馆\footnote{陈春声:《族群分类与地域认同:1640—1940年韩江流域民众“客家观念”的演变》}。在过去的乡村生活中,村民们大多聚居在讲同一语言的村庄里,因而族群意识不深,不如地域意识强烈。随着大批客家、福佬人来到汕头定居,两族商人、知识分子、市民朝夕相处,每天都在体验着异质文化。在日复一日的碰撞下,双方认识到了自己与对方的不同,都产生了强烈的族群意识,并进而认可了和而不同的共存生活。

土客战争和汕头开埠分别使广府与客家、客家与福佬接受了在南粤大地上和而不同的共存之道。接下来,便需要一种理念将三族统合在一起了。1907年初,顺德广府学者黄节在上海出版《广东地理乡土教科书》,激起轩然大波。黄节系与陈澧齐名的南粤大儒朱次琦(关于朱次琦,详见下一章)之再传弟子,以诗闻名。在这本教科书里,黄节公然在第十二课《人种》中以图表形式将“客家”、“福狫”划为“外来种族”,引发粤东客家、福佬精英的强烈愤慨。是年3月29日,《岭东日报》上发表了一篇题为《广东乡土历史客家福佬非汉种辨》的文章,对黄节的观点大加批判:

\begin{quote}
广东乡土历史教科书一种,为粤人黄晦闻(节)所作,其第二课(按:原文如此)广东种族,有曰客家福老二族,非粤种,亦非汉种。其参考书复曰:此两种殆周职官方所谓七闽之种。不知其何故出此?岂其有意污蔑欤?不然,何失实之甚也!……黄氏此书,乃欲以饷粤省儿童,使其先入于心,早怀成见,益启其阋墙之衅,以为亡国之媒,非黄氏贻之祸乎?考之事实已相违,施之教育又不合。黄氏苟出于有心蔑视也则已,如其为无心之差误,则吾望其速为改正也\footnote{转引自程美宝:《地域文化与国家认同:晚清以来“广东文化”观的形成》,页85—86}!
\end{quote}

与此同时,全粤数十县的客家、福佬知识分子一同发起声势浩大的抗议活动,要求清帝国学部将黄节的教科书查禁。对他们来说,称自己的民系为“非粤非汉”的外来种族着实是一种无法忍受的侮辱,因为客家人和福佬人已在南粤的土地上生活了许久。在兴宁县,有个名叫胡曦的客家士人甚至在病危之际与友人彻夜谈论此事,直至瞑目。在巨大压力下,清帝国学部只得要求上海道台处理此事。在1908再版的《广东地理乡土教科书》中,黄节不得不将有关客家、福佬系“外来种族”的论述全部删去,方平息此番争端。至此,广府精英默认了客家、福佬人并非“非粤非汉”的“外来种族”这一“事实”。南粤的广府、客家、福佬(潮汕)三族各自承认了其余两者的南粤人身份及“华夏正统传人”的身份,并一同以此种认同和岭北做出身份切割。经过长时间的磨合、冲突,经过长时间的磨合、冲突,以三族为主体的南粤文明终于定型。在20世纪,三族中都将涌现出保卫南粤自由与尊严的伟大英雄。

南粤的三族共同体定型了。那么,语言不通的三族之间应当如何交流?三者将采取何种语言作为共同语?在下一节中,我们便将讨论这一至关重要的问题。

\section{民族语言与民族史:学海堂、粤史、粤语、粤文}

\indent 在17—19世纪,随着欧洲民族国家的崛起,西方各国纷纷制定国语,并在小学教育中加以推广。在如法国这样的国家,标准法语抹平了国内各民族的语言差异,并制造出了面目狰狞的大一统利维坦。在20世纪,“中国”的推行“国语”运动如出一辙,甚至变本加厉,直到今天仍以“推广普通话”的面目挤压着粤语的生存空间。但在南粤,粤语广州话演变为共同语却完全是自发秩序演化的结果,与政治强权毫无关系。这一演化发生于17—19世纪,与欧洲民族国家的崛起处于同一时代。此种演化是如何发生的?欲了解这一点,我们便需追溯粤语的历史。

粤语系一种由百越语言与北方汉语混合而成的克里奥尔语,不能与岭北语言互通。它音调丰富,铿锵有力,既拥有大量百越语言的底层词汇,也比较完好地保留了中古汉语的读音\footnote{关于粤语中的百越底层,参见李敬忠:《粤语中的百越语成分问题》}。早在12世纪末,流寓广西的吴越人周去非就已注意到南粤不但拥有自己的语言,甚至还有自己发明出来的“俗字”。据周去非记载,当时的南粤有个表示“瘦弱”之意的“奀”字\footnote{周去非:《岭外代答》卷4}。在今天的粤文中,该字仍在被使用。成书于1535年的《广东通志初稿》中也记录了不少南粤俗字:

\begin{quote}
奀音勒,不大,谓瘦也。

无曰冇,音耄,谓与有相反也\footnote{转引自程美宝:《地域文化与国家认同:晚清以来“广东文化”观的形成》,页119}。
\end{quote}

除“奀”之外,此书还提到了今日粤文中的常用字“冇”。此外,书中亦记载许多今已消失的“俗字”。由是可见,南粤自古以来不但与岭北语言不同,在文字上亦有不小的差异。然而,因南粤士大夫需要学习帝国的“官话”方能进入官场,如何讲好非母语的“官话”遂成为令他们头痛的问题。对岭北帝国的皇帝而言,粤、闽籍官员所讲的语言殊为难懂。为解决此问题,雍正帝于1728年特意“下旨”,令在南粤、闽越设立教授“官话”的“正音书院”。六年后,雍正帝又令从江右、吴越挑选通晓“官话”的士人赴粤、闽任教。短时间内,粤、闽两地便出现了超过2000座正音书院、书馆、社学。由“正音”学校的数量如此之多来看,此次“正音”运动的目标当是粤、闽的全体读书人。不过,此次运动在1735年雍正帝病死后不久便松弛下来,继位的乾隆帝对之兴趣寥寥,各“正音”机构亦虚应故事,学生寥寥无几。至1745年,乾隆帝干脆“下旨”裁撤了“正音”机构\footnote{邓洪波:《正音书院与清代的官话运动》}。

虽然清帝国的“正音”运动惨淡收场,但18—19世纪前期的南粤士大夫仍以会讲一口“官话”为融入上流社会的身份标志。不管他们所讲的“官话”有多重的口音,他们都以之为荣。19世纪初的《增城县志》就提供了ni 样一个案例:


\begin{quote}
(增城)士大夫见客,不屑方言,多以正音。
\end{quote}

从岭北帝国的立场上看,所谓“方言”即指粤语,“正音”则指“官话”。由此段文字可知,增城士大夫视粤语为“方言”,对之采取不屑一顾的态度,宁愿在见客时讲“官话”。非但如此,当地的广府士大夫还十分歧视不通“官话”的客家人,称:


\begin{quote}

至若客民隶籍者,虽世阅数传,乡音无改,入耳嘈嘈,不问而知为异籍也\footnote{转引自程美宝:《地域文化与国家认同:晚清以来“广东文化”观的形成》,页113}。
\end{quote}

不过,无论南粤士大夫如何对自己的母语不屑一顾,粤语毕竟是通行于南粤社会的通用语言。士大夫们在互相接触时可以讲“官话”,但在与自己的家人、仆人交谈时必须讲母语。在广阔的南粤社会中,粤语也占据着统治地位。在18—19世纪前期的世界最大城市广州,粤语广州话是理所当然的通用语。不唯广府居民讲广州话,许多客居广州的客家、潮州客商亦通晓广州话。1782年,一本名为《分韵撮要》的粤语韵书在广州出版。这本作者不详的书系统整理了广州府南海、顺德一带的粤语音韵,是现存最早的粤语韵书之一\footnote{彭小川:《粤语韵书"分韵撮要"及其韵母系统》}。可见,在18世纪后期,仍有南粤学者视粤语为一门需要认真对待的语言,而非“俚俗”的“南蛮方言”。1820年,南粤史上著名的学术机构学海堂在广州成立,使局势为之一变。

学海堂的创办者系时任清两广总督的朴学大师阮元(江淮仪征人)。自1817年在广州上任起,阮元即致力于在南粤传播经学、考据学,以期将南粤士大夫带入当时清帝国的学术主流中。阮元通过与十三行行商合作获取了充足的经费。1820年三月二日,阮元将其亲书的三字匾额“学海堂”挂于城西文澜书院,开始招生授课。1824年十一月,经历时三个月的工程,由阮元监修的学海堂在越秀山半山腰建成。1826年,阮元调任云贵总督。临行前,他亲自为学海堂制定章程,规定:

\begin{quote}
管理学海堂,本部堂酌派出学长吴兰修(嘉应人)、赵均(顺德人)、林柏桐(番禺人)、曾钊(南海人)、徐荣(广州汉军八旗)、熊景星(南海人)、马福安(顺德人)、吴应逵(鹤山人)共八人同司课事。其有出仕等事,再由七人公举补额\footnote{转引自郑连聪:《阮元与学海堂研究》}。
\end{quote}

阮元任命的八名“学长”皆为南粤人,他们组成了学海堂的管理层,并负有教育任务。学海堂不授科举之学,专门教授经学、史学、考据学、文学、天文历算等实际学问,为有志于学术、对出仕帝国兴趣不深的南粤青年提供学习场所。学海堂学长工资不高,大都凭着一股为南粤“成就后进,教育英才”的精神努力工作。学长的选补至为严格,写作“浮艳诲淫之辞”和鸦片成瘾者不得就任,“乡评”是比才学更重要的入选标准。学海堂的授课采取“季课制”,每年按季分为四课,每课由两名学长负责,称“管课学长”。每季第一个月上旬,管课学长召集八学长共拟本课考题,将题纸发给学生。该季结束时,管课学长再将题纸收上,分发各学长评阅,排定学生名次\footnote{郑连聪:《阮元与学海堂研究》}。可见,学海堂并不强调应试教育,因为学生在每季三个月的充足时间用于答题。欲完成这些题目,则需在此期间认真听课、努力思考。学海堂强调的是真正的学术,而非机械的应试。1834年,学海堂又规定学长由学生中挑选十名品行优良者为“专课肄业生”。专课肄业生可自八学长中任择一师“谒见请业”,并挑选自己感兴趣的一门经史学问展开精深的研究。此培养模式颇类西式大学的研究生教育,学长与专课肄业生的关系就如导师与学生一般\footnote{转引自程美宝:《地域文化与国家认同:晚清以来“广东文化”观的形成》,页85—86}。此后,学海堂的运作日益走上轨道,进而涌现出一大批影响南粤历史走向的人才。其中,以吴兰修、梁廷柟、陈澧三人对南粤的民族发明最为重要。

吴兰修系学海堂首批学长之一,乃嘉应州客家人,曾于1808年中举人,并在粤西信宜任儒学训导。然而,他无意于进一步深入官场,却对算学和南粤历史很感兴趣。1818—1822年间,阮元组织一批学者编纂了卷帙浩繁的道光本《广东通志》,精通南粤历史的吴兰修被聘为总校勘官。出任学海堂学长后,他更是一心钻研学问,大量购书,自建“书巢”藏书数万卷。特别令人感动的是,深爱着南粤的吴兰修“以乡梓之邦,数典宜核”,花十年时间精研南汉国史,写出了《南汉纪》、《南汉地理志》、《南汉金石志》等不朽著作。1873年,他以85岁高龄去世,留下了等身的著作及一套以南粤为主体叙述历史的方法\footnote{李兆洛:《吴石华南汉纪序》,《养一斋集》卷2}。此外,又有梁廷柟对南粤史进行了更深入的研究。

梁廷柟系顺德伦教人,生于1796年。出身中产家庭的他早年科举不顺,仅任粤东澄海县教谕,遂和吴兰修一样放弃仕途,积极钻研学问。1829年,梁廷柟完成十八卷的《南汉书》,是为首部模仿帝国官修断代史书创作的南粤本土断代史。该书分帝王本纪、后妃、诸王公主、诸臣、杂传、女子、宦官、方外、叛逆、外传十类,各本纪、传记前皆有褒贬人物的“论赞”,其内容之丰富、考证之严密超过了吴兰修的《南汉纪》一书,系研究南汉国史的必读之书。1833年,梁廷柟又依靠极其有限的资料撰成讲述南越国诸帝王事迹的《南越五主传》,是为南粤史上首部严谨的南越国史研究专著。写作过程中,梁廷柟曾因资料不足大为苦恼。书成之后,他在序言中这样说:


\begin{quote}
生长是邦,愤悱若此,安用考据为哉\footnote{梁廷柟:《南越五主传》序}?
\end{quote}

梁廷柟生长于南粤而难考乡邦之史,由此感到极度愤懑,其爱粤之心跃然纸上,足可使百余年后的我们深深感动。1840年,梁廷柟就任学海堂学长。此后,他积极钻研西方历史,于1844、1848年先后撰成美国史专著《合省国说》和英国史专著《兰仑说》。1852年,他全力帮助家乡的修志事业,参与了《顺德县志》的编纂。九年后,梁廷柟在家乡去世,时年66岁\footnote{关于梁廷柟之生平,参见吴宝祥:《梁廷柟年谱简编》}。

若说吴兰修、梁廷柟开启了南粤民族史学,那么陈澧就开启了南粤的民族语言学。陈澧的祖先世居吴越绍兴,其祖父迁居南粤。他的祖父和父亲皆因未能进入广东户籍而不得参与科举,直到他这代方“占籍为番禺县人”。陈澧于1810年生于广州。做为一个吴越移民后代,陈澧以广州人自居,亦被人视为粤人。1832年,年仅23岁的他考中举人,一时春风得意。两年后,他进入学海堂求学,成为学海堂的首届“专课肄业生”之一。但此后二十年内,他六赴北京会试,皆告失利,遂断绝出仕念头,一心钻研学术。陈澧兴趣广博,于经学、天文、地理、算术、音律、文学、书法无所不通,人称“东塾先生”。1840年,他被聘为学海堂学长,教出了不少优秀的学生。陈澧反对清儒细碎的考据之学,提出读书当以“通经致用”为目的。这一主张贯穿他的一生,直到他在1881年以72岁之龄去世\footnote{孙克强、杨传庆、裴哲主编:《清人词话下》}。陈澧曾在晚年撰有《广州音说》一文,论证粤语广州话即系中古汉语。在文中,他这样说:

\begin{quote}
广州方言合于隋唐韵书切语,为他方所不及者,约有数端。余广州人也,请略言之……广音四声皆分清浊,故读古书切语,了然无疑也。余考古韵书切语有年,而知广州方音之善,故特举而论之,非自私其乡也。他方之人宦游广州者甚多,能为广州语者亦不少。试取古韵书切语核之,则知余之言不谬也……至广中人声音之所以善者,盖千余年来中原之人徙居广中,今之广音实系隋唐时中原古音,故以隋唐韵书切语核之,而密合如此也。请以质之海内审音者\footnote{转引自程美宝:《地域文化与国家认同:晚清以来“广东文化”观的形成》,页117}。
\end{quote}

从纯粹的语言学角度来看,陈澧称广州话语音与“隋唐古音”全然相合肯定不符合事实。但从民族发明的角度上看,他却是用系统的学院派方法将粤语广州话发明为足以傲视岭北一切语言的“华夏正音”,进而将粤人论述为华夏文明的唯一正统传人。在这套论证下,南粤士大夫便无需再将“官话”视为一种高于自己母语的语言,反而可以站在粤语一边歧视“官话”。最令人称奇的是,陈澧的这套论证方法被他的一位名叫温仲和的客家学生原封不动地学去了。温仲和是嘉应人,曾于1878年被选为学海堂专课肄业生,师从于陈澧。1889年,温仲和考中进士,于次年担任《嘉应州志》总纂。在这本方志中,温仲和如是说:

\begin{quote}
仲和昔侍先师番禺陈京卿(澧),尝谓之曰:嘉应之话,多隋唐以前古音……夫昔之传经者,既以方言证经,则今考方言自宜借经相证。其间相同者,盖十之八九,以此愈足证明客家为中原衣冠之遗,而其言语皆合中原之音韵\footnote{转引自程美宝:《地域文化与国家认同:晚清以来“广东文化”观的形成》,页71—72}。
\end{quote}

此段文字中,温仲和比他的老师陈澧更大胆,直接将自己的母语客家话论证为比粤语还要存古的“隋唐以前古音”。这样一来,客家话也被“严谨”地“考证”成了“华夏正音”,与粤语一同获得了足以傲视岭北语言的资本。陈澧和温仲和这一对师徒携手为广府、客家两族的民族发明做出积极贡献,可称得上是南粤史上的一段佳话。

自1881年陈澧去世后,学海堂师生便再无一人享有他那样的声誉。19世纪末,随着西学广泛传入东亚,东亚传统的经史知识已不再是清帝国士子的必修学问。在此情形下,学海堂完成了它的历史使命,于1903年停办。在学海堂存在的八十三年里,它为南粤培养出了一大批优秀的才俊,其中许多人将会影响南粤的历史走向。在随后的章节中,我们将看到他们的事迹。学海堂的另一个重要贡献,便是开创了南粤的民族史学和民族语言学。它为无意于出仕清帝国的南粤精英提供了一个研究南粤历史、语言、文化的稳定平台,成为19世纪南粤士大夫发明民族的中心。如果说学海堂是当时南粤精英发明民族的中心,那么在南粤民众中便存在着更为强劲的民族发明力量。形形色色的粤语文学作品与基督教书籍的在ni 一时期的涌现,是这种力量的直观反映。

粤语文学的历史源远流长。在南粤漫长的文明史上,虽然精英阶层多以汉语文言文写作,但民间歌谣和民间宗教仪式的用语大都为粤语。在16世纪以前,南粤缺乏一个稳固的儒化士大夫阶层,因而无人将这些粤语歌谣、符咒用文字系统地记录下来。在成书于17世纪后期的《广东新语》中,我们方能看到南粤士人屈大均对乡土歌谣较详细的记载:

\begin{quote}

粤俗好歌。凡有吉庆,必唱歌以为欢乐。以不露题中一字,语多双关,而中有挂折者为善。挂折者,挂一人名于中,字相连而意不相连者也。其歌也,辞不必全雅,平仄不必全叶,以俚言土音衬贴之。唱一句或延半刻,曼节长声,自迴自复,不肯一往而尽。辞必极其艳,情必极其至,使人喜悦悲酸而不能已已,此其为善之大端也……名曰摸鱼歌,或妇女岁时聚会,则使簪师唱之\footnote{屈大均:《广东新语》卷12}。
\end{quote}

屈大均在此提及的“摸鱼歌”很可能泛指当时的所有粤语歌谣,包括小调、抒情歌曲和较长的叙事歌(称“龙舟”)。由屈大均称歌中含有“俚言土音”来看,这些歌曲是用粤语演唱的。屈大均在后文中记录的一段歌词证实了这种判断:

\begin{quote}

大头竹笋作三桠,咁好后生无置家。咁好早禾无入米,咁好攀枝无晾花\footnote{屈大均:《广东新语》卷12}。
\end{quote}

这首情歌站在一位女子的角度,窃喜地赞叹一个英俊后生尚未名草有主。由歌中如此直接地表达女子对英俊男子的喜爱之情来看,此歌当不会被儒家士大夫公开赞赏,却能受到饮食男女的追捧。由“咁好”这一粤语词反复出现来看,此歌定是被南粤大众以口语反复传唱的作品。1713年,一部名为《花笺记》的奇书在东莞出版,将粤语文学带入了南粤士大夫的生活中。

《花笺记》一书作者不详,内容为书生梁亦仙与两位美女杨瑶仙、刘玉卿间缠绵悱恻的爱情故事。全书以南粤传统民谣形式“木鱼歌”的形式写就,系一本“木鱼书”。该书共五十九回,每回有十余句至二百余句不等的唱词\footnote{王钊宇总纂:《岭南文化百科全书》,页295}。书中大量运用“咁”、“唔”、“点”、“睇”、“堆埋”一类沿用至今的粤语词汇,颇为贴近当时的粤语口语\footnote{程美宝:《地域文化与国家认同:晚清以来“广东文化”观的形成》,页123}。因该书所讲的系颇为典雅的才子佳人故事,又用生动的生活语言写成,因而受到许多南粤士大夫的追捧。他们惊喜地发现,自己的母语写就的文学竟然如此美丽动人。18世纪中期,一位名叫钟映雪的东莞士人在读过《花笺记》后曾激动地说:

\begin{quote}
予幼时闻人说:“读书人案头无《西厢》、《花笺》二书,便非会读书人。”此语真是知言,想见此公亦不俗\footnote{转引自程美宝:《地域文化与国家认同:晚清以来“广东文化”观的形成》,页128}。
\end{quote}

由钟映雪幼时便已听人说不读《花笺记》“便非会读书人”来看,此书在南粤士大夫中着实大受欢迎。在此,《花笺记》已被钟映雪提到与名著《西厢记》等量齐观的地位。接下来,钟映雪实在难掩他对《花笺记》的喜爱,兴奋地写道:

\begin{quote}

《花笺记》当与美人读之……

《花笺记》当与名士读之……

《花笺记》当向明窗净己读之……

《花笺记》当以精笔妙墨点之……

《花笺记》当以锦囊贮之……

《花笺记》当以素缣写之……
\end{quote}

直到三百多年后的今天,我们仍能通过这些文字感受到钟映雪在阅读到以自己母语写就的佳作后那份自心底涌出的快乐与激动。钟映雪不但极力推崇《花笺记》,还身体力行地用粤语创作木鱼书。他曾著有与《花笺记》齐名的作品《二荷花史》,讲述一个书生与两位佳人的奇幻情缘\footnote{李思德:《中外艺术辞典》,页662}。至19世纪早期,一位大才子横空出世,将粤语文学推上新高峰,他的名字叫做招子庸。招子庸系南海人,于1793年出生于一个耕读传家的小康家庭。他曾于1816年中举人,远赴山东任数县县丞,后因不满于清帝国官场的浊乱辞官归乡\footnote{钟哲平:《粤韵清音·广府说唱文学》,页153}。此后,招子庸绝意于仕进,每日浪迹于珠江上的风月场所,流连于花船之间,以不修边幅闻名。每当端午节斗龙舟时,他便簪石榴花跣足立于船头,左手执旗、右手擂鼓,狂傲之态颇令世俗惊骇\footnote{程美宝:《地域文化与国家认同:晚清以来“广东文化”观的形成》,页131}。招子庸又与学海堂学者过从甚密,曾师从于学海堂学长张维屏(番禺人,1838年任学长),并与另一学长徐荣结为密友。当时,学海堂学者中颇有一些与风尘女子结下不解之缘者,如大名鼎鼎的陈澧便曾和一位名叫柳小怜的名妓有过一段情感纠葛。至于张维屏更系一风流才子,时人称之为“风流教主”。他时常“偕二三知己载酒珠江”,与一班狂士、歌妓通宵狂欢\footnote{程美宝:《地域文化与国家认同:晚清以来“广东文化”观的形成》,页132}。在招子庸的带动下,学海堂学者在治学之余时常写下一些风月文字,记录他们在花船中的艳遇与情愁。这些文字大多为用粤语创作的“粤讴”,因为他们只有用母语方能贴切地表达自己内心最柔软的情绪。只需看看招子庸的两首著名粤讴,我们便能明白这种情感有多么美好和真挚:

\begin{quote}
结丝萝

清水灯心煲白果,果然清白,怕乜你心多。白纸共薄荷,包俾过我。薄情如纸,你话奈乜谁何。圆眼沙梨包几个,眼底共你离开,暂且放疏。丝线共花针,你话点穿得眼过。真正系错。总要同针合线,正结得丝萝\footnote{转引自程美宝:《地域文化与国家认同:晚清以来“广东文化”观的形成》,页130}。
\end{quote}

\begin{quote}
吊秋喜

听见你话死,实在见疑思。何苦轻生得咁痴?你系为人客死,心唔怪得你。死因钱债,叫我怎不伤悲!你平日,当我系知心,亦该同我讲句,做乜交情三两个月,都冇句言词?往日个种恩情丢了落水,纵有金银烧尽带不到阴司!可惜漂泊在青楼,辜负你一世;烟花场上冇日开眉,你名叫秋喜,只望等到秋来还有喜意,做乜才过冬至后,就被雪霜欺?今日无力春风唔共你争得啖一口气,落花无主咁就葬在春泥。此后情思有梦你便频频寄,或者尽我呢点穷心慰吓故知!泉路茫茫你双脚又咁细,黄泉无客店问你向灭谁栖?青山白骨唔知凭谁祭?衰杨残月空听个只杜鹃啼。未必有个知心人来共你掷纸,清明空恨个页纸钱飞。罢咯,不若当你作义妻,来送你入寺。等你孤魂无主,仗吓佛力扶持。你便哀恳个位慈云施吓佛偈,等你转通来生誓不做客妻。若系冤债未偿,再罚你落粉地,你便拣过一个多情早早见机。我若共你未断情缘,重还有相会日子。须谨记,念吓前恩义!讲到销魂两个字,共你死过都唔迟\footnote{李默:《多情曲》,页26—27}。
\end{quote}

第一首粤讴通过谐音将“清白”、“情薄”联系起来,暗喻招子庸的一位心上人负心薄情,并表达他的悔恨幽怨之意。第二首则是招子庸凭吊一位名叫“秋喜”的轻生青楼女子的作品,情感深厚悲凉。两首粤讴所表达的缠绵情意,其真挚、婉转留待各位读者细细体会,一切尽在不言中。唯有用母语创作出的诗词,方能表现出如此热烈、细腻的情感。1828年,招子庸所作歌曲集《粤讴》在广州出版,共收录粤讴一百二十余首,其中多为咏叹珠江花船歌妓遭遇、男女情思之作。此书极受南粤士人欢迎,在其后一个世纪内至少再版了八次之多,模仿其《粤讴》的歌曲集如《再粤讴》、《新粤讴解心》等亦所在多有。1847年,招子庸去世,留下了一笔丰厚的遗产:经过他的努力,南粤士大夫已不再视粤语歌谣为民间俚俗小调、不再视粤语为“俚言土音”、不再视粤字为不入流的“俗字”。至1870年代,一种混在着粤文、北方白话文、文言文、被称为“三及第”的全新文体在南粤诞生,其标志为1870年代出版于广州的小说集《俗话倾谈》。

《俗话倾谈》之作者系四会人邵彬儒。他一生仕途不顺,在广州、三水、佛山等地以讲书劝善为生。他曾于1875年与几个士人在广州成立善社,专门用浅白的粤语口语劝人戒抽鸦片。《俗话倾谈》有初集、二集,分别刊刻于1870、1871年,收录了大批民间传闻逸话,其大旨乃讲说因果报应、劝善惩恶。书中叙述部分混杂三种语言,人物对话则用粤语和北方白话写成,其中以粤语为多。此外,书中偶尔加入一个对故事情节进行评价的“画外音”,以粤语或文言文写就。在该书序言中,邵彬儒ni 样解释他以粤语写作的动机:

\begin{quote}
或有谈及因果报应,则有听、有不听焉,且有抽身而去者矣。非言语不通,事事情未得听也。惟讲得有趣,方能入人耳、动人心,而留人余步矣。善打鼓者,多打鼓也。善讲古者,须谈别致。讲得深奥,妇孺难知。惟以俗情俗语之、说通之,而人皆易晓矣,且津津有味矣\footnote{邵彬儒:《俗话倾谈自序》}。
\end{quote}

在此,邵彬儒将该书的潜在读者定为普通的南粤民众,指出唯有用贴近生活的粤语写作,方能引起大众兴趣、打动大众之心。只消看一段书中文字,大家便可体会此种“三及第”文体是何等地平易近人:

\begin{quote}
香山县有一人姓明,两兄弟,兄名克德,弟名俊德。父母先亡,遗下家产值数千金。克德娶妻凌氏,知情达理,女中之君子也,上能敬夫,下能爱叔。俊德十七八岁,尚未成婚,在家管理耕种。

克德交两个朋友,一个姓钱,一个姓赵。两人不是正经人物,本系无赖之徒,到来一味奉承,想贪饮食。克德又唔明白,以姓钱为知心,以姓赵为知己(克德心盲,又遇瞳人反皆,所以,唔望得真自己,又唔望得真人)。赵钱两人得意遇时,讲三都七国本事非凡。克德本来唔好性情,遇人得罪佢,就一肚火气。钱赵不去泼水,反去添油,话:“驶乜怕佢呀!有咁丢驾就打佢,奈乜何就告佢亦易事。”姓钱话:“兵房师爷系我姐夫。”姓赵话:“三班总头系我老契。”克德拍掌喜曰:“有咁样人事,随便车天。”满斟一杯劝姓钱曰:“好手足。”又斟一杯劝姓赵曰:“好兄弟。”三人畅饮,劈口高歌,或猜拳,或大笑。克德大声曰:“喊我细佬来,快的赶去炙烧酒、杀鸡,唔得急,将廿只鸭蛋打破,湿半斤虾米,切一两腊肉丝,发猛火,洗锅仔,快的炒熟来!\footnote{邵彬儒:《骨肉试真情》,《俗话倾谈二集》卷上}”
\end{quote}

寥寥几笔,不辨是非的主人公和奸猾的赵某、钱某便跃然纸上,生动得如在人眼前。对南粤大众来说,此种活生生的画面感唯有粤语方能勾勒得出。至此,南粤精英与大众便通过共同的母语紧密联系在了一起\footnote{关于《俗化倾谈》的详细研究,可参看李婉薇:《清末民初的粤语书写》,页219—239}。值得注意的是,在《俗话倾谈》二集出版的同年,第一部运用粤语的粤剧剧本《芙蓉屏》诞生。在此之前,南粤本土的戏剧剧本皆用“官话”书写,不能被称为“粤剧”。《芙蓉屏》问世后,大批真正的“粤剧”开始出现。到1910年代,粤语已成为粤剧唱词的主体\footnote{程美宝:《地域文化与国家认同:晚清以来“广东文化”观的形成》,页136—138}。《俗话倾谈》和《芙蓉屏》问世的时代正是第二次鸦片战争结束后不久。此后,各式各样的“三及第”文学便如雨后春笋般冒了出来。在两次鸦片战争前后,大批西方传教士涌入南粤,书写了大量纯粤语的传教书籍(关于这些书籍的详细情况,我们将在本书下一章中讨论)。这些书籍乃是纯粤语出版物的先声。1890年代,在清帝国展开“维新变法”运动前后,东亚大陆各地开始出现白话报纸、教科书,南粤亦不例外。曾任清帝国驻日、英参赞、驻美国旧金山领事的嘉应客家学者黄遵宪亦积极投身维新事业,将大量粤语、客家话词写入自己的诗歌中,并振聋发聩地在《杂感》发出了“我手写我口”的呐喊:

\begin{quote}

我手写我口,古岂能拘牵。即今流俗语,我若登简编\footnote{赵艳红编著:《中国文学简史》,页332}。
\end{quote}

所谓“我手写我口”,便是用自己的笔堂堂正正地写下自己的母语,将其完全转化为正式的书面语。至此,粤语已完全被南粤精英视为一种极其正式的语言。它不但被发明为最正统的“华夏正音”,更是能够以之书写优美诗篇、故事、编撰报纸、教科书的正规语言。南粤士大夫在互相交谈时热衷于说官话的现象一去不返了,他们从儿女情长中找到了粤语的美丽、在广大民众中发现了粤语的生命力。而这一切的发生,不过是不到一百年间的事情,这着实是个无比伟大的奇迹。1903年,曾积极参与“维新变法”,后先后流亡日本、澳门、曾师从于日本维新旗手福泽谕吉的新会学者陈子褒用纯粤语书写下了面向妇女儿童的启蒙教材《妇孺三四五字经》,该书迅速风靡整个珠三角。在1900年代,当清帝国境内其它区域的儿童还在以“官话”念诵《朱子家训》中的“黎明即起,洒扫庭除”等语句时,珠三角的儿童已在先生的指导下用母语朗读“早起身,下床去。先洒水,后扫地”\footnote{程美宝:《地域文化与国家认同:晚清以来“广东文化”观的形成》,页157—159}。在这朗朗读书声中,粤语在南粤精英和大众心目中超越了“官话”,成为必须以性命守护的神圣语言,而这一语言的标准语音则是广州音,因为广州是南粤无可置疑的中心。若有岭北人移民至南粤,假如他和他的家族不懂粤语、不认同南粤,那么无论经过多少代人、哪怕已经入籍南粤,也只能被视为“捞头”、“北佬”。相反,若有移民认真学习粤语、归化南粤、为保卫南粤的自由与尊严做出贡献,那么便会理所当然地被视为“自己友”,这便是南粤文明伟大而神圣的语言认同。1911年,清帝国崩溃,广东宣告独立。当时,有人指出:

\begin{quote}
凡省城教员教授、议士辩论、官府谈判,俱是用广东人,讲广东话,故无论何乡何县,皆以学习省话为最要……(若外地人不懂广州话)一到省城,无论学界政界、工界商界,讲话既多误会,且有笑我为乡下佬者\footnote{程美宝:《地域文化与国家认同:晚清以来“广东文化”观的形成》,页155}。
\end{quote}

南粤的民族共同语,至此定型。














