\chapter{第二次北属与士氏时代}

\section{第二次北属:汉帝国治下的南粤}

\indent 公元前111年,汉武帝以极其残暴的手段灭亡了南越国,南粤再度沦入帝国之手,进入了第二次北属时期。在南越国的故土上,汉帝国设置了南海、苍梧、郁林、合浦、交趾、九真、日南七郡。次年,汉军又渡过琼州海峡占领海南岛,增置儋耳、珠崖两郡。作为帝国官吏的驻地,郡城往往是帝国在一地的神经节点。暴汉在南粤设置的郡数竟达暴秦的三倍,可见其控制南粤的决心应是强于暴秦的。

公元前106年,汉武帝分帝国为朔方、兖州、青州、豫州、徐州、冀州、幽州、并州、扬州、荆州、益州、凉州、交趾等十三州,每州设刺史一人。其中,交趾的范围大致相当于今日的两广与越南,刺史驻于西江上游的苍梧郡。由南粤未立州名,而是仅称交趾来看,汉帝国眼中的南粤实为一片与中原迥异的异域。密布雨林和沼泽的南粤蚊虫众多,缺乏抗体的入粤北人往往因此死于疟疾\footnote{马立博:《虎、米、丝、泥:帝制晚期华南的环境与经济》,南京:江苏人民出版社,页68。}。受困于此的北人恐惧地将疟疾称为“瘴气”,并将南粤描绘为可怕的蛮瘴之地。然而,南粤丰富的物产以及海外贸易带来的种种珍宝又激起了汉帝国官员的贪欲,使许多官员视入粤做官为风险和收益并存之事。在粤奸的为虎作伥之下,汉帝国官员对南粤进行了敲骨吸髓式的压榨,给我们的祖先造成了巨大的苦难。

如上一节所述,曾俘杀吕嘉丞相的大粤奸都稽因此“大功”被汉武帝封为“临蔡侯”。不久后,其子襄继承侯位。公元前104年,即南越国灭亡后的第七年,“临蔡侯”襄便因在番禺劫掠而被处死。在海南岛,汉帝国的珠崖太守因当地妇女多蓄长发,竟丧心病狂地“缚妇女割头取发”,制作假发并出售牟利。至汉武帝末年,珠崖太守孙幸又大肆征调当地特产“广幅布”(一种比汉朝布宽三倍的精美布匹),致使民不堪命,群起攻打郡城,击毙了孙幸\footnote{胡守为:《岭南古史》,页160—161。}。此次起义虽然被汉帝国残暴镇压下去,但同样的事件在其后七年内即在岛上发生了六起。公元前82年,汉昭帝被迫撤销儋耳郡。然而,岛上百姓的反抗仍未停止。公元前53年,珠崖郡九个县联合起义,汉宣帝命护军都尉张禄统兵渡海镇压。此后数年内,汉帝国因珠崖战事损失万余军队、耗资三亿钱,仍无法平息当地的反抗。公元前46年,汉元帝终于宣布“罢珠崖郡”,撤走海南岛上的汉人军队、官民\footnote{胡守为:《岭南古史》,页46—47。}。海南岛持续六十四年的抗汉战争,至此以光荣的胜利告终。

两汉之际,南粤因与帝国中心山海远隔,未受大洪水波及。新莽末期,天下大乱,交趾牧(西汉成帝时改刺史为州牧)邓让闭关自守。公元30年,邓让举交趾七郡“归附”于东汉光武帝刘秀。在东汉帝国治下,南粤百姓依然承受着残酷的压榨和盘剥。很快,一场伟大的起义便爆发了。

当时,在相当于今日越南北部的交趾、九真、日南三郡,史前时代以来的传统社会结构仍未被打破。在安阳王、南越国、暴秦、西汉帝国的治下,当地的部落酋长“雒侯”、“雒将”仍依靠百越人的传统习惯法治理各部落。公元39年,贪暴好杀的帝国官僚苏定任交趾太守,大肆蹂躏当地民众。次年正月,麊冷县雒将之女征侧的丈夫诗索被苏定杀害。忍无可忍的征侧与其妹征贰遂起兵反抗,攻破交趾郡城。九真、日南、合浦三郡之民纷纷起义响应二征的义举,连克六十五城\footnote{郭振铎、张笑梅:《越南通史》,页158—159。}。一时之间,越南北部及雷州半岛皆脱离了汉帝国的统治,交趾七郡中有四郡获得了自由。征侧乃称王,定都麊冷\footnote{《钦定越史通鉴纲目》卷2,页10。}。公元41年冬,大惊失色的汉光武帝急忙拜东汉“开国功臣”马援为伏波将军,统兵万余南下镇压。马援时年七十余岁,乃一善于用兵的老将\footnote{陈重金:《越南通史》,页30。}。仓促起事的交趾部落民人数虽多,仍难以抵御马援的进攻。公元42年,马援率军随山开路千余里、沿海而进,与征王之军决战于交趾郡之浪泊。经过激战,征王不敌,退保禁溪,东汉军“斩首数万人”、获降者万余。马援复进兵禁溪,屠杀数千人、招降二万余人,起义军至此瓦解。公元43年,不愿做降虏的征王姐妹撤退至福禄县,投喝江(底江、红河连接处)自尽。至公元44年,马援又率两万军队南进九真郡,屠杀五千余名坚持战斗的起义军余部。为夸耀帝国的“武功”,马援于九真郡居封县竖立了两根铜柱。至此,壮烈的二征起义被东汉帝国的屠刀彻底镇压了下去。在与起义军的战斗中,汉军付出了沉重的代价。至公元44年秋汉军班师北上时,已陷入了“军吏经瘴疫死者十四五”的窘境。

在越南尚未独立的时代,尤其在交州、广州尚未分治的时代,南粤和越南是一难以分割的共同体。两者作为一个整体一同抗击过秦帝国的入侵、一同处于南越国治下、一同被汉帝国占领、一同被划为“交趾”。在征王发动起义后,雷州半岛的居民亦与越南北部的居民一起投入了抗击暴汉的战斗。因此,二征起义不但是越南史上的光辉一页,亦是南粤史上的光辉一页,更是百越先民抗击残暴帝国武断之治的标志性事件。对于二征,越南史学家曾有如下评价:

\begin{quote}
	征侧、征贰以女子,一呼而九真、日南、合浦及岭外六十五城皆应之。其立国称王,易如反掌……惜乎继赵之后至吴氏之前千余年间,男子徒自低头束手,为北人臣仆,曾不愧二征之女子。吁,可谓自弃矣\footnote{“赵”指南越国,“吴氏”指越南吴朝。关于吴朝,详见下文。转引自陈重金:《越南通史》,页30。}!
\end{quote}

此段评价热烈地赞颂了二征的伟大,亦对越南男子徒自屈居于北人之下感到惋惜。黎氏所言,乃出于对帝国横行肆虐之义愤,不无夸张之处。事实上,在这之后的岁月中,我们的伟大祖先与越南人的伟大祖先一直并肩战斗、屡仆屡起,涌现出了无数伟大的男女英雄。

在东汉帝国治下,南粤仍被帝国视为化外之地,承受着沉重的压榨。东汉初年,汉帝国于今湖南郴州设桂阳郡,郡境所辖范围包括岭南之含洭、浈阳一带(相当于今粤北英德之一部)。由于两地距郡治路途遥远,帝国公事往来皆大肆征发民船,百姓苦之。据站在帝国上的史料称,桂阳太守卫飒为革除此弊,下令“凿山通道五百里”,因而“役省劳息”\footnote{胡守为:《岭南古史》,页48。}。然而,此种书于帝国史籍中的“惠政”究竟能否真正及于普通南粤百姓,实在值得怀疑。至于 nï 种在崎岖山地展开的浩大工程到底使多少粤民失去生命,则更非帝国关心之事。此外,又有东汉宫廷在南粤征收荔枝、龙眼以供其享乐的虐政。为保证这些岭南佳果新鲜地运达北方,南粤人被迫承担残酷的徭役,“奔腾险阻,死者继路”\footnote{这一虐政,直至公元105年方被汉和帝取消。见阮元:(道光)《广东通志》卷181《前事略一》}。至于南粤百姓对东汉帝国暴虐统治的反抗,则是史不绝书。178年,交趾、合浦之百越部族“乌浒蛮”起兵反汉,交趾人梁龙聚众数万响应之。此次大规模起义持续四年才被汉军镇压下去,梁龙壮烈战死\footnote{《钦定越史通鉴纲目》卷2,页23。}。184年,不堪压榨的交趾屯兵起义,杀交趾刺史周喁。战事持续一年,方被汉帝国“抚”平。关于激起此次起义的原因,史书称:

\begin{quote}
	为刺史者,以其地有明珠、翠羽、犀象、玳瑁、异香、美材之物,率无清行。财计盈给,辄求迁代,故吏民皆叛之\footnote{《钦定越史通鉴纲目》卷2,页24。}。
\end{quote}

由此可见,贪求南粤珍宝的汉帝国官僚对我们的祖先进行了残酷的压榨。他们无耻地将榨取、抢夺来的粤人财物据为己有,并以之贿赂上官以求升迁。汉帝国贪婪暴虐的统治,乃是激起南粤人反抗的主因。在汉帝国持续近四百年的残酷统治下,我们的伟大祖先没有甘作降虏,而是发动了一次又一次的反抗。此外,他们还延续着百越时代以来的航海传统,继续扬帆出海,为南粤不断探索海外世界。关于当时南粤的海外航线,史籍有详细记载:

\begin{quote}
	自日南障塞、徐闻、合浦船行可五月,有都元国(今苏门答腊);又船行可四月,有邑卢没国(今缅甸境内);又船行可二十余日,有谌离国;步行可十余日,有夫甘都卢国(今缅甸境内)。自夫甘都卢国船行可二月余,有黄支国(今印度马德拉斯一带)……黄支船行可八月,到皮宗(今马六甲一带)……黄支之南有已程不国(今斯里兰卡),汉之译使自此还矣\footnote{班固:《汉书》卷28《地理志第八下》}。
\end{quote}

由此段记载可知,当时南粤先民们的海船不但已能纵横于南海之上,更能通过马六甲海峡、进入印度洋,到达缅甸、印度和斯里兰卡。如果我们将眼光投向当时的整个世界,便会发现令人惊叹和自豪的事实:两千年前南粤先民航海路线的西端,正与公元1世纪罗马帝国商船贸易路线的东端重合。这说明,此时的南粤已经加入到了整个世界的贸易体系中,与世界文明的中心环地中海地区紧紧联系在一起。据《罗马帝国衰亡史》记载,在公元1世纪时,每年夏至都会有约120艘商船从埃及出发,渡海到达印度河斯里兰卡,与当地的“亚洲远邦商人”贸易,并于年底携带货物回到埃及、再经地中海运回罗马\footnote{吉本:《罗马帝国衰亡史》}。这些与罗马人贸易的亚洲商人中,不乏南粤海商。更值得注意的是,南粤海商很可能早在奥古斯都时代就已(公元前27年至公元14年)到达过罗马帝国境内\footnote{据罗马人的记载,在奥古斯都时代,远方的塞里斯人(汉朝人)、印度人都曾遣使奉献珍宝,要求与罗马通商。见《广东通史》,页280。我们可以大胆推测,这些进入罗马的“汉朝人”,很有可能是从海路到达罗马的南粤海商。}。更有明确的记载表明,罗马商人曾在公元166年到达日南,献上象牙、犀角、玳瑁等珍宝\footnote{范晔:《后汉书》卷88《西域传》}。海外贸易使亚、欧、非三洲的种种珍奇货物大批输入南粤,其中包括被称为“璧流离”的蓝宝石、五颜六色的玻璃(流离)制品、华美的珊瑚、琥珀、玛瑙、水晶、名贵的海外香料。至于象牙、犀角、玳瑁之类的珍宝,更是数不胜数。海外贸易给南粤带来了巨量的财富,亦激起了汉帝国官僚的贪欲。在这些寄生虫的盘剥下,我们的祖先依然能不断冲破险阻,纵横于广阔的海上,创造伟大的航海事业,并将南粤与世界连为一体。两千年后的我们读史至此,不能不掩卷肃立,对祖先们致以深切的敬意。

经一百余年的挣扎,罪恶的东汉帝国终于走向了它的末日。公元184年,黄巾之乱在冀州爆发,汉末大洪水开始。同年,李进任交趾刺史,是为近四百年来首个担任该官职的粤人。在此紧急关头,南粤再次处于历史的节点中。面对迫在眉睫的洪水冲击,一位伟大的南粤守护者即将横空出世。他的名字,叫做士燮。

\section{士氏时代:土豪守护下的南粤}

\indent 士燮,字彦威,苍梧广信(今广西梧州,广信系苍梧郡治)人。据说,士燮的祖先本为鲁国人,两汉之际因避新莽之乱迁居南粤,遂以苍梧广信为籍贯。士燮先祖之各代人物已不可考,只知其父士赐曾于汉桓帝时(146—167年)曾为日南太守\footnote{胡守为:《岭南古史》,页72。}。士氏家族究竟是否真的出自北方,实属疑问。其无法考证的祖先谱系,暗示了这一家族很可能是彻头彻尾的土著。所谓的北方出身,则很可能是士氏为在门阀观念兴起的东汉帝国争取话语权而伪造出来的身份。在此后的南粤历史中,本土土豪们曾屡次采用此种伪造手段与帝国周旋、为南粤谋取利益,实为一种保全本土共同体的政治策略。由士赐以本地人的身份在交趾为官来看,士氏对这一策略的运用是相当成功的。

作为一名官宦子弟,士燮曾于少年时代游学于东汉帝都洛阳,师从儒者刘子奇习《左氏春秋》\footnote{陈寿:《三国志》卷49《吴书四·士燮传》}。士赐去世后,他被派往蜀地的巫县任县令。至187年,又升任交趾太守。早在士赐为官时,士氏即已是南粤望族。至此时,士氏在南粤的地位更为稳固。当时,洪水已开始吞没东汉帝国,东汉在南粤的统治岌岌可危。在士燮任交趾太守的同年,首个南粤本地出身的交趾刺史李进离任,吴越会稽人朱符代之。朱符系一个残暴贪婪的帝国官僚。他以乡人担任南粤各地长吏,用重税对南粤先民们施以残酷的压榨\footnote{《钦定越史通鉴纲目》卷2,页29。}。191年,即关东群雄起兵讨伐董卓的次年,粤人的怒火被彻底点燃。在南粤各地,我们英勇的祖先发动了规模空前的大起义、攻城拔寨,矛头直指朱符。朱符狼狈地逃亡海上,仍未能逃脱正义的审判,最终被起义军追上击毙\footnote{《钦定越史通鉴纲目》卷2。朱符毙命之年份,参见邱普艳:《士燮与儒学在交趾的传播》,《平顶山学院学报》2005年第6期,页12—13。}。虽然史籍中未记载士氏一族在此次起义中扮演的角色,但我们有理由相信,士氏与起义有密切的联系:朱符死后,士燮很快便向汉廷请求任其弟士壹为合浦太守、士䵋为九真太守、士武为南海太守,自身难保、对南粤鞭长莫及的汉廷只得答应。在此之前,士壹为交趾郡督邮、士䵋为徐闻县令(属合浦郡)。可见,士氏早在起义爆发前即已在南粤有相当的势力。而在汉廷满足士燮的要求后,南粤的社会秩序很快便得到恢复\footnote{胡守为:《岭南古史》,页73。}。这一切似表明,士燮在发动起义的南粤先民当中有着相当高的威望,且有控制起义军的能力。对于其时士氏在南粤的威望,史书称:

\begin{quote}
	(士)燮兄弟并为列郡,雄长一州。偏在万里,威尊无上,出入鸣钟磬,备具威仪。茄箫鼓吹,车骑满道,胡人夹毂焚烧香者,常有数十。妻妾乘辎軿,子弟从兵骑,当时贵重,震服百蛮,尉他(赵佗之别称)不足踰也\footnote{陈寿:《三国志》卷49《吴书四·士燮传》}。
\end{quote}

至此,交趾七郡中不但已有南海、合浦、交趾、九真四郡被士氏兄弟完全控制,士氏一族的威望更使他们能“雄长”交趾一州,获得足以比肩南越武帝的地位。由被称为“胡人”的在粤外国人亦对士氏兄弟顶礼膜拜来看,士氏亦注意经营海上外贸与国际交流,延续着南粤的海洋文明。至此,经过长达302年的第二次北属时期,南粤终于再次回到了粤人手中。南粤,光复了。

刚刚光复不久,南粤便开始直面生死存亡的考验。公元190年,刘表被任命为荆州刺史。在汉末群雄中,刘表以“贤”名著称,曾积极参与东汉士人反对宦官的党锢之祸。在掌管荆州后,刘表亦对争霸战争不甚积极,摆出一副以仁义治国的面目。然而在面对南粤时,刘表完全是一副凶恶的嘴脸,无时无刻不想将南粤变为自己的领地。公元198年,刘表夺取了长沙、零陵、桂阳三郡。其中,桂阳郡辖区包括浈阳、含洭、曲江三县,浈阳、含洭在今英德境内,曲江相当于今之韶关。刘表的控制区实已越过南岭天险,到达粤北\footnote{胡守为:《岭南古史》,页59。}。为遏制刘表向南扩张的势头,当时已控制汉廷的曹操派出张津前往南粤担任交趾刺史。张津到达南粤后不得不遵从士燮之意,与其联名“上疏”汉廷,要求将“交趾”改称“交州”以提高南粤的政治地位\footnote{对于交趾是否曾改称交州,后世观点不同。阮元所修《广东通志》认为确有此事,胡守为则以为无。今从阮说。}。对于士燮的这一新要求,鞭长莫及的汉廷自然只能满足。不久后,刘表发兵南侵,意图吞并南粤,张津率兵迎战。然而,张津本人热衷于烧香事鬼神、读“邪道书”,毫无指挥能力,其部下亦纪律涣散、十分厌战。公元203年,张津被部将区景杀死。同年,苍梧太守史璜去世\footnote{阮元:(道光)《广东通志》卷181《前事略一》}。野心勃勃的刘表乃趁机遣零陵(今湖南永州)人赖恭为交州刺史、长沙人吴巨为苍梧太守,欲一举控制南粤。关键时刻,曹操为制衡刘表再度出手,迫使汉献帝向士燮发布了一道玺书:

\begin{quote}
	交州绝域,南带江海,上恩不宣,下义壅隔。知逆贼刘表又遣赖恭窥看南土,今以燮为绥南中郎将,董督七郡,领交趾太守如故\footnote{陈寿:《三国志》卷49《吴书四·士燮传》}。
\end{quote}

收到玺书后,士燮遣部下张旻北上向汉廷朝贡,接受了任命\footnote{陈寿:《三国志》卷49《吴书四·士燮传》}。此道玺书的内容颇值得回味。士燮本为交趾太守,然曹操并未将其提升为交州刺史,仅授予其一“绥南中郎将”的称号。至于允许士燮“董督七郡”,则表明士氏对交州南海、苍梧、郁林、合浦、交趾、九真、日南的控制已获得汉廷认可。曹操明白,若在承认士燮控制交州七郡的基础上再授予其交州刺史之职,则帝国将很有可能完全无法控制南粤。曹操此番举动的真实意图乃利用士燮牵制刘表,绝非有爱于南粤。士燮亦通过接受汉廷封官的方式有限度地与曹操合作,从而抵御迫在眉睫的刘表势力入侵、捍卫南粤的自由。由此可见,士燮与曹操之间的互相利用完全是机会主义的,与百越时代之侯与魏国联合反楚之举非常相似。士氏时代的南粤实为一股足以与汉末群雄平起平坐的强大力量,受到了曹操的密切重视。

赖恭、吴巨进入南粤后,很快便开始自相残杀。刘表任命的苍梧太守吴巨乃一野心勃勃之人。他收罗了杀死张津的叛将区景,起兵逐走交州刺史赖恭,盘踞于苍梧郡治广信、拥部曲五千余人,与士燮形成对峙之局。吴巨在苍梧的割据并未持续太久。公元208年,刘表之子刘琮向曹操投降。同年,孙权、刘备联军在赤壁之战中大破曹操。公元210年,挟赤壁之战新胜之威的孙权任步骘为交州刺史,孙吴对南粤事务的插手由此开始。次年,步骘率武吏一千人赴任\footnote{胡守为:《岭南古史》,页65。}。面对孙吴的威势,吴巨不敢正面抵抗,遂派人将步骘迎入南粤。然而,步骘为杜绝后患,便将吴巨、区景二人邀请至自己的驻地会面,并将两人斩于厅前。其后,步骘纠集两万大军沿西江东下,于高要峡击溃吴巨旧部的抵抗,其残部逃亡粤西高凉。吴巨在南粤的势力至此被消灭,士氏直接暴露在了吴军的锋锐之下\footnote{胡守为:《岭南古史》,页66。}。

对南粤来说,吴巨与孙吴都属于外来势力,唯有士燮才是守护本土的土豪。孙吴与吴巨的战争不过是两股侵略者间的火并,孙吴与士燮间的斗争则能决定南粤的历史走向。面对孙吴军队大兵压境,士燮选择了暂时妥协的策略,将步骘迎入南海郡,名义上“归顺”孙吴。步骘到达南海后,以被西汉侵略军烧毁的南越国故都番禺之遗址为基础,大肆修筑番禺城,以之为据点。公元217年,步骘更将交州州治由广信迁至番禺\footnote{胡守为:《岭南古史》,页69。}。这样,孙吴便在南粤的核心区域打下了一颗牢固的钉子。在此形势下,士燮为保护南粤的安宁依然极力维持与孙吴的关系,遣其子士廞入质孙权。出于对士氏的忌惮,孙权亦对其极力讨好,以士廞为武昌太守,封士燮、士壹之子为中郎将。公元221年,士燮又诱蜀汉益州郡土豪雍闿缚太守张裔、归附孙吴。孙权因此升士燮为卫将军、封龙编侯;以其弟士壹为偏将军,封都乡侯。

然而,孙权对于士氏的怀柔仅仅是一种卑鄙的伪装。事实上,他与刘表一样,一直想将南粤据为己有。公元220年,孙权以悍将吕岱代步骘为交州刺史。吕岱一到南粤便采取了比步骘更具进攻性的姿态,迅速招降了盘踞于高凉的吴巨余党。同年,粤北浈阳人王金于南海郡境内发动反吴起义,被吕岱血腥镇压。王金被俘送吴中斩首,起义军遭屠杀、俘虏者高达一万人以上\footnote{阮元:(道光)《广东通志》卷181《前事略一》}。不过,此时的吕岱仍十分畏惧士燮,不敢对士氏贸然动手,以免使自己沦为与朱符相同的下场。

公元226年,士燮以九十高龄溘然长逝,离开了他守护了36年的南粤。令孙权与吕岱最为恐惧的南粤守护者既然已经死去,南粤新的苦难便即将降临了。这一年,孙吴对南粤发动了全面的攻势。在吕岱的提议下,孙权首先将南粤分割为交州、广州两部分,以交趾、九真、日南为交州,以合浦以北各郡为广州。此外,孙权任命吕岱为广州刺史、戴良为交州刺史。至于士氏一族担任的官职则全被取消,仅将九真太守之职授予士燮之子士徽\footnote{胡守为:《士燮家族及其在交州的统治》}。如此一来,坐镇南海番禺的吕岱便能控制广州之地,进而对抗以交趾郡为核心区域的士氏。此时,士氏家族与南粤进入了生死存亡的关头。

在此危机时刻,士燮的继承人士徽却反应迟缓,并未迅速组织起有效抵抗。直至戴良行至合浦,士徽才声明不愿离开交趾,自署交趾太守,发兵阻击戴良入境。然而,抗击外敌的战争尚未正式打响,士氏内部又已先自乱阵脚。原士燮属吏、交趾人桓邻系主和派,不愿对吴用兵。士徽起兵后,劝士徽遵从孙吴之命、迎接戴良。士徽闻之大怒,鞭杀桓邻。桓邻之兄桓治及其子乃发兵攻打士徽,围攻士徽于交趾郡城达数月之久。桓治因久攻不下,乃与士徽相约和亲,各自罢兵。此次南粤内战中,交战双方都动员了自己的核心武装、由宗族子弟组成的“宗兵”参战\footnote{胡守为:《士燮家族及其在交州的统治》}。大量本该抗击孙吴入侵的南粤勇士还未与敌人交手,便白白地在同胞相残的内战中失去了性命,这是极度令人痛心的。

在士徽与桓治进行内战时,吕岱已做好了进攻交趾的军事准备。在场内战刚刚结束,吕岱的军队便从广州出发,直逼交趾。阴险狡诈的吕岱在军中带上了士徽的堂弟、士壹之子中郎将士匡。士匡乃吕岱旧友,当时正在吴中为官。吕岱将其带在军中,系为了通过他招降士徽。吕岱首先致函士徽,“告喻祸福”,接着又令士匡出使交趾游说士徽,称只要士徽交出交趾太守之职便可保无事。当时,刚刚经历了内战的士氏已元气大伤,难以组织起顽强抵抗。在权衡利弊之后,士徽出于对亲人的信任同意放下武器。然而,这不过是吕岱早已设计好的无耻骗局。在吕岱到达交趾后,一场极度卑鄙的阴谋上演了:为表达投降诚意,士徽携兄弟六人肉袒出城迎接吕岱。吕岱假模假样对他们慰勉了一番,令他们复服回城。次日晨,吕岱在城外架设帐幕,邀请士氏兄弟入见。当时,幕中宾客满座。毫无戒备的士徽进入幕中后,吕岱忽然起身宣读孙权的诏书,历数士徽所谓的“罪过”,将其当场推出斩首,传首武昌。

对于我们淳朴勇敢的祖先们来说,侵略者的卑鄙着实超乎相像。直到这时,他们才终于如梦初醒。原与士徽兵戎相见的桓治联合士徽帐下大将甘醴、统率南粤吏民发动了迟到的反抗。然而,大势已去,一切都太晚了。已经控制了南粤局势的吕岱血腥地镇压了这次起义,彻底消灭了士氏在南粤的势力,随即将交州、广州重新合二为一\footnote{陈寿:《三国志》卷60《吴书十五*吕岱传》}。数年后,孙权借口士壹、士䵋“犯法”,将他们残忍地杀害。除曾被士燮送入吴中为质的士廞系病死外,士氏一族的男丁多不得善终。士匡的下场史籍未载,他很有可能在被孙吴榨取完价值后杀害\footnote{胡守为:《岭南古史》,页79。}。随着士氏悲惨地凋零,南粤的自由又一次失去了,粤人又一次陷入了帝国的魔掌。

在南粤史中,士氏时代持续了36年。这一时代虽然短暂,却无疑具有重要地位:在士燮的奋力拼搏下,南粤驱逐东汉流官、躲过了汉末大洪水的侵袭。作为一名守护乡土的南粤土豪,士燮不但保护了南粤的海洋传统,更殚精竭虑地周旋于刘表、曹操、孙吴、蜀汉等岭北强权之间,保卫了南粤来之不易的珍贵自立。只要他在世一天,他便如一尊高大的守护神般保卫着南粤的百姓与河山,使侵略者不敢吞并南粤。他死后,粤人在侵略者面前未能团结,惜败于孙吴帝国的卑鄙阴谋之下。这一惨痛的历史教训,值得我们永远铭记。


