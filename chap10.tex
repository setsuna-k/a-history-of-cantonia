\chapter{南粤发明的启动:十六、十七世纪的宗族化运动}

\section{魏校的毁淫祠运动}

公元1513年初,一支闽越漳州的商贸船队从马六甲起航,他们的目的地是南粤沿岸。船队主人被称为“蔡老大”,是个常年往返于琉球、闽、粤、马六甲之间的海商。对于他而言,这次出海不过是又一次平常的商业航行。然而,船上一位名叫乔治·欧维士(Jorge Álvares)的葡萄牙商人却使这次旅途显得别具意义。葡萄牙人在两年前刚刚攻占南亚交通要道马六甲,他们十分希望继续北航,找到《马可·波罗游记》中所记载的那个神秘东方世界。在葡属马六甲总督阿尔布科尔科(Albuquerque)的命令下,欧维士以“商务代理人”的身份登上蔡老大的商船,承担起发现新世界的使命。不久后,船队抵达一处名为Tamao的荒岛,欧维士在此登陆,立下一块刻有葡萄牙国徽的“发现碑”,并于次年返回马六甲。Tamao岛即是屯门岛,位于珠江口东南侧,乃南粤的岛屿。就这样,葡萄牙人“发现”了南粤,南粤首次接触到了处在大航海时代的西方文明,从此与全球连为一体\footnote{关于欧维士航行至屯门岛之事,参见张伟保:《略论葡萄牙人在中国东南沿海的活动(1513—1552)》;林梅村:《寻找屯门岛》}。波澜壮阔的南粤近代史,由此拉开序幕。

当葡萄牙人到达南粤时,南粤社会内部正进行着一场史无前例的变革。这场伟大变革重塑了南粤的社会形态,使南粤得以成为今天的模样。在讨论西方人与南粤的接触前,我么首先要明白当时南粤的情形。因此,本章便先讨论这次变革。

1520年,吴越苏州人魏校来到广州,就任明帝国的广东提学官。所谓“提学官”,乃明帝国省级教育行政长官,负责督查一省各府、州、县学的教学质量。因提学官往往兼任按察副使(一省司法之副长官),所以又被明人俗称为“提学副使”。魏校是个原教旨主义的洪武社会主义者,对于当时南粤境内种种不合“朝廷礼制”的“淫祠”十分敌视。朱元璋早在1370年便曾下令,除明帝国载于“祀典”的神祗外,帝国境内不准崇拜别的神。那些供奉不被帝国允许祭祀的神的庙宇、神坛,被称为“淫祠”\footnote{井上彻:《魏校的捣毁淫祠令研究——广东民间信仰与儒教》,《史林》2003年第2期}。如前所述,古代南粤一直有着丰富的本土信仰体系。我们的祖先在对悦城龙母、波罗神、三山神、雷祖神的崇拜中曾多次打退岭北帝国在意识形态领域的侵犯,捍卫了南粤的宗教信仰。明初,朱元璋虽曾改造南粤社会,但由于南粤本土信仰实在树大根深,他亦不敢轻举妄动。这样,直到16世纪早期,南粤仍是个遍布本土庙宇、神坛的地方,儒家思想完全无法扎根。 

在魏校tay 来,ni di 本土信仰通通都hay “淫邪”的。koy 要继承朱元璋的未竟事业,将粤人在意识形态上彻底变为帝国的降虏。1521年十一月,魏校发布了捣毁广州城内“淫祠”的命令:

\begin{quote}

照得:广城淫祠,所在布列,扇惑民俗,耗蠹民财,莫斯为盛。社学教化首务也。久废不修,无以培养人才,表正风俗。当职怵然于衷,拟合就行,仰广州府抄案委官,亲诣坊巷。凡神祠佛宇不载于祀典,不关风教、及原无敕额者尽数拆除,择其宽厂者改建东、西、南、北、中、东南、西南社学七区,复旧武社学一区。仍量留数处,以备兴废举坠。其余地基堪以变卖,木植可以改造者,收贮价银工料,在官以充修理之费。斯实崇正黜邪举一而两便者也\footnote{魏校:《庄渠遗书》卷9}。

\end{quote}

命令中,魏校视遍布广州的神庙、佛寺为“淫祠”,下令将大部分拆毁,并改造为教授官方儒学以培养科举士大夫的基层学校社学。此外,“淫祠”的地基也要售卖,从而为官府赚取修建社学的经费。魏校在命令中提到的七所社学全由佛寺或道教祠宇改建而来。各社学不但建有校舍,还有“学田”,这些学田系被没收的“淫祠”田土。同时,魏校特别保留了历史悠久的光孝寺、玄妙观,下令将广州的所有佛僧、道士皆归入这些寺观。至于未持有明帝国下发的“度牒”(出家许可证)的僧道,则全被强制还俗\footnote{井上彻:《魏校的捣毁淫祠令研究——广东民间信仰与儒教》,《史林》2003年第2期}。

不久后,魏校进一步扩大运动规模,将毁淫祠的范围扩展到广州城外的农村。不但广州附廓县番禺、南海的淫祠被下令全部拆毁,连附近的高明、四会、增城、新会、从化、新宁六县也要一体遵从。此外,就连粤西雷州府、廉州府及粤北南雄府也接到魏校的命令,被要求将境内“淫祠及废寺观尽数折毁”\footnote{井上彻:《魏校的捣毁淫祠令研究——广东民间信仰与儒教》,《史林》2003年第2期}。在南粤各地,明dayguog 官僚发动皂吏、民夫冲进一间间本土神庙,砸毁神像、砍断木柱、推倒墙壁,将之改建成社学。魏校甚至曾一度想拆毁供奉着六祖慧能真身的韶州南华寺,并确曾将六祖的衣钵敲碎,可谓丧心病狂至极\footnote{任建敏:《从“理学名山”到“文翰名山”——16世纪西樵山历史变迁研究》,页43}。在魏校的蹂躏下,南粤的本土信仰遭遇了空前浩劫。

1522年,魏校离任。对于他究竟破坏了多少间“淫祠”,史籍未载。但从他兴建的社学数量上,可以一窥此次运动的规模。据明帝国出版于1535年的《广东通志初稿》记载,广东十府的社学数为:广州府207所、韶州府54所、南雄府19所、惠州府32所、潮州府15所、肇庆府59所、高州府93所、廉州府6所、雷州府15所、琼州府185所。可见,毁淫祠运动在广州府和海南岛(琼州府)进行得最为彻底。在魏校来粤之前,南粤境内几无社学,广州府的南海、番禺两县更是一间都没有。而至1535年时,南海、番禺竟各有105所、48所\footnote{井上彻:《魏校的捣毁淫祠令研究——广东民间信仰与儒教》,《史林》2003年第2期}。明帝国对南粤本土信仰的摧残与迫害之深,由是可知。

不过,值得注意的是,到1535年时,这些社学已大多因难以维持而陷入废弃状态。与此同时,一间间粤人自己建立的书院、祠堂正拔地而起。在这十余年间究竟发生了什么?欲明白这一问题,便需追踪当时的南粤精英是如何应对毁淫祠运动的。

\section{大礼议与粤绅:湛甘泉、霍韬、方献夫}

现在,让我们将目光稍稍前移一小段时间,看一看1510年代的南海县西樵山(在今佛山市境内)发生了什么。当时,名叫霍韬(佛山石头村人)、方献夫(南海丹灶人)、湛若水(增城人)的三名南粤士大夫正隐居在这座山上。他们皆对其时在位的皇帝明武宗深为不满,因此选择离开官场,入栖此山。

在儒家士大夫眼中,明武宗是个离经叛道的皇帝。他不喜朝政,时常和一群宠臣一同四处游乐。他还曾不顾朝臣劝阻,偷偷溜过长城,前往晋北与蒙古人作战。1512年,对朝政深感失望的方献夫从北京吏部辞职,回到南粤,入居风景绝佳的西樵山大科峰。1514年,方献夫的好友霍韬考中进士,随即辞官不就,回到家乡,于次年来到西樵山读书。1517年,正在家乡为母守丧的湛若水亦移居西樵山,从而为一幕壮阔的历史活剧拉开了序幕\footnote{任建敏:《从“理学名山”到“文翰名山”——16世纪西樵山历史变迁研究》,页64—107}。

湛若水,字元明,号甘泉,世人多称之“甘泉先生”。1466年,甘泉出生于增城的一个土豪家庭。他幼年丧父,由母亲抚养成人,有着与陈白沙相似的成长经历。1492年,二十七岁的他通过广东乡试,成为举人。在这之后,他和青年白沙一样对人生意义产生怀疑,因此放弃科举,于1494年来到新会,师从时年六十七岁的陈白沙。是年,白沙因爱徒林光决意前往山东做官将之逐出师门,心情低迷。甘泉的到来,无疑给年迈的白沙带来了一股春风。聪明好学的甘泉在白沙的耳提面命之下迅速掌握其学说,深受白沙喜爱,一跃成为白沙最信任的弟子之一。去世前夕,白沙特意将他与朋友、学生一同在海边垂钓的场所“江门钓台”作为衣钵传给甘泉,以示甘泉乃其学术继承人\footnote{朱鸿林:《读张诩"白沙先生行状"》,《岭南学报》新第1期}。

白沙去世后,甘泉本想继承其师的事业,永不出仕。然而,在母亲的严命之下,他不得不于1505年进京赶考,一举成功。入仕后,甘泉对为官无甚兴趣,却被王守仁(阳明)的讲学吸引。当时,正在为官的阳明已开始在北京宣扬其唯意志论色彩浓厚的心学,开门授徒。甘泉虽对阳明的讲学行为十分赞赏,且与之结为好友,却对阳明学中过分强调个人意志的观点相当不满。为反击阳明,他提出了“随处体认天理”的宗旨,亦开始讲学授徒。据甘泉自述,他的学说与阳明的不同之处在于:

\begin{quote}

阳明与吾言心不同,阳明所谓心,指方寸而言。吾之谓心者,体万物而不遗者也\footnote{张廷玉:《明史》卷283《列传一百七十一湛若水传》}。

\end{quote}

空疏而近似成功学的阳明学的出现暗示着诸夏文明季候的衰落。相形之下,甘泉学强调对万事万物的体认,显然更为笃实。这一学说,无疑与白沙遍观群书、提倡“学贵自得”颇有关系。此后数十年间,甘泉虽然一直保持着对阳明的尊重,但始终坚持自己的观点,成为明帝国境内除阳明之外的二号大儒,制造了“天下学徒,不归王则归湛”的盛况\footnote{张夏:《雒闽源流录》卷14}。对于甘泉在明帝国境内四处讲学的情形,本书不拟详细讨论。我们只需注意,当甘泉在1517年来到西樵山时,他已是个名动天下的大学者了。

甘泉在西樵山保持着讲学的习惯。他的行为影响了方献夫和霍韬。很快,三人便在山上建立书院,一边开馆授徒,一边互相往来研讨学问\footnote{罗一星:《明清佛山经济发展与社会变迁》,页86}。西樵山林深瀑长、云岚时起,乃一处绝佳盛景。在这样的美景中,三人日复一日地讲学、会面、游玩,培养了深厚的友谊,亦产生了相近的政治立场。

1521年,明武宗病死。因其无嗣,帝位由身在湖北的兴王继承,是为明世宗。明世宗登基后,武宗朝的佞臣被一扫而空。湛、霍、方三人认为,这一变局象征着朝政的革新,他们出山的时候到了。一年内,三人相继进入北京朝中,随即卷入激烈的政治斗争。当时,明帝国正面临着空前的宪法危机,那便是明史中著名的“大礼议”事件。

明世宗刚到北京时,朝臣便希望他追认武宗之父孝宗为“皇考”,而以其生父为“皇叔考”。朝臣们意在维护明帝国皇位传承的法统,却遭到明世宗的激烈抵制。世宗不但执意追尊生父为“皇考”,还要将其追尊为帝、使其牌位能被放入太庙。争端最终以流血收场。1524年七月,二百余名大臣在皇宫左顺门外大声嚎哭,声震阙廷,是为“左顺门事件”。明世宗下令锦衣卫将其中134人逮捕,施以杖刑,活活打死17人。至此,明世宗获得“大礼议”之争的全面胜利。

在“大礼议”的过程中,有五名官员坚定地站在明世宗一边,霍韬、方献夫身列其中。湛甘泉虽一度站在朝臣一边,但态度暧昧,很快便转向中立,未参与“左顺门事件”。事后,明世宗对见风使舵的甘泉颇不满意,将其调往南京\footnote{黎业明:《思想与政治:湛若水与“大礼议”之关系述略》,《深圳大学学报》(人文社会科学版)2009年第5期}。对霍、方二人,他则投桃报李、委以重任。此后,两人平步青云,霍韬官拜礼部尚书,于1540年病逝于北京;方献夫更于1532—1534年间进入内阁,成为明世宗的宠臣,后辞职归西樵山,于1544年病逝。相形之下,湛甘泉仕途不顺,此后一直在南京为官,但他的学术成就却远远超过霍、方。在后半生,甘泉一直以传播其师陈白沙的学说为己任,先后于南京、扬州、增城、广州、南岳衡山开设书院。1540年辞官归粤后,他依然讲学不倦。史载,当他在1560年以九十五岁高龄去世时,竟有学生3900余人\footnote{张廷玉:《明史》卷196《列传八十四方献夫传》;卷277《列传一百二十八霍韬传》;卷283《列传一百七十一湛若水传》}。甘泉学的广泛传播,无疑有力地阻碍了阳明学的发展、延缓了东亚大陆文明的衰败,实为我南粤对诸夏的又一大贡献。

对明帝国而言,大礼议是一次没有战争的“靖难之变”(按:即明成祖篡位之事),明世宗“乾纲独断”地破坏了帝国的法统。在帝国文官集团眼中,迎合明世宗的湛甘泉、霍韬、方献夫实乃小人。然而,若我们站在南粤的立场上,便能对此事作出完全不同的解读。在历次北属时期,南粤一直被帝国视同化外,粤人则近似“化外之民”。对南粤精英来说,帝国的法统实则与南粤自身的利益关系不大。若能利用帝国的宪法危机使粤人进入帝国权力核心,从而在帝国霸占南粤的现状下为粤人获得话语权、谋取更多利益,亦不失为一件好事。事实上,霍韬便是这一策略的忠实执行者。在下一节中,我们便将探究以他为代表的南粤精英是如何做的。

\section{南粤精英的宗族建构}

1523年,霍韬因在“大礼议”中支持明世宗而饱受同僚攻击,被迫暂时回粤躲避风头。甫一回粤,他便开始进行一系列令人目瞪口呆的活动。当时的西樵山上,原有一座名叫宝峰寺的佛寺。在刚刚过去的毁淫祠运动中,该寺因僧人被附近居民举报称有淫乱行为而惨遭拆毁,其田产亦成为无主之地。霍韬迅速出手买下三百亩寺田,随即率百余名家人迁入西樵山中,尽力经营这批田产。经一年余的劳作,这批田产已为霍氏带来可观的收入。霍韬遂以这些收入为经费,为族人修建祠堂。1525年正月一日,富丽堂皇、存续至今的霍氏大宗祠在佛山石头村完工,祠堂的建筑用地亦为被霍韬买下的原“淫祠”之地。在霍韬的设计下,祠堂正中仿照《朱子家礼》的规定,摆放其始祖夫妇、高祖父母、曾祖父母、祖父母、父母的牌位\footnote{任建敏:《从“理学名山”到“文翰名山”——16世纪西樵山历史变迁研究》,页172—173。《朱子家礼》规定,祠堂正中应方高、曾、祖、考四代牌位。霍韬的摆放方式和《朱子家礼》略有不同。}。大宗祠建成时,霍韬规定:族内设家长一人“总摄家事”、宗子一人“惟主祭祀”。若有人贤能,可兼任宗子、家长。霍韬特意将其兄霍隆立为家长,以示自己虽为高官,但在家族中仍应遵守小共同体的伦理道德,服从兄长。除为族人提供以儒礼祭祀祖先的场所外,大宗祠还有救济贫困族人的功能,族中田产不足四十亩者可每年从大宗祠领取十石粮食\footnote{罗一星:《明清佛山经济发展与社会变迁》,页102}。这样,霍氏族人便再无饥寒之忧。

除建宗祠、立家长、宗子外,霍韬还为族人设计了“考功”制度。他在家长之下设“田纲领”、“司货”两职,由族人轮流担任,每年一换。田纲领的任务乃组织族中农业生产,计算业农族人在一年中的收获量,以亩入十石为“上功”、七石为“中功”、五石为“下功”。司货则统计从商族人在一年中的营业额,以获田五亩、银三十两为“上最”,田二亩、银十五两为“中最”,田一亩、银五两为“下最”。族人需将其收入交给田纲领、司货,由两者存入粮仓以充宗族经费。田纲领、司货还要将一岁收支情形、族人“功最”禀告家长,以使家长可依之奖惩族人。家长对族人的奖惩亦要遵循严格规定,绝不可滥用权力。每年元旦,家长要将族中男丁聚于大宗祠中举行“岁报功最”仪式。仪式开始时,家长立于祖宗牌位旁,族人立于两侧廊内。接着,每人“以次升堂,各报岁功。报毕,趋两廊序立。”凡报“上功”、“上最”者,家长祝酒于祖先牌位,并将超额收入的十分之一赏给此人作为个人财产。报“中功”、“中最”、“下功”、“下最”者,无赏无罚。至于游手好闲、一年中毫无所获者被称为“无庸”,要被司货拉出队列并跪在堂下向祖先请罪。对连续三年皆“无庸”者,更要用荆条鞭打二十下以示惩戒\footnote{罗一星:《明清佛山经济发展与社会变迁》,页104—105}。

“考功”制度之外,又有“会膳”制度。霍氏一族内有许多小家庭,这些家庭并无一同进餐的习惯。大宗祠建成后,霍韬开始与族人商议举族聚餐之事。至1526年初,用于聚餐的厨房修建完毕,霍韬乃于同年二月带领全族百余人举行首次“会膳”。据霍韬规定,会膳于每月朔(一日)、望(十五日)举行,即每年举行二十四次。凡逢会膳日,全族男女老幼需先着正装拜谒大宗祠,接着对家长行两拜之礼,再以长幼之序交相参拜,其后方可入座。在此过程中,设男女礼生各二人。礼生若发现举止失仪者,便要禀告族长,剥夺此人参加会膳的资格,令其在堂下一直跪到会膳结束。开始用餐前,家长会命礼生宣读有功族人的善行及有过族人的劣迹。会膳中,每八人一桌,男女分区而坐、不得同桌。虽然男女分坐,但两者所用之桌大小一致,菜品亦皆为每桌肉三碟、菜两碟,唯女桌无酒。可见,霍韬虽然强调男女有别,但绝无苛待女性之处。直到今天,中原的许多地区仍然不准女人上桌吃饭。而早在五百年前,我南粤的宗族便在饮食上对男女一视同仁,并无不准女人上桌进餐的愚昧做法。南粤文明程度之远胜岭北,由此可窥一斑\footnote{罗一星:《明清佛山经济发展与社会变迁》,页106—109}。

除上述两种制度外,霍韬还十分重视对宗族子弟的教育。1525年十月,他在大宗祠旁建成石头书院,用以教育年满十八的族中子弟。入学者不但要进修文化知识,还必须习农事、耕种学田,以收获物赈济同乡贫民。1527年,霍韬被明世宗召回北京,但他仍以书信指导族人的生活、工作和教育。1530年,霍韬因母亲去世回粤服三年之丧,开始在西樵山的宝峰寺遗址上修建新书院。两年后,颇具规模的四峰书院建成\footnote{罗一星:《明清佛山经济发展与社会变迁》,页112}。该书院与湛甘泉的大科书院、方献夫的石泉书院鼎足而三,成为西樵山上的文化中心\footnote{任建敏:《从“理学名山”到“文翰名山”——16世纪西樵山历史变迁研究》,页93}。霍韬将族中子弟移入其中,延请名师来教,并亲自授课。课余时间,诸子弟“耘菜灌园”,过着自给自足的生活\footnote{罗一星:《明清佛山经济发展与社会变迁》,页113}。

为给自己兴建的书院披上“合法”外衣,霍韬迎合明帝国“毁淫祠”、立社学的政策,将这些书院称为“社学”。由于他是明世宗的宠臣,广东的明帝国官员不得不默许他这种做法。霍韬对子弟的教育取得了喜人成就,他的九子中除第四子、第五子早夭外,其他七人均为秀才、举人,其中次子霍与瑕更于1559年中进士,任吴越慈溪知县,因力抗权臣严嵩而一度丢官\footnote{罗一星:《明清佛山经济发展与社会变迁》,页114}。霍氏子孙的强烈正义感,与霍韬对他们的反复规训是分不开的。1533年,霍韬获授吏部侍郎,广东官员无不由其选拔。当时,霍韬族人多有借其名横暴乡里者。对此,霍韬在家信中反复教导,语气至为诚恳。以下试举数\footnote{转引自罗一星:《明清佛山经济发展与社会变迁》,页117—118}例:

\begin{quote}

只愿兄弟子侄勿生事,为我累。家中如此尽够了,若不知足,是得罪天地神明也。

我居此地,当以廉介率百官。如辞守取予不严,赃官何所警戒?

予之不德,固惟日恐畏,真如临深,真如履薄。如兄弟亦幸深体此心,谨身慎行,齐整家法,不可非议。俾予早早致仕(退休)回去,保全今名,乡人称之曰:“我岭南士夫保有终誉惟某氏一家而已”,岂不美哉?

\end{quote}

除以书信规劝外,霍韬还在1536—1537年间支持南海知县黄正色(吴越江阴人)严厉打击虐待乡人的霍氏族人。在霍韬的软硬兼施下,石头霍氏产生了认真劳作、努力经商、崇尚正义的家风。当时,有部分南粤士绅督家不严,放纵其子弟肆意妄为。例如,1520年代初任阁臣的南海人梁储就曾放纵子弟肆意妄为,其子梁次摅甚至曾为争夺田产而率家丁屠杀富豪杨端一族二百余人\footnote{赵翼著、王树民校证:《廿二史札记校证》卷34,页825}。与这种家族相比,霍氏的家风无疑是相当端正的。

霍韬借其在“大礼议”事件中获得的政治地位收购“淫祠”地基、田产,建立祠堂、书院、经营族产的做法,为16世纪的南粤精英提供了一套建设宗族共同体的样板。这一样板,被学界称为“霍韬模式”。当时,有许多粤绅在霍韬的影响下建立宗族、创设书院。例如,湛甘泉和香山人黄佐(关于黄佐,详见下节)便曾在广州白云山大量收购被毁佛寺的地产,分别建立甘泉、泰泉两书院\footnote{屈大均:《广东新语》卷17}。原教旨洪武社会主义分子魏校曾欲通过毁“淫祠”运动在精神上彻底征服南粤,却只能给粤绅建设宗族共同体提供方便,这实为一极大的讽刺。缺乏格局感的他在后洪武时代采取洪武社会主义政策,便只能导致这种南辕北辙的结局。粤人的宗族共同体利用明帝国的话语体系包装自己,像公司一样经营产业,又用朱子学保持族人的道德感。西方文明及珠江口沙田开发带来的无限商机使这些宗族获得了充足的利源,宗族成员又通过祭拜共同祖先紧密团结在一起,将宗族的实力变得越来越强。那么,宗族是如何建构共同祖先的?珠江口沙田开发又是怎么一回事?这些,便是下一节将讨论的问题。

\section{沙田上的祖先发明者:南粤小华夏的诞生}

今日的珠江三角洲无疑是南粤的经济中心。这里不但坐落着广州、东莞、深圳、珠海等重要城市,亦遍布着繁华的村镇、密集的厂房。然而在一千年前,珠三角的大部分土地仍是一片汪洋。那时,珠江口是个深入内陆150公里的巨大河口湾,广州城南不远便是大海。在相当于今日东莞西部、深圳西部、顺德、中山、新会的地方,只有一点星罗棋布的海岛冒出海平面,今天的珠海则是个四面环海的巨大岛屿。1120年代末,靖康之变发生,许多岭北难民翻越大庾岭,经南雄珠玑巷进入粤北。紧接着,湘赣流寇大举入掠粤北,大批粤北居民又逃入珠三角,使珠三角人口激增,陷入了地多人少的窘境。因此,珠三角先民在12世纪大举修筑堤围,令许多土地得以免受海水、江水侵蚀,成为肥沃良田。当时,南海九江镇修建了延续至今的著名农业水利工程“桑园围”。当地居民以新填土地为鱼塘,在鱼塘周围建起堤围,并于堤上种植桑树。桑树可巩固堤围,桑叶可养蚕取丝,蚕粪可做鱼粮,鱼粪又能积为塘泥、为桑树施肥,由此形成良性循环的“桑基鱼塘”生态系统。此后七百余年间,桑园围屡经扩建。至20世纪,桑园围已成长为周长68.85公里、围内面积133.75平方公里、捍卫良田1500公顷的巨型工程\footnote{邓芬:《桑园围——珠江三角洲最大的堤围工程》,《农业考古》2006年第1期}。

除建造“桑基鱼塘”工程外,珠三角先民开垦新土的另一种方式是开发沙田。沙田的开发过程是艰辛的,一块成熟沙田的形成期往往长达数十甚至上百年。若欲开发沙田,首先要在欲填海的目标水域修建堤围,并向其中投掷大量石块,从而制造被称为“底基”的石质地基。其后数年至数十年间,精心设计的堤围会改变水流,将珠江带来的泥沙不断引入堤围内底基上,形成泥滩。泥滩成型后要迅速种上芦苇、水草进行加固。再经过几十年,若泥滩仍不移位,便成为了可种植水稻的坚固陆地“沙田”\footnote{关于沙田制造法,参见科大卫:《皇帝与祖宗:华南的国家与宗族》,页321}。到16世纪末,相当于今日珠海的那个大岛已与陆地连接起来,今日珠三角的基本格局成型了\footnote{刘志伟:《地域空间中的国家秩序——珠江三角洲“沙田——民田”格局的形成》,《清史研究》1999年第2期}。这是世界上除荷兰外最大的填海造陆工程,乃我们的伟大祖先创造的文明奇迹\footnote{徐承恩:《郁躁的城邦:香港民族源流史》,页54}。

15、16世纪这沙田开发的最后阶段,亦为沙田开发的最高峰。1450年,黄萧养之乱结束,珠三角土豪的自治权得到明帝国默许。由于沙田开发是项费时费力的工程,非能长期动员大批人力的殷实之家难以进行。因此,这两个世纪的沙田开发实为一场土豪间的圈地竞赛。在珠江口西岸的顺德县、香山县(相当于今日中山、珠海、澳门),竞争尤其激烈,土豪们往往召集大批人员乘船抢夺沙田,导致械斗经常发生,促进了珠三角社会基层的武装化\footnote{井上彻:《明末珠江三角洲的乡绅与宗族》,《中国社会历史评论》第十卷}。在16世纪,沙田开发的艰难、械斗的频繁促使参与其中的土豪共同体纷纷将自己打扮成儒化宗族,从而加强共同体的团结与战斗力。对于这些土豪来说,发明一个令人骄傲的祖先以凝聚族人实为要务。而当时流传于南粤的“南雄珠玑巷移民传说”,正是发明祖先的绝好材料。该传说称:宋帝国治粤时期,有一在杭州经商的粤北南雄商人黄某与一胡姓妃子私奔,两人隐居于南雄珠玑巷。珠玑巷是一条南北走向的古老街巷,乃由岭北翻越大庾岭入粤后的必经之地,往来南北的富商大贾、文人学士都曾路经此巷。据说,此巷居民亦都是岭北移民之后。宋帝闻知胡妃逃跑后,派兵入粤捕捉,惧怕株连的33户97名珠玑巷居民乃南下广州、新会等地躲避,胡妃则赴井自尽。在史籍中,我们甚至还能看到一则据说系由“冈州(今新会)知县李丛芳”发出的准许珠玑巷居民入籍的告示:

\begin{quote}
	

普天之下,莫非王土。率土之滨,莫非王臣。贡生罗贵等九十七人,既无过失,准迁移安插广州、冈州、大良都等处,方可准案增立图甲,以定户籍。现辟处以结庐,辟地以种食,合应赋税办役差粮毋违,仍取具供结册,连路引繓赴冈州\footnote{转引自科大卫:《皇帝与祖宗:华南的国家与宗族》,页85}。

\end{quote}

这一告示无疑是后世伪造的,因为图甲(里甲的别称)是明帝国的制度,不可能出现在宋帝国治粤时期。至于传说中皇妃与富商私奔的故事更是带有浓厚的浪漫色彩,亦很难说是事实。纵然如此,这一传说仍然带有诗意的真实,它反映了两宋之际难民从岭北、粤北逃至珠三角的历史。南雄珠玑巷是许多岭北难民入粤时的必经之路。经数百年时间,南雄珠玑巷已成为大批珠三角家族共同的历史记忆,而这一传说则是由此历史记忆演化而来的。事实上,这些家族中有北方血统者早已归化南粤文明,变得与粤人毫无二致。珠三角的居民们自己也说不清到底谁是北人后裔、谁是南粤土著。只有口耳相传的“南雄珠玑巷移民传说”,暗示着这些居民半信半疑地认为自己有来自北方的祖先。对致力于开发沙田的南粤土豪来说,若能借此将自己的家族塑造为岭北古华夏贵胄之后,无疑能极大地提升族人的荣誉感,并使自己的家族在帝国内部获得更多话语权。香山学者黄佐便为我们提供了一个绝妙的祖先发明案例。

1490年,黄佐出生于香山县一个读书人家庭。这一家庭的早期历史模糊不清,直到黄佐的祖父黄瑜在15世纪中期中了举人后才有文字记载。黄瑜曾任惠州长乐县令,但因为人耿直,很快便得罪上司,辞官而去。黄佐的父亲黄畿自小便在严谨的家风中长大,一生究心于儒、佛、道的学问,拒不入仕。在父祖的耳濡目染下,黄佐养成了笃实、正直的性格。他于1520年考中进士,并在其后的“大礼议”中站在朝臣一边反对明世宗,辞官还乡。1523年冬,他在归乡途中专程前往吴越绍兴拜会当时名动天下的王守仁(阳明),与之“食息与俱”,论学七天。黄佐无论如何都不能同意阳明“知行合一”的说法,认为“知犹目也、行犹足也”,学者必需“先知后行”,即先笃实地积累学问,而后方可行动。强调唯意志论的阳明见无法说服黄佐,只得黯然叹称:“直谅多闻,吾益友也。\footnote{朱鸿林:《儒者思想与出处》,页308}”四年后,黄佐在广州再次会见阳明。其时阳明已然病危,急欲收黄佐为徒,被黄佐断然拒绝\footnote{朱鸿林:《儒者思想与出处》,页316}。

做为一名虔诚的朱子学信徒,黄佐不但坚决反对阳明学,亦积极投身于建构儒化乡土共同体的实践。由于他自小生长在沙田开发事业如火如荼的香山县,因此对当地人建构共同体以提高乡族凝聚力的需求十分了解。他整理了父祖有关祖先的文字,又增添了不少内容,将自己的家族发明为岭北高官之后:

\begin{quote}

若吾宗之所自出,相传为蜀汉将军忠之裔。唐末有鷟者,隐居有奇操,石晋征拜谏议大夫。值乱,徙入筠州。入宋,子孙益衍,巍科膴仕,往往而有。其昭然可据者,则谏议裔孙,度支员外郎汉卿为一世……汉卿生某,某生某,二世皆阙其名。某生处士文敬,文敬生迪功郎重载,重载生朝奉,即楚州监税雍,雍生元西台御史宪昭,以直谏驰声朝署。会禁汉人、南人不得蓄兵器,犯者论死,乃上疏言:“天生五材,谁能去兵,苟以南北异视,人人疑惧,为变非小。”忤虏君臣意,贬岭南,卒于途。子从简藐然孤孑入广,留家南海之西濠,是为始迁祖也。元末左丞何真起兵卫乡闾,众推率为副,累有功,官至宣慰司副使\footnote{刘志伟:《从乡豪历史到士人记忆——由黄佐"自叙先世行状"看明代地方势力的转变》,《历史研究》2006年第6期}。

\end{quote}

此段文字中,黄佐称自己的家族很可能是蜀汉名将黄忠之后,唐末五代时曾有名黄鷟者入仕后晋,后避乱迁居蜀地之筠州。宋帝国时期,黄氏族人多有参加科举并入仕者。至元帝国时期,有名黄宪昭者为元帝国御史,因得罪蒙元君臣被贬岭南,卒于途中。其子黄从简只身入粤,成为香山黄氏的“始迁祖”。其后,黄从简追随何真起兵,一跃而为何真的副手。

这一记载的可靠性极其令人怀疑,因为元明之际关于何真的史料中并无关于黄从简的任何记录。黄佐的祖父黄瑜曾留下一段文字记载黄从简的事迹,这是现存史料中关于此人的最早记载。黄瑜仅称黄从简乃何真麾下一骁将,并未提及其北方祖先,更未说过此人乃元帝国御史之子。可见,黄从简以上的黄氏世系极有可能是由黄佐编造的。假若历史上果真存在黄从简这个人,那么由何真政权的性质来看,此人很可能只是一个在乡里颇有威望的南粤本地土豪。

黄佐并不仅止步于发明岭北高官祖先,他还要重构粤人的历史,将粤人为华夏最正统的后裔。1561年,离去世仅有五年的黄佐完成了系统叙述南粤风土人情的巨著《广东通志》。在书中,黄佐这样叙述南粤历史:

\begin{quote}

汉,粤人俗好相攻击。秦徙中原之民,使与百粤杂处。

晋,南土温湿,多有气毒……革奢务啬,南域改观。

南朝,民户不多,俚僚猥杂,卷握之资,富兼十世。

隋,土地下湿,多瘴疠,其人性并轻悍,权结箕踞,乃其旧风。

唐,闾阎朴雾,士女云流讴歌,有霸道之余甿,俗得华风之杂。

宋,海舶贸易,商贾交凑,尚淫祀,多瘴毒……民物岁滋,声教日洽。

元,今之交广,古之邹鲁。

本朝,衣冠礼乐,无异中州。声华日盛,民勤于食……

盖自汉末建安至于东晋永嘉之际,中国之人,避地者多入岭表,子孙往往家焉。其流风遗韵,衣冠习气,熏陶渐染,故习渐变,而俗庶几中州\footnote{黄佐:(嘉靖)《广东通志》卷20,转引自程美宝:《地域文化与国家认同:晚清以来“广东文化”观的形成》,页54—55}。

\end{quote}

此种历史叙述模式,将南粤史描述为一段在岭北移民“熏陶”下不断“开化”的故事。黄佐提出,正是在岭北移民“衣冠习气”的影响下,南粤才一步步变为“无异中州”的华夏世界。然而,我们只需回想一下16世纪之前的南粤史,便能明白这种说法有多不可靠。如前所述,直到15世纪陈白沙在世的时代,南粤依然被岭北帝国视为迥异于华夏文明的“化外”之地。南粤在16世纪的华夏化与其说是岭北移民的“恩赐”,不如说是霍韬、黄佐这类南粤精英自己选择的路径。南粤精英的自我华夏化必然伴随着历史记忆的重构,而发明祖先、发明历史就是重构历史记忆的重要手段。黄佐通过将自己的家族打造为岭北高官之后、将南粤打造为被岭北移民不断“开化”之地提出了一套完整的南粤史叙述模式,这一叙述模式无疑是陈白沙提出的小华夏路线的自然延伸。这种历史叙事模式若安置在由“霍韬模式”建构出来的宗族上,便形成了既有深厚历史传统、又有组织度的儒化共同体。在16世纪,一批批南粤土豪便通过这种方式建构宗族、发明祖先,开启了南粤自我华夏化的道路。珠三角大批从事沙田开发的宗族都自称乃珠玑巷移民之后,亦有部分人视自己为崔与之、李昴英等南粤儒者的后代\footnote{科大卫:《皇帝与祖宗:华南的国家与宗族》,页84}。对于如此众多的人都称自己为珠玑巷移民后代的现象,霍韬曾表示不屑一顾,视之为无稽之谈,并老实地承认自己的祖先是在明初靠卖鸭蛋起家的土著\footnote{罗一星:《明清佛山经济发展与社会变迁》,页96}。然而在他去世后,他的儿子霍与瑕居然建造了一块“始祖南雄珠玑巷以来寓墓”,视霍氏为来自三晋太原的岭北移民之后\footnote{任建敏:《从“理学名山”到“文翰名山”——16世纪西樵山历史变迁研究》,页170}。

在16世纪以前,帝国并不允许庶民祭祀始祖。据朱元璋规定,庶民不可建设祠堂,只能在家中祭三世祖,品官亦只可在家中设家庙祭四世祖。许多南粤百姓只能将自家祖先的灵位安放于佛寺,并在坟墓前祭拜祖先。自1520年代起,魏校的毁淫祠运动严重打击了佛教在南粤的势力,南粤精英又借在“大礼议”中获得的政治影响力大建祠堂、祭祀发明出来的北方始祖。对此既成事实,明廷所能做的唯有追认。1536年,明世宗批准礼部尚书夏言(江右贵溪人)准许“天下臣民”祭祀始祖的建议,南粤宗族化运动的政治障碍得以彻底解除\footnote{对于明世宗究竟有无批准夏言的建议,学界颇有争论。然而,学界都认同夏言的建议确实大大推动了南粤的宗族化运动。关于学界对此问题的争论,参见任建敏:《从“理学名山”到“文翰名山”——16世纪西樵山历史变迁研究》,页14—22}。

通过发明祖先、建设宗族,南粤精英利用来自帝国的话语体系给自己的乡土共同体披上了一层“合法”外衣,并能从此不受帝国阻碍地进行共同体建设。事实上,宗族往往不止是血缘共同体,几个完全没有血缘关系的家庭也可以通过发明共同的北方祖先、修建同一个大宗祠而团结为一个共同体。受霍韬在石头村建设宗族的影响,佛山镇内数量众多、势力分散、缺乏血缘联系的霍姓亦团结起来,发明了一个名叫“正一郎”的祖先,将其事迹写进他们共同编纂的族谱。他们称,正一郎是14世纪时由南雄珠玑巷迁居到佛山的移民,此人子孙众多,繁衍出了佛山镇内的大批霍姓人口。到17世纪,佛山霍氏祠堂已经建成,佛山镇上的霍姓们在同一场所祭祖,真的自视为血脉相连的一家人了\footnote{科大卫:《皇帝与祖宗:华南的国家与宗族》,页156—157}。在开发沙田的活动中,如此庞大的宗族共同体必然是颇具竞争力的。

从16世纪前期到17世纪后期,经过约一百五十年的时间,宗族构建的浪潮从珠三角逐步扩散到整个广东。与西方世界的贸易和沙田开发给南粤土豪带来了大量财富,使他们得以放手建设宗族。17世纪末,南粤著名学者屈大均完成了他的名著《广东新语》。在书中,他这样描述他所看到的南粤社会:

\begin{quote}

岭南之著姓右族于广州为盛,广之世于乡为盛。其土沃而人繁,或一乡一姓、或一乡二三姓,自唐宋以来,蝉连而居,安其土,乐其谣俗,鲜有迁徙他邦者。其大小宗祖祢皆有祠,代为堂构,以壮丽相高。每千人之族,祠数十所;小姓单家,族人不满百者,亦有祠数所。其曰大宗祠者,始祖之庙也。庶人而有始祖之庙,追远也,收族也,追孝也。收族,仁也。匪谮也,匪谄也\footnote{屈大均:《广东新语》卷17}。

今无女巫,惟阳春有之,然亦自为女巫,不为人作女巫也。盖妇女病,辄跳神,愈则以身为赛,垂髾盛色,缠结非常,头戴鸟毛之冠,缀以璎珞,一舞一歌,回环宛转,观者无不称艳。盖自以身为媚,乃为敬神之至云\footnote{屈大均:《广东新语》卷6}。

\end{quote}

可见,17世纪后期的广东已是一个异常儒化、华夏化的世界。安土重迁的粤人自视华夏显贵后裔,修建大量祠堂以祭拜虚构的岭北祖先。百越传统的巫术在广东已近乎绝迹,仅在冯冼时代曾为南粤政治中心的阳春有所残留。慎终追远的粤人以无数祠堂为社会凝结核,形成了一个个星罗棋布的宗族共同体。这些共同体不但包含做为祭祀中心的祠堂,还有用以赈济贫困族人和乡民的社仓、用以自卫的堡垒,并能在需要时迅速召集族中壮丁出战。宋儒在《朱子家礼》中所构想的那个理想社会在南粤变成了现实。这种社会的组织度虽远远不如诸夏封建时代,但无疑是华夏世界彻底原子化之前的最后一道障碍,其组织度远远高于中原散沙。从此开始,南粤变成了小华夏,乃是最纯正的华夏世界。粤人不但拥有延续数万年的百越传统、是百越文明中的伟大一员,同时还是华夏文明最为正统的嫡传。在粤人眼中,当岭北的华夏文明已在蛮族和僭主的蹂躏下濒临灭绝时,是百越出身的南粤接续了华夏文明,成为诸夏之选锋、华夏世界之波兰,南粤完全有足够的理由蔑视岭北的软骨头与降虏们。这一观念虽然是被发明出来的,但又最为符合事实,因而一直延续至今,成为粤人自外于岭北的重要依据。就这样,与此前历史上的粤人有着从未间断的延续性,同时又肩负着崭新传统、有着崭新面貌的近现代南粤民族诞生了,南粤伟大的民族发明进程开始了。

然而,在粤东的韩江流域、粤西北的深山、广西与海南,宗族化的过程并非如此一帆风顺。在西方世界接触南粤、明帝国侵占南粤的大背景下,这些地区的历史进程变得愈加复杂,在16—17世纪间演出了无数血泪与光荣并存的历史活剧。







