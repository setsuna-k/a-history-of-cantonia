\chapter{西方人到达南粤:葡萄牙、西班牙、荷兰、英国}

\section{葡萄牙人来粤}

\indent 1513年,葡萄牙商人乔治·欧维士受到马六甲总督阿尔布科尔科的派遣,做为首个造访南粤的西方人在珠江口东南侧的屯门岛登陆,波澜壮阔的南粤近代史由此拉开帷幕。至此,葡萄牙人终于完成了他们长期以来的夙愿:找到《马可·波罗游记》中记载的那个神秘的东方帝国。南粤的富庶令欧维士极为兴奋。他在当年返回马六甲后,葡萄牙人对南粤的热情被完全激发起来。1515年,另一位葡商拉斐尔·佩雷斯特雷洛(Fafael Perestrello)到达屯门岛,于次年秋返回马六甲,赚取了二十倍的利润。拉斐尔对葡方如是描述他所见到的粤人:

\begin{quote}

“中国人”希望与葡萄牙人和平友好,他们是一个非常善良的民族\footnote{《广东通史》古代下册,页285}。

\end{quote}

由拉斐尔的记载可知,我南粤人一向怀有开放的胸襟,对于侵害我们自由的岭北帝国坚决抗争、对前来贸易的西方友人则竭诚欢迎。由于对南粤印象良好,一支由8艘船只组成的葡萄牙贸易船队在舰队司令费尔南·佩雷斯·德安拉德拉(Fernao Peres de Andrade)的指挥下于1517年6月17日由马六甲起航,驶往南粤,由里斯本赶来的葡萄牙国王曼努埃尔一世之特使托·皮雷斯(Tome Peres)亦随船同行。8月15日,葡萄牙船队抵达屯门岛,费尔南前往南头城(今深圳南头古城)与明帝国水师接洽,自称“佛郎机人”,要求“朝见中国国王”\footnote{《剑桥中国明代史》,页311}。9月底,船队到达广州怀远驿(今荔湾区第十八甫路)江面,抛锚停泊,鸣放礼炮、在桅杆上升起旗帜。史载,当时葡船鸣炮如雷,一度引起广州市民恐慌\footnote{《广东通史》古代下册,页285}。这是粤人首次领略近代西方文明的强大。

闻知葡萄牙人欲“朝见”明帝,明两广总督陈金遂一面将葡使一行安置于光孝寺学习礼仪,一面将此事报知朝廷。次年正月,明廷对此事做出如下的处理结果:

\begin{quote}

广东镇巡等官以海南诸番无所谓佛郎机者,况使者无本国文书,未可信,乃留其使者以请。下礼部议处,得旨:“令谕还国,其方物给予之。\footnote{《明武宗实录》卷158,正德十三年正月壬寅条}”

\end{quote}

早在葡萄牙人来粤前四五百年的第五次北属时期,我南粤的海商便曾到达西西里、西班牙。而直到16世纪,明帝国君臣仍然昧于世界大势,荒唐可笑地以“海南诸番无所谓佛郎机者”为理由拒绝与葡人交往。南粤河山被如此愚昧的岭北帝国侵占,可谓悲惨至极。当时,明帝国的海禁较洪武时期已有所松动。朱元璋死后,其子朱棣于1403年复设广东、福建、浙江市舶司,准许安南(越南)、暹罗、爪哇、琉球、日本、占城等国以朝贡名义至粤、闽、吴贸易,是为“贡舶贸易”。“贡舶贸易”有严格的时间限定,各国商船只能在明帝国规定的“贡期”内航至东南三越港口,贸易量颇受限制。至于沿海之民出海进行的“商舶贸易”则一直被明帝国视为非法,不断遭到取缔。虽然如此,向往自由的粤人、闽人仍纷纷冒着生命危险冲破海禁,前往南洋、日本,并曾在14世纪末于苏门答腊建立延续约半个世纪的旧港政权。许多外国海商亦不甘忍受明帝国在“贡舶”体制下对他们的疯狂盘剥,绕开市舶司直接与粤闽沿海百姓交易。1514年,即葡人首次来粤后第二年,明帝国当局竟诬之为“奸民勾结外夷”,下令广东抚按官“禁约番船,非贡期而至者即阻回”。此令一出,广州的海外贸易一时陷入停顿。明帝国驻粤官员惊慌地发现,他们突然无法抽取到足够多的商税了,而这些商税本是用来充作明帝国驻粤侵略军的军费的。因此,明广东当局不得不于1517年恢复旧例,允许外国商船“不拘年分”来粤贸易,广州港迅速恢复了繁荣\footnote{《广东通史》古代下册,页266}。费尔南指挥的葡萄牙舰队在同年到达广州江面时,看到的正是一副热闹景象。然而如前所述,愚昧的明帝国君臣在1518年仍然拒绝了葡萄牙人的来访,葡人遂决定采取强硬态度。当年6月,费尔南辞职归国,其弟西蒙(Simao)代之。西蒙于1519年8月率由印度科钦出发的船只抵达屯门岛,随即在岛上建起一座设有火炮的石木结构要塞,并派兵劫掠、勒索过往船只\footnote{《广东通史》古代下册,页287}。西蒙的威吓姿态终于收到效果。1520年,葡萄牙使团获准北上。在葡使皮雷斯和翻译火者亚三的率领下,他们于当年5月抵达南京\footnote{对于火者亚三的真实身份,学界历来聚讼不已。有学者视之为南洋穆斯林粤侨,近年亦有人认为此人实为吴越苏州商人傅永祥,理由是史书中记载的火者亚三与傅永祥的人生经历重合度非常高。参见金国平、吴志良:《“火者亚三”生平考略——传说与事实》,《明史研究论丛》(第十辑)。然而,以火者亚三为傅永祥的说法无法解释一条关键记载,即明人称火者亚三“高鼻深目如回回状”。金、吴的研究提到了这则记载,却没有对其进行分析。“火者”系穆斯林对一类官员的称呼(见张维华:《明史欧洲四国传注释》,页9),“亚三”当为阿拉伯名“哈桑”之转译。笔者认为,火者亚三系穆斯林的可能性更大,唯不知其具体国籍。}。当时,明武宗正在南京南巡,遂得以召见葡使,并对穆斯林出身的火者亚三颇有好感,向其学习了葡萄牙语或马来语,视之为宠臣\footnote{对于明武宗具体学习了哪种语言的考证,参见金国平、吴志良:《“火者亚三”生平考略——传说与事实》}。不久后,使团被命令前往北京等待皇帝还京。在北京,明帝国官僚摆出一副“天朝上国”的架子,对葡萄牙使团十分无礼。他们不但强令葡萄牙人在每个月阴历的初一、十五两天匍匐于紫禁城城墙下\footnote{《剑桥中国明代史》,页313},还曾因火者亚三拒绝向礼部官员跪拜而对其施以杖刑\footnote{金国平、吴志良:《“火者亚三”生平考略——传说与事实》}。1521年初,明武宗回到北京,于4月19日病死。继任的明世宗立即对使团痛下杀手,将火者亚三处斩,并把葡使赶到广州、投入监狱。第二年,皮雷斯惨死于狱\footnote{郭福祥、左远波:《中国皇帝与洋人》,页88}。

关押、杀害葡使的同时,明帝国又强令屯门岛上的葡人离境。葡人拒不执行,理由是他们正在与粤人贸易。明军立即发起进攻,但被击退,葡船亦被击沉一艘。6月,明军发起第二次攻势,又被打退。9月3日,明海道副使汪鋐决定与葡人决战,遂集结50艘战船、4000余兵力,对屯门岛上仅有6艘船葡萄牙人展开总攻。明军舰队以绝对优势兵力对葡船形成半圆形包围圈,点燃满载干柴的旧船发动火攻,并以善于泅水者潜水凿击葡船。经数日激战,葡人寡不敌众,仅余3艘大船,于9月7日夜冒大雨强行突围而出,至10月底回到马六甲,是为东西交通史上著名的“屯门海战”\footnote{《广东通史》古代下册,页288}。

在屯门海战的消息传到里斯本之前,曼努埃尔一世已派出由马尔廷·科廷奥(Martim Continho)率领的第二个使团搭乘4艘船出使明帝国。1522年7月,使团船队到达珠江口西侧的香山县西草湾,未表现出任何敌意,却遭到明帝国水师的突袭。在汪鋐的指使下,明备倭指挥柯荣、百户王应恩率96艘战船尾随而至,意图置葡使于死地。葡人以外交任务为重,拒不还击,直至明军以火船进攻方才还手。经激战,葡船退至潲州,折舰二艘,余者突围而去,于10月中旬回到马六甲,是为“西草湾海战”。此役,明军伤亡惨重,百户王应恩阵亡。葡方战死35人、被俘42人\footnote{《广东通史》古代下册,页289}。被俘葡人被带上枷锁,遭到明帝国残酷虐待,并于1523年秋全被公开杀死\footnote{《剑桥中国明代史》下卷,页314},首级亦遭示众\footnote{徐日久:《五边典则》卷24}。汪鋐将缴获的葡萄牙火炮献给明廷,称之为“佛郎机”,由此引发了东亚仿制西式火炮“佛郎机”的浪潮,推动了东亚的军事技术革新\footnote{张廷玉:《明史》卷325《外国六佛郎机》}。

对于屯门、西草湾海战,明帝国自然大事吹嘘,事之为大捷。近现代的大一统史观亦视之为“中国军队抗击西方殖民主义侵略势力的一次重大胜利”,对之大加赞扬。但若站在南粤的立场上,我们便能得到完全不同的结论。葡人来粤,无疑将南粤拉入了1500年后逐步形成的全球体系。对于来粤贸易的葡人,我们的祖先表现出了十足的友善态度,与我们展开和平贸易。然而,愚昧自大的明帝国却采用扣押和袭击使节、虐杀战俘等种种卑鄙手段对待来粤葡人。在葡人被迫还击的行动中,我南粤百姓惨遭伤害。明帝国的倒行逆施不但生生斩断了粤葡间的正常交往,还使两者的真正深入接触推迟了三十余年。屯门海战后,明帝国严禁外国人来粤贸易,使葡人、东南亚人的对明贸易重心转移到闽越的漳州浯屿、月港和吴越的双屿。直到1542年,广州城门上还张贴着由金字写成的明世宗“圣谕”:“虬髯大眼之人不准入其疆域。”南粤千万年来发达的海洋贸易遭遇了严重挫折。然而,我们向往自由、向往海洋的祖先注定不会让ni 一局面长久维持。很快,葡萄牙人便将重返南粤,掀起更大的波澜。

\section{圣方济·沙勿略与澳门城邦的建立}

\indent 1548年,明浙江巡抚、大吴奸朱纨攻陷双屿岛,血腥屠杀了岛上包括八百名葡萄牙人在内的一万二千名各国百姓。双屿本是葡人精心经营的吴越港口。在被摧毁前,当地有葡萄牙城防司令、王室大法官、市政议员,是个标准的葡式城邦,乃日本—东亚大陆—马六甲三角贸易的枢纽,每年往来该地的各国商船多达上千艘。次年,朱纨又在闽越浯屿、走马溪杀害数百名葡、吴、闽海商,并酷刑处死96名俘虏\footnote{执经生:《痛心!被明帝国摧毁的吴越澳门!》(注:执经生前辈的文章未能找到,但此事在《明史·列传第九十三》中以维护明帝国的角度有简略记载。—— Setsuna K)}。至此,葡人因在吴、闽的贸易全部碰壁,遂转回南粤。

在1540年代,虽然明帝国官方严禁粤葡人贸易,但根本无法阻止我们祖先冲破牢笼的正义行动。当时,南粤沿海走私成风,明帝国官僚根本无力禁止。只要葡人不靠近广州,他们便对粤人和葡人的贸易采取睁一只眼闭一只眼的态度。时人如是记载其时的沿海情形:

\begin{quote}

(粤人)更番往来私舶,杂诸夷中为交易。首领人皆高鼻白皙,广人能辨识之……甚至官军商纪,亦与交通\footnote{黄佐:《广东通史》卷66}。

\end{quote}

此段记载中提到的游鱼洲即今日广州海珠区琶洲,距广州城仅五里。由此可见,当时广州近郊的走私贸易异常活跃。广州城外的乡民不但与葡人交易各种被明帝国称为“违禁物”的世界各地特产,一些违法分子甚至还拐卖儿童、与葡人进行人口交易\footnote{《广东通史》古代下册,页291}。人口贩卖是丑恶的犯罪行为,当予否定。然而,此种在贸易中必然产生的现象之罪恶程度较之岭北帝国对粤人残酷至极的屠戮与镇压,无疑远远不如。而粤葡间正常的“走私”贸易带给南粤百姓的巨大利益,则是显而易见的,无数粤人因之获得工作机会与财富,并加深了对世界和西方文明的认识。

除游鱼洲外,上川岛亦是粤葡贸易的重要地点。上川岛位于广海卫(今台山广海镇)以南洋面,被葡人称为“圣约翰岛”(St John’s Island)。当地粤商云集,“以各自的投资交易获利”,葡人亦不时航至岛上,搭建茅屋、开设商场。随着葡人的到来,上川岛亦受到天主教耶稣会的关注。1552年,西班牙耶稣会士圣方济·沙勿略(Francis Xavier, St.)在上川岛登陆,一位伟大的传教士从此与南粤结下不解之缘。

1506年,沙勿略出生于西班牙纳瓦拉。他于1534年听从耶稣会创始人依那爵·罗耀拉(Ignacio de Loyola)的布道,皈依天主。1540年,沙勿略受教宗指派前往印度传播福音。此后,他于1549年登陆日本,获西本州大名大内义隆接见,得以在当地自由传教。然而,因为日本文化深受“中国”影响,有日本人奇怪为何“中国人”不皈依基督教,并拒绝信教。受此刺激,沙勿略决定将福音传播到“中华帝国”去,从此打开东亚皈依主的大门。1551年,他在回航印度的途中遭遇台风,一度漂至上川岛。回到印度后,他收到伊那爵的信件,被任命为耶稣会“印度区及周边”的省会长。至此,沙勿略看清了自己的使命:他必须全力打开“中华帝国”的大门,哪怕前面有再多的险阻也万死不辞\footnote{格拉茨:《现代天主教百科全书》,页726}。

1552年10月26日,沙勿略在上川岛登陆,他计划与葡萄牙商人一同进入南粤。然而,葡萄牙人不喜欢他这个西班牙人,对其入粤传教之事百般阻挠,沙勿略只得伺机偷渡。当时,沙勿略栖居于一间四面透风的茅屋,身边只有一个教名为安东尼的粤人基督徒随侍左右。艰苦的行程、恶劣的环境使他染上疟疾,很快就病倒了。重病之中,沙勿略仍未放弃将福音带给南粤的信念。躺在草床上的他一面让安东尼给他放血以减轻痛苦,一面不停祷告。在一封书于11月13日的信中,沙勿略对友人说:

\begin{quote}

你一定当知者有一事,且万勿疑虑,即魔鬼不容冠有耶稣圣名之修会——耶稣会进入“中国”,此事是一定的。我在上川岛上,书此仍令你知悉。对于此事,你万不要疑虑。因魔鬼所兴起以往之阻挡,及现在每日所发生的难处,用我笔墨,我终不能详言也。还有一事,我们该当确定知之者,即因天主之帮助、圣宠、恩佑,对于此事魔鬼必将失败,天主为自己之愈大光荣,用一卑微之工具如余者,要压服魔鬼之骄傲也\footnote{周天:《跋涉:明清之际耶稣会的在华传教》,页16}。

\end{quote}

沙勿略宁可牺牲一切也要将福音传播到南粤的精神,足以令粤人为之动容落泪。同年12月2日,在坚定的信仰中,沙勿略迎接了天国的召唤,于上川岛荣归主怀,时年五十五岁。他的遗体于次年被运抵印度果安放,至今仍奇迹般地没有朽坏。他的去世地点竖立着墓碑,以供此后的传教士和南粤天主教徒凭吊。教宗则将其封圣,称“圣方济·沙勿略”。耶稣会士与南粤的第一次接触,就这样悲壮地落下了帷幕\footnote{周天:《跋涉:明清之际耶稣会的在华传教》,页16}。

沙勿略去世后不久,上川岛上的粤葡贸易点也遭到毁灭。1554年,明帝国强制封锁上川岛,禁止粤葡互市,葡人被迫转往浪白澳(古岛名,在今珠海三灶岛之西北)。1549年,三十名由闽越逃出的葡人在浪白澳登陆,于此建立定居点,将之发展为葡萄牙商船往返马六甲与日本的中转站\footnote{《广东通史》古代下册,页293}。到1555—1556年间,浪白澳的葡萄牙居民已有300—400人。这些远涉重洋而来的冒险者居住在粗糙的茅屋中,性情粗犷,经常彼此私斗。在岛上耶稣会士的调节下,这些人才勉强能做到团结一致、不再自相残杀\footnote{《剑桥中国明代史》,页319}。浪白澳并非良港,乃风涛险恶、水土贫瘠之地\footnote{《广东通史》古代下册,页294}。在此情况下,葡人急需在南粤沿海获得一个土地肥沃的优良港口,建立一处规范的贸易城邦。粤葡两个伟大文明的结晶澳门城邦,由此诞生。

澳门位于香山县南端。对其地势,史籍云:

\begin{quote}

澳门一岛状如莲花,香山近处有路名关闸砂。直出抵澳,若莲茎焉\footnote{《广东通史》古代下册,页294}。

\end{quote}

正如此段材料所言,澳门正如香山县伸向南海中的一朵莲花。澳门本来不与大陆连通的岛屿。在16世纪,因沙田开发和珠江口淤泥堆积,澳门岛与香山县连在一起,变为澳门半岛。在半岛以南有四座呈棋盘状排列的高山突出海面,海水呈十字型纵贯其间,被称为“十字门”\footnote{《广东通史》古代下册,页295}。如此优良的地势构成了天然良港,此地因而向来为越南、暹罗对粤贸易的必经之处\footnote{黄鸿钊:《澳门史》,页102}。在葡人口中,澳门被称为“Macau”。对于这一称呼的由来,人们众说纷纭。有人认为,此处有一海神妈祖庙,葡人因而称之“妈港”,即Macau。又有人指出,Macau实为粤语“舶口”之转译\footnote{《广东通史》古代下册,页295}。此外,还有一个戏谑的说法:据说,当葡人最初在澳门登陆时,他们曾用葡语询问当地居民此为何地。当地居民不懂葡语,遂用粤语回答“乜鸠?”葡人以为这就是此地的地名,乃称之Macau\footnote{谭元亨:《广州十三行——明清300年艰难曲折的外贸之路》,页4}。

无论真相如何,早在1535年便曾有葡人到达澳门进行贸易。然而,由于当时明帝国对葡人严格防范,这批最早的登陆者很快就离去了\footnote{黄鸿钊:《澳门史》,页104}。1553年,葡人以500两白银贿赂明海道副使汪柏,借口货物被海水打湿、欲上岸晾晒货物,获得在澳门的暂居权\footnote{《广东通史》古代下册,页295}。葡人的到来使澳门很快成为繁华之地。到1556年,澳门葡人更获准进入广州城贸易。据著名的耶稣会士利玛窦记载,当时葡人赴广州进行贸易的情形是:

\begin{quote}

葡萄牙人已经奠定了一年两次市集的习惯,一次是在一月,展销从印度来船只所携带的货物;一次是在六月末,销售从日本运来的商品。这些市集不再像从前那样在澳门港或岛上举行,而是在省城本身之内举行……这些公开市场的时间一般规定为两个月,便长长加以延长\footnote{转引自谭元亨:《广州十三行——明清300年艰难曲折的外贸之路》,页6}。

\end{quote}

明帝国虽然允许澳门葡人在广州贸易,但仍蛮横地不准粤葡之间进行自由交易。在1556年,明帝国任命广州城内的十三家商行做为对葡贸易的垄断者。这些商行包括广州本地人商馆五家、闽越泉州人商馆五家、吴越徽州人商馆三家,统称“十三行”\footnote{谭元亨:《广州十三行——明清300年艰难曲折的外贸之路》,页7}。在此后三百年间,十三行将成为沟通南粤与西方世界的重要窗口,创造惊人的财富。

广州城内的对葡贸易使我们的祖先获得了极大的益处。1557年,在广州商人的争取下,明帝国同意正式将澳门租给葡萄牙人,每年向其收取500两租金,澳门城邦的历史由是正式揭幕\footnote{黄鸿钊:《澳门史》,页108—109}。

在澳门城邦建立之初,大批外国人蜂拥而来。短短八年内,澳门居民已膨胀到近万人之多,其中包括900名葡萄牙人及数千名马来亚人、印度人、非洲人\footnote{黄鸿钊:《澳门史》,页120}。澳门建城系葡国商人在粤人帮助下努力达成的结果,与葡萄牙政府无涉,因此城邦事务全由本地葡人自治。1560年,澳门葡人自行选举产生了一个由驻地首领、法官和四名商人组成的委员会,是为日后澳门议事会的雏形。二十三年后,澳门葡人举行正式选举,组建议事会。次年,葡属印度总督承认澳门议事会为自治机构,允许议事会全权管理当地政治、经济、司法事务,城邦特殊重大事务则召集市民大会讨论并表决。议事会一般由3名议员、2名法官、1名理事官组成,3年一任,可连任一次,担任者需年满40岁。每次议事会选举皆于圣诞节前后在议事亭举行,所有居澳葡人都有选举权。此外,葡国则向澳门派遣总督、大法官、主教,此三者被视为议事会的当然议员\footnote{黄鸿钊:《澳门史》,页121—122}。在议事会的策划下,居澳葡人组成自卫组织“保安队”,并于1557—1626年间建成6座炮台,将澳门变为一座坚固的要塞。澳门总督麾下亦有100—480名葡兵,乃保卫澳门城邦的正规武装力量\footnote{黄鸿钊:《澳门史》,页124}。

澳门还有许多粤人居民,他们亦是城邦的重要组成部分。在愚昧颟顸的岭北帝国眼中,这些人是“勾结外夷”的“奸民”、“匪贼”。然而对南粤来说,他们无疑是最早一批与西方世界深入接触的伟大先驱。城邦建立后,大批粤人以及部分闽人迁居澳门。在葡萄牙人往返于广州、澳门之间时,是他们充当“引水人”,于珠江口的风涛中驾船引路;在澳门葡人和广州十三行及明帝国官僚交涉时,是他们努力学习葡语,充当“通事”(翻译);在建设城邦的过程中,是学会了西洋建筑技术的他们建起了一座座精美的西式建筑,并将南粤的建筑风格糅合其中,动工于1602—1603年、完工于1637年的著名建筑澳门圣保禄大教堂就是他们与耶稣会士携手完成的杰作。这座美轮美奂的教堂的样式为:

\begin{quote}

在它的建筑中也有文化融汇的因素……这种欧亚混合物从造型与象征性方面均得到了证实:怪兽的身躯是一条“中国”式的龙;滴水沿嘴是“中国”的雄狮,象征着力量和勇气……表现人物的大型石雕显露了东方人或实施工程者的长相模式,在解释主要代表性结构的象征意义上采用了东方的植物\footnote{程美宝:《澳门作为飞地的“危”与“机”——16—19世纪华洋交往中的小人物》,《河南大学学报》(社会科学版)2012年第5期}。

\end{quote}

圣保禄教堂被南粤人以粤语谐音亲切地称为“大三巴”。不幸的是,大三巴于1835年的一场大火中烧毁,今天仅存一面前壁。这面壮观的前壁被称为“大三巴牌坊”,乃今日澳门最重要的地标之一。通过凭吊大三巴牌坊,我们足以体认到澳门城邦的伟大:它是南粤与西方文明的壮阔结晶,没有葡萄牙人带来的西方文明,澳门城邦不会诞生;没有我们伟大祖先对西方文明与自由的向往,澳门城邦亦不可能繁荣昌盛。在澳门城邦建立之后,澳门历史便不再属于南粤史的一部分,而是成为了一座葡萄牙城邦。然而,这座城邦与南粤有着深深的联系,它永远是与南粤血肉相连的亲邦。

大三巴展现了天主教会在澳门的巨大影响力。1576年,教宗下令建立澳门主教区,管辖“中国”、日本、越南的传教事业,澳门一跃而为葡人在远东传教事业的中心。葡萄牙教士们使许多居住在澳门的粤人皈依天主,这些南粤基督徒中的一部分人继承了沙勿略的未竟事业,回到被明帝国侵占的家乡传播福音\footnote{黄鸿钊:《澳门史》,页124}。1583年,两名伟大的耶稣会士从澳门出发,深入南粤内地传教,他们即将在南粤掀起一场认识世界的思想风暴。他们二人,便是来自意大利的罗明坚(Michel Ruggieri)与利玛窦(Matteo Ricci)。

\section{南粤认识世界:罗明坚、利玛窦、黄明沙、钟鸣仁}


\indent 我南粤历史悠久,航海便利,一直是接收世界中心(环地中海文明圈)由海路向东亚输出文明的桥头堡。早在第四次北属时期,唐帝国治下的南粤就已有来自西亚、东欧的景教(基督教聂斯托里派)徒。他们居住于广州的“蕃坊”,在879年的黄巢屠城中损失殆尽。至第五次北属时期,又有来自欧洲的基督徒来到广州定居。他们被元帝国视为色目人,以蒙古语呼之为“也里可温”。14世纪后期,洪武社会肆虐于南粤,也里可温被禁止自相婚娶,其宗教信仰亦被明帝国视为非法,这一群体遂泯于粤人之中,归于消亡\footnote{《中华文化通志84 第九典宗教与民俗 基督教犹太教志》,页50—51}。

广州的景教徒与也里可温多为欧洲、西亚人,粤人很少。直至16世纪,基督教都未曾深入南粤社会。1579年,意大利籍耶稣会士罗明坚来到澳门,基督教扎根南粤的历史进程由是开始。

1543年,罗明坚出生于意大利中南部的斯品纳佐拉城。早年他曾为法律博士,后于1572年在罗马加入耶稣会。1578年,完成了哲学、神学课程的他在里斯本领受神父之职,乘“圣路易”号船前往耶稣会在远东的总部印度果阿。在他的随行人员中,有年轻教士巴范济(Francois Pasio,意大利人)和利马窦\footnote{周天:《跋涉:明清之际耶稣会的在华传教》,页18—19}。

在果阿,罗明坚被指派前往南粤传教,进而敲开明帝国对基督教封闭的大门,巴范济和利玛窦则暂时留在果阿待命。1579年7月,罗明坚乘船抵达澳门。此后两年间,他伪装为葡萄牙商人,多次进入广州城调查情况。在罗明坚眼中,广州人口众多、物产丰富、贸易发达,往来如梭的河船与熙来攘往的行人构成了繁华的城市图景。此外,他还注意到明政府在南粤的横行霸道:

\begin{quote}

皇帝一人的收入就超过欧洲所有国王和领主——甚至还要加上非洲君主的岁入……官员的权利很大,法规都掌握在他们手中,百姓动辄就被申斥惩治\footnote{周天:《跋涉:明清之际耶稣会的在华传教》,页19}。

\end{quote}

对于驻粤明帝国官僚的颟顸腐朽,罗明坚亦有详细描述:

\begin{quote}

政府似乎既蔑视外国人又害怕外国人,尽可能地远离外国事务,对外国人的防范措施严格而有效,不经批准出国或擅自带进外国人的“中国人”是要处以死刑的,对外国事务不想听、不想理解,与之有关的任何东西都无法钻进他们的耳朵。贿赂是公开的秘密,任何事情都有桌面上和私底下两种解决办法,后者的基础就是送钱或送礼\footnote{周天:《跋涉:明清之际耶稣会的在华传教》,页20}。

\end{quote}

在明帝国愚昧至极的闭关政策下,南粤面向世界的大门被迫紧闭。我南粤不但要承受明帝国君臣敲骨吸髓的盘剥,还要被这群岭北寄生虫强行拉进愚昧的深渊,着实令人愤怒至极。1582年,巴范济、利玛窦奉命由果阿来到澳门协助罗明坚。直到这时,耶稣会士仍无法进入南粤\footnote{周天:《跋涉:明清之际耶稣会的在华传教》,页21}。

同年末,事情终于迎来转机。深谙明帝国官场现状的罗明坚以西式服装、三棱镜、报时钟贿赂了贪得无厌的明两广总督陈瑞,总算获得入粤许可。1583年2月,罗明坚与巴范济抵达广州,遵从当地官员的劝告剃光头发,以佛僧装扮出现于公共场合。不幸的是,陈瑞突然在这时被明廷免职,其对耶稣会士的邀请亦化为泡影,二人只得失魂落魄地返回澳门。很快,巴范济便被派往日本传教。临行前,他这样提醒罗明坚:“归根到底,在‘中国’送礼花钱就好办事。\footnote{周天:《跋涉:明清之际耶稣会的在华传教》,页22—23}”

罗明坚对巴范济的提醒心领神会,并着手实施进一步贿赂。数个月后,金银财宝果然再次打动贪婪的明帝国官僚,继陈瑞担任的两广总督郭应聘致函澳门,允许耶稣会士在广州以西九十公里处的肇庆居留。1583年9月10日,以僧装打扮的罗明坚与利玛窦抵达肇庆,栖居于一栋小屋。很快,他们便在西江畔的九层浮屠崇禧塔下建起了一座教堂。在教堂开工时,肇庆城轰动了。我们对未知世界充满好奇心的祖先蜂拥而至,将工地围了个水泄不通。利玛窦记录当时的情形称:

\begin{quote}
总施工开始时,各阶层大群好奇的人都被吸引来了,有的甚至来自远方,显然被有关外国教士的故事所打动……神父们就对他们极尽殷勤,行事时力求赢得他们的好感和友谊。他们称为无价宝石的玻璃三棱镜,凡是要看的都让看,还有书籍、圣母像和其他的欧洲产品,都由于新奇而被认为是漂亮非凡\footnote{金尼阁:《利玛窦中国札记》,页164—165}。

\end{quote}

次年,教堂完工,定名为“仙花寺”。首个在仙花寺受洗的粤人是个肇庆城中的穷人。此人身患不治之症,被家人抛弃等死,罗明坚和利玛窦便将他带回教堂,一边照料他,一边告诉他基督教的基本教理。此人很快便做好准备,接受洗礼,并在数日后安然逝去。除传教外,罗明坚还用汉字写了一部语言通俗的教理著作《天主圣教实录》,以此吸引更多人入教\footnote{金尼阁:《利玛窦中国札记》,页170—172}。两位传教士的真诚态度打动了肇庆居民,人们对传教士的态度由陌生和敌视转为尊敬。在路过和谈到教堂时,人们充满了敬畏之情\footnote{周天:《跋涉:明清之际耶稣会的在华传教》,页25}。每日来到教堂的百姓和官员越来越多,他们或向罗、利二人提出种种刁钻古怪的问题、或参观精美的西洋器物。持续不断的接待活动令二位传教士的经费难以为继。经商量后,罗明坚决定回澳门筹款,肇庆的传教事业则由利玛窦独自支撑\footnote{周天:《跋涉:明清之际耶稣会的在华传教》,页26}。虽然访问教堂的人很多,但人们往往是为了满足好奇心,真正有兴趣皈依天主的并不多。利玛窦为此绞尽脑汁,终于想出了一个办法,远东最早的一幅全球地图随之诞生。

1552年,利玛窦出生于意大利中部的马切拉塔城。他的父亲是一名商人,在教皇领地有产业。利玛窦从小就与基督教结下不解之缘。他自幼接受神学教育,并修习了语法、经院哲学、文学、数学等学问。十六岁时,他前往罗马求学,并在那里加入耶稣会。到1577年,二十五岁的利玛窦已是个学识渊博的虔诚基督徒。是年,他中断学业加入东方传教团,与罗明坚、巴范济一同离开欧洲前往印度果阿。在果阿,他反对当地耶稣会士禁止印度人担任教士的做法,认为神父不应轻视、敌视当地居民。这种对各地居民一视同仁的态度,亦在利玛窦来粤后显现出来。面对肇庆人的敌视,他和罗明坚以真诚打动了他们。当粤人未对基督教教义表现出太大兴趣时,他决定首先向粤人传授西方的科学知识以赢得粤人的好感。当时,在教堂接待世上挂着一幅用拉丁文标注的世界地图。这一地图上包含着16世纪时欧洲人所了解的全部世界,包括地球上除澳洲、南极洲外的所有陆地。来访者对这一地图表示极为震惊,他们不但惊讶地看出原来华夏并不在世界中心,而且发现东亚传统的“天圆地方”观无法解释地图所展示的球形世界。人们对地图的热情使利玛窦意识到,这是向粤人展示世界的真实图景、赢取粤人尊重的好机会。他重制了一份以汉字标注、带有经纬线的世界地图,并将明帝国的位置调到地图中部以提高人们对地图的接受度,还用铜、铁制造了天球仪和地球仪。这一地图,被利玛窦命名为《坤舆万国全图》\footnote{金尼阁:《利玛窦中国札记》,页180—181}。据利玛窦记载,当肇庆百姓看到这幅世界地图时,他们的反映是:

\begin{quote}
当他们头一次看见我们的世界地图时,一些无学识的人讥笑它,拿它开心,但更有教育的人却不一样,特别是当他们研究了相应于南北回归线的纬线、子午线和赤道的位置时。再者,他们得知五大地区的对称,读到很多不同民族的风俗,看到许多地名和他们古代作家所取的名字完全一致,这时候他们承认那张地图确实表示世界的大小和形状。从此之后,他们对欧洲的教育制度有了更高的评价\footnote{金尼阁:《利玛窦中国札记》,页181}。

\end{quote}

《坤舆万国全图》获得了巨大成功。我们的祖先第一次知道了地球的真实形状,在地图上看到了浩瀚的太平洋、大西洋与辽阔的美洲新大陆。此后,他们将扬帆起航,前往他们曾在地图上看到的地方,将南粤的种子撒向五洲七海,在波澜壮阔的大航海中创造文明奇迹,使南粤成为欧洲、美洲各国的海上邻邦。而使我们祖先最初认识到世界的大小、样貌的人,正是远涉重洋来到南粤的利玛窦。对南粤来说,利玛窦不愧是一个有巨大贡献的文明输入者,今天的南粤人绝不应忘记他传播文明的伟大贡献。

此后数年间,利玛窦全力工作。到1589年,南粤的受洗者已达数十人。正当利玛窦踌躇满志,准备大展身手时,一场意外的打击突然降临。是年,贪婪的新任明两广总督刘继文看上了仙花寺所占地段,欲将其据为己有。他卑鄙地诬陷利玛窦是澳门葡人派来的间谍,胡说利玛窦“假奇巧之淫技,博黎民之欢心”,执意要将其赶走。经艰难交涉,刘继文好不容易才同意在粤北韶州拨一块土地给利玛窦。利玛窦刚离开肇庆,刘继文就在仙花寺前立上一块石碑,上书“广东刘继文以廉俸购自夷人”,可谓无耻至极。同年9月,利玛窦到达韶州,在当地一块九百平米大小的土地上建立教堂,继续传播教义和科学知识。粤北各地的士绅百姓纷纷慕名而来,兴致勃勃地参观利玛窦展示的欧洲壁画、三棱镜、挂钟,认定他是个神通广大的博学之士\footnote{周天:《跋涉:明清之际耶稣会的在华传教》,页32}。利玛窦还曾赴英德、南雄传教,南雄的首位入教者葛盛华更为传播福音的事业积极奔走,发展了六个教徒\footnote{叶农:《明末天主教在广东地区的传播与发展》}。此外,利玛窦还在韶州交到了一个至关重要的朋友,那便是吴越常熟士人瞿汝夔(字太素)。瞿汝夔出身于官宦世家,早年因与兄嫂通奸乱伦被逐出家门\footnote{关于瞿汝夔乱伦事,参见黄一农:《两头蛇——明末清初的第一代天主教徒》}。此后他携妻妾仆人周游各地、遍访名师,终于在韶州遇上利玛窦。瞿汝夔当时正痴迷于炼金、丹药之术。他搬到利玛窦身边居住,不停送礼、请其吃饭,想让利玛窦教他炼银之法。然而,他很快便被利玛窦掌握的西方科学知识所震慑,开始认真地追随利氏学习算术、历法、几何,并用帮助利氏用汉文翻译了古希腊欧几里得《几何原本》的第一卷。瞿汝夔亦醉心于基督教教义,可由于他不忍休掉自己唯一的妾,因而不具备受洗资格。不过,他仍热心地给利玛窦提了一条至关重要的建议,指出传教士应该打扮成儒生才能受到明帝国境内精英的尊重,从而更便利地传教。利玛窦听从他的建议,在瞿汝夔的指导下认真学习了《四书》,并梳起发髻、穿上儒者的绸袍。从此,他在与韶州知府相见时不再需要像普通平民一样全程跪着,而是能以秀才拜见之礼行事。1595年,利玛窦更获准离开南粤、北上南昌传教。此时,韶州的传教点已经颇具规模,当地先后有两名葡萄牙教士、四名意大利教士、两名南粤修士(韶州人黄明沙、新会人钟鸣仁)进行活动。著名的吴越上海基督徒徐光启,便是在此前后于韶州认识郭居静神父(Lazzaro Cattaneo,意大利籍耶稣会士)并接触到基督教的\footnote{叶农:《明末天主教在广东地区的传播与发展》}。此后十五年间,利玛窦历尽险阻,终于到达明帝国的中心北京,并在那里和徐光启谱写了一段东西交通史上的佳话。这一段故事不属于南粤史范畴,本书不拟详谈。做为文明传播者,利玛窦完成了在南粤播撒种子的使命。他不但使粤人认识到了天主,亦向粤人打开了认识世界的大门。

利玛窦北上后,黄明沙和钟鸣仁接过他在南粤的事业,全力传播福音。然而,愚昧残暴的明帝国以阴森的目光时刻注视着南粤的基督徒。在明帝国官僚眼中,这些粤人教徒是“勾结外夷”的“奸民”,必须严加“整肃”。1596年,韶州城内几个顽固守旧的秀才指使一伙喝醉的无赖袭击了韶州教堂,在打伤数人后被神父和修士们击退。第二天,这几个秀才恶人先告状,煽动起一群无赖沿街叫嚷,称他们遭到了神父和修士的袭击。明帝国官僚不问是非曲直,蛮横地杖责了教堂中的两个仆人,并给钟鸣仁带上沉重的头枷示众一天\footnote{利玛窦:《耶稣会与天主教进入中国史》}。此后,明帝国对南粤基督徒的迫害日甚一日。

1605年,明帝国突然声称澳门葡人将要攻打广州,无故拆除广州城墙周围的千余所民房,并将当时正在广州办事、生病发高烧的黄明沙当做葡人间谍逮捕,一同被抓走的还有两名仆人、两名教徒。明帝国官吏对他们进行了惨无人道的折磨,用夹棍狠夹他们的脚踝,逼他们承认自己是间谍。除黄明沙外,所有受刑者都痛得放声大哭,而黄明沙不但不哭号,反而用粤语温柔地安慰其他四人要镇定,决不能在严刑下说谎、背叛天主。此后一年内,黄明沙在狱中遭遇了无穷无尽的酷刑,被打得血肉模糊、遍体鳞伤。明帝国官吏一次次用木板疯狂地殴打他,而他只回答说自己他一名天主教徒,现在他耶稣会士、利玛窦的门徒。这时,他已将灵魂交给天主、决定为了他的信仰与南粤的自由殉难了。1606年3月31日,黄明沙在狱中荣归主怀,时年33岁。据利玛窦记载,他走的那一刻,正是耶稣上十字架的时刻\footnote{利玛窦:《耶稣会与天主教进入中国史》,页399—404}。

1612年,传教士被驱逐出韶州,教堂亦被关闭。此后,他们来到南雄继续活动。四年后,东亚基督教史上著名的“南京教案”爆发,正在南京活动的钟鸣仁及其弟钟鸣礼连同21名教徒一同被明军逮捕、饱受拷打讯问,南雄的传教点亦被强制关闭\footnote{关于“南京教案”,参见周天:《跋涉:明清之际耶稣会的在华传教》,页111—120}。1622年,钟鸣仁在杭州悲惨地病逝\footnote{黎小江、莫世祥主编:《澳门大辞典》,页626}。此后直至明帝国灭亡,南粤的教会一直未能恢复元气。明帝国视南粤为海防重地,严格限制外籍传教士入境。到1636年时,由于南粤境内基督徒实在太少,他们的数量甚至都无法得到有效统计。至1640年,除海南岛外,南粤境内竟再无一个外籍传教士\footnote{叶农:《明末天主教在广东地区的传播与发展》}。

在明帝国愚昧野蛮的迫害下,南粤的基督教活动暂时沉寂了。然而,罗明坚、利玛窦已将认识世界的大门向南粤敞开,并将种子撒入南粤大地,黄明沙与钟鸣仁便是这种子结出的果实。岭北强权永远无法阻止粤人对自由与文明的向往。在16—17世纪,粤人不顾明帝国的百般阻挠,纷纷扬帆起航,掀起了探索世界的巨浪。

\section{西班牙人、荷兰人、英国人来粤}

\indent 1565—1571年间,西班牙将领莱加斯皮(Miguel Lopez de Legaspi)率远征队占领菲律宾群岛。菲律宾与南粤隔南海相望,由首府马尼拉乘船只用数日便能抵达南粤、闽越。至此,西班牙人已站在了南粤的大门口。西班牙人占领菲律宾有三个目的:一、与葡萄牙分享东方贸易;二、使菲律宾基督教化;三、获得日后进攻明帝国的基地。在西班牙人看来,澳门葡人垄断了西方对粤贸易。他们希望绕开葡人,直接前往广州与明政府对话,获得粤西之间的直接贸易渠道。

1575年6月,两名西班牙使节从马尼拉出发抵达广州,要求对粤通商,被明两广总督刘尧诲拒绝,是为西班牙人首次登陆南粤\footnote{黄鸿剑:《中英关系史》,页4}。三年后,圣方济各会神父彼得·德·奥法罗(Fryer Peter de Alfaro)率团至广州,要求在广州传教。广州的明帝国官僚将他们送到广西梧州,在此会见刘尧诲。在会面中,刘尧诲仍像上次一样持顽固态度,拒绝他们的传教请求,将他们送回广州。此后,这些传教士中的一部分去了澳门,另一部分经闽越泉州返回菲律宾\footnote{《广东通史》古代下册,页540}。

两次碰壁使菲律宾西人失去耐心,决定对明动武。1570年代,西班牙人一边对明交涉,一边详细侦查闽越海防。他们认为,腐朽落后的明帝国军队根本不堪一击。1576年,菲律宾总督弗朗西斯科·桑德(Fransiscon de Sande)向菲利普二世提出,只需要4000—6000名装备精良的西班牙军人配合部分日本、明朝海盗便能轻松击败明帝国。1580年,西班牙国王菲利普二世吞并葡萄牙,组建“无敌舰队”,使西班牙一跃而为欧洲首强。此后,菲律宾西人开始详细讨论对明作战问题。1586年4月20日,继任总督圣迭戈·维拉(Santiago de Vera)在马尼拉召集菲律宾的西班牙政治、军事、宗教、公民领袖举行会议,议定了一项详细的攻明计划。该计划的内容是,组织1万至1.2万人的西班牙军队,再加上由美洲殖民地调来的5000至6000名印第安人及同等数量的日本兵(由在日传教士负责招募),建立一支有2万至2.5万人的远征军。此外,如有可能的话,亦要邀请葡萄牙人参加远征。维拉总督将亲率远征军登陆闽越,同时葡军应从澳门北上进攻南粤。西、葡两军分别攻占闽、粤后,将分头北上、直取北京。在两军推进过程中,沿途的明帝国官方机构要被留以充作傀儡政府统治当地人口。此外,该计划还强调远征军不能滥杀无辜,应对占领区民众采取温和谨慎的态度,以方便日后传教工作的开展\footnote{Sameul Howley, “The Spanish Plan to Conquer China”}。

这项宏大的计划被写入备忘录,送往西班牙本土讨论。然而,当时的西班牙帝国正忙于准备进攻英国之事,无暇向远东派出足够兵力,这一计划被迫搁浅。1588年,因“无敌舰队”征英失败,元气大伤的西班牙帝国更无可能对明出兵,该计划遂告流产\footnote{《广东通史》古代下册,页541}。据后世军事史学者推演,如果该计划果真实施,西葡联军会因人数过少而失败。但如果他们的推进速度足够快,他们有可能逼近北京。无论如何,闽、粤是几乎可以肯定会被联军占领的。若历史果真如此,南粤文明的历史走向将发生极大改变,极有可能彻底脱离岭北帝国控制,并在近代以民族国家的形式脱离葡萄牙帝国,从而傲立于世界诸国当中。可惜的是,历史没有如果。随着西班牙征明计划的流产,南粤错过了脱离岭北帝国的机会。此后,南粤仍将长期忍受岭北帝国的残酷压迫。

1588年后,西班牙人不再言征明之事,而是回到老路,寻求对粤通商的可能。1598年,西班牙使者唐·胡安·卡穆迪奥(Don Juan Camudio)从菲律宾来到南粤,要求与澳门葡人一样在广州贸易。卡穆迪奥大概采取了贿赂手段,因为这次明帝国居然同意了他的要求,允许他在广州以南的虎跳门(今珠海市斗门区蠕蛛仔,西班牙人称之为Pinal)居留。当时,葡萄牙本土虽被西班牙吞并,但两国的海外殖民地仍各行其事、互相攻击,澳门葡人与虎跳门西人皆对对方十分警惕。一支西班牙舰队随即进驻虎跳门,击败前来袭扰的葡萄牙人,并在广州购买了足够的货物,于1599年返回马尼拉。见西舰已返航,明帝国便毫无信义地出尔反尔,派兵将虎跳门的西人定居点焚烧一空。当时,菲律宾西人刚刚遭遇进攻柬埔寨失败的挫折,无力报复。西人对粤贸易的尝试,至此彻底失败\footnote{《广东通史》古代下册,页542}。

继西班牙人而来的是荷兰人。1601年,荷兰人范·内克(Van Neck)率舰队出现于亚洲海面。这支舰队首先袭击了菲律宾,不利而退。接着又驶近澳门,被葡人拒绝入城。走投无路的他们只得前往广州,与驻粤明帝国税使宦官李凤交涉。在粤人看来,属日耳曼系的荷人“须发皆赤,目睛圆”,人高马大,与拉丁系的西人、葡人外貌完全不同,遂称荷人为“红毛鬼”。由于不明荷人底细,对欧洲文明一无所知的李凤拒绝了荷人的贸易要求,令其离开。荷人仅有船两艘,无力动武,遂扬帆南去。他们在澳门附近海面遭遇葡舰袭击,被俘20人,其中17人被处死、3人被送往马六甲。荷人对此极度愤怒,决定报复葡萄牙。1602年,荷兰东印度公司成立,其宗旨系为荷兰在东方取得最大之贸易自由、最大限度地伤害西葡两国在东方的利益。此后,荷、葡两国舰船便在亚洲冲突不断\footnote{《广东通史》古代下册,页543—544}。1619年,荷军攻占爪哇岛西北部,建立巴达维亚城(雅加达),由此获得进攻澳门的陆上基地。当时,经葡人多年经营的澳门已成为东亚最重要的贸易港口,澳门—广州、澳门—印度果阿—里斯本、澳门—日本长崎、澳门—马尼拉—美洲阿卡普尔科、澳门—东南亚各港的国际贸易路线皆告开辟。通过澳门城邦的繁荣贸易,葡人获利极大,向世界市场输出了东亚生丝、丝织品、陶瓷,对南粤输入了白银、香料\footnote{《广东通史》古代下册,页556}。对于如此重要的港口城市,荷人极欲据为己有。1622年6月24日,荷军以17艘战船、1300兵力强攻澳门,在这座要塞化的城邦下遭遇惨败。此战荷军阵亡136人、负伤126人、折舰4艘,而葡人不过战死4人。为纪念这次保卫澳门城邦的重大胜利,葡人特意将6月24日定为澳门的“城市日”。1627年,4艘荷军战船对澳门展开第二次袭击,又告失败。荷人在澳门惨败后,遂转向闽越,于1622年占领澎湖、1624年占领台湾岛西南部。至此,荷兰人暂时疏离了南粤,开始与闽越深入交流。在闽荷交往中,一股强大的闽人海上势力砰然崛起,那便是17世纪中期赫赫有名的郑氏集团。关于郑氏集团与粤人的关系,笔者将在下一章中详细论述\footnote{《广东通史》古代下册,页544—545}。

在荷兰东印度公司成立前两年,英国亦成立了东印度公司,取得从好望角到麦哲伦海峡间亚、非、美三洲的所有贸易专营权。1622年,英国曾派两艘战舰支援荷军对澳门的进攻。然而在次年,荷兰驻安汶岛(今属印尼)总督听信当地英国商人勾结日本雇佣兵准备刺杀他的传言,处死了岛上的10名英国人、10名日本人和1名葡萄牙人,英荷遂告决裂\footnote{赵苏苏主编:《英汉百科专门词典》,页45}。1625年,英国东印度公司与葡萄牙果阿总督签署协议,允许英人在印度对葡贸易、通过澳门对粤直接贸易。然而,葡萄牙澳门总督却在第二年下令阻止来澳英商北上。至于愚昧的明帝国,则更不会主动与英人接触。此后十年间,英人苦于对粤贸易无门,急欲打开僵局。1635年底,英王查理一世绕开东印度公司,将对明、印度贸易全权授予科腾商团(Courteen Association),任命约翰·威代尔为私商首席代表,率一支由3艘商船、1艘战船组成的小舰队远航南粤打开市场。

1637年6月27日,威代尔舰队航抵澳门以西的横琴岛,葡人不准其进入澳门。威代尔遂将舰队泊于氹仔岛,并不断驶入珠江口进行侦查。8月8日,威代尔舰队驶至珠江口内的虎门亚娘鞋岛,驻守虎门炮台的明军向其鸣炮。威代尔下令升起英王旗,开炮还击。经数小时激战,明军惨败,英军攻克虎门炮台,缴获火炮35门。驻粤明帝国官僚受此震慑,便同意与威代尔进行贸易,并允许英舰在广州内河停泊,威代尔乃派3人携款赴广州城内购货。然而,明广东水师却在9月10日突然背信弃义地扣留入城英商、以3艘战船突袭英船,迫使毫无战斗准备的威代尔舰队驶离广州。9月19日,威代尔舰队展开报复,在虎门地区焚明船三艘,登陆焚一市集、夺猪30头。21日,他们又攻克并炸毁阿娘鞋炮台,焚明方大船一艘。完成上述任务后,舰队凯旋回到澳门附近\footnote{萨苏、李峰:《中国海魂 从郑和到钓鱼岛》,页38}。

威代尔系一英国私商。他率领的小舰队竟能如此轻易地屡屡击溃明军,明帝国之腐朽无能可见一斑。不过,威代尔虽大获全胜,却无法再进行对粤贸易,亦还未救出被扣押在广州的3名手下。于是,他只得请求澳门葡人出面对明斡旋。在葡人帮助下,英商代表与明政府于11月22日在广州达成协议,英方需赔偿白银2800两、明方则放还人质。这一赔款数远远低于英商在广州赚得的收益。11月30日,威代尔派一艘船携在广州购买的巨量货物先行回英,其它船只则继续前往亚齐(今印尼苏门答腊西北部)、印度进行贸易\footnote{萨苏、李峰:《中国海魂 从郑和到钓鱼岛》,页38;《广东通史》古代下册,页547}。

从16世纪前期到17世纪中期,葡萄牙人、西班牙人、荷兰人、英国人相继来到南粤,意图与南粤展开正常的贸易往来。然而,因愚昧腐朽、狂妄自大的明帝国侵占着南粤,此种交往遭遇了巨大阻碍。纵然如此,西方文明仍给南粤带来了巨大的贸易机会、将南粤拉入全球体系,并使粤人了解了基督教、认识到了世界的真实样貌与大小。至于伟大的澳门城邦,更是南粤文明与西方文明交流后产生的瑰丽结晶。在这一时期,大航海时代的西方文明已向南粤伸出了友谊之手。若无我们热爱自由的祖先不断冲破明帝国的阻力、积极回应,则南粤绝无可能在与西方文明的早期交往中获得无比巨大的成就。在下一章,我们便将一睹同时期粤人的动向,看看我们的祖先是以何等昂扬的姿态迈入大航海时代的。




